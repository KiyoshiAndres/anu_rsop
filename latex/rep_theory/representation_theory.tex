\documentclass[a4paper]{article}

\usepackage{geometry}
\usepackage{xcolor}
\usepackage{color,soul}
\usepackage{mathrsfs}
\usepackage{tikz-cd}
\usepackage{tcolorbox}
\usepackage{amsthm}
\usepackage{amsmath}
\usepackage{amssymb}
\usepackage{fancyhdr}
\pagestyle{fancy}
\usepackage{wrapfig}
\usepackage{tabularx}
\usepackage[font={small, sf},labelfont=bf]{caption}


\usepackage{titlesec}
\titleformat{\section}[block]{\color{black}\Large\bfseries\filcenter}{}{1em}{}



\setlength{\headheight}{40pt}
\usepackage{graphicx}
\usepackage[export]{adjustbox}


\usepackage{imakeidx}
\makeindex[columns=2, title=Index, intoc]


\usepackage{hyperref}
\usepackage{cleveref}

\hypersetup{
	colorlinks=true,
	linkcolor=blue,
	filecolor=magenta,      
	urlcolor=cyan,
	pdftitle={Sharelatex Example},
	pdfpagemode=FullScreen,
}

\urlstyle{same}


\newcommand{\FHom}{\mathcal{H} o m}
\newcommand{\sheafification}{\mathcal{F}^+}
\newcommand{\Id}{\textrm{Id}}
\newcommand{\Hom}{\textrm{Hom}}
\newcommand{\Spec}{\textrm{Spec }}
\newcommand{\Sch}{\textrm{Sch}}
\newcommand{\Set}{\textrm{Set}}
\newcommand{\Grass}{\textrm{Grass}}
\newcommand{\Mod}{\textrm{Mod}}
\newcommand{\id}{\textrm{id}}
\newcommand{\Ob}{\textrm{Ob}}
\newcommand{\Ext}{\textrm{Ext}}
\newcommand{\Tor}{\textrm{Tor}}
\newcommand{\Invlim}{\lim\limits_{\longleftarrow}}
\newcommand{\Dirlim}{\lim\limits_{\longrightarrow}}
\newcommand{\Gal}{\textrm{Gal}}
\newcommand{\F}{\mathscr{F}}
\newcommand{\res}{\text{res}}
\newcommand{\primep}{\mathfrak{p}}
\newcommand{\bigo}{\mathcal{O}}
\newcommand\smallo{
	\mathchoice
	{{\scriptstyle\mathcal{O}}}% \displaystyle
	{{\scriptstyle\mathcal{O}}}% \textstyle
	{{\scriptscriptstyle\mathcal{O}}}% \scriptstyle
	{\scalebox{.7}{$\scriptscriptstyle\mathcal{O}$}}%\scriptscriptstyle
}
\newcommand{\Aut}{\textrm{Aut}}
\newcommand{\projs}{\mathbf{Proj}(S)}
\newcommand{\Proj}{\mathbf{Proj}}
\newcommand{\Ass}{\textrm{Ass}}
\newcommand{\Div}{\textrm{Div}}
\newcommand{\pdiv}{\textrm{div}}
\newcommand{\Cov}{\textrm{Cov}}
\newcommand{\Op}{\textrm{Op}}


\newtheoremstyle{theorem}{}{}{}{}{\color{blue}\bfseries}{.}{ }{}
\theoremstyle{theorem}
\newtheorem{theorem}{Theorem}[section]
\newtheorem{lemma}[theorem]{Lemma}
\newtheorem{corollary}[theorem]{Corollary}
\newtheorem{proposition}[theorem]{Proposition}

\theoremstyle{definition}
\newtheorem{definition}{Definition} [section]
\newtheorem{example}{Example}
\newtheorem{xca}[theorem]{Exercise}



\theoremstyle{remark}
\newtheorem{remark}{Remark}

\newtheoremstyle{gremark}{}{}{}{}{\color{red}\bfseries}{.}{ }{}
\theoremstyle{gremark}
\newtheorem{gremark}{Future Expansion Necessary}

\newtheoremstyle{discussion}{}{}{}{}{\color{orange}\bfseries}{.}{ }{}
\theoremstyle{discussion}
\newtheorem{discussion}{Discussion}

\newtheoremstyle{notation}{}{}{}{}{\color{orange}\bfseries}{.}{ }{}
\theoremstyle{notation}
\newtheorem{notation}{Notation}


\newtcolorbox{mybox}[3][]
{
	colframe = #2!25,
	colback  = #2!10,
	coltitle = #2!20!black,  
	#1,
}




\begin{document}
	
	
	
	\section*{Definitions}
	
	
	\begin{definition}
		The \textbf{Regular Representation} of $A$ is given by $\rho: A\to End(A) $, with 
		$\newline\rho(a)b=ab$.
	\end{definition}
	
	\begin{definition}
		A representation is called \textbf{irreducible} or \textbf{simple}, if the only subrepresentations are $0$ and $V$.
	\end{definition}

	\begin{definition}
		A representation is called \textbf{semi-simple} if it is a direct sum of irreducible representations.
	\end{definition}

	\begin{definition}
		A non-zero representation of $A$ is said to be \textbf{indecomposable} if it cannot be written as a direct sum of two non-zero representations.
	\end{definition}

	\begin{lemma}
		Let $V_1$ and $V_2$ be representations of an algebra $A$ over a field $F$. Let $\phi: V_1\to V_2$ be a non-zero morphism. Then
		\begin{enumerate}
			\item If $V_1$ is irreducible, $\phi$ is injective.
			\item If $V_2$ is irreducible, $\phi$ is surjective.
		\end{enumerate}
	\end{lemma}

	

	\begin{definition}
		Let $\mathfrak{g}$ be a vector space, and let $[,]:\mathfrak{g}\times\mathfrak{g}\to \mathfrak{g}$ be a pairing. Then $(\mathfrak{g},[,])$ is said to be a \textbf{Lie Algebra} if it satisfies the following two properties:
		\begin{enumerate}
			\item $[,]$ is skew-symmetric bilinear.
			\item $[,]$ satisfies the \textbf{Jacobi Identity}: 
			$$[[a,b],c]+[[b,c],a]+[[c,a],b]=0.$$
		\end{enumerate}
	\end{definition}

	\begin{definition}
		Let $V$ be a vector space. Then, the \textbf{General Lie Algebra} of $V$ is defined as $End(V)$ together with $[a,b]=ab-ba$. It is denoted by $\mathfrak{gl}(V)$.
	\end{definition}
	
	\begin{proposition}
		Let $g(t)$ be a differentiable family of automorphisms of an algebra $A$ over $\mathbb{R}$ or $\mathbb{C}$, parametrized by $t\in(-\epsilon,\epsilon)$, such that $g'(0)=Id$. Then $g'(0):A\to A$ is a derivation. Conversely, if $D$ is a derivation, then $e^{tD}$ is a one-parameter family of Automorphisms.
	\end{proposition}

	\begin{definition}
		Let $\mathfrak{g}$ be a Lie algebra with basis $x_i$ and define $[x_i,x_j]=\sum_k c_{ij}^k x_k$. Then the \textbf{Universal Enveloping Algebra} of $\mathfrak{g}$, denoted by $\mathcal{U}(\mathfrak{g})$
	\end{definition}
	

	
	
\end{document}