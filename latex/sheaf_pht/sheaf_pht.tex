\documentclass[12pt]{article}

\usepackage{geometry}
\usepackage{xcolor}
\usepackage{color,soul}
\usepackage{mathrsfs}
\usepackage{tikz-cd}
\usepackage{tcolorbox}
\usepackage{amsthm}
\usepackage{amsmath}
\usepackage{amssymb}
\usepackage{fancyhdr}
\pagestyle{fancy}
\usepackage{graphicx}


\usepackage{titlesec}
\titleformat{\section}[block]{\color{black}\Large\bfseries\filcenter}{}{1em}{}



\setlength{\headheight}{40pt}
\usepackage{graphicx}

\usepackage{imakeidx}
\makeindex[columns=2, title=Index, intoc]


\usepackage{hyperref}
\usepackage{cleveref}

\hypersetup{
	colorlinks=true,
	linkcolor=blue,
	filecolor=magenta,      
	urlcolor=cyan,
	pdftitle={Sharelatex Example},
	pdfpagemode=FullScreen,
}

\urlstyle{same}


\newcommand{\FHom}{\mathcal{H} o m}
\newcommand{\sheafification}{\mathcal{F}^+}
\newcommand{\Id}{\textrm{Id}}
\newcommand{\Hom}{\textrm{Hom}}
\newcommand{\Spec}{\textrm{Spec }}
\newcommand{\Sch}{\textrm{Sch}}
\newcommand{\Set}{\textrm{Set}}
\newcommand{\Grass}{\textrm{Grass}}
\newcommand{\Mod}{\textrm{Mod}}
\newcommand{\id}{\textrm{id}}
\newcommand{\Ob}{\textrm{Ob}}
\newcommand{\Ext}{\textrm{Ext}}
\newcommand{\Tor}{\textrm{Tor}}
\newcommand{\Invlim}{\lim\limits_{\longleftarrow}}
\newcommand{\Dirlim}{\lim\limits_{\longrightarrow}}
\newcommand{\Gal}{\textrm{Gal}}
\newcommand{\F}{\mathscr{F}}
\newcommand{\res}{\text{res}}
\newcommand{\primep}{\mathfrak{p}}
\newcommand{\bigo}{\mathcal{O}}
\newcommand\smallo{
	\mathchoice
	{{\scriptstyle\mathcal{O}}}% \displaystyle
	{{\scriptstyle\mathcal{O}}}% \textstyle
	{{\scriptscriptstyle\mathcal{O}}}% \scriptstyle
	{\scalebox{.7}{$\scriptscriptstyle\mathcal{O}$}}%\scriptscriptstyle
}
\newcommand{\Aut}{\textrm{Aut}}
\newcommand{\projs}{\mathbf{Proj}(S)}
\newcommand{\Proj}{\mathbf{Proj}}
\newcommand{\Ass}{\textrm{Ass}}
\newcommand{\Div}{\textrm{Div}}
\newcommand{\pdiv}{\textrm{div}}
\newcommand{\Cov}{\textrm{Cov}}
\newcommand{\Op}{\textrm{Op}}


\newtheoremstyle{theorem}{}{}{}{}{\color{blue}\bfseries}{.}{ }{}
\theoremstyle{theorem}
\newtheorem{theorem}{Theorem}[section]
\newtheorem{lemma}[theorem]{Lemma}
\newtheorem{corollary}[theorem]{Corollary}
\newtheorem{proposition}[theorem]{Proposition}

\theoremstyle{definition}
\newtheorem{definition}{Definition} [section]
\newtheorem{example}{Example}
\newtheorem{xca}[theorem]{Exercise}



\theoremstyle{remark}
\newtheorem{remark}{Remark}

\newtheoremstyle{gremark}{}{}{}{}{\color{red}\bfseries}{.}{ }{}
\theoremstyle{gremark}
\newtheorem{gremark}{Future Expansion Necessary}

\newtheoremstyle{discussion}{}{}{}{}{\color{orange}\bfseries}{.}{ }{}
\theoremstyle{discussion}
\newtheorem{discussion}{Discussion}

\newtheoremstyle{notation}{}{}{}{}{\color{orange}\bfseries}{.}{ }{}
\theoremstyle{notation}
\newtheorem{notation}{Notation}

\bibliographystyle{apalike}

\newtcolorbox{mybox}[3][]
{
	colframe = #2!25,
	colback  = #2!10,
	coltitle = #2!20!black,  
	#1,
}




\title{Derived PHT}
\author{Kiyoshi Andres Takeuchi Romo}

\begin{document}
	
	\maketitle
	
	
	In our analysis of projections of a $3$-Manifold $X$, we would like to reconstruct our Manifold from its projections onto a direction $v\in S^2$. Each projection map is close to giving a Fiber Bundle, but the size of the fibers vary at each point. So we explore two alternatives: Sheaves and Fibrations.
	
	\begin{definition}
		A \textbf{Fiber Bundle} is a 
	\end{definition}

	\begin{definition}
		A \textbf{Fibration} is a 
	\end{definition}

	\begin{definition}
		A \textbf{Sheaf} is a 
	\end{definition}
	
	\section*{Abstract Sheaf Theory}
	
	A presheaf is an organizational tool that allows you to specify information of some structure locally or on open sets.
	
	A sheaf is a presheaf that has the property that the information you have locally can glue into global information.
	
	I will thus begin by giving an example of a presheaf that isn't a sheaf, and a presheaf that is a sheaf.
	
	\begin{example}
		
	\end{example}

	\begin{example}
		
	\end{example}

	Now, it is somewhat clear from the definition of sheaf that it only depends on the notion of open covers. 
	
	\begin{definition}
		\textbf{Grothendieck Topology}
		\textbf{Site}
	\end{definition}

	As the name suggests, a Grothendieck Topology is a way to give some topological aspect to a Category. But in fact, it's not the full axioms of a topology that are passed on, rather only the information of "open covers". This is because open covers are enough to define sheaves.
	
		\begin{example}
		The most important example is that given by the open sets of a toplogical space $X$. The objects of the category will be denoted by $Op(X)$ and will consist of open subsets of $X$. Morphisms are inclusions.
		\end{example}
	
	\begin{definition}
		A \textbf{presheaf} is a contravariant functor $\mathcal{F}:\mathcal{C}\to \mathcal{D}$, where $\mathcal{C}$ is a Site $\mathcal{D}$ is usually taken to be the category of sets, groups, rings, etc.
	\end{definition}

	The main reason behind calling contravariant functors presheaves, is because it suggests that we want to treat that contravariant functor as a sheaf. 

	\begin{definition}
		
	\end{definition}

	\begin{proposition}
		\textbf{Sheafification}
	\end{proposition}

	\begin{example}
		\textbf{constant sheaf}
	\end{example}

	\begin{example}
		
	\end{example}

	Sheaves are 
	
	
	\section{Lifting Properties}
	
	\begin{definition}
		
	\end{definition}

	\begin{definition}
		
	\end{definition}

	\begin{example}
		
	\end{example}

	\begin{example}
		
	\end{example}

	\section{Fibrations}
	
	

	\section{Derived and Homotopy Categories}

	\begin{definition}
		
	\end{definition}
	
	
	\section{Applications}
	
	\section{Loop Spaces}
	
	
	
	
	\begin{example}
		Let $X$ be a PL $3$-Manifold, and let $v\in S^2$. We denote the projection of $X$ in the direction of $v$ as $X_v$.  Define a presheaf $\mathcal{F}:X\to \mathbb{R}^2$ by $\mathcal{F}(U)=PHT(U)$.
	\end{example}
	
	
	
	
	\section*{Magnitude Homology}
	
	\begin{definition}
		
	\end{definition}
	
	\begin{proposition}
		
	\end{proposition}
	
	
	
	
	
	
	
	
\end{document}