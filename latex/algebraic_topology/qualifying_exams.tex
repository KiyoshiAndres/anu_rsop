\documentclass[a4paper]{article}

\usepackage{geometry}
\usepackage{xcolor}
\usepackage{color,soul}
\usepackage{mathrsfs}
\usepackage{tikz-cd}
\usepackage{tcolorbox}
\usepackage{amsthm}
\usepackage{amsmath}
\usepackage{amssymb}
\usepackage{fancyhdr}
\pagestyle{fancy}
\usepackage{graphicx}


\usepackage{titlesec}
\titleformat{\section}[block]{\color{black}\Large\bfseries\filcenter}{}{1em}{}



\setlength{\headheight}{40pt}
\usepackage{graphicx}



\usepackage{imakeidx}
\makeindex[columns=2, title=Index, intoc]


\usepackage{hyperref}
\usepackage{cleveref}

\hypersetup{
	colorlinks=true,
	linkcolor=blue,
	filecolor=magenta,      
	urlcolor=cyan,
	pdftitle={Sharelatex Example},
	pdfpagemode=FullScreen,
}

\urlstyle{same}


\newcommand{\FHom}{\mathcal{H} o m}
\newcommand{\sheafification}{\mathcal{F}^+}
\newcommand{\Id}{\textrm{Id}}
\newcommand{\Hom}{\textrm{Hom}}
\newcommand{\Spec}{\textrm{Spec }}
\newcommand{\Sch}{\textrm{Sch}}
\newcommand{\Set}{\textrm{Set}}
\newcommand{\Grass}{\textrm{Grass}}
\newcommand{\Mod}{\textrm{Mod}}
\newcommand{\id}{\textrm{id}}
\newcommand{\Ob}{\textrm{Ob}}
\newcommand{\Ext}{\textrm{Ext}}
\newcommand{\Tor}{\textrm{Tor}}
\newcommand{\Invlim}{\lim\limits_{\longleftarrow}}
\newcommand{\Dirlim}{\lim\limits_{\longrightarrow}}
\newcommand{\Gal}{\textrm{Gal}}
\newcommand{\F}{\mathscr{F}}
\newcommand{\res}{\text{res}}
\newcommand{\primep}{\mathfrak{p}}
\newcommand{\bigo}{\mathcal{O}}
\newcommand\smallo{
	\mathchoice
	{{\scriptstyle\mathcal{O}}}% \displaystyle
	{{\scriptstyle\mathcal{O}}}% \textstyle
	{{\scriptscriptstyle\mathcal{O}}}% \scriptstyle
	{\scalebox{.7}{$\scriptscriptstyle\mathcal{O}$}}%\scriptscriptstyle
}
\newcommand{\Aut}{\textrm{Aut}}
\newcommand{\projs}{\mathbf{Proj}(S)}
\newcommand{\Proj}{\mathbf{Proj}}
\newcommand{\Ass}{\textrm{Ass}}
\newcommand{\Div}{\textrm{Div}}
\newcommand{\pdiv}{\textrm{div}}
\newcommand{\Cov}{\textrm{Cov}}
\newcommand{\Op}{\textrm{Op}}


\newtheoremstyle{theorem}{}{}{}{}{\color{blue}\bfseries}{.}{ }{}
\theoremstyle{theorem}
\newtheorem{theorem}{Theorem}[section]
\newtheorem{lemma}[theorem]{Lemma}
\newtheorem{corollary}[theorem]{Corollary}
\newtheorem{proposition}[theorem]{Proposition}

\theoremstyle{definition}
\newtheorem{definition}{Definition} [section]
\newtheorem{example}{Example}
\newtheorem{xca}[theorem]{Exercise}



\theoremstyle{remark}
\newtheorem{remark}{Remark}

\newtheoremstyle{gremark}{}{}{}{}{\color{red}\bfseries}{.}{ }{}
\theoremstyle{gremark}
\newtheorem{gremark}{Future Expansion Necessary}

\newtheoremstyle{discussion}{}{}{}{}{\color{orange}\bfseries}{.}{ }{}
\theoremstyle{discussion}
\newtheorem{discussion}{Discussion}

\newtheoremstyle{notation}{}{}{}{}{\color{orange}\bfseries}{.}{ }{}
\theoremstyle{notation}
\newtheorem{notation}{Notation}


\newtcolorbox{mybox}[3][]
{
	colframe = #2!25,
	colback  = #2!10,
	coltitle = #2!20!black,  
	#1,
}




\begin{document}
	
\section*{Path Connected}

\begin{proposition}
	A disjoint union of locally path-connected spaces is locally path-connected.
\end{proposition}

\begin{proposition}
	A locally contractible space is locally path-connected.
\end{proposition}

\begin{proposition}
	A CW space is locally contractible.
\end{proposition}

\begin{proposition}
	The quotient of a locally path-connected space is locally path-connected.
\end{proposition}

\section{Covering Space}

\begin{definition}
	A \textbf{universal covering} is a simply connected covering. They are unique up to homeomorphism.
\end{definition}

\begin{proposition}
	A connected topological space has a universal cover iff it is locally path-connected and semilocally simply connected.
\end{proposition}

\begin{proposition}
	The number of sheets of a covering space $p : ( \widetilde{X}, x_0)\to (X, x_0)$
	with $X$ and $\widetilde{X}$ path-connected equals the index of $p_*(\pi_1(\widetilde{X}, x_0))$ in $\pi_1(X, x_0)$ .
\end{proposition}

\begin{proposition}
	Let $p : (\widetilde{X}, \widetilde{x}_0) \to (X, x_0) $ be a path-connected covering space of the path-connected, locally path-connected space $X$, and let $H = p_* \left( \pi_1(\widetilde{X}, \widetilde{x}_0) \right) \subset \pi_1(X, x_0)$. Then:
	
	\begin{enumerate}
		\item The covering space \( p : \widetilde{X} \to X \) is normal (regular) if and only if \( H \) is a normal subgroup of \( \pi_1(X, x_0) \).
		
		\item The group of deck transformations \( \text{Deck}(\widetilde{X}/X) \) is isomorphic to the quotient \( N(H)/H \), where \( N(H) \) is the normalizer of \( H \) in \( \pi_1(X, x_0) \).
	\end{enumerate}
	
	\noindent In particular, if \( \widetilde{X} \to X \) is a normal covering, then
	\[
	\text{Deck}(\widetilde{X}/X) \cong \pi_1(X, x_0)/H.
	\]
	Hence, for the universal cover \( \widetilde{X} \to X \), we have
	\[
	\text{Deck}(\widetilde{X}/X) \cong \pi_1(X, x_0).
	\]
\end{proposition}

\begin{lemma}
	The number of sheets of the universal covering space is 
\end{lemma}

\begin{proposition}
	Deck transformations don't have fixed points.
\end{proposition}

\pagebreak

\section{CW Complexes}

\begin{proposition}
	A CW complex is semi-locally simply connected.
\end{proposition}

\begin{proposition}
	A connected CW complex has a universal cover.
\end{proposition}

\begin{proposition}
	A covering space of a CW complex is also a CW complex, with cells projecting homeomorphically to cells.
\end{proposition}

\begin{proposition}
	If $X$ is a finite CW complex and if $Y\to X$ is a $n$-sheeted covering then $Y$ is a finite CW complex and $\Xi(Y)=n\cdot\Xi(X)$. 
\end{proposition}


\pagebreak

\section{Named Theorems}

\begin{definition}
	The Lefschetz Number
	$$\Lambda(f):=\sum_n \textrm{tr}(f_*:H_n(X;\mathbb{Q})\to H_n(X;\mathbb{Q})) $$
\end{definition}

\begin{theorem}[Lefschetz Fixed Point Theorem]
	If $X$ is a triangulable space or a retract of a simplicial complex, and if $f:X\to X$ is continuous, then if $\Lambda(f)\neq 0$, $f$ has a fixed point.
\end{theorem}

\pagebreak

\section{Problems}

\begin{example}
	Suppose that $X$ is a finite connected CW complex such that $\pi_1(X)$ is finite and nontrivial. Prove that the universal covering $\widetilde{X}$ of $X$ cannot be contractible.
\end{example}

\begin{proof}
	Since $X$ is a connected CW complex it has a universal cover, which is also a CW complex since $X$ is a CW complex. Since $\pi_1(X)$ is finite, the universal cover has a finite number of sheets, and since $X$ is a finite CW complex, each sheet has finite cells. So $\widetilde{X}$ is a finite CW complex. 
	\medbreak
	
	Suppose for contradiction that $\widetilde{X}$ is contractible. Then $H_0(\widetilde{X})=\mathbb{Z}$ and $H_i(\widetilde{X})=0$ for $i>0$. Since $f$ is continuous, and $\widetilde{X}$ is simply connected and hence connected,
	Let $p\in \widetilde{X}$ and since $\widetilde{X}$ is connected there is only one generator of $H_0(\widetilde{X};\mathbb{Q}) $, $[p]$. Then $[p]=1$. And $f_*$ maps $[p]$ to $[f(p)]$ but since $f$ is continuous and $f(p)\in\widetilde{X}$, $[f(p)]=[p]$ hence $f_*$ is the identity and its trace must be $1$.  	
	Thus any continuous self map $f:\widetilde{X}\to\widetilde{X}$ has Lefschetz number $1$, and thus has a fixed point.  
	Whic is a contradiction since $\pi_1(X)$ is non-trivial and is isomorphic to the group of Deck transformations of $\widetilde{X}$ and thus there is a non-trivial Deck transformation, and Deck transformations don't have fixed points. 
	
	Alternatively, by computing $\chi(\widetilde{X})$ using homology and the fact that $\widetilde{X}$ is contractible, $1=\chi(\widetilde{X})=|\pi_1(X)|\cdot \chi(X)$ so that since $\pi_1(X)$ is non-trivial this is impossible.
\end{proof}
	
	
\end{document}