\documentclass[12pt]{article}

\usepackage{geometry}
\usepackage{xcolor}
\usepackage{color,soul}
\usepackage{mathrsfs}
\usepackage{tikz-cd}
\usepackage{tcolorbox}
\usepackage{amsthm}
\usepackage{amsmath}
\usepackage{amssymb}
\usepackage{fancyhdr}
\pagestyle{fancy}
\usepackage{graphicx}


\usepackage{titlesec}
\titleformat{\section}[block]{\color{black}\Large\bfseries\filcenter}{}{1em}{}



\setlength{\headheight}{40pt}
\usepackage{graphicx}

\usepackage{imakeidx}
\makeindex[columns=2, title=Index, intoc]


\usepackage{hyperref}
\usepackage{cleveref}

\hypersetup{
	colorlinks=true,
	linkcolor=blue,
	filecolor=magenta,      
	urlcolor=cyan,
	pdftitle={Sharelatex Example},
	pdfpagemode=FullScreen,
}

\urlstyle{same}


\newcommand{\FHom}{\mathcal{H} o m}
\newcommand{\sheafification}{\mathcal{F}^+}
\newcommand{\Id}{\textrm{Id}}
\newcommand{\Hom}{\textrm{Hom}}
\newcommand{\Spec}{\textrm{Spec }}
\newcommand{\Sch}{\textrm{Sch}}
\newcommand{\Set}{\textrm{Set}}
\newcommand{\Grass}{\textrm{Grass}}
\newcommand{\Mod}{\textrm{Mod}}
\newcommand{\id}{\textrm{id}}
\newcommand{\Ob}{\textrm{Ob}}
\newcommand{\Ext}{\textrm{Ext}}
\newcommand{\Tor}{\textrm{Tor}}
\newcommand{\Invlim}{\lim\limits_{\longleftarrow}}
\newcommand{\Dirlim}{\lim\limits_{\longrightarrow}}
\newcommand{\Gal}{\textrm{Gal}}
\newcommand{\F}{\mathscr{F}}
\newcommand{\res}{\text{res}}
\newcommand{\primep}{\mathfrak{p}}
\newcommand{\bigo}{\mathcal{O}}
\newcommand\smallo{
	\mathchoice
	{{\scriptstyle\mathcal{O}}}% \displaystyle
	{{\scriptstyle\mathcal{O}}}% \textstyle
	{{\scriptscriptstyle\mathcal{O}}}% \scriptstyle
	{\scalebox{.7}{$\scriptscriptstyle\mathcal{O}$}}%\scriptscriptstyle
}
\newcommand{\Aut}{\textrm{Aut}}
\newcommand{\projs}{\mathbf{Proj}(S)}
\newcommand{\Proj}{\mathbf{Proj}}
\newcommand{\Ass}{\textrm{Ass}}
\newcommand{\Div}{\textrm{Div}}
\newcommand{\pdiv}{\textrm{div}}
\newcommand{\Cov}{\textrm{Cov}}
\newcommand{\Op}{\textrm{Op}}


\newtheoremstyle{theorem}{}{}{}{}{\color{blue}\bfseries}{.}{ }{}
\theoremstyle{theorem}
\newtheorem{theorem}{Theorem}[section]
\newtheorem{lemma}[theorem]{Lemma}
\newtheorem{corollary}[theorem]{Corollary}
\newtheorem{proposition}[theorem]{Proposition}

\theoremstyle{definition}
\newtheorem{definition}{Definition} [section]
\newtheorem{example}{Example}
\newtheorem{xca}[theorem]{Exercise}



\theoremstyle{remark}
\newtheorem{remark}{Remark}

\newtheoremstyle{gremark}{}{}{}{}{\color{red}\bfseries}{.}{ }{}
\theoremstyle{gremark}
\newtheorem{gremark}{Future Expansion Necessary}

\newtheoremstyle{discussion}{}{}{}{}{\color{orange}\bfseries}{.}{ }{}
\theoremstyle{discussion}
\newtheorem{discussion}{Discussion}

\newtheoremstyle{notation}{}{}{}{}{\color{orange}\bfseries}{.}{ }{}
\theoremstyle{notation}
\newtheorem{notation}{Notation}

\bibliographystyle{apalike}

\newtcolorbox{mybox}[3][]
{
	colframe = #2!25,
	colback  = #2!10,
	coltitle = #2!20!black,  
	#1,
}




\title{Projections of PL 3-Manifold}
\author{Kiyoshi Andres Takeuchi Romo}

\begin{document}
	
	\maketitle
	
	\section*{PL-Manifolds}
	
	\begin{proposition}
		Let $X$ be a PL 3-Manifold. There exists a PL-Manifold that is the topological convex hull of $X$.
	\end{proposition}

	\begin{proposition}
		Let $X$ be the convex hull of a PL 3-Manifold. The critical points determine its structure uniquely.
	\end{proposition}


	\section*{Projections}
	
	Let $X$ be a PL 3-Manifold. Let $v\in S^2$. We denote the projection of $X$ in the direction of $v$ by $X_v$. 
	
	\begin{proposition}
		Let $x\in X_v$. Then, for any allowable direction $u\in S^2$, $h_u(x)=h_u(y)$ for any $y\in p_v^{-1}(x)$.
	\end{proposition}

	\begin{proof}
		Let $y=(y\cdot v)v+(y-(y\cdot v)v)$. Then, letting $x=(y-(y\cdot v)v)$ be the projection, we write $y=(y\cdot v)v+x$.
		Now, the allowable directions for the height function of $X_v$ are $u\in S^2$ for which $u\cdot v=0$, so that $y\cdot u=(y\cdot v)v\cdot u + x\cdot u=x\cdot u$. Therefore we have our conclusion.
	\end{proof}

	\subsection{Convex}
	
	\begin{proposition}
		Let $X$ be convex. Then, for any $u\in S^2$ such that $u\cdot v=0$, we have that the diagrams $D_0(X,u)=D_0(X_v,u)$.
	\end{proposition}

	\begin{proof}
		There is a unique off-diagonal point in each diagram with birth at the minimum.
	\end{proof}

	\begin{corollary}
		We can recover $D_0(X,u)$ for any $u\in S^2$.
	\end{corollary}

	\begin{proof}
		Start with any $u\in S^2$, then use any $v\in S^2$ orthogonal to 
	\end{proof}

	\begin{corollary}
		Let $X$ be convex. Then the structure of $X$ can be recovered from the projections.
	\end{corollary}

	\begin{proof}
		Since we have $D_0(X,u)$ for all $u\in S^2$ by the previous corollary, and we know that $D_1(X,u)$ and $D_2(X,u)$ are trivial by convexity, we are in the setting of .
	\end{proof}

	Now, more generally we will not have that equality $D(X,u)=D(X_v,u)$. But in the case of simply connectedness, we have 
	
	\begin{proposition}
		Let $X$ be convex. Then, it suffices to study a single great circle of directions to recover the structure of $X$.
	\end{proposition}

	\begin{proof}
		Fix some $w\in S^2$. We identify the set $u\in S^2$ such that $u\cdot w=0$ with $S^1$. Pick any $m\in S^2$. Then there will be some $u\in S^1$ for which $m\cdot u=0$ (explicitely take $\pm m\times w$). Pick one of them, and $D_0(X_u,m)=D_0(X,m)$.
	\end{proof}

	\subsection{Simply Connected}

	\begin{proposition}
		Let $X$ be simply connected. Let $v\in S^2$. Then, we have the following table:
		
		$Ord_0(X_v, w)\subseteq Ord_0(X,w)$.
	\end{proposition}

	\begin{proof}
		
	\end{proof}
	
	\subsection*{Visibility of Vertices}
	
	\begin{definition}
		Let $X$ be a PL 3-Manifold. Let $x\in X$ be a critical point. $x$ is said to be \textbf{visible} if $x$ is a critical point for some projection $p_v$ and $h_u$ such that $u\cdot v=0$.
	\end{definition}

	\begin{definition}
		A point $x\in X$ is said to be \textbf{twice visible} if it is visible from two distinct directions $v_1,v_2\in S^2$.
	\end{definition}
	
	\begin{proposition}
		Let $X$ be a convex PL 3-Manifold. Every critical point is twice visible.
	\end{proposition}

	\begin{proposition}
		Let $X$ be a convex PL 3-Manifold. If a critical point is twice visible, it can be uniquely identified.
	\end{proposition}

	\subsection*{Visibility of Links}

	\begin{definition}
		visibility of link
	\end{definition}

	\begin{proposition}
		Let $X$ be a PL 3-Manifold. If every critical point is twice visible, then one can uniquely determine its structure.
	\end{proposition}

	\begin{corollary}
		One can recover the convex hull of a PL 3-Manifold from its projections.
	\end{corollary}

	
	
	
	
	
	
	
	
\end{document}