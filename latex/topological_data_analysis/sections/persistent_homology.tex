\documentclass[../main.tex]{subfiles}

\begin{document}
	
	The quintessential reference for this section is \cite{hatcher2002algebraic}. 
	
	
	There are a bunch of Homology groups, so it's easier to define them algebraically, and then name them by their respective chain complexes.
	
	\begin{definition}
		convex hull
	\end{definition}
	
	
	
	\begin{definition}
		Let $X$ be a topological space, and $A\subseteq X$ a subspace. Then, the \textbf{relative homology groups}\index{Homology!Relative} $H_n(X,A)$ are given by the chain complex defined by 
		$$C_n(X,A)=C_n(X)/C_n(A).$$
	\end{definition}

	
		This next section is dedicated to defining the alpha complex.
		
		\begin{definition}
			
		\end{definition}

	\begin{definition}
		alpha complex. 
	\end{definition}

	
	There's a number of theorems we can prove for relative homology groups.
	
	
	
	\begin{example}[How to draw a persistence diagram]
		
	\end{example}
	
	

	
	\subsection{Persistent Homology Morse Theory}
	
	
	
	\subsection{Extended Persistent Homology}
	
		Let $p=[p_1,\ldots, p_d]\in \mathbb{Z}^d$ and let $l=[l_1,\ldots,l_d]\in \mathbb{Z}/2\mathbb{Z}^d$
	
	
	
	
	\subsection{Persistent Homology}
	
		Let $X$ be a $\Delta$-complex. 
		
		
	
	
	
		\ifSubfilesClassLoaded{
			\bibliographystyle{apalike}
			\bibliography{../refs}
		}{}
	
\end{document}
