\documentclass[../main.tex]{subfiles}

\begin{document}
	
	The idea of Morse theory is that you can obtain some information about a geometric space from the functions defined on it. For example, we can have unbounded continuous functions on $\mathbb{R}$, yet this is not possible in $S^1$ since it is compact \cite{matsumoto2002introduction}.
	
	Smooth manifolds, functions between smooth manifolds, and other smooth manifold concepts are defined as in \cite[150]{lee2003introduction}. We will usually assume functions $f$ of manifolds to be smooth, or at least twice differentiable so that their Hessians exist.
	
	\begin{definition}
		A critical point $p_0\in M$ is said to be \textbf{non-degenerate} for the function $f$ if its \textbf{Hessian} 
		\begin{equation}
			H_f=
		\end{equation}
		
		
		$H_f(p_0)$ has non-zero determinant at $p_0$.
	\end{definition}
	
	I would like to keep some visual examples on hand. To that end, consider the following surfaces:
	
	\begin{figure}[H]
		\begin{center}
			%% Creator: Matplotlib, PGF backend
%%
%% To include the figure in your LaTeX document, write
%%   \input{<filename>.pgf}
%%
%% Make sure the required packages are loaded in your preamble
%%   \usepackage{pgf}
%%
%% Also ensure that all the required font packages are loaded; for instance,
%% the lmodern package is sometimes necessary when using math font.
%%   \usepackage{lmodern}
%%
%% Figures using additional raster images can only be included by \input if
%% they are in the same directory as the main LaTeX file. For loading figures
%% from other directories you can use the `import` package
%%   \usepackage{import}
%%
%% and then include the figures with
%%   \import{<path to file>}{<filename>.pgf}
%%
%% Matplotlib used the following preamble
%%   \def\mathdefault#1{#1}
%%   \everymath=\expandafter{\the\everymath\displaystyle}
%%   
%%   \makeatletter\@ifpackageloaded{underscore}{}{\usepackage[strings]{underscore}}\makeatother
%%
\begingroup%
\makeatletter%
\begin{pgfpicture}%
\pgfpathrectangle{\pgfpointorigin}{\pgfqpoint{6.400000in}{4.800000in}}%
\pgfusepath{use as bounding box, clip}%
\begin{pgfscope}%
\pgfsetbuttcap%
\pgfsetmiterjoin%
\definecolor{currentfill}{rgb}{1.000000,1.000000,1.000000}%
\pgfsetfillcolor{currentfill}%
\pgfsetlinewidth{0.000000pt}%
\definecolor{currentstroke}{rgb}{1.000000,1.000000,1.000000}%
\pgfsetstrokecolor{currentstroke}%
\pgfsetdash{}{0pt}%
\pgfpathmoveto{\pgfqpoint{0.000000in}{0.000000in}}%
\pgfpathlineto{\pgfqpoint{6.400000in}{0.000000in}}%
\pgfpathlineto{\pgfqpoint{6.400000in}{4.800000in}}%
\pgfpathlineto{\pgfqpoint{0.000000in}{4.800000in}}%
\pgfpathlineto{\pgfqpoint{0.000000in}{0.000000in}}%
\pgfpathclose%
\pgfusepath{fill}%
\end{pgfscope}%
\begin{pgfscope}%
\pgfsetbuttcap%
\pgfsetmiterjoin%
\definecolor{currentfill}{rgb}{1.000000,1.000000,1.000000}%
\pgfsetfillcolor{currentfill}%
\pgfsetlinewidth{0.000000pt}%
\definecolor{currentstroke}{rgb}{0.000000,0.000000,0.000000}%
\pgfsetstrokecolor{currentstroke}%
\pgfsetstrokeopacity{0.000000}%
\pgfsetdash{}{0pt}%
\pgfpathmoveto{\pgfqpoint{1.072000in}{0.528000in}}%
\pgfpathlineto{\pgfqpoint{4.768000in}{0.528000in}}%
\pgfpathlineto{\pgfqpoint{4.768000in}{4.224000in}}%
\pgfpathlineto{\pgfqpoint{1.072000in}{4.224000in}}%
\pgfpathlineto{\pgfqpoint{1.072000in}{0.528000in}}%
\pgfpathclose%
\pgfusepath{fill}%
\end{pgfscope}%
\begin{pgfscope}%
\pgfsetbuttcap%
\pgfsetmiterjoin%
\definecolor{currentfill}{rgb}{0.950000,0.950000,0.950000}%
\pgfsetfillcolor{currentfill}%
\pgfsetfillopacity{0.500000}%
\pgfsetlinewidth{1.003750pt}%
\definecolor{currentstroke}{rgb}{0.950000,0.950000,0.950000}%
\pgfsetstrokecolor{currentstroke}%
\pgfsetstrokeopacity{0.500000}%
\pgfsetdash{}{0pt}%
\pgfpathmoveto{\pgfqpoint{1.351073in}{1.439315in}}%
\pgfpathlineto{\pgfqpoint{2.571613in}{2.462396in}}%
\pgfpathlineto{\pgfqpoint{2.554647in}{3.937861in}}%
\pgfpathlineto{\pgfqpoint{1.275698in}{3.004542in}}%
\pgfusepath{stroke,fill}%
\end{pgfscope}%
\begin{pgfscope}%
\pgfsetbuttcap%
\pgfsetmiterjoin%
\definecolor{currentfill}{rgb}{0.900000,0.900000,0.900000}%
\pgfsetfillcolor{currentfill}%
\pgfsetfillopacity{0.500000}%
\pgfsetlinewidth{1.003750pt}%
\definecolor{currentstroke}{rgb}{0.900000,0.900000,0.900000}%
\pgfsetstrokecolor{currentstroke}%
\pgfsetstrokeopacity{0.500000}%
\pgfsetdash{}{0pt}%
\pgfpathmoveto{\pgfqpoint{2.571613in}{2.462396in}}%
\pgfpathlineto{\pgfqpoint{4.530144in}{1.893128in}}%
\pgfpathlineto{\pgfqpoint{4.600038in}{3.419414in}}%
\pgfpathlineto{\pgfqpoint{2.554647in}{3.937861in}}%
\pgfusepath{stroke,fill}%
\end{pgfscope}%
\begin{pgfscope}%
\pgfsetbuttcap%
\pgfsetmiterjoin%
\definecolor{currentfill}{rgb}{0.925000,0.925000,0.925000}%
\pgfsetfillcolor{currentfill}%
\pgfsetfillopacity{0.500000}%
\pgfsetlinewidth{1.003750pt}%
\definecolor{currentstroke}{rgb}{0.925000,0.925000,0.925000}%
\pgfsetstrokecolor{currentstroke}%
\pgfsetstrokeopacity{0.500000}%
\pgfsetdash{}{0pt}%
\pgfpathmoveto{\pgfqpoint{1.351073in}{1.439315in}}%
\pgfpathlineto{\pgfqpoint{3.427212in}{0.761248in}}%
\pgfpathlineto{\pgfqpoint{4.530144in}{1.893128in}}%
\pgfpathlineto{\pgfqpoint{2.571613in}{2.462396in}}%
\pgfusepath{stroke,fill}%
\end{pgfscope}%
\begin{pgfscope}%
\pgfsetbuttcap%
\pgfsetroundjoin%
\pgfsetlinewidth{0.803000pt}%
\definecolor{currentstroke}{rgb}{0.690196,0.690196,0.690196}%
\pgfsetstrokecolor{currentstroke}%
\pgfsetdash{}{0pt}%
\pgfpathmoveto{\pgfqpoint{1.655086in}{1.340025in}}%
\pgfpathlineto{\pgfqpoint{2.859486in}{2.378723in}}%
\pgfpathlineto{\pgfqpoint{2.854746in}{3.861795in}}%
\pgfusepath{stroke}%
\end{pgfscope}%
\begin{pgfscope}%
\pgfsetbuttcap%
\pgfsetroundjoin%
\pgfsetlinewidth{0.803000pt}%
\definecolor{currentstroke}{rgb}{0.690196,0.690196,0.690196}%
\pgfsetstrokecolor{currentstroke}%
\pgfsetdash{}{0pt}%
\pgfpathmoveto{\pgfqpoint{2.016630in}{1.221944in}}%
\pgfpathlineto{\pgfqpoint{3.201353in}{2.279356in}}%
\pgfpathlineto{\pgfqpoint{3.211373in}{3.771400in}}%
\pgfusepath{stroke}%
\end{pgfscope}%
\begin{pgfscope}%
\pgfsetbuttcap%
\pgfsetroundjoin%
\pgfsetlinewidth{0.803000pt}%
\definecolor{currentstroke}{rgb}{0.690196,0.690196,0.690196}%
\pgfsetstrokecolor{currentstroke}%
\pgfsetdash{}{0pt}%
\pgfpathmoveto{\pgfqpoint{2.385003in}{1.101634in}}%
\pgfpathlineto{\pgfqpoint{3.549136in}{2.178269in}}%
\pgfpathlineto{\pgfqpoint{3.574442in}{3.679373in}}%
\pgfusepath{stroke}%
\end{pgfscope}%
\begin{pgfscope}%
\pgfsetbuttcap%
\pgfsetroundjoin%
\pgfsetlinewidth{0.803000pt}%
\definecolor{currentstroke}{rgb}{0.690196,0.690196,0.690196}%
\pgfsetstrokecolor{currentstroke}%
\pgfsetdash{}{0pt}%
\pgfpathmoveto{\pgfqpoint{2.760401in}{0.979029in}}%
\pgfpathlineto{\pgfqpoint{3.902990in}{2.075417in}}%
\pgfpathlineto{\pgfqpoint{3.944129in}{3.585668in}}%
\pgfusepath{stroke}%
\end{pgfscope}%
\begin{pgfscope}%
\pgfsetbuttcap%
\pgfsetroundjoin%
\pgfsetlinewidth{0.803000pt}%
\definecolor{currentstroke}{rgb}{0.690196,0.690196,0.690196}%
\pgfsetstrokecolor{currentstroke}%
\pgfsetdash{}{0pt}%
\pgfpathmoveto{\pgfqpoint{3.143027in}{0.854063in}}%
\pgfpathlineto{\pgfqpoint{4.263078in}{1.970754in}}%
\pgfpathlineto{\pgfqpoint{4.320618in}{3.490239in}}%
\pgfusepath{stroke}%
\end{pgfscope}%
\begin{pgfscope}%
\pgfsetbuttcap%
\pgfsetroundjoin%
\pgfsetlinewidth{0.803000pt}%
\definecolor{currentstroke}{rgb}{0.690196,0.690196,0.690196}%
\pgfsetstrokecolor{currentstroke}%
\pgfsetdash{}{0pt}%
\pgfpathmoveto{\pgfqpoint{1.486794in}{3.158591in}}%
\pgfpathlineto{\pgfqpoint{1.551891in}{1.607644in}}%
\pgfpathlineto{\pgfqpoint{3.609350in}{0.948166in}}%
\pgfusepath{stroke}%
\end{pgfscope}%
\begin{pgfscope}%
\pgfsetbuttcap%
\pgfsetroundjoin%
\pgfsetlinewidth{0.803000pt}%
\definecolor{currentstroke}{rgb}{0.690196,0.690196,0.690196}%
\pgfsetstrokecolor{currentstroke}%
\pgfsetdash{}{0pt}%
\pgfpathmoveto{\pgfqpoint{1.726150in}{3.333262in}}%
\pgfpathlineto{\pgfqpoint{1.779897in}{1.798764in}}%
\pgfpathlineto{\pgfqpoint{3.815826in}{1.160062in}}%
\pgfusepath{stroke}%
\end{pgfscope}%
\begin{pgfscope}%
\pgfsetbuttcap%
\pgfsetroundjoin%
\pgfsetlinewidth{0.803000pt}%
\definecolor{currentstroke}{rgb}{0.690196,0.690196,0.690196}%
\pgfsetstrokecolor{currentstroke}%
\pgfsetdash{}{0pt}%
\pgfpathmoveto{\pgfqpoint{1.957889in}{3.502375in}}%
\pgfpathlineto{\pgfqpoint{2.000956in}{1.984060in}}%
\pgfpathlineto{\pgfqpoint{4.015686in}{1.365167in}}%
\pgfusepath{stroke}%
\end{pgfscope}%
\begin{pgfscope}%
\pgfsetbuttcap%
\pgfsetroundjoin%
\pgfsetlinewidth{0.803000pt}%
\definecolor{currentstroke}{rgb}{0.690196,0.690196,0.690196}%
\pgfsetstrokecolor{currentstroke}%
\pgfsetdash{}{0pt}%
\pgfpathmoveto{\pgfqpoint{2.182368in}{3.666189in}}%
\pgfpathlineto{\pgfqpoint{2.215380in}{2.163794in}}%
\pgfpathlineto{\pgfqpoint{4.209242in}{1.563803in}}%
\pgfusepath{stroke}%
\end{pgfscope}%
\begin{pgfscope}%
\pgfsetbuttcap%
\pgfsetroundjoin%
\pgfsetlinewidth{0.803000pt}%
\definecolor{currentstroke}{rgb}{0.690196,0.690196,0.690196}%
\pgfsetstrokecolor{currentstroke}%
\pgfsetdash{}{0pt}%
\pgfpathmoveto{\pgfqpoint{2.399923in}{3.824951in}}%
\pgfpathlineto{\pgfqpoint{2.423463in}{2.338213in}}%
\pgfpathlineto{\pgfqpoint{4.396788in}{1.756271in}}%
\pgfusepath{stroke}%
\end{pgfscope}%
\begin{pgfscope}%
\pgfsetbuttcap%
\pgfsetroundjoin%
\pgfsetlinewidth{0.803000pt}%
\definecolor{currentstroke}{rgb}{0.690196,0.690196,0.690196}%
\pgfsetstrokecolor{currentstroke}%
\pgfsetdash{}{0pt}%
\pgfpathmoveto{\pgfqpoint{4.531484in}{1.922370in}}%
\pgfpathlineto{\pgfqpoint{2.571288in}{2.490723in}}%
\pgfpathlineto{\pgfqpoint{1.349632in}{1.469254in}}%
\pgfusepath{stroke}%
\end{pgfscope}%
\begin{pgfscope}%
\pgfsetbuttcap%
\pgfsetroundjoin%
\pgfsetlinewidth{0.803000pt}%
\definecolor{currentstroke}{rgb}{0.690196,0.690196,0.690196}%
\pgfsetstrokecolor{currentstroke}%
\pgfsetdash{}{0pt}%
\pgfpathmoveto{\pgfqpoint{4.538664in}{2.079179in}}%
\pgfpathlineto{\pgfqpoint{2.569541in}{2.642585in}}%
\pgfpathlineto{\pgfqpoint{1.341899in}{1.629830in}}%
\pgfusepath{stroke}%
\end{pgfscope}%
\begin{pgfscope}%
\pgfsetbuttcap%
\pgfsetroundjoin%
\pgfsetlinewidth{0.803000pt}%
\definecolor{currentstroke}{rgb}{0.690196,0.690196,0.690196}%
\pgfsetstrokecolor{currentstroke}%
\pgfsetdash{}{0pt}%
\pgfpathmoveto{\pgfqpoint{4.545911in}{2.237436in}}%
\pgfpathlineto{\pgfqpoint{2.567780in}{2.795784in}}%
\pgfpathlineto{\pgfqpoint{1.334092in}{1.791947in}}%
\pgfusepath{stroke}%
\end{pgfscope}%
\begin{pgfscope}%
\pgfsetbuttcap%
\pgfsetroundjoin%
\pgfsetlinewidth{0.803000pt}%
\definecolor{currentstroke}{rgb}{0.690196,0.690196,0.690196}%
\pgfsetstrokecolor{currentstroke}%
\pgfsetdash{}{0pt}%
\pgfpathmoveto{\pgfqpoint{4.553226in}{2.397163in}}%
\pgfpathlineto{\pgfqpoint{2.566002in}{2.950337in}}%
\pgfpathlineto{\pgfqpoint{1.326210in}{1.955626in}}%
\pgfusepath{stroke}%
\end{pgfscope}%
\begin{pgfscope}%
\pgfsetbuttcap%
\pgfsetroundjoin%
\pgfsetlinewidth{0.803000pt}%
\definecolor{currentstroke}{rgb}{0.690196,0.690196,0.690196}%
\pgfsetstrokecolor{currentstroke}%
\pgfsetdash{}{0pt}%
\pgfpathmoveto{\pgfqpoint{4.560608in}{2.558379in}}%
\pgfpathlineto{\pgfqpoint{2.564209in}{3.106262in}}%
\pgfpathlineto{\pgfqpoint{1.318251in}{2.120890in}}%
\pgfusepath{stroke}%
\end{pgfscope}%
\begin{pgfscope}%
\pgfsetbuttcap%
\pgfsetroundjoin%
\pgfsetlinewidth{0.803000pt}%
\definecolor{currentstroke}{rgb}{0.690196,0.690196,0.690196}%
\pgfsetstrokecolor{currentstroke}%
\pgfsetdash{}{0pt}%
\pgfpathmoveto{\pgfqpoint{4.568060in}{2.721106in}}%
\pgfpathlineto{\pgfqpoint{2.562400in}{3.263577in}}%
\pgfpathlineto{\pgfqpoint{1.310215in}{2.287762in}}%
\pgfusepath{stroke}%
\end{pgfscope}%
\begin{pgfscope}%
\pgfsetbuttcap%
\pgfsetroundjoin%
\pgfsetlinewidth{0.803000pt}%
\definecolor{currentstroke}{rgb}{0.690196,0.690196,0.690196}%
\pgfsetstrokecolor{currentstroke}%
\pgfsetdash{}{0pt}%
\pgfpathmoveto{\pgfqpoint{4.575582in}{2.885364in}}%
\pgfpathlineto{\pgfqpoint{2.560575in}{3.422302in}}%
\pgfpathlineto{\pgfqpoint{1.302101in}{2.456266in}}%
\pgfusepath{stroke}%
\end{pgfscope}%
\begin{pgfscope}%
\pgfsetbuttcap%
\pgfsetroundjoin%
\pgfsetlinewidth{0.803000pt}%
\definecolor{currentstroke}{rgb}{0.690196,0.690196,0.690196}%
\pgfsetstrokecolor{currentstroke}%
\pgfsetdash{}{0pt}%
\pgfpathmoveto{\pgfqpoint{4.583175in}{3.051176in}}%
\pgfpathlineto{\pgfqpoint{2.558734in}{3.582454in}}%
\pgfpathlineto{\pgfqpoint{1.293906in}{2.626426in}}%
\pgfusepath{stroke}%
\end{pgfscope}%
\begin{pgfscope}%
\pgfsetbuttcap%
\pgfsetroundjoin%
\pgfsetlinewidth{0.803000pt}%
\definecolor{currentstroke}{rgb}{0.690196,0.690196,0.690196}%
\pgfsetstrokecolor{currentstroke}%
\pgfsetdash{}{0pt}%
\pgfpathmoveto{\pgfqpoint{4.590840in}{3.218564in}}%
\pgfpathlineto{\pgfqpoint{2.556875in}{3.744054in}}%
\pgfpathlineto{\pgfqpoint{1.285631in}{2.798266in}}%
\pgfusepath{stroke}%
\end{pgfscope}%
\begin{pgfscope}%
\pgfsetbuttcap%
\pgfsetroundjoin%
\pgfsetlinewidth{0.803000pt}%
\definecolor{currentstroke}{rgb}{0.690196,0.690196,0.690196}%
\pgfsetstrokecolor{currentstroke}%
\pgfsetdash{}{0pt}%
\pgfpathmoveto{\pgfqpoint{4.598578in}{3.387550in}}%
\pgfpathlineto{\pgfqpoint{2.555000in}{3.907121in}}%
\pgfpathlineto{\pgfqpoint{1.277274in}{2.971811in}}%
\pgfusepath{stroke}%
\end{pgfscope}%
\begin{pgfscope}%
\pgfsetrectcap%
\pgfsetroundjoin%
\pgfsetlinewidth{0.803000pt}%
\definecolor{currentstroke}{rgb}{0.000000,0.000000,0.000000}%
\pgfsetstrokecolor{currentstroke}%
\pgfsetdash{}{0pt}%
\pgfpathmoveto{\pgfqpoint{1.351073in}{1.439315in}}%
\pgfpathlineto{\pgfqpoint{3.427212in}{0.761248in}}%
\pgfusepath{stroke}%
\end{pgfscope}%
\begin{pgfscope}%
\pgfsetrectcap%
\pgfsetroundjoin%
\pgfsetlinewidth{0.803000pt}%
\definecolor{currentstroke}{rgb}{0.000000,0.000000,0.000000}%
\pgfsetstrokecolor{currentstroke}%
\pgfsetdash{}{0pt}%
\pgfpathmoveto{\pgfqpoint{1.665577in}{1.349073in}}%
\pgfpathlineto{\pgfqpoint{1.634057in}{1.321889in}}%
\pgfusepath{stroke}%
\end{pgfscope}%
\begin{pgfscope}%
\definecolor{textcolor}{rgb}{0.000000,0.000000,0.000000}%
\pgfsetstrokecolor{textcolor}%
\pgfsetfillcolor{textcolor}%
\pgftext[x=1.550695in,y=1.120588in,,top]{\color{textcolor}{\rmfamily\fontsize{10.000000}{12.000000}\selectfont\catcode`\^=\active\def^{\ifmmode\sp\else\^{}\fi}\catcode`\%=\active\def%{\%}$\mathdefault{\ensuremath{-}4}$}}%
\end{pgfscope}%
\begin{pgfscope}%
\pgfsetrectcap%
\pgfsetroundjoin%
\pgfsetlinewidth{0.803000pt}%
\definecolor{currentstroke}{rgb}{0.000000,0.000000,0.000000}%
\pgfsetstrokecolor{currentstroke}%
\pgfsetdash{}{0pt}%
\pgfpathmoveto{\pgfqpoint{2.026958in}{1.231163in}}%
\pgfpathlineto{\pgfqpoint{1.995928in}{1.203468in}}%
\pgfusepath{stroke}%
\end{pgfscope}%
\begin{pgfscope}%
\definecolor{textcolor}{rgb}{0.000000,0.000000,0.000000}%
\pgfsetstrokecolor{textcolor}%
\pgfsetfillcolor{textcolor}%
\pgftext[x=1.912621in,y=1.000002in,,top]{\color{textcolor}{\rmfamily\fontsize{10.000000}{12.000000}\selectfont\catcode`\^=\active\def^{\ifmmode\sp\else\^{}\fi}\catcode`\%=\active\def%{\%}$\mathdefault{\ensuremath{-}2}$}}%
\end{pgfscope}%
\begin{pgfscope}%
\pgfsetrectcap%
\pgfsetroundjoin%
\pgfsetlinewidth{0.803000pt}%
\definecolor{currentstroke}{rgb}{0.000000,0.000000,0.000000}%
\pgfsetstrokecolor{currentstroke}%
\pgfsetdash{}{0pt}%
\pgfpathmoveto{\pgfqpoint{2.395160in}{1.111027in}}%
\pgfpathlineto{\pgfqpoint{2.364645in}{1.082806in}}%
\pgfusepath{stroke}%
\end{pgfscope}%
\begin{pgfscope}%
\definecolor{textcolor}{rgb}{0.000000,0.000000,0.000000}%
\pgfsetstrokecolor{textcolor}%
\pgfsetfillcolor{textcolor}%
\pgftext[x=2.281411in,y=0.877129in,,top]{\color{textcolor}{\rmfamily\fontsize{10.000000}{12.000000}\selectfont\catcode`\^=\active\def^{\ifmmode\sp\else\^{}\fi}\catcode`\%=\active\def%{\%}$\mathdefault{0}$}}%
\end{pgfscope}%
\begin{pgfscope}%
\pgfsetrectcap%
\pgfsetroundjoin%
\pgfsetlinewidth{0.803000pt}%
\definecolor{currentstroke}{rgb}{0.000000,0.000000,0.000000}%
\pgfsetstrokecolor{currentstroke}%
\pgfsetdash{}{0pt}%
\pgfpathmoveto{\pgfqpoint{2.770378in}{0.988602in}}%
\pgfpathlineto{\pgfqpoint{2.740404in}{0.959840in}}%
\pgfusepath{stroke}%
\end{pgfscope}%
\begin{pgfscope}%
\definecolor{textcolor}{rgb}{0.000000,0.000000,0.000000}%
\pgfsetstrokecolor{textcolor}%
\pgfsetfillcolor{textcolor}%
\pgftext[x=2.657259in,y=0.751904in,,top]{\color{textcolor}{\rmfamily\fontsize{10.000000}{12.000000}\selectfont\catcode`\^=\active\def^{\ifmmode\sp\else\^{}\fi}\catcode`\%=\active\def%{\%}$\mathdefault{2}$}}%
\end{pgfscope}%
\begin{pgfscope}%
\pgfsetrectcap%
\pgfsetroundjoin%
\pgfsetlinewidth{0.803000pt}%
\definecolor{currentstroke}{rgb}{0.000000,0.000000,0.000000}%
\pgfsetstrokecolor{currentstroke}%
\pgfsetdash{}{0pt}%
\pgfpathmoveto{\pgfqpoint{3.152815in}{0.863821in}}%
\pgfpathlineto{\pgfqpoint{3.123408in}{0.834503in}}%
\pgfusepath{stroke}%
\end{pgfscope}%
\begin{pgfscope}%
\definecolor{textcolor}{rgb}{0.000000,0.000000,0.000000}%
\pgfsetstrokecolor{textcolor}%
\pgfsetfillcolor{textcolor}%
\pgftext[x=3.040373in,y=0.624259in,,top]{\color{textcolor}{\rmfamily\fontsize{10.000000}{12.000000}\selectfont\catcode`\^=\active\def^{\ifmmode\sp\else\^{}\fi}\catcode`\%=\active\def%{\%}$\mathdefault{4}$}}%
\end{pgfscope}%
\begin{pgfscope}%
\pgfsetrectcap%
\pgfsetroundjoin%
\pgfsetlinewidth{0.803000pt}%
\definecolor{currentstroke}{rgb}{0.000000,0.000000,0.000000}%
\pgfsetstrokecolor{currentstroke}%
\pgfsetdash{}{0pt}%
\pgfpathmoveto{\pgfqpoint{4.530144in}{1.893128in}}%
\pgfpathlineto{\pgfqpoint{3.427212in}{0.761248in}}%
\pgfusepath{stroke}%
\end{pgfscope}%
\begin{pgfscope}%
\pgfsetrectcap%
\pgfsetroundjoin%
\pgfsetlinewidth{0.803000pt}%
\definecolor{currentstroke}{rgb}{0.000000,0.000000,0.000000}%
\pgfsetstrokecolor{currentstroke}%
\pgfsetdash{}{0pt}%
\pgfpathmoveto{\pgfqpoint{3.592019in}{0.953721in}}%
\pgfpathlineto{\pgfqpoint{3.644056in}{0.937042in}}%
\pgfusepath{stroke}%
\end{pgfscope}%
\begin{pgfscope}%
\definecolor{textcolor}{rgb}{0.000000,0.000000,0.000000}%
\pgfsetstrokecolor{textcolor}%
\pgfsetfillcolor{textcolor}%
\pgftext[x=3.786829in,y=0.763340in,,top]{\color{textcolor}{\rmfamily\fontsize{10.000000}{12.000000}\selectfont\catcode`\^=\active\def^{\ifmmode\sp\else\^{}\fi}\catcode`\%=\active\def%{\%}$\mathdefault{\ensuremath{-}4}$}}%
\end{pgfscope}%
\begin{pgfscope}%
\pgfsetrectcap%
\pgfsetroundjoin%
\pgfsetlinewidth{0.803000pt}%
\definecolor{currentstroke}{rgb}{0.000000,0.000000,0.000000}%
\pgfsetstrokecolor{currentstroke}%
\pgfsetdash{}{0pt}%
\pgfpathmoveto{\pgfqpoint{3.798691in}{1.165438in}}%
\pgfpathlineto{\pgfqpoint{3.850141in}{1.149297in}}%
\pgfusepath{stroke}%
\end{pgfscope}%
\begin{pgfscope}%
\definecolor{textcolor}{rgb}{0.000000,0.000000,0.000000}%
\pgfsetstrokecolor{textcolor}%
\pgfsetfillcolor{textcolor}%
\pgftext[x=3.990534in,y=0.978371in,,top]{\color{textcolor}{\rmfamily\fontsize{10.000000}{12.000000}\selectfont\catcode`\^=\active\def^{\ifmmode\sp\else\^{}\fi}\catcode`\%=\active\def%{\%}$\mathdefault{\ensuremath{-}2}$}}%
\end{pgfscope}%
\begin{pgfscope}%
\pgfsetrectcap%
\pgfsetroundjoin%
\pgfsetlinewidth{0.803000pt}%
\definecolor{currentstroke}{rgb}{0.000000,0.000000,0.000000}%
\pgfsetstrokecolor{currentstroke}%
\pgfsetdash{}{0pt}%
\pgfpathmoveto{\pgfqpoint{3.998742in}{1.370372in}}%
\pgfpathlineto{\pgfqpoint{4.049616in}{1.354745in}}%
\pgfusepath{stroke}%
\end{pgfscope}%
\begin{pgfscope}%
\definecolor{textcolor}{rgb}{0.000000,0.000000,0.000000}%
\pgfsetstrokecolor{textcolor}%
\pgfsetfillcolor{textcolor}%
\pgftext[x=4.187708in,y=1.186507in,,top]{\color{textcolor}{\rmfamily\fontsize{10.000000}{12.000000}\selectfont\catcode`\^=\active\def^{\ifmmode\sp\else\^{}\fi}\catcode`\%=\active\def%{\%}$\mathdefault{0}$}}%
\end{pgfscope}%
\begin{pgfscope}%
\pgfsetrectcap%
\pgfsetroundjoin%
\pgfsetlinewidth{0.803000pt}%
\definecolor{currentstroke}{rgb}{0.000000,0.000000,0.000000}%
\pgfsetstrokecolor{currentstroke}%
\pgfsetdash{}{0pt}%
\pgfpathmoveto{\pgfqpoint{4.192487in}{1.568845in}}%
\pgfpathlineto{\pgfqpoint{4.242794in}{1.553707in}}%
\pgfusepath{stroke}%
\end{pgfscope}%
\begin{pgfscope}%
\definecolor{textcolor}{rgb}{0.000000,0.000000,0.000000}%
\pgfsetstrokecolor{textcolor}%
\pgfsetfillcolor{textcolor}%
\pgftext[x=4.378659in,y=1.388075in,,top]{\color{textcolor}{\rmfamily\fontsize{10.000000}{12.000000}\selectfont\catcode`\^=\active\def^{\ifmmode\sp\else\^{}\fi}\catcode`\%=\active\def%{\%}$\mathdefault{2}$}}%
\end{pgfscope}%
\begin{pgfscope}%
\pgfsetrectcap%
\pgfsetroundjoin%
\pgfsetlinewidth{0.803000pt}%
\definecolor{currentstroke}{rgb}{0.000000,0.000000,0.000000}%
\pgfsetstrokecolor{currentstroke}%
\pgfsetdash{}{0pt}%
\pgfpathmoveto{\pgfqpoint{4.380217in}{1.761158in}}%
\pgfpathlineto{\pgfqpoint{4.429968in}{1.746486in}}%
\pgfusepath{stroke}%
\end{pgfscope}%
\begin{pgfscope}%
\definecolor{textcolor}{rgb}{0.000000,0.000000,0.000000}%
\pgfsetstrokecolor{textcolor}%
\pgfsetfillcolor{textcolor}%
\pgftext[x=4.563677in,y=1.583380in,,top]{\color{textcolor}{\rmfamily\fontsize{10.000000}{12.000000}\selectfont\catcode`\^=\active\def^{\ifmmode\sp\else\^{}\fi}\catcode`\%=\active\def%{\%}$\mathdefault{4}$}}%
\end{pgfscope}%
\begin{pgfscope}%
\pgfsetrectcap%
\pgfsetroundjoin%
\pgfsetlinewidth{0.803000pt}%
\definecolor{currentstroke}{rgb}{0.000000,0.000000,0.000000}%
\pgfsetstrokecolor{currentstroke}%
\pgfsetdash{}{0pt}%
\pgfpathmoveto{\pgfqpoint{4.530144in}{1.893128in}}%
\pgfpathlineto{\pgfqpoint{4.600038in}{3.419414in}}%
\pgfusepath{stroke}%
\end{pgfscope}%
\begin{pgfscope}%
\pgfsetrectcap%
\pgfsetroundjoin%
\pgfsetlinewidth{0.803000pt}%
\definecolor{currentstroke}{rgb}{0.000000,0.000000,0.000000}%
\pgfsetstrokecolor{currentstroke}%
\pgfsetdash{}{0pt}%
\pgfpathmoveto{\pgfqpoint{4.515031in}{1.927140in}}%
\pgfpathlineto{\pgfqpoint{4.564427in}{1.912818in}}%
\pgfusepath{stroke}%
\end{pgfscope}%
\begin{pgfscope}%
\definecolor{textcolor}{rgb}{0.000000,0.000000,0.000000}%
\pgfsetstrokecolor{textcolor}%
\pgfsetfillcolor{textcolor}%
\pgftext[x=4.785150in,y=1.958494in,,top]{\color{textcolor}{\rmfamily\fontsize{10.000000}{12.000000}\selectfont\catcode`\^=\active\def^{\ifmmode\sp\else\^{}\fi}\catcode`\%=\active\def%{\%}-1.01}}%
\end{pgfscope}%
\begin{pgfscope}%
\pgfsetrectcap%
\pgfsetroundjoin%
\pgfsetlinewidth{0.803000pt}%
\definecolor{currentstroke}{rgb}{0.000000,0.000000,0.000000}%
\pgfsetstrokecolor{currentstroke}%
\pgfsetdash{}{0pt}%
\pgfpathmoveto{\pgfqpoint{4.522134in}{2.083909in}}%
\pgfpathlineto{\pgfqpoint{4.571765in}{2.069708in}}%
\pgfusepath{stroke}%
\end{pgfscope}%
\begin{pgfscope}%
\definecolor{textcolor}{rgb}{0.000000,0.000000,0.000000}%
\pgfsetstrokecolor{textcolor}%
\pgfsetfillcolor{textcolor}%
\pgftext[x=4.793472in,y=2.114996in,,top]{\color{textcolor}{\rmfamily\fontsize{10.000000}{12.000000}\selectfont\catcode`\^=\active\def^{\ifmmode\sp\else\^{}\fi}\catcode`\%=\active\def%{\%}-0.79}}%
\end{pgfscope}%
\begin{pgfscope}%
\pgfsetrectcap%
\pgfsetroundjoin%
\pgfsetlinewidth{0.803000pt}%
\definecolor{currentstroke}{rgb}{0.000000,0.000000,0.000000}%
\pgfsetstrokecolor{currentstroke}%
\pgfsetdash{}{0pt}%
\pgfpathmoveto{\pgfqpoint{4.529302in}{2.242125in}}%
\pgfpathlineto{\pgfqpoint{4.579171in}{2.228049in}}%
\pgfusepath{stroke}%
\end{pgfscope}%
\begin{pgfscope}%
\definecolor{textcolor}{rgb}{0.000000,0.000000,0.000000}%
\pgfsetstrokecolor{textcolor}%
\pgfsetfillcolor{textcolor}%
\pgftext[x=4.801870in,y=2.272939in,,top]{\color{textcolor}{\rmfamily\fontsize{10.000000}{12.000000}\selectfont\catcode`\^=\active\def^{\ifmmode\sp\else\^{}\fi}\catcode`\%=\active\def%{\%}-0.56}}%
\end{pgfscope}%
\begin{pgfscope}%
\pgfsetrectcap%
\pgfsetroundjoin%
\pgfsetlinewidth{0.803000pt}%
\definecolor{currentstroke}{rgb}{0.000000,0.000000,0.000000}%
\pgfsetstrokecolor{currentstroke}%
\pgfsetdash{}{0pt}%
\pgfpathmoveto{\pgfqpoint{4.536536in}{2.401809in}}%
\pgfpathlineto{\pgfqpoint{4.586646in}{2.387860in}}%
\pgfusepath{stroke}%
\end{pgfscope}%
\begin{pgfscope}%
\definecolor{textcolor}{rgb}{0.000000,0.000000,0.000000}%
\pgfsetstrokecolor{textcolor}%
\pgfsetfillcolor{textcolor}%
\pgftext[x=4.810347in,y=2.432344in,,top]{\color{textcolor}{\rmfamily\fontsize{10.000000}{12.000000}\selectfont\catcode`\^=\active\def^{\ifmmode\sp\else\^{}\fi}\catcode`\%=\active\def%{\%}-0.34}}%
\end{pgfscope}%
\begin{pgfscope}%
\pgfsetrectcap%
\pgfsetroundjoin%
\pgfsetlinewidth{0.803000pt}%
\definecolor{currentstroke}{rgb}{0.000000,0.000000,0.000000}%
\pgfsetstrokecolor{currentstroke}%
\pgfsetdash{}{0pt}%
\pgfpathmoveto{\pgfqpoint{4.543838in}{2.562982in}}%
\pgfpathlineto{\pgfqpoint{4.594190in}{2.549163in}}%
\pgfusepath{stroke}%
\end{pgfscope}%
\begin{pgfscope}%
\definecolor{textcolor}{rgb}{0.000000,0.000000,0.000000}%
\pgfsetstrokecolor{textcolor}%
\pgfsetfillcolor{textcolor}%
\pgftext[x=4.818902in,y=2.593231in,,top]{\color{textcolor}{\rmfamily\fontsize{10.000000}{12.000000}\selectfont\catcode`\^=\active\def^{\ifmmode\sp\else\^{}\fi}\catcode`\%=\active\def%{\%}-0.11}}%
\end{pgfscope}%
\begin{pgfscope}%
\pgfsetrectcap%
\pgfsetroundjoin%
\pgfsetlinewidth{0.803000pt}%
\definecolor{currentstroke}{rgb}{0.000000,0.000000,0.000000}%
\pgfsetstrokecolor{currentstroke}%
\pgfsetdash{}{0pt}%
\pgfpathmoveto{\pgfqpoint{4.551208in}{2.725664in}}%
\pgfpathlineto{\pgfqpoint{4.601805in}{2.711979in}}%
\pgfusepath{stroke}%
\end{pgfscope}%
\begin{pgfscope}%
\definecolor{textcolor}{rgb}{0.000000,0.000000,0.000000}%
\pgfsetstrokecolor{textcolor}%
\pgfsetfillcolor{textcolor}%
\pgftext[x=4.827537in,y=2.755620in,,top]{\color{textcolor}{\rmfamily\fontsize{10.000000}{12.000000}\selectfont\catcode`\^=\active\def^{\ifmmode\sp\else\^{}\fi}\catcode`\%=\active\def%{\%}0.11}}%
\end{pgfscope}%
\begin{pgfscope}%
\pgfsetrectcap%
\pgfsetroundjoin%
\pgfsetlinewidth{0.803000pt}%
\definecolor{currentstroke}{rgb}{0.000000,0.000000,0.000000}%
\pgfsetstrokecolor{currentstroke}%
\pgfsetdash{}{0pt}%
\pgfpathmoveto{\pgfqpoint{4.558647in}{2.889877in}}%
\pgfpathlineto{\pgfqpoint{4.609492in}{2.876328in}}%
\pgfusepath{stroke}%
\end{pgfscope}%
\begin{pgfscope}%
\definecolor{textcolor}{rgb}{0.000000,0.000000,0.000000}%
\pgfsetstrokecolor{textcolor}%
\pgfsetfillcolor{textcolor}%
\pgftext[x=4.836253in,y=2.919534in,,top]{\color{textcolor}{\rmfamily\fontsize{10.000000}{12.000000}\selectfont\catcode`\^=\active\def^{\ifmmode\sp\else\^{}\fi}\catcode`\%=\active\def%{\%}0.34}}%
\end{pgfscope}%
\begin{pgfscope}%
\pgfsetrectcap%
\pgfsetroundjoin%
\pgfsetlinewidth{0.803000pt}%
\definecolor{currentstroke}{rgb}{0.000000,0.000000,0.000000}%
\pgfsetstrokecolor{currentstroke}%
\pgfsetdash{}{0pt}%
\pgfpathmoveto{\pgfqpoint{4.566157in}{3.055642in}}%
\pgfpathlineto{\pgfqpoint{4.617252in}{3.042233in}}%
\pgfusepath{stroke}%
\end{pgfscope}%
\begin{pgfscope}%
\definecolor{textcolor}{rgb}{0.000000,0.000000,0.000000}%
\pgfsetstrokecolor{textcolor}%
\pgfsetfillcolor{textcolor}%
\pgftext[x=4.845052in,y=3.084993in,,top]{\color{textcolor}{\rmfamily\fontsize{10.000000}{12.000000}\selectfont\catcode`\^=\active\def^{\ifmmode\sp\else\^{}\fi}\catcode`\%=\active\def%{\%}0.56}}%
\end{pgfscope}%
\begin{pgfscope}%
\pgfsetrectcap%
\pgfsetroundjoin%
\pgfsetlinewidth{0.803000pt}%
\definecolor{currentstroke}{rgb}{0.000000,0.000000,0.000000}%
\pgfsetstrokecolor{currentstroke}%
\pgfsetdash{}{0pt}%
\pgfpathmoveto{\pgfqpoint{4.573738in}{3.222982in}}%
\pgfpathlineto{\pgfqpoint{4.625086in}{3.209716in}}%
\pgfusepath{stroke}%
\end{pgfscope}%
\begin{pgfscope}%
\definecolor{textcolor}{rgb}{0.000000,0.000000,0.000000}%
\pgfsetstrokecolor{textcolor}%
\pgfsetfillcolor{textcolor}%
\pgftext[x=4.853934in,y=3.252019in,,top]{\color{textcolor}{\rmfamily\fontsize{10.000000}{12.000000}\selectfont\catcode`\^=\active\def^{\ifmmode\sp\else\^{}\fi}\catcode`\%=\active\def%{\%}0.79}}%
\end{pgfscope}%
\begin{pgfscope}%
\pgfsetrectcap%
\pgfsetroundjoin%
\pgfsetlinewidth{0.803000pt}%
\definecolor{currentstroke}{rgb}{0.000000,0.000000,0.000000}%
\pgfsetstrokecolor{currentstroke}%
\pgfsetdash{}{0pt}%
\pgfpathmoveto{\pgfqpoint{4.581392in}{3.391919in}}%
\pgfpathlineto{\pgfqpoint{4.632994in}{3.378800in}}%
\pgfusepath{stroke}%
\end{pgfscope}%
\begin{pgfscope}%
\definecolor{textcolor}{rgb}{0.000000,0.000000,0.000000}%
\pgfsetstrokecolor{textcolor}%
\pgfsetfillcolor{textcolor}%
\pgftext[x=4.862900in,y=3.420636in,,top]{\color{textcolor}{\rmfamily\fontsize{10.000000}{12.000000}\selectfont\catcode`\^=\active\def^{\ifmmode\sp\else\^{}\fi}\catcode`\%=\active\def%{\%}1.01}}%
\end{pgfscope}%
\begin{pgfscope}%
\pgfpathrectangle{\pgfqpoint{1.072000in}{0.528000in}}{\pgfqpoint{3.696000in}{3.696000in}}%
\pgfusepath{clip}%
\pgfsetbuttcap%
\pgfsetroundjoin%
\definecolor{currentfill}{rgb}{0.271104,0.360011,0.807095}%
\pgfsetfillcolor{currentfill}%
\pgfsetlinewidth{0.000000pt}%
\definecolor{currentstroke}{rgb}{0.000000,0.000000,0.000000}%
\pgfsetstrokecolor{currentstroke}%
\pgfsetdash{}{0pt}%
\pgfpathmoveto{\pgfqpoint{2.952824in}{2.288451in}}%
\pgfpathlineto{\pgfqpoint{2.996030in}{2.250611in}}%
\pgfpathlineto{\pgfqpoint{3.020954in}{2.320147in}}%
\pgfpathlineto{\pgfqpoint{2.977764in}{2.373998in}}%
\pgfpathlineto{\pgfqpoint{2.952824in}{2.288451in}}%
\pgfpathclose%
\pgfusepath{fill}%
\end{pgfscope}%
\begin{pgfscope}%
\pgfpathrectangle{\pgfqpoint{1.072000in}{0.528000in}}{\pgfqpoint{3.696000in}{3.696000in}}%
\pgfusepath{clip}%
\pgfsetbuttcap%
\pgfsetroundjoin%
\definecolor{currentfill}{rgb}{0.294718,0.393542,0.834384}%
\pgfsetfillcolor{currentfill}%
\pgfsetlinewidth{0.000000pt}%
\definecolor{currentstroke}{rgb}{0.000000,0.000000,0.000000}%
\pgfsetstrokecolor{currentstroke}%
\pgfsetdash{}{0pt}%
\pgfpathmoveto{\pgfqpoint{3.020954in}{2.320147in}}%
\pgfpathlineto{\pgfqpoint{3.064167in}{2.278109in}}%
\pgfpathlineto{\pgfqpoint{3.089108in}{2.364221in}}%
\pgfpathlineto{\pgfqpoint{3.045920in}{2.420064in}}%
\pgfpathlineto{\pgfqpoint{3.020954in}{2.320147in}}%
\pgfpathclose%
\pgfusepath{fill}%
\end{pgfscope}%
\begin{pgfscope}%
\pgfpathrectangle{\pgfqpoint{1.072000in}{0.528000in}}{\pgfqpoint{3.696000in}{3.696000in}}%
\pgfusepath{clip}%
\pgfsetbuttcap%
\pgfsetroundjoin%
\definecolor{currentfill}{rgb}{0.309060,0.413498,0.850128}%
\pgfsetfillcolor{currentfill}%
\pgfsetlinewidth{0.000000pt}%
\definecolor{currentstroke}{rgb}{0.000000,0.000000,0.000000}%
\pgfsetstrokecolor{currentstroke}%
\pgfsetdash{}{0pt}%
\pgfpathmoveto{\pgfqpoint{2.909623in}{2.340155in}}%
\pgfpathlineto{\pgfqpoint{2.952824in}{2.288451in}}%
\pgfpathlineto{\pgfqpoint{2.977764in}{2.373998in}}%
\pgfpathlineto{\pgfqpoint{2.934550in}{2.440882in}}%
\pgfpathlineto{\pgfqpoint{2.909623in}{2.340155in}}%
\pgfpathclose%
\pgfusepath{fill}%
\end{pgfscope}%
\begin{pgfscope}%
\pgfpathrectangle{\pgfqpoint{1.072000in}{0.528000in}}{\pgfqpoint{3.696000in}{3.696000in}}%
\pgfusepath{clip}%
\pgfsetbuttcap%
\pgfsetroundjoin%
\definecolor{currentfill}{rgb}{0.333490,0.446265,0.874452}%
\pgfsetfillcolor{currentfill}%
\pgfsetlinewidth{0.000000pt}%
\definecolor{currentstroke}{rgb}{0.000000,0.000000,0.000000}%
\pgfsetstrokecolor{currentstroke}%
\pgfsetdash{}{0pt}%
\pgfpathmoveto{\pgfqpoint{2.977764in}{2.373998in}}%
\pgfpathlineto{\pgfqpoint{3.020954in}{2.320147in}}%
\pgfpathlineto{\pgfqpoint{3.045920in}{2.420064in}}%
\pgfpathlineto{\pgfqpoint{3.002728in}{2.487134in}}%
\pgfpathlineto{\pgfqpoint{2.977764in}{2.373998in}}%
\pgfpathclose%
\pgfusepath{fill}%
\end{pgfscope}%
\begin{pgfscope}%
\pgfpathrectangle{\pgfqpoint{1.072000in}{0.528000in}}{\pgfqpoint{3.696000in}{3.696000in}}%
\pgfusepath{clip}%
\pgfsetbuttcap%
\pgfsetroundjoin%
\definecolor{currentfill}{rgb}{0.271104,0.360011,0.807095}%
\pgfsetfillcolor{currentfill}%
\pgfsetlinewidth{0.000000pt}%
\definecolor{currentstroke}{rgb}{0.000000,0.000000,0.000000}%
\pgfsetstrokecolor{currentstroke}%
\pgfsetdash{}{0pt}%
\pgfpathmoveto{\pgfqpoint{3.064167in}{2.278109in}}%
\pgfpathlineto{\pgfqpoint{3.107440in}{2.246290in}}%
\pgfpathlineto{\pgfqpoint{3.132329in}{2.318541in}}%
\pgfpathlineto{\pgfqpoint{3.089108in}{2.364221in}}%
\pgfpathlineto{\pgfqpoint{3.064167in}{2.278109in}}%
\pgfpathclose%
\pgfusepath{fill}%
\end{pgfscope}%
\begin{pgfscope}%
\pgfpathrectangle{\pgfqpoint{1.072000in}{0.528000in}}{\pgfqpoint{3.696000in}{3.696000in}}%
\pgfusepath{clip}%
\pgfsetbuttcap%
\pgfsetroundjoin%
\definecolor{currentfill}{rgb}{0.257234,0.339661,0.789661}%
\pgfsetfillcolor{currentfill}%
\pgfsetlinewidth{0.000000pt}%
\definecolor{currentstroke}{rgb}{0.000000,0.000000,0.000000}%
\pgfsetstrokecolor{currentstroke}%
\pgfsetdash{}{0pt}%
\pgfpathmoveto{\pgfqpoint{2.884652in}{2.265870in}}%
\pgfpathlineto{\pgfqpoint{2.927847in}{2.231933in}}%
\pgfpathlineto{\pgfqpoint{2.952824in}{2.288451in}}%
\pgfpathlineto{\pgfqpoint{2.909623in}{2.340155in}}%
\pgfpathlineto{\pgfqpoint{2.884652in}{2.265870in}}%
\pgfpathclose%
\pgfusepath{fill}%
\end{pgfscope}%
\begin{pgfscope}%
\pgfpathrectangle{\pgfqpoint{1.072000in}{0.528000in}}{\pgfqpoint{3.696000in}{3.696000in}}%
\pgfusepath{clip}%
\pgfsetbuttcap%
\pgfsetroundjoin%
\definecolor{currentfill}{rgb}{0.248091,0.326013,0.777669}%
\pgfsetfillcolor{currentfill}%
\pgfsetlinewidth{0.000000pt}%
\definecolor{currentstroke}{rgb}{0.000000,0.000000,0.000000}%
\pgfsetstrokecolor{currentstroke}%
\pgfsetdash{}{0pt}%
\pgfpathmoveto{\pgfqpoint{2.996030in}{2.250611in}}%
\pgfpathlineto{\pgfqpoint{3.039291in}{2.224848in}}%
\pgfpathlineto{\pgfqpoint{3.064167in}{2.278109in}}%
\pgfpathlineto{\pgfqpoint{3.020954in}{2.320147in}}%
\pgfpathlineto{\pgfqpoint{2.996030in}{2.250611in}}%
\pgfpathclose%
\pgfusepath{fill}%
\end{pgfscope}%
\begin{pgfscope}%
\pgfpathrectangle{\pgfqpoint{1.072000in}{0.528000in}}{\pgfqpoint{3.696000in}{3.696000in}}%
\pgfusepath{clip}%
\pgfsetbuttcap%
\pgfsetroundjoin%
\definecolor{currentfill}{rgb}{0.289996,0.386836,0.828926}%
\pgfsetfillcolor{currentfill}%
\pgfsetlinewidth{0.000000pt}%
\definecolor{currentstroke}{rgb}{0.000000,0.000000,0.000000}%
\pgfsetstrokecolor{currentstroke}%
\pgfsetdash{}{0pt}%
\pgfpathmoveto{\pgfqpoint{2.841436in}{2.316071in}}%
\pgfpathlineto{\pgfqpoint{2.884652in}{2.265870in}}%
\pgfpathlineto{\pgfqpoint{2.909623in}{2.340155in}}%
\pgfpathlineto{\pgfqpoint{2.866373in}{2.407016in}}%
\pgfpathlineto{\pgfqpoint{2.841436in}{2.316071in}}%
\pgfpathclose%
\pgfusepath{fill}%
\end{pgfscope}%
\begin{pgfscope}%
\pgfpathrectangle{\pgfqpoint{1.072000in}{0.528000in}}{\pgfqpoint{3.696000in}{3.696000in}}%
\pgfusepath{clip}%
\pgfsetbuttcap%
\pgfsetroundjoin%
\definecolor{currentfill}{rgb}{0.358415,0.478426,0.896795}%
\pgfsetfillcolor{currentfill}%
\pgfsetlinewidth{0.000000pt}%
\definecolor{currentstroke}{rgb}{0.000000,0.000000,0.000000}%
\pgfsetstrokecolor{currentstroke}%
\pgfsetdash{}{0pt}%
\pgfpathmoveto{\pgfqpoint{2.866373in}{2.407016in}}%
\pgfpathlineto{\pgfqpoint{2.909623in}{2.340155in}}%
\pgfpathlineto{\pgfqpoint{2.934550in}{2.440882in}}%
\pgfpathlineto{\pgfqpoint{2.891263in}{2.521511in}}%
\pgfpathlineto{\pgfqpoint{2.866373in}{2.407016in}}%
\pgfpathclose%
\pgfusepath{fill}%
\end{pgfscope}%
\begin{pgfscope}%
\pgfpathrectangle{\pgfqpoint{1.072000in}{0.528000in}}{\pgfqpoint{3.696000in}{3.696000in}}%
\pgfusepath{clip}%
\pgfsetbuttcap%
\pgfsetroundjoin%
\definecolor{currentfill}{rgb}{0.388852,0.516298,0.921373}%
\pgfsetfillcolor{currentfill}%
\pgfsetlinewidth{0.000000pt}%
\definecolor{currentstroke}{rgb}{0.000000,0.000000,0.000000}%
\pgfsetstrokecolor{currentstroke}%
\pgfsetdash{}{0pt}%
\pgfpathmoveto{\pgfqpoint{2.934550in}{2.440882in}}%
\pgfpathlineto{\pgfqpoint{2.977764in}{2.373998in}}%
\pgfpathlineto{\pgfqpoint{3.002728in}{2.487134in}}%
\pgfpathlineto{\pgfqpoint{2.959490in}{2.566103in}}%
\pgfpathlineto{\pgfqpoint{2.934550in}{2.440882in}}%
\pgfpathclose%
\pgfusepath{fill}%
\end{pgfscope}%
\begin{pgfscope}%
\pgfpathrectangle{\pgfqpoint{1.072000in}{0.528000in}}{\pgfqpoint{3.696000in}{3.696000in}}%
\pgfusepath{clip}%
\pgfsetbuttcap%
\pgfsetroundjoin%
\definecolor{currentfill}{rgb}{0.248091,0.326013,0.777669}%
\pgfsetfillcolor{currentfill}%
\pgfsetlinewidth{0.000000pt}%
\definecolor{currentstroke}{rgb}{0.000000,0.000000,0.000000}%
\pgfsetstrokecolor{currentstroke}%
\pgfsetdash{}{0pt}%
\pgfpathmoveto{\pgfqpoint{3.107440in}{2.246290in}}%
\pgfpathlineto{\pgfqpoint{3.150803in}{2.222856in}}%
\pgfpathlineto{\pgfqpoint{3.175612in}{2.281685in}}%
\pgfpathlineto{\pgfqpoint{3.132329in}{2.318541in}}%
\pgfpathlineto{\pgfqpoint{3.107440in}{2.246290in}}%
\pgfpathclose%
\pgfusepath{fill}%
\end{pgfscope}%
\begin{pgfscope}%
\pgfpathrectangle{\pgfqpoint{1.072000in}{0.528000in}}{\pgfqpoint{3.696000in}{3.696000in}}%
\pgfusepath{clip}%
\pgfsetbuttcap%
\pgfsetroundjoin%
\definecolor{currentfill}{rgb}{0.238948,0.312365,0.765676}%
\pgfsetfillcolor{currentfill}%
\pgfsetlinewidth{0.000000pt}%
\definecolor{currentstroke}{rgb}{0.000000,0.000000,0.000000}%
\pgfsetstrokecolor{currentstroke}%
\pgfsetdash{}{0pt}%
\pgfpathmoveto{\pgfqpoint{2.927847in}{2.231933in}}%
\pgfpathlineto{\pgfqpoint{2.971083in}{2.212366in}}%
\pgfpathlineto{\pgfqpoint{2.996030in}{2.250611in}}%
\pgfpathlineto{\pgfqpoint{2.952824in}{2.288451in}}%
\pgfpathlineto{\pgfqpoint{2.927847in}{2.231933in}}%
\pgfpathclose%
\pgfusepath{fill}%
\end{pgfscope}%
\begin{pgfscope}%
\pgfpathrectangle{\pgfqpoint{1.072000in}{0.528000in}}{\pgfqpoint{3.696000in}{3.696000in}}%
\pgfusepath{clip}%
\pgfsetbuttcap%
\pgfsetroundjoin%
\definecolor{currentfill}{rgb}{0.280550,0.373423,0.818011}%
\pgfsetfillcolor{currentfill}%
\pgfsetlinewidth{0.000000pt}%
\definecolor{currentstroke}{rgb}{0.000000,0.000000,0.000000}%
\pgfsetstrokecolor{currentstroke}%
\pgfsetdash{}{0pt}%
\pgfpathmoveto{\pgfqpoint{2.773154in}{2.299899in}}%
\pgfpathlineto{\pgfqpoint{2.816396in}{2.249753in}}%
\pgfpathlineto{\pgfqpoint{2.841436in}{2.316071in}}%
\pgfpathlineto{\pgfqpoint{2.798132in}{2.383797in}}%
\pgfpathlineto{\pgfqpoint{2.773154in}{2.299899in}}%
\pgfpathclose%
\pgfusepath{fill}%
\end{pgfscope}%
\begin{pgfscope}%
\pgfpathrectangle{\pgfqpoint{1.072000in}{0.528000in}}{\pgfqpoint{3.696000in}{3.696000in}}%
\pgfusepath{clip}%
\pgfsetbuttcap%
\pgfsetroundjoin%
\definecolor{currentfill}{rgb}{0.248091,0.326013,0.777669}%
\pgfsetfillcolor{currentfill}%
\pgfsetlinewidth{0.000000pt}%
\definecolor{currentstroke}{rgb}{0.000000,0.000000,0.000000}%
\pgfsetstrokecolor{currentstroke}%
\pgfsetdash{}{0pt}%
\pgfpathmoveto{\pgfqpoint{2.816396in}{2.249753in}}%
\pgfpathlineto{\pgfqpoint{2.859576in}{2.218536in}}%
\pgfpathlineto{\pgfqpoint{2.884652in}{2.265870in}}%
\pgfpathlineto{\pgfqpoint{2.841436in}{2.316071in}}%
\pgfpathlineto{\pgfqpoint{2.816396in}{2.249753in}}%
\pgfpathclose%
\pgfusepath{fill}%
\end{pgfscope}%
\begin{pgfscope}%
\pgfpathrectangle{\pgfqpoint{1.072000in}{0.528000in}}{\pgfqpoint{3.696000in}{3.696000in}}%
\pgfusepath{clip}%
\pgfsetbuttcap%
\pgfsetroundjoin%
\definecolor{currentfill}{rgb}{0.338377,0.452819,0.879317}%
\pgfsetfillcolor{currentfill}%
\pgfsetlinewidth{0.000000pt}%
\definecolor{currentstroke}{rgb}{0.000000,0.000000,0.000000}%
\pgfsetstrokecolor{currentstroke}%
\pgfsetdash{}{0pt}%
\pgfpathmoveto{\pgfqpoint{2.798132in}{2.383797in}}%
\pgfpathlineto{\pgfqpoint{2.841436in}{2.316071in}}%
\pgfpathlineto{\pgfqpoint{2.866373in}{2.407016in}}%
\pgfpathlineto{\pgfqpoint{2.823013in}{2.489688in}}%
\pgfpathlineto{\pgfqpoint{2.798132in}{2.383797in}}%
\pgfpathclose%
\pgfusepath{fill}%
\end{pgfscope}%
\begin{pgfscope}%
\pgfpathrectangle{\pgfqpoint{1.072000in}{0.528000in}}{\pgfqpoint{3.696000in}{3.696000in}}%
\pgfusepath{clip}%
\pgfsetbuttcap%
\pgfsetroundjoin%
\definecolor{currentfill}{rgb}{0.234377,0.305542,0.759680}%
\pgfsetfillcolor{currentfill}%
\pgfsetlinewidth{0.000000pt}%
\definecolor{currentstroke}{rgb}{0.000000,0.000000,0.000000}%
\pgfsetstrokecolor{currentstroke}%
\pgfsetdash{}{0pt}%
\pgfpathmoveto{\pgfqpoint{3.039291in}{2.224848in}}%
\pgfpathlineto{\pgfqpoint{3.082646in}{2.209029in}}%
\pgfpathlineto{\pgfqpoint{3.107440in}{2.246290in}}%
\pgfpathlineto{\pgfqpoint{3.064167in}{2.278109in}}%
\pgfpathlineto{\pgfqpoint{3.039291in}{2.224848in}}%
\pgfpathclose%
\pgfusepath{fill}%
\end{pgfscope}%
\begin{pgfscope}%
\pgfpathrectangle{\pgfqpoint{1.072000in}{0.528000in}}{\pgfqpoint{3.696000in}{3.696000in}}%
\pgfusepath{clip}%
\pgfsetbuttcap%
\pgfsetroundjoin%
\definecolor{currentfill}{rgb}{0.229806,0.298718,0.753683}%
\pgfsetfillcolor{currentfill}%
\pgfsetlinewidth{0.000000pt}%
\definecolor{currentstroke}{rgb}{0.000000,0.000000,0.000000}%
\pgfsetstrokecolor{currentstroke}%
\pgfsetdash{}{0pt}%
\pgfpathmoveto{\pgfqpoint{2.859576in}{2.218536in}}%
\pgfpathlineto{\pgfqpoint{2.902770in}{2.204359in}}%
\pgfpathlineto{\pgfqpoint{2.927847in}{2.231933in}}%
\pgfpathlineto{\pgfqpoint{2.884652in}{2.265870in}}%
\pgfpathlineto{\pgfqpoint{2.859576in}{2.218536in}}%
\pgfpathclose%
\pgfusepath{fill}%
\end{pgfscope}%
\begin{pgfscope}%
\pgfpathrectangle{\pgfqpoint{1.072000in}{0.528000in}}{\pgfqpoint{3.696000in}{3.696000in}}%
\pgfusepath{clip}%
\pgfsetbuttcap%
\pgfsetroundjoin%
\definecolor{currentfill}{rgb}{0.238948,0.312365,0.765676}%
\pgfsetfillcolor{currentfill}%
\pgfsetlinewidth{0.000000pt}%
\definecolor{currentstroke}{rgb}{0.000000,0.000000,0.000000}%
\pgfsetstrokecolor{currentstroke}%
\pgfsetdash{}{0pt}%
\pgfpathmoveto{\pgfqpoint{3.150803in}{2.222856in}}%
\pgfpathlineto{\pgfqpoint{3.194275in}{2.205866in}}%
\pgfpathlineto{\pgfqpoint{3.218982in}{2.252157in}}%
\pgfpathlineto{\pgfqpoint{3.175612in}{2.281685in}}%
\pgfpathlineto{\pgfqpoint{3.150803in}{2.222856in}}%
\pgfpathclose%
\pgfusepath{fill}%
\end{pgfscope}%
\begin{pgfscope}%
\pgfpathrectangle{\pgfqpoint{1.072000in}{0.528000in}}{\pgfqpoint{3.696000in}{3.696000in}}%
\pgfusepath{clip}%
\pgfsetbuttcap%
\pgfsetroundjoin%
\definecolor{currentfill}{rgb}{0.333490,0.446265,0.874452}%
\pgfsetfillcolor{currentfill}%
\pgfsetlinewidth{0.000000pt}%
\definecolor{currentstroke}{rgb}{0.000000,0.000000,0.000000}%
\pgfsetstrokecolor{currentstroke}%
\pgfsetdash{}{0pt}%
\pgfpathmoveto{\pgfqpoint{2.729773in}{2.370067in}}%
\pgfpathlineto{\pgfqpoint{2.773154in}{2.299899in}}%
\pgfpathlineto{\pgfqpoint{2.798132in}{2.383797in}}%
\pgfpathlineto{\pgfqpoint{2.754672in}{2.469520in}}%
\pgfpathlineto{\pgfqpoint{2.729773in}{2.370067in}}%
\pgfpathclose%
\pgfusepath{fill}%
\end{pgfscope}%
\begin{pgfscope}%
\pgfpathrectangle{\pgfqpoint{1.072000in}{0.528000in}}{\pgfqpoint{3.696000in}{3.696000in}}%
\pgfusepath{clip}%
\pgfsetbuttcap%
\pgfsetroundjoin%
\definecolor{currentfill}{rgb}{0.425199,0.559058,0.946061}%
\pgfsetfillcolor{currentfill}%
\pgfsetlinewidth{0.000000pt}%
\definecolor{currentstroke}{rgb}{0.000000,0.000000,0.000000}%
\pgfsetstrokecolor{currentstroke}%
\pgfsetdash{}{0pt}%
\pgfpathmoveto{\pgfqpoint{2.823013in}{2.489688in}}%
\pgfpathlineto{\pgfqpoint{2.866373in}{2.407016in}}%
\pgfpathlineto{\pgfqpoint{2.891263in}{2.521511in}}%
\pgfpathlineto{\pgfqpoint{2.847851in}{2.615967in}}%
\pgfpathlineto{\pgfqpoint{2.823013in}{2.489688in}}%
\pgfpathclose%
\pgfusepath{fill}%
\end{pgfscope}%
\begin{pgfscope}%
\pgfpathrectangle{\pgfqpoint{1.072000in}{0.528000in}}{\pgfqpoint{3.696000in}{3.696000in}}%
\pgfusepath{clip}%
\pgfsetbuttcap%
\pgfsetroundjoin%
\definecolor{currentfill}{rgb}{0.243520,0.319189,0.771672}%
\pgfsetfillcolor{currentfill}%
\pgfsetlinewidth{0.000000pt}%
\definecolor{currentstroke}{rgb}{0.000000,0.000000,0.000000}%
\pgfsetstrokecolor{currentstroke}%
\pgfsetdash{}{0pt}%
\pgfpathmoveto{\pgfqpoint{2.748028in}{2.238382in}}%
\pgfpathlineto{\pgfqpoint{2.791199in}{2.207847in}}%
\pgfpathlineto{\pgfqpoint{2.816396in}{2.249753in}}%
\pgfpathlineto{\pgfqpoint{2.773154in}{2.299899in}}%
\pgfpathlineto{\pgfqpoint{2.748028in}{2.238382in}}%
\pgfpathclose%
\pgfusepath{fill}%
\end{pgfscope}%
\begin{pgfscope}%
\pgfpathrectangle{\pgfqpoint{1.072000in}{0.528000in}}{\pgfqpoint{3.696000in}{3.696000in}}%
\pgfusepath{clip}%
\pgfsetbuttcap%
\pgfsetroundjoin%
\definecolor{currentfill}{rgb}{0.275827,0.366717,0.812553}%
\pgfsetfillcolor{currentfill}%
\pgfsetlinewidth{0.000000pt}%
\definecolor{currentstroke}{rgb}{0.000000,0.000000,0.000000}%
\pgfsetstrokecolor{currentstroke}%
\pgfsetdash{}{0pt}%
\pgfpathmoveto{\pgfqpoint{2.704738in}{2.290654in}}%
\pgfpathlineto{\pgfqpoint{2.748028in}{2.238382in}}%
\pgfpathlineto{\pgfqpoint{2.773154in}{2.299899in}}%
\pgfpathlineto{\pgfqpoint{2.729773in}{2.370067in}}%
\pgfpathlineto{\pgfqpoint{2.704738in}{2.290654in}}%
\pgfpathclose%
\pgfusepath{fill}%
\end{pgfscope}%
\begin{pgfscope}%
\pgfpathrectangle{\pgfqpoint{1.072000in}{0.528000in}}{\pgfqpoint{3.696000in}{3.696000in}}%
\pgfusepath{clip}%
\pgfsetbuttcap%
\pgfsetroundjoin%
\definecolor{currentfill}{rgb}{0.229806,0.298718,0.753683}%
\pgfsetfillcolor{currentfill}%
\pgfsetlinewidth{0.000000pt}%
\definecolor{currentstroke}{rgb}{0.000000,0.000000,0.000000}%
\pgfsetstrokecolor{currentstroke}%
\pgfsetdash{}{0pt}%
\pgfpathmoveto{\pgfqpoint{2.971083in}{2.212366in}}%
\pgfpathlineto{\pgfqpoint{3.014411in}{2.204801in}}%
\pgfpathlineto{\pgfqpoint{3.039291in}{2.224848in}}%
\pgfpathlineto{\pgfqpoint{2.996030in}{2.250611in}}%
\pgfpathlineto{\pgfqpoint{2.971083in}{2.212366in}}%
\pgfpathclose%
\pgfusepath{fill}%
\end{pgfscope}%
\begin{pgfscope}%
\pgfpathrectangle{\pgfqpoint{1.072000in}{0.528000in}}{\pgfqpoint{3.696000in}{3.696000in}}%
\pgfusepath{clip}%
\pgfsetbuttcap%
\pgfsetroundjoin%
\definecolor{currentfill}{rgb}{0.451739,0.588181,0.960201}%
\pgfsetfillcolor{currentfill}%
\pgfsetlinewidth{0.000000pt}%
\definecolor{currentstroke}{rgb}{0.000000,0.000000,0.000000}%
\pgfsetstrokecolor{currentstroke}%
\pgfsetdash{}{0pt}%
\pgfpathmoveto{\pgfqpoint{2.891263in}{2.521511in}}%
\pgfpathlineto{\pgfqpoint{2.934550in}{2.440882in}}%
\pgfpathlineto{\pgfqpoint{2.959490in}{2.566103in}}%
\pgfpathlineto{\pgfqpoint{2.916161in}{2.657132in}}%
\pgfpathlineto{\pgfqpoint{2.891263in}{2.521511in}}%
\pgfpathclose%
\pgfusepath{fill}%
\end{pgfscope}%
\begin{pgfscope}%
\pgfpathrectangle{\pgfqpoint{1.072000in}{0.528000in}}{\pgfqpoint{3.696000in}{3.696000in}}%
\pgfusepath{clip}%
\pgfsetbuttcap%
\pgfsetroundjoin%
\definecolor{currentfill}{rgb}{0.409611,0.540759,0.935545}%
\pgfsetfillcolor{currentfill}%
\pgfsetlinewidth{0.000000pt}%
\definecolor{currentstroke}{rgb}{0.000000,0.000000,0.000000}%
\pgfsetstrokecolor{currentstroke}%
\pgfsetdash{}{0pt}%
\pgfpathmoveto{\pgfqpoint{2.754672in}{2.469520in}}%
\pgfpathlineto{\pgfqpoint{2.798132in}{2.383797in}}%
\pgfpathlineto{\pgfqpoint{2.823013in}{2.489688in}}%
\pgfpathlineto{\pgfqpoint{2.779486in}{2.588060in}}%
\pgfpathlineto{\pgfqpoint{2.754672in}{2.469520in}}%
\pgfpathclose%
\pgfusepath{fill}%
\end{pgfscope}%
\begin{pgfscope}%
\pgfpathrectangle{\pgfqpoint{1.072000in}{0.528000in}}{\pgfqpoint{3.696000in}{3.696000in}}%
\pgfusepath{clip}%
\pgfsetbuttcap%
\pgfsetroundjoin%
\definecolor{currentfill}{rgb}{0.229806,0.298718,0.753683}%
\pgfsetfillcolor{currentfill}%
\pgfsetlinewidth{0.000000pt}%
\definecolor{currentstroke}{rgb}{0.000000,0.000000,0.000000}%
\pgfsetstrokecolor{currentstroke}%
\pgfsetdash{}{0pt}%
\pgfpathmoveto{\pgfqpoint{3.082646in}{2.209029in}}%
\pgfpathlineto{\pgfqpoint{3.126122in}{2.200828in}}%
\pgfpathlineto{\pgfqpoint{3.150803in}{2.222856in}}%
\pgfpathlineto{\pgfqpoint{3.107440in}{2.246290in}}%
\pgfpathlineto{\pgfqpoint{3.082646in}{2.209029in}}%
\pgfpathclose%
\pgfusepath{fill}%
\end{pgfscope}%
\begin{pgfscope}%
\pgfpathrectangle{\pgfqpoint{1.072000in}{0.528000in}}{\pgfqpoint{3.696000in}{3.696000in}}%
\pgfusepath{clip}%
\pgfsetbuttcap%
\pgfsetroundjoin%
\definecolor{currentfill}{rgb}{0.229806,0.298718,0.753683}%
\pgfsetfillcolor{currentfill}%
\pgfsetlinewidth{0.000000pt}%
\definecolor{currentstroke}{rgb}{0.000000,0.000000,0.000000}%
\pgfsetstrokecolor{currentstroke}%
\pgfsetdash{}{0pt}%
\pgfpathmoveto{\pgfqpoint{2.791199in}{2.207847in}}%
\pgfpathlineto{\pgfqpoint{2.834340in}{2.197308in}}%
\pgfpathlineto{\pgfqpoint{2.859576in}{2.218536in}}%
\pgfpathlineto{\pgfqpoint{2.816396in}{2.249753in}}%
\pgfpathlineto{\pgfqpoint{2.791199in}{2.207847in}}%
\pgfpathclose%
\pgfusepath{fill}%
\end{pgfscope}%
\begin{pgfscope}%
\pgfpathrectangle{\pgfqpoint{1.072000in}{0.528000in}}{\pgfqpoint{3.696000in}{3.696000in}}%
\pgfusepath{clip}%
\pgfsetbuttcap%
\pgfsetroundjoin%
\definecolor{currentfill}{rgb}{0.333490,0.446265,0.874452}%
\pgfsetfillcolor{currentfill}%
\pgfsetlinewidth{0.000000pt}%
\definecolor{currentstroke}{rgb}{0.000000,0.000000,0.000000}%
\pgfsetstrokecolor{currentstroke}%
\pgfsetdash{}{0pt}%
\pgfpathmoveto{\pgfqpoint{2.661242in}{2.365427in}}%
\pgfpathlineto{\pgfqpoint{2.704738in}{2.290654in}}%
\pgfpathlineto{\pgfqpoint{2.729773in}{2.370067in}}%
\pgfpathlineto{\pgfqpoint{2.686176in}{2.460400in}}%
\pgfpathlineto{\pgfqpoint{2.661242in}{2.365427in}}%
\pgfpathclose%
\pgfusepath{fill}%
\end{pgfscope}%
\begin{pgfscope}%
\pgfpathrectangle{\pgfqpoint{1.072000in}{0.528000in}}{\pgfqpoint{3.696000in}{3.696000in}}%
\pgfusepath{clip}%
\pgfsetbuttcap%
\pgfsetroundjoin%
\definecolor{currentfill}{rgb}{0.229806,0.298718,0.753683}%
\pgfsetfillcolor{currentfill}%
\pgfsetlinewidth{0.000000pt}%
\definecolor{currentstroke}{rgb}{0.000000,0.000000,0.000000}%
\pgfsetstrokecolor{currentstroke}%
\pgfsetdash{}{0pt}%
\pgfpathmoveto{\pgfqpoint{3.194275in}{2.205866in}}%
\pgfpathlineto{\pgfqpoint{3.237866in}{2.193389in}}%
\pgfpathlineto{\pgfqpoint{3.262453in}{2.228421in}}%
\pgfpathlineto{\pgfqpoint{3.218982in}{2.252157in}}%
\pgfpathlineto{\pgfqpoint{3.194275in}{2.205866in}}%
\pgfpathclose%
\pgfusepath{fill}%
\end{pgfscope}%
\begin{pgfscope}%
\pgfpathrectangle{\pgfqpoint{1.072000in}{0.528000in}}{\pgfqpoint{3.696000in}{3.696000in}}%
\pgfusepath{clip}%
\pgfsetbuttcap%
\pgfsetroundjoin%
\definecolor{currentfill}{rgb}{0.243520,0.319189,0.771672}%
\pgfsetfillcolor{currentfill}%
\pgfsetlinewidth{0.000000pt}%
\definecolor{currentstroke}{rgb}{0.000000,0.000000,0.000000}%
\pgfsetstrokecolor{currentstroke}%
\pgfsetdash{}{0pt}%
\pgfpathmoveto{\pgfqpoint{2.679525in}{2.231168in}}%
\pgfpathlineto{\pgfqpoint{2.722709in}{2.198517in}}%
\pgfpathlineto{\pgfqpoint{2.748028in}{2.238382in}}%
\pgfpathlineto{\pgfqpoint{2.704738in}{2.290654in}}%
\pgfpathlineto{\pgfqpoint{2.679525in}{2.231168in}}%
\pgfpathclose%
\pgfusepath{fill}%
\end{pgfscope}%
\begin{pgfscope}%
\pgfpathrectangle{\pgfqpoint{1.072000in}{0.528000in}}{\pgfqpoint{3.696000in}{3.696000in}}%
\pgfusepath{clip}%
\pgfsetbuttcap%
\pgfsetroundjoin%
\definecolor{currentfill}{rgb}{0.229806,0.298718,0.753683}%
\pgfsetfillcolor{currentfill}%
\pgfsetlinewidth{0.000000pt}%
\definecolor{currentstroke}{rgb}{0.000000,0.000000,0.000000}%
\pgfsetstrokecolor{currentstroke}%
\pgfsetdash{}{0pt}%
\pgfpathmoveto{\pgfqpoint{2.902770in}{2.204359in}}%
\pgfpathlineto{\pgfqpoint{2.946045in}{2.204710in}}%
\pgfpathlineto{\pgfqpoint{2.971083in}{2.212366in}}%
\pgfpathlineto{\pgfqpoint{2.927847in}{2.231933in}}%
\pgfpathlineto{\pgfqpoint{2.902770in}{2.204359in}}%
\pgfpathclose%
\pgfusepath{fill}%
\end{pgfscope}%
\begin{pgfscope}%
\pgfpathrectangle{\pgfqpoint{1.072000in}{0.528000in}}{\pgfqpoint{3.696000in}{3.696000in}}%
\pgfusepath{clip}%
\pgfsetbuttcap%
\pgfsetroundjoin%
\definecolor{currentfill}{rgb}{0.280550,0.373423,0.818011}%
\pgfsetfillcolor{currentfill}%
\pgfsetlinewidth{0.000000pt}%
\definecolor{currentstroke}{rgb}{0.000000,0.000000,0.000000}%
\pgfsetstrokecolor{currentstroke}%
\pgfsetdash{}{0pt}%
\pgfpathmoveto{\pgfqpoint{2.636146in}{2.288347in}}%
\pgfpathlineto{\pgfqpoint{2.679525in}{2.231168in}}%
\pgfpathlineto{\pgfqpoint{2.704738in}{2.290654in}}%
\pgfpathlineto{\pgfqpoint{2.661242in}{2.365427in}}%
\pgfpathlineto{\pgfqpoint{2.636146in}{2.288347in}}%
\pgfpathclose%
\pgfusepath{fill}%
\end{pgfscope}%
\begin{pgfscope}%
\pgfpathrectangle{\pgfqpoint{1.072000in}{0.528000in}}{\pgfqpoint{3.696000in}{3.696000in}}%
\pgfusepath{clip}%
\pgfsetbuttcap%
\pgfsetroundjoin%
\definecolor{currentfill}{rgb}{0.404421,0.534643,0.932002}%
\pgfsetfillcolor{currentfill}%
\pgfsetlinewidth{0.000000pt}%
\definecolor{currentstroke}{rgb}{0.000000,0.000000,0.000000}%
\pgfsetstrokecolor{currentstroke}%
\pgfsetdash{}{0pt}%
\pgfpathmoveto{\pgfqpoint{2.686176in}{2.460400in}}%
\pgfpathlineto{\pgfqpoint{2.729773in}{2.370067in}}%
\pgfpathlineto{\pgfqpoint{2.754672in}{2.469520in}}%
\pgfpathlineto{\pgfqpoint{2.710993in}{2.572799in}}%
\pgfpathlineto{\pgfqpoint{2.686176in}{2.460400in}}%
\pgfpathclose%
\pgfusepath{fill}%
\end{pgfscope}%
\begin{pgfscope}%
\pgfpathrectangle{\pgfqpoint{1.072000in}{0.528000in}}{\pgfqpoint{3.696000in}{3.696000in}}%
\pgfusepath{clip}%
\pgfsetbuttcap%
\pgfsetroundjoin%
\definecolor{currentfill}{rgb}{0.229806,0.298718,0.753683}%
\pgfsetfillcolor{currentfill}%
\pgfsetlinewidth{0.000000pt}%
\definecolor{currentstroke}{rgb}{0.000000,0.000000,0.000000}%
\pgfsetstrokecolor{currentstroke}%
\pgfsetdash{}{0pt}%
\pgfpathmoveto{\pgfqpoint{3.014411in}{2.204801in}}%
\pgfpathlineto{\pgfqpoint{3.057870in}{2.206571in}}%
\pgfpathlineto{\pgfqpoint{3.082646in}{2.209029in}}%
\pgfpathlineto{\pgfqpoint{3.039291in}{2.224848in}}%
\pgfpathlineto{\pgfqpoint{3.014411in}{2.204801in}}%
\pgfpathclose%
\pgfusepath{fill}%
\end{pgfscope}%
\begin{pgfscope}%
\pgfpathrectangle{\pgfqpoint{1.072000in}{0.528000in}}{\pgfqpoint{3.696000in}{3.696000in}}%
\pgfusepath{clip}%
\pgfsetbuttcap%
\pgfsetroundjoin%
\definecolor{currentfill}{rgb}{0.532568,0.669801,0.990393}%
\pgfsetfillcolor{currentfill}%
\pgfsetlinewidth{0.000000pt}%
\definecolor{currentstroke}{rgb}{0.000000,0.000000,0.000000}%
\pgfsetstrokecolor{currentstroke}%
\pgfsetdash{}{0pt}%
\pgfpathmoveto{\pgfqpoint{2.847851in}{2.615967in}}%
\pgfpathlineto{\pgfqpoint{2.891263in}{2.521511in}}%
\pgfpathlineto{\pgfqpoint{2.916161in}{2.657132in}}%
\pgfpathlineto{\pgfqpoint{2.872697in}{2.759775in}}%
\pgfpathlineto{\pgfqpoint{2.847851in}{2.615967in}}%
\pgfpathclose%
\pgfusepath{fill}%
\end{pgfscope}%
\begin{pgfscope}%
\pgfpathrectangle{\pgfqpoint{1.072000in}{0.528000in}}{\pgfqpoint{3.696000in}{3.696000in}}%
\pgfusepath{clip}%
\pgfsetbuttcap%
\pgfsetroundjoin%
\definecolor{currentfill}{rgb}{0.505423,0.643995,0.983157}%
\pgfsetfillcolor{currentfill}%
\pgfsetlinewidth{0.000000pt}%
\definecolor{currentstroke}{rgb}{0.000000,0.000000,0.000000}%
\pgfsetstrokecolor{currentstroke}%
\pgfsetdash{}{0pt}%
\pgfpathmoveto{\pgfqpoint{2.779486in}{2.588060in}}%
\pgfpathlineto{\pgfqpoint{2.823013in}{2.489688in}}%
\pgfpathlineto{\pgfqpoint{2.847851in}{2.615967in}}%
\pgfpathlineto{\pgfqpoint{2.804266in}{2.723597in}}%
\pgfpathlineto{\pgfqpoint{2.779486in}{2.588060in}}%
\pgfpathclose%
\pgfusepath{fill}%
\end{pgfscope}%
\begin{pgfscope}%
\pgfpathrectangle{\pgfqpoint{1.072000in}{0.528000in}}{\pgfqpoint{3.696000in}{3.696000in}}%
\pgfusepath{clip}%
\pgfsetbuttcap%
\pgfsetroundjoin%
\definecolor{currentfill}{rgb}{0.229806,0.298718,0.753683}%
\pgfsetfillcolor{currentfill}%
\pgfsetlinewidth{0.000000pt}%
\definecolor{currentstroke}{rgb}{0.000000,0.000000,0.000000}%
\pgfsetstrokecolor{currentstroke}%
\pgfsetdash{}{0pt}%
\pgfpathmoveto{\pgfqpoint{2.722709in}{2.198517in}}%
\pgfpathlineto{\pgfqpoint{2.765800in}{2.188977in}}%
\pgfpathlineto{\pgfqpoint{2.791199in}{2.207847in}}%
\pgfpathlineto{\pgfqpoint{2.748028in}{2.238382in}}%
\pgfpathlineto{\pgfqpoint{2.722709in}{2.198517in}}%
\pgfpathclose%
\pgfusepath{fill}%
\end{pgfscope}%
\begin{pgfscope}%
\pgfpathrectangle{\pgfqpoint{1.072000in}{0.528000in}}{\pgfqpoint{3.696000in}{3.696000in}}%
\pgfusepath{clip}%
\pgfsetbuttcap%
\pgfsetroundjoin%
\definecolor{currentfill}{rgb}{0.229806,0.298718,0.753683}%
\pgfsetfillcolor{currentfill}%
\pgfsetlinewidth{0.000000pt}%
\definecolor{currentstroke}{rgb}{0.000000,0.000000,0.000000}%
\pgfsetstrokecolor{currentstroke}%
\pgfsetdash{}{0pt}%
\pgfpathmoveto{\pgfqpoint{3.126122in}{2.200828in}}%
\pgfpathlineto{\pgfqpoint{3.169734in}{2.197867in}}%
\pgfpathlineto{\pgfqpoint{3.194275in}{2.205866in}}%
\pgfpathlineto{\pgfqpoint{3.150803in}{2.222856in}}%
\pgfpathlineto{\pgfqpoint{3.126122in}{2.200828in}}%
\pgfpathclose%
\pgfusepath{fill}%
\end{pgfscope}%
\begin{pgfscope}%
\pgfpathrectangle{\pgfqpoint{1.072000in}{0.528000in}}{\pgfqpoint{3.696000in}{3.696000in}}%
\pgfusepath{clip}%
\pgfsetbuttcap%
\pgfsetroundjoin%
\definecolor{currentfill}{rgb}{0.229806,0.298718,0.753683}%
\pgfsetfillcolor{currentfill}%
\pgfsetlinewidth{0.000000pt}%
\definecolor{currentstroke}{rgb}{0.000000,0.000000,0.000000}%
\pgfsetstrokecolor{currentstroke}%
\pgfsetdash{}{0pt}%
\pgfpathmoveto{\pgfqpoint{2.834340in}{2.197308in}}%
\pgfpathlineto{\pgfqpoint{2.877533in}{2.204227in}}%
\pgfpathlineto{\pgfqpoint{2.902770in}{2.204359in}}%
\pgfpathlineto{\pgfqpoint{2.859576in}{2.218536in}}%
\pgfpathlineto{\pgfqpoint{2.834340in}{2.197308in}}%
\pgfpathclose%
\pgfusepath{fill}%
\end{pgfscope}%
\begin{pgfscope}%
\pgfpathrectangle{\pgfqpoint{1.072000in}{0.528000in}}{\pgfqpoint{3.696000in}{3.696000in}}%
\pgfusepath{clip}%
\pgfsetbuttcap%
\pgfsetroundjoin%
\definecolor{currentfill}{rgb}{0.229806,0.298718,0.753683}%
\pgfsetfillcolor{currentfill}%
\pgfsetlinewidth{0.000000pt}%
\definecolor{currentstroke}{rgb}{0.000000,0.000000,0.000000}%
\pgfsetstrokecolor{currentstroke}%
\pgfsetdash{}{0pt}%
\pgfpathmoveto{\pgfqpoint{3.237866in}{2.193389in}}%
\pgfpathlineto{\pgfqpoint{3.281578in}{2.183615in}}%
\pgfpathlineto{\pgfqpoint{3.306033in}{2.208996in}}%
\pgfpathlineto{\pgfqpoint{3.262453in}{2.228421in}}%
\pgfpathlineto{\pgfqpoint{3.237866in}{2.193389in}}%
\pgfpathclose%
\pgfusepath{fill}%
\end{pgfscope}%
\begin{pgfscope}%
\pgfpathrectangle{\pgfqpoint{1.072000in}{0.528000in}}{\pgfqpoint{3.696000in}{3.696000in}}%
\pgfusepath{clip}%
\pgfsetbuttcap%
\pgfsetroundjoin%
\definecolor{currentfill}{rgb}{0.248091,0.326013,0.777669}%
\pgfsetfillcolor{currentfill}%
\pgfsetlinewidth{0.000000pt}%
\definecolor{currentstroke}{rgb}{0.000000,0.000000,0.000000}%
\pgfsetstrokecolor{currentstroke}%
\pgfsetdash{}{0pt}%
\pgfpathmoveto{\pgfqpoint{2.610855in}{2.228716in}}%
\pgfpathlineto{\pgfqpoint{2.654098in}{2.190548in}}%
\pgfpathlineto{\pgfqpoint{2.679525in}{2.231168in}}%
\pgfpathlineto{\pgfqpoint{2.636146in}{2.288347in}}%
\pgfpathlineto{\pgfqpoint{2.610855in}{2.228716in}}%
\pgfpathclose%
\pgfusepath{fill}%
\end{pgfscope}%
\begin{pgfscope}%
\pgfpathrectangle{\pgfqpoint{1.072000in}{0.528000in}}{\pgfqpoint{3.696000in}{3.696000in}}%
\pgfusepath{clip}%
\pgfsetbuttcap%
\pgfsetroundjoin%
\definecolor{currentfill}{rgb}{0.338377,0.452819,0.879317}%
\pgfsetfillcolor{currentfill}%
\pgfsetlinewidth{0.000000pt}%
\definecolor{currentstroke}{rgb}{0.000000,0.000000,0.000000}%
\pgfsetstrokecolor{currentstroke}%
\pgfsetdash{}{0pt}%
\pgfpathmoveto{\pgfqpoint{2.592479in}{2.370305in}}%
\pgfpathlineto{\pgfqpoint{2.636146in}{2.288347in}}%
\pgfpathlineto{\pgfqpoint{2.661242in}{2.365427in}}%
\pgfpathlineto{\pgfqpoint{2.617458in}{2.462347in}}%
\pgfpathlineto{\pgfqpoint{2.592479in}{2.370305in}}%
\pgfpathclose%
\pgfusepath{fill}%
\end{pgfscope}%
\begin{pgfscope}%
\pgfpathrectangle{\pgfqpoint{1.072000in}{0.528000in}}{\pgfqpoint{3.696000in}{3.696000in}}%
\pgfusepath{clip}%
\pgfsetbuttcap%
\pgfsetroundjoin%
\definecolor{currentfill}{rgb}{0.494638,0.633022,0.978983}%
\pgfsetfillcolor{currentfill}%
\pgfsetlinewidth{0.000000pt}%
\definecolor{currentstroke}{rgb}{0.000000,0.000000,0.000000}%
\pgfsetstrokecolor{currentstroke}%
\pgfsetdash{}{0pt}%
\pgfpathmoveto{\pgfqpoint{2.710993in}{2.572799in}}%
\pgfpathlineto{\pgfqpoint{2.754672in}{2.469520in}}%
\pgfpathlineto{\pgfqpoint{2.779486in}{2.588060in}}%
\pgfpathlineto{\pgfqpoint{2.735740in}{2.701152in}}%
\pgfpathlineto{\pgfqpoint{2.710993in}{2.572799in}}%
\pgfpathclose%
\pgfusepath{fill}%
\end{pgfscope}%
\begin{pgfscope}%
\pgfpathrectangle{\pgfqpoint{1.072000in}{0.528000in}}{\pgfqpoint{3.696000in}{3.696000in}}%
\pgfusepath{clip}%
\pgfsetbuttcap%
\pgfsetroundjoin%
\definecolor{currentfill}{rgb}{0.409611,0.540759,0.935545}%
\pgfsetfillcolor{currentfill}%
\pgfsetlinewidth{0.000000pt}%
\definecolor{currentstroke}{rgb}{0.000000,0.000000,0.000000}%
\pgfsetstrokecolor{currentstroke}%
\pgfsetdash{}{0pt}%
\pgfpathmoveto{\pgfqpoint{2.617458in}{2.462347in}}%
\pgfpathlineto{\pgfqpoint{2.661242in}{2.365427in}}%
\pgfpathlineto{\pgfqpoint{2.686176in}{2.460400in}}%
\pgfpathlineto{\pgfqpoint{2.642298in}{2.569978in}}%
\pgfpathlineto{\pgfqpoint{2.617458in}{2.462347in}}%
\pgfpathclose%
\pgfusepath{fill}%
\end{pgfscope}%
\begin{pgfscope}%
\pgfpathrectangle{\pgfqpoint{1.072000in}{0.528000in}}{\pgfqpoint{3.696000in}{3.696000in}}%
\pgfusepath{clip}%
\pgfsetbuttcap%
\pgfsetroundjoin%
\definecolor{currentfill}{rgb}{0.289996,0.386836,0.828926}%
\pgfsetfillcolor{currentfill}%
\pgfsetlinewidth{0.000000pt}%
\definecolor{currentstroke}{rgb}{0.000000,0.000000,0.000000}%
\pgfsetstrokecolor{currentstroke}%
\pgfsetdash{}{0pt}%
\pgfpathmoveto{\pgfqpoint{2.567325in}{2.293987in}}%
\pgfpathlineto{\pgfqpoint{2.610855in}{2.228716in}}%
\pgfpathlineto{\pgfqpoint{2.636146in}{2.288347in}}%
\pgfpathlineto{\pgfqpoint{2.592479in}{2.370305in}}%
\pgfpathlineto{\pgfqpoint{2.567325in}{2.293987in}}%
\pgfpathclose%
\pgfusepath{fill}%
\end{pgfscope}%
\begin{pgfscope}%
\pgfpathrectangle{\pgfqpoint{1.072000in}{0.528000in}}{\pgfqpoint{3.696000in}{3.696000in}}%
\pgfusepath{clip}%
\pgfsetbuttcap%
\pgfsetroundjoin%
\definecolor{currentfill}{rgb}{0.229806,0.298718,0.753683}%
\pgfsetfillcolor{currentfill}%
\pgfsetlinewidth{0.000000pt}%
\definecolor{currentstroke}{rgb}{0.000000,0.000000,0.000000}%
\pgfsetstrokecolor{currentstroke}%
\pgfsetdash{}{0pt}%
\pgfpathmoveto{\pgfqpoint{2.654098in}{2.190548in}}%
\pgfpathlineto{\pgfqpoint{2.697162in}{2.178595in}}%
\pgfpathlineto{\pgfqpoint{2.722709in}{2.198517in}}%
\pgfpathlineto{\pgfqpoint{2.679525in}{2.231168in}}%
\pgfpathlineto{\pgfqpoint{2.654098in}{2.190548in}}%
\pgfpathclose%
\pgfusepath{fill}%
\end{pgfscope}%
\begin{pgfscope}%
\pgfpathrectangle{\pgfqpoint{1.072000in}{0.528000in}}{\pgfqpoint{3.696000in}{3.696000in}}%
\pgfusepath{clip}%
\pgfsetbuttcap%
\pgfsetroundjoin%
\definecolor{currentfill}{rgb}{0.234377,0.305542,0.759680}%
\pgfsetfillcolor{currentfill}%
\pgfsetlinewidth{0.000000pt}%
\definecolor{currentstroke}{rgb}{0.000000,0.000000,0.000000}%
\pgfsetstrokecolor{currentstroke}%
\pgfsetdash{}{0pt}%
\pgfpathmoveto{\pgfqpoint{2.946045in}{2.204710in}}%
\pgfpathlineto{\pgfqpoint{2.989452in}{2.216641in}}%
\pgfpathlineto{\pgfqpoint{3.014411in}{2.204801in}}%
\pgfpathlineto{\pgfqpoint{2.971083in}{2.212366in}}%
\pgfpathlineto{\pgfqpoint{2.946045in}{2.204710in}}%
\pgfpathclose%
\pgfusepath{fill}%
\end{pgfscope}%
\begin{pgfscope}%
\pgfpathrectangle{\pgfqpoint{1.072000in}{0.528000in}}{\pgfqpoint{3.696000in}{3.696000in}}%
\pgfusepath{clip}%
\pgfsetbuttcap%
\pgfsetroundjoin%
\definecolor{currentfill}{rgb}{0.238948,0.312365,0.765676}%
\pgfsetfillcolor{currentfill}%
\pgfsetlinewidth{0.000000pt}%
\definecolor{currentstroke}{rgb}{0.000000,0.000000,0.000000}%
\pgfsetstrokecolor{currentstroke}%
\pgfsetdash{}{0pt}%
\pgfpathmoveto{\pgfqpoint{3.057870in}{2.206571in}}%
\pgfpathlineto{\pgfqpoint{3.101485in}{2.214872in}}%
\pgfpathlineto{\pgfqpoint{3.126122in}{2.200828in}}%
\pgfpathlineto{\pgfqpoint{3.082646in}{2.209029in}}%
\pgfpathlineto{\pgfqpoint{3.057870in}{2.206571in}}%
\pgfpathclose%
\pgfusepath{fill}%
\end{pgfscope}%
\begin{pgfscope}%
\pgfpathrectangle{\pgfqpoint{1.072000in}{0.528000in}}{\pgfqpoint{3.696000in}{3.696000in}}%
\pgfusepath{clip}%
\pgfsetbuttcap%
\pgfsetroundjoin%
\definecolor{currentfill}{rgb}{0.229806,0.298718,0.753683}%
\pgfsetfillcolor{currentfill}%
\pgfsetlinewidth{0.000000pt}%
\definecolor{currentstroke}{rgb}{0.000000,0.000000,0.000000}%
\pgfsetstrokecolor{currentstroke}%
\pgfsetdash{}{0pt}%
\pgfpathmoveto{\pgfqpoint{2.765800in}{2.188977in}}%
\pgfpathlineto{\pgfqpoint{2.808895in}{2.200134in}}%
\pgfpathlineto{\pgfqpoint{2.834340in}{2.197308in}}%
\pgfpathlineto{\pgfqpoint{2.791199in}{2.207847in}}%
\pgfpathlineto{\pgfqpoint{2.765800in}{2.188977in}}%
\pgfpathclose%
\pgfusepath{fill}%
\end{pgfscope}%
\begin{pgfscope}%
\pgfpathrectangle{\pgfqpoint{1.072000in}{0.528000in}}{\pgfqpoint{3.696000in}{3.696000in}}%
\pgfusepath{clip}%
\pgfsetbuttcap%
\pgfsetroundjoin%
\definecolor{currentfill}{rgb}{0.494638,0.633022,0.978983}%
\pgfsetfillcolor{currentfill}%
\pgfsetlinewidth{0.000000pt}%
\definecolor{currentstroke}{rgb}{0.000000,0.000000,0.000000}%
\pgfsetstrokecolor{currentstroke}%
\pgfsetdash{}{0pt}%
\pgfpathmoveto{\pgfqpoint{2.642298in}{2.569978in}}%
\pgfpathlineto{\pgfqpoint{2.686176in}{2.460400in}}%
\pgfpathlineto{\pgfqpoint{2.710993in}{2.572799in}}%
\pgfpathlineto{\pgfqpoint{2.667042in}{2.692192in}}%
\pgfpathlineto{\pgfqpoint{2.642298in}{2.569978in}}%
\pgfpathclose%
\pgfusepath{fill}%
\end{pgfscope}%
\begin{pgfscope}%
\pgfpathrectangle{\pgfqpoint{1.072000in}{0.528000in}}{\pgfqpoint{3.696000in}{3.696000in}}%
\pgfusepath{clip}%
\pgfsetbuttcap%
\pgfsetroundjoin%
\definecolor{currentfill}{rgb}{0.229806,0.298718,0.753683}%
\pgfsetfillcolor{currentfill}%
\pgfsetlinewidth{0.000000pt}%
\definecolor{currentstroke}{rgb}{0.000000,0.000000,0.000000}%
\pgfsetstrokecolor{currentstroke}%
\pgfsetdash{}{0pt}%
\pgfpathmoveto{\pgfqpoint{3.169734in}{2.197867in}}%
\pgfpathlineto{\pgfqpoint{3.213486in}{2.197846in}}%
\pgfpathlineto{\pgfqpoint{3.237866in}{2.193389in}}%
\pgfpathlineto{\pgfqpoint{3.194275in}{2.205866in}}%
\pgfpathlineto{\pgfqpoint{3.169734in}{2.197867in}}%
\pgfpathclose%
\pgfusepath{fill}%
\end{pgfscope}%
\begin{pgfscope}%
\pgfpathrectangle{\pgfqpoint{1.072000in}{0.528000in}}{\pgfqpoint{3.696000in}{3.696000in}}%
\pgfusepath{clip}%
\pgfsetbuttcap%
\pgfsetroundjoin%
\definecolor{currentfill}{rgb}{0.229806,0.298718,0.753683}%
\pgfsetfillcolor{currentfill}%
\pgfsetlinewidth{0.000000pt}%
\definecolor{currentstroke}{rgb}{0.000000,0.000000,0.000000}%
\pgfsetstrokecolor{currentstroke}%
\pgfsetdash{}{0pt}%
\pgfpathmoveto{\pgfqpoint{3.281578in}{2.183615in}}%
\pgfpathlineto{\pgfqpoint{3.325403in}{2.174943in}}%
\pgfpathlineto{\pgfqpoint{3.349723in}{2.192551in}}%
\pgfpathlineto{\pgfqpoint{3.306033in}{2.208996in}}%
\pgfpathlineto{\pgfqpoint{3.281578in}{2.183615in}}%
\pgfpathclose%
\pgfusepath{fill}%
\end{pgfscope}%
\begin{pgfscope}%
\pgfpathrectangle{\pgfqpoint{1.072000in}{0.528000in}}{\pgfqpoint{3.696000in}{3.696000in}}%
\pgfusepath{clip}%
\pgfsetbuttcap%
\pgfsetroundjoin%
\definecolor{currentfill}{rgb}{0.257234,0.339661,0.789661}%
\pgfsetfillcolor{currentfill}%
\pgfsetlinewidth{0.000000pt}%
\definecolor{currentstroke}{rgb}{0.000000,0.000000,0.000000}%
\pgfsetstrokecolor{currentstroke}%
\pgfsetdash{}{0pt}%
\pgfpathmoveto{\pgfqpoint{2.541967in}{2.232762in}}%
\pgfpathlineto{\pgfqpoint{2.585341in}{2.185297in}}%
\pgfpathlineto{\pgfqpoint{2.610855in}{2.228716in}}%
\pgfpathlineto{\pgfqpoint{2.567325in}{2.293987in}}%
\pgfpathlineto{\pgfqpoint{2.541967in}{2.232762in}}%
\pgfpathclose%
\pgfusepath{fill}%
\end{pgfscope}%
\begin{pgfscope}%
\pgfpathrectangle{\pgfqpoint{1.072000in}{0.528000in}}{\pgfqpoint{3.696000in}{3.696000in}}%
\pgfusepath{clip}%
\pgfsetbuttcap%
\pgfsetroundjoin%
\definecolor{currentfill}{rgb}{0.624703,0.748318,0.998719}%
\pgfsetfillcolor{currentfill}%
\pgfsetlinewidth{0.000000pt}%
\definecolor{currentstroke}{rgb}{0.000000,0.000000,0.000000}%
\pgfsetstrokecolor{currentstroke}%
\pgfsetdash{}{0pt}%
\pgfpathmoveto{\pgfqpoint{2.804266in}{2.723597in}}%
\pgfpathlineto{\pgfqpoint{2.847851in}{2.615967in}}%
\pgfpathlineto{\pgfqpoint{2.872697in}{2.759775in}}%
\pgfpathlineto{\pgfqpoint{2.829062in}{2.872891in}}%
\pgfpathlineto{\pgfqpoint{2.804266in}{2.723597in}}%
\pgfpathclose%
\pgfusepath{fill}%
\end{pgfscope}%
\begin{pgfscope}%
\pgfpathrectangle{\pgfqpoint{1.072000in}{0.528000in}}{\pgfqpoint{3.696000in}{3.696000in}}%
\pgfusepath{clip}%
\pgfsetbuttcap%
\pgfsetroundjoin%
\definecolor{currentfill}{rgb}{0.229806,0.298718,0.753683}%
\pgfsetfillcolor{currentfill}%
\pgfsetlinewidth{0.000000pt}%
\definecolor{currentstroke}{rgb}{0.000000,0.000000,0.000000}%
\pgfsetstrokecolor{currentstroke}%
\pgfsetdash{}{0pt}%
\pgfpathmoveto{\pgfqpoint{2.585341in}{2.185297in}}%
\pgfpathlineto{\pgfqpoint{2.628432in}{2.166928in}}%
\pgfpathlineto{\pgfqpoint{2.654098in}{2.190548in}}%
\pgfpathlineto{\pgfqpoint{2.610855in}{2.228716in}}%
\pgfpathlineto{\pgfqpoint{2.585341in}{2.185297in}}%
\pgfpathclose%
\pgfusepath{fill}%
\end{pgfscope}%
\begin{pgfscope}%
\pgfpathrectangle{\pgfqpoint{1.072000in}{0.528000in}}{\pgfqpoint{3.696000in}{3.696000in}}%
\pgfusepath{clip}%
\pgfsetbuttcap%
\pgfsetroundjoin%
\definecolor{currentfill}{rgb}{0.603162,0.731527,0.999565}%
\pgfsetfillcolor{currentfill}%
\pgfsetlinewidth{0.000000pt}%
\definecolor{currentstroke}{rgb}{0.000000,0.000000,0.000000}%
\pgfsetstrokecolor{currentstroke}%
\pgfsetdash{}{0pt}%
\pgfpathmoveto{\pgfqpoint{2.735740in}{2.701152in}}%
\pgfpathlineto{\pgfqpoint{2.779486in}{2.588060in}}%
\pgfpathlineto{\pgfqpoint{2.804266in}{2.723597in}}%
\pgfpathlineto{\pgfqpoint{2.760469in}{2.842937in}}%
\pgfpathlineto{\pgfqpoint{2.735740in}{2.701152in}}%
\pgfpathclose%
\pgfusepath{fill}%
\end{pgfscope}%
\begin{pgfscope}%
\pgfpathrectangle{\pgfqpoint{1.072000in}{0.528000in}}{\pgfqpoint{3.696000in}{3.696000in}}%
\pgfusepath{clip}%
\pgfsetbuttcap%
\pgfsetroundjoin%
\definecolor{currentfill}{rgb}{0.358415,0.478426,0.896795}%
\pgfsetfillcolor{currentfill}%
\pgfsetlinewidth{0.000000pt}%
\definecolor{currentstroke}{rgb}{0.000000,0.000000,0.000000}%
\pgfsetstrokecolor{currentstroke}%
\pgfsetdash{}{0pt}%
\pgfpathmoveto{\pgfqpoint{2.523413in}{2.385894in}}%
\pgfpathlineto{\pgfqpoint{2.567325in}{2.293987in}}%
\pgfpathlineto{\pgfqpoint{2.592479in}{2.370305in}}%
\pgfpathlineto{\pgfqpoint{2.548444in}{2.476005in}}%
\pgfpathlineto{\pgfqpoint{2.523413in}{2.385894in}}%
\pgfpathclose%
\pgfusepath{fill}%
\end{pgfscope}%
\begin{pgfscope}%
\pgfpathrectangle{\pgfqpoint{1.072000in}{0.528000in}}{\pgfqpoint{3.696000in}{3.696000in}}%
\pgfusepath{clip}%
\pgfsetbuttcap%
\pgfsetroundjoin%
\definecolor{currentfill}{rgb}{0.425199,0.559058,0.946061}%
\pgfsetfillcolor{currentfill}%
\pgfsetlinewidth{0.000000pt}%
\definecolor{currentstroke}{rgb}{0.000000,0.000000,0.000000}%
\pgfsetstrokecolor{currentstroke}%
\pgfsetdash{}{0pt}%
\pgfpathmoveto{\pgfqpoint{2.548444in}{2.476005in}}%
\pgfpathlineto{\pgfqpoint{2.592479in}{2.370305in}}%
\pgfpathlineto{\pgfqpoint{2.617458in}{2.462347in}}%
\pgfpathlineto{\pgfqpoint{2.573324in}{2.579845in}}%
\pgfpathlineto{\pgfqpoint{2.548444in}{2.476005in}}%
\pgfpathclose%
\pgfusepath{fill}%
\end{pgfscope}%
\begin{pgfscope}%
\pgfpathrectangle{\pgfqpoint{1.072000in}{0.528000in}}{\pgfqpoint{3.696000in}{3.696000in}}%
\pgfusepath{clip}%
\pgfsetbuttcap%
\pgfsetroundjoin%
\definecolor{currentfill}{rgb}{0.243520,0.319189,0.771672}%
\pgfsetfillcolor{currentfill}%
\pgfsetlinewidth{0.000000pt}%
\definecolor{currentstroke}{rgb}{0.000000,0.000000,0.000000}%
\pgfsetstrokecolor{currentstroke}%
\pgfsetdash{}{0pt}%
\pgfpathmoveto{\pgfqpoint{2.877533in}{2.204227in}}%
\pgfpathlineto{\pgfqpoint{2.920850in}{2.225463in}}%
\pgfpathlineto{\pgfqpoint{2.946045in}{2.204710in}}%
\pgfpathlineto{\pgfqpoint{2.902770in}{2.204359in}}%
\pgfpathlineto{\pgfqpoint{2.877533in}{2.204227in}}%
\pgfpathclose%
\pgfusepath{fill}%
\end{pgfscope}%
\begin{pgfscope}%
\pgfpathrectangle{\pgfqpoint{1.072000in}{0.528000in}}{\pgfqpoint{3.696000in}{3.696000in}}%
\pgfusepath{clip}%
\pgfsetbuttcap%
\pgfsetroundjoin%
\definecolor{currentfill}{rgb}{0.229806,0.298718,0.753683}%
\pgfsetfillcolor{currentfill}%
\pgfsetlinewidth{0.000000pt}%
\definecolor{currentstroke}{rgb}{0.000000,0.000000,0.000000}%
\pgfsetstrokecolor{currentstroke}%
\pgfsetdash{}{0pt}%
\pgfpathmoveto{\pgfqpoint{2.697162in}{2.178595in}}%
\pgfpathlineto{\pgfqpoint{2.740161in}{2.190763in}}%
\pgfpathlineto{\pgfqpoint{2.765800in}{2.188977in}}%
\pgfpathlineto{\pgfqpoint{2.722709in}{2.198517in}}%
\pgfpathlineto{\pgfqpoint{2.697162in}{2.178595in}}%
\pgfpathclose%
\pgfusepath{fill}%
\end{pgfscope}%
\begin{pgfscope}%
\pgfpathrectangle{\pgfqpoint{1.072000in}{0.528000in}}{\pgfqpoint{3.696000in}{3.696000in}}%
\pgfusepath{clip}%
\pgfsetbuttcap%
\pgfsetroundjoin%
\definecolor{currentfill}{rgb}{0.309060,0.413498,0.850128}%
\pgfsetfillcolor{currentfill}%
\pgfsetlinewidth{0.000000pt}%
\definecolor{currentstroke}{rgb}{0.000000,0.000000,0.000000}%
\pgfsetstrokecolor{currentstroke}%
\pgfsetdash{}{0pt}%
\pgfpathmoveto{\pgfqpoint{2.498200in}{2.309454in}}%
\pgfpathlineto{\pgfqpoint{2.541967in}{2.232762in}}%
\pgfpathlineto{\pgfqpoint{2.567325in}{2.293987in}}%
\pgfpathlineto{\pgfqpoint{2.523413in}{2.385894in}}%
\pgfpathlineto{\pgfqpoint{2.498200in}{2.309454in}}%
\pgfpathclose%
\pgfusepath{fill}%
\end{pgfscope}%
\begin{pgfscope}%
\pgfpathrectangle{\pgfqpoint{1.072000in}{0.528000in}}{\pgfqpoint{3.696000in}{3.696000in}}%
\pgfusepath{clip}%
\pgfsetbuttcap%
\pgfsetroundjoin%
\definecolor{currentfill}{rgb}{0.597777,0.727330,0.999777}%
\pgfsetfillcolor{currentfill}%
\pgfsetlinewidth{0.000000pt}%
\definecolor{currentstroke}{rgb}{0.000000,0.000000,0.000000}%
\pgfsetstrokecolor{currentstroke}%
\pgfsetdash{}{0pt}%
\pgfpathmoveto{\pgfqpoint{2.667042in}{2.692192in}}%
\pgfpathlineto{\pgfqpoint{2.710993in}{2.572799in}}%
\pgfpathlineto{\pgfqpoint{2.735740in}{2.701152in}}%
\pgfpathlineto{\pgfqpoint{2.691738in}{2.827053in}}%
\pgfpathlineto{\pgfqpoint{2.667042in}{2.692192in}}%
\pgfpathclose%
\pgfusepath{fill}%
\end{pgfscope}%
\begin{pgfscope}%
\pgfpathrectangle{\pgfqpoint{1.072000in}{0.528000in}}{\pgfqpoint{3.696000in}{3.696000in}}%
\pgfusepath{clip}%
\pgfsetbuttcap%
\pgfsetroundjoin%
\definecolor{currentfill}{rgb}{0.505423,0.643995,0.983157}%
\pgfsetfillcolor{currentfill}%
\pgfsetlinewidth{0.000000pt}%
\definecolor{currentstroke}{rgb}{0.000000,0.000000,0.000000}%
\pgfsetstrokecolor{currentstroke}%
\pgfsetdash{}{0pt}%
\pgfpathmoveto{\pgfqpoint{2.573324in}{2.579845in}}%
\pgfpathlineto{\pgfqpoint{2.617458in}{2.462347in}}%
\pgfpathlineto{\pgfqpoint{2.642298in}{2.569978in}}%
\pgfpathlineto{\pgfqpoint{2.598089in}{2.696739in}}%
\pgfpathlineto{\pgfqpoint{2.573324in}{2.579845in}}%
\pgfpathclose%
\pgfusepath{fill}%
\end{pgfscope}%
\begin{pgfscope}%
\pgfpathrectangle{\pgfqpoint{1.072000in}{0.528000in}}{\pgfqpoint{3.696000in}{3.696000in}}%
\pgfusepath{clip}%
\pgfsetbuttcap%
\pgfsetroundjoin%
\definecolor{currentfill}{rgb}{0.252663,0.332837,0.783665}%
\pgfsetfillcolor{currentfill}%
\pgfsetlinewidth{0.000000pt}%
\definecolor{currentstroke}{rgb}{0.000000,0.000000,0.000000}%
\pgfsetstrokecolor{currentstroke}%
\pgfsetdash{}{0pt}%
\pgfpathmoveto{\pgfqpoint{2.989452in}{2.216641in}}%
\pgfpathlineto{\pgfqpoint{3.033033in}{2.236959in}}%
\pgfpathlineto{\pgfqpoint{3.057870in}{2.206571in}}%
\pgfpathlineto{\pgfqpoint{3.014411in}{2.204801in}}%
\pgfpathlineto{\pgfqpoint{2.989452in}{2.216641in}}%
\pgfpathclose%
\pgfusepath{fill}%
\end{pgfscope}%
\begin{pgfscope}%
\pgfpathrectangle{\pgfqpoint{1.072000in}{0.528000in}}{\pgfqpoint{3.696000in}{3.696000in}}%
\pgfusepath{clip}%
\pgfsetbuttcap%
\pgfsetroundjoin%
\definecolor{currentfill}{rgb}{0.229806,0.298718,0.753683}%
\pgfsetfillcolor{currentfill}%
\pgfsetlinewidth{0.000000pt}%
\definecolor{currentstroke}{rgb}{0.000000,0.000000,0.000000}%
\pgfsetstrokecolor{currentstroke}%
\pgfsetdash{}{0pt}%
\pgfpathmoveto{\pgfqpoint{3.325403in}{2.174943in}}%
\pgfpathlineto{\pgfqpoint{3.369333in}{2.166055in}}%
\pgfpathlineto{\pgfqpoint{3.393523in}{2.177975in}}%
\pgfpathlineto{\pgfqpoint{3.349723in}{2.192551in}}%
\pgfpathlineto{\pgfqpoint{3.325403in}{2.174943in}}%
\pgfpathclose%
\pgfusepath{fill}%
\end{pgfscope}%
\begin{pgfscope}%
\pgfpathrectangle{\pgfqpoint{1.072000in}{0.528000in}}{\pgfqpoint{3.696000in}{3.696000in}}%
\pgfusepath{clip}%
\pgfsetbuttcap%
\pgfsetroundjoin%
\definecolor{currentfill}{rgb}{0.238948,0.312365,0.765676}%
\pgfsetfillcolor{currentfill}%
\pgfsetlinewidth{0.000000pt}%
\definecolor{currentstroke}{rgb}{0.000000,0.000000,0.000000}%
\pgfsetstrokecolor{currentstroke}%
\pgfsetdash{}{0pt}%
\pgfpathmoveto{\pgfqpoint{3.213486in}{2.197846in}}%
\pgfpathlineto{\pgfqpoint{3.257374in}{2.198649in}}%
\pgfpathlineto{\pgfqpoint{3.281578in}{2.183615in}}%
\pgfpathlineto{\pgfqpoint{3.237866in}{2.193389in}}%
\pgfpathlineto{\pgfqpoint{3.213486in}{2.197846in}}%
\pgfpathclose%
\pgfusepath{fill}%
\end{pgfscope}%
\begin{pgfscope}%
\pgfpathrectangle{\pgfqpoint{1.072000in}{0.528000in}}{\pgfqpoint{3.696000in}{3.696000in}}%
\pgfusepath{clip}%
\pgfsetbuttcap%
\pgfsetroundjoin%
\definecolor{currentfill}{rgb}{0.238948,0.312365,0.765676}%
\pgfsetfillcolor{currentfill}%
\pgfsetlinewidth{0.000000pt}%
\definecolor{currentstroke}{rgb}{0.000000,0.000000,0.000000}%
\pgfsetstrokecolor{currentstroke}%
\pgfsetdash{}{0pt}%
\pgfpathmoveto{\pgfqpoint{2.516388in}{2.185355in}}%
\pgfpathlineto{\pgfqpoint{2.559590in}{2.156215in}}%
\pgfpathlineto{\pgfqpoint{2.585341in}{2.185297in}}%
\pgfpathlineto{\pgfqpoint{2.541967in}{2.232762in}}%
\pgfpathlineto{\pgfqpoint{2.516388in}{2.185355in}}%
\pgfpathclose%
\pgfusepath{fill}%
\end{pgfscope}%
\begin{pgfscope}%
\pgfpathrectangle{\pgfqpoint{1.072000in}{0.528000in}}{\pgfqpoint{3.696000in}{3.696000in}}%
\pgfusepath{clip}%
\pgfsetbuttcap%
\pgfsetroundjoin%
\definecolor{currentfill}{rgb}{0.248091,0.326013,0.777669}%
\pgfsetfillcolor{currentfill}%
\pgfsetlinewidth{0.000000pt}%
\definecolor{currentstroke}{rgb}{0.000000,0.000000,0.000000}%
\pgfsetstrokecolor{currentstroke}%
\pgfsetdash{}{0pt}%
\pgfpathmoveto{\pgfqpoint{3.101485in}{2.214872in}}%
\pgfpathlineto{\pgfqpoint{3.145268in}{2.226915in}}%
\pgfpathlineto{\pgfqpoint{3.169734in}{2.197867in}}%
\pgfpathlineto{\pgfqpoint{3.126122in}{2.200828in}}%
\pgfpathlineto{\pgfqpoint{3.101485in}{2.214872in}}%
\pgfpathclose%
\pgfusepath{fill}%
\end{pgfscope}%
\begin{pgfscope}%
\pgfpathrectangle{\pgfqpoint{1.072000in}{0.528000in}}{\pgfqpoint{3.696000in}{3.696000in}}%
\pgfusepath{clip}%
\pgfsetbuttcap%
\pgfsetroundjoin%
\definecolor{currentfill}{rgb}{0.229806,0.298718,0.753683}%
\pgfsetfillcolor{currentfill}%
\pgfsetlinewidth{0.000000pt}%
\definecolor{currentstroke}{rgb}{0.000000,0.000000,0.000000}%
\pgfsetstrokecolor{currentstroke}%
\pgfsetdash{}{0pt}%
\pgfpathmoveto{\pgfqpoint{2.628432in}{2.166928in}}%
\pgfpathlineto{\pgfqpoint{2.671363in}{2.176104in}}%
\pgfpathlineto{\pgfqpoint{2.697162in}{2.178595in}}%
\pgfpathlineto{\pgfqpoint{2.654098in}{2.190548in}}%
\pgfpathlineto{\pgfqpoint{2.628432in}{2.166928in}}%
\pgfpathclose%
\pgfusepath{fill}%
\end{pgfscope}%
\begin{pgfscope}%
\pgfpathrectangle{\pgfqpoint{1.072000in}{0.528000in}}{\pgfqpoint{3.696000in}{3.696000in}}%
\pgfusepath{clip}%
\pgfsetbuttcap%
\pgfsetroundjoin%
\definecolor{currentfill}{rgb}{0.271104,0.360011,0.807095}%
\pgfsetfillcolor{currentfill}%
\pgfsetlinewidth{0.000000pt}%
\definecolor{currentstroke}{rgb}{0.000000,0.000000,0.000000}%
\pgfsetstrokecolor{currentstroke}%
\pgfsetdash{}{0pt}%
\pgfpathmoveto{\pgfqpoint{2.472784in}{2.245988in}}%
\pgfpathlineto{\pgfqpoint{2.516388in}{2.185355in}}%
\pgfpathlineto{\pgfqpoint{2.541967in}{2.232762in}}%
\pgfpathlineto{\pgfqpoint{2.498200in}{2.309454in}}%
\pgfpathlineto{\pgfqpoint{2.472784in}{2.245988in}}%
\pgfpathclose%
\pgfusepath{fill}%
\end{pgfscope}%
\begin{pgfscope}%
\pgfpathrectangle{\pgfqpoint{1.072000in}{0.528000in}}{\pgfqpoint{3.696000in}{3.696000in}}%
\pgfusepath{clip}%
\pgfsetbuttcap%
\pgfsetroundjoin%
\definecolor{currentfill}{rgb}{0.248091,0.326013,0.777669}%
\pgfsetfillcolor{currentfill}%
\pgfsetlinewidth{0.000000pt}%
\definecolor{currentstroke}{rgb}{0.000000,0.000000,0.000000}%
\pgfsetstrokecolor{currentstroke}%
\pgfsetdash{}{0pt}%
\pgfpathmoveto{\pgfqpoint{2.808895in}{2.200134in}}%
\pgfpathlineto{\pgfqpoint{2.852084in}{2.228783in}}%
\pgfpathlineto{\pgfqpoint{2.877533in}{2.204227in}}%
\pgfpathlineto{\pgfqpoint{2.834340in}{2.197308in}}%
\pgfpathlineto{\pgfqpoint{2.808895in}{2.200134in}}%
\pgfpathclose%
\pgfusepath{fill}%
\end{pgfscope}%
\begin{pgfscope}%
\pgfpathrectangle{\pgfqpoint{1.072000in}{0.528000in}}{\pgfqpoint{3.696000in}{3.696000in}}%
\pgfusepath{clip}%
\pgfsetbuttcap%
\pgfsetroundjoin%
\definecolor{currentfill}{rgb}{0.603162,0.731527,0.999565}%
\pgfsetfillcolor{currentfill}%
\pgfsetlinewidth{0.000000pt}%
\definecolor{currentstroke}{rgb}{0.000000,0.000000,0.000000}%
\pgfsetstrokecolor{currentstroke}%
\pgfsetdash{}{0pt}%
\pgfpathmoveto{\pgfqpoint{2.598089in}{2.696739in}}%
\pgfpathlineto{\pgfqpoint{2.642298in}{2.569978in}}%
\pgfpathlineto{\pgfqpoint{2.667042in}{2.692192in}}%
\pgfpathlineto{\pgfqpoint{2.622785in}{2.825201in}}%
\pgfpathlineto{\pgfqpoint{2.598089in}{2.696739in}}%
\pgfpathclose%
\pgfusepath{fill}%
\end{pgfscope}%
\begin{pgfscope}%
\pgfpathrectangle{\pgfqpoint{1.072000in}{0.528000in}}{\pgfqpoint{3.696000in}{3.696000in}}%
\pgfusepath{clip}%
\pgfsetbuttcap%
\pgfsetroundjoin%
\definecolor{currentfill}{rgb}{0.724041,0.814910,0.975651}%
\pgfsetfillcolor{currentfill}%
\pgfsetlinewidth{0.000000pt}%
\definecolor{currentstroke}{rgb}{0.000000,0.000000,0.000000}%
\pgfsetstrokecolor{currentstroke}%
\pgfsetdash{}{0pt}%
\pgfpathmoveto{\pgfqpoint{2.760469in}{2.842937in}}%
\pgfpathlineto{\pgfqpoint{2.804266in}{2.723597in}}%
\pgfpathlineto{\pgfqpoint{2.829062in}{2.872891in}}%
\pgfpathlineto{\pgfqpoint{2.785226in}{2.994597in}}%
\pgfpathlineto{\pgfqpoint{2.760469in}{2.842937in}}%
\pgfpathclose%
\pgfusepath{fill}%
\end{pgfscope}%
\begin{pgfscope}%
\pgfpathrectangle{\pgfqpoint{1.072000in}{0.528000in}}{\pgfqpoint{3.696000in}{3.696000in}}%
\pgfusepath{clip}%
\pgfsetbuttcap%
\pgfsetroundjoin%
\definecolor{currentfill}{rgb}{0.388852,0.516298,0.921373}%
\pgfsetfillcolor{currentfill}%
\pgfsetlinewidth{0.000000pt}%
\definecolor{currentstroke}{rgb}{0.000000,0.000000,0.000000}%
\pgfsetstrokecolor{currentstroke}%
\pgfsetdash{}{0pt}%
\pgfpathmoveto{\pgfqpoint{2.453952in}{2.413974in}}%
\pgfpathlineto{\pgfqpoint{2.498200in}{2.309454in}}%
\pgfpathlineto{\pgfqpoint{2.523413in}{2.385894in}}%
\pgfpathlineto{\pgfqpoint{2.479048in}{2.502526in}}%
\pgfpathlineto{\pgfqpoint{2.453952in}{2.413974in}}%
\pgfpathclose%
\pgfusepath{fill}%
\end{pgfscope}%
\begin{pgfscope}%
\pgfpathrectangle{\pgfqpoint{1.072000in}{0.528000in}}{\pgfqpoint{3.696000in}{3.696000in}}%
\pgfusepath{clip}%
\pgfsetbuttcap%
\pgfsetroundjoin%
\definecolor{currentfill}{rgb}{0.229806,0.298718,0.753683}%
\pgfsetfillcolor{currentfill}%
\pgfsetlinewidth{0.000000pt}%
\definecolor{currentstroke}{rgb}{0.000000,0.000000,0.000000}%
\pgfsetstrokecolor{currentstroke}%
\pgfsetdash{}{0pt}%
\pgfpathmoveto{\pgfqpoint{2.559590in}{2.156215in}}%
\pgfpathlineto{\pgfqpoint{2.602516in}{2.157816in}}%
\pgfpathlineto{\pgfqpoint{2.628432in}{2.166928in}}%
\pgfpathlineto{\pgfqpoint{2.585341in}{2.185297in}}%
\pgfpathlineto{\pgfqpoint{2.559590in}{2.156215in}}%
\pgfpathclose%
\pgfusepath{fill}%
\end{pgfscope}%
\begin{pgfscope}%
\pgfpathrectangle{\pgfqpoint{1.072000in}{0.528000in}}{\pgfqpoint{3.696000in}{3.696000in}}%
\pgfusepath{clip}%
\pgfsetbuttcap%
\pgfsetroundjoin%
\definecolor{currentfill}{rgb}{0.451739,0.588181,0.960201}%
\pgfsetfillcolor{currentfill}%
\pgfsetlinewidth{0.000000pt}%
\definecolor{currentstroke}{rgb}{0.000000,0.000000,0.000000}%
\pgfsetstrokecolor{currentstroke}%
\pgfsetdash{}{0pt}%
\pgfpathmoveto{\pgfqpoint{2.479048in}{2.502526in}}%
\pgfpathlineto{\pgfqpoint{2.523413in}{2.385894in}}%
\pgfpathlineto{\pgfqpoint{2.548444in}{2.476005in}}%
\pgfpathlineto{\pgfqpoint{2.503986in}{2.603052in}}%
\pgfpathlineto{\pgfqpoint{2.479048in}{2.502526in}}%
\pgfpathclose%
\pgfusepath{fill}%
\end{pgfscope}%
\begin{pgfscope}%
\pgfpathrectangle{\pgfqpoint{1.072000in}{0.528000in}}{\pgfqpoint{3.696000in}{3.696000in}}%
\pgfusepath{clip}%
\pgfsetbuttcap%
\pgfsetroundjoin%
\definecolor{currentfill}{rgb}{0.229806,0.298718,0.753683}%
\pgfsetfillcolor{currentfill}%
\pgfsetlinewidth{0.000000pt}%
\definecolor{currentstroke}{rgb}{0.000000,0.000000,0.000000}%
\pgfsetstrokecolor{currentstroke}%
\pgfsetdash{}{0pt}%
\pgfpathmoveto{\pgfqpoint{3.369333in}{2.166055in}}%
\pgfpathlineto{\pgfqpoint{3.413353in}{2.155969in}}%
\pgfpathlineto{\pgfqpoint{3.437427in}{2.164432in}}%
\pgfpathlineto{\pgfqpoint{3.393523in}{2.177975in}}%
\pgfpathlineto{\pgfqpoint{3.369333in}{2.166055in}}%
\pgfpathclose%
\pgfusepath{fill}%
\end{pgfscope}%
\begin{pgfscope}%
\pgfpathrectangle{\pgfqpoint{1.072000in}{0.528000in}}{\pgfqpoint{3.696000in}{3.696000in}}%
\pgfusepath{clip}%
\pgfsetbuttcap%
\pgfsetroundjoin%
\definecolor{currentfill}{rgb}{0.333490,0.446265,0.874452}%
\pgfsetfillcolor{currentfill}%
\pgfsetlinewidth{0.000000pt}%
\definecolor{currentstroke}{rgb}{0.000000,0.000000,0.000000}%
\pgfsetstrokecolor{currentstroke}%
\pgfsetdash{}{0pt}%
\pgfpathmoveto{\pgfqpoint{2.428673in}{2.337253in}}%
\pgfpathlineto{\pgfqpoint{2.472784in}{2.245988in}}%
\pgfpathlineto{\pgfqpoint{2.498200in}{2.309454in}}%
\pgfpathlineto{\pgfqpoint{2.453952in}{2.413974in}}%
\pgfpathlineto{\pgfqpoint{2.428673in}{2.337253in}}%
\pgfpathclose%
\pgfusepath{fill}%
\end{pgfscope}%
\begin{pgfscope}%
\pgfpathrectangle{\pgfqpoint{1.072000in}{0.528000in}}{\pgfqpoint{3.696000in}{3.696000in}}%
\pgfusepath{clip}%
\pgfsetbuttcap%
\pgfsetroundjoin%
\definecolor{currentfill}{rgb}{0.713852,0.808857,0.979386}%
\pgfsetfillcolor{currentfill}%
\pgfsetlinewidth{0.000000pt}%
\definecolor{currentstroke}{rgb}{0.000000,0.000000,0.000000}%
\pgfsetstrokecolor{currentstroke}%
\pgfsetdash{}{0pt}%
\pgfpathmoveto{\pgfqpoint{2.691738in}{2.827053in}}%
\pgfpathlineto{\pgfqpoint{2.735740in}{2.701152in}}%
\pgfpathlineto{\pgfqpoint{2.760469in}{2.842937in}}%
\pgfpathlineto{\pgfqpoint{2.716434in}{2.971673in}}%
\pgfpathlineto{\pgfqpoint{2.691738in}{2.827053in}}%
\pgfpathclose%
\pgfusepath{fill}%
\end{pgfscope}%
\begin{pgfscope}%
\pgfpathrectangle{\pgfqpoint{1.072000in}{0.528000in}}{\pgfqpoint{3.696000in}{3.696000in}}%
\pgfusepath{clip}%
\pgfsetbuttcap%
\pgfsetroundjoin%
\definecolor{currentfill}{rgb}{0.252663,0.332837,0.783665}%
\pgfsetfillcolor{currentfill}%
\pgfsetlinewidth{0.000000pt}%
\definecolor{currentstroke}{rgb}{0.000000,0.000000,0.000000}%
\pgfsetstrokecolor{currentstroke}%
\pgfsetdash{}{0pt}%
\pgfpathmoveto{\pgfqpoint{2.740161in}{2.190763in}}%
\pgfpathlineto{\pgfqpoint{2.783202in}{2.223941in}}%
\pgfpathlineto{\pgfqpoint{2.808895in}{2.200134in}}%
\pgfpathlineto{\pgfqpoint{2.765800in}{2.188977in}}%
\pgfpathlineto{\pgfqpoint{2.740161in}{2.190763in}}%
\pgfpathclose%
\pgfusepath{fill}%
\end{pgfscope}%
\begin{pgfscope}%
\pgfpathrectangle{\pgfqpoint{1.072000in}{0.528000in}}{\pgfqpoint{3.696000in}{3.696000in}}%
\pgfusepath{clip}%
\pgfsetbuttcap%
\pgfsetroundjoin%
\definecolor{currentfill}{rgb}{0.266381,0.353304,0.801637}%
\pgfsetfillcolor{currentfill}%
\pgfsetlinewidth{0.000000pt}%
\definecolor{currentstroke}{rgb}{0.000000,0.000000,0.000000}%
\pgfsetstrokecolor{currentstroke}%
\pgfsetdash{}{0pt}%
\pgfpathmoveto{\pgfqpoint{2.920850in}{2.225463in}}%
\pgfpathlineto{\pgfqpoint{2.964345in}{2.257494in}}%
\pgfpathlineto{\pgfqpoint{2.989452in}{2.216641in}}%
\pgfpathlineto{\pgfqpoint{2.946045in}{2.204710in}}%
\pgfpathlineto{\pgfqpoint{2.920850in}{2.225463in}}%
\pgfpathclose%
\pgfusepath{fill}%
\end{pgfscope}%
\begin{pgfscope}%
\pgfpathrectangle{\pgfqpoint{1.072000in}{0.528000in}}{\pgfqpoint{3.696000in}{3.696000in}}%
\pgfusepath{clip}%
\pgfsetbuttcap%
\pgfsetroundjoin%
\definecolor{currentfill}{rgb}{0.248091,0.326013,0.777669}%
\pgfsetfillcolor{currentfill}%
\pgfsetlinewidth{0.000000pt}%
\definecolor{currentstroke}{rgb}{0.000000,0.000000,0.000000}%
\pgfsetstrokecolor{currentstroke}%
\pgfsetdash{}{0pt}%
\pgfpathmoveto{\pgfqpoint{2.447153in}{2.194293in}}%
\pgfpathlineto{\pgfqpoint{2.490581in}{2.149993in}}%
\pgfpathlineto{\pgfqpoint{2.516388in}{2.185355in}}%
\pgfpathlineto{\pgfqpoint{2.472784in}{2.245988in}}%
\pgfpathlineto{\pgfqpoint{2.447153in}{2.194293in}}%
\pgfpathclose%
\pgfusepath{fill}%
\end{pgfscope}%
\begin{pgfscope}%
\pgfpathrectangle{\pgfqpoint{1.072000in}{0.528000in}}{\pgfqpoint{3.696000in}{3.696000in}}%
\pgfusepath{clip}%
\pgfsetbuttcap%
\pgfsetroundjoin%
\definecolor{currentfill}{rgb}{0.532568,0.669801,0.990393}%
\pgfsetfillcolor{currentfill}%
\pgfsetlinewidth{0.000000pt}%
\definecolor{currentstroke}{rgb}{0.000000,0.000000,0.000000}%
\pgfsetstrokecolor{currentstroke}%
\pgfsetdash{}{0pt}%
\pgfpathmoveto{\pgfqpoint{2.503986in}{2.603052in}}%
\pgfpathlineto{\pgfqpoint{2.548444in}{2.476005in}}%
\pgfpathlineto{\pgfqpoint{2.573324in}{2.579845in}}%
\pgfpathlineto{\pgfqpoint{2.528799in}{2.715085in}}%
\pgfpathlineto{\pgfqpoint{2.503986in}{2.603052in}}%
\pgfpathclose%
\pgfusepath{fill}%
\end{pgfscope}%
\begin{pgfscope}%
\pgfpathrectangle{\pgfqpoint{1.072000in}{0.528000in}}{\pgfqpoint{3.696000in}{3.696000in}}%
\pgfusepath{clip}%
\pgfsetbuttcap%
\pgfsetroundjoin%
\definecolor{currentfill}{rgb}{0.243520,0.319189,0.771672}%
\pgfsetfillcolor{currentfill}%
\pgfsetlinewidth{0.000000pt}%
\definecolor{currentstroke}{rgb}{0.000000,0.000000,0.000000}%
\pgfsetstrokecolor{currentstroke}%
\pgfsetdash{}{0pt}%
\pgfpathmoveto{\pgfqpoint{3.257374in}{2.198649in}}%
\pgfpathlineto{\pgfqpoint{3.301382in}{2.198435in}}%
\pgfpathlineto{\pgfqpoint{3.325403in}{2.174943in}}%
\pgfpathlineto{\pgfqpoint{3.281578in}{2.183615in}}%
\pgfpathlineto{\pgfqpoint{3.257374in}{2.198649in}}%
\pgfpathclose%
\pgfusepath{fill}%
\end{pgfscope}%
\begin{pgfscope}%
\pgfpathrectangle{\pgfqpoint{1.072000in}{0.528000in}}{\pgfqpoint{3.696000in}{3.696000in}}%
\pgfusepath{clip}%
\pgfsetbuttcap%
\pgfsetroundjoin%
\definecolor{currentfill}{rgb}{0.229806,0.298718,0.753683}%
\pgfsetfillcolor{currentfill}%
\pgfsetlinewidth{0.000000pt}%
\definecolor{currentstroke}{rgb}{0.000000,0.000000,0.000000}%
\pgfsetstrokecolor{currentstroke}%
\pgfsetdash{}{0pt}%
\pgfpathmoveto{\pgfqpoint{2.490581in}{2.149993in}}%
\pgfpathlineto{\pgfqpoint{2.533599in}{2.139115in}}%
\pgfpathlineto{\pgfqpoint{2.559590in}{2.156215in}}%
\pgfpathlineto{\pgfqpoint{2.516388in}{2.185355in}}%
\pgfpathlineto{\pgfqpoint{2.490581in}{2.149993in}}%
\pgfpathclose%
\pgfusepath{fill}%
\end{pgfscope}%
\begin{pgfscope}%
\pgfpathrectangle{\pgfqpoint{1.072000in}{0.528000in}}{\pgfqpoint{3.696000in}{3.696000in}}%
\pgfusepath{clip}%
\pgfsetbuttcap%
\pgfsetroundjoin%
\definecolor{currentfill}{rgb}{0.266381,0.353304,0.801637}%
\pgfsetfillcolor{currentfill}%
\pgfsetlinewidth{0.000000pt}%
\definecolor{currentstroke}{rgb}{0.000000,0.000000,0.000000}%
\pgfsetstrokecolor{currentstroke}%
\pgfsetdash{}{0pt}%
\pgfpathmoveto{\pgfqpoint{3.145268in}{2.226915in}}%
\pgfpathlineto{\pgfqpoint{3.189217in}{2.240065in}}%
\pgfpathlineto{\pgfqpoint{3.213486in}{2.197846in}}%
\pgfpathlineto{\pgfqpoint{3.169734in}{2.197867in}}%
\pgfpathlineto{\pgfqpoint{3.145268in}{2.226915in}}%
\pgfpathclose%
\pgfusepath{fill}%
\end{pgfscope}%
\begin{pgfscope}%
\pgfpathrectangle{\pgfqpoint{1.072000in}{0.528000in}}{\pgfqpoint{3.696000in}{3.696000in}}%
\pgfusepath{clip}%
\pgfsetbuttcap%
\pgfsetroundjoin%
\definecolor{currentfill}{rgb}{0.275827,0.366717,0.812553}%
\pgfsetfillcolor{currentfill}%
\pgfsetlinewidth{0.000000pt}%
\definecolor{currentstroke}{rgb}{0.000000,0.000000,0.000000}%
\pgfsetstrokecolor{currentstroke}%
\pgfsetdash{}{0pt}%
\pgfpathmoveto{\pgfqpoint{3.033033in}{2.236959in}}%
\pgfpathlineto{\pgfqpoint{3.076808in}{2.262407in}}%
\pgfpathlineto{\pgfqpoint{3.101485in}{2.214872in}}%
\pgfpathlineto{\pgfqpoint{3.057870in}{2.206571in}}%
\pgfpathlineto{\pgfqpoint{3.033033in}{2.236959in}}%
\pgfpathclose%
\pgfusepath{fill}%
\end{pgfscope}%
\begin{pgfscope}%
\pgfpathrectangle{\pgfqpoint{1.072000in}{0.528000in}}{\pgfqpoint{3.696000in}{3.696000in}}%
\pgfusepath{clip}%
\pgfsetbuttcap%
\pgfsetroundjoin%
\definecolor{currentfill}{rgb}{0.294718,0.393542,0.834384}%
\pgfsetfillcolor{currentfill}%
\pgfsetlinewidth{0.000000pt}%
\definecolor{currentstroke}{rgb}{0.000000,0.000000,0.000000}%
\pgfsetstrokecolor{currentstroke}%
\pgfsetdash{}{0pt}%
\pgfpathmoveto{\pgfqpoint{2.403194in}{2.271706in}}%
\pgfpathlineto{\pgfqpoint{2.447153in}{2.194293in}}%
\pgfpathlineto{\pgfqpoint{2.472784in}{2.245988in}}%
\pgfpathlineto{\pgfqpoint{2.428673in}{2.337253in}}%
\pgfpathlineto{\pgfqpoint{2.403194in}{2.271706in}}%
\pgfpathclose%
\pgfusepath{fill}%
\end{pgfscope}%
\begin{pgfscope}%
\pgfpathrectangle{\pgfqpoint{1.072000in}{0.528000in}}{\pgfqpoint{3.696000in}{3.696000in}}%
\pgfusepath{clip}%
\pgfsetbuttcap%
\pgfsetroundjoin%
\definecolor{currentfill}{rgb}{0.248091,0.326013,0.777669}%
\pgfsetfillcolor{currentfill}%
\pgfsetlinewidth{0.000000pt}%
\definecolor{currentstroke}{rgb}{0.000000,0.000000,0.000000}%
\pgfsetstrokecolor{currentstroke}%
\pgfsetdash{}{0pt}%
\pgfpathmoveto{\pgfqpoint{2.671363in}{2.176104in}}%
\pgfpathlineto{\pgfqpoint{2.714260in}{2.210028in}}%
\pgfpathlineto{\pgfqpoint{2.740161in}{2.190763in}}%
\pgfpathlineto{\pgfqpoint{2.697162in}{2.178595in}}%
\pgfpathlineto{\pgfqpoint{2.671363in}{2.176104in}}%
\pgfpathclose%
\pgfusepath{fill}%
\end{pgfscope}%
\begin{pgfscope}%
\pgfpathrectangle{\pgfqpoint{1.072000in}{0.528000in}}{\pgfqpoint{3.696000in}{3.696000in}}%
\pgfusepath{clip}%
\pgfsetbuttcap%
\pgfsetroundjoin%
\definecolor{currentfill}{rgb}{0.713852,0.808857,0.979386}%
\pgfsetfillcolor{currentfill}%
\pgfsetlinewidth{0.000000pt}%
\definecolor{currentstroke}{rgb}{0.000000,0.000000,0.000000}%
\pgfsetstrokecolor{currentstroke}%
\pgfsetdash{}{0pt}%
\pgfpathmoveto{\pgfqpoint{2.622785in}{2.825201in}}%
\pgfpathlineto{\pgfqpoint{2.667042in}{2.692192in}}%
\pgfpathlineto{\pgfqpoint{2.691738in}{2.827053in}}%
\pgfpathlineto{\pgfqpoint{2.647457in}{2.962895in}}%
\pgfpathlineto{\pgfqpoint{2.622785in}{2.825201in}}%
\pgfpathclose%
\pgfusepath{fill}%
\end{pgfscope}%
\begin{pgfscope}%
\pgfpathrectangle{\pgfqpoint{1.072000in}{0.528000in}}{\pgfqpoint{3.696000in}{3.696000in}}%
\pgfusepath{clip}%
\pgfsetbuttcap%
\pgfsetroundjoin%
\definecolor{currentfill}{rgb}{0.624703,0.748318,0.998719}%
\pgfsetfillcolor{currentfill}%
\pgfsetlinewidth{0.000000pt}%
\definecolor{currentstroke}{rgb}{0.000000,0.000000,0.000000}%
\pgfsetstrokecolor{currentstroke}%
\pgfsetdash{}{0pt}%
\pgfpathmoveto{\pgfqpoint{2.528799in}{2.715085in}}%
\pgfpathlineto{\pgfqpoint{2.573324in}{2.579845in}}%
\pgfpathlineto{\pgfqpoint{2.598089in}{2.696739in}}%
\pgfpathlineto{\pgfqpoint{2.553526in}{2.837464in}}%
\pgfpathlineto{\pgfqpoint{2.528799in}{2.715085in}}%
\pgfpathclose%
\pgfusepath{fill}%
\end{pgfscope}%
\begin{pgfscope}%
\pgfpathrectangle{\pgfqpoint{1.072000in}{0.528000in}}{\pgfqpoint{3.696000in}{3.696000in}}%
\pgfusepath{clip}%
\pgfsetbuttcap%
\pgfsetroundjoin%
\definecolor{currentfill}{rgb}{0.229806,0.298718,0.753683}%
\pgfsetfillcolor{currentfill}%
\pgfsetlinewidth{0.000000pt}%
\definecolor{currentstroke}{rgb}{0.000000,0.000000,0.000000}%
\pgfsetstrokecolor{currentstroke}%
\pgfsetdash{}{0pt}%
\pgfpathmoveto{\pgfqpoint{3.413353in}{2.155969in}}%
\pgfpathlineto{\pgfqpoint{3.457450in}{2.144079in}}%
\pgfpathlineto{\pgfqpoint{3.481434in}{2.151405in}}%
\pgfpathlineto{\pgfqpoint{3.437427in}{2.164432in}}%
\pgfpathlineto{\pgfqpoint{3.413353in}{2.155969in}}%
\pgfpathclose%
\pgfusepath{fill}%
\end{pgfscope}%
\begin{pgfscope}%
\pgfpathrectangle{\pgfqpoint{1.072000in}{0.528000in}}{\pgfqpoint{3.696000in}{3.696000in}}%
\pgfusepath{clip}%
\pgfsetbuttcap%
\pgfsetroundjoin%
\definecolor{currentfill}{rgb}{0.243520,0.319189,0.771672}%
\pgfsetfillcolor{currentfill}%
\pgfsetlinewidth{0.000000pt}%
\definecolor{currentstroke}{rgb}{0.000000,0.000000,0.000000}%
\pgfsetstrokecolor{currentstroke}%
\pgfsetdash{}{0pt}%
\pgfpathmoveto{\pgfqpoint{2.602516in}{2.157816in}}%
\pgfpathlineto{\pgfqpoint{2.645302in}{2.187927in}}%
\pgfpathlineto{\pgfqpoint{2.671363in}{2.176104in}}%
\pgfpathlineto{\pgfqpoint{2.628432in}{2.166928in}}%
\pgfpathlineto{\pgfqpoint{2.602516in}{2.157816in}}%
\pgfpathclose%
\pgfusepath{fill}%
\end{pgfscope}%
\begin{pgfscope}%
\pgfpathrectangle{\pgfqpoint{1.072000in}{0.528000in}}{\pgfqpoint{3.696000in}{3.696000in}}%
\pgfusepath{clip}%
\pgfsetbuttcap%
\pgfsetroundjoin%
\definecolor{currentfill}{rgb}{0.280550,0.373423,0.818011}%
\pgfsetfillcolor{currentfill}%
\pgfsetlinewidth{0.000000pt}%
\definecolor{currentstroke}{rgb}{0.000000,0.000000,0.000000}%
\pgfsetstrokecolor{currentstroke}%
\pgfsetdash{}{0pt}%
\pgfpathmoveto{\pgfqpoint{2.852084in}{2.228783in}}%
\pgfpathlineto{\pgfqpoint{2.895443in}{2.271152in}}%
\pgfpathlineto{\pgfqpoint{2.920850in}{2.225463in}}%
\pgfpathlineto{\pgfqpoint{2.877533in}{2.204227in}}%
\pgfpathlineto{\pgfqpoint{2.852084in}{2.228783in}}%
\pgfpathclose%
\pgfusepath{fill}%
\end{pgfscope}%
\begin{pgfscope}%
\pgfpathrectangle{\pgfqpoint{1.072000in}{0.528000in}}{\pgfqpoint{3.696000in}{3.696000in}}%
\pgfusepath{clip}%
\pgfsetbuttcap%
\pgfsetroundjoin%
\definecolor{currentfill}{rgb}{0.252663,0.332837,0.783665}%
\pgfsetfillcolor{currentfill}%
\pgfsetlinewidth{0.000000pt}%
\definecolor{currentstroke}{rgb}{0.000000,0.000000,0.000000}%
\pgfsetstrokecolor{currentstroke}%
\pgfsetdash{}{0pt}%
\pgfpathmoveto{\pgfqpoint{3.301382in}{2.198435in}}%
\pgfpathlineto{\pgfqpoint{3.345492in}{2.195708in}}%
\pgfpathlineto{\pgfqpoint{3.369333in}{2.166055in}}%
\pgfpathlineto{\pgfqpoint{3.325403in}{2.174943in}}%
\pgfpathlineto{\pgfqpoint{3.301382in}{2.198435in}}%
\pgfpathclose%
\pgfusepath{fill}%
\end{pgfscope}%
\begin{pgfscope}%
\pgfpathrectangle{\pgfqpoint{1.072000in}{0.528000in}}{\pgfqpoint{3.696000in}{3.696000in}}%
\pgfusepath{clip}%
\pgfsetbuttcap%
\pgfsetroundjoin%
\definecolor{currentfill}{rgb}{0.234377,0.305542,0.759680}%
\pgfsetfillcolor{currentfill}%
\pgfsetlinewidth{0.000000pt}%
\definecolor{currentstroke}{rgb}{0.000000,0.000000,0.000000}%
\pgfsetstrokecolor{currentstroke}%
\pgfsetdash{}{0pt}%
\pgfpathmoveto{\pgfqpoint{2.421304in}{2.152784in}}%
\pgfpathlineto{\pgfqpoint{2.464548in}{2.124551in}}%
\pgfpathlineto{\pgfqpoint{2.490581in}{2.149993in}}%
\pgfpathlineto{\pgfqpoint{2.447153in}{2.194293in}}%
\pgfpathlineto{\pgfqpoint{2.421304in}{2.152784in}}%
\pgfpathclose%
\pgfusepath{fill}%
\end{pgfscope}%
\begin{pgfscope}%
\pgfpathrectangle{\pgfqpoint{1.072000in}{0.528000in}}{\pgfqpoint{3.696000in}{3.696000in}}%
\pgfusepath{clip}%
\pgfsetbuttcap%
\pgfsetroundjoin%
\definecolor{currentfill}{rgb}{0.271104,0.360011,0.807095}%
\pgfsetfillcolor{currentfill}%
\pgfsetlinewidth{0.000000pt}%
\definecolor{currentstroke}{rgb}{0.000000,0.000000,0.000000}%
\pgfsetstrokecolor{currentstroke}%
\pgfsetdash{}{0pt}%
\pgfpathmoveto{\pgfqpoint{2.377508in}{2.216280in}}%
\pgfpathlineto{\pgfqpoint{2.421304in}{2.152784in}}%
\pgfpathlineto{\pgfqpoint{2.447153in}{2.194293in}}%
\pgfpathlineto{\pgfqpoint{2.403194in}{2.271706in}}%
\pgfpathlineto{\pgfqpoint{2.377508in}{2.216280in}}%
\pgfpathclose%
\pgfusepath{fill}%
\end{pgfscope}%
\begin{pgfscope}%
\pgfpathrectangle{\pgfqpoint{1.072000in}{0.528000in}}{\pgfqpoint{3.696000in}{3.696000in}}%
\pgfusepath{clip}%
\pgfsetbuttcap%
\pgfsetroundjoin%
\definecolor{currentfill}{rgb}{0.822420,0.856898,0.910795}%
\pgfsetfillcolor{currentfill}%
\pgfsetlinewidth{0.000000pt}%
\definecolor{currentstroke}{rgb}{0.000000,0.000000,0.000000}%
\pgfsetstrokecolor{currentstroke}%
\pgfsetdash{}{0pt}%
\pgfpathmoveto{\pgfqpoint{2.716434in}{2.971673in}}%
\pgfpathlineto{\pgfqpoint{2.760469in}{2.842937in}}%
\pgfpathlineto{\pgfqpoint{2.785226in}{2.994597in}}%
\pgfpathlineto{\pgfqpoint{2.741175in}{3.122256in}}%
\pgfpathlineto{\pgfqpoint{2.716434in}{2.971673in}}%
\pgfpathclose%
\pgfusepath{fill}%
\end{pgfscope}%
\begin{pgfscope}%
\pgfpathrectangle{\pgfqpoint{1.072000in}{0.528000in}}{\pgfqpoint{3.696000in}{3.696000in}}%
\pgfusepath{clip}%
\pgfsetbuttcap%
\pgfsetroundjoin%
\definecolor{currentfill}{rgb}{0.430507,0.564883,0.948889}%
\pgfsetfillcolor{currentfill}%
\pgfsetlinewidth{0.000000pt}%
\definecolor{currentstroke}{rgb}{0.000000,0.000000,0.000000}%
\pgfsetstrokecolor{currentstroke}%
\pgfsetdash{}{0pt}%
\pgfpathmoveto{\pgfqpoint{2.383986in}{2.456612in}}%
\pgfpathlineto{\pgfqpoint{2.428673in}{2.337253in}}%
\pgfpathlineto{\pgfqpoint{2.453952in}{2.413974in}}%
\pgfpathlineto{\pgfqpoint{2.409169in}{2.543348in}}%
\pgfpathlineto{\pgfqpoint{2.383986in}{2.456612in}}%
\pgfpathclose%
\pgfusepath{fill}%
\end{pgfscope}%
\begin{pgfscope}%
\pgfpathrectangle{\pgfqpoint{1.072000in}{0.528000in}}{\pgfqpoint{3.696000in}{3.696000in}}%
\pgfusepath{clip}%
\pgfsetbuttcap%
\pgfsetroundjoin%
\definecolor{currentfill}{rgb}{0.494638,0.633022,0.978983}%
\pgfsetfillcolor{currentfill}%
\pgfsetlinewidth{0.000000pt}%
\definecolor{currentstroke}{rgb}{0.000000,0.000000,0.000000}%
\pgfsetstrokecolor{currentstroke}%
\pgfsetdash{}{0pt}%
\pgfpathmoveto{\pgfqpoint{2.409169in}{2.543348in}}%
\pgfpathlineto{\pgfqpoint{2.453952in}{2.413974in}}%
\pgfpathlineto{\pgfqpoint{2.479048in}{2.502526in}}%
\pgfpathlineto{\pgfqpoint{2.434191in}{2.640495in}}%
\pgfpathlineto{\pgfqpoint{2.409169in}{2.543348in}}%
\pgfpathclose%
\pgfusepath{fill}%
\end{pgfscope}%
\begin{pgfscope}%
\pgfpathrectangle{\pgfqpoint{1.072000in}{0.528000in}}{\pgfqpoint{3.696000in}{3.696000in}}%
\pgfusepath{clip}%
\pgfsetbuttcap%
\pgfsetroundjoin%
\definecolor{currentfill}{rgb}{0.234377,0.305542,0.759680}%
\pgfsetfillcolor{currentfill}%
\pgfsetlinewidth{0.000000pt}%
\definecolor{currentstroke}{rgb}{0.000000,0.000000,0.000000}%
\pgfsetstrokecolor{currentstroke}%
\pgfsetdash{}{0pt}%
\pgfpathmoveto{\pgfqpoint{2.533599in}{2.139115in}}%
\pgfpathlineto{\pgfqpoint{2.576348in}{2.160278in}}%
\pgfpathlineto{\pgfqpoint{2.602516in}{2.157816in}}%
\pgfpathlineto{\pgfqpoint{2.559590in}{2.156215in}}%
\pgfpathlineto{\pgfqpoint{2.533599in}{2.139115in}}%
\pgfpathclose%
\pgfusepath{fill}%
\end{pgfscope}%
\begin{pgfscope}%
\pgfpathrectangle{\pgfqpoint{1.072000in}{0.528000in}}{\pgfqpoint{3.696000in}{3.696000in}}%
\pgfusepath{clip}%
\pgfsetbuttcap%
\pgfsetroundjoin%
\definecolor{currentfill}{rgb}{0.724041,0.814910,0.975651}%
\pgfsetfillcolor{currentfill}%
\pgfsetlinewidth{0.000000pt}%
\definecolor{currentstroke}{rgb}{0.000000,0.000000,0.000000}%
\pgfsetstrokecolor{currentstroke}%
\pgfsetdash{}{0pt}%
\pgfpathmoveto{\pgfqpoint{2.553526in}{2.837464in}}%
\pgfpathlineto{\pgfqpoint{2.598089in}{2.696739in}}%
\pgfpathlineto{\pgfqpoint{2.622785in}{2.825201in}}%
\pgfpathlineto{\pgfqpoint{2.578212in}{2.968286in}}%
\pgfpathlineto{\pgfqpoint{2.553526in}{2.837464in}}%
\pgfpathclose%
\pgfusepath{fill}%
\end{pgfscope}%
\begin{pgfscope}%
\pgfpathrectangle{\pgfqpoint{1.072000in}{0.528000in}}{\pgfqpoint{3.696000in}{3.696000in}}%
\pgfusepath{clip}%
\pgfsetbuttcap%
\pgfsetroundjoin%
\definecolor{currentfill}{rgb}{0.378598,0.503856,0.913692}%
\pgfsetfillcolor{currentfill}%
\pgfsetlinewidth{0.000000pt}%
\definecolor{currentstroke}{rgb}{0.000000,0.000000,0.000000}%
\pgfsetstrokecolor{currentstroke}%
\pgfsetdash{}{0pt}%
\pgfpathmoveto{\pgfqpoint{2.358620in}{2.380151in}}%
\pgfpathlineto{\pgfqpoint{2.403194in}{2.271706in}}%
\pgfpathlineto{\pgfqpoint{2.428673in}{2.337253in}}%
\pgfpathlineto{\pgfqpoint{2.383986in}{2.456612in}}%
\pgfpathlineto{\pgfqpoint{2.358620in}{2.380151in}}%
\pgfpathclose%
\pgfusepath{fill}%
\end{pgfscope}%
\begin{pgfscope}%
\pgfpathrectangle{\pgfqpoint{1.072000in}{0.528000in}}{\pgfqpoint{3.696000in}{3.696000in}}%
\pgfusepath{clip}%
\pgfsetbuttcap%
\pgfsetroundjoin%
\definecolor{currentfill}{rgb}{0.570616,0.704109,0.997195}%
\pgfsetfillcolor{currentfill}%
\pgfsetlinewidth{0.000000pt}%
\definecolor{currentstroke}{rgb}{0.000000,0.000000,0.000000}%
\pgfsetstrokecolor{currentstroke}%
\pgfsetdash{}{0pt}%
\pgfpathmoveto{\pgfqpoint{2.434191in}{2.640495in}}%
\pgfpathlineto{\pgfqpoint{2.479048in}{2.502526in}}%
\pgfpathlineto{\pgfqpoint{2.503986in}{2.603052in}}%
\pgfpathlineto{\pgfqpoint{2.459082in}{2.747692in}}%
\pgfpathlineto{\pgfqpoint{2.434191in}{2.640495in}}%
\pgfpathclose%
\pgfusepath{fill}%
\end{pgfscope}%
\begin{pgfscope}%
\pgfpathrectangle{\pgfqpoint{1.072000in}{0.528000in}}{\pgfqpoint{3.696000in}{3.696000in}}%
\pgfusepath{clip}%
\pgfsetbuttcap%
\pgfsetroundjoin%
\definecolor{currentfill}{rgb}{0.229806,0.298718,0.753683}%
\pgfsetfillcolor{currentfill}%
\pgfsetlinewidth{0.000000pt}%
\definecolor{currentstroke}{rgb}{0.000000,0.000000,0.000000}%
\pgfsetstrokecolor{currentstroke}%
\pgfsetdash{}{0pt}%
\pgfpathmoveto{\pgfqpoint{3.457450in}{2.144079in}}%
\pgfpathlineto{\pgfqpoint{3.501617in}{2.130179in}}%
\pgfpathlineto{\pgfqpoint{3.525541in}{2.138719in}}%
\pgfpathlineto{\pgfqpoint{3.481434in}{2.151405in}}%
\pgfpathlineto{\pgfqpoint{3.457450in}{2.144079in}}%
\pgfpathclose%
\pgfusepath{fill}%
\end{pgfscope}%
\begin{pgfscope}%
\pgfpathrectangle{\pgfqpoint{1.072000in}{0.528000in}}{\pgfqpoint{3.696000in}{3.696000in}}%
\pgfusepath{clip}%
\pgfsetbuttcap%
\pgfsetroundjoin%
\definecolor{currentfill}{rgb}{0.280550,0.373423,0.818011}%
\pgfsetfillcolor{currentfill}%
\pgfsetlinewidth{0.000000pt}%
\definecolor{currentstroke}{rgb}{0.000000,0.000000,0.000000}%
\pgfsetstrokecolor{currentstroke}%
\pgfsetdash{}{0pt}%
\pgfpathmoveto{\pgfqpoint{3.189217in}{2.240065in}}%
\pgfpathlineto{\pgfqpoint{3.233318in}{2.251944in}}%
\pgfpathlineto{\pgfqpoint{3.257374in}{2.198649in}}%
\pgfpathlineto{\pgfqpoint{3.213486in}{2.197846in}}%
\pgfpathlineto{\pgfqpoint{3.189217in}{2.240065in}}%
\pgfpathclose%
\pgfusepath{fill}%
\end{pgfscope}%
\begin{pgfscope}%
\pgfpathrectangle{\pgfqpoint{1.072000in}{0.528000in}}{\pgfqpoint{3.696000in}{3.696000in}}%
\pgfusepath{clip}%
\pgfsetbuttcap%
\pgfsetroundjoin%
\definecolor{currentfill}{rgb}{0.304174,0.406945,0.845263}%
\pgfsetfillcolor{currentfill}%
\pgfsetlinewidth{0.000000pt}%
\definecolor{currentstroke}{rgb}{0.000000,0.000000,0.000000}%
\pgfsetstrokecolor{currentstroke}%
\pgfsetdash{}{0pt}%
\pgfpathmoveto{\pgfqpoint{2.964345in}{2.257494in}}%
\pgfpathlineto{\pgfqpoint{3.008057in}{2.296619in}}%
\pgfpathlineto{\pgfqpoint{3.033033in}{2.236959in}}%
\pgfpathlineto{\pgfqpoint{2.989452in}{2.216641in}}%
\pgfpathlineto{\pgfqpoint{2.964345in}{2.257494in}}%
\pgfpathclose%
\pgfusepath{fill}%
\end{pgfscope}%
\begin{pgfscope}%
\pgfpathrectangle{\pgfqpoint{1.072000in}{0.528000in}}{\pgfqpoint{3.696000in}{3.696000in}}%
\pgfusepath{clip}%
\pgfsetbuttcap%
\pgfsetroundjoin%
\definecolor{currentfill}{rgb}{0.813693,0.854282,0.918480}%
\pgfsetfillcolor{currentfill}%
\pgfsetlinewidth{0.000000pt}%
\definecolor{currentstroke}{rgb}{0.000000,0.000000,0.000000}%
\pgfsetstrokecolor{currentstroke}%
\pgfsetdash{}{0pt}%
\pgfpathmoveto{\pgfqpoint{2.647457in}{2.962895in}}%
\pgfpathlineto{\pgfqpoint{2.691738in}{2.827053in}}%
\pgfpathlineto{\pgfqpoint{2.716434in}{2.971673in}}%
\pgfpathlineto{\pgfqpoint{2.672153in}{3.106646in}}%
\pgfpathlineto{\pgfqpoint{2.647457in}{2.962895in}}%
\pgfpathclose%
\pgfusepath{fill}%
\end{pgfscope}%
\begin{pgfscope}%
\pgfpathrectangle{\pgfqpoint{1.072000in}{0.528000in}}{\pgfqpoint{3.696000in}{3.696000in}}%
\pgfusepath{clip}%
\pgfsetbuttcap%
\pgfsetroundjoin%
\definecolor{currentfill}{rgb}{0.289996,0.386836,0.828926}%
\pgfsetfillcolor{currentfill}%
\pgfsetlinewidth{0.000000pt}%
\definecolor{currentstroke}{rgb}{0.000000,0.000000,0.000000}%
\pgfsetstrokecolor{currentstroke}%
\pgfsetdash{}{0pt}%
\pgfpathmoveto{\pgfqpoint{2.783202in}{2.223941in}}%
\pgfpathlineto{\pgfqpoint{2.826383in}{2.274242in}}%
\pgfpathlineto{\pgfqpoint{2.852084in}{2.228783in}}%
\pgfpathlineto{\pgfqpoint{2.808895in}{2.200134in}}%
\pgfpathlineto{\pgfqpoint{2.783202in}{2.223941in}}%
\pgfpathclose%
\pgfusepath{fill}%
\end{pgfscope}%
\begin{pgfscope}%
\pgfpathrectangle{\pgfqpoint{1.072000in}{0.528000in}}{\pgfqpoint{3.696000in}{3.696000in}}%
\pgfusepath{clip}%
\pgfsetbuttcap%
\pgfsetroundjoin%
\definecolor{currentfill}{rgb}{0.229806,0.298718,0.753683}%
\pgfsetfillcolor{currentfill}%
\pgfsetlinewidth{0.000000pt}%
\definecolor{currentstroke}{rgb}{0.000000,0.000000,0.000000}%
\pgfsetstrokecolor{currentstroke}%
\pgfsetdash{}{0pt}%
\pgfpathmoveto{\pgfqpoint{2.464548in}{2.124551in}}%
\pgfpathlineto{\pgfqpoint{2.507374in}{2.131339in}}%
\pgfpathlineto{\pgfqpoint{2.533599in}{2.139115in}}%
\pgfpathlineto{\pgfqpoint{2.490581in}{2.149993in}}%
\pgfpathlineto{\pgfqpoint{2.464548in}{2.124551in}}%
\pgfpathclose%
\pgfusepath{fill}%
\end{pgfscope}%
\begin{pgfscope}%
\pgfpathrectangle{\pgfqpoint{1.072000in}{0.528000in}}{\pgfqpoint{3.696000in}{3.696000in}}%
\pgfusepath{clip}%
\pgfsetbuttcap%
\pgfsetroundjoin%
\definecolor{currentfill}{rgb}{0.304174,0.406945,0.845263}%
\pgfsetfillcolor{currentfill}%
\pgfsetlinewidth{0.000000pt}%
\definecolor{currentstroke}{rgb}{0.000000,0.000000,0.000000}%
\pgfsetstrokecolor{currentstroke}%
\pgfsetdash{}{0pt}%
\pgfpathmoveto{\pgfqpoint{3.076808in}{2.262407in}}%
\pgfpathlineto{\pgfqpoint{3.120784in}{2.289829in}}%
\pgfpathlineto{\pgfqpoint{3.145268in}{2.226915in}}%
\pgfpathlineto{\pgfqpoint{3.101485in}{2.214872in}}%
\pgfpathlineto{\pgfqpoint{3.076808in}{2.262407in}}%
\pgfpathclose%
\pgfusepath{fill}%
\end{pgfscope}%
\begin{pgfscope}%
\pgfpathrectangle{\pgfqpoint{1.072000in}{0.528000in}}{\pgfqpoint{3.696000in}{3.696000in}}%
\pgfusepath{clip}%
\pgfsetbuttcap%
\pgfsetroundjoin%
\definecolor{currentfill}{rgb}{0.333490,0.446265,0.874452}%
\pgfsetfillcolor{currentfill}%
\pgfsetlinewidth{0.000000pt}%
\definecolor{currentstroke}{rgb}{0.000000,0.000000,0.000000}%
\pgfsetstrokecolor{currentstroke}%
\pgfsetdash{}{0pt}%
\pgfpathmoveto{\pgfqpoint{2.333058in}{2.313420in}}%
\pgfpathlineto{\pgfqpoint{2.377508in}{2.216280in}}%
\pgfpathlineto{\pgfqpoint{2.403194in}{2.271706in}}%
\pgfpathlineto{\pgfqpoint{2.358620in}{2.380151in}}%
\pgfpathlineto{\pgfqpoint{2.333058in}{2.313420in}}%
\pgfpathclose%
\pgfusepath{fill}%
\end{pgfscope}%
\begin{pgfscope}%
\pgfpathrectangle{\pgfqpoint{1.072000in}{0.528000in}}{\pgfqpoint{3.696000in}{3.696000in}}%
\pgfusepath{clip}%
\pgfsetbuttcap%
\pgfsetroundjoin%
\definecolor{currentfill}{rgb}{0.656683,0.771806,0.994914}%
\pgfsetfillcolor{currentfill}%
\pgfsetlinewidth{0.000000pt}%
\definecolor{currentstroke}{rgb}{0.000000,0.000000,0.000000}%
\pgfsetstrokecolor{currentstroke}%
\pgfsetdash{}{0pt}%
\pgfpathmoveto{\pgfqpoint{2.459082in}{2.747692in}}%
\pgfpathlineto{\pgfqpoint{2.503986in}{2.603052in}}%
\pgfpathlineto{\pgfqpoint{2.528799in}{2.715085in}}%
\pgfpathlineto{\pgfqpoint{2.483876in}{2.863995in}}%
\pgfpathlineto{\pgfqpoint{2.459082in}{2.747692in}}%
\pgfpathclose%
\pgfusepath{fill}%
\end{pgfscope}%
\begin{pgfscope}%
\pgfpathrectangle{\pgfqpoint{1.072000in}{0.528000in}}{\pgfqpoint{3.696000in}{3.696000in}}%
\pgfusepath{clip}%
\pgfsetbuttcap%
\pgfsetroundjoin%
\definecolor{currentfill}{rgb}{0.257234,0.339661,0.789661}%
\pgfsetfillcolor{currentfill}%
\pgfsetlinewidth{0.000000pt}%
\definecolor{currentstroke}{rgb}{0.000000,0.000000,0.000000}%
\pgfsetstrokecolor{currentstroke}%
\pgfsetdash{}{0pt}%
\pgfpathmoveto{\pgfqpoint{3.345492in}{2.195708in}}%
\pgfpathlineto{\pgfqpoint{3.389679in}{2.189364in}}%
\pgfpathlineto{\pgfqpoint{3.413353in}{2.155969in}}%
\pgfpathlineto{\pgfqpoint{3.369333in}{2.166055in}}%
\pgfpathlineto{\pgfqpoint{3.345492in}{2.195708in}}%
\pgfpathclose%
\pgfusepath{fill}%
\end{pgfscope}%
\begin{pgfscope}%
\pgfpathrectangle{\pgfqpoint{1.072000in}{0.528000in}}{\pgfqpoint{3.696000in}{3.696000in}}%
\pgfusepath{clip}%
\pgfsetbuttcap%
\pgfsetroundjoin%
\definecolor{currentfill}{rgb}{0.248091,0.326013,0.777669}%
\pgfsetfillcolor{currentfill}%
\pgfsetlinewidth{0.000000pt}%
\definecolor{currentstroke}{rgb}{0.000000,0.000000,0.000000}%
\pgfsetstrokecolor{currentstroke}%
\pgfsetdash{}{0pt}%
\pgfpathmoveto{\pgfqpoint{2.351614in}{2.169642in}}%
\pgfpathlineto{\pgfqpoint{2.395245in}{2.119631in}}%
\pgfpathlineto{\pgfqpoint{2.421304in}{2.152784in}}%
\pgfpathlineto{\pgfqpoint{2.377508in}{2.216280in}}%
\pgfpathlineto{\pgfqpoint{2.351614in}{2.169642in}}%
\pgfpathclose%
\pgfusepath{fill}%
\end{pgfscope}%
\begin{pgfscope}%
\pgfpathrectangle{\pgfqpoint{1.072000in}{0.528000in}}{\pgfqpoint{3.696000in}{3.696000in}}%
\pgfusepath{clip}%
\pgfsetbuttcap%
\pgfsetroundjoin%
\definecolor{currentfill}{rgb}{0.229806,0.298718,0.753683}%
\pgfsetfillcolor{currentfill}%
\pgfsetlinewidth{0.000000pt}%
\definecolor{currentstroke}{rgb}{0.000000,0.000000,0.000000}%
\pgfsetstrokecolor{currentstroke}%
\pgfsetdash{}{0pt}%
\pgfpathmoveto{\pgfqpoint{3.501617in}{2.130179in}}%
\pgfpathlineto{\pgfqpoint{3.545847in}{2.114467in}}%
\pgfpathlineto{\pgfqpoint{3.569756in}{2.126552in}}%
\pgfpathlineto{\pgfqpoint{3.525541in}{2.138719in}}%
\pgfpathlineto{\pgfqpoint{3.501617in}{2.130179in}}%
\pgfpathclose%
\pgfusepath{fill}%
\end{pgfscope}%
\begin{pgfscope}%
\pgfpathrectangle{\pgfqpoint{1.072000in}{0.528000in}}{\pgfqpoint{3.696000in}{3.696000in}}%
\pgfusepath{clip}%
\pgfsetbuttcap%
\pgfsetroundjoin%
\definecolor{currentfill}{rgb}{0.289996,0.386836,0.828926}%
\pgfsetfillcolor{currentfill}%
\pgfsetlinewidth{0.000000pt}%
\definecolor{currentstroke}{rgb}{0.000000,0.000000,0.000000}%
\pgfsetstrokecolor{currentstroke}%
\pgfsetdash{}{0pt}%
\pgfpathmoveto{\pgfqpoint{2.714260in}{2.210028in}}%
\pgfpathlineto{\pgfqpoint{2.757239in}{2.264872in}}%
\pgfpathlineto{\pgfqpoint{2.783202in}{2.223941in}}%
\pgfpathlineto{\pgfqpoint{2.740161in}{2.190763in}}%
\pgfpathlineto{\pgfqpoint{2.714260in}{2.210028in}}%
\pgfpathclose%
\pgfusepath{fill}%
\end{pgfscope}%
\begin{pgfscope}%
\pgfpathrectangle{\pgfqpoint{1.072000in}{0.528000in}}{\pgfqpoint{3.696000in}{3.696000in}}%
\pgfusepath{clip}%
\pgfsetbuttcap%
\pgfsetroundjoin%
\definecolor{currentfill}{rgb}{0.229806,0.298718,0.753683}%
\pgfsetfillcolor{currentfill}%
\pgfsetlinewidth{0.000000pt}%
\definecolor{currentstroke}{rgb}{0.000000,0.000000,0.000000}%
\pgfsetstrokecolor{currentstroke}%
\pgfsetdash{}{0pt}%
\pgfpathmoveto{\pgfqpoint{2.395245in}{2.119631in}}%
\pgfpathlineto{\pgfqpoint{2.438301in}{2.106708in}}%
\pgfpathlineto{\pgfqpoint{2.464548in}{2.124551in}}%
\pgfpathlineto{\pgfqpoint{2.421304in}{2.152784in}}%
\pgfpathlineto{\pgfqpoint{2.395245in}{2.119631in}}%
\pgfpathclose%
\pgfusepath{fill}%
\end{pgfscope}%
\begin{pgfscope}%
\pgfpathrectangle{\pgfqpoint{1.072000in}{0.528000in}}{\pgfqpoint{3.696000in}{3.696000in}}%
\pgfusepath{clip}%
\pgfsetbuttcap%
\pgfsetroundjoin%
\definecolor{currentfill}{rgb}{0.822420,0.856898,0.910795}%
\pgfsetfillcolor{currentfill}%
\pgfsetlinewidth{0.000000pt}%
\definecolor{currentstroke}{rgb}{0.000000,0.000000,0.000000}%
\pgfsetstrokecolor{currentstroke}%
\pgfsetdash{}{0pt}%
\pgfpathmoveto{\pgfqpoint{2.578212in}{2.968286in}}%
\pgfpathlineto{\pgfqpoint{2.622785in}{2.825201in}}%
\pgfpathlineto{\pgfqpoint{2.647457in}{2.962895in}}%
\pgfpathlineto{\pgfqpoint{2.602903in}{3.104895in}}%
\pgfpathlineto{\pgfqpoint{2.578212in}{2.968286in}}%
\pgfpathclose%
\pgfusepath{fill}%
\end{pgfscope}%
\begin{pgfscope}%
\pgfpathrectangle{\pgfqpoint{1.072000in}{0.528000in}}{\pgfqpoint{3.696000in}{3.696000in}}%
\pgfusepath{clip}%
\pgfsetbuttcap%
\pgfsetroundjoin%
\definecolor{currentfill}{rgb}{0.304174,0.406945,0.845263}%
\pgfsetfillcolor{currentfill}%
\pgfsetlinewidth{0.000000pt}%
\definecolor{currentstroke}{rgb}{0.000000,0.000000,0.000000}%
\pgfsetstrokecolor{currentstroke}%
\pgfsetdash{}{0pt}%
\pgfpathmoveto{\pgfqpoint{2.307293in}{2.255575in}}%
\pgfpathlineto{\pgfqpoint{2.351614in}{2.169642in}}%
\pgfpathlineto{\pgfqpoint{2.377508in}{2.216280in}}%
\pgfpathlineto{\pgfqpoint{2.333058in}{2.313420in}}%
\pgfpathlineto{\pgfqpoint{2.307293in}{2.255575in}}%
\pgfpathclose%
\pgfusepath{fill}%
\end{pgfscope}%
\begin{pgfscope}%
\pgfpathrectangle{\pgfqpoint{1.072000in}{0.528000in}}{\pgfqpoint{3.696000in}{3.696000in}}%
\pgfusepath{clip}%
\pgfsetbuttcap%
\pgfsetroundjoin%
\definecolor{currentfill}{rgb}{0.748682,0.827679,0.963334}%
\pgfsetfillcolor{currentfill}%
\pgfsetlinewidth{0.000000pt}%
\definecolor{currentstroke}{rgb}{0.000000,0.000000,0.000000}%
\pgfsetstrokecolor{currentstroke}%
\pgfsetdash{}{0pt}%
\pgfpathmoveto{\pgfqpoint{2.483876in}{2.863995in}}%
\pgfpathlineto{\pgfqpoint{2.528799in}{2.715085in}}%
\pgfpathlineto{\pgfqpoint{2.553526in}{2.837464in}}%
\pgfpathlineto{\pgfqpoint{2.508615in}{2.987824in}}%
\pgfpathlineto{\pgfqpoint{2.483876in}{2.863995in}}%
\pgfpathclose%
\pgfusepath{fill}%
\end{pgfscope}%
\begin{pgfscope}%
\pgfpathrectangle{\pgfqpoint{1.072000in}{0.528000in}}{\pgfqpoint{3.696000in}{3.696000in}}%
\pgfusepath{clip}%
\pgfsetbuttcap%
\pgfsetroundjoin%
\definecolor{currentfill}{rgb}{0.280550,0.373423,0.818011}%
\pgfsetfillcolor{currentfill}%
\pgfsetlinewidth{0.000000pt}%
\definecolor{currentstroke}{rgb}{0.000000,0.000000,0.000000}%
\pgfsetstrokecolor{currentstroke}%
\pgfsetdash{}{0pt}%
\pgfpathmoveto{\pgfqpoint{2.645302in}{2.187927in}}%
\pgfpathlineto{\pgfqpoint{2.688086in}{2.243023in}}%
\pgfpathlineto{\pgfqpoint{2.714260in}{2.210028in}}%
\pgfpathlineto{\pgfqpoint{2.671363in}{2.176104in}}%
\pgfpathlineto{\pgfqpoint{2.645302in}{2.187927in}}%
\pgfpathclose%
\pgfusepath{fill}%
\end{pgfscope}%
\begin{pgfscope}%
\pgfpathrectangle{\pgfqpoint{1.072000in}{0.528000in}}{\pgfqpoint{3.696000in}{3.696000in}}%
\pgfusepath{clip}%
\pgfsetbuttcap%
\pgfsetroundjoin%
\definecolor{currentfill}{rgb}{0.299441,0.400248,0.839842}%
\pgfsetfillcolor{currentfill}%
\pgfsetlinewidth{0.000000pt}%
\definecolor{currentstroke}{rgb}{0.000000,0.000000,0.000000}%
\pgfsetstrokecolor{currentstroke}%
\pgfsetdash{}{0pt}%
\pgfpathmoveto{\pgfqpoint{3.233318in}{2.251944in}}%
\pgfpathlineto{\pgfqpoint{3.277549in}{2.260519in}}%
\pgfpathlineto{\pgfqpoint{3.301382in}{2.198435in}}%
\pgfpathlineto{\pgfqpoint{3.257374in}{2.198649in}}%
\pgfpathlineto{\pgfqpoint{3.233318in}{2.251944in}}%
\pgfpathclose%
\pgfusepath{fill}%
\end{pgfscope}%
\begin{pgfscope}%
\pgfpathrectangle{\pgfqpoint{1.072000in}{0.528000in}}{\pgfqpoint{3.696000in}{3.696000in}}%
\pgfusepath{clip}%
\pgfsetbuttcap%
\pgfsetroundjoin%
\definecolor{currentfill}{rgb}{0.328604,0.439712,0.869587}%
\pgfsetfillcolor{currentfill}%
\pgfsetlinewidth{0.000000pt}%
\definecolor{currentstroke}{rgb}{0.000000,0.000000,0.000000}%
\pgfsetstrokecolor{currentstroke}%
\pgfsetdash{}{0pt}%
\pgfpathmoveto{\pgfqpoint{2.895443in}{2.271152in}}%
\pgfpathlineto{\pgfqpoint{2.939028in}{2.323147in}}%
\pgfpathlineto{\pgfqpoint{2.964345in}{2.257494in}}%
\pgfpathlineto{\pgfqpoint{2.920850in}{2.225463in}}%
\pgfpathlineto{\pgfqpoint{2.895443in}{2.271152in}}%
\pgfpathclose%
\pgfusepath{fill}%
\end{pgfscope}%
\begin{pgfscope}%
\pgfpathrectangle{\pgfqpoint{1.072000in}{0.528000in}}{\pgfqpoint{3.696000in}{3.696000in}}%
\pgfusepath{clip}%
\pgfsetbuttcap%
\pgfsetroundjoin%
\definecolor{currentfill}{rgb}{0.261805,0.346484,0.795658}%
\pgfsetfillcolor{currentfill}%
\pgfsetlinewidth{0.000000pt}%
\definecolor{currentstroke}{rgb}{0.000000,0.000000,0.000000}%
\pgfsetstrokecolor{currentstroke}%
\pgfsetdash{}{0pt}%
\pgfpathmoveto{\pgfqpoint{3.389679in}{2.189364in}}%
\pgfpathlineto{\pgfqpoint{3.433923in}{2.178726in}}%
\pgfpathlineto{\pgfqpoint{3.457450in}{2.144079in}}%
\pgfpathlineto{\pgfqpoint{3.413353in}{2.155969in}}%
\pgfpathlineto{\pgfqpoint{3.389679in}{2.189364in}}%
\pgfpathclose%
\pgfusepath{fill}%
\end{pgfscope}%
\begin{pgfscope}%
\pgfpathrectangle{\pgfqpoint{1.072000in}{0.528000in}}{\pgfqpoint{3.696000in}{3.696000in}}%
\pgfusepath{clip}%
\pgfsetbuttcap%
\pgfsetroundjoin%
\definecolor{currentfill}{rgb}{0.906154,0.842091,0.806151}%
\pgfsetfillcolor{currentfill}%
\pgfsetlinewidth{0.000000pt}%
\definecolor{currentstroke}{rgb}{0.000000,0.000000,0.000000}%
\pgfsetstrokecolor{currentstroke}%
\pgfsetdash{}{0pt}%
\pgfpathmoveto{\pgfqpoint{2.672153in}{3.106646in}}%
\pgfpathlineto{\pgfqpoint{2.716434in}{2.971673in}}%
\pgfpathlineto{\pgfqpoint{2.741175in}{3.122256in}}%
\pgfpathlineto{\pgfqpoint{2.696915in}{3.252506in}}%
\pgfpathlineto{\pgfqpoint{2.672153in}{3.106646in}}%
\pgfpathclose%
\pgfusepath{fill}%
\end{pgfscope}%
\begin{pgfscope}%
\pgfpathrectangle{\pgfqpoint{1.072000in}{0.528000in}}{\pgfqpoint{3.696000in}{3.696000in}}%
\pgfusepath{clip}%
\pgfsetbuttcap%
\pgfsetroundjoin%
\definecolor{currentfill}{rgb}{0.266381,0.353304,0.801637}%
\pgfsetfillcolor{currentfill}%
\pgfsetlinewidth{0.000000pt}%
\definecolor{currentstroke}{rgb}{0.000000,0.000000,0.000000}%
\pgfsetstrokecolor{currentstroke}%
\pgfsetdash{}{0pt}%
\pgfpathmoveto{\pgfqpoint{2.576348in}{2.160278in}}%
\pgfpathlineto{\pgfqpoint{2.618980in}{2.210551in}}%
\pgfpathlineto{\pgfqpoint{2.645302in}{2.187927in}}%
\pgfpathlineto{\pgfqpoint{2.602516in}{2.157816in}}%
\pgfpathlineto{\pgfqpoint{2.576348in}{2.160278in}}%
\pgfpathclose%
\pgfusepath{fill}%
\end{pgfscope}%
\begin{pgfscope}%
\pgfpathrectangle{\pgfqpoint{1.072000in}{0.528000in}}{\pgfqpoint{3.696000in}{3.696000in}}%
\pgfusepath{clip}%
\pgfsetbuttcap%
\pgfsetroundjoin%
\definecolor{currentfill}{rgb}{0.229806,0.298718,0.753683}%
\pgfsetfillcolor{currentfill}%
\pgfsetlinewidth{0.000000pt}%
\definecolor{currentstroke}{rgb}{0.000000,0.000000,0.000000}%
\pgfsetstrokecolor{currentstroke}%
\pgfsetdash{}{0pt}%
\pgfpathmoveto{\pgfqpoint{3.545847in}{2.114467in}}%
\pgfpathlineto{\pgfqpoint{3.590145in}{2.097544in}}%
\pgfpathlineto{\pgfqpoint{3.614088in}{2.115425in}}%
\pgfpathlineto{\pgfqpoint{3.569756in}{2.126552in}}%
\pgfpathlineto{\pgfqpoint{3.545847in}{2.114467in}}%
\pgfpathclose%
\pgfusepath{fill}%
\end{pgfscope}%
\begin{pgfscope}%
\pgfpathrectangle{\pgfqpoint{1.072000in}{0.528000in}}{\pgfqpoint{3.696000in}{3.696000in}}%
\pgfusepath{clip}%
\pgfsetbuttcap%
\pgfsetroundjoin%
\definecolor{currentfill}{rgb}{0.252663,0.332837,0.783665}%
\pgfsetfillcolor{currentfill}%
\pgfsetlinewidth{0.000000pt}%
\definecolor{currentstroke}{rgb}{0.000000,0.000000,0.000000}%
\pgfsetstrokecolor{currentstroke}%
\pgfsetdash{}{0pt}%
\pgfpathmoveto{\pgfqpoint{2.507374in}{2.131339in}}%
\pgfpathlineto{\pgfqpoint{2.549941in}{2.171128in}}%
\pgfpathlineto{\pgfqpoint{2.576348in}{2.160278in}}%
\pgfpathlineto{\pgfqpoint{2.533599in}{2.139115in}}%
\pgfpathlineto{\pgfqpoint{2.507374in}{2.131339in}}%
\pgfpathclose%
\pgfusepath{fill}%
\end{pgfscope}%
\begin{pgfscope}%
\pgfpathrectangle{\pgfqpoint{1.072000in}{0.528000in}}{\pgfqpoint{3.696000in}{3.696000in}}%
\pgfusepath{clip}%
\pgfsetbuttcap%
\pgfsetroundjoin%
\definecolor{currentfill}{rgb}{0.238948,0.312365,0.765676}%
\pgfsetfillcolor{currentfill}%
\pgfsetlinewidth{0.000000pt}%
\definecolor{currentstroke}{rgb}{0.000000,0.000000,0.000000}%
\pgfsetstrokecolor{currentstroke}%
\pgfsetdash{}{0pt}%
\pgfpathmoveto{\pgfqpoint{2.325520in}{2.130300in}}%
\pgfpathlineto{\pgfqpoint{2.368989in}{2.092895in}}%
\pgfpathlineto{\pgfqpoint{2.395245in}{2.119631in}}%
\pgfpathlineto{\pgfqpoint{2.351614in}{2.169642in}}%
\pgfpathlineto{\pgfqpoint{2.325520in}{2.130300in}}%
\pgfpathclose%
\pgfusepath{fill}%
\end{pgfscope}%
\begin{pgfscope}%
\pgfpathrectangle{\pgfqpoint{1.072000in}{0.528000in}}{\pgfqpoint{3.696000in}{3.696000in}}%
\pgfusepath{clip}%
\pgfsetbuttcap%
\pgfsetroundjoin%
\definecolor{currentfill}{rgb}{0.238948,0.312365,0.765676}%
\pgfsetfillcolor{currentfill}%
\pgfsetlinewidth{0.000000pt}%
\definecolor{currentstroke}{rgb}{0.000000,0.000000,0.000000}%
\pgfsetstrokecolor{currentstroke}%
\pgfsetdash{}{0pt}%
\pgfpathmoveto{\pgfqpoint{2.438301in}{2.106708in}}%
\pgfpathlineto{\pgfqpoint{2.480935in}{2.130083in}}%
\pgfpathlineto{\pgfqpoint{2.507374in}{2.131339in}}%
\pgfpathlineto{\pgfqpoint{2.464548in}{2.124551in}}%
\pgfpathlineto{\pgfqpoint{2.438301in}{2.106708in}}%
\pgfpathclose%
\pgfusepath{fill}%
\end{pgfscope}%
\begin{pgfscope}%
\pgfpathrectangle{\pgfqpoint{1.072000in}{0.528000in}}{\pgfqpoint{3.696000in}{3.696000in}}%
\pgfusepath{clip}%
\pgfsetbuttcap%
\pgfsetroundjoin%
\definecolor{currentfill}{rgb}{0.333490,0.446265,0.874452}%
\pgfsetfillcolor{currentfill}%
\pgfsetlinewidth{0.000000pt}%
\definecolor{currentstroke}{rgb}{0.000000,0.000000,0.000000}%
\pgfsetstrokecolor{currentstroke}%
\pgfsetdash{}{0pt}%
\pgfpathmoveto{\pgfqpoint{3.120784in}{2.289829in}}%
\pgfpathlineto{\pgfqpoint{3.164954in}{2.316306in}}%
\pgfpathlineto{\pgfqpoint{3.189217in}{2.240065in}}%
\pgfpathlineto{\pgfqpoint{3.145268in}{2.226915in}}%
\pgfpathlineto{\pgfqpoint{3.120784in}{2.289829in}}%
\pgfpathclose%
\pgfusepath{fill}%
\end{pgfscope}%
\begin{pgfscope}%
\pgfpathrectangle{\pgfqpoint{1.072000in}{0.528000in}}{\pgfqpoint{3.696000in}{3.696000in}}%
\pgfusepath{clip}%
\pgfsetbuttcap%
\pgfsetroundjoin%
\definecolor{currentfill}{rgb}{0.343278,0.459354,0.884122}%
\pgfsetfillcolor{currentfill}%
\pgfsetlinewidth{0.000000pt}%
\definecolor{currentstroke}{rgb}{0.000000,0.000000,0.000000}%
\pgfsetstrokecolor{currentstroke}%
\pgfsetdash{}{0pt}%
\pgfpathmoveto{\pgfqpoint{3.008057in}{2.296619in}}%
\pgfpathlineto{\pgfqpoint{3.052004in}{2.339167in}}%
\pgfpathlineto{\pgfqpoint{3.076808in}{2.262407in}}%
\pgfpathlineto{\pgfqpoint{3.033033in}{2.236959in}}%
\pgfpathlineto{\pgfqpoint{3.008057in}{2.296619in}}%
\pgfpathclose%
\pgfusepath{fill}%
\end{pgfscope}%
\begin{pgfscope}%
\pgfpathrectangle{\pgfqpoint{1.072000in}{0.528000in}}{\pgfqpoint{3.696000in}{3.696000in}}%
\pgfusepath{clip}%
\pgfsetbuttcap%
\pgfsetroundjoin%
\definecolor{currentfill}{rgb}{0.280550,0.373423,0.818011}%
\pgfsetfillcolor{currentfill}%
\pgfsetlinewidth{0.000000pt}%
\definecolor{currentstroke}{rgb}{0.000000,0.000000,0.000000}%
\pgfsetstrokecolor{currentstroke}%
\pgfsetdash{}{0pt}%
\pgfpathmoveto{\pgfqpoint{2.281325in}{2.205572in}}%
\pgfpathlineto{\pgfqpoint{2.325520in}{2.130300in}}%
\pgfpathlineto{\pgfqpoint{2.351614in}{2.169642in}}%
\pgfpathlineto{\pgfqpoint{2.307293in}{2.255575in}}%
\pgfpathlineto{\pgfqpoint{2.281325in}{2.205572in}}%
\pgfpathclose%
\pgfusepath{fill}%
\end{pgfscope}%
\begin{pgfscope}%
\pgfpathrectangle{\pgfqpoint{1.072000in}{0.528000in}}{\pgfqpoint{3.696000in}{3.696000in}}%
\pgfusepath{clip}%
\pgfsetbuttcap%
\pgfsetroundjoin%
\definecolor{currentfill}{rgb}{0.839351,0.861167,0.894494}%
\pgfsetfillcolor{currentfill}%
\pgfsetlinewidth{0.000000pt}%
\definecolor{currentstroke}{rgb}{0.000000,0.000000,0.000000}%
\pgfsetstrokecolor{currentstroke}%
\pgfsetdash{}{0pt}%
\pgfpathmoveto{\pgfqpoint{2.508615in}{2.987824in}}%
\pgfpathlineto{\pgfqpoint{2.553526in}{2.837464in}}%
\pgfpathlineto{\pgfqpoint{2.578212in}{2.968286in}}%
\pgfpathlineto{\pgfqpoint{2.533343in}{3.116939in}}%
\pgfpathlineto{\pgfqpoint{2.508615in}{2.987824in}}%
\pgfpathclose%
\pgfusepath{fill}%
\end{pgfscope}%
\begin{pgfscope}%
\pgfpathrectangle{\pgfqpoint{1.072000in}{0.528000in}}{\pgfqpoint{3.696000in}{3.696000in}}%
\pgfusepath{clip}%
\pgfsetbuttcap%
\pgfsetroundjoin%
\definecolor{currentfill}{rgb}{0.906154,0.842091,0.806151}%
\pgfsetfillcolor{currentfill}%
\pgfsetlinewidth{0.000000pt}%
\definecolor{currentstroke}{rgb}{0.000000,0.000000,0.000000}%
\pgfsetstrokecolor{currentstroke}%
\pgfsetdash{}{0pt}%
\pgfpathmoveto{\pgfqpoint{2.602903in}{3.104895in}}%
\pgfpathlineto{\pgfqpoint{2.647457in}{2.962895in}}%
\pgfpathlineto{\pgfqpoint{2.672153in}{3.106646in}}%
\pgfpathlineto{\pgfqpoint{2.627643in}{3.243918in}}%
\pgfpathlineto{\pgfqpoint{2.602903in}{3.104895in}}%
\pgfpathclose%
\pgfusepath{fill}%
\end{pgfscope}%
\begin{pgfscope}%
\pgfpathrectangle{\pgfqpoint{1.072000in}{0.528000in}}{\pgfqpoint{3.696000in}{3.696000in}}%
\pgfusepath{clip}%
\pgfsetbuttcap%
\pgfsetroundjoin%
\definecolor{currentfill}{rgb}{0.261805,0.346484,0.795658}%
\pgfsetfillcolor{currentfill}%
\pgfsetlinewidth{0.000000pt}%
\definecolor{currentstroke}{rgb}{0.000000,0.000000,0.000000}%
\pgfsetstrokecolor{currentstroke}%
\pgfsetdash{}{0pt}%
\pgfpathmoveto{\pgfqpoint{3.433923in}{2.178726in}}%
\pgfpathlineto{\pgfqpoint{3.478203in}{2.163557in}}%
\pgfpathlineto{\pgfqpoint{3.501617in}{2.130179in}}%
\pgfpathlineto{\pgfqpoint{3.457450in}{2.144079in}}%
\pgfpathlineto{\pgfqpoint{3.433923in}{2.178726in}}%
\pgfpathclose%
\pgfusepath{fill}%
\end{pgfscope}%
\begin{pgfscope}%
\pgfpathrectangle{\pgfqpoint{1.072000in}{0.528000in}}{\pgfqpoint{3.696000in}{3.696000in}}%
\pgfusepath{clip}%
\pgfsetbuttcap%
\pgfsetroundjoin%
\definecolor{currentfill}{rgb}{0.229806,0.298718,0.753683}%
\pgfsetfillcolor{currentfill}%
\pgfsetlinewidth{0.000000pt}%
\definecolor{currentstroke}{rgb}{0.000000,0.000000,0.000000}%
\pgfsetstrokecolor{currentstroke}%
\pgfsetdash{}{0pt}%
\pgfpathmoveto{\pgfqpoint{2.368989in}{2.092895in}}%
\pgfpathlineto{\pgfqpoint{2.411863in}{2.094083in}}%
\pgfpathlineto{\pgfqpoint{2.438301in}{2.106708in}}%
\pgfpathlineto{\pgfqpoint{2.395245in}{2.119631in}}%
\pgfpathlineto{\pgfqpoint{2.368989in}{2.092895in}}%
\pgfpathclose%
\pgfusepath{fill}%
\end{pgfscope}%
\begin{pgfscope}%
\pgfpathrectangle{\pgfqpoint{1.072000in}{0.528000in}}{\pgfqpoint{3.696000in}{3.696000in}}%
\pgfusepath{clip}%
\pgfsetbuttcap%
\pgfsetroundjoin%
\definecolor{currentfill}{rgb}{0.343278,0.459354,0.884122}%
\pgfsetfillcolor{currentfill}%
\pgfsetlinewidth{0.000000pt}%
\definecolor{currentstroke}{rgb}{0.000000,0.000000,0.000000}%
\pgfsetstrokecolor{currentstroke}%
\pgfsetdash{}{0pt}%
\pgfpathmoveto{\pgfqpoint{2.826383in}{2.274242in}}%
\pgfpathlineto{\pgfqpoint{2.869781in}{2.337255in}}%
\pgfpathlineto{\pgfqpoint{2.895443in}{2.271152in}}%
\pgfpathlineto{\pgfqpoint{2.852084in}{2.228783in}}%
\pgfpathlineto{\pgfqpoint{2.826383in}{2.274242in}}%
\pgfpathclose%
\pgfusepath{fill}%
\end{pgfscope}%
\begin{pgfscope}%
\pgfpathrectangle{\pgfqpoint{1.072000in}{0.528000in}}{\pgfqpoint{3.696000in}{3.696000in}}%
\pgfusepath{clip}%
\pgfsetbuttcap%
\pgfsetroundjoin%
\definecolor{currentfill}{rgb}{0.229806,0.298718,0.753683}%
\pgfsetfillcolor{currentfill}%
\pgfsetlinewidth{0.000000pt}%
\definecolor{currentstroke}{rgb}{0.000000,0.000000,0.000000}%
\pgfsetstrokecolor{currentstroke}%
\pgfsetdash{}{0pt}%
\pgfpathmoveto{\pgfqpoint{3.590145in}{2.097544in}}%
\pgfpathlineto{\pgfqpoint{3.634527in}{2.080392in}}%
\pgfpathlineto{\pgfqpoint{3.658561in}{2.106181in}}%
\pgfpathlineto{\pgfqpoint{3.614088in}{2.115425in}}%
\pgfpathlineto{\pgfqpoint{3.590145in}{2.097544in}}%
\pgfpathclose%
\pgfusepath{fill}%
\end{pgfscope}%
\begin{pgfscope}%
\pgfpathrectangle{\pgfqpoint{1.072000in}{0.528000in}}{\pgfqpoint{3.696000in}{3.696000in}}%
\pgfusepath{clip}%
\pgfsetbuttcap%
\pgfsetroundjoin%
\definecolor{currentfill}{rgb}{0.313946,0.420052,0.854993}%
\pgfsetfillcolor{currentfill}%
\pgfsetlinewidth{0.000000pt}%
\definecolor{currentstroke}{rgb}{0.000000,0.000000,0.000000}%
\pgfsetstrokecolor{currentstroke}%
\pgfsetdash{}{0pt}%
\pgfpathmoveto{\pgfqpoint{3.277549in}{2.260519in}}%
\pgfpathlineto{\pgfqpoint{3.321882in}{2.264160in}}%
\pgfpathlineto{\pgfqpoint{3.345492in}{2.195708in}}%
\pgfpathlineto{\pgfqpoint{3.301382in}{2.198435in}}%
\pgfpathlineto{\pgfqpoint{3.277549in}{2.260519in}}%
\pgfpathclose%
\pgfusepath{fill}%
\end{pgfscope}%
\begin{pgfscope}%
\pgfpathrectangle{\pgfqpoint{1.072000in}{0.528000in}}{\pgfqpoint{3.696000in}{3.696000in}}%
\pgfusepath{clip}%
\pgfsetbuttcap%
\pgfsetroundjoin%
\definecolor{currentfill}{rgb}{0.916071,0.833977,0.788693}%
\pgfsetfillcolor{currentfill}%
\pgfsetlinewidth{0.000000pt}%
\definecolor{currentstroke}{rgb}{0.000000,0.000000,0.000000}%
\pgfsetstrokecolor{currentstroke}%
\pgfsetdash{}{0pt}%
\pgfpathmoveto{\pgfqpoint{2.533343in}{3.116939in}}%
\pgfpathlineto{\pgfqpoint{2.578212in}{2.968286in}}%
\pgfpathlineto{\pgfqpoint{2.602903in}{3.104895in}}%
\pgfpathlineto{\pgfqpoint{2.558106in}{3.248452in}}%
\pgfpathlineto{\pgfqpoint{2.533343in}{3.116939in}}%
\pgfpathclose%
\pgfusepath{fill}%
\end{pgfscope}%
\begin{pgfscope}%
\pgfpathrectangle{\pgfqpoint{1.072000in}{0.528000in}}{\pgfqpoint{3.696000in}{3.696000in}}%
\pgfusepath{clip}%
\pgfsetbuttcap%
\pgfsetroundjoin%
\definecolor{currentfill}{rgb}{0.229806,0.298718,0.753683}%
\pgfsetfillcolor{currentfill}%
\pgfsetlinewidth{0.000000pt}%
\definecolor{currentstroke}{rgb}{0.000000,0.000000,0.000000}%
\pgfsetstrokecolor{currentstroke}%
\pgfsetdash{}{0pt}%
\pgfpathmoveto{\pgfqpoint{2.299238in}{2.096717in}}%
\pgfpathlineto{\pgfqpoint{2.342557in}{2.070640in}}%
\pgfpathlineto{\pgfqpoint{2.368989in}{2.092895in}}%
\pgfpathlineto{\pgfqpoint{2.325520in}{2.130300in}}%
\pgfpathlineto{\pgfqpoint{2.299238in}{2.096717in}}%
\pgfpathclose%
\pgfusepath{fill}%
\end{pgfscope}%
\begin{pgfscope}%
\pgfpathrectangle{\pgfqpoint{1.072000in}{0.528000in}}{\pgfqpoint{3.696000in}{3.696000in}}%
\pgfusepath{clip}%
\pgfsetbuttcap%
\pgfsetroundjoin%
\definecolor{currentfill}{rgb}{0.261805,0.346484,0.795658}%
\pgfsetfillcolor{currentfill}%
\pgfsetlinewidth{0.000000pt}%
\definecolor{currentstroke}{rgb}{0.000000,0.000000,0.000000}%
\pgfsetstrokecolor{currentstroke}%
\pgfsetdash{}{0pt}%
\pgfpathmoveto{\pgfqpoint{2.255158in}{2.162277in}}%
\pgfpathlineto{\pgfqpoint{2.299238in}{2.096717in}}%
\pgfpathlineto{\pgfqpoint{2.325520in}{2.130300in}}%
\pgfpathlineto{\pgfqpoint{2.281325in}{2.205572in}}%
\pgfpathlineto{\pgfqpoint{2.255158in}{2.162277in}}%
\pgfpathclose%
\pgfusepath{fill}%
\end{pgfscope}%
\begin{pgfscope}%
\pgfpathrectangle{\pgfqpoint{1.072000in}{0.528000in}}{\pgfqpoint{3.696000in}{3.696000in}}%
\pgfusepath{clip}%
\pgfsetbuttcap%
\pgfsetroundjoin%
\definecolor{currentfill}{rgb}{0.257234,0.339661,0.789661}%
\pgfsetfillcolor{currentfill}%
\pgfsetlinewidth{0.000000pt}%
\definecolor{currentstroke}{rgb}{0.000000,0.000000,0.000000}%
\pgfsetstrokecolor{currentstroke}%
\pgfsetdash{}{0pt}%
\pgfpathmoveto{\pgfqpoint{3.478203in}{2.163557in}}%
\pgfpathlineto{\pgfqpoint{3.522505in}{2.144074in}}%
\pgfpathlineto{\pgfqpoint{3.545847in}{2.114467in}}%
\pgfpathlineto{\pgfqpoint{3.501617in}{2.130179in}}%
\pgfpathlineto{\pgfqpoint{3.478203in}{2.163557in}}%
\pgfpathclose%
\pgfusepath{fill}%
\end{pgfscope}%
\begin{pgfscope}%
\pgfpathrectangle{\pgfqpoint{1.072000in}{0.528000in}}{\pgfqpoint{3.696000in}{3.696000in}}%
\pgfusepath{clip}%
\pgfsetbuttcap%
\pgfsetroundjoin%
\definecolor{currentfill}{rgb}{0.348323,0.465711,0.888346}%
\pgfsetfillcolor{currentfill}%
\pgfsetlinewidth{0.000000pt}%
\definecolor{currentstroke}{rgb}{0.000000,0.000000,0.000000}%
\pgfsetstrokecolor{currentstroke}%
\pgfsetdash{}{0pt}%
\pgfpathmoveto{\pgfqpoint{2.757239in}{2.264872in}}%
\pgfpathlineto{\pgfqpoint{2.800404in}{2.336048in}}%
\pgfpathlineto{\pgfqpoint{2.826383in}{2.274242in}}%
\pgfpathlineto{\pgfqpoint{2.783202in}{2.223941in}}%
\pgfpathlineto{\pgfqpoint{2.757239in}{2.264872in}}%
\pgfpathclose%
\pgfusepath{fill}%
\end{pgfscope}%
\begin{pgfscope}%
\pgfpathrectangle{\pgfqpoint{1.072000in}{0.528000in}}{\pgfqpoint{3.696000in}{3.696000in}}%
\pgfusepath{clip}%
\pgfsetbuttcap%
\pgfsetroundjoin%
\definecolor{currentfill}{rgb}{0.959518,0.766973,0.674145}%
\pgfsetfillcolor{currentfill}%
\pgfsetlinewidth{0.000000pt}%
\definecolor{currentstroke}{rgb}{0.000000,0.000000,0.000000}%
\pgfsetstrokecolor{currentstroke}%
\pgfsetdash{}{0pt}%
\pgfpathmoveto{\pgfqpoint{2.627643in}{3.243918in}}%
\pgfpathlineto{\pgfqpoint{2.672153in}{3.106646in}}%
\pgfpathlineto{\pgfqpoint{2.696915in}{3.252506in}}%
\pgfpathlineto{\pgfqpoint{2.652473in}{3.381351in}}%
\pgfpathlineto{\pgfqpoint{2.627643in}{3.243918in}}%
\pgfpathclose%
\pgfusepath{fill}%
\end{pgfscope}%
\begin{pgfscope}%
\pgfpathrectangle{\pgfqpoint{1.072000in}{0.528000in}}{\pgfqpoint{3.696000in}{3.696000in}}%
\pgfusepath{clip}%
\pgfsetbuttcap%
\pgfsetroundjoin%
\definecolor{currentfill}{rgb}{0.229806,0.298718,0.753683}%
\pgfsetfillcolor{currentfill}%
\pgfsetlinewidth{0.000000pt}%
\definecolor{currentstroke}{rgb}{0.000000,0.000000,0.000000}%
\pgfsetstrokecolor{currentstroke}%
\pgfsetdash{}{0pt}%
\pgfpathmoveto{\pgfqpoint{3.634527in}{2.080392in}}%
\pgfpathlineto{\pgfqpoint{3.679018in}{2.064337in}}%
\pgfpathlineto{\pgfqpoint{3.703205in}{2.099948in}}%
\pgfpathlineto{\pgfqpoint{3.658561in}{2.106181in}}%
\pgfpathlineto{\pgfqpoint{3.634527in}{2.080392in}}%
\pgfpathclose%
\pgfusepath{fill}%
\end{pgfscope}%
\begin{pgfscope}%
\pgfpathrectangle{\pgfqpoint{1.072000in}{0.528000in}}{\pgfqpoint{3.696000in}{3.696000in}}%
\pgfusepath{clip}%
\pgfsetbuttcap%
\pgfsetroundjoin%
\definecolor{currentfill}{rgb}{0.363461,0.484784,0.901019}%
\pgfsetfillcolor{currentfill}%
\pgfsetlinewidth{0.000000pt}%
\definecolor{currentstroke}{rgb}{0.000000,0.000000,0.000000}%
\pgfsetstrokecolor{currentstroke}%
\pgfsetdash{}{0pt}%
\pgfpathmoveto{\pgfqpoint{3.164954in}{2.316306in}}%
\pgfpathlineto{\pgfqpoint{3.209297in}{2.339263in}}%
\pgfpathlineto{\pgfqpoint{3.233318in}{2.251944in}}%
\pgfpathlineto{\pgfqpoint{3.189217in}{2.240065in}}%
\pgfpathlineto{\pgfqpoint{3.164954in}{2.316306in}}%
\pgfpathclose%
\pgfusepath{fill}%
\end{pgfscope}%
\begin{pgfscope}%
\pgfpathrectangle{\pgfqpoint{1.072000in}{0.528000in}}{\pgfqpoint{3.696000in}{3.696000in}}%
\pgfusepath{clip}%
\pgfsetbuttcap%
\pgfsetroundjoin%
\definecolor{currentfill}{rgb}{0.378598,0.503856,0.913692}%
\pgfsetfillcolor{currentfill}%
\pgfsetlinewidth{0.000000pt}%
\definecolor{currentstroke}{rgb}{0.000000,0.000000,0.000000}%
\pgfsetstrokecolor{currentstroke}%
\pgfsetdash{}{0pt}%
\pgfpathmoveto{\pgfqpoint{2.939028in}{2.323147in}}%
\pgfpathlineto{\pgfqpoint{2.982876in}{2.380590in}}%
\pgfpathlineto{\pgfqpoint{3.008057in}{2.296619in}}%
\pgfpathlineto{\pgfqpoint{2.964345in}{2.257494in}}%
\pgfpathlineto{\pgfqpoint{2.939028in}{2.323147in}}%
\pgfpathclose%
\pgfusepath{fill}%
\end{pgfscope}%
\begin{pgfscope}%
\pgfpathrectangle{\pgfqpoint{1.072000in}{0.528000in}}{\pgfqpoint{3.696000in}{3.696000in}}%
\pgfusepath{clip}%
\pgfsetbuttcap%
\pgfsetroundjoin%
\definecolor{currentfill}{rgb}{0.323718,0.433158,0.864722}%
\pgfsetfillcolor{currentfill}%
\pgfsetlinewidth{0.000000pt}%
\definecolor{currentstroke}{rgb}{0.000000,0.000000,0.000000}%
\pgfsetstrokecolor{currentstroke}%
\pgfsetdash{}{0pt}%
\pgfpathmoveto{\pgfqpoint{3.321882in}{2.264160in}}%
\pgfpathlineto{\pgfqpoint{3.366284in}{2.261678in}}%
\pgfpathlineto{\pgfqpoint{3.389679in}{2.189364in}}%
\pgfpathlineto{\pgfqpoint{3.345492in}{2.195708in}}%
\pgfpathlineto{\pgfqpoint{3.321882in}{2.264160in}}%
\pgfpathclose%
\pgfusepath{fill}%
\end{pgfscope}%
\begin{pgfscope}%
\pgfpathrectangle{\pgfqpoint{1.072000in}{0.528000in}}{\pgfqpoint{3.696000in}{3.696000in}}%
\pgfusepath{clip}%
\pgfsetbuttcap%
\pgfsetroundjoin%
\definecolor{currentfill}{rgb}{0.343278,0.459354,0.884122}%
\pgfsetfillcolor{currentfill}%
\pgfsetlinewidth{0.000000pt}%
\definecolor{currentstroke}{rgb}{0.000000,0.000000,0.000000}%
\pgfsetstrokecolor{currentstroke}%
\pgfsetdash{}{0pt}%
\pgfpathmoveto{\pgfqpoint{2.688086in}{2.243023in}}%
\pgfpathlineto{\pgfqpoint{2.730994in}{2.318540in}}%
\pgfpathlineto{\pgfqpoint{2.757239in}{2.264872in}}%
\pgfpathlineto{\pgfqpoint{2.714260in}{2.210028in}}%
\pgfpathlineto{\pgfqpoint{2.688086in}{2.243023in}}%
\pgfpathclose%
\pgfusepath{fill}%
\end{pgfscope}%
\begin{pgfscope}%
\pgfpathrectangle{\pgfqpoint{1.072000in}{0.528000in}}{\pgfqpoint{3.696000in}{3.696000in}}%
\pgfusepath{clip}%
\pgfsetbuttcap%
\pgfsetroundjoin%
\definecolor{currentfill}{rgb}{0.248091,0.326013,0.777669}%
\pgfsetfillcolor{currentfill}%
\pgfsetlinewidth{0.000000pt}%
\definecolor{currentstroke}{rgb}{0.000000,0.000000,0.000000}%
\pgfsetstrokecolor{currentstroke}%
\pgfsetdash{}{0pt}%
\pgfpathmoveto{\pgfqpoint{2.411863in}{2.094083in}}%
\pgfpathlineto{\pgfqpoint{2.454305in}{2.132554in}}%
\pgfpathlineto{\pgfqpoint{2.480935in}{2.130083in}}%
\pgfpathlineto{\pgfqpoint{2.438301in}{2.106708in}}%
\pgfpathlineto{\pgfqpoint{2.411863in}{2.094083in}}%
\pgfpathclose%
\pgfusepath{fill}%
\end{pgfscope}%
\begin{pgfscope}%
\pgfpathrectangle{\pgfqpoint{1.072000in}{0.528000in}}{\pgfqpoint{3.696000in}{3.696000in}}%
\pgfusepath{clip}%
\pgfsetbuttcap%
\pgfsetroundjoin%
\definecolor{currentfill}{rgb}{0.252663,0.332837,0.783665}%
\pgfsetfillcolor{currentfill}%
\pgfsetlinewidth{0.000000pt}%
\definecolor{currentstroke}{rgb}{0.000000,0.000000,0.000000}%
\pgfsetstrokecolor{currentstroke}%
\pgfsetdash{}{0pt}%
\pgfpathmoveto{\pgfqpoint{3.522505in}{2.144074in}}%
\pgfpathlineto{\pgfqpoint{3.566824in}{2.120940in}}%
\pgfpathlineto{\pgfqpoint{3.590145in}{2.097544in}}%
\pgfpathlineto{\pgfqpoint{3.545847in}{2.114467in}}%
\pgfpathlineto{\pgfqpoint{3.522505in}{2.144074in}}%
\pgfpathclose%
\pgfusepath{fill}%
\end{pgfscope}%
\begin{pgfscope}%
\pgfpathrectangle{\pgfqpoint{1.072000in}{0.528000in}}{\pgfqpoint{3.696000in}{3.696000in}}%
\pgfusepath{clip}%
\pgfsetbuttcap%
\pgfsetroundjoin%
\definecolor{currentfill}{rgb}{0.275827,0.366717,0.812553}%
\pgfsetfillcolor{currentfill}%
\pgfsetlinewidth{0.000000pt}%
\definecolor{currentstroke}{rgb}{0.000000,0.000000,0.000000}%
\pgfsetstrokecolor{currentstroke}%
\pgfsetdash{}{0pt}%
\pgfpathmoveto{\pgfqpoint{2.480935in}{2.130083in}}%
\pgfpathlineto{\pgfqpoint{2.523316in}{2.187109in}}%
\pgfpathlineto{\pgfqpoint{2.549941in}{2.171128in}}%
\pgfpathlineto{\pgfqpoint{2.507374in}{2.131339in}}%
\pgfpathlineto{\pgfqpoint{2.480935in}{2.130083in}}%
\pgfpathclose%
\pgfusepath{fill}%
\end{pgfscope}%
\begin{pgfscope}%
\pgfpathrectangle{\pgfqpoint{1.072000in}{0.528000in}}{\pgfqpoint{3.696000in}{3.696000in}}%
\pgfusepath{clip}%
\pgfsetbuttcap%
\pgfsetroundjoin%
\definecolor{currentfill}{rgb}{0.328604,0.439712,0.869587}%
\pgfsetfillcolor{currentfill}%
\pgfsetlinewidth{0.000000pt}%
\definecolor{currentstroke}{rgb}{0.000000,0.000000,0.000000}%
\pgfsetstrokecolor{currentstroke}%
\pgfsetdash{}{0pt}%
\pgfpathmoveto{\pgfqpoint{2.618980in}{2.210551in}}%
\pgfpathlineto{\pgfqpoint{2.661643in}{2.285667in}}%
\pgfpathlineto{\pgfqpoint{2.688086in}{2.243023in}}%
\pgfpathlineto{\pgfqpoint{2.645302in}{2.187927in}}%
\pgfpathlineto{\pgfqpoint{2.618980in}{2.210551in}}%
\pgfpathclose%
\pgfusepath{fill}%
\end{pgfscope}%
\begin{pgfscope}%
\pgfpathrectangle{\pgfqpoint{1.072000in}{0.528000in}}{\pgfqpoint{3.696000in}{3.696000in}}%
\pgfusepath{clip}%
\pgfsetbuttcap%
\pgfsetroundjoin%
\definecolor{currentfill}{rgb}{0.304174,0.406945,0.845263}%
\pgfsetfillcolor{currentfill}%
\pgfsetlinewidth{0.000000pt}%
\definecolor{currentstroke}{rgb}{0.000000,0.000000,0.000000}%
\pgfsetstrokecolor{currentstroke}%
\pgfsetdash{}{0pt}%
\pgfpathmoveto{\pgfqpoint{2.549941in}{2.171128in}}%
\pgfpathlineto{\pgfqpoint{2.592412in}{2.240284in}}%
\pgfpathlineto{\pgfqpoint{2.618980in}{2.210551in}}%
\pgfpathlineto{\pgfqpoint{2.576348in}{2.160278in}}%
\pgfpathlineto{\pgfqpoint{2.549941in}{2.171128in}}%
\pgfpathclose%
\pgfusepath{fill}%
\end{pgfscope}%
\begin{pgfscope}%
\pgfpathrectangle{\pgfqpoint{1.072000in}{0.528000in}}{\pgfqpoint{3.696000in}{3.696000in}}%
\pgfusepath{clip}%
\pgfsetbuttcap%
\pgfsetroundjoin%
\definecolor{currentfill}{rgb}{0.229806,0.298718,0.753683}%
\pgfsetfillcolor{currentfill}%
\pgfsetlinewidth{0.000000pt}%
\definecolor{currentstroke}{rgb}{0.000000,0.000000,0.000000}%
\pgfsetstrokecolor{currentstroke}%
\pgfsetdash{}{0pt}%
\pgfpathmoveto{\pgfqpoint{2.342557in}{2.070640in}}%
\pgfpathlineto{\pgfqpoint{2.385257in}{2.084369in}}%
\pgfpathlineto{\pgfqpoint{2.411863in}{2.094083in}}%
\pgfpathlineto{\pgfqpoint{2.368989in}{2.092895in}}%
\pgfpathlineto{\pgfqpoint{2.342557in}{2.070640in}}%
\pgfpathclose%
\pgfusepath{fill}%
\end{pgfscope}%
\begin{pgfscope}%
\pgfpathrectangle{\pgfqpoint{1.072000in}{0.528000in}}{\pgfqpoint{3.696000in}{3.696000in}}%
\pgfusepath{clip}%
\pgfsetbuttcap%
\pgfsetroundjoin%
\definecolor{currentfill}{rgb}{0.388852,0.516298,0.921373}%
\pgfsetfillcolor{currentfill}%
\pgfsetlinewidth{0.000000pt}%
\definecolor{currentstroke}{rgb}{0.000000,0.000000,0.000000}%
\pgfsetstrokecolor{currentstroke}%
\pgfsetdash{}{0pt}%
\pgfpathmoveto{\pgfqpoint{3.052004in}{2.339167in}}%
\pgfpathlineto{\pgfqpoint{3.096189in}{2.381669in}}%
\pgfpathlineto{\pgfqpoint{3.120784in}{2.289829in}}%
\pgfpathlineto{\pgfqpoint{3.076808in}{2.262407in}}%
\pgfpathlineto{\pgfqpoint{3.052004in}{2.339167in}}%
\pgfpathclose%
\pgfusepath{fill}%
\end{pgfscope}%
\begin{pgfscope}%
\pgfpathrectangle{\pgfqpoint{1.072000in}{0.528000in}}{\pgfqpoint{3.696000in}{3.696000in}}%
\pgfusepath{clip}%
\pgfsetbuttcap%
\pgfsetroundjoin%
\definecolor{currentfill}{rgb}{0.960581,0.762501,0.667964}%
\pgfsetfillcolor{currentfill}%
\pgfsetlinewidth{0.000000pt}%
\definecolor{currentstroke}{rgb}{0.000000,0.000000,0.000000}%
\pgfsetstrokecolor{currentstroke}%
\pgfsetdash{}{0pt}%
\pgfpathmoveto{\pgfqpoint{2.558106in}{3.248452in}}%
\pgfpathlineto{\pgfqpoint{2.602903in}{3.104895in}}%
\pgfpathlineto{\pgfqpoint{2.627643in}{3.243918in}}%
\pgfpathlineto{\pgfqpoint{2.582945in}{3.378891in}}%
\pgfpathlineto{\pgfqpoint{2.558106in}{3.248452in}}%
\pgfpathclose%
\pgfusepath{fill}%
\end{pgfscope}%
\begin{pgfscope}%
\pgfpathrectangle{\pgfqpoint{1.072000in}{0.528000in}}{\pgfqpoint{3.696000in}{3.696000in}}%
\pgfusepath{clip}%
\pgfsetbuttcap%
\pgfsetroundjoin%
\definecolor{currentfill}{rgb}{0.252663,0.332837,0.783665}%
\pgfsetfillcolor{currentfill}%
\pgfsetlinewidth{0.000000pt}%
\definecolor{currentstroke}{rgb}{0.000000,0.000000,0.000000}%
\pgfsetstrokecolor{currentstroke}%
\pgfsetdash{}{0pt}%
\pgfpathmoveto{\pgfqpoint{2.228801in}{2.124553in}}%
\pgfpathlineto{\pgfqpoint{2.272784in}{2.067411in}}%
\pgfpathlineto{\pgfqpoint{2.299238in}{2.096717in}}%
\pgfpathlineto{\pgfqpoint{2.255158in}{2.162277in}}%
\pgfpathlineto{\pgfqpoint{2.228801in}{2.124553in}}%
\pgfpathclose%
\pgfusepath{fill}%
\end{pgfscope}%
\begin{pgfscope}%
\pgfpathrectangle{\pgfqpoint{1.072000in}{0.528000in}}{\pgfqpoint{3.696000in}{3.696000in}}%
\pgfusepath{clip}%
\pgfsetbuttcap%
\pgfsetroundjoin%
\definecolor{currentfill}{rgb}{0.229806,0.298718,0.753683}%
\pgfsetfillcolor{currentfill}%
\pgfsetlinewidth{0.000000pt}%
\definecolor{currentstroke}{rgb}{0.000000,0.000000,0.000000}%
\pgfsetstrokecolor{currentstroke}%
\pgfsetdash{}{0pt}%
\pgfpathmoveto{\pgfqpoint{3.679018in}{2.064337in}}%
\pgfpathlineto{\pgfqpoint{3.723660in}{2.050999in}}%
\pgfpathlineto{\pgfqpoint{3.748065in}{2.098083in}}%
\pgfpathlineto{\pgfqpoint{3.703205in}{2.099948in}}%
\pgfpathlineto{\pgfqpoint{3.679018in}{2.064337in}}%
\pgfpathclose%
\pgfusepath{fill}%
\end{pgfscope}%
\begin{pgfscope}%
\pgfpathrectangle{\pgfqpoint{1.072000in}{0.528000in}}{\pgfqpoint{3.696000in}{3.696000in}}%
\pgfusepath{clip}%
\pgfsetbuttcap%
\pgfsetroundjoin%
\definecolor{currentfill}{rgb}{0.229806,0.298718,0.753683}%
\pgfsetfillcolor{currentfill}%
\pgfsetlinewidth{0.000000pt}%
\definecolor{currentstroke}{rgb}{0.000000,0.000000,0.000000}%
\pgfsetstrokecolor{currentstroke}%
\pgfsetdash{}{0pt}%
\pgfpathmoveto{\pgfqpoint{2.272784in}{2.067411in}}%
\pgfpathlineto{\pgfqpoint{2.315970in}{2.051051in}}%
\pgfpathlineto{\pgfqpoint{2.342557in}{2.070640in}}%
\pgfpathlineto{\pgfqpoint{2.299238in}{2.096717in}}%
\pgfpathlineto{\pgfqpoint{2.272784in}{2.067411in}}%
\pgfpathclose%
\pgfusepath{fill}%
\end{pgfscope}%
\begin{pgfscope}%
\pgfpathrectangle{\pgfqpoint{1.072000in}{0.528000in}}{\pgfqpoint{3.696000in}{3.696000in}}%
\pgfusepath{clip}%
\pgfsetbuttcap%
\pgfsetroundjoin%
\definecolor{currentfill}{rgb}{0.243520,0.319189,0.771672}%
\pgfsetfillcolor{currentfill}%
\pgfsetlinewidth{0.000000pt}%
\definecolor{currentstroke}{rgb}{0.000000,0.000000,0.000000}%
\pgfsetstrokecolor{currentstroke}%
\pgfsetdash{}{0pt}%
\pgfpathmoveto{\pgfqpoint{3.566824in}{2.120940in}}%
\pgfpathlineto{\pgfqpoint{3.611168in}{2.095248in}}%
\pgfpathlineto{\pgfqpoint{3.634527in}{2.080392in}}%
\pgfpathlineto{\pgfqpoint{3.590145in}{2.097544in}}%
\pgfpathlineto{\pgfqpoint{3.566824in}{2.120940in}}%
\pgfpathclose%
\pgfusepath{fill}%
\end{pgfscope}%
\begin{pgfscope}%
\pgfpathrectangle{\pgfqpoint{1.072000in}{0.528000in}}{\pgfqpoint{3.696000in}{3.696000in}}%
\pgfusepath{clip}%
\pgfsetbuttcap%
\pgfsetroundjoin%
\definecolor{currentfill}{rgb}{0.328604,0.439712,0.869587}%
\pgfsetfillcolor{currentfill}%
\pgfsetlinewidth{0.000000pt}%
\definecolor{currentstroke}{rgb}{0.000000,0.000000,0.000000}%
\pgfsetstrokecolor{currentstroke}%
\pgfsetdash{}{0pt}%
\pgfpathmoveto{\pgfqpoint{3.366284in}{2.261678in}}%
\pgfpathlineto{\pgfqpoint{3.410723in}{2.252346in}}%
\pgfpathlineto{\pgfqpoint{3.433923in}{2.178726in}}%
\pgfpathlineto{\pgfqpoint{3.389679in}{2.189364in}}%
\pgfpathlineto{\pgfqpoint{3.366284in}{2.261678in}}%
\pgfpathclose%
\pgfusepath{fill}%
\end{pgfscope}%
\begin{pgfscope}%
\pgfpathrectangle{\pgfqpoint{1.072000in}{0.528000in}}{\pgfqpoint{3.696000in}{3.696000in}}%
\pgfusepath{clip}%
\pgfsetbuttcap%
\pgfsetroundjoin%
\definecolor{currentfill}{rgb}{0.409611,0.540759,0.935545}%
\pgfsetfillcolor{currentfill}%
\pgfsetlinewidth{0.000000pt}%
\definecolor{currentstroke}{rgb}{0.000000,0.000000,0.000000}%
\pgfsetstrokecolor{currentstroke}%
\pgfsetdash{}{0pt}%
\pgfpathmoveto{\pgfqpoint{2.869781in}{2.337255in}}%
\pgfpathlineto{\pgfqpoint{2.913454in}{2.408326in}}%
\pgfpathlineto{\pgfqpoint{2.939028in}{2.323147in}}%
\pgfpathlineto{\pgfqpoint{2.895443in}{2.271152in}}%
\pgfpathlineto{\pgfqpoint{2.869781in}{2.337255in}}%
\pgfpathclose%
\pgfusepath{fill}%
\end{pgfscope}%
\begin{pgfscope}%
\pgfpathrectangle{\pgfqpoint{1.072000in}{0.528000in}}{\pgfqpoint{3.696000in}{3.696000in}}%
\pgfusepath{clip}%
\pgfsetbuttcap%
\pgfsetroundjoin%
\definecolor{currentfill}{rgb}{0.388852,0.516298,0.921373}%
\pgfsetfillcolor{currentfill}%
\pgfsetlinewidth{0.000000pt}%
\definecolor{currentstroke}{rgb}{0.000000,0.000000,0.000000}%
\pgfsetstrokecolor{currentstroke}%
\pgfsetdash{}{0pt}%
\pgfpathmoveto{\pgfqpoint{3.209297in}{2.339263in}}%
\pgfpathlineto{\pgfqpoint{3.253782in}{2.356541in}}%
\pgfpathlineto{\pgfqpoint{3.277549in}{2.260519in}}%
\pgfpathlineto{\pgfqpoint{3.233318in}{2.251944in}}%
\pgfpathlineto{\pgfqpoint{3.209297in}{2.339263in}}%
\pgfpathclose%
\pgfusepath{fill}%
\end{pgfscope}%
\begin{pgfscope}%
\pgfpathrectangle{\pgfqpoint{1.072000in}{0.528000in}}{\pgfqpoint{3.696000in}{3.696000in}}%
\pgfusepath{clip}%
\pgfsetbuttcap%
\pgfsetroundjoin%
\definecolor{currentfill}{rgb}{0.967317,0.657471,0.538160}%
\pgfsetfillcolor{currentfill}%
\pgfsetlinewidth{0.000000pt}%
\definecolor{currentstroke}{rgb}{0.000000,0.000000,0.000000}%
\pgfsetstrokecolor{currentstroke}%
\pgfsetdash{}{0pt}%
\pgfpathmoveto{\pgfqpoint{2.582945in}{3.378891in}}%
\pgfpathlineto{\pgfqpoint{2.627643in}{3.243918in}}%
\pgfpathlineto{\pgfqpoint{2.652473in}{3.381351in}}%
\pgfpathlineto{\pgfqpoint{2.607899in}{3.504306in}}%
\pgfpathlineto{\pgfqpoint{2.582945in}{3.378891in}}%
\pgfpathclose%
\pgfusepath{fill}%
\end{pgfscope}%
\begin{pgfscope}%
\pgfpathrectangle{\pgfqpoint{1.072000in}{0.528000in}}{\pgfqpoint{3.696000in}{3.696000in}}%
\pgfusepath{clip}%
\pgfsetbuttcap%
\pgfsetroundjoin%
\definecolor{currentfill}{rgb}{0.238948,0.312365,0.765676}%
\pgfsetfillcolor{currentfill}%
\pgfsetlinewidth{0.000000pt}%
\definecolor{currentstroke}{rgb}{0.000000,0.000000,0.000000}%
\pgfsetstrokecolor{currentstroke}%
\pgfsetdash{}{0pt}%
\pgfpathmoveto{\pgfqpoint{3.611168in}{2.095248in}}%
\pgfpathlineto{\pgfqpoint{3.655556in}{2.068497in}}%
\pgfpathlineto{\pgfqpoint{3.679018in}{2.064337in}}%
\pgfpathlineto{\pgfqpoint{3.634527in}{2.080392in}}%
\pgfpathlineto{\pgfqpoint{3.611168in}{2.095248in}}%
\pgfpathclose%
\pgfusepath{fill}%
\end{pgfscope}%
\begin{pgfscope}%
\pgfpathrectangle{\pgfqpoint{1.072000in}{0.528000in}}{\pgfqpoint{3.696000in}{3.696000in}}%
\pgfusepath{clip}%
\pgfsetbuttcap%
\pgfsetroundjoin%
\definecolor{currentfill}{rgb}{0.238948,0.312365,0.765676}%
\pgfsetfillcolor{currentfill}%
\pgfsetlinewidth{0.000000pt}%
\definecolor{currentstroke}{rgb}{0.000000,0.000000,0.000000}%
\pgfsetstrokecolor{currentstroke}%
\pgfsetdash{}{0pt}%
\pgfpathmoveto{\pgfqpoint{3.723660in}{2.050999in}}%
\pgfpathlineto{\pgfqpoint{3.768506in}{2.042219in}}%
\pgfpathlineto{\pgfqpoint{3.793194in}{2.102099in}}%
\pgfpathlineto{\pgfqpoint{3.748065in}{2.098083in}}%
\pgfpathlineto{\pgfqpoint{3.723660in}{2.050999in}}%
\pgfpathclose%
\pgfusepath{fill}%
\end{pgfscope}%
\begin{pgfscope}%
\pgfpathrectangle{\pgfqpoint{1.072000in}{0.528000in}}{\pgfqpoint{3.696000in}{3.696000in}}%
\pgfusepath{clip}%
\pgfsetbuttcap%
\pgfsetroundjoin%
\definecolor{currentfill}{rgb}{0.328604,0.439712,0.869587}%
\pgfsetfillcolor{currentfill}%
\pgfsetlinewidth{0.000000pt}%
\definecolor{currentstroke}{rgb}{0.000000,0.000000,0.000000}%
\pgfsetstrokecolor{currentstroke}%
\pgfsetdash{}{0pt}%
\pgfpathmoveto{\pgfqpoint{3.410723in}{2.252346in}}%
\pgfpathlineto{\pgfqpoint{3.455169in}{2.235909in}}%
\pgfpathlineto{\pgfqpoint{3.478203in}{2.163557in}}%
\pgfpathlineto{\pgfqpoint{3.433923in}{2.178726in}}%
\pgfpathlineto{\pgfqpoint{3.410723in}{2.252346in}}%
\pgfpathclose%
\pgfusepath{fill}%
\end{pgfscope}%
\begin{pgfscope}%
\pgfpathrectangle{\pgfqpoint{1.072000in}{0.528000in}}{\pgfqpoint{3.696000in}{3.696000in}}%
\pgfusepath{clip}%
\pgfsetbuttcap%
\pgfsetroundjoin%
\definecolor{currentfill}{rgb}{0.243520,0.319189,0.771672}%
\pgfsetfillcolor{currentfill}%
\pgfsetlinewidth{0.000000pt}%
\definecolor{currentstroke}{rgb}{0.000000,0.000000,0.000000}%
\pgfsetstrokecolor{currentstroke}%
\pgfsetdash{}{0pt}%
\pgfpathmoveto{\pgfqpoint{2.202264in}{2.091354in}}%
\pgfpathlineto{\pgfqpoint{2.246175in}{2.041044in}}%
\pgfpathlineto{\pgfqpoint{2.272784in}{2.067411in}}%
\pgfpathlineto{\pgfqpoint{2.228801in}{2.124553in}}%
\pgfpathlineto{\pgfqpoint{2.202264in}{2.091354in}}%
\pgfpathclose%
\pgfusepath{fill}%
\end{pgfscope}%
\begin{pgfscope}%
\pgfpathrectangle{\pgfqpoint{1.072000in}{0.528000in}}{\pgfqpoint{3.696000in}{3.696000in}}%
\pgfusepath{clip}%
\pgfsetbuttcap%
\pgfsetroundjoin%
\definecolor{currentfill}{rgb}{0.238948,0.312365,0.765676}%
\pgfsetfillcolor{currentfill}%
\pgfsetlinewidth{0.000000pt}%
\definecolor{currentstroke}{rgb}{0.000000,0.000000,0.000000}%
\pgfsetstrokecolor{currentstroke}%
\pgfsetdash{}{0pt}%
\pgfpathmoveto{\pgfqpoint{2.315970in}{2.051051in}}%
\pgfpathlineto{\pgfqpoint{2.358516in}{2.075440in}}%
\pgfpathlineto{\pgfqpoint{2.385257in}{2.084369in}}%
\pgfpathlineto{\pgfqpoint{2.342557in}{2.070640in}}%
\pgfpathlineto{\pgfqpoint{2.315970in}{2.051051in}}%
\pgfpathclose%
\pgfusepath{fill}%
\end{pgfscope}%
\begin{pgfscope}%
\pgfpathrectangle{\pgfqpoint{1.072000in}{0.528000in}}{\pgfqpoint{3.696000in}{3.696000in}}%
\pgfusepath{clip}%
\pgfsetbuttcap%
\pgfsetroundjoin%
\definecolor{currentfill}{rgb}{0.266381,0.353304,0.801637}%
\pgfsetfillcolor{currentfill}%
\pgfsetlinewidth{0.000000pt}%
\definecolor{currentstroke}{rgb}{0.000000,0.000000,0.000000}%
\pgfsetstrokecolor{currentstroke}%
\pgfsetdash{}{0pt}%
\pgfpathmoveto{\pgfqpoint{2.385257in}{2.084369in}}%
\pgfpathlineto{\pgfqpoint{2.427519in}{2.136108in}}%
\pgfpathlineto{\pgfqpoint{2.454305in}{2.132554in}}%
\pgfpathlineto{\pgfqpoint{2.411863in}{2.094083in}}%
\pgfpathlineto{\pgfqpoint{2.385257in}{2.084369in}}%
\pgfpathclose%
\pgfusepath{fill}%
\end{pgfscope}%
\begin{pgfscope}%
\pgfpathrectangle{\pgfqpoint{1.072000in}{0.528000in}}{\pgfqpoint{3.696000in}{3.696000in}}%
\pgfusepath{clip}%
\pgfsetbuttcap%
\pgfsetroundjoin%
\definecolor{currentfill}{rgb}{0.441123,0.576532,0.954545}%
\pgfsetfillcolor{currentfill}%
\pgfsetlinewidth{0.000000pt}%
\definecolor{currentstroke}{rgb}{0.000000,0.000000,0.000000}%
\pgfsetstrokecolor{currentstroke}%
\pgfsetdash{}{0pt}%
\pgfpathmoveto{\pgfqpoint{2.982876in}{2.380590in}}%
\pgfpathlineto{\pgfqpoint{3.027000in}{2.439443in}}%
\pgfpathlineto{\pgfqpoint{3.052004in}{2.339167in}}%
\pgfpathlineto{\pgfqpoint{3.008057in}{2.296619in}}%
\pgfpathlineto{\pgfqpoint{2.982876in}{2.380590in}}%
\pgfpathclose%
\pgfusepath{fill}%
\end{pgfscope}%
\begin{pgfscope}%
\pgfpathrectangle{\pgfqpoint{1.072000in}{0.528000in}}{\pgfqpoint{3.696000in}{3.696000in}}%
\pgfusepath{clip}%
\pgfsetbuttcap%
\pgfsetroundjoin%
\definecolor{currentfill}{rgb}{0.229806,0.298718,0.753683}%
\pgfsetfillcolor{currentfill}%
\pgfsetlinewidth{0.000000pt}%
\definecolor{currentstroke}{rgb}{0.000000,0.000000,0.000000}%
\pgfsetstrokecolor{currentstroke}%
\pgfsetdash{}{0pt}%
\pgfpathmoveto{\pgfqpoint{3.655556in}{2.068497in}}%
\pgfpathlineto{\pgfqpoint{3.700024in}{2.042536in}}%
\pgfpathlineto{\pgfqpoint{3.723660in}{2.050999in}}%
\pgfpathlineto{\pgfqpoint{3.679018in}{2.064337in}}%
\pgfpathlineto{\pgfqpoint{3.655556in}{2.068497in}}%
\pgfpathclose%
\pgfusepath{fill}%
\end{pgfscope}%
\begin{pgfscope}%
\pgfpathrectangle{\pgfqpoint{1.072000in}{0.528000in}}{\pgfqpoint{3.696000in}{3.696000in}}%
\pgfusepath{clip}%
\pgfsetbuttcap%
\pgfsetroundjoin%
\definecolor{currentfill}{rgb}{0.425199,0.559058,0.946061}%
\pgfsetfillcolor{currentfill}%
\pgfsetlinewidth{0.000000pt}%
\definecolor{currentstroke}{rgb}{0.000000,0.000000,0.000000}%
\pgfsetstrokecolor{currentstroke}%
\pgfsetdash{}{0pt}%
\pgfpathmoveto{\pgfqpoint{2.800404in}{2.336048in}}%
\pgfpathlineto{\pgfqpoint{2.843836in}{2.418500in}}%
\pgfpathlineto{\pgfqpoint{2.869781in}{2.337255in}}%
\pgfpathlineto{\pgfqpoint{2.826383in}{2.274242in}}%
\pgfpathlineto{\pgfqpoint{2.800404in}{2.336048in}}%
\pgfpathclose%
\pgfusepath{fill}%
\end{pgfscope}%
\begin{pgfscope}%
\pgfpathrectangle{\pgfqpoint{1.072000in}{0.528000in}}{\pgfqpoint{3.696000in}{3.696000in}}%
\pgfusepath{clip}%
\pgfsetbuttcap%
\pgfsetroundjoin%
\definecolor{currentfill}{rgb}{0.430507,0.564883,0.948889}%
\pgfsetfillcolor{currentfill}%
\pgfsetlinewidth{0.000000pt}%
\definecolor{currentstroke}{rgb}{0.000000,0.000000,0.000000}%
\pgfsetstrokecolor{currentstroke}%
\pgfsetdash{}{0pt}%
\pgfpathmoveto{\pgfqpoint{3.096189in}{2.381669in}}%
\pgfpathlineto{\pgfqpoint{3.140596in}{2.420997in}}%
\pgfpathlineto{\pgfqpoint{3.164954in}{2.316306in}}%
\pgfpathlineto{\pgfqpoint{3.120784in}{2.289829in}}%
\pgfpathlineto{\pgfqpoint{3.096189in}{2.381669in}}%
\pgfpathclose%
\pgfusepath{fill}%
\end{pgfscope}%
\begin{pgfscope}%
\pgfpathrectangle{\pgfqpoint{1.072000in}{0.528000in}}{\pgfqpoint{3.696000in}{3.696000in}}%
\pgfusepath{clip}%
\pgfsetbuttcap%
\pgfsetroundjoin%
\definecolor{currentfill}{rgb}{0.229806,0.298718,0.753683}%
\pgfsetfillcolor{currentfill}%
\pgfsetlinewidth{0.000000pt}%
\definecolor{currentstroke}{rgb}{0.000000,0.000000,0.000000}%
\pgfsetstrokecolor{currentstroke}%
\pgfsetdash{}{0pt}%
\pgfpathmoveto{\pgfqpoint{2.246175in}{2.041044in}}%
\pgfpathlineto{\pgfqpoint{2.289255in}{2.032517in}}%
\pgfpathlineto{\pgfqpoint{2.315970in}{2.051051in}}%
\pgfpathlineto{\pgfqpoint{2.272784in}{2.067411in}}%
\pgfpathlineto{\pgfqpoint{2.246175in}{2.041044in}}%
\pgfpathclose%
\pgfusepath{fill}%
\end{pgfscope}%
\begin{pgfscope}%
\pgfpathrectangle{\pgfqpoint{1.072000in}{0.528000in}}{\pgfqpoint{3.696000in}{3.696000in}}%
\pgfusepath{clip}%
\pgfsetbuttcap%
\pgfsetroundjoin%
\definecolor{currentfill}{rgb}{0.304174,0.406945,0.845263}%
\pgfsetfillcolor{currentfill}%
\pgfsetlinewidth{0.000000pt}%
\definecolor{currentstroke}{rgb}{0.000000,0.000000,0.000000}%
\pgfsetstrokecolor{currentstroke}%
\pgfsetdash{}{0pt}%
\pgfpathmoveto{\pgfqpoint{2.454305in}{2.132554in}}%
\pgfpathlineto{\pgfqpoint{2.496504in}{2.205056in}}%
\pgfpathlineto{\pgfqpoint{2.523316in}{2.187109in}}%
\pgfpathlineto{\pgfqpoint{2.480935in}{2.130083in}}%
\pgfpathlineto{\pgfqpoint{2.454305in}{2.132554in}}%
\pgfpathclose%
\pgfusepath{fill}%
\end{pgfscope}%
\begin{pgfscope}%
\pgfpathrectangle{\pgfqpoint{1.072000in}{0.528000in}}{\pgfqpoint{3.696000in}{3.696000in}}%
\pgfusepath{clip}%
\pgfsetbuttcap%
\pgfsetroundjoin%
\definecolor{currentfill}{rgb}{0.323718,0.433158,0.864722}%
\pgfsetfillcolor{currentfill}%
\pgfsetlinewidth{0.000000pt}%
\definecolor{currentstroke}{rgb}{0.000000,0.000000,0.000000}%
\pgfsetstrokecolor{currentstroke}%
\pgfsetdash{}{0pt}%
\pgfpathmoveto{\pgfqpoint{3.455169in}{2.235909in}}%
\pgfpathlineto{\pgfqpoint{3.499599in}{2.212592in}}%
\pgfpathlineto{\pgfqpoint{3.522505in}{2.144074in}}%
\pgfpathlineto{\pgfqpoint{3.478203in}{2.163557in}}%
\pgfpathlineto{\pgfqpoint{3.455169in}{2.235909in}}%
\pgfpathclose%
\pgfusepath{fill}%
\end{pgfscope}%
\begin{pgfscope}%
\pgfpathrectangle{\pgfqpoint{1.072000in}{0.528000in}}{\pgfqpoint{3.696000in}{3.696000in}}%
\pgfusepath{clip}%
\pgfsetbuttcap%
\pgfsetroundjoin%
\definecolor{currentfill}{rgb}{0.409611,0.540759,0.935545}%
\pgfsetfillcolor{currentfill}%
\pgfsetlinewidth{0.000000pt}%
\definecolor{currentstroke}{rgb}{0.000000,0.000000,0.000000}%
\pgfsetstrokecolor{currentstroke}%
\pgfsetdash{}{0pt}%
\pgfpathmoveto{\pgfqpoint{3.253782in}{2.356541in}}%
\pgfpathlineto{\pgfqpoint{3.298372in}{2.366442in}}%
\pgfpathlineto{\pgfqpoint{3.321882in}{2.264160in}}%
\pgfpathlineto{\pgfqpoint{3.277549in}{2.260519in}}%
\pgfpathlineto{\pgfqpoint{3.253782in}{2.356541in}}%
\pgfpathclose%
\pgfusepath{fill}%
\end{pgfscope}%
\begin{pgfscope}%
\pgfpathrectangle{\pgfqpoint{1.072000in}{0.528000in}}{\pgfqpoint{3.696000in}{3.696000in}}%
\pgfusepath{clip}%
\pgfsetbuttcap%
\pgfsetroundjoin%
\definecolor{currentfill}{rgb}{0.343278,0.459354,0.884122}%
\pgfsetfillcolor{currentfill}%
\pgfsetlinewidth{0.000000pt}%
\definecolor{currentstroke}{rgb}{0.000000,0.000000,0.000000}%
\pgfsetstrokecolor{currentstroke}%
\pgfsetdash{}{0pt}%
\pgfpathmoveto{\pgfqpoint{2.523316in}{2.187109in}}%
\pgfpathlineto{\pgfqpoint{2.565624in}{2.273452in}}%
\pgfpathlineto{\pgfqpoint{2.592412in}{2.240284in}}%
\pgfpathlineto{\pgfqpoint{2.549941in}{2.171128in}}%
\pgfpathlineto{\pgfqpoint{2.523316in}{2.187109in}}%
\pgfpathclose%
\pgfusepath{fill}%
\end{pgfscope}%
\begin{pgfscope}%
\pgfpathrectangle{\pgfqpoint{1.072000in}{0.528000in}}{\pgfqpoint{3.696000in}{3.696000in}}%
\pgfusepath{clip}%
\pgfsetbuttcap%
\pgfsetroundjoin%
\definecolor{currentfill}{rgb}{0.425199,0.559058,0.946061}%
\pgfsetfillcolor{currentfill}%
\pgfsetlinewidth{0.000000pt}%
\definecolor{currentstroke}{rgb}{0.000000,0.000000,0.000000}%
\pgfsetstrokecolor{currentstroke}%
\pgfsetdash{}{0pt}%
\pgfpathmoveto{\pgfqpoint{2.730994in}{2.318540in}}%
\pgfpathlineto{\pgfqpoint{2.774136in}{2.409169in}}%
\pgfpathlineto{\pgfqpoint{2.800404in}{2.336048in}}%
\pgfpathlineto{\pgfqpoint{2.757239in}{2.264872in}}%
\pgfpathlineto{\pgfqpoint{2.730994in}{2.318540in}}%
\pgfpathclose%
\pgfusepath{fill}%
\end{pgfscope}%
\begin{pgfscope}%
\pgfpathrectangle{\pgfqpoint{1.072000in}{0.528000in}}{\pgfqpoint{3.696000in}{3.696000in}}%
\pgfusepath{clip}%
\pgfsetbuttcap%
\pgfsetroundjoin%
\definecolor{currentfill}{rgb}{0.248091,0.326013,0.777669}%
\pgfsetfillcolor{currentfill}%
\pgfsetlinewidth{0.000000pt}%
\definecolor{currentstroke}{rgb}{0.000000,0.000000,0.000000}%
\pgfsetstrokecolor{currentstroke}%
\pgfsetdash{}{0pt}%
\pgfpathmoveto{\pgfqpoint{3.768506in}{2.042219in}}%
\pgfpathlineto{\pgfqpoint{3.813624in}{2.039967in}}%
\pgfpathlineto{\pgfqpoint{3.838655in}{2.113577in}}%
\pgfpathlineto{\pgfqpoint{3.793194in}{2.102099in}}%
\pgfpathlineto{\pgfqpoint{3.768506in}{2.042219in}}%
\pgfpathclose%
\pgfusepath{fill}%
\end{pgfscope}%
\begin{pgfscope}%
\pgfpathrectangle{\pgfqpoint{1.072000in}{0.528000in}}{\pgfqpoint{3.696000in}{3.696000in}}%
\pgfusepath{clip}%
\pgfsetbuttcap%
\pgfsetroundjoin%
\definecolor{currentfill}{rgb}{0.229806,0.298718,0.753683}%
\pgfsetfillcolor{currentfill}%
\pgfsetlinewidth{0.000000pt}%
\definecolor{currentstroke}{rgb}{0.000000,0.000000,0.000000}%
\pgfsetstrokecolor{currentstroke}%
\pgfsetdash{}{0pt}%
\pgfpathmoveto{\pgfqpoint{3.700024in}{2.042536in}}%
\pgfpathlineto{\pgfqpoint{3.744623in}{2.019506in}}%
\pgfpathlineto{\pgfqpoint{3.768506in}{2.042219in}}%
\pgfpathlineto{\pgfqpoint{3.723660in}{2.050999in}}%
\pgfpathlineto{\pgfqpoint{3.700024in}{2.042536in}}%
\pgfpathclose%
\pgfusepath{fill}%
\end{pgfscope}%
\begin{pgfscope}%
\pgfpathrectangle{\pgfqpoint{1.072000in}{0.528000in}}{\pgfqpoint{3.696000in}{3.696000in}}%
\pgfusepath{clip}%
\pgfsetbuttcap%
\pgfsetroundjoin%
\definecolor{currentfill}{rgb}{0.378598,0.503856,0.913692}%
\pgfsetfillcolor{currentfill}%
\pgfsetlinewidth{0.000000pt}%
\definecolor{currentstroke}{rgb}{0.000000,0.000000,0.000000}%
\pgfsetstrokecolor{currentstroke}%
\pgfsetdash{}{0pt}%
\pgfpathmoveto{\pgfqpoint{2.592412in}{2.240284in}}%
\pgfpathlineto{\pgfqpoint{2.634950in}{2.333787in}}%
\pgfpathlineto{\pgfqpoint{2.661643in}{2.285667in}}%
\pgfpathlineto{\pgfqpoint{2.618980in}{2.210551in}}%
\pgfpathlineto{\pgfqpoint{2.592412in}{2.240284in}}%
\pgfpathclose%
\pgfusepath{fill}%
\end{pgfscope}%
\begin{pgfscope}%
\pgfpathrectangle{\pgfqpoint{1.072000in}{0.528000in}}{\pgfqpoint{3.696000in}{3.696000in}}%
\pgfusepath{clip}%
\pgfsetbuttcap%
\pgfsetroundjoin%
\definecolor{currentfill}{rgb}{0.409611,0.540759,0.935545}%
\pgfsetfillcolor{currentfill}%
\pgfsetlinewidth{0.000000pt}%
\definecolor{currentstroke}{rgb}{0.000000,0.000000,0.000000}%
\pgfsetstrokecolor{currentstroke}%
\pgfsetdash{}{0pt}%
\pgfpathmoveto{\pgfqpoint{2.661643in}{2.285667in}}%
\pgfpathlineto{\pgfqpoint{2.704474in}{2.380303in}}%
\pgfpathlineto{\pgfqpoint{2.730994in}{2.318540in}}%
\pgfpathlineto{\pgfqpoint{2.688086in}{2.243023in}}%
\pgfpathlineto{\pgfqpoint{2.661643in}{2.285667in}}%
\pgfpathclose%
\pgfusepath{fill}%
\end{pgfscope}%
\begin{pgfscope}%
\pgfpathrectangle{\pgfqpoint{1.072000in}{0.528000in}}{\pgfqpoint{3.696000in}{3.696000in}}%
\pgfusepath{clip}%
\pgfsetbuttcap%
\pgfsetroundjoin%
\definecolor{currentfill}{rgb}{0.313946,0.420052,0.854993}%
\pgfsetfillcolor{currentfill}%
\pgfsetlinewidth{0.000000pt}%
\definecolor{currentstroke}{rgb}{0.000000,0.000000,0.000000}%
\pgfsetstrokecolor{currentstroke}%
\pgfsetdash{}{0pt}%
\pgfpathmoveto{\pgfqpoint{3.499599in}{2.212592in}}%
\pgfpathlineto{\pgfqpoint{3.543997in}{2.183095in}}%
\pgfpathlineto{\pgfqpoint{3.566824in}{2.120940in}}%
\pgfpathlineto{\pgfqpoint{3.522505in}{2.144074in}}%
\pgfpathlineto{\pgfqpoint{3.499599in}{2.212592in}}%
\pgfpathclose%
\pgfusepath{fill}%
\end{pgfscope}%
\begin{pgfscope}%
\pgfpathrectangle{\pgfqpoint{1.072000in}{0.528000in}}{\pgfqpoint{3.696000in}{3.696000in}}%
\pgfusepath{clip}%
\pgfsetbuttcap%
\pgfsetroundjoin%
\definecolor{currentfill}{rgb}{0.238948,0.312365,0.765676}%
\pgfsetfillcolor{currentfill}%
\pgfsetlinewidth{0.000000pt}%
\definecolor{currentstroke}{rgb}{0.000000,0.000000,0.000000}%
\pgfsetstrokecolor{currentstroke}%
\pgfsetdash{}{0pt}%
\pgfpathmoveto{\pgfqpoint{2.175555in}{2.061787in}}%
\pgfpathlineto{\pgfqpoint{2.219428in}{2.016500in}}%
\pgfpathlineto{\pgfqpoint{2.246175in}{2.041044in}}%
\pgfpathlineto{\pgfqpoint{2.202264in}{2.091354in}}%
\pgfpathlineto{\pgfqpoint{2.175555in}{2.061787in}}%
\pgfpathclose%
\pgfusepath{fill}%
\end{pgfscope}%
\begin{pgfscope}%
\pgfpathrectangle{\pgfqpoint{1.072000in}{0.528000in}}{\pgfqpoint{3.696000in}{3.696000in}}%
\pgfusepath{clip}%
\pgfsetbuttcap%
\pgfsetroundjoin%
\definecolor{currentfill}{rgb}{0.243520,0.319189,0.771672}%
\pgfsetfillcolor{currentfill}%
\pgfsetlinewidth{0.000000pt}%
\definecolor{currentstroke}{rgb}{0.000000,0.000000,0.000000}%
\pgfsetstrokecolor{currentstroke}%
\pgfsetdash{}{0pt}%
\pgfpathmoveto{\pgfqpoint{2.289255in}{2.032517in}}%
\pgfpathlineto{\pgfqpoint{2.331670in}{2.065443in}}%
\pgfpathlineto{\pgfqpoint{2.358516in}{2.075440in}}%
\pgfpathlineto{\pgfqpoint{2.315970in}{2.051051in}}%
\pgfpathlineto{\pgfqpoint{2.289255in}{2.032517in}}%
\pgfpathclose%
\pgfusepath{fill}%
\end{pgfscope}%
\begin{pgfscope}%
\pgfpathrectangle{\pgfqpoint{1.072000in}{0.528000in}}{\pgfqpoint{3.696000in}{3.696000in}}%
\pgfusepath{clip}%
\pgfsetbuttcap%
\pgfsetroundjoin%
\definecolor{currentfill}{rgb}{0.299441,0.400248,0.839842}%
\pgfsetfillcolor{currentfill}%
\pgfsetlinewidth{0.000000pt}%
\definecolor{currentstroke}{rgb}{0.000000,0.000000,0.000000}%
\pgfsetstrokecolor{currentstroke}%
\pgfsetdash{}{0pt}%
\pgfpathmoveto{\pgfqpoint{3.543997in}{2.183095in}}%
\pgfpathlineto{\pgfqpoint{3.588360in}{2.148587in}}%
\pgfpathlineto{\pgfqpoint{3.611168in}{2.095248in}}%
\pgfpathlineto{\pgfqpoint{3.566824in}{2.120940in}}%
\pgfpathlineto{\pgfqpoint{3.543997in}{2.183095in}}%
\pgfpathclose%
\pgfusepath{fill}%
\end{pgfscope}%
\begin{pgfscope}%
\pgfpathrectangle{\pgfqpoint{1.072000in}{0.528000in}}{\pgfqpoint{3.696000in}{3.696000in}}%
\pgfusepath{clip}%
\pgfsetbuttcap%
\pgfsetroundjoin%
\definecolor{currentfill}{rgb}{0.229806,0.298718,0.753683}%
\pgfsetfillcolor{currentfill}%
\pgfsetlinewidth{0.000000pt}%
\definecolor{currentstroke}{rgb}{0.000000,0.000000,0.000000}%
\pgfsetstrokecolor{currentstroke}%
\pgfsetdash{}{0pt}%
\pgfpathmoveto{\pgfqpoint{2.219428in}{2.016500in}}%
\pgfpathlineto{\pgfqpoint{2.262436in}{2.013711in}}%
\pgfpathlineto{\pgfqpoint{2.289255in}{2.032517in}}%
\pgfpathlineto{\pgfqpoint{2.246175in}{2.041044in}}%
\pgfpathlineto{\pgfqpoint{2.219428in}{2.016500in}}%
\pgfpathclose%
\pgfusepath{fill}%
\end{pgfscope}%
\begin{pgfscope}%
\pgfpathrectangle{\pgfqpoint{1.072000in}{0.528000in}}{\pgfqpoint{3.696000in}{3.696000in}}%
\pgfusepath{clip}%
\pgfsetbuttcap%
\pgfsetroundjoin%
\definecolor{currentfill}{rgb}{0.229806,0.298718,0.753683}%
\pgfsetfillcolor{currentfill}%
\pgfsetlinewidth{0.000000pt}%
\definecolor{currentstroke}{rgb}{0.000000,0.000000,0.000000}%
\pgfsetstrokecolor{currentstroke}%
\pgfsetdash{}{0pt}%
\pgfpathmoveto{\pgfqpoint{3.744623in}{2.019506in}}%
\pgfpathlineto{\pgfqpoint{3.789422in}{2.001748in}}%
\pgfpathlineto{\pgfqpoint{3.813624in}{2.039967in}}%
\pgfpathlineto{\pgfqpoint{3.768506in}{2.042219in}}%
\pgfpathlineto{\pgfqpoint{3.744623in}{2.019506in}}%
\pgfpathclose%
\pgfusepath{fill}%
\end{pgfscope}%
\begin{pgfscope}%
\pgfpathrectangle{\pgfqpoint{1.072000in}{0.528000in}}{\pgfqpoint{3.696000in}{3.696000in}}%
\pgfusepath{clip}%
\pgfsetbuttcap%
\pgfsetroundjoin%
\definecolor{currentfill}{rgb}{0.483854,0.622050,0.974808}%
\pgfsetfillcolor{currentfill}%
\pgfsetlinewidth{0.000000pt}%
\definecolor{currentstroke}{rgb}{0.000000,0.000000,0.000000}%
\pgfsetstrokecolor{currentstroke}%
\pgfsetdash{}{0pt}%
\pgfpathmoveto{\pgfqpoint{2.913454in}{2.408326in}}%
\pgfpathlineto{\pgfqpoint{2.957436in}{2.482828in}}%
\pgfpathlineto{\pgfqpoint{2.982876in}{2.380590in}}%
\pgfpathlineto{\pgfqpoint{2.939028in}{2.323147in}}%
\pgfpathlineto{\pgfqpoint{2.913454in}{2.408326in}}%
\pgfpathclose%
\pgfusepath{fill}%
\end{pgfscope}%
\begin{pgfscope}%
\pgfpathrectangle{\pgfqpoint{1.072000in}{0.528000in}}{\pgfqpoint{3.696000in}{3.696000in}}%
\pgfusepath{clip}%
\pgfsetbuttcap%
\pgfsetroundjoin%
\definecolor{currentfill}{rgb}{0.280550,0.373423,0.818011}%
\pgfsetfillcolor{currentfill}%
\pgfsetlinewidth{0.000000pt}%
\definecolor{currentstroke}{rgb}{0.000000,0.000000,0.000000}%
\pgfsetstrokecolor{currentstroke}%
\pgfsetdash{}{0pt}%
\pgfpathmoveto{\pgfqpoint{2.358516in}{2.075440in}}%
\pgfpathlineto{\pgfqpoint{2.400611in}{2.138352in}}%
\pgfpathlineto{\pgfqpoint{2.427519in}{2.136108in}}%
\pgfpathlineto{\pgfqpoint{2.385257in}{2.084369in}}%
\pgfpathlineto{\pgfqpoint{2.358516in}{2.075440in}}%
\pgfpathclose%
\pgfusepath{fill}%
\end{pgfscope}%
\begin{pgfscope}%
\pgfpathrectangle{\pgfqpoint{1.072000in}{0.528000in}}{\pgfqpoint{3.696000in}{3.696000in}}%
\pgfusepath{clip}%
\pgfsetbuttcap%
\pgfsetroundjoin%
\definecolor{currentfill}{rgb}{0.425199,0.559058,0.946061}%
\pgfsetfillcolor{currentfill}%
\pgfsetlinewidth{0.000000pt}%
\definecolor{currentstroke}{rgb}{0.000000,0.000000,0.000000}%
\pgfsetstrokecolor{currentstroke}%
\pgfsetdash{}{0pt}%
\pgfpathmoveto{\pgfqpoint{3.298372in}{2.366442in}}%
\pgfpathlineto{\pgfqpoint{3.343027in}{2.367743in}}%
\pgfpathlineto{\pgfqpoint{3.366284in}{2.261678in}}%
\pgfpathlineto{\pgfqpoint{3.321882in}{2.264160in}}%
\pgfpathlineto{\pgfqpoint{3.298372in}{2.366442in}}%
\pgfpathclose%
\pgfusepath{fill}%
\end{pgfscope}%
\begin{pgfscope}%
\pgfpathrectangle{\pgfqpoint{1.072000in}{0.528000in}}{\pgfqpoint{3.696000in}{3.696000in}}%
\pgfusepath{clip}%
\pgfsetbuttcap%
\pgfsetroundjoin%
\definecolor{currentfill}{rgb}{0.473070,0.611077,0.970634}%
\pgfsetfillcolor{currentfill}%
\pgfsetlinewidth{0.000000pt}%
\definecolor{currentstroke}{rgb}{0.000000,0.000000,0.000000}%
\pgfsetstrokecolor{currentstroke}%
\pgfsetdash{}{0pt}%
\pgfpathmoveto{\pgfqpoint{3.140596in}{2.420997in}}%
\pgfpathlineto{\pgfqpoint{3.185198in}{2.454466in}}%
\pgfpathlineto{\pgfqpoint{3.209297in}{2.339263in}}%
\pgfpathlineto{\pgfqpoint{3.164954in}{2.316306in}}%
\pgfpathlineto{\pgfqpoint{3.140596in}{2.420997in}}%
\pgfpathclose%
\pgfusepath{fill}%
\end{pgfscope}%
\begin{pgfscope}%
\pgfpathrectangle{\pgfqpoint{1.072000in}{0.528000in}}{\pgfqpoint{3.696000in}{3.696000in}}%
\pgfusepath{clip}%
\pgfsetbuttcap%
\pgfsetroundjoin%
\definecolor{currentfill}{rgb}{0.280550,0.373423,0.818011}%
\pgfsetfillcolor{currentfill}%
\pgfsetlinewidth{0.000000pt}%
\definecolor{currentstroke}{rgb}{0.000000,0.000000,0.000000}%
\pgfsetstrokecolor{currentstroke}%
\pgfsetdash{}{0pt}%
\pgfpathmoveto{\pgfqpoint{3.588360in}{2.148587in}}%
\pgfpathlineto{\pgfqpoint{3.632702in}{2.110689in}}%
\pgfpathlineto{\pgfqpoint{3.655556in}{2.068497in}}%
\pgfpathlineto{\pgfqpoint{3.611168in}{2.095248in}}%
\pgfpathlineto{\pgfqpoint{3.588360in}{2.148587in}}%
\pgfpathclose%
\pgfusepath{fill}%
\end{pgfscope}%
\begin{pgfscope}%
\pgfpathrectangle{\pgfqpoint{1.072000in}{0.528000in}}{\pgfqpoint{3.696000in}{3.696000in}}%
\pgfusepath{clip}%
\pgfsetbuttcap%
\pgfsetroundjoin%
\definecolor{currentfill}{rgb}{0.271104,0.360011,0.807095}%
\pgfsetfillcolor{currentfill}%
\pgfsetlinewidth{0.000000pt}%
\definecolor{currentstroke}{rgb}{0.000000,0.000000,0.000000}%
\pgfsetstrokecolor{currentstroke}%
\pgfsetdash{}{0pt}%
\pgfpathmoveto{\pgfqpoint{3.813624in}{2.039967in}}%
\pgfpathlineto{\pgfqpoint{3.859092in}{2.046239in}}%
\pgfpathlineto{\pgfqpoint{3.884521in}{2.134063in}}%
\pgfpathlineto{\pgfqpoint{3.838655in}{2.113577in}}%
\pgfpathlineto{\pgfqpoint{3.813624in}{2.039967in}}%
\pgfpathclose%
\pgfusepath{fill}%
\end{pgfscope}%
\begin{pgfscope}%
\pgfpathrectangle{\pgfqpoint{1.072000in}{0.528000in}}{\pgfqpoint{3.696000in}{3.696000in}}%
\pgfusepath{clip}%
\pgfsetbuttcap%
\pgfsetroundjoin%
\definecolor{currentfill}{rgb}{0.505423,0.643995,0.983157}%
\pgfsetfillcolor{currentfill}%
\pgfsetlinewidth{0.000000pt}%
\definecolor{currentstroke}{rgb}{0.000000,0.000000,0.000000}%
\pgfsetstrokecolor{currentstroke}%
\pgfsetdash{}{0pt}%
\pgfpathmoveto{\pgfqpoint{3.027000in}{2.439443in}}%
\pgfpathlineto{\pgfqpoint{3.071399in}{2.495995in}}%
\pgfpathlineto{\pgfqpoint{3.096189in}{2.381669in}}%
\pgfpathlineto{\pgfqpoint{3.052004in}{2.339167in}}%
\pgfpathlineto{\pgfqpoint{3.027000in}{2.439443in}}%
\pgfpathclose%
\pgfusepath{fill}%
\end{pgfscope}%
\begin{pgfscope}%
\pgfpathrectangle{\pgfqpoint{1.072000in}{0.528000in}}{\pgfqpoint{3.696000in}{3.696000in}}%
\pgfusepath{clip}%
\pgfsetbuttcap%
\pgfsetroundjoin%
\definecolor{currentfill}{rgb}{0.266381,0.353304,0.801637}%
\pgfsetfillcolor{currentfill}%
\pgfsetlinewidth{0.000000pt}%
\definecolor{currentstroke}{rgb}{0.000000,0.000000,0.000000}%
\pgfsetstrokecolor{currentstroke}%
\pgfsetdash{}{0pt}%
\pgfpathmoveto{\pgfqpoint{3.632702in}{2.110689in}}%
\pgfpathlineto{\pgfqpoint{3.677049in}{2.071437in}}%
\pgfpathlineto{\pgfqpoint{3.700024in}{2.042536in}}%
\pgfpathlineto{\pgfqpoint{3.655556in}{2.068497in}}%
\pgfpathlineto{\pgfqpoint{3.632702in}{2.110689in}}%
\pgfpathclose%
\pgfusepath{fill}%
\end{pgfscope}%
\begin{pgfscope}%
\pgfpathrectangle{\pgfqpoint{1.072000in}{0.528000in}}{\pgfqpoint{3.696000in}{3.696000in}}%
\pgfusepath{clip}%
\pgfsetbuttcap%
\pgfsetroundjoin%
\definecolor{currentfill}{rgb}{0.333490,0.446265,0.874452}%
\pgfsetfillcolor{currentfill}%
\pgfsetlinewidth{0.000000pt}%
\definecolor{currentstroke}{rgb}{0.000000,0.000000,0.000000}%
\pgfsetstrokecolor{currentstroke}%
\pgfsetdash{}{0pt}%
\pgfpathmoveto{\pgfqpoint{2.427519in}{2.136108in}}%
\pgfpathlineto{\pgfqpoint{2.469541in}{2.222039in}}%
\pgfpathlineto{\pgfqpoint{2.496504in}{2.205056in}}%
\pgfpathlineto{\pgfqpoint{2.454305in}{2.132554in}}%
\pgfpathlineto{\pgfqpoint{2.427519in}{2.136108in}}%
\pgfpathclose%
\pgfusepath{fill}%
\end{pgfscope}%
\begin{pgfscope}%
\pgfpathrectangle{\pgfqpoint{1.072000in}{0.528000in}}{\pgfqpoint{3.696000in}{3.696000in}}%
\pgfusepath{clip}%
\pgfsetbuttcap%
\pgfsetroundjoin%
\definecolor{currentfill}{rgb}{0.238948,0.312365,0.765676}%
\pgfsetfillcolor{currentfill}%
\pgfsetlinewidth{0.000000pt}%
\definecolor{currentstroke}{rgb}{0.000000,0.000000,0.000000}%
\pgfsetstrokecolor{currentstroke}%
\pgfsetdash{}{0pt}%
\pgfpathmoveto{\pgfqpoint{2.148681in}{2.035178in}}%
\pgfpathlineto{\pgfqpoint{2.192558in}{1.992938in}}%
\pgfpathlineto{\pgfqpoint{2.219428in}{2.016500in}}%
\pgfpathlineto{\pgfqpoint{2.175555in}{2.061787in}}%
\pgfpathlineto{\pgfqpoint{2.148681in}{2.035178in}}%
\pgfpathclose%
\pgfusepath{fill}%
\end{pgfscope}%
\begin{pgfscope}%
\pgfpathrectangle{\pgfqpoint{1.072000in}{0.528000in}}{\pgfqpoint{3.696000in}{3.696000in}}%
\pgfusepath{clip}%
\pgfsetbuttcap%
\pgfsetroundjoin%
\definecolor{currentfill}{rgb}{0.234377,0.305542,0.759680}%
\pgfsetfillcolor{currentfill}%
\pgfsetlinewidth{0.000000pt}%
\definecolor{currentstroke}{rgb}{0.000000,0.000000,0.000000}%
\pgfsetstrokecolor{currentstroke}%
\pgfsetdash{}{0pt}%
\pgfpathmoveto{\pgfqpoint{3.789422in}{2.001748in}}%
\pgfpathlineto{\pgfqpoint{3.834504in}{1.991692in}}%
\pgfpathlineto{\pgfqpoint{3.859092in}{2.046239in}}%
\pgfpathlineto{\pgfqpoint{3.813624in}{2.039967in}}%
\pgfpathlineto{\pgfqpoint{3.789422in}{2.001748in}}%
\pgfpathclose%
\pgfusepath{fill}%
\end{pgfscope}%
\begin{pgfscope}%
\pgfpathrectangle{\pgfqpoint{1.072000in}{0.528000in}}{\pgfqpoint{3.696000in}{3.696000in}}%
\pgfusepath{clip}%
\pgfsetbuttcap%
\pgfsetroundjoin%
\definecolor{currentfill}{rgb}{0.248091,0.326013,0.777669}%
\pgfsetfillcolor{currentfill}%
\pgfsetlinewidth{0.000000pt}%
\definecolor{currentstroke}{rgb}{0.000000,0.000000,0.000000}%
\pgfsetstrokecolor{currentstroke}%
\pgfsetdash{}{0pt}%
\pgfpathmoveto{\pgfqpoint{3.677049in}{2.071437in}}%
\pgfpathlineto{\pgfqpoint{3.721450in}{2.033226in}}%
\pgfpathlineto{\pgfqpoint{3.744623in}{2.019506in}}%
\pgfpathlineto{\pgfqpoint{3.700024in}{2.042536in}}%
\pgfpathlineto{\pgfqpoint{3.677049in}{2.071437in}}%
\pgfpathclose%
\pgfusepath{fill}%
\end{pgfscope}%
\begin{pgfscope}%
\pgfpathrectangle{\pgfqpoint{1.072000in}{0.528000in}}{\pgfqpoint{3.696000in}{3.696000in}}%
\pgfusepath{clip}%
\pgfsetbuttcap%
\pgfsetroundjoin%
\definecolor{currentfill}{rgb}{0.430507,0.564883,0.948889}%
\pgfsetfillcolor{currentfill}%
\pgfsetlinewidth{0.000000pt}%
\definecolor{currentstroke}{rgb}{0.000000,0.000000,0.000000}%
\pgfsetstrokecolor{currentstroke}%
\pgfsetdash{}{0pt}%
\pgfpathmoveto{\pgfqpoint{3.343027in}{2.367743in}}%
\pgfpathlineto{\pgfqpoint{3.387703in}{2.359703in}}%
\pgfpathlineto{\pgfqpoint{3.410723in}{2.252346in}}%
\pgfpathlineto{\pgfqpoint{3.366284in}{2.261678in}}%
\pgfpathlineto{\pgfqpoint{3.343027in}{2.367743in}}%
\pgfpathclose%
\pgfusepath{fill}%
\end{pgfscope}%
\begin{pgfscope}%
\pgfpathrectangle{\pgfqpoint{1.072000in}{0.528000in}}{\pgfqpoint{3.696000in}{3.696000in}}%
\pgfusepath{clip}%
\pgfsetbuttcap%
\pgfsetroundjoin%
\definecolor{currentfill}{rgb}{0.510824,0.649397,0.985079}%
\pgfsetfillcolor{currentfill}%
\pgfsetlinewidth{0.000000pt}%
\definecolor{currentstroke}{rgb}{0.000000,0.000000,0.000000}%
\pgfsetstrokecolor{currentstroke}%
\pgfsetdash{}{0pt}%
\pgfpathmoveto{\pgfqpoint{2.843836in}{2.418500in}}%
\pgfpathlineto{\pgfqpoint{2.887592in}{2.507022in}}%
\pgfpathlineto{\pgfqpoint{2.913454in}{2.408326in}}%
\pgfpathlineto{\pgfqpoint{2.869781in}{2.337255in}}%
\pgfpathlineto{\pgfqpoint{2.843836in}{2.418500in}}%
\pgfpathclose%
\pgfusepath{fill}%
\end{pgfscope}%
\begin{pgfscope}%
\pgfpathrectangle{\pgfqpoint{1.072000in}{0.528000in}}{\pgfqpoint{3.696000in}{3.696000in}}%
\pgfusepath{clip}%
\pgfsetbuttcap%
\pgfsetroundjoin%
\definecolor{currentfill}{rgb}{0.252663,0.332837,0.783665}%
\pgfsetfillcolor{currentfill}%
\pgfsetlinewidth{0.000000pt}%
\definecolor{currentstroke}{rgb}{0.000000,0.000000,0.000000}%
\pgfsetstrokecolor{currentstroke}%
\pgfsetdash{}{0pt}%
\pgfpathmoveto{\pgfqpoint{2.262436in}{2.013711in}}%
\pgfpathlineto{\pgfqpoint{2.304750in}{2.052869in}}%
\pgfpathlineto{\pgfqpoint{2.331670in}{2.065443in}}%
\pgfpathlineto{\pgfqpoint{2.289255in}{2.032517in}}%
\pgfpathlineto{\pgfqpoint{2.262436in}{2.013711in}}%
\pgfpathclose%
\pgfusepath{fill}%
\end{pgfscope}%
\begin{pgfscope}%
\pgfpathrectangle{\pgfqpoint{1.072000in}{0.528000in}}{\pgfqpoint{3.696000in}{3.696000in}}%
\pgfusepath{clip}%
\pgfsetbuttcap%
\pgfsetroundjoin%
\definecolor{currentfill}{rgb}{0.229806,0.298718,0.753683}%
\pgfsetfillcolor{currentfill}%
\pgfsetlinewidth{0.000000pt}%
\definecolor{currentstroke}{rgb}{0.000000,0.000000,0.000000}%
\pgfsetstrokecolor{currentstroke}%
\pgfsetdash{}{0pt}%
\pgfpathmoveto{\pgfqpoint{2.192558in}{1.992938in}}%
\pgfpathlineto{\pgfqpoint{2.235533in}{1.993641in}}%
\pgfpathlineto{\pgfqpoint{2.262436in}{2.013711in}}%
\pgfpathlineto{\pgfqpoint{2.219428in}{2.016500in}}%
\pgfpathlineto{\pgfqpoint{2.192558in}{1.992938in}}%
\pgfpathclose%
\pgfusepath{fill}%
\end{pgfscope}%
\begin{pgfscope}%
\pgfpathrectangle{\pgfqpoint{1.072000in}{0.528000in}}{\pgfqpoint{3.696000in}{3.696000in}}%
\pgfusepath{clip}%
\pgfsetbuttcap%
\pgfsetroundjoin%
\definecolor{currentfill}{rgb}{0.388852,0.516298,0.921373}%
\pgfsetfillcolor{currentfill}%
\pgfsetlinewidth{0.000000pt}%
\definecolor{currentstroke}{rgb}{0.000000,0.000000,0.000000}%
\pgfsetstrokecolor{currentstroke}%
\pgfsetdash{}{0pt}%
\pgfpathmoveto{\pgfqpoint{2.496504in}{2.205056in}}%
\pgfpathlineto{\pgfqpoint{2.538650in}{2.306576in}}%
\pgfpathlineto{\pgfqpoint{2.565624in}{2.273452in}}%
\pgfpathlineto{\pgfqpoint{2.523316in}{2.187109in}}%
\pgfpathlineto{\pgfqpoint{2.496504in}{2.205056in}}%
\pgfpathclose%
\pgfusepath{fill}%
\end{pgfscope}%
\begin{pgfscope}%
\pgfpathrectangle{\pgfqpoint{1.072000in}{0.528000in}}{\pgfqpoint{3.696000in}{3.696000in}}%
\pgfusepath{clip}%
\pgfsetbuttcap%
\pgfsetroundjoin%
\definecolor{currentfill}{rgb}{0.238948,0.312365,0.765676}%
\pgfsetfillcolor{currentfill}%
\pgfsetlinewidth{0.000000pt}%
\definecolor{currentstroke}{rgb}{0.000000,0.000000,0.000000}%
\pgfsetstrokecolor{currentstroke}%
\pgfsetdash{}{0pt}%
\pgfpathmoveto{\pgfqpoint{3.721450in}{2.033226in}}%
\pgfpathlineto{\pgfqpoint{3.765972in}{1.998729in}}%
\pgfpathlineto{\pgfqpoint{3.789422in}{2.001748in}}%
\pgfpathlineto{\pgfqpoint{3.744623in}{2.019506in}}%
\pgfpathlineto{\pgfqpoint{3.721450in}{2.033226in}}%
\pgfpathclose%
\pgfusepath{fill}%
\end{pgfscope}%
\begin{pgfscope}%
\pgfpathrectangle{\pgfqpoint{1.072000in}{0.528000in}}{\pgfqpoint{3.696000in}{3.696000in}}%
\pgfusepath{clip}%
\pgfsetbuttcap%
\pgfsetroundjoin%
\definecolor{currentfill}{rgb}{0.510824,0.649397,0.985079}%
\pgfsetfillcolor{currentfill}%
\pgfsetlinewidth{0.000000pt}%
\definecolor{currentstroke}{rgb}{0.000000,0.000000,0.000000}%
\pgfsetstrokecolor{currentstroke}%
\pgfsetdash{}{0pt}%
\pgfpathmoveto{\pgfqpoint{3.185198in}{2.454466in}}%
\pgfpathlineto{\pgfqpoint{3.229957in}{2.479885in}}%
\pgfpathlineto{\pgfqpoint{3.253782in}{2.356541in}}%
\pgfpathlineto{\pgfqpoint{3.209297in}{2.339263in}}%
\pgfpathlineto{\pgfqpoint{3.185198in}{2.454466in}}%
\pgfpathclose%
\pgfusepath{fill}%
\end{pgfscope}%
\begin{pgfscope}%
\pgfpathrectangle{\pgfqpoint{1.072000in}{0.528000in}}{\pgfqpoint{3.696000in}{3.696000in}}%
\pgfusepath{clip}%
\pgfsetbuttcap%
\pgfsetroundjoin%
\definecolor{currentfill}{rgb}{0.294718,0.393542,0.834384}%
\pgfsetfillcolor{currentfill}%
\pgfsetlinewidth{0.000000pt}%
\definecolor{currentstroke}{rgb}{0.000000,0.000000,0.000000}%
\pgfsetstrokecolor{currentstroke}%
\pgfsetdash{}{0pt}%
\pgfpathmoveto{\pgfqpoint{3.859092in}{2.046239in}}%
\pgfpathlineto{\pgfqpoint{3.904996in}{2.062933in}}%
\pgfpathlineto{\pgfqpoint{3.930867in}{2.164956in}}%
\pgfpathlineto{\pgfqpoint{3.884521in}{2.134063in}}%
\pgfpathlineto{\pgfqpoint{3.859092in}{2.046239in}}%
\pgfpathclose%
\pgfusepath{fill}%
\end{pgfscope}%
\begin{pgfscope}%
\pgfpathrectangle{\pgfqpoint{1.072000in}{0.528000in}}{\pgfqpoint{3.696000in}{3.696000in}}%
\pgfusepath{clip}%
\pgfsetbuttcap%
\pgfsetroundjoin%
\definecolor{currentfill}{rgb}{0.299441,0.400248,0.839842}%
\pgfsetfillcolor{currentfill}%
\pgfsetlinewidth{0.000000pt}%
\definecolor{currentstroke}{rgb}{0.000000,0.000000,0.000000}%
\pgfsetstrokecolor{currentstroke}%
\pgfsetdash{}{0pt}%
\pgfpathmoveto{\pgfqpoint{2.331670in}{2.065443in}}%
\pgfpathlineto{\pgfqpoint{2.373619in}{2.137238in}}%
\pgfpathlineto{\pgfqpoint{2.400611in}{2.138352in}}%
\pgfpathlineto{\pgfqpoint{2.358516in}{2.075440in}}%
\pgfpathlineto{\pgfqpoint{2.331670in}{2.065443in}}%
\pgfpathclose%
\pgfusepath{fill}%
\end{pgfscope}%
\begin{pgfscope}%
\pgfpathrectangle{\pgfqpoint{1.072000in}{0.528000in}}{\pgfqpoint{3.696000in}{3.696000in}}%
\pgfusepath{clip}%
\pgfsetbuttcap%
\pgfsetroundjoin%
\definecolor{currentfill}{rgb}{0.430507,0.564883,0.948889}%
\pgfsetfillcolor{currentfill}%
\pgfsetlinewidth{0.000000pt}%
\definecolor{currentstroke}{rgb}{0.000000,0.000000,0.000000}%
\pgfsetstrokecolor{currentstroke}%
\pgfsetdash{}{0pt}%
\pgfpathmoveto{\pgfqpoint{3.387703in}{2.359703in}}%
\pgfpathlineto{\pgfqpoint{3.432363in}{2.342063in}}%
\pgfpathlineto{\pgfqpoint{3.455169in}{2.235909in}}%
\pgfpathlineto{\pgfqpoint{3.410723in}{2.252346in}}%
\pgfpathlineto{\pgfqpoint{3.387703in}{2.359703in}}%
\pgfpathclose%
\pgfusepath{fill}%
\end{pgfscope}%
\begin{pgfscope}%
\pgfpathrectangle{\pgfqpoint{1.072000in}{0.528000in}}{\pgfqpoint{3.696000in}{3.696000in}}%
\pgfusepath{clip}%
\pgfsetbuttcap%
\pgfsetroundjoin%
\definecolor{currentfill}{rgb}{0.521696,0.659599,0.987736}%
\pgfsetfillcolor{currentfill}%
\pgfsetlinewidth{0.000000pt}%
\definecolor{currentstroke}{rgb}{0.000000,0.000000,0.000000}%
\pgfsetstrokecolor{currentstroke}%
\pgfsetdash{}{0pt}%
\pgfpathmoveto{\pgfqpoint{2.774136in}{2.409169in}}%
\pgfpathlineto{\pgfqpoint{2.817596in}{2.509201in}}%
\pgfpathlineto{\pgfqpoint{2.843836in}{2.418500in}}%
\pgfpathlineto{\pgfqpoint{2.800404in}{2.336048in}}%
\pgfpathlineto{\pgfqpoint{2.774136in}{2.409169in}}%
\pgfpathclose%
\pgfusepath{fill}%
\end{pgfscope}%
\begin{pgfscope}%
\pgfpathrectangle{\pgfqpoint{1.072000in}{0.528000in}}{\pgfqpoint{3.696000in}{3.696000in}}%
\pgfusepath{clip}%
\pgfsetbuttcap%
\pgfsetroundjoin%
\definecolor{currentfill}{rgb}{0.441123,0.576532,0.954545}%
\pgfsetfillcolor{currentfill}%
\pgfsetlinewidth{0.000000pt}%
\definecolor{currentstroke}{rgb}{0.000000,0.000000,0.000000}%
\pgfsetstrokecolor{currentstroke}%
\pgfsetdash{}{0pt}%
\pgfpathmoveto{\pgfqpoint{2.565624in}{2.273452in}}%
\pgfpathlineto{\pgfqpoint{2.608033in}{2.383340in}}%
\pgfpathlineto{\pgfqpoint{2.634950in}{2.333787in}}%
\pgfpathlineto{\pgfqpoint{2.592412in}{2.240284in}}%
\pgfpathlineto{\pgfqpoint{2.565624in}{2.273452in}}%
\pgfpathclose%
\pgfusepath{fill}%
\end{pgfscope}%
\begin{pgfscope}%
\pgfpathrectangle{\pgfqpoint{1.072000in}{0.528000in}}{\pgfqpoint{3.696000in}{3.696000in}}%
\pgfusepath{clip}%
\pgfsetbuttcap%
\pgfsetroundjoin%
\definecolor{currentfill}{rgb}{0.248091,0.326013,0.777669}%
\pgfsetfillcolor{currentfill}%
\pgfsetlinewidth{0.000000pt}%
\definecolor{currentstroke}{rgb}{0.000000,0.000000,0.000000}%
\pgfsetstrokecolor{currentstroke}%
\pgfsetdash{}{0pt}%
\pgfpathmoveto{\pgfqpoint{3.834504in}{1.991692in}}%
\pgfpathlineto{\pgfqpoint{3.879963in}{1.991732in}}%
\pgfpathlineto{\pgfqpoint{3.904996in}{2.062933in}}%
\pgfpathlineto{\pgfqpoint{3.859092in}{2.046239in}}%
\pgfpathlineto{\pgfqpoint{3.834504in}{1.991692in}}%
\pgfpathclose%
\pgfusepath{fill}%
\end{pgfscope}%
\begin{pgfscope}%
\pgfpathrectangle{\pgfqpoint{1.072000in}{0.528000in}}{\pgfqpoint{3.696000in}{3.696000in}}%
\pgfusepath{clip}%
\pgfsetbuttcap%
\pgfsetroundjoin%
\definecolor{currentfill}{rgb}{0.234377,0.305542,0.759680}%
\pgfsetfillcolor{currentfill}%
\pgfsetlinewidth{0.000000pt}%
\definecolor{currentstroke}{rgb}{0.000000,0.000000,0.000000}%
\pgfsetstrokecolor{currentstroke}%
\pgfsetdash{}{0pt}%
\pgfpathmoveto{\pgfqpoint{2.121646in}{2.011107in}}%
\pgfpathlineto{\pgfqpoint{2.165573in}{1.969838in}}%
\pgfpathlineto{\pgfqpoint{2.192558in}{1.992938in}}%
\pgfpathlineto{\pgfqpoint{2.148681in}{2.035178in}}%
\pgfpathlineto{\pgfqpoint{2.121646in}{2.011107in}}%
\pgfpathclose%
\pgfusepath{fill}%
\end{pgfscope}%
\begin{pgfscope}%
\pgfpathrectangle{\pgfqpoint{1.072000in}{0.528000in}}{\pgfqpoint{3.696000in}{3.696000in}}%
\pgfusepath{clip}%
\pgfsetbuttcap%
\pgfsetroundjoin%
\definecolor{currentfill}{rgb}{0.229806,0.298718,0.753683}%
\pgfsetfillcolor{currentfill}%
\pgfsetlinewidth{0.000000pt}%
\definecolor{currentstroke}{rgb}{0.000000,0.000000,0.000000}%
\pgfsetstrokecolor{currentstroke}%
\pgfsetdash{}{0pt}%
\pgfpathmoveto{\pgfqpoint{3.765972in}{1.998729in}}%
\pgfpathlineto{\pgfqpoint{3.810699in}{1.970786in}}%
\pgfpathlineto{\pgfqpoint{3.834504in}{1.991692in}}%
\pgfpathlineto{\pgfqpoint{3.789422in}{2.001748in}}%
\pgfpathlineto{\pgfqpoint{3.765972in}{1.998729in}}%
\pgfpathclose%
\pgfusepath{fill}%
\end{pgfscope}%
\begin{pgfscope}%
\pgfpathrectangle{\pgfqpoint{1.072000in}{0.528000in}}{\pgfqpoint{3.696000in}{3.696000in}}%
\pgfusepath{clip}%
\pgfsetbuttcap%
\pgfsetroundjoin%
\definecolor{currentfill}{rgb}{0.483854,0.622050,0.974808}%
\pgfsetfillcolor{currentfill}%
\pgfsetlinewidth{0.000000pt}%
\definecolor{currentstroke}{rgb}{0.000000,0.000000,0.000000}%
\pgfsetstrokecolor{currentstroke}%
\pgfsetdash{}{0pt}%
\pgfpathmoveto{\pgfqpoint{2.634950in}{2.333787in}}%
\pgfpathlineto{\pgfqpoint{2.677698in}{2.445541in}}%
\pgfpathlineto{\pgfqpoint{2.704474in}{2.380303in}}%
\pgfpathlineto{\pgfqpoint{2.661643in}{2.285667in}}%
\pgfpathlineto{\pgfqpoint{2.634950in}{2.333787in}}%
\pgfpathclose%
\pgfusepath{fill}%
\end{pgfscope}%
\begin{pgfscope}%
\pgfpathrectangle{\pgfqpoint{1.072000in}{0.528000in}}{\pgfqpoint{3.696000in}{3.696000in}}%
\pgfusepath{clip}%
\pgfsetbuttcap%
\pgfsetroundjoin%
\definecolor{currentfill}{rgb}{0.510824,0.649397,0.985079}%
\pgfsetfillcolor{currentfill}%
\pgfsetlinewidth{0.000000pt}%
\definecolor{currentstroke}{rgb}{0.000000,0.000000,0.000000}%
\pgfsetstrokecolor{currentstroke}%
\pgfsetdash{}{0pt}%
\pgfpathmoveto{\pgfqpoint{2.704474in}{2.380303in}}%
\pgfpathlineto{\pgfqpoint{2.747588in}{2.488415in}}%
\pgfpathlineto{\pgfqpoint{2.774136in}{2.409169in}}%
\pgfpathlineto{\pgfqpoint{2.730994in}{2.318540in}}%
\pgfpathlineto{\pgfqpoint{2.704474in}{2.380303in}}%
\pgfpathclose%
\pgfusepath{fill}%
\end{pgfscope}%
\begin{pgfscope}%
\pgfpathrectangle{\pgfqpoint{1.072000in}{0.528000in}}{\pgfqpoint{3.696000in}{3.696000in}}%
\pgfusepath{clip}%
\pgfsetbuttcap%
\pgfsetroundjoin%
\definecolor{currentfill}{rgb}{0.565182,0.699438,0.996635}%
\pgfsetfillcolor{currentfill}%
\pgfsetlinewidth{0.000000pt}%
\definecolor{currentstroke}{rgb}{0.000000,0.000000,0.000000}%
\pgfsetstrokecolor{currentstroke}%
\pgfsetdash{}{0pt}%
\pgfpathmoveto{\pgfqpoint{2.957436in}{2.482828in}}%
\pgfpathlineto{\pgfqpoint{3.001736in}{2.556415in}}%
\pgfpathlineto{\pgfqpoint{3.027000in}{2.439443in}}%
\pgfpathlineto{\pgfqpoint{2.982876in}{2.380590in}}%
\pgfpathlineto{\pgfqpoint{2.957436in}{2.482828in}}%
\pgfpathclose%
\pgfusepath{fill}%
\end{pgfscope}%
\begin{pgfscope}%
\pgfpathrectangle{\pgfqpoint{1.072000in}{0.528000in}}{\pgfqpoint{3.696000in}{3.696000in}}%
\pgfusepath{clip}%
\pgfsetbuttcap%
\pgfsetroundjoin%
\definecolor{currentfill}{rgb}{0.425199,0.559058,0.946061}%
\pgfsetfillcolor{currentfill}%
\pgfsetlinewidth{0.000000pt}%
\definecolor{currentstroke}{rgb}{0.000000,0.000000,0.000000}%
\pgfsetstrokecolor{currentstroke}%
\pgfsetdash{}{0pt}%
\pgfpathmoveto{\pgfqpoint{3.432363in}{2.342063in}}%
\pgfpathlineto{\pgfqpoint{3.476972in}{2.315041in}}%
\pgfpathlineto{\pgfqpoint{3.499599in}{2.212592in}}%
\pgfpathlineto{\pgfqpoint{3.455169in}{2.235909in}}%
\pgfpathlineto{\pgfqpoint{3.432363in}{2.342063in}}%
\pgfpathclose%
\pgfusepath{fill}%
\end{pgfscope}%
\begin{pgfscope}%
\pgfpathrectangle{\pgfqpoint{1.072000in}{0.528000in}}{\pgfqpoint{3.696000in}{3.696000in}}%
\pgfusepath{clip}%
\pgfsetbuttcap%
\pgfsetroundjoin%
\definecolor{currentfill}{rgb}{0.565182,0.699438,0.996635}%
\pgfsetfillcolor{currentfill}%
\pgfsetlinewidth{0.000000pt}%
\definecolor{currentstroke}{rgb}{0.000000,0.000000,0.000000}%
\pgfsetstrokecolor{currentstroke}%
\pgfsetdash{}{0pt}%
\pgfpathmoveto{\pgfqpoint{3.071399in}{2.495995in}}%
\pgfpathlineto{\pgfqpoint{3.116049in}{2.547007in}}%
\pgfpathlineto{\pgfqpoint{3.140596in}{2.420997in}}%
\pgfpathlineto{\pgfqpoint{3.096189in}{2.381669in}}%
\pgfpathlineto{\pgfqpoint{3.071399in}{2.495995in}}%
\pgfpathclose%
\pgfusepath{fill}%
\end{pgfscope}%
\begin{pgfscope}%
\pgfpathrectangle{\pgfqpoint{1.072000in}{0.528000in}}{\pgfqpoint{3.696000in}{3.696000in}}%
\pgfusepath{clip}%
\pgfsetbuttcap%
\pgfsetroundjoin%
\definecolor{currentfill}{rgb}{0.229806,0.298718,0.753683}%
\pgfsetfillcolor{currentfill}%
\pgfsetlinewidth{0.000000pt}%
\definecolor{currentstroke}{rgb}{0.000000,0.000000,0.000000}%
\pgfsetstrokecolor{currentstroke}%
\pgfsetdash{}{0pt}%
\pgfpathmoveto{\pgfqpoint{2.165573in}{1.969838in}}%
\pgfpathlineto{\pgfqpoint{2.208563in}{1.971695in}}%
\pgfpathlineto{\pgfqpoint{2.235533in}{1.993641in}}%
\pgfpathlineto{\pgfqpoint{2.192558in}{1.992938in}}%
\pgfpathlineto{\pgfqpoint{2.165573in}{1.969838in}}%
\pgfpathclose%
\pgfusepath{fill}%
\end{pgfscope}%
\begin{pgfscope}%
\pgfpathrectangle{\pgfqpoint{1.072000in}{0.528000in}}{\pgfqpoint{3.696000in}{3.696000in}}%
\pgfusepath{clip}%
\pgfsetbuttcap%
\pgfsetroundjoin%
\definecolor{currentfill}{rgb}{0.363461,0.484784,0.901019}%
\pgfsetfillcolor{currentfill}%
\pgfsetlinewidth{0.000000pt}%
\definecolor{currentstroke}{rgb}{0.000000,0.000000,0.000000}%
\pgfsetstrokecolor{currentstroke}%
\pgfsetdash{}{0pt}%
\pgfpathmoveto{\pgfqpoint{2.400611in}{2.138352in}}%
\pgfpathlineto{\pgfqpoint{2.442469in}{2.235467in}}%
\pgfpathlineto{\pgfqpoint{2.469541in}{2.222039in}}%
\pgfpathlineto{\pgfqpoint{2.427519in}{2.136108in}}%
\pgfpathlineto{\pgfqpoint{2.400611in}{2.138352in}}%
\pgfpathclose%
\pgfusepath{fill}%
\end{pgfscope}%
\begin{pgfscope}%
\pgfpathrectangle{\pgfqpoint{1.072000in}{0.528000in}}{\pgfqpoint{3.696000in}{3.696000in}}%
\pgfusepath{clip}%
\pgfsetbuttcap%
\pgfsetroundjoin%
\definecolor{currentfill}{rgb}{0.257234,0.339661,0.789661}%
\pgfsetfillcolor{currentfill}%
\pgfsetlinewidth{0.000000pt}%
\definecolor{currentstroke}{rgb}{0.000000,0.000000,0.000000}%
\pgfsetstrokecolor{currentstroke}%
\pgfsetdash{}{0pt}%
\pgfpathmoveto{\pgfqpoint{2.235533in}{1.993641in}}%
\pgfpathlineto{\pgfqpoint{2.277784in}{2.036598in}}%
\pgfpathlineto{\pgfqpoint{2.304750in}{2.052869in}}%
\pgfpathlineto{\pgfqpoint{2.262436in}{2.013711in}}%
\pgfpathlineto{\pgfqpoint{2.235533in}{1.993641in}}%
\pgfpathclose%
\pgfusepath{fill}%
\end{pgfscope}%
\begin{pgfscope}%
\pgfpathrectangle{\pgfqpoint{1.072000in}{0.528000in}}{\pgfqpoint{3.696000in}{3.696000in}}%
\pgfusepath{clip}%
\pgfsetbuttcap%
\pgfsetroundjoin%
\definecolor{currentfill}{rgb}{0.229806,0.298718,0.753683}%
\pgfsetfillcolor{currentfill}%
\pgfsetlinewidth{0.000000pt}%
\definecolor{currentstroke}{rgb}{0.000000,0.000000,0.000000}%
\pgfsetstrokecolor{currentstroke}%
\pgfsetdash{}{0pt}%
\pgfpathmoveto{\pgfqpoint{3.810699in}{1.970786in}}%
\pgfpathlineto{\pgfqpoint{3.855731in}{1.952268in}}%
\pgfpathlineto{\pgfqpoint{3.879963in}{1.991732in}}%
\pgfpathlineto{\pgfqpoint{3.834504in}{1.991692in}}%
\pgfpathlineto{\pgfqpoint{3.810699in}{1.970786in}}%
\pgfpathclose%
\pgfusepath{fill}%
\end{pgfscope}%
\begin{pgfscope}%
\pgfpathrectangle{\pgfqpoint{1.072000in}{0.528000in}}{\pgfqpoint{3.696000in}{3.696000in}}%
\pgfusepath{clip}%
\pgfsetbuttcap%
\pgfsetroundjoin%
\definecolor{currentfill}{rgb}{0.409611,0.540759,0.935545}%
\pgfsetfillcolor{currentfill}%
\pgfsetlinewidth{0.000000pt}%
\definecolor{currentstroke}{rgb}{0.000000,0.000000,0.000000}%
\pgfsetstrokecolor{currentstroke}%
\pgfsetdash{}{0pt}%
\pgfpathmoveto{\pgfqpoint{3.476972in}{2.315041in}}%
\pgfpathlineto{\pgfqpoint{3.521505in}{2.279339in}}%
\pgfpathlineto{\pgfqpoint{3.543997in}{2.183095in}}%
\pgfpathlineto{\pgfqpoint{3.499599in}{2.212592in}}%
\pgfpathlineto{\pgfqpoint{3.476972in}{2.315041in}}%
\pgfpathclose%
\pgfusepath{fill}%
\end{pgfscope}%
\begin{pgfscope}%
\pgfpathrectangle{\pgfqpoint{1.072000in}{0.528000in}}{\pgfqpoint{3.696000in}{3.696000in}}%
\pgfusepath{clip}%
\pgfsetbuttcap%
\pgfsetroundjoin%
\definecolor{currentfill}{rgb}{0.543440,0.680003,0.993051}%
\pgfsetfillcolor{currentfill}%
\pgfsetlinewidth{0.000000pt}%
\definecolor{currentstroke}{rgb}{0.000000,0.000000,0.000000}%
\pgfsetstrokecolor{currentstroke}%
\pgfsetdash{}{0pt}%
\pgfpathmoveto{\pgfqpoint{3.229957in}{2.479885in}}%
\pgfpathlineto{\pgfqpoint{3.274828in}{2.495572in}}%
\pgfpathlineto{\pgfqpoint{3.298372in}{2.366442in}}%
\pgfpathlineto{\pgfqpoint{3.253782in}{2.356541in}}%
\pgfpathlineto{\pgfqpoint{3.229957in}{2.479885in}}%
\pgfpathclose%
\pgfusepath{fill}%
\end{pgfscope}%
\begin{pgfscope}%
\pgfpathrectangle{\pgfqpoint{1.072000in}{0.528000in}}{\pgfqpoint{3.696000in}{3.696000in}}%
\pgfusepath{clip}%
\pgfsetbuttcap%
\pgfsetroundjoin%
\definecolor{currentfill}{rgb}{0.388852,0.516298,0.921373}%
\pgfsetfillcolor{currentfill}%
\pgfsetlinewidth{0.000000pt}%
\definecolor{currentstroke}{rgb}{0.000000,0.000000,0.000000}%
\pgfsetstrokecolor{currentstroke}%
\pgfsetdash{}{0pt}%
\pgfpathmoveto{\pgfqpoint{3.521505in}{2.279339in}}%
\pgfpathlineto{\pgfqpoint{3.565949in}{2.236150in}}%
\pgfpathlineto{\pgfqpoint{3.588360in}{2.148587in}}%
\pgfpathlineto{\pgfqpoint{3.543997in}{2.183095in}}%
\pgfpathlineto{\pgfqpoint{3.521505in}{2.279339in}}%
\pgfpathclose%
\pgfusepath{fill}%
\end{pgfscope}%
\begin{pgfscope}%
\pgfpathrectangle{\pgfqpoint{1.072000in}{0.528000in}}{\pgfqpoint{3.696000in}{3.696000in}}%
\pgfusepath{clip}%
\pgfsetbuttcap%
\pgfsetroundjoin%
\definecolor{currentfill}{rgb}{0.271104,0.360011,0.807095}%
\pgfsetfillcolor{currentfill}%
\pgfsetlinewidth{0.000000pt}%
\definecolor{currentstroke}{rgb}{0.000000,0.000000,0.000000}%
\pgfsetstrokecolor{currentstroke}%
\pgfsetdash{}{0pt}%
\pgfpathmoveto{\pgfqpoint{3.879963in}{1.991732in}}%
\pgfpathlineto{\pgfqpoint{3.925903in}{2.004077in}}%
\pgfpathlineto{\pgfqpoint{3.951426in}{2.091716in}}%
\pgfpathlineto{\pgfqpoint{3.904996in}{2.062933in}}%
\pgfpathlineto{\pgfqpoint{3.879963in}{1.991732in}}%
\pgfpathclose%
\pgfusepath{fill}%
\end{pgfscope}%
\begin{pgfscope}%
\pgfpathrectangle{\pgfqpoint{1.072000in}{0.528000in}}{\pgfqpoint{3.696000in}{3.696000in}}%
\pgfusepath{clip}%
\pgfsetbuttcap%
\pgfsetroundjoin%
\definecolor{currentfill}{rgb}{0.234377,0.305542,0.759680}%
\pgfsetfillcolor{currentfill}%
\pgfsetlinewidth{0.000000pt}%
\definecolor{currentstroke}{rgb}{0.000000,0.000000,0.000000}%
\pgfsetstrokecolor{currentstroke}%
\pgfsetdash{}{0pt}%
\pgfpathmoveto{\pgfqpoint{2.094444in}{1.989437in}}%
\pgfpathlineto{\pgfqpoint{2.138475in}{1.947023in}}%
\pgfpathlineto{\pgfqpoint{2.165573in}{1.969838in}}%
\pgfpathlineto{\pgfqpoint{2.121646in}{2.011107in}}%
\pgfpathlineto{\pgfqpoint{2.094444in}{1.989437in}}%
\pgfpathclose%
\pgfusepath{fill}%
\end{pgfscope}%
\begin{pgfscope}%
\pgfpathrectangle{\pgfqpoint{1.072000in}{0.528000in}}{\pgfqpoint{3.696000in}{3.696000in}}%
\pgfusepath{clip}%
\pgfsetbuttcap%
\pgfsetroundjoin%
\definecolor{currentfill}{rgb}{0.363461,0.484784,0.901019}%
\pgfsetfillcolor{currentfill}%
\pgfsetlinewidth{0.000000pt}%
\definecolor{currentstroke}{rgb}{0.000000,0.000000,0.000000}%
\pgfsetstrokecolor{currentstroke}%
\pgfsetdash{}{0pt}%
\pgfpathmoveto{\pgfqpoint{3.565949in}{2.236150in}}%
\pgfpathlineto{\pgfqpoint{3.610305in}{2.187152in}}%
\pgfpathlineto{\pgfqpoint{3.632702in}{2.110689in}}%
\pgfpathlineto{\pgfqpoint{3.588360in}{2.148587in}}%
\pgfpathlineto{\pgfqpoint{3.565949in}{2.236150in}}%
\pgfpathclose%
\pgfusepath{fill}%
\end{pgfscope}%
\begin{pgfscope}%
\pgfpathrectangle{\pgfqpoint{1.072000in}{0.528000in}}{\pgfqpoint{3.696000in}{3.696000in}}%
\pgfusepath{clip}%
\pgfsetbuttcap%
\pgfsetroundjoin%
\definecolor{currentfill}{rgb}{0.333490,0.446265,0.874452}%
\pgfsetfillcolor{currentfill}%
\pgfsetlinewidth{0.000000pt}%
\definecolor{currentstroke}{rgb}{0.000000,0.000000,0.000000}%
\pgfsetstrokecolor{currentstroke}%
\pgfsetdash{}{0pt}%
\pgfpathmoveto{\pgfqpoint{3.904996in}{2.062933in}}%
\pgfpathlineto{\pgfqpoint{3.951426in}{2.091716in}}%
\pgfpathlineto{\pgfqpoint{3.977769in}{2.207386in}}%
\pgfpathlineto{\pgfqpoint{3.930867in}{2.164956in}}%
\pgfpathlineto{\pgfqpoint{3.904996in}{2.062933in}}%
\pgfpathclose%
\pgfusepath{fill}%
\end{pgfscope}%
\begin{pgfscope}%
\pgfpathrectangle{\pgfqpoint{1.072000in}{0.528000in}}{\pgfqpoint{3.696000in}{3.696000in}}%
\pgfusepath{clip}%
\pgfsetbuttcap%
\pgfsetroundjoin%
\definecolor{currentfill}{rgb}{0.313946,0.420052,0.854993}%
\pgfsetfillcolor{currentfill}%
\pgfsetlinewidth{0.000000pt}%
\definecolor{currentstroke}{rgb}{0.000000,0.000000,0.000000}%
\pgfsetstrokecolor{currentstroke}%
\pgfsetdash{}{0pt}%
\pgfpathmoveto{\pgfqpoint{2.304750in}{2.052869in}}%
\pgfpathlineto{\pgfqpoint{2.346581in}{2.131118in}}%
\pgfpathlineto{\pgfqpoint{2.373619in}{2.137238in}}%
\pgfpathlineto{\pgfqpoint{2.331670in}{2.065443in}}%
\pgfpathlineto{\pgfqpoint{2.304750in}{2.052869in}}%
\pgfpathclose%
\pgfusepath{fill}%
\end{pgfscope}%
\begin{pgfscope}%
\pgfpathrectangle{\pgfqpoint{1.072000in}{0.528000in}}{\pgfqpoint{3.696000in}{3.696000in}}%
\pgfusepath{clip}%
\pgfsetbuttcap%
\pgfsetroundjoin%
\definecolor{currentfill}{rgb}{0.333490,0.446265,0.874452}%
\pgfsetfillcolor{currentfill}%
\pgfsetlinewidth{0.000000pt}%
\definecolor{currentstroke}{rgb}{0.000000,0.000000,0.000000}%
\pgfsetstrokecolor{currentstroke}%
\pgfsetdash{}{0pt}%
\pgfpathmoveto{\pgfqpoint{3.610305in}{2.187152in}}%
\pgfpathlineto{\pgfqpoint{3.654595in}{2.134499in}}%
\pgfpathlineto{\pgfqpoint{3.677049in}{2.071437in}}%
\pgfpathlineto{\pgfqpoint{3.632702in}{2.110689in}}%
\pgfpathlineto{\pgfqpoint{3.610305in}{2.187152in}}%
\pgfpathclose%
\pgfusepath{fill}%
\end{pgfscope}%
\begin{pgfscope}%
\pgfpathrectangle{\pgfqpoint{1.072000in}{0.528000in}}{\pgfqpoint{3.696000in}{3.696000in}}%
\pgfusepath{clip}%
\pgfsetbuttcap%
\pgfsetroundjoin%
\definecolor{currentfill}{rgb}{0.430507,0.564883,0.948889}%
\pgfsetfillcolor{currentfill}%
\pgfsetlinewidth{0.000000pt}%
\definecolor{currentstroke}{rgb}{0.000000,0.000000,0.000000}%
\pgfsetstrokecolor{currentstroke}%
\pgfsetdash{}{0pt}%
\pgfpathmoveto{\pgfqpoint{2.469541in}{2.222039in}}%
\pgfpathlineto{\pgfqpoint{2.511531in}{2.336515in}}%
\pgfpathlineto{\pgfqpoint{2.538650in}{2.306576in}}%
\pgfpathlineto{\pgfqpoint{2.496504in}{2.205056in}}%
\pgfpathlineto{\pgfqpoint{2.469541in}{2.222039in}}%
\pgfpathclose%
\pgfusepath{fill}%
\end{pgfscope}%
\begin{pgfscope}%
\pgfpathrectangle{\pgfqpoint{1.072000in}{0.528000in}}{\pgfqpoint{3.696000in}{3.696000in}}%
\pgfusepath{clip}%
\pgfsetbuttcap%
\pgfsetroundjoin%
\definecolor{currentfill}{rgb}{0.304174,0.406945,0.845263}%
\pgfsetfillcolor{currentfill}%
\pgfsetlinewidth{0.000000pt}%
\definecolor{currentstroke}{rgb}{0.000000,0.000000,0.000000}%
\pgfsetstrokecolor{currentstroke}%
\pgfsetdash{}{0pt}%
\pgfpathmoveto{\pgfqpoint{3.654595in}{2.134499in}}%
\pgfpathlineto{\pgfqpoint{3.698857in}{2.080777in}}%
\pgfpathlineto{\pgfqpoint{3.721450in}{2.033226in}}%
\pgfpathlineto{\pgfqpoint{3.677049in}{2.071437in}}%
\pgfpathlineto{\pgfqpoint{3.654595in}{2.134499in}}%
\pgfpathclose%
\pgfusepath{fill}%
\end{pgfscope}%
\begin{pgfscope}%
\pgfpathrectangle{\pgfqpoint{1.072000in}{0.528000in}}{\pgfqpoint{3.696000in}{3.696000in}}%
\pgfusepath{clip}%
\pgfsetbuttcap%
\pgfsetroundjoin%
\definecolor{currentfill}{rgb}{0.275827,0.366717,0.812553}%
\pgfsetfillcolor{currentfill}%
\pgfsetlinewidth{0.000000pt}%
\definecolor{currentstroke}{rgb}{0.000000,0.000000,0.000000}%
\pgfsetstrokecolor{currentstroke}%
\pgfsetdash{}{0pt}%
\pgfpathmoveto{\pgfqpoint{3.698857in}{2.080777in}}%
\pgfpathlineto{\pgfqpoint{3.743156in}{2.028934in}}%
\pgfpathlineto{\pgfqpoint{3.765972in}{1.998729in}}%
\pgfpathlineto{\pgfqpoint{3.721450in}{2.033226in}}%
\pgfpathlineto{\pgfqpoint{3.698857in}{2.080777in}}%
\pgfpathclose%
\pgfusepath{fill}%
\end{pgfscope}%
\begin{pgfscope}%
\pgfpathrectangle{\pgfqpoint{1.072000in}{0.528000in}}{\pgfqpoint{3.696000in}{3.696000in}}%
\pgfusepath{clip}%
\pgfsetbuttcap%
\pgfsetroundjoin%
\definecolor{currentfill}{rgb}{0.603162,0.731527,0.999565}%
\pgfsetfillcolor{currentfill}%
\pgfsetlinewidth{0.000000pt}%
\definecolor{currentstroke}{rgb}{0.000000,0.000000,0.000000}%
\pgfsetstrokecolor{currentstroke}%
\pgfsetdash{}{0pt}%
\pgfpathmoveto{\pgfqpoint{2.887592in}{2.507022in}}%
\pgfpathlineto{\pgfqpoint{2.931702in}{2.596574in}}%
\pgfpathlineto{\pgfqpoint{2.957436in}{2.482828in}}%
\pgfpathlineto{\pgfqpoint{2.913454in}{2.408326in}}%
\pgfpathlineto{\pgfqpoint{2.887592in}{2.507022in}}%
\pgfpathclose%
\pgfusepath{fill}%
\end{pgfscope}%
\begin{pgfscope}%
\pgfpathrectangle{\pgfqpoint{1.072000in}{0.528000in}}{\pgfqpoint{3.696000in}{3.696000in}}%
\pgfusepath{clip}%
\pgfsetbuttcap%
\pgfsetroundjoin%
\definecolor{currentfill}{rgb}{0.238948,0.312365,0.765676}%
\pgfsetfillcolor{currentfill}%
\pgfsetlinewidth{0.000000pt}%
\definecolor{currentstroke}{rgb}{0.000000,0.000000,0.000000}%
\pgfsetstrokecolor{currentstroke}%
\pgfsetdash{}{0pt}%
\pgfpathmoveto{\pgfqpoint{3.855731in}{1.952268in}}%
\pgfpathlineto{\pgfqpoint{3.901183in}{1.945928in}}%
\pgfpathlineto{\pgfqpoint{3.925903in}{2.004077in}}%
\pgfpathlineto{\pgfqpoint{3.879963in}{1.991732in}}%
\pgfpathlineto{\pgfqpoint{3.855731in}{1.952268in}}%
\pgfpathclose%
\pgfusepath{fill}%
\end{pgfscope}%
\begin{pgfscope}%
\pgfpathrectangle{\pgfqpoint{1.072000in}{0.528000in}}{\pgfqpoint{3.696000in}{3.696000in}}%
\pgfusepath{clip}%
\pgfsetbuttcap%
\pgfsetroundjoin%
\definecolor{currentfill}{rgb}{0.252663,0.332837,0.783665}%
\pgfsetfillcolor{currentfill}%
\pgfsetlinewidth{0.000000pt}%
\definecolor{currentstroke}{rgb}{0.000000,0.000000,0.000000}%
\pgfsetstrokecolor{currentstroke}%
\pgfsetdash{}{0pt}%
\pgfpathmoveto{\pgfqpoint{3.743156in}{2.028934in}}%
\pgfpathlineto{\pgfqpoint{3.787574in}{1.982177in}}%
\pgfpathlineto{\pgfqpoint{3.810699in}{1.970786in}}%
\pgfpathlineto{\pgfqpoint{3.765972in}{1.998729in}}%
\pgfpathlineto{\pgfqpoint{3.743156in}{2.028934in}}%
\pgfpathclose%
\pgfusepath{fill}%
\end{pgfscope}%
\begin{pgfscope}%
\pgfpathrectangle{\pgfqpoint{1.072000in}{0.528000in}}{\pgfqpoint{3.696000in}{3.696000in}}%
\pgfusepath{clip}%
\pgfsetbuttcap%
\pgfsetroundjoin%
\definecolor{currentfill}{rgb}{0.229806,0.298718,0.753683}%
\pgfsetfillcolor{currentfill}%
\pgfsetlinewidth{0.000000pt}%
\definecolor{currentstroke}{rgb}{0.000000,0.000000,0.000000}%
\pgfsetstrokecolor{currentstroke}%
\pgfsetdash{}{0pt}%
\pgfpathmoveto{\pgfqpoint{2.138475in}{1.947023in}}%
\pgfpathlineto{\pgfqpoint{2.181534in}{1.947659in}}%
\pgfpathlineto{\pgfqpoint{2.208563in}{1.971695in}}%
\pgfpathlineto{\pgfqpoint{2.165573in}{1.969838in}}%
\pgfpathlineto{\pgfqpoint{2.138475in}{1.947023in}}%
\pgfpathclose%
\pgfusepath{fill}%
\end{pgfscope}%
\begin{pgfscope}%
\pgfpathrectangle{\pgfqpoint{1.072000in}{0.528000in}}{\pgfqpoint{3.696000in}{3.696000in}}%
\pgfusepath{clip}%
\pgfsetbuttcap%
\pgfsetroundjoin%
\definecolor{currentfill}{rgb}{0.261805,0.346484,0.795658}%
\pgfsetfillcolor{currentfill}%
\pgfsetlinewidth{0.000000pt}%
\definecolor{currentstroke}{rgb}{0.000000,0.000000,0.000000}%
\pgfsetstrokecolor{currentstroke}%
\pgfsetdash{}{0pt}%
\pgfpathmoveto{\pgfqpoint{2.208563in}{1.971695in}}%
\pgfpathlineto{\pgfqpoint{2.250796in}{2.015942in}}%
\pgfpathlineto{\pgfqpoint{2.277784in}{2.036598in}}%
\pgfpathlineto{\pgfqpoint{2.235533in}{1.993641in}}%
\pgfpathlineto{\pgfqpoint{2.208563in}{1.971695in}}%
\pgfpathclose%
\pgfusepath{fill}%
\end{pgfscope}%
\begin{pgfscope}%
\pgfpathrectangle{\pgfqpoint{1.072000in}{0.528000in}}{\pgfqpoint{3.696000in}{3.696000in}}%
\pgfusepath{clip}%
\pgfsetbuttcap%
\pgfsetroundjoin%
\definecolor{currentfill}{rgb}{0.565182,0.699438,0.996635}%
\pgfsetfillcolor{currentfill}%
\pgfsetlinewidth{0.000000pt}%
\definecolor{currentstroke}{rgb}{0.000000,0.000000,0.000000}%
\pgfsetstrokecolor{currentstroke}%
\pgfsetdash{}{0pt}%
\pgfpathmoveto{\pgfqpoint{3.274828in}{2.495572in}}%
\pgfpathlineto{\pgfqpoint{3.319761in}{2.500348in}}%
\pgfpathlineto{\pgfqpoint{3.343027in}{2.367743in}}%
\pgfpathlineto{\pgfqpoint{3.298372in}{2.366442in}}%
\pgfpathlineto{\pgfqpoint{3.274828in}{2.495572in}}%
\pgfpathclose%
\pgfusepath{fill}%
\end{pgfscope}%
\begin{pgfscope}%
\pgfpathrectangle{\pgfqpoint{1.072000in}{0.528000in}}{\pgfqpoint{3.696000in}{3.696000in}}%
\pgfusepath{clip}%
\pgfsetbuttcap%
\pgfsetroundjoin%
\definecolor{currentfill}{rgb}{0.234377,0.305542,0.759680}%
\pgfsetfillcolor{currentfill}%
\pgfsetlinewidth{0.000000pt}%
\definecolor{currentstroke}{rgb}{0.000000,0.000000,0.000000}%
\pgfsetstrokecolor{currentstroke}%
\pgfsetdash{}{0pt}%
\pgfpathmoveto{\pgfqpoint{3.787574in}{1.982177in}}%
\pgfpathlineto{\pgfqpoint{3.832217in}{1.943831in}}%
\pgfpathlineto{\pgfqpoint{3.855731in}{1.952268in}}%
\pgfpathlineto{\pgfqpoint{3.810699in}{1.970786in}}%
\pgfpathlineto{\pgfqpoint{3.787574in}{1.982177in}}%
\pgfpathclose%
\pgfusepath{fill}%
\end{pgfscope}%
\begin{pgfscope}%
\pgfpathrectangle{\pgfqpoint{1.072000in}{0.528000in}}{\pgfqpoint{3.696000in}{3.696000in}}%
\pgfusepath{clip}%
\pgfsetbuttcap%
\pgfsetroundjoin%
\definecolor{currentfill}{rgb}{0.619318,0.744121,0.998931}%
\pgfsetfillcolor{currentfill}%
\pgfsetlinewidth{0.000000pt}%
\definecolor{currentstroke}{rgb}{0.000000,0.000000,0.000000}%
\pgfsetstrokecolor{currentstroke}%
\pgfsetdash{}{0pt}%
\pgfpathmoveto{\pgfqpoint{3.116049in}{2.547007in}}%
\pgfpathlineto{\pgfqpoint{3.160917in}{2.589796in}}%
\pgfpathlineto{\pgfqpoint{3.185198in}{2.454466in}}%
\pgfpathlineto{\pgfqpoint{3.140596in}{2.420997in}}%
\pgfpathlineto{\pgfqpoint{3.116049in}{2.547007in}}%
\pgfpathclose%
\pgfusepath{fill}%
\end{pgfscope}%
\begin{pgfscope}%
\pgfpathrectangle{\pgfqpoint{1.072000in}{0.528000in}}{\pgfqpoint{3.696000in}{3.696000in}}%
\pgfusepath{clip}%
\pgfsetbuttcap%
\pgfsetroundjoin%
\definecolor{currentfill}{rgb}{0.505423,0.643995,0.983157}%
\pgfsetfillcolor{currentfill}%
\pgfsetlinewidth{0.000000pt}%
\definecolor{currentstroke}{rgb}{0.000000,0.000000,0.000000}%
\pgfsetstrokecolor{currentstroke}%
\pgfsetdash{}{0pt}%
\pgfpathmoveto{\pgfqpoint{2.538650in}{2.306576in}}%
\pgfpathlineto{\pgfqpoint{2.580930in}{2.430607in}}%
\pgfpathlineto{\pgfqpoint{2.608033in}{2.383340in}}%
\pgfpathlineto{\pgfqpoint{2.565624in}{2.273452in}}%
\pgfpathlineto{\pgfqpoint{2.538650in}{2.306576in}}%
\pgfpathclose%
\pgfusepath{fill}%
\end{pgfscope}%
\begin{pgfscope}%
\pgfpathrectangle{\pgfqpoint{1.072000in}{0.528000in}}{\pgfqpoint{3.696000in}{3.696000in}}%
\pgfusepath{clip}%
\pgfsetbuttcap%
\pgfsetroundjoin%
\definecolor{currentfill}{rgb}{0.388852,0.516298,0.921373}%
\pgfsetfillcolor{currentfill}%
\pgfsetlinewidth{0.000000pt}%
\definecolor{currentstroke}{rgb}{0.000000,0.000000,0.000000}%
\pgfsetstrokecolor{currentstroke}%
\pgfsetdash{}{0pt}%
\pgfpathmoveto{\pgfqpoint{2.373619in}{2.137238in}}%
\pgfpathlineto{\pgfqpoint{2.415330in}{2.243160in}}%
\pgfpathlineto{\pgfqpoint{2.442469in}{2.235467in}}%
\pgfpathlineto{\pgfqpoint{2.400611in}{2.138352in}}%
\pgfpathlineto{\pgfqpoint{2.373619in}{2.137238in}}%
\pgfpathclose%
\pgfusepath{fill}%
\end{pgfscope}%
\begin{pgfscope}%
\pgfpathrectangle{\pgfqpoint{1.072000in}{0.528000in}}{\pgfqpoint{3.696000in}{3.696000in}}%
\pgfusepath{clip}%
\pgfsetbuttcap%
\pgfsetroundjoin%
\definecolor{currentfill}{rgb}{0.640828,0.760752,0.997846}%
\pgfsetfillcolor{currentfill}%
\pgfsetlinewidth{0.000000pt}%
\definecolor{currentstroke}{rgb}{0.000000,0.000000,0.000000}%
\pgfsetstrokecolor{currentstroke}%
\pgfsetdash{}{0pt}%
\pgfpathmoveto{\pgfqpoint{3.001736in}{2.556415in}}%
\pgfpathlineto{\pgfqpoint{3.046346in}{2.625224in}}%
\pgfpathlineto{\pgfqpoint{3.071399in}{2.495995in}}%
\pgfpathlineto{\pgfqpoint{3.027000in}{2.439443in}}%
\pgfpathlineto{\pgfqpoint{3.001736in}{2.556415in}}%
\pgfpathclose%
\pgfusepath{fill}%
\end{pgfscope}%
\begin{pgfscope}%
\pgfpathrectangle{\pgfqpoint{1.072000in}{0.528000in}}{\pgfqpoint{3.696000in}{3.696000in}}%
\pgfusepath{clip}%
\pgfsetbuttcap%
\pgfsetroundjoin%
\definecolor{currentfill}{rgb}{0.238948,0.312365,0.765676}%
\pgfsetfillcolor{currentfill}%
\pgfsetlinewidth{0.000000pt}%
\definecolor{currentstroke}{rgb}{0.000000,0.000000,0.000000}%
\pgfsetstrokecolor{currentstroke}%
\pgfsetdash{}{0pt}%
\pgfpathmoveto{\pgfqpoint{2.067067in}{1.970323in}}%
\pgfpathlineto{\pgfqpoint{2.111260in}{1.924672in}}%
\pgfpathlineto{\pgfqpoint{2.138475in}{1.947023in}}%
\pgfpathlineto{\pgfqpoint{2.094444in}{1.989437in}}%
\pgfpathlineto{\pgfqpoint{2.067067in}{1.970323in}}%
\pgfpathclose%
\pgfusepath{fill}%
\end{pgfscope}%
\begin{pgfscope}%
\pgfpathrectangle{\pgfqpoint{1.072000in}{0.528000in}}{\pgfqpoint{3.696000in}{3.696000in}}%
\pgfusepath{clip}%
\pgfsetbuttcap%
\pgfsetroundjoin%
\definecolor{currentfill}{rgb}{0.630089,0.752516,0.998508}%
\pgfsetfillcolor{currentfill}%
\pgfsetlinewidth{0.000000pt}%
\definecolor{currentstroke}{rgb}{0.000000,0.000000,0.000000}%
\pgfsetstrokecolor{currentstroke}%
\pgfsetdash{}{0pt}%
\pgfpathmoveto{\pgfqpoint{2.817596in}{2.509201in}}%
\pgfpathlineto{\pgfqpoint{2.861429in}{2.612884in}}%
\pgfpathlineto{\pgfqpoint{2.887592in}{2.507022in}}%
\pgfpathlineto{\pgfqpoint{2.843836in}{2.418500in}}%
\pgfpathlineto{\pgfqpoint{2.817596in}{2.509201in}}%
\pgfpathclose%
\pgfusepath{fill}%
\end{pgfscope}%
\begin{pgfscope}%
\pgfpathrectangle{\pgfqpoint{1.072000in}{0.528000in}}{\pgfqpoint{3.696000in}{3.696000in}}%
\pgfusepath{clip}%
\pgfsetbuttcap%
\pgfsetroundjoin%
\definecolor{currentfill}{rgb}{0.309060,0.413498,0.850128}%
\pgfsetfillcolor{currentfill}%
\pgfsetlinewidth{0.000000pt}%
\definecolor{currentstroke}{rgb}{0.000000,0.000000,0.000000}%
\pgfsetstrokecolor{currentstroke}%
\pgfsetdash{}{0pt}%
\pgfpathmoveto{\pgfqpoint{3.925903in}{2.004077in}}%
\pgfpathlineto{\pgfqpoint{3.972427in}{2.030601in}}%
\pgfpathlineto{\pgfqpoint{3.998470in}{2.133881in}}%
\pgfpathlineto{\pgfqpoint{3.951426in}{2.091716in}}%
\pgfpathlineto{\pgfqpoint{3.925903in}{2.004077in}}%
\pgfpathclose%
\pgfusepath{fill}%
\end{pgfscope}%
\begin{pgfscope}%
\pgfpathrectangle{\pgfqpoint{1.072000in}{0.528000in}}{\pgfqpoint{3.696000in}{3.696000in}}%
\pgfusepath{clip}%
\pgfsetbuttcap%
\pgfsetroundjoin%
\definecolor{currentfill}{rgb}{0.229806,0.298718,0.753683}%
\pgfsetfillcolor{currentfill}%
\pgfsetlinewidth{0.000000pt}%
\definecolor{currentstroke}{rgb}{0.000000,0.000000,0.000000}%
\pgfsetstrokecolor{currentstroke}%
\pgfsetdash{}{0pt}%
\pgfpathmoveto{\pgfqpoint{3.832217in}{1.943831in}}%
\pgfpathlineto{\pgfqpoint{3.877205in}{1.917177in}}%
\pgfpathlineto{\pgfqpoint{3.901183in}{1.945928in}}%
\pgfpathlineto{\pgfqpoint{3.855731in}{1.952268in}}%
\pgfpathlineto{\pgfqpoint{3.832217in}{1.943831in}}%
\pgfpathclose%
\pgfusepath{fill}%
\end{pgfscope}%
\begin{pgfscope}%
\pgfpathrectangle{\pgfqpoint{1.072000in}{0.528000in}}{\pgfqpoint{3.696000in}{3.696000in}}%
\pgfusepath{clip}%
\pgfsetbuttcap%
\pgfsetroundjoin%
\definecolor{currentfill}{rgb}{0.323718,0.433158,0.864722}%
\pgfsetfillcolor{currentfill}%
\pgfsetlinewidth{0.000000pt}%
\definecolor{currentstroke}{rgb}{0.000000,0.000000,0.000000}%
\pgfsetstrokecolor{currentstroke}%
\pgfsetdash{}{0pt}%
\pgfpathmoveto{\pgfqpoint{2.277784in}{2.036598in}}%
\pgfpathlineto{\pgfqpoint{2.319529in}{2.118785in}}%
\pgfpathlineto{\pgfqpoint{2.346581in}{2.131118in}}%
\pgfpathlineto{\pgfqpoint{2.304750in}{2.052869in}}%
\pgfpathlineto{\pgfqpoint{2.277784in}{2.036598in}}%
\pgfpathclose%
\pgfusepath{fill}%
\end{pgfscope}%
\begin{pgfscope}%
\pgfpathrectangle{\pgfqpoint{1.072000in}{0.528000in}}{\pgfqpoint{3.696000in}{3.696000in}}%
\pgfusepath{clip}%
\pgfsetbuttcap%
\pgfsetroundjoin%
\definecolor{currentfill}{rgb}{0.257234,0.339661,0.789661}%
\pgfsetfillcolor{currentfill}%
\pgfsetlinewidth{0.000000pt}%
\definecolor{currentstroke}{rgb}{0.000000,0.000000,0.000000}%
\pgfsetstrokecolor{currentstroke}%
\pgfsetdash{}{0pt}%
\pgfpathmoveto{\pgfqpoint{3.901183in}{1.945928in}}%
\pgfpathlineto{\pgfqpoint{3.947174in}{1.954224in}}%
\pgfpathlineto{\pgfqpoint{3.972427in}{2.030601in}}%
\pgfpathlineto{\pgfqpoint{3.925903in}{2.004077in}}%
\pgfpathlineto{\pgfqpoint{3.901183in}{1.945928in}}%
\pgfpathclose%
\pgfusepath{fill}%
\end{pgfscope}%
\begin{pgfscope}%
\pgfpathrectangle{\pgfqpoint{1.072000in}{0.528000in}}{\pgfqpoint{3.696000in}{3.696000in}}%
\pgfusepath{clip}%
\pgfsetbuttcap%
\pgfsetroundjoin%
\definecolor{currentfill}{rgb}{0.565182,0.699438,0.996635}%
\pgfsetfillcolor{currentfill}%
\pgfsetlinewidth{0.000000pt}%
\definecolor{currentstroke}{rgb}{0.000000,0.000000,0.000000}%
\pgfsetstrokecolor{currentstroke}%
\pgfsetdash{}{0pt}%
\pgfpathmoveto{\pgfqpoint{2.608033in}{2.383340in}}%
\pgfpathlineto{\pgfqpoint{2.650696in}{2.509899in}}%
\pgfpathlineto{\pgfqpoint{2.677698in}{2.445541in}}%
\pgfpathlineto{\pgfqpoint{2.634950in}{2.333787in}}%
\pgfpathlineto{\pgfqpoint{2.608033in}{2.383340in}}%
\pgfpathclose%
\pgfusepath{fill}%
\end{pgfscope}%
\begin{pgfscope}%
\pgfpathrectangle{\pgfqpoint{1.072000in}{0.528000in}}{\pgfqpoint{3.696000in}{3.696000in}}%
\pgfusepath{clip}%
\pgfsetbuttcap%
\pgfsetroundjoin%
\definecolor{currentfill}{rgb}{0.576051,0.708780,0.997755}%
\pgfsetfillcolor{currentfill}%
\pgfsetlinewidth{0.000000pt}%
\definecolor{currentstroke}{rgb}{0.000000,0.000000,0.000000}%
\pgfsetstrokecolor{currentstroke}%
\pgfsetdash{}{0pt}%
\pgfpathmoveto{\pgfqpoint{3.319761in}{2.500348in}}%
\pgfpathlineto{\pgfqpoint{3.364708in}{2.493507in}}%
\pgfpathlineto{\pgfqpoint{3.387703in}{2.359703in}}%
\pgfpathlineto{\pgfqpoint{3.343027in}{2.367743in}}%
\pgfpathlineto{\pgfqpoint{3.319761in}{2.500348in}}%
\pgfpathclose%
\pgfusepath{fill}%
\end{pgfscope}%
\begin{pgfscope}%
\pgfpathrectangle{\pgfqpoint{1.072000in}{0.528000in}}{\pgfqpoint{3.696000in}{3.696000in}}%
\pgfusepath{clip}%
\pgfsetbuttcap%
\pgfsetroundjoin%
\definecolor{currentfill}{rgb}{0.229806,0.298718,0.753683}%
\pgfsetfillcolor{currentfill}%
\pgfsetlinewidth{0.000000pt}%
\definecolor{currentstroke}{rgb}{0.000000,0.000000,0.000000}%
\pgfsetstrokecolor{currentstroke}%
\pgfsetdash{}{0pt}%
\pgfpathmoveto{\pgfqpoint{2.111260in}{1.924672in}}%
\pgfpathlineto{\pgfqpoint{2.154445in}{1.921732in}}%
\pgfpathlineto{\pgfqpoint{2.181534in}{1.947659in}}%
\pgfpathlineto{\pgfqpoint{2.138475in}{1.947023in}}%
\pgfpathlineto{\pgfqpoint{2.111260in}{1.924672in}}%
\pgfpathclose%
\pgfusepath{fill}%
\end{pgfscope}%
\begin{pgfscope}%
\pgfpathrectangle{\pgfqpoint{1.072000in}{0.528000in}}{\pgfqpoint{3.696000in}{3.696000in}}%
\pgfusepath{clip}%
\pgfsetbuttcap%
\pgfsetroundjoin%
\definecolor{currentfill}{rgb}{0.388852,0.516298,0.921373}%
\pgfsetfillcolor{currentfill}%
\pgfsetlinewidth{0.000000pt}%
\definecolor{currentstroke}{rgb}{0.000000,0.000000,0.000000}%
\pgfsetstrokecolor{currentstroke}%
\pgfsetdash{}{0pt}%
\pgfpathmoveto{\pgfqpoint{3.951426in}{2.091716in}}%
\pgfpathlineto{\pgfqpoint{3.998470in}{2.133881in}}%
\pgfpathlineto{\pgfqpoint{4.025296in}{2.262090in}}%
\pgfpathlineto{\pgfqpoint{3.977769in}{2.207386in}}%
\pgfpathlineto{\pgfqpoint{3.951426in}{2.091716in}}%
\pgfpathclose%
\pgfusepath{fill}%
\end{pgfscope}%
\begin{pgfscope}%
\pgfpathrectangle{\pgfqpoint{1.072000in}{0.528000in}}{\pgfqpoint{3.696000in}{3.696000in}}%
\pgfusepath{clip}%
\pgfsetbuttcap%
\pgfsetroundjoin%
\definecolor{currentfill}{rgb}{0.630089,0.752516,0.998508}%
\pgfsetfillcolor{currentfill}%
\pgfsetlinewidth{0.000000pt}%
\definecolor{currentstroke}{rgb}{0.000000,0.000000,0.000000}%
\pgfsetstrokecolor{currentstroke}%
\pgfsetdash{}{0pt}%
\pgfpathmoveto{\pgfqpoint{2.747588in}{2.488415in}}%
\pgfpathlineto{\pgfqpoint{2.791069in}{2.603622in}}%
\pgfpathlineto{\pgfqpoint{2.817596in}{2.509201in}}%
\pgfpathlineto{\pgfqpoint{2.774136in}{2.409169in}}%
\pgfpathlineto{\pgfqpoint{2.747588in}{2.488415in}}%
\pgfpathclose%
\pgfusepath{fill}%
\end{pgfscope}%
\begin{pgfscope}%
\pgfpathrectangle{\pgfqpoint{1.072000in}{0.528000in}}{\pgfqpoint{3.696000in}{3.696000in}}%
\pgfusepath{clip}%
\pgfsetbuttcap%
\pgfsetroundjoin%
\definecolor{currentfill}{rgb}{0.603162,0.731527,0.999565}%
\pgfsetfillcolor{currentfill}%
\pgfsetlinewidth{0.000000pt}%
\definecolor{currentstroke}{rgb}{0.000000,0.000000,0.000000}%
\pgfsetstrokecolor{currentstroke}%
\pgfsetdash{}{0pt}%
\pgfpathmoveto{\pgfqpoint{2.677698in}{2.445541in}}%
\pgfpathlineto{\pgfqpoint{2.720778in}{2.568746in}}%
\pgfpathlineto{\pgfqpoint{2.747588in}{2.488415in}}%
\pgfpathlineto{\pgfqpoint{2.704474in}{2.380303in}}%
\pgfpathlineto{\pgfqpoint{2.677698in}{2.445541in}}%
\pgfpathclose%
\pgfusepath{fill}%
\end{pgfscope}%
\begin{pgfscope}%
\pgfpathrectangle{\pgfqpoint{1.072000in}{0.528000in}}{\pgfqpoint{3.696000in}{3.696000in}}%
\pgfusepath{clip}%
\pgfsetbuttcap%
\pgfsetroundjoin%
\definecolor{currentfill}{rgb}{0.261805,0.346484,0.795658}%
\pgfsetfillcolor{currentfill}%
\pgfsetlinewidth{0.000000pt}%
\definecolor{currentstroke}{rgb}{0.000000,0.000000,0.000000}%
\pgfsetstrokecolor{currentstroke}%
\pgfsetdash{}{0pt}%
\pgfpathmoveto{\pgfqpoint{2.181534in}{1.947659in}}%
\pgfpathlineto{\pgfqpoint{2.223798in}{1.990655in}}%
\pgfpathlineto{\pgfqpoint{2.250796in}{2.015942in}}%
\pgfpathlineto{\pgfqpoint{2.208563in}{1.971695in}}%
\pgfpathlineto{\pgfqpoint{2.181534in}{1.947659in}}%
\pgfpathclose%
\pgfusepath{fill}%
\end{pgfscope}%
\begin{pgfscope}%
\pgfpathrectangle{\pgfqpoint{1.072000in}{0.528000in}}{\pgfqpoint{3.696000in}{3.696000in}}%
\pgfusepath{clip}%
\pgfsetbuttcap%
\pgfsetroundjoin%
\definecolor{currentfill}{rgb}{0.229806,0.298718,0.753683}%
\pgfsetfillcolor{currentfill}%
\pgfsetlinewidth{0.000000pt}%
\definecolor{currentstroke}{rgb}{0.000000,0.000000,0.000000}%
\pgfsetstrokecolor{currentstroke}%
\pgfsetdash{}{0pt}%
\pgfpathmoveto{\pgfqpoint{3.877205in}{1.917177in}}%
\pgfpathlineto{\pgfqpoint{3.922671in}{1.905263in}}%
\pgfpathlineto{\pgfqpoint{3.947174in}{1.954224in}}%
\pgfpathlineto{\pgfqpoint{3.901183in}{1.945928in}}%
\pgfpathlineto{\pgfqpoint{3.877205in}{1.917177in}}%
\pgfpathclose%
\pgfusepath{fill}%
\end{pgfscope}%
\begin{pgfscope}%
\pgfpathrectangle{\pgfqpoint{1.072000in}{0.528000in}}{\pgfqpoint{3.696000in}{3.696000in}}%
\pgfusepath{clip}%
\pgfsetbuttcap%
\pgfsetroundjoin%
\definecolor{currentfill}{rgb}{0.473070,0.611077,0.970634}%
\pgfsetfillcolor{currentfill}%
\pgfsetlinewidth{0.000000pt}%
\definecolor{currentstroke}{rgb}{0.000000,0.000000,0.000000}%
\pgfsetstrokecolor{currentstroke}%
\pgfsetdash{}{0pt}%
\pgfpathmoveto{\pgfqpoint{2.442469in}{2.235467in}}%
\pgfpathlineto{\pgfqpoint{2.484309in}{2.360562in}}%
\pgfpathlineto{\pgfqpoint{2.511531in}{2.336515in}}%
\pgfpathlineto{\pgfqpoint{2.469541in}{2.222039in}}%
\pgfpathlineto{\pgfqpoint{2.442469in}{2.235467in}}%
\pgfpathclose%
\pgfusepath{fill}%
\end{pgfscope}%
\begin{pgfscope}%
\pgfpathrectangle{\pgfqpoint{1.072000in}{0.528000in}}{\pgfqpoint{3.696000in}{3.696000in}}%
\pgfusepath{clip}%
\pgfsetbuttcap%
\pgfsetroundjoin%
\definecolor{currentfill}{rgb}{0.576051,0.708780,0.997755}%
\pgfsetfillcolor{currentfill}%
\pgfsetlinewidth{0.000000pt}%
\definecolor{currentstroke}{rgb}{0.000000,0.000000,0.000000}%
\pgfsetstrokecolor{currentstroke}%
\pgfsetdash{}{0pt}%
\pgfpathmoveto{\pgfqpoint{3.364708in}{2.493507in}}%
\pgfpathlineto{\pgfqpoint{3.409620in}{2.474799in}}%
\pgfpathlineto{\pgfqpoint{3.432363in}{2.342063in}}%
\pgfpathlineto{\pgfqpoint{3.387703in}{2.359703in}}%
\pgfpathlineto{\pgfqpoint{3.364708in}{2.493507in}}%
\pgfpathclose%
\pgfusepath{fill}%
\end{pgfscope}%
\begin{pgfscope}%
\pgfpathrectangle{\pgfqpoint{1.072000in}{0.528000in}}{\pgfqpoint{3.696000in}{3.696000in}}%
\pgfusepath{clip}%
\pgfsetbuttcap%
\pgfsetroundjoin%
\definecolor{currentfill}{rgb}{0.661968,0.775491,0.993937}%
\pgfsetfillcolor{currentfill}%
\pgfsetlinewidth{0.000000pt}%
\definecolor{currentstroke}{rgb}{0.000000,0.000000,0.000000}%
\pgfsetstrokecolor{currentstroke}%
\pgfsetdash{}{0pt}%
\pgfpathmoveto{\pgfqpoint{3.160917in}{2.589796in}}%
\pgfpathlineto{\pgfqpoint{3.205957in}{2.622259in}}%
\pgfpathlineto{\pgfqpoint{3.229957in}{2.479885in}}%
\pgfpathlineto{\pgfqpoint{3.185198in}{2.454466in}}%
\pgfpathlineto{\pgfqpoint{3.160917in}{2.589796in}}%
\pgfpathclose%
\pgfusepath{fill}%
\end{pgfscope}%
\begin{pgfscope}%
\pgfpathrectangle{\pgfqpoint{1.072000in}{0.528000in}}{\pgfqpoint{3.696000in}{3.696000in}}%
\pgfusepath{clip}%
\pgfsetbuttcap%
\pgfsetroundjoin%
\definecolor{currentfill}{rgb}{0.238948,0.312365,0.765676}%
\pgfsetfillcolor{currentfill}%
\pgfsetlinewidth{0.000000pt}%
\definecolor{currentstroke}{rgb}{0.000000,0.000000,0.000000}%
\pgfsetstrokecolor{currentstroke}%
\pgfsetdash{}{0pt}%
\pgfpathmoveto{\pgfqpoint{2.039493in}{1.954206in}}%
\pgfpathlineto{\pgfqpoint{2.083910in}{1.903314in}}%
\pgfpathlineto{\pgfqpoint{2.111260in}{1.924672in}}%
\pgfpathlineto{\pgfqpoint{2.067067in}{1.970323in}}%
\pgfpathlineto{\pgfqpoint{2.039493in}{1.954206in}}%
\pgfpathclose%
\pgfusepath{fill}%
\end{pgfscope}%
\begin{pgfscope}%
\pgfpathrectangle{\pgfqpoint{1.072000in}{0.528000in}}{\pgfqpoint{3.696000in}{3.696000in}}%
\pgfusepath{clip}%
\pgfsetbuttcap%
\pgfsetroundjoin%
\definecolor{currentfill}{rgb}{0.409611,0.540759,0.935545}%
\pgfsetfillcolor{currentfill}%
\pgfsetlinewidth{0.000000pt}%
\definecolor{currentstroke}{rgb}{0.000000,0.000000,0.000000}%
\pgfsetstrokecolor{currentstroke}%
\pgfsetdash{}{0pt}%
\pgfpathmoveto{\pgfqpoint{2.346581in}{2.131118in}}%
\pgfpathlineto{\pgfqpoint{2.388164in}{2.243398in}}%
\pgfpathlineto{\pgfqpoint{2.415330in}{2.243160in}}%
\pgfpathlineto{\pgfqpoint{2.373619in}{2.137238in}}%
\pgfpathlineto{\pgfqpoint{2.346581in}{2.131118in}}%
\pgfpathclose%
\pgfusepath{fill}%
\end{pgfscope}%
\begin{pgfscope}%
\pgfpathrectangle{\pgfqpoint{1.072000in}{0.528000in}}{\pgfqpoint{3.696000in}{3.696000in}}%
\pgfusepath{clip}%
\pgfsetbuttcap%
\pgfsetroundjoin%
\definecolor{currentfill}{rgb}{0.698454,0.799450,0.984577}%
\pgfsetfillcolor{currentfill}%
\pgfsetlinewidth{0.000000pt}%
\definecolor{currentstroke}{rgb}{0.000000,0.000000,0.000000}%
\pgfsetstrokecolor{currentstroke}%
\pgfsetdash{}{0pt}%
\pgfpathmoveto{\pgfqpoint{2.931702in}{2.596574in}}%
\pgfpathlineto{\pgfqpoint{2.976172in}{2.682563in}}%
\pgfpathlineto{\pgfqpoint{3.001736in}{2.556415in}}%
\pgfpathlineto{\pgfqpoint{2.957436in}{2.482828in}}%
\pgfpathlineto{\pgfqpoint{2.931702in}{2.596574in}}%
\pgfpathclose%
\pgfusepath{fill}%
\end{pgfscope}%
\begin{pgfscope}%
\pgfpathrectangle{\pgfqpoint{1.072000in}{0.528000in}}{\pgfqpoint{3.696000in}{3.696000in}}%
\pgfusepath{clip}%
\pgfsetbuttcap%
\pgfsetroundjoin%
\definecolor{currentfill}{rgb}{0.328604,0.439712,0.869587}%
\pgfsetfillcolor{currentfill}%
\pgfsetlinewidth{0.000000pt}%
\definecolor{currentstroke}{rgb}{0.000000,0.000000,0.000000}%
\pgfsetstrokecolor{currentstroke}%
\pgfsetdash{}{0pt}%
\pgfpathmoveto{\pgfqpoint{2.250796in}{2.015942in}}%
\pgfpathlineto{\pgfqpoint{2.292493in}{2.099494in}}%
\pgfpathlineto{\pgfqpoint{2.319529in}{2.118785in}}%
\pgfpathlineto{\pgfqpoint{2.277784in}{2.036598in}}%
\pgfpathlineto{\pgfqpoint{2.250796in}{2.015942in}}%
\pgfpathclose%
\pgfusepath{fill}%
\end{pgfscope}%
\begin{pgfscope}%
\pgfpathrectangle{\pgfqpoint{1.072000in}{0.528000in}}{\pgfqpoint{3.696000in}{3.696000in}}%
\pgfusepath{clip}%
\pgfsetbuttcap%
\pgfsetroundjoin%
\definecolor{currentfill}{rgb}{0.266381,0.353304,0.801637}%
\pgfsetfillcolor{currentfill}%
\pgfsetlinewidth{0.000000pt}%
\definecolor{currentstroke}{rgb}{0.000000,0.000000,0.000000}%
\pgfsetstrokecolor{currentstroke}%
\pgfsetdash{}{0pt}%
\pgfpathmoveto{\pgfqpoint{3.764995in}{2.022957in}}%
\pgfpathlineto{\pgfqpoint{3.809301in}{1.964211in}}%
\pgfpathlineto{\pgfqpoint{3.832217in}{1.943831in}}%
\pgfpathlineto{\pgfqpoint{3.787574in}{1.982177in}}%
\pgfpathlineto{\pgfqpoint{3.764995in}{2.022957in}}%
\pgfpathclose%
\pgfusepath{fill}%
\end{pgfscope}%
\begin{pgfscope}%
\pgfpathrectangle{\pgfqpoint{1.072000in}{0.528000in}}{\pgfqpoint{3.696000in}{3.696000in}}%
\pgfusepath{clip}%
\pgfsetbuttcap%
\pgfsetroundjoin%
\definecolor{currentfill}{rgb}{0.304174,0.406945,0.845263}%
\pgfsetfillcolor{currentfill}%
\pgfsetlinewidth{0.000000pt}%
\definecolor{currentstroke}{rgb}{0.000000,0.000000,0.000000}%
\pgfsetstrokecolor{currentstroke}%
\pgfsetdash{}{0pt}%
\pgfpathmoveto{\pgfqpoint{3.720825in}{2.088752in}}%
\pgfpathlineto{\pgfqpoint{3.764995in}{2.022957in}}%
\pgfpathlineto{\pgfqpoint{3.787574in}{1.982177in}}%
\pgfpathlineto{\pgfqpoint{3.743156in}{2.028934in}}%
\pgfpathlineto{\pgfqpoint{3.720825in}{2.088752in}}%
\pgfpathclose%
\pgfusepath{fill}%
\end{pgfscope}%
\begin{pgfscope}%
\pgfpathrectangle{\pgfqpoint{1.072000in}{0.528000in}}{\pgfqpoint{3.696000in}{3.696000in}}%
\pgfusepath{clip}%
\pgfsetbuttcap%
\pgfsetroundjoin%
\definecolor{currentfill}{rgb}{0.565182,0.699438,0.996635}%
\pgfsetfillcolor{currentfill}%
\pgfsetlinewidth{0.000000pt}%
\definecolor{currentstroke}{rgb}{0.000000,0.000000,0.000000}%
\pgfsetstrokecolor{currentstroke}%
\pgfsetdash{}{0pt}%
\pgfpathmoveto{\pgfqpoint{3.409620in}{2.474799in}}%
\pgfpathlineto{\pgfqpoint{3.454454in}{2.444422in}}%
\pgfpathlineto{\pgfqpoint{3.476972in}{2.315041in}}%
\pgfpathlineto{\pgfqpoint{3.432363in}{2.342063in}}%
\pgfpathlineto{\pgfqpoint{3.409620in}{2.474799in}}%
\pgfpathclose%
\pgfusepath{fill}%
\end{pgfscope}%
\begin{pgfscope}%
\pgfpathrectangle{\pgfqpoint{1.072000in}{0.528000in}}{\pgfqpoint{3.696000in}{3.696000in}}%
\pgfusepath{clip}%
\pgfsetbuttcap%
\pgfsetroundjoin%
\definecolor{currentfill}{rgb}{0.229806,0.298718,0.753683}%
\pgfsetfillcolor{currentfill}%
\pgfsetlinewidth{0.000000pt}%
\definecolor{currentstroke}{rgb}{0.000000,0.000000,0.000000}%
\pgfsetstrokecolor{currentstroke}%
\pgfsetdash{}{0pt}%
\pgfpathmoveto{\pgfqpoint{2.083910in}{1.903314in}}%
\pgfpathlineto{\pgfqpoint{2.127286in}{1.894516in}}%
\pgfpathlineto{\pgfqpoint{2.154445in}{1.921732in}}%
\pgfpathlineto{\pgfqpoint{2.111260in}{1.924672in}}%
\pgfpathlineto{\pgfqpoint{2.083910in}{1.903314in}}%
\pgfpathclose%
\pgfusepath{fill}%
\end{pgfscope}%
\begin{pgfscope}%
\pgfpathrectangle{\pgfqpoint{1.072000in}{0.528000in}}{\pgfqpoint{3.696000in}{3.696000in}}%
\pgfusepath{clip}%
\pgfsetbuttcap%
\pgfsetroundjoin%
\definecolor{currentfill}{rgb}{0.289996,0.386836,0.828926}%
\pgfsetfillcolor{currentfill}%
\pgfsetlinewidth{0.000000pt}%
\definecolor{currentstroke}{rgb}{0.000000,0.000000,0.000000}%
\pgfsetstrokecolor{currentstroke}%
\pgfsetdash{}{0pt}%
\pgfpathmoveto{\pgfqpoint{3.947174in}{1.954224in}}%
\pgfpathlineto{\pgfqpoint{3.993824in}{1.979153in}}%
\pgfpathlineto{\pgfqpoint{4.019637in}{2.072684in}}%
\pgfpathlineto{\pgfqpoint{3.972427in}{2.030601in}}%
\pgfpathlineto{\pgfqpoint{3.947174in}{1.954224in}}%
\pgfpathclose%
\pgfusepath{fill}%
\end{pgfscope}%
\begin{pgfscope}%
\pgfpathrectangle{\pgfqpoint{1.072000in}{0.528000in}}{\pgfqpoint{3.696000in}{3.696000in}}%
\pgfusepath{clip}%
\pgfsetbuttcap%
\pgfsetroundjoin%
\definecolor{currentfill}{rgb}{0.343278,0.459354,0.884122}%
\pgfsetfillcolor{currentfill}%
\pgfsetlinewidth{0.000000pt}%
\definecolor{currentstroke}{rgb}{0.000000,0.000000,0.000000}%
\pgfsetstrokecolor{currentstroke}%
\pgfsetdash{}{0pt}%
\pgfpathmoveto{\pgfqpoint{3.676684in}{2.157853in}}%
\pgfpathlineto{\pgfqpoint{3.720825in}{2.088752in}}%
\pgfpathlineto{\pgfqpoint{3.743156in}{2.028934in}}%
\pgfpathlineto{\pgfqpoint{3.698857in}{2.080777in}}%
\pgfpathlineto{\pgfqpoint{3.676684in}{2.157853in}}%
\pgfpathclose%
\pgfusepath{fill}%
\end{pgfscope}%
\begin{pgfscope}%
\pgfpathrectangle{\pgfqpoint{1.072000in}{0.528000in}}{\pgfqpoint{3.696000in}{3.696000in}}%
\pgfusepath{clip}%
\pgfsetbuttcap%
\pgfsetroundjoin%
\definecolor{currentfill}{rgb}{0.243520,0.319189,0.771672}%
\pgfsetfillcolor{currentfill}%
\pgfsetlinewidth{0.000000pt}%
\definecolor{currentstroke}{rgb}{0.000000,0.000000,0.000000}%
\pgfsetstrokecolor{currentstroke}%
\pgfsetdash{}{0pt}%
\pgfpathmoveto{\pgfqpoint{3.809301in}{1.964211in}}%
\pgfpathlineto{\pgfqpoint{3.853867in}{1.916300in}}%
\pgfpathlineto{\pgfqpoint{3.877205in}{1.917177in}}%
\pgfpathlineto{\pgfqpoint{3.832217in}{1.943831in}}%
\pgfpathlineto{\pgfqpoint{3.809301in}{1.964211in}}%
\pgfpathclose%
\pgfusepath{fill}%
\end{pgfscope}%
\begin{pgfscope}%
\pgfpathrectangle{\pgfqpoint{1.072000in}{0.528000in}}{\pgfqpoint{3.696000in}{3.696000in}}%
\pgfusepath{clip}%
\pgfsetbuttcap%
\pgfsetroundjoin%
\definecolor{currentfill}{rgb}{0.257234,0.339661,0.789661}%
\pgfsetfillcolor{currentfill}%
\pgfsetlinewidth{0.000000pt}%
\definecolor{currentstroke}{rgb}{0.000000,0.000000,0.000000}%
\pgfsetstrokecolor{currentstroke}%
\pgfsetdash{}{0pt}%
\pgfpathmoveto{\pgfqpoint{2.154445in}{1.921732in}}%
\pgfpathlineto{\pgfqpoint{2.196799in}{1.960947in}}%
\pgfpathlineto{\pgfqpoint{2.223798in}{1.990655in}}%
\pgfpathlineto{\pgfqpoint{2.181534in}{1.947659in}}%
\pgfpathlineto{\pgfqpoint{2.154445in}{1.921732in}}%
\pgfpathclose%
\pgfusepath{fill}%
\end{pgfscope}%
\begin{pgfscope}%
\pgfpathrectangle{\pgfqpoint{1.072000in}{0.528000in}}{\pgfqpoint{3.696000in}{3.696000in}}%
\pgfusepath{clip}%
\pgfsetbuttcap%
\pgfsetroundjoin%
\definecolor{currentfill}{rgb}{0.358415,0.478426,0.896795}%
\pgfsetfillcolor{currentfill}%
\pgfsetlinewidth{0.000000pt}%
\definecolor{currentstroke}{rgb}{0.000000,0.000000,0.000000}%
\pgfsetstrokecolor{currentstroke}%
\pgfsetdash{}{0pt}%
\pgfpathmoveto{\pgfqpoint{3.972427in}{2.030601in}}%
\pgfpathlineto{\pgfqpoint{4.019637in}{2.072684in}}%
\pgfpathlineto{\pgfqpoint{4.046208in}{2.190217in}}%
\pgfpathlineto{\pgfqpoint{3.998470in}{2.133881in}}%
\pgfpathlineto{\pgfqpoint{3.972427in}{2.030601in}}%
\pgfpathclose%
\pgfusepath{fill}%
\end{pgfscope}%
\begin{pgfscope}%
\pgfpathrectangle{\pgfqpoint{1.072000in}{0.528000in}}{\pgfqpoint{3.696000in}{3.696000in}}%
\pgfusepath{clip}%
\pgfsetbuttcap%
\pgfsetroundjoin%
\definecolor{currentfill}{rgb}{0.388852,0.516298,0.921373}%
\pgfsetfillcolor{currentfill}%
\pgfsetlinewidth{0.000000pt}%
\definecolor{currentstroke}{rgb}{0.000000,0.000000,0.000000}%
\pgfsetstrokecolor{currentstroke}%
\pgfsetdash{}{0pt}%
\pgfpathmoveto{\pgfqpoint{3.632492in}{2.226747in}}%
\pgfpathlineto{\pgfqpoint{3.676684in}{2.157853in}}%
\pgfpathlineto{\pgfqpoint{3.698857in}{2.080777in}}%
\pgfpathlineto{\pgfqpoint{3.654595in}{2.134499in}}%
\pgfpathlineto{\pgfqpoint{3.632492in}{2.226747in}}%
\pgfpathclose%
\pgfusepath{fill}%
\end{pgfscope}%
\begin{pgfscope}%
\pgfpathrectangle{\pgfqpoint{1.072000in}{0.528000in}}{\pgfqpoint{3.696000in}{3.696000in}}%
\pgfusepath{clip}%
\pgfsetbuttcap%
\pgfsetroundjoin%
\definecolor{currentfill}{rgb}{0.708720,0.805721,0.981117}%
\pgfsetfillcolor{currentfill}%
\pgfsetlinewidth{0.000000pt}%
\definecolor{currentstroke}{rgb}{0.000000,0.000000,0.000000}%
\pgfsetstrokecolor{currentstroke}%
\pgfsetdash{}{0pt}%
\pgfpathmoveto{\pgfqpoint{3.046346in}{2.625224in}}%
\pgfpathlineto{\pgfqpoint{3.091235in}{2.686021in}}%
\pgfpathlineto{\pgfqpoint{3.116049in}{2.547007in}}%
\pgfpathlineto{\pgfqpoint{3.071399in}{2.495995in}}%
\pgfpathlineto{\pgfqpoint{3.046346in}{2.625224in}}%
\pgfpathclose%
\pgfusepath{fill}%
\end{pgfscope}%
\begin{pgfscope}%
\pgfpathrectangle{\pgfqpoint{1.072000in}{0.528000in}}{\pgfqpoint{3.696000in}{3.696000in}}%
\pgfusepath{clip}%
\pgfsetbuttcap%
\pgfsetroundjoin%
\definecolor{currentfill}{rgb}{0.543440,0.680003,0.993051}%
\pgfsetfillcolor{currentfill}%
\pgfsetlinewidth{0.000000pt}%
\definecolor{currentstroke}{rgb}{0.000000,0.000000,0.000000}%
\pgfsetstrokecolor{currentstroke}%
\pgfsetdash{}{0pt}%
\pgfpathmoveto{\pgfqpoint{3.454454in}{2.444422in}}%
\pgfpathlineto{\pgfqpoint{3.499174in}{2.403032in}}%
\pgfpathlineto{\pgfqpoint{3.521505in}{2.279339in}}%
\pgfpathlineto{\pgfqpoint{3.476972in}{2.315041in}}%
\pgfpathlineto{\pgfqpoint{3.454454in}{2.444422in}}%
\pgfpathclose%
\pgfusepath{fill}%
\end{pgfscope}%
\begin{pgfscope}%
\pgfpathrectangle{\pgfqpoint{1.072000in}{0.528000in}}{\pgfqpoint{3.696000in}{3.696000in}}%
\pgfusepath{clip}%
\pgfsetbuttcap%
\pgfsetroundjoin%
\definecolor{currentfill}{rgb}{0.248091,0.326013,0.777669}%
\pgfsetfillcolor{currentfill}%
\pgfsetlinewidth{0.000000pt}%
\definecolor{currentstroke}{rgb}{0.000000,0.000000,0.000000}%
\pgfsetstrokecolor{currentstroke}%
\pgfsetdash{}{0pt}%
\pgfpathmoveto{\pgfqpoint{3.922671in}{1.905263in}}%
\pgfpathlineto{\pgfqpoint{3.968752in}{1.910712in}}%
\pgfpathlineto{\pgfqpoint{3.993824in}{1.979153in}}%
\pgfpathlineto{\pgfqpoint{3.947174in}{1.954224in}}%
\pgfpathlineto{\pgfqpoint{3.922671in}{1.905263in}}%
\pgfpathclose%
\pgfusepath{fill}%
\end{pgfscope}%
\begin{pgfscope}%
\pgfpathrectangle{\pgfqpoint{1.072000in}{0.528000in}}{\pgfqpoint{3.696000in}{3.696000in}}%
\pgfusepath{clip}%
\pgfsetbuttcap%
\pgfsetroundjoin%
\definecolor{currentfill}{rgb}{0.430507,0.564883,0.948889}%
\pgfsetfillcolor{currentfill}%
\pgfsetlinewidth{0.000000pt}%
\definecolor{currentstroke}{rgb}{0.000000,0.000000,0.000000}%
\pgfsetstrokecolor{currentstroke}%
\pgfsetdash{}{0pt}%
\pgfpathmoveto{\pgfqpoint{3.588193in}{2.292281in}}%
\pgfpathlineto{\pgfqpoint{3.632492in}{2.226747in}}%
\pgfpathlineto{\pgfqpoint{3.654595in}{2.134499in}}%
\pgfpathlineto{\pgfqpoint{3.610305in}{2.187152in}}%
\pgfpathlineto{\pgfqpoint{3.588193in}{2.292281in}}%
\pgfpathclose%
\pgfusepath{fill}%
\end{pgfscope}%
\begin{pgfscope}%
\pgfpathrectangle{\pgfqpoint{1.072000in}{0.528000in}}{\pgfqpoint{3.696000in}{3.696000in}}%
\pgfusepath{clip}%
\pgfsetbuttcap%
\pgfsetroundjoin%
\definecolor{currentfill}{rgb}{0.565182,0.699438,0.996635}%
\pgfsetfillcolor{currentfill}%
\pgfsetlinewidth{0.000000pt}%
\definecolor{currentstroke}{rgb}{0.000000,0.000000,0.000000}%
\pgfsetstrokecolor{currentstroke}%
\pgfsetdash{}{0pt}%
\pgfpathmoveto{\pgfqpoint{2.511531in}{2.336515in}}%
\pgfpathlineto{\pgfqpoint{2.553684in}{2.472329in}}%
\pgfpathlineto{\pgfqpoint{2.580930in}{2.430607in}}%
\pgfpathlineto{\pgfqpoint{2.538650in}{2.306576in}}%
\pgfpathlineto{\pgfqpoint{2.511531in}{2.336515in}}%
\pgfpathclose%
\pgfusepath{fill}%
\end{pgfscope}%
\begin{pgfscope}%
\pgfpathrectangle{\pgfqpoint{1.072000in}{0.528000in}}{\pgfqpoint{3.696000in}{3.696000in}}%
\pgfusepath{clip}%
\pgfsetbuttcap%
\pgfsetroundjoin%
\definecolor{currentfill}{rgb}{0.229806,0.298718,0.753683}%
\pgfsetfillcolor{currentfill}%
\pgfsetlinewidth{0.000000pt}%
\definecolor{currentstroke}{rgb}{0.000000,0.000000,0.000000}%
\pgfsetstrokecolor{currentstroke}%
\pgfsetdash{}{0pt}%
\pgfpathmoveto{\pgfqpoint{3.853867in}{1.916300in}}%
\pgfpathlineto{\pgfqpoint{3.898838in}{1.882856in}}%
\pgfpathlineto{\pgfqpoint{3.922671in}{1.905263in}}%
\pgfpathlineto{\pgfqpoint{3.877205in}{1.917177in}}%
\pgfpathlineto{\pgfqpoint{3.853867in}{1.916300in}}%
\pgfpathclose%
\pgfusepath{fill}%
\end{pgfscope}%
\begin{pgfscope}%
\pgfpathrectangle{\pgfqpoint{1.072000in}{0.528000in}}{\pgfqpoint{3.696000in}{3.696000in}}%
\pgfusepath{clip}%
\pgfsetbuttcap%
\pgfsetroundjoin%
\definecolor{currentfill}{rgb}{0.510824,0.649397,0.985079}%
\pgfsetfillcolor{currentfill}%
\pgfsetlinewidth{0.000000pt}%
\definecolor{currentstroke}{rgb}{0.000000,0.000000,0.000000}%
\pgfsetstrokecolor{currentstroke}%
\pgfsetdash{}{0pt}%
\pgfpathmoveto{\pgfqpoint{3.499174in}{2.403032in}}%
\pgfpathlineto{\pgfqpoint{3.543757in}{2.351768in}}%
\pgfpathlineto{\pgfqpoint{3.565949in}{2.236150in}}%
\pgfpathlineto{\pgfqpoint{3.521505in}{2.279339in}}%
\pgfpathlineto{\pgfqpoint{3.499174in}{2.403032in}}%
\pgfpathclose%
\pgfusepath{fill}%
\end{pgfscope}%
\begin{pgfscope}%
\pgfpathrectangle{\pgfqpoint{1.072000in}{0.528000in}}{\pgfqpoint{3.696000in}{3.696000in}}%
\pgfusepath{clip}%
\pgfsetbuttcap%
\pgfsetroundjoin%
\definecolor{currentfill}{rgb}{0.473070,0.611077,0.970634}%
\pgfsetfillcolor{currentfill}%
\pgfsetlinewidth{0.000000pt}%
\definecolor{currentstroke}{rgb}{0.000000,0.000000,0.000000}%
\pgfsetstrokecolor{currentstroke}%
\pgfsetdash{}{0pt}%
\pgfpathmoveto{\pgfqpoint{3.543757in}{2.351768in}}%
\pgfpathlineto{\pgfqpoint{3.588193in}{2.292281in}}%
\pgfpathlineto{\pgfqpoint{3.610305in}{2.187152in}}%
\pgfpathlineto{\pgfqpoint{3.565949in}{2.236150in}}%
\pgfpathlineto{\pgfqpoint{3.543757in}{2.351768in}}%
\pgfpathclose%
\pgfusepath{fill}%
\end{pgfscope}%
\begin{pgfscope}%
\pgfpathrectangle{\pgfqpoint{1.072000in}{0.528000in}}{\pgfqpoint{3.696000in}{3.696000in}}%
\pgfusepath{clip}%
\pgfsetbuttcap%
\pgfsetroundjoin%
\definecolor{currentfill}{rgb}{0.451739,0.588181,0.960201}%
\pgfsetfillcolor{currentfill}%
\pgfsetlinewidth{0.000000pt}%
\definecolor{currentstroke}{rgb}{0.000000,0.000000,0.000000}%
\pgfsetstrokecolor{currentstroke}%
\pgfsetdash{}{0pt}%
\pgfpathmoveto{\pgfqpoint{3.998470in}{2.133881in}}%
\pgfpathlineto{\pgfqpoint{4.046208in}{2.190217in}}%
\pgfpathlineto{\pgfqpoint{4.073506in}{2.329291in}}%
\pgfpathlineto{\pgfqpoint{4.025296in}{2.262090in}}%
\pgfpathlineto{\pgfqpoint{3.998470in}{2.133881in}}%
\pgfpathclose%
\pgfusepath{fill}%
\end{pgfscope}%
\begin{pgfscope}%
\pgfpathrectangle{\pgfqpoint{1.072000in}{0.528000in}}{\pgfqpoint{3.696000in}{3.696000in}}%
\pgfusepath{clip}%
\pgfsetbuttcap%
\pgfsetroundjoin%
\definecolor{currentfill}{rgb}{0.243520,0.319189,0.771672}%
\pgfsetfillcolor{currentfill}%
\pgfsetlinewidth{0.000000pt}%
\definecolor{currentstroke}{rgb}{0.000000,0.000000,0.000000}%
\pgfsetstrokecolor{currentstroke}%
\pgfsetdash{}{0pt}%
\pgfpathmoveto{\pgfqpoint{2.011697in}{1.941792in}}%
\pgfpathlineto{\pgfqpoint{2.056400in}{1.883803in}}%
\pgfpathlineto{\pgfqpoint{2.083910in}{1.903314in}}%
\pgfpathlineto{\pgfqpoint{2.039493in}{1.954206in}}%
\pgfpathlineto{\pgfqpoint{2.011697in}{1.941792in}}%
\pgfpathclose%
\pgfusepath{fill}%
\end{pgfscope}%
\begin{pgfscope}%
\pgfpathrectangle{\pgfqpoint{1.072000in}{0.528000in}}{\pgfqpoint{3.696000in}{3.696000in}}%
\pgfusepath{clip}%
\pgfsetbuttcap%
\pgfsetroundjoin%
\definecolor{currentfill}{rgb}{0.698454,0.799450,0.984577}%
\pgfsetfillcolor{currentfill}%
\pgfsetlinewidth{0.000000pt}%
\definecolor{currentstroke}{rgb}{0.000000,0.000000,0.000000}%
\pgfsetstrokecolor{currentstroke}%
\pgfsetdash{}{0pt}%
\pgfpathmoveto{\pgfqpoint{3.205957in}{2.622259in}}%
\pgfpathlineto{\pgfqpoint{3.251117in}{2.642852in}}%
\pgfpathlineto{\pgfqpoint{3.274828in}{2.495572in}}%
\pgfpathlineto{\pgfqpoint{3.229957in}{2.479885in}}%
\pgfpathlineto{\pgfqpoint{3.205957in}{2.622259in}}%
\pgfpathclose%
\pgfusepath{fill}%
\end{pgfscope}%
\begin{pgfscope}%
\pgfpathrectangle{\pgfqpoint{1.072000in}{0.528000in}}{\pgfqpoint{3.696000in}{3.696000in}}%
\pgfusepath{clip}%
\pgfsetbuttcap%
\pgfsetroundjoin%
\definecolor{currentfill}{rgb}{0.728970,0.817464,0.973188}%
\pgfsetfillcolor{currentfill}%
\pgfsetlinewidth{0.000000pt}%
\definecolor{currentstroke}{rgb}{0.000000,0.000000,0.000000}%
\pgfsetstrokecolor{currentstroke}%
\pgfsetdash{}{0pt}%
\pgfpathmoveto{\pgfqpoint{2.861429in}{2.612884in}}%
\pgfpathlineto{\pgfqpoint{2.905660in}{2.714794in}}%
\pgfpathlineto{\pgfqpoint{2.931702in}{2.596574in}}%
\pgfpathlineto{\pgfqpoint{2.887592in}{2.507022in}}%
\pgfpathlineto{\pgfqpoint{2.861429in}{2.612884in}}%
\pgfpathclose%
\pgfusepath{fill}%
\end{pgfscope}%
\begin{pgfscope}%
\pgfpathrectangle{\pgfqpoint{1.072000in}{0.528000in}}{\pgfqpoint{3.696000in}{3.696000in}}%
\pgfusepath{clip}%
\pgfsetbuttcap%
\pgfsetroundjoin%
\definecolor{currentfill}{rgb}{0.328604,0.439712,0.869587}%
\pgfsetfillcolor{currentfill}%
\pgfsetlinewidth{0.000000pt}%
\definecolor{currentstroke}{rgb}{0.000000,0.000000,0.000000}%
\pgfsetstrokecolor{currentstroke}%
\pgfsetdash{}{0pt}%
\pgfpathmoveto{\pgfqpoint{2.223798in}{1.990655in}}%
\pgfpathlineto{\pgfqpoint{2.265493in}{2.072979in}}%
\pgfpathlineto{\pgfqpoint{2.292493in}{2.099494in}}%
\pgfpathlineto{\pgfqpoint{2.250796in}{2.015942in}}%
\pgfpathlineto{\pgfqpoint{2.223798in}{1.990655in}}%
\pgfpathclose%
\pgfusepath{fill}%
\end{pgfscope}%
\begin{pgfscope}%
\pgfpathrectangle{\pgfqpoint{1.072000in}{0.528000in}}{\pgfqpoint{3.696000in}{3.696000in}}%
\pgfusepath{clip}%
\pgfsetbuttcap%
\pgfsetroundjoin%
\definecolor{currentfill}{rgb}{0.229806,0.298718,0.753683}%
\pgfsetfillcolor{currentfill}%
\pgfsetlinewidth{0.000000pt}%
\definecolor{currentstroke}{rgb}{0.000000,0.000000,0.000000}%
\pgfsetstrokecolor{currentstroke}%
\pgfsetdash{}{0pt}%
\pgfpathmoveto{\pgfqpoint{3.898838in}{1.882856in}}%
\pgfpathlineto{\pgfqpoint{3.944364in}{1.867139in}}%
\pgfpathlineto{\pgfqpoint{3.968752in}{1.910712in}}%
\pgfpathlineto{\pgfqpoint{3.922671in}{1.905263in}}%
\pgfpathlineto{\pgfqpoint{3.898838in}{1.882856in}}%
\pgfpathclose%
\pgfusepath{fill}%
\end{pgfscope}%
\begin{pgfscope}%
\pgfpathrectangle{\pgfqpoint{1.072000in}{0.528000in}}{\pgfqpoint{3.696000in}{3.696000in}}%
\pgfusepath{clip}%
\pgfsetbuttcap%
\pgfsetroundjoin%
\definecolor{currentfill}{rgb}{0.510824,0.649397,0.985079}%
\pgfsetfillcolor{currentfill}%
\pgfsetlinewidth{0.000000pt}%
\definecolor{currentstroke}{rgb}{0.000000,0.000000,0.000000}%
\pgfsetstrokecolor{currentstroke}%
\pgfsetdash{}{0pt}%
\pgfpathmoveto{\pgfqpoint{2.415330in}{2.243160in}}%
\pgfpathlineto{\pgfqpoint{2.457032in}{2.376504in}}%
\pgfpathlineto{\pgfqpoint{2.484309in}{2.360562in}}%
\pgfpathlineto{\pgfqpoint{2.442469in}{2.235467in}}%
\pgfpathlineto{\pgfqpoint{2.415330in}{2.243160in}}%
\pgfpathclose%
\pgfusepath{fill}%
\end{pgfscope}%
\begin{pgfscope}%
\pgfpathrectangle{\pgfqpoint{1.072000in}{0.528000in}}{\pgfqpoint{3.696000in}{3.696000in}}%
\pgfusepath{clip}%
\pgfsetbuttcap%
\pgfsetroundjoin%
\definecolor{currentfill}{rgb}{0.229806,0.298718,0.753683}%
\pgfsetfillcolor{currentfill}%
\pgfsetlinewidth{0.000000pt}%
\definecolor{currentstroke}{rgb}{0.000000,0.000000,0.000000}%
\pgfsetstrokecolor{currentstroke}%
\pgfsetdash{}{0pt}%
\pgfpathmoveto{\pgfqpoint{2.056400in}{1.883803in}}%
\pgfpathlineto{\pgfqpoint{2.100034in}{1.867003in}}%
\pgfpathlineto{\pgfqpoint{2.127286in}{1.894516in}}%
\pgfpathlineto{\pgfqpoint{2.083910in}{1.903314in}}%
\pgfpathlineto{\pgfqpoint{2.056400in}{1.883803in}}%
\pgfpathclose%
\pgfusepath{fill}%
\end{pgfscope}%
\begin{pgfscope}%
\pgfpathrectangle{\pgfqpoint{1.072000in}{0.528000in}}{\pgfqpoint{3.696000in}{3.696000in}}%
\pgfusepath{clip}%
\pgfsetbuttcap%
\pgfsetroundjoin%
\definecolor{currentfill}{rgb}{0.252663,0.332837,0.783665}%
\pgfsetfillcolor{currentfill}%
\pgfsetlinewidth{0.000000pt}%
\definecolor{currentstroke}{rgb}{0.000000,0.000000,0.000000}%
\pgfsetstrokecolor{currentstroke}%
\pgfsetdash{}{0pt}%
\pgfpathmoveto{\pgfqpoint{2.127286in}{1.894516in}}%
\pgfpathlineto{\pgfqpoint{2.169791in}{1.927482in}}%
\pgfpathlineto{\pgfqpoint{2.196799in}{1.960947in}}%
\pgfpathlineto{\pgfqpoint{2.154445in}{1.921732in}}%
\pgfpathlineto{\pgfqpoint{2.127286in}{1.894516in}}%
\pgfpathclose%
\pgfusepath{fill}%
\end{pgfscope}%
\begin{pgfscope}%
\pgfpathrectangle{\pgfqpoint{1.072000in}{0.528000in}}{\pgfqpoint{3.696000in}{3.696000in}}%
\pgfusepath{clip}%
\pgfsetbuttcap%
\pgfsetroundjoin%
\definecolor{currentfill}{rgb}{0.425199,0.559058,0.946061}%
\pgfsetfillcolor{currentfill}%
\pgfsetlinewidth{0.000000pt}%
\definecolor{currentstroke}{rgb}{0.000000,0.000000,0.000000}%
\pgfsetstrokecolor{currentstroke}%
\pgfsetdash{}{0pt}%
\pgfpathmoveto{\pgfqpoint{2.319529in}{2.118785in}}%
\pgfpathlineto{\pgfqpoint{2.361011in}{2.234938in}}%
\pgfpathlineto{\pgfqpoint{2.388164in}{2.243398in}}%
\pgfpathlineto{\pgfqpoint{2.346581in}{2.131118in}}%
\pgfpathlineto{\pgfqpoint{2.319529in}{2.118785in}}%
\pgfpathclose%
\pgfusepath{fill}%
\end{pgfscope}%
\begin{pgfscope}%
\pgfpathrectangle{\pgfqpoint{1.072000in}{0.528000in}}{\pgfqpoint{3.696000in}{3.696000in}}%
\pgfusepath{clip}%
\pgfsetbuttcap%
\pgfsetroundjoin%
\definecolor{currentfill}{rgb}{0.640828,0.760752,0.997846}%
\pgfsetfillcolor{currentfill}%
\pgfsetlinewidth{0.000000pt}%
\definecolor{currentstroke}{rgb}{0.000000,0.000000,0.000000}%
\pgfsetstrokecolor{currentstroke}%
\pgfsetdash{}{0pt}%
\pgfpathmoveto{\pgfqpoint{2.580930in}{2.430607in}}%
\pgfpathlineto{\pgfqpoint{2.623506in}{2.569503in}}%
\pgfpathlineto{\pgfqpoint{2.650696in}{2.509899in}}%
\pgfpathlineto{\pgfqpoint{2.608033in}{2.383340in}}%
\pgfpathlineto{\pgfqpoint{2.580930in}{2.430607in}}%
\pgfpathclose%
\pgfusepath{fill}%
\end{pgfscope}%
\begin{pgfscope}%
\pgfpathrectangle{\pgfqpoint{1.072000in}{0.528000in}}{\pgfqpoint{3.696000in}{3.696000in}}%
\pgfusepath{clip}%
\pgfsetbuttcap%
\pgfsetroundjoin%
\definecolor{currentfill}{rgb}{0.280550,0.373423,0.818011}%
\pgfsetfillcolor{currentfill}%
\pgfsetlinewidth{0.000000pt}%
\definecolor{currentstroke}{rgb}{0.000000,0.000000,0.000000}%
\pgfsetstrokecolor{currentstroke}%
\pgfsetdash{}{0pt}%
\pgfpathmoveto{\pgfqpoint{3.968752in}{1.910712in}}%
\pgfpathlineto{\pgfqpoint{4.015579in}{1.935528in}}%
\pgfpathlineto{\pgfqpoint{4.041242in}{2.022075in}}%
\pgfpathlineto{\pgfqpoint{3.993824in}{1.979153in}}%
\pgfpathlineto{\pgfqpoint{3.968752in}{1.910712in}}%
\pgfpathclose%
\pgfusepath{fill}%
\end{pgfscope}%
\begin{pgfscope}%
\pgfpathrectangle{\pgfqpoint{1.072000in}{0.528000in}}{\pgfqpoint{3.696000in}{3.696000in}}%
\pgfusepath{clip}%
\pgfsetbuttcap%
\pgfsetroundjoin%
\definecolor{currentfill}{rgb}{0.338377,0.452819,0.879317}%
\pgfsetfillcolor{currentfill}%
\pgfsetlinewidth{0.000000pt}%
\definecolor{currentstroke}{rgb}{0.000000,0.000000,0.000000}%
\pgfsetstrokecolor{currentstroke}%
\pgfsetdash{}{0pt}%
\pgfpathmoveto{\pgfqpoint{3.993824in}{1.979153in}}%
\pgfpathlineto{\pgfqpoint{4.041242in}{2.022075in}}%
\pgfpathlineto{\pgfqpoint{4.067619in}{2.131068in}}%
\pgfpathlineto{\pgfqpoint{4.019637in}{2.072684in}}%
\pgfpathlineto{\pgfqpoint{3.993824in}{1.979153in}}%
\pgfpathclose%
\pgfusepath{fill}%
\end{pgfscope}%
\begin{pgfscope}%
\pgfpathrectangle{\pgfqpoint{1.072000in}{0.528000in}}{\pgfqpoint{3.696000in}{3.696000in}}%
\pgfusepath{clip}%
\pgfsetbuttcap%
\pgfsetroundjoin%
\definecolor{currentfill}{rgb}{0.743754,0.825125,0.965798}%
\pgfsetfillcolor{currentfill}%
\pgfsetlinewidth{0.000000pt}%
\definecolor{currentstroke}{rgb}{0.000000,0.000000,0.000000}%
\pgfsetstrokecolor{currentstroke}%
\pgfsetdash{}{0pt}%
\pgfpathmoveto{\pgfqpoint{2.791069in}{2.603622in}}%
\pgfpathlineto{\pgfqpoint{2.834972in}{2.719632in}}%
\pgfpathlineto{\pgfqpoint{2.861429in}{2.612884in}}%
\pgfpathlineto{\pgfqpoint{2.817596in}{2.509201in}}%
\pgfpathlineto{\pgfqpoint{2.791069in}{2.603622in}}%
\pgfpathclose%
\pgfusepath{fill}%
\end{pgfscope}%
\begin{pgfscope}%
\pgfpathrectangle{\pgfqpoint{1.072000in}{0.528000in}}{\pgfqpoint{3.696000in}{3.696000in}}%
\pgfusepath{clip}%
\pgfsetbuttcap%
\pgfsetroundjoin%
\definecolor{currentfill}{rgb}{0.698454,0.799450,0.984577}%
\pgfsetfillcolor{currentfill}%
\pgfsetlinewidth{0.000000pt}%
\definecolor{currentstroke}{rgb}{0.000000,0.000000,0.000000}%
\pgfsetstrokecolor{currentstroke}%
\pgfsetdash{}{0pt}%
\pgfpathmoveto{\pgfqpoint{2.650696in}{2.509899in}}%
\pgfpathlineto{\pgfqpoint{2.693739in}{2.645564in}}%
\pgfpathlineto{\pgfqpoint{2.720778in}{2.568746in}}%
\pgfpathlineto{\pgfqpoint{2.677698in}{2.445541in}}%
\pgfpathlineto{\pgfqpoint{2.650696in}{2.509899in}}%
\pgfpathclose%
\pgfusepath{fill}%
\end{pgfscope}%
\begin{pgfscope}%
\pgfpathrectangle{\pgfqpoint{1.072000in}{0.528000in}}{\pgfqpoint{3.696000in}{3.696000in}}%
\pgfusepath{clip}%
\pgfsetbuttcap%
\pgfsetroundjoin%
\definecolor{currentfill}{rgb}{0.728970,0.817464,0.973188}%
\pgfsetfillcolor{currentfill}%
\pgfsetlinewidth{0.000000pt}%
\definecolor{currentstroke}{rgb}{0.000000,0.000000,0.000000}%
\pgfsetstrokecolor{currentstroke}%
\pgfsetdash{}{0pt}%
\pgfpathmoveto{\pgfqpoint{2.720778in}{2.568746in}}%
\pgfpathlineto{\pgfqpoint{2.764276in}{2.696340in}}%
\pgfpathlineto{\pgfqpoint{2.791069in}{2.603622in}}%
\pgfpathlineto{\pgfqpoint{2.747588in}{2.488415in}}%
\pgfpathlineto{\pgfqpoint{2.720778in}{2.568746in}}%
\pgfpathclose%
\pgfusepath{fill}%
\end{pgfscope}%
\begin{pgfscope}%
\pgfpathrectangle{\pgfqpoint{1.072000in}{0.528000in}}{\pgfqpoint{3.696000in}{3.696000in}}%
\pgfusepath{clip}%
\pgfsetbuttcap%
\pgfsetroundjoin%
\definecolor{currentfill}{rgb}{0.718985,0.811993,0.977656}%
\pgfsetfillcolor{currentfill}%
\pgfsetlinewidth{0.000000pt}%
\definecolor{currentstroke}{rgb}{0.000000,0.000000,0.000000}%
\pgfsetstrokecolor{currentstroke}%
\pgfsetdash{}{0pt}%
\pgfpathmoveto{\pgfqpoint{3.251117in}{2.642852in}}%
\pgfpathlineto{\pgfqpoint{3.296344in}{2.650535in}}%
\pgfpathlineto{\pgfqpoint{3.319761in}{2.500348in}}%
\pgfpathlineto{\pgfqpoint{3.274828in}{2.495572in}}%
\pgfpathlineto{\pgfqpoint{3.251117in}{2.642852in}}%
\pgfpathclose%
\pgfusepath{fill}%
\end{pgfscope}%
\begin{pgfscope}%
\pgfpathrectangle{\pgfqpoint{1.072000in}{0.528000in}}{\pgfqpoint{3.696000in}{3.696000in}}%
\pgfusepath{clip}%
\pgfsetbuttcap%
\pgfsetroundjoin%
\definecolor{currentfill}{rgb}{0.763363,0.835092,0.955658}%
\pgfsetfillcolor{currentfill}%
\pgfsetlinewidth{0.000000pt}%
\definecolor{currentstroke}{rgb}{0.000000,0.000000,0.000000}%
\pgfsetstrokecolor{currentstroke}%
\pgfsetdash{}{0pt}%
\pgfpathmoveto{\pgfqpoint{3.091235in}{2.686021in}}%
\pgfpathlineto{\pgfqpoint{3.136363in}{2.736266in}}%
\pgfpathlineto{\pgfqpoint{3.160917in}{2.589796in}}%
\pgfpathlineto{\pgfqpoint{3.116049in}{2.547007in}}%
\pgfpathlineto{\pgfqpoint{3.091235in}{2.686021in}}%
\pgfpathclose%
\pgfusepath{fill}%
\end{pgfscope}%
\begin{pgfscope}%
\pgfpathrectangle{\pgfqpoint{1.072000in}{0.528000in}}{\pgfqpoint{3.696000in}{3.696000in}}%
\pgfusepath{clip}%
\pgfsetbuttcap%
\pgfsetroundjoin%
\definecolor{currentfill}{rgb}{0.248091,0.326013,0.777669}%
\pgfsetfillcolor{currentfill}%
\pgfsetlinewidth{0.000000pt}%
\definecolor{currentstroke}{rgb}{0.000000,0.000000,0.000000}%
\pgfsetstrokecolor{currentstroke}%
\pgfsetdash{}{0pt}%
\pgfpathmoveto{\pgfqpoint{3.831042in}{1.940501in}}%
\pgfpathlineto{\pgfqpoint{3.875564in}{1.884914in}}%
\pgfpathlineto{\pgfqpoint{3.898838in}{1.882856in}}%
\pgfpathlineto{\pgfqpoint{3.853867in}{1.916300in}}%
\pgfpathlineto{\pgfqpoint{3.831042in}{1.940501in}}%
\pgfpathclose%
\pgfusepath{fill}%
\end{pgfscope}%
\begin{pgfscope}%
\pgfpathrectangle{\pgfqpoint{1.072000in}{0.528000in}}{\pgfqpoint{3.696000in}{3.696000in}}%
\pgfusepath{clip}%
\pgfsetbuttcap%
\pgfsetroundjoin%
\definecolor{currentfill}{rgb}{0.280550,0.373423,0.818011}%
\pgfsetfillcolor{currentfill}%
\pgfsetlinewidth{0.000000pt}%
\definecolor{currentstroke}{rgb}{0.000000,0.000000,0.000000}%
\pgfsetstrokecolor{currentstroke}%
\pgfsetdash{}{0pt}%
\pgfpathmoveto{\pgfqpoint{3.786839in}{2.009888in}}%
\pgfpathlineto{\pgfqpoint{3.831042in}{1.940501in}}%
\pgfpathlineto{\pgfqpoint{3.853867in}{1.916300in}}%
\pgfpathlineto{\pgfqpoint{3.809301in}{1.964211in}}%
\pgfpathlineto{\pgfqpoint{3.786839in}{2.009888in}}%
\pgfpathclose%
\pgfusepath{fill}%
\end{pgfscope}%
\begin{pgfscope}%
\pgfpathrectangle{\pgfqpoint{1.072000in}{0.528000in}}{\pgfqpoint{3.696000in}{3.696000in}}%
\pgfusepath{clip}%
\pgfsetbuttcap%
\pgfsetroundjoin%
\definecolor{currentfill}{rgb}{0.777378,0.840921,0.946149}%
\pgfsetfillcolor{currentfill}%
\pgfsetlinewidth{0.000000pt}%
\definecolor{currentstroke}{rgb}{0.000000,0.000000,0.000000}%
\pgfsetstrokecolor{currentstroke}%
\pgfsetdash{}{0pt}%
\pgfpathmoveto{\pgfqpoint{2.976172in}{2.682563in}}%
\pgfpathlineto{\pgfqpoint{3.020982in}{2.761076in}}%
\pgfpathlineto{\pgfqpoint{3.046346in}{2.625224in}}%
\pgfpathlineto{\pgfqpoint{3.001736in}{2.556415in}}%
\pgfpathlineto{\pgfqpoint{2.976172in}{2.682563in}}%
\pgfpathclose%
\pgfusepath{fill}%
\end{pgfscope}%
\begin{pgfscope}%
\pgfpathrectangle{\pgfqpoint{1.072000in}{0.528000in}}{\pgfqpoint{3.696000in}{3.696000in}}%
\pgfusepath{clip}%
\pgfsetbuttcap%
\pgfsetroundjoin%
\definecolor{currentfill}{rgb}{0.425199,0.559058,0.946061}%
\pgfsetfillcolor{currentfill}%
\pgfsetlinewidth{0.000000pt}%
\definecolor{currentstroke}{rgb}{0.000000,0.000000,0.000000}%
\pgfsetstrokecolor{currentstroke}%
\pgfsetdash{}{0pt}%
\pgfpathmoveto{\pgfqpoint{4.019637in}{2.072684in}}%
\pgfpathlineto{\pgfqpoint{4.067619in}{2.131068in}}%
\pgfpathlineto{\pgfqpoint{4.094703in}{2.260877in}}%
\pgfpathlineto{\pgfqpoint{4.046208in}{2.190217in}}%
\pgfpathlineto{\pgfqpoint{4.019637in}{2.072684in}}%
\pgfpathclose%
\pgfusepath{fill}%
\end{pgfscope}%
\begin{pgfscope}%
\pgfpathrectangle{\pgfqpoint{1.072000in}{0.528000in}}{\pgfqpoint{3.696000in}{3.696000in}}%
\pgfusepath{clip}%
\pgfsetbuttcap%
\pgfsetroundjoin%
\definecolor{currentfill}{rgb}{0.243520,0.319189,0.771672}%
\pgfsetfillcolor{currentfill}%
\pgfsetlinewidth{0.000000pt}%
\definecolor{currentstroke}{rgb}{0.000000,0.000000,0.000000}%
\pgfsetstrokecolor{currentstroke}%
\pgfsetdash{}{0pt}%
\pgfpathmoveto{\pgfqpoint{3.944364in}{1.867139in}}%
\pgfpathlineto{\pgfqpoint{3.990598in}{1.871814in}}%
\pgfpathlineto{\pgfqpoint{4.015579in}{1.935528in}}%
\pgfpathlineto{\pgfqpoint{3.968752in}{1.910712in}}%
\pgfpathlineto{\pgfqpoint{3.944364in}{1.867139in}}%
\pgfpathclose%
\pgfusepath{fill}%
\end{pgfscope}%
\begin{pgfscope}%
\pgfpathrectangle{\pgfqpoint{1.072000in}{0.528000in}}{\pgfqpoint{3.696000in}{3.696000in}}%
\pgfusepath{clip}%
\pgfsetbuttcap%
\pgfsetroundjoin%
\definecolor{currentfill}{rgb}{0.323718,0.433158,0.864722}%
\pgfsetfillcolor{currentfill}%
\pgfsetlinewidth{0.000000pt}%
\definecolor{currentstroke}{rgb}{0.000000,0.000000,0.000000}%
\pgfsetstrokecolor{currentstroke}%
\pgfsetdash{}{0pt}%
\pgfpathmoveto{\pgfqpoint{2.196799in}{1.960947in}}%
\pgfpathlineto{\pgfqpoint{2.238542in}{2.039452in}}%
\pgfpathlineto{\pgfqpoint{2.265493in}{2.072979in}}%
\pgfpathlineto{\pgfqpoint{2.223798in}{1.990655in}}%
\pgfpathlineto{\pgfqpoint{2.196799in}{1.960947in}}%
\pgfpathclose%
\pgfusepath{fill}%
\end{pgfscope}%
\begin{pgfscope}%
\pgfpathrectangle{\pgfqpoint{1.072000in}{0.528000in}}{\pgfqpoint{3.696000in}{3.696000in}}%
\pgfusepath{clip}%
\pgfsetbuttcap%
\pgfsetroundjoin%
\definecolor{currentfill}{rgb}{0.252663,0.332837,0.783665}%
\pgfsetfillcolor{currentfill}%
\pgfsetlinewidth{0.000000pt}%
\definecolor{currentstroke}{rgb}{0.000000,0.000000,0.000000}%
\pgfsetstrokecolor{currentstroke}%
\pgfsetdash{}{0pt}%
\pgfpathmoveto{\pgfqpoint{1.983639in}{1.934010in}}%
\pgfpathlineto{\pgfqpoint{2.028692in}{1.867284in}}%
\pgfpathlineto{\pgfqpoint{2.056400in}{1.883803in}}%
\pgfpathlineto{\pgfqpoint{2.011697in}{1.941792in}}%
\pgfpathlineto{\pgfqpoint{1.983639in}{1.934010in}}%
\pgfpathclose%
\pgfusepath{fill}%
\end{pgfscope}%
\begin{pgfscope}%
\pgfpathrectangle{\pgfqpoint{1.072000in}{0.528000in}}{\pgfqpoint{3.696000in}{3.696000in}}%
\pgfusepath{clip}%
\pgfsetbuttcap%
\pgfsetroundjoin%
\definecolor{currentfill}{rgb}{0.243520,0.319189,0.771672}%
\pgfsetfillcolor{currentfill}%
\pgfsetlinewidth{0.000000pt}%
\definecolor{currentstroke}{rgb}{0.000000,0.000000,0.000000}%
\pgfsetstrokecolor{currentstroke}%
\pgfsetdash{}{0pt}%
\pgfpathmoveto{\pgfqpoint{2.100034in}{1.867003in}}%
\pgfpathlineto{\pgfqpoint{2.142756in}{1.891360in}}%
\pgfpathlineto{\pgfqpoint{2.169791in}{1.927482in}}%
\pgfpathlineto{\pgfqpoint{2.127286in}{1.894516in}}%
\pgfpathlineto{\pgfqpoint{2.100034in}{1.867003in}}%
\pgfpathclose%
\pgfusepath{fill}%
\end{pgfscope}%
\begin{pgfscope}%
\pgfpathrectangle{\pgfqpoint{1.072000in}{0.528000in}}{\pgfqpoint{3.696000in}{3.696000in}}%
\pgfusepath{clip}%
\pgfsetbuttcap%
\pgfsetroundjoin%
\definecolor{currentfill}{rgb}{0.328604,0.439712,0.869587}%
\pgfsetfillcolor{currentfill}%
\pgfsetlinewidth{0.000000pt}%
\definecolor{currentstroke}{rgb}{0.000000,0.000000,0.000000}%
\pgfsetstrokecolor{currentstroke}%
\pgfsetdash{}{0pt}%
\pgfpathmoveto{\pgfqpoint{3.742804in}{2.088868in}}%
\pgfpathlineto{\pgfqpoint{3.786839in}{2.009888in}}%
\pgfpathlineto{\pgfqpoint{3.809301in}{1.964211in}}%
\pgfpathlineto{\pgfqpoint{3.764995in}{2.022957in}}%
\pgfpathlineto{\pgfqpoint{3.742804in}{2.088868in}}%
\pgfpathclose%
\pgfusepath{fill}%
\end{pgfscope}%
\begin{pgfscope}%
\pgfpathrectangle{\pgfqpoint{1.072000in}{0.528000in}}{\pgfqpoint{3.696000in}{3.696000in}}%
\pgfusepath{clip}%
\pgfsetbuttcap%
\pgfsetroundjoin%
\definecolor{currentfill}{rgb}{0.229806,0.298718,0.753683}%
\pgfsetfillcolor{currentfill}%
\pgfsetlinewidth{0.000000pt}%
\definecolor{currentstroke}{rgb}{0.000000,0.000000,0.000000}%
\pgfsetstrokecolor{currentstroke}%
\pgfsetdash{}{0pt}%
\pgfpathmoveto{\pgfqpoint{3.875564in}{1.884914in}}%
\pgfpathlineto{\pgfqpoint{3.920569in}{1.847028in}}%
\pgfpathlineto{\pgfqpoint{3.944364in}{1.867139in}}%
\pgfpathlineto{\pgfqpoint{3.898838in}{1.882856in}}%
\pgfpathlineto{\pgfqpoint{3.875564in}{1.884914in}}%
\pgfpathclose%
\pgfusepath{fill}%
\end{pgfscope}%
\begin{pgfscope}%
\pgfpathrectangle{\pgfqpoint{1.072000in}{0.528000in}}{\pgfqpoint{3.696000in}{3.696000in}}%
\pgfusepath{clip}%
\pgfsetbuttcap%
\pgfsetroundjoin%
\definecolor{currentfill}{rgb}{0.229806,0.298718,0.753683}%
\pgfsetfillcolor{currentfill}%
\pgfsetlinewidth{0.000000pt}%
\definecolor{currentstroke}{rgb}{0.000000,0.000000,0.000000}%
\pgfsetstrokecolor{currentstroke}%
\pgfsetdash{}{0pt}%
\pgfpathmoveto{\pgfqpoint{2.028692in}{1.867284in}}%
\pgfpathlineto{\pgfqpoint{2.072652in}{1.840535in}}%
\pgfpathlineto{\pgfqpoint{2.100034in}{1.867003in}}%
\pgfpathlineto{\pgfqpoint{2.056400in}{1.883803in}}%
\pgfpathlineto{\pgfqpoint{2.028692in}{1.867284in}}%
\pgfpathclose%
\pgfusepath{fill}%
\end{pgfscope}%
\begin{pgfscope}%
\pgfpathrectangle{\pgfqpoint{1.072000in}{0.528000in}}{\pgfqpoint{3.696000in}{3.696000in}}%
\pgfusepath{clip}%
\pgfsetbuttcap%
\pgfsetroundjoin%
\definecolor{currentfill}{rgb}{0.619318,0.744121,0.998931}%
\pgfsetfillcolor{currentfill}%
\pgfsetlinewidth{0.000000pt}%
\definecolor{currentstroke}{rgb}{0.000000,0.000000,0.000000}%
\pgfsetstrokecolor{currentstroke}%
\pgfsetdash{}{0pt}%
\pgfpathmoveto{\pgfqpoint{2.484309in}{2.360562in}}%
\pgfpathlineto{\pgfqpoint{2.526340in}{2.505800in}}%
\pgfpathlineto{\pgfqpoint{2.553684in}{2.472329in}}%
\pgfpathlineto{\pgfqpoint{2.511531in}{2.336515in}}%
\pgfpathlineto{\pgfqpoint{2.484309in}{2.360562in}}%
\pgfpathclose%
\pgfusepath{fill}%
\end{pgfscope}%
\begin{pgfscope}%
\pgfpathrectangle{\pgfqpoint{1.072000in}{0.528000in}}{\pgfqpoint{3.696000in}{3.696000in}}%
\pgfusepath{clip}%
\pgfsetbuttcap%
\pgfsetroundjoin%
\definecolor{currentfill}{rgb}{0.430507,0.564883,0.948889}%
\pgfsetfillcolor{currentfill}%
\pgfsetlinewidth{0.000000pt}%
\definecolor{currentstroke}{rgb}{0.000000,0.000000,0.000000}%
\pgfsetstrokecolor{currentstroke}%
\pgfsetdash{}{0pt}%
\pgfpathmoveto{\pgfqpoint{2.292493in}{2.099494in}}%
\pgfpathlineto{\pgfqpoint{2.333905in}{2.217018in}}%
\pgfpathlineto{\pgfqpoint{2.361011in}{2.234938in}}%
\pgfpathlineto{\pgfqpoint{2.319529in}{2.118785in}}%
\pgfpathlineto{\pgfqpoint{2.292493in}{2.099494in}}%
\pgfpathclose%
\pgfusepath{fill}%
\end{pgfscope}%
\begin{pgfscope}%
\pgfpathrectangle{\pgfqpoint{1.072000in}{0.528000in}}{\pgfqpoint{3.696000in}{3.696000in}}%
\pgfusepath{clip}%
\pgfsetbuttcap%
\pgfsetroundjoin%
\definecolor{currentfill}{rgb}{0.378598,0.503856,0.913692}%
\pgfsetfillcolor{currentfill}%
\pgfsetlinewidth{0.000000pt}%
\definecolor{currentstroke}{rgb}{0.000000,0.000000,0.000000}%
\pgfsetstrokecolor{currentstroke}%
\pgfsetdash{}{0pt}%
\pgfpathmoveto{\pgfqpoint{3.698809in}{2.173173in}}%
\pgfpathlineto{\pgfqpoint{3.742804in}{2.088868in}}%
\pgfpathlineto{\pgfqpoint{3.764995in}{2.022957in}}%
\pgfpathlineto{\pgfqpoint{3.720825in}{2.088752in}}%
\pgfpathlineto{\pgfqpoint{3.698809in}{2.173173in}}%
\pgfpathclose%
\pgfusepath{fill}%
\end{pgfscope}%
\begin{pgfscope}%
\pgfpathrectangle{\pgfqpoint{1.072000in}{0.528000in}}{\pgfqpoint{3.696000in}{3.696000in}}%
\pgfusepath{clip}%
\pgfsetbuttcap%
\pgfsetroundjoin%
\definecolor{currentfill}{rgb}{0.543440,0.680003,0.993051}%
\pgfsetfillcolor{currentfill}%
\pgfsetlinewidth{0.000000pt}%
\definecolor{currentstroke}{rgb}{0.000000,0.000000,0.000000}%
\pgfsetstrokecolor{currentstroke}%
\pgfsetdash{}{0pt}%
\pgfpathmoveto{\pgfqpoint{2.388164in}{2.243398in}}%
\pgfpathlineto{\pgfqpoint{2.429742in}{2.382636in}}%
\pgfpathlineto{\pgfqpoint{2.457032in}{2.376504in}}%
\pgfpathlineto{\pgfqpoint{2.415330in}{2.243160in}}%
\pgfpathlineto{\pgfqpoint{2.388164in}{2.243398in}}%
\pgfpathclose%
\pgfusepath{fill}%
\end{pgfscope}%
\begin{pgfscope}%
\pgfpathrectangle{\pgfqpoint{1.072000in}{0.528000in}}{\pgfqpoint{3.696000in}{3.696000in}}%
\pgfusepath{clip}%
\pgfsetbuttcap%
\pgfsetroundjoin%
\definecolor{currentfill}{rgb}{0.532568,0.669801,0.990393}%
\pgfsetfillcolor{currentfill}%
\pgfsetlinewidth{0.000000pt}%
\definecolor{currentstroke}{rgb}{0.000000,0.000000,0.000000}%
\pgfsetstrokecolor{currentstroke}%
\pgfsetdash{}{0pt}%
\pgfpathmoveto{\pgfqpoint{4.046208in}{2.190217in}}%
\pgfpathlineto{\pgfqpoint{4.094703in}{2.260877in}}%
\pgfpathlineto{\pgfqpoint{4.122437in}{2.408593in}}%
\pgfpathlineto{\pgfqpoint{4.073506in}{2.329291in}}%
\pgfpathlineto{\pgfqpoint{4.046208in}{2.190217in}}%
\pgfpathclose%
\pgfusepath{fill}%
\end{pgfscope}%
\begin{pgfscope}%
\pgfpathrectangle{\pgfqpoint{1.072000in}{0.528000in}}{\pgfqpoint{3.696000in}{3.696000in}}%
\pgfusepath{clip}%
\pgfsetbuttcap%
\pgfsetroundjoin%
\definecolor{currentfill}{rgb}{0.728970,0.817464,0.973188}%
\pgfsetfillcolor{currentfill}%
\pgfsetlinewidth{0.000000pt}%
\definecolor{currentstroke}{rgb}{0.000000,0.000000,0.000000}%
\pgfsetstrokecolor{currentstroke}%
\pgfsetdash{}{0pt}%
\pgfpathmoveto{\pgfqpoint{3.296344in}{2.650535in}}%
\pgfpathlineto{\pgfqpoint{3.341581in}{2.644705in}}%
\pgfpathlineto{\pgfqpoint{3.364708in}{2.493507in}}%
\pgfpathlineto{\pgfqpoint{3.319761in}{2.500348in}}%
\pgfpathlineto{\pgfqpoint{3.296344in}{2.650535in}}%
\pgfpathclose%
\pgfusepath{fill}%
\end{pgfscope}%
\begin{pgfscope}%
\pgfpathrectangle{\pgfqpoint{1.072000in}{0.528000in}}{\pgfqpoint{3.696000in}{3.696000in}}%
\pgfusepath{clip}%
\pgfsetbuttcap%
\pgfsetroundjoin%
\definecolor{currentfill}{rgb}{0.229806,0.298718,0.753683}%
\pgfsetfillcolor{currentfill}%
\pgfsetlinewidth{0.000000pt}%
\definecolor{currentstroke}{rgb}{0.000000,0.000000,0.000000}%
\pgfsetstrokecolor{currentstroke}%
\pgfsetdash{}{0pt}%
\pgfpathmoveto{\pgfqpoint{3.920569in}{1.847028in}}%
\pgfpathlineto{\pgfqpoint{3.966228in}{1.830187in}}%
\pgfpathlineto{\pgfqpoint{3.990598in}{1.871814in}}%
\pgfpathlineto{\pgfqpoint{3.944364in}{1.867139in}}%
\pgfpathlineto{\pgfqpoint{3.920569in}{1.847028in}}%
\pgfpathclose%
\pgfusepath{fill}%
\end{pgfscope}%
\begin{pgfscope}%
\pgfpathrectangle{\pgfqpoint{1.072000in}{0.528000in}}{\pgfqpoint{3.696000in}{3.696000in}}%
\pgfusepath{clip}%
\pgfsetbuttcap%
\pgfsetroundjoin%
\definecolor{currentfill}{rgb}{0.441123,0.576532,0.954545}%
\pgfsetfillcolor{currentfill}%
\pgfsetlinewidth{0.000000pt}%
\definecolor{currentstroke}{rgb}{0.000000,0.000000,0.000000}%
\pgfsetstrokecolor{currentstroke}%
\pgfsetdash{}{0pt}%
\pgfpathmoveto{\pgfqpoint{3.654751in}{2.258697in}}%
\pgfpathlineto{\pgfqpoint{3.698809in}{2.173173in}}%
\pgfpathlineto{\pgfqpoint{3.720825in}{2.088752in}}%
\pgfpathlineto{\pgfqpoint{3.676684in}{2.157853in}}%
\pgfpathlineto{\pgfqpoint{3.654751in}{2.258697in}}%
\pgfpathclose%
\pgfusepath{fill}%
\end{pgfscope}%
\begin{pgfscope}%
\pgfpathrectangle{\pgfqpoint{1.072000in}{0.528000in}}{\pgfqpoint{3.696000in}{3.696000in}}%
\pgfusepath{clip}%
\pgfsetbuttcap%
\pgfsetroundjoin%
\definecolor{currentfill}{rgb}{0.313946,0.420052,0.854993}%
\pgfsetfillcolor{currentfill}%
\pgfsetlinewidth{0.000000pt}%
\definecolor{currentstroke}{rgb}{0.000000,0.000000,0.000000}%
\pgfsetstrokecolor{currentstroke}%
\pgfsetdash{}{0pt}%
\pgfpathmoveto{\pgfqpoint{2.169791in}{1.927482in}}%
\pgfpathlineto{\pgfqpoint{2.211641in}{1.999611in}}%
\pgfpathlineto{\pgfqpoint{2.238542in}{2.039452in}}%
\pgfpathlineto{\pgfqpoint{2.196799in}{1.960947in}}%
\pgfpathlineto{\pgfqpoint{2.169791in}{1.927482in}}%
\pgfpathclose%
\pgfusepath{fill}%
\end{pgfscope}%
\begin{pgfscope}%
\pgfpathrectangle{\pgfqpoint{1.072000in}{0.528000in}}{\pgfqpoint{3.696000in}{3.696000in}}%
\pgfusepath{clip}%
\pgfsetbuttcap%
\pgfsetroundjoin%
\definecolor{currentfill}{rgb}{0.238948,0.312365,0.765676}%
\pgfsetfillcolor{currentfill}%
\pgfsetlinewidth{0.000000pt}%
\definecolor{currentstroke}{rgb}{0.000000,0.000000,0.000000}%
\pgfsetstrokecolor{currentstroke}%
\pgfsetdash{}{0pt}%
\pgfpathmoveto{\pgfqpoint{2.072652in}{1.840535in}}%
\pgfpathlineto{\pgfqpoint{2.115664in}{1.854089in}}%
\pgfpathlineto{\pgfqpoint{2.142756in}{1.891360in}}%
\pgfpathlineto{\pgfqpoint{2.100034in}{1.867003in}}%
\pgfpathlineto{\pgfqpoint{2.072652in}{1.840535in}}%
\pgfpathclose%
\pgfusepath{fill}%
\end{pgfscope}%
\begin{pgfscope}%
\pgfpathrectangle{\pgfqpoint{1.072000in}{0.528000in}}{\pgfqpoint{3.696000in}{3.696000in}}%
\pgfusepath{clip}%
\pgfsetbuttcap%
\pgfsetroundjoin%
\definecolor{currentfill}{rgb}{0.333490,0.446265,0.874452}%
\pgfsetfillcolor{currentfill}%
\pgfsetlinewidth{0.000000pt}%
\definecolor{currentstroke}{rgb}{0.000000,0.000000,0.000000}%
\pgfsetstrokecolor{currentstroke}%
\pgfsetdash{}{0pt}%
\pgfpathmoveto{\pgfqpoint{4.015579in}{1.935528in}}%
\pgfpathlineto{\pgfqpoint{4.063272in}{1.980914in}}%
\pgfpathlineto{\pgfqpoint{4.089522in}{2.083558in}}%
\pgfpathlineto{\pgfqpoint{4.041242in}{2.022075in}}%
\pgfpathlineto{\pgfqpoint{4.015579in}{1.935528in}}%
\pgfpathclose%
\pgfusepath{fill}%
\end{pgfscope}%
\begin{pgfscope}%
\pgfpathrectangle{\pgfqpoint{1.072000in}{0.528000in}}{\pgfqpoint{3.696000in}{3.696000in}}%
\pgfusepath{clip}%
\pgfsetbuttcap%
\pgfsetroundjoin%
\definecolor{currentfill}{rgb}{0.275827,0.366717,0.812553}%
\pgfsetfillcolor{currentfill}%
\pgfsetlinewidth{0.000000pt}%
\definecolor{currentstroke}{rgb}{0.000000,0.000000,0.000000}%
\pgfsetstrokecolor{currentstroke}%
\pgfsetdash{}{0pt}%
\pgfpathmoveto{\pgfqpoint{3.990598in}{1.871814in}}%
\pgfpathlineto{\pgfqpoint{4.037682in}{1.898752in}}%
\pgfpathlineto{\pgfqpoint{4.063272in}{1.980914in}}%
\pgfpathlineto{\pgfqpoint{4.015579in}{1.935528in}}%
\pgfpathlineto{\pgfqpoint{3.990598in}{1.871814in}}%
\pgfpathclose%
\pgfusepath{fill}%
\end{pgfscope}%
\begin{pgfscope}%
\pgfpathrectangle{\pgfqpoint{1.072000in}{0.528000in}}{\pgfqpoint{3.696000in}{3.696000in}}%
\pgfusepath{clip}%
\pgfsetbuttcap%
\pgfsetroundjoin%
\definecolor{currentfill}{rgb}{0.505423,0.643995,0.983157}%
\pgfsetfillcolor{currentfill}%
\pgfsetlinewidth{0.000000pt}%
\definecolor{currentstroke}{rgb}{0.000000,0.000000,0.000000}%
\pgfsetstrokecolor{currentstroke}%
\pgfsetdash{}{0pt}%
\pgfpathmoveto{\pgfqpoint{3.610557in}{2.341691in}}%
\pgfpathlineto{\pgfqpoint{3.654751in}{2.258697in}}%
\pgfpathlineto{\pgfqpoint{3.676684in}{2.157853in}}%
\pgfpathlineto{\pgfqpoint{3.632492in}{2.226747in}}%
\pgfpathlineto{\pgfqpoint{3.610557in}{2.341691in}}%
\pgfpathclose%
\pgfusepath{fill}%
\end{pgfscope}%
\begin{pgfscope}%
\pgfpathrectangle{\pgfqpoint{1.072000in}{0.528000in}}{\pgfqpoint{3.696000in}{3.696000in}}%
\pgfusepath{clip}%
\pgfsetbuttcap%
\pgfsetroundjoin%
\definecolor{currentfill}{rgb}{0.818056,0.855590,0.914638}%
\pgfsetfillcolor{currentfill}%
\pgfsetlinewidth{0.000000pt}%
\definecolor{currentstroke}{rgb}{0.000000,0.000000,0.000000}%
\pgfsetstrokecolor{currentstroke}%
\pgfsetdash{}{0pt}%
\pgfpathmoveto{\pgfqpoint{2.905660in}{2.714794in}}%
\pgfpathlineto{\pgfqpoint{2.950288in}{2.810158in}}%
\pgfpathlineto{\pgfqpoint{2.976172in}{2.682563in}}%
\pgfpathlineto{\pgfqpoint{2.931702in}{2.596574in}}%
\pgfpathlineto{\pgfqpoint{2.905660in}{2.714794in}}%
\pgfpathclose%
\pgfusepath{fill}%
\end{pgfscope}%
\begin{pgfscope}%
\pgfpathrectangle{\pgfqpoint{1.072000in}{0.528000in}}{\pgfqpoint{3.696000in}{3.696000in}}%
\pgfusepath{clip}%
\pgfsetbuttcap%
\pgfsetroundjoin%
\definecolor{currentfill}{rgb}{0.728970,0.817464,0.973188}%
\pgfsetfillcolor{currentfill}%
\pgfsetlinewidth{0.000000pt}%
\definecolor{currentstroke}{rgb}{0.000000,0.000000,0.000000}%
\pgfsetstrokecolor{currentstroke}%
\pgfsetdash{}{0pt}%
\pgfpathmoveto{\pgfqpoint{3.341581in}{2.644705in}}%
\pgfpathlineto{\pgfqpoint{3.386772in}{2.625146in}}%
\pgfpathlineto{\pgfqpoint{3.409620in}{2.474799in}}%
\pgfpathlineto{\pgfqpoint{3.364708in}{2.493507in}}%
\pgfpathlineto{\pgfqpoint{3.341581in}{2.644705in}}%
\pgfpathclose%
\pgfusepath{fill}%
\end{pgfscope}%
\begin{pgfscope}%
\pgfpathrectangle{\pgfqpoint{1.072000in}{0.528000in}}{\pgfqpoint{3.696000in}{3.696000in}}%
\pgfusepath{clip}%
\pgfsetbuttcap%
\pgfsetroundjoin%
\definecolor{currentfill}{rgb}{0.229806,0.298718,0.753683}%
\pgfsetfillcolor{currentfill}%
\pgfsetlinewidth{0.000000pt}%
\definecolor{currentstroke}{rgb}{0.000000,0.000000,0.000000}%
\pgfsetstrokecolor{currentstroke}%
\pgfsetdash{}{0pt}%
\pgfpathmoveto{\pgfqpoint{2.000740in}{1.855136in}}%
\pgfpathlineto{\pgfqpoint{2.045094in}{1.816753in}}%
\pgfpathlineto{\pgfqpoint{2.072652in}{1.840535in}}%
\pgfpathlineto{\pgfqpoint{2.028692in}{1.867284in}}%
\pgfpathlineto{\pgfqpoint{2.000740in}{1.855136in}}%
\pgfpathclose%
\pgfusepath{fill}%
\end{pgfscope}%
\begin{pgfscope}%
\pgfpathrectangle{\pgfqpoint{1.072000in}{0.528000in}}{\pgfqpoint{3.696000in}{3.696000in}}%
\pgfusepath{clip}%
\pgfsetbuttcap%
\pgfsetroundjoin%
\definecolor{currentfill}{rgb}{0.261805,0.346484,0.795658}%
\pgfsetfillcolor{currentfill}%
\pgfsetlinewidth{0.000000pt}%
\definecolor{currentstroke}{rgb}{0.000000,0.000000,0.000000}%
\pgfsetstrokecolor{currentstroke}%
\pgfsetdash{}{0pt}%
\pgfpathmoveto{\pgfqpoint{1.955276in}{1.931957in}}%
\pgfpathlineto{\pgfqpoint{2.000740in}{1.855136in}}%
\pgfpathlineto{\pgfqpoint{2.028692in}{1.867284in}}%
\pgfpathlineto{\pgfqpoint{1.983639in}{1.934010in}}%
\pgfpathlineto{\pgfqpoint{1.955276in}{1.931957in}}%
\pgfpathclose%
\pgfusepath{fill}%
\end{pgfscope}%
\begin{pgfscope}%
\pgfpathrectangle{\pgfqpoint{1.072000in}{0.528000in}}{\pgfqpoint{3.696000in}{3.696000in}}%
\pgfusepath{clip}%
\pgfsetbuttcap%
\pgfsetroundjoin%
\definecolor{currentfill}{rgb}{0.809329,0.852974,0.922323}%
\pgfsetfillcolor{currentfill}%
\pgfsetlinewidth{0.000000pt}%
\definecolor{currentstroke}{rgb}{0.000000,0.000000,0.000000}%
\pgfsetstrokecolor{currentstroke}%
\pgfsetdash{}{0pt}%
\pgfpathmoveto{\pgfqpoint{3.136363in}{2.736266in}}%
\pgfpathlineto{\pgfqpoint{3.181679in}{2.774102in}}%
\pgfpathlineto{\pgfqpoint{3.205957in}{2.622259in}}%
\pgfpathlineto{\pgfqpoint{3.160917in}{2.589796in}}%
\pgfpathlineto{\pgfqpoint{3.136363in}{2.736266in}}%
\pgfpathclose%
\pgfusepath{fill}%
\end{pgfscope}%
\begin{pgfscope}%
\pgfpathrectangle{\pgfqpoint{1.072000in}{0.528000in}}{\pgfqpoint{3.696000in}{3.696000in}}%
\pgfusepath{clip}%
\pgfsetbuttcap%
\pgfsetroundjoin%
\definecolor{currentfill}{rgb}{0.409611,0.540759,0.935545}%
\pgfsetfillcolor{currentfill}%
\pgfsetlinewidth{0.000000pt}%
\definecolor{currentstroke}{rgb}{0.000000,0.000000,0.000000}%
\pgfsetstrokecolor{currentstroke}%
\pgfsetdash{}{0pt}%
\pgfpathmoveto{\pgfqpoint{4.041242in}{2.022075in}}%
\pgfpathlineto{\pgfqpoint{4.089522in}{2.083558in}}%
\pgfpathlineto{\pgfqpoint{4.116440in}{2.205716in}}%
\pgfpathlineto{\pgfqpoint{4.067619in}{2.131068in}}%
\pgfpathlineto{\pgfqpoint{4.041242in}{2.022075in}}%
\pgfpathclose%
\pgfusepath{fill}%
\end{pgfscope}%
\begin{pgfscope}%
\pgfpathrectangle{\pgfqpoint{1.072000in}{0.528000in}}{\pgfqpoint{3.696000in}{3.696000in}}%
\pgfusepath{clip}%
\pgfsetbuttcap%
\pgfsetroundjoin%
\definecolor{currentfill}{rgb}{0.708720,0.805721,0.981117}%
\pgfsetfillcolor{currentfill}%
\pgfsetlinewidth{0.000000pt}%
\definecolor{currentstroke}{rgb}{0.000000,0.000000,0.000000}%
\pgfsetstrokecolor{currentstroke}%
\pgfsetdash{}{0pt}%
\pgfpathmoveto{\pgfqpoint{2.553684in}{2.472329in}}%
\pgfpathlineto{\pgfqpoint{2.596174in}{2.621099in}}%
\pgfpathlineto{\pgfqpoint{2.623506in}{2.569503in}}%
\pgfpathlineto{\pgfqpoint{2.580930in}{2.430607in}}%
\pgfpathlineto{\pgfqpoint{2.553684in}{2.472329in}}%
\pgfpathclose%
\pgfusepath{fill}%
\end{pgfscope}%
\begin{pgfscope}%
\pgfpathrectangle{\pgfqpoint{1.072000in}{0.528000in}}{\pgfqpoint{3.696000in}{3.696000in}}%
\pgfusepath{clip}%
\pgfsetbuttcap%
\pgfsetroundjoin%
\definecolor{currentfill}{rgb}{0.430507,0.564883,0.948889}%
\pgfsetfillcolor{currentfill}%
\pgfsetlinewidth{0.000000pt}%
\definecolor{currentstroke}{rgb}{0.000000,0.000000,0.000000}%
\pgfsetstrokecolor{currentstroke}%
\pgfsetdash{}{0pt}%
\pgfpathmoveto{\pgfqpoint{2.265493in}{2.072979in}}%
\pgfpathlineto{\pgfqpoint{2.306871in}{2.189359in}}%
\pgfpathlineto{\pgfqpoint{2.333905in}{2.217018in}}%
\pgfpathlineto{\pgfqpoint{2.292493in}{2.099494in}}%
\pgfpathlineto{\pgfqpoint{2.265493in}{2.072979in}}%
\pgfpathclose%
\pgfusepath{fill}%
\end{pgfscope}%
\begin{pgfscope}%
\pgfpathrectangle{\pgfqpoint{1.072000in}{0.528000in}}{\pgfqpoint{3.696000in}{3.696000in}}%
\pgfusepath{clip}%
\pgfsetbuttcap%
\pgfsetroundjoin%
\definecolor{currentfill}{rgb}{0.565182,0.699438,0.996635}%
\pgfsetfillcolor{currentfill}%
\pgfsetlinewidth{0.000000pt}%
\definecolor{currentstroke}{rgb}{0.000000,0.000000,0.000000}%
\pgfsetstrokecolor{currentstroke}%
\pgfsetdash{}{0pt}%
\pgfpathmoveto{\pgfqpoint{3.566179in}{2.418898in}}%
\pgfpathlineto{\pgfqpoint{3.610557in}{2.341691in}}%
\pgfpathlineto{\pgfqpoint{3.632492in}{2.226747in}}%
\pgfpathlineto{\pgfqpoint{3.588193in}{2.292281in}}%
\pgfpathlineto{\pgfqpoint{3.566179in}{2.418898in}}%
\pgfpathclose%
\pgfusepath{fill}%
\end{pgfscope}%
\begin{pgfscope}%
\pgfpathrectangle{\pgfqpoint{1.072000in}{0.528000in}}{\pgfqpoint{3.696000in}{3.696000in}}%
\pgfusepath{clip}%
\pgfsetbuttcap%
\pgfsetroundjoin%
\definecolor{currentfill}{rgb}{0.252663,0.332837,0.783665}%
\pgfsetfillcolor{currentfill}%
\pgfsetlinewidth{0.000000pt}%
\definecolor{currentstroke}{rgb}{0.000000,0.000000,0.000000}%
\pgfsetstrokecolor{currentstroke}%
\pgfsetdash{}{0pt}%
\pgfpathmoveto{\pgfqpoint{3.852717in}{1.908317in}}%
\pgfpathlineto{\pgfqpoint{3.897254in}{1.847957in}}%
\pgfpathlineto{\pgfqpoint{3.920569in}{1.847028in}}%
\pgfpathlineto{\pgfqpoint{3.875564in}{1.884914in}}%
\pgfpathlineto{\pgfqpoint{3.852717in}{1.908317in}}%
\pgfpathclose%
\pgfusepath{fill}%
\end{pgfscope}%
\begin{pgfscope}%
\pgfpathrectangle{\pgfqpoint{1.072000in}{0.528000in}}{\pgfqpoint{3.696000in}{3.696000in}}%
\pgfusepath{clip}%
\pgfsetbuttcap%
\pgfsetroundjoin%
\definecolor{currentfill}{rgb}{0.243520,0.319189,0.771672}%
\pgfsetfillcolor{currentfill}%
\pgfsetlinewidth{0.000000pt}%
\definecolor{currentstroke}{rgb}{0.000000,0.000000,0.000000}%
\pgfsetstrokecolor{currentstroke}%
\pgfsetdash{}{0pt}%
\pgfpathmoveto{\pgfqpoint{3.966228in}{1.830187in}}%
\pgfpathlineto{\pgfqpoint{4.012706in}{1.836948in}}%
\pgfpathlineto{\pgfqpoint{4.037682in}{1.898752in}}%
\pgfpathlineto{\pgfqpoint{3.990598in}{1.871814in}}%
\pgfpathlineto{\pgfqpoint{3.966228in}{1.830187in}}%
\pgfpathclose%
\pgfusepath{fill}%
\end{pgfscope}%
\begin{pgfscope}%
\pgfpathrectangle{\pgfqpoint{1.072000in}{0.528000in}}{\pgfqpoint{3.696000in}{3.696000in}}%
\pgfusepath{clip}%
\pgfsetbuttcap%
\pgfsetroundjoin%
\definecolor{currentfill}{rgb}{0.299441,0.400248,0.839842}%
\pgfsetfillcolor{currentfill}%
\pgfsetlinewidth{0.000000pt}%
\definecolor{currentstroke}{rgb}{0.000000,0.000000,0.000000}%
\pgfsetstrokecolor{currentstroke}%
\pgfsetdash{}{0pt}%
\pgfpathmoveto{\pgfqpoint{2.142756in}{1.891360in}}%
\pgfpathlineto{\pgfqpoint{2.184778in}{1.954628in}}%
\pgfpathlineto{\pgfqpoint{2.211641in}{1.999611in}}%
\pgfpathlineto{\pgfqpoint{2.169791in}{1.927482in}}%
\pgfpathlineto{\pgfqpoint{2.142756in}{1.891360in}}%
\pgfpathclose%
\pgfusepath{fill}%
\end{pgfscope}%
\begin{pgfscope}%
\pgfpathrectangle{\pgfqpoint{1.072000in}{0.528000in}}{\pgfqpoint{3.696000in}{3.696000in}}%
\pgfusepath{clip}%
\pgfsetbuttcap%
\pgfsetroundjoin%
\definecolor{currentfill}{rgb}{0.718985,0.811993,0.977656}%
\pgfsetfillcolor{currentfill}%
\pgfsetlinewidth{0.000000pt}%
\definecolor{currentstroke}{rgb}{0.000000,0.000000,0.000000}%
\pgfsetstrokecolor{currentstroke}%
\pgfsetdash{}{0pt}%
\pgfpathmoveto{\pgfqpoint{3.386772in}{2.625146in}}%
\pgfpathlineto{\pgfqpoint{3.431868in}{2.592012in}}%
\pgfpathlineto{\pgfqpoint{3.454454in}{2.444422in}}%
\pgfpathlineto{\pgfqpoint{3.409620in}{2.474799in}}%
\pgfpathlineto{\pgfqpoint{3.386772in}{2.625146in}}%
\pgfpathclose%
\pgfusepath{fill}%
\end{pgfscope}%
\begin{pgfscope}%
\pgfpathrectangle{\pgfqpoint{1.072000in}{0.528000in}}{\pgfqpoint{3.696000in}{3.696000in}}%
\pgfusepath{clip}%
\pgfsetbuttcap%
\pgfsetroundjoin%
\definecolor{currentfill}{rgb}{0.289996,0.386836,0.828926}%
\pgfsetfillcolor{currentfill}%
\pgfsetlinewidth{0.000000pt}%
\definecolor{currentstroke}{rgb}{0.000000,0.000000,0.000000}%
\pgfsetstrokecolor{currentstroke}%
\pgfsetdash{}{0pt}%
\pgfpathmoveto{\pgfqpoint{3.808581in}{1.985941in}}%
\pgfpathlineto{\pgfqpoint{3.852717in}{1.908317in}}%
\pgfpathlineto{\pgfqpoint{3.875564in}{1.884914in}}%
\pgfpathlineto{\pgfqpoint{3.831042in}{1.940501in}}%
\pgfpathlineto{\pgfqpoint{3.808581in}{1.985941in}}%
\pgfpathclose%
\pgfusepath{fill}%
\end{pgfscope}%
\begin{pgfscope}%
\pgfpathrectangle{\pgfqpoint{1.072000in}{0.528000in}}{\pgfqpoint{3.696000in}{3.696000in}}%
\pgfusepath{clip}%
\pgfsetbuttcap%
\pgfsetroundjoin%
\definecolor{currentfill}{rgb}{0.229806,0.298718,0.753683}%
\pgfsetfillcolor{currentfill}%
\pgfsetlinewidth{0.000000pt}%
\definecolor{currentstroke}{rgb}{0.000000,0.000000,0.000000}%
\pgfsetstrokecolor{currentstroke}%
\pgfsetdash{}{0pt}%
\pgfpathmoveto{\pgfqpoint{3.897254in}{1.847957in}}%
\pgfpathlineto{\pgfqpoint{3.942376in}{1.808901in}}%
\pgfpathlineto{\pgfqpoint{3.966228in}{1.830187in}}%
\pgfpathlineto{\pgfqpoint{3.920569in}{1.847028in}}%
\pgfpathlineto{\pgfqpoint{3.897254in}{1.847957in}}%
\pgfpathclose%
\pgfusepath{fill}%
\end{pgfscope}%
\begin{pgfscope}%
\pgfpathrectangle{\pgfqpoint{1.072000in}{0.528000in}}{\pgfqpoint{3.696000in}{3.696000in}}%
\pgfusepath{clip}%
\pgfsetbuttcap%
\pgfsetroundjoin%
\definecolor{currentfill}{rgb}{0.619318,0.744121,0.998931}%
\pgfsetfillcolor{currentfill}%
\pgfsetlinewidth{0.000000pt}%
\definecolor{currentstroke}{rgb}{0.000000,0.000000,0.000000}%
\pgfsetstrokecolor{currentstroke}%
\pgfsetdash{}{0pt}%
\pgfpathmoveto{\pgfqpoint{3.521600in}{2.487643in}}%
\pgfpathlineto{\pgfqpoint{3.566179in}{2.418898in}}%
\pgfpathlineto{\pgfqpoint{3.588193in}{2.292281in}}%
\pgfpathlineto{\pgfqpoint{3.543757in}{2.351768in}}%
\pgfpathlineto{\pgfqpoint{3.521600in}{2.487643in}}%
\pgfpathclose%
\pgfusepath{fill}%
\end{pgfscope}%
\begin{pgfscope}%
\pgfpathrectangle{\pgfqpoint{1.072000in}{0.528000in}}{\pgfqpoint{3.696000in}{3.696000in}}%
\pgfusepath{clip}%
\pgfsetbuttcap%
\pgfsetroundjoin%
\definecolor{currentfill}{rgb}{0.839351,0.861167,0.894494}%
\pgfsetfillcolor{currentfill}%
\pgfsetlinewidth{0.000000pt}%
\definecolor{currentstroke}{rgb}{0.000000,0.000000,0.000000}%
\pgfsetstrokecolor{currentstroke}%
\pgfsetdash{}{0pt}%
\pgfpathmoveto{\pgfqpoint{3.020982in}{2.761076in}}%
\pgfpathlineto{\pgfqpoint{3.066099in}{2.829021in}}%
\pgfpathlineto{\pgfqpoint{3.091235in}{2.686021in}}%
\pgfpathlineto{\pgfqpoint{3.046346in}{2.625224in}}%
\pgfpathlineto{\pgfqpoint{3.020982in}{2.761076in}}%
\pgfpathclose%
\pgfusepath{fill}%
\end{pgfscope}%
\begin{pgfscope}%
\pgfpathrectangle{\pgfqpoint{1.072000in}{0.528000in}}{\pgfqpoint{3.696000in}{3.696000in}}%
\pgfusepath{clip}%
\pgfsetbuttcap%
\pgfsetroundjoin%
\definecolor{currentfill}{rgb}{0.698454,0.799450,0.984577}%
\pgfsetfillcolor{currentfill}%
\pgfsetlinewidth{0.000000pt}%
\definecolor{currentstroke}{rgb}{0.000000,0.000000,0.000000}%
\pgfsetstrokecolor{currentstroke}%
\pgfsetdash{}{0pt}%
\pgfpathmoveto{\pgfqpoint{3.431868in}{2.592012in}}%
\pgfpathlineto{\pgfqpoint{3.476823in}{2.545849in}}%
\pgfpathlineto{\pgfqpoint{3.499174in}{2.403032in}}%
\pgfpathlineto{\pgfqpoint{3.454454in}{2.444422in}}%
\pgfpathlineto{\pgfqpoint{3.431868in}{2.592012in}}%
\pgfpathclose%
\pgfusepath{fill}%
\end{pgfscope}%
\begin{pgfscope}%
\pgfpathrectangle{\pgfqpoint{1.072000in}{0.528000in}}{\pgfqpoint{3.696000in}{3.696000in}}%
\pgfusepath{clip}%
\pgfsetbuttcap%
\pgfsetroundjoin%
\definecolor{currentfill}{rgb}{0.505423,0.643995,0.983157}%
\pgfsetfillcolor{currentfill}%
\pgfsetlinewidth{0.000000pt}%
\definecolor{currentstroke}{rgb}{0.000000,0.000000,0.000000}%
\pgfsetstrokecolor{currentstroke}%
\pgfsetdash{}{0pt}%
\pgfpathmoveto{\pgfqpoint{4.067619in}{2.131068in}}%
\pgfpathlineto{\pgfqpoint{4.116440in}{2.205716in}}%
\pgfpathlineto{\pgfqpoint{4.143993in}{2.345262in}}%
\pgfpathlineto{\pgfqpoint{4.094703in}{2.260877in}}%
\pgfpathlineto{\pgfqpoint{4.067619in}{2.131068in}}%
\pgfpathclose%
\pgfusepath{fill}%
\end{pgfscope}%
\begin{pgfscope}%
\pgfpathrectangle{\pgfqpoint{1.072000in}{0.528000in}}{\pgfqpoint{3.696000in}{3.696000in}}%
\pgfusepath{clip}%
\pgfsetbuttcap%
\pgfsetroundjoin%
\definecolor{currentfill}{rgb}{0.661968,0.775491,0.993937}%
\pgfsetfillcolor{currentfill}%
\pgfsetlinewidth{0.000000pt}%
\definecolor{currentstroke}{rgb}{0.000000,0.000000,0.000000}%
\pgfsetstrokecolor{currentstroke}%
\pgfsetdash{}{0pt}%
\pgfpathmoveto{\pgfqpoint{3.476823in}{2.545849in}}%
\pgfpathlineto{\pgfqpoint{3.521600in}{2.487643in}}%
\pgfpathlineto{\pgfqpoint{3.543757in}{2.351768in}}%
\pgfpathlineto{\pgfqpoint{3.499174in}{2.403032in}}%
\pgfpathlineto{\pgfqpoint{3.476823in}{2.545849in}}%
\pgfpathclose%
\pgfusepath{fill}%
\end{pgfscope}%
\begin{pgfscope}%
\pgfpathrectangle{\pgfqpoint{1.072000in}{0.528000in}}{\pgfqpoint{3.696000in}{3.696000in}}%
\pgfusepath{clip}%
\pgfsetbuttcap%
\pgfsetroundjoin%
\definecolor{currentfill}{rgb}{0.229806,0.298718,0.753683}%
\pgfsetfillcolor{currentfill}%
\pgfsetlinewidth{0.000000pt}%
\definecolor{currentstroke}{rgb}{0.000000,0.000000,0.000000}%
\pgfsetstrokecolor{currentstroke}%
\pgfsetdash{}{0pt}%
\pgfpathmoveto{\pgfqpoint{2.045094in}{1.816753in}}%
\pgfpathlineto{\pgfqpoint{2.088466in}{1.817543in}}%
\pgfpathlineto{\pgfqpoint{2.115664in}{1.854089in}}%
\pgfpathlineto{\pgfqpoint{2.072652in}{1.840535in}}%
\pgfpathlineto{\pgfqpoint{2.045094in}{1.816753in}}%
\pgfpathclose%
\pgfusepath{fill}%
\end{pgfscope}%
\begin{pgfscope}%
\pgfpathrectangle{\pgfqpoint{1.072000in}{0.528000in}}{\pgfqpoint{3.696000in}{3.696000in}}%
\pgfusepath{clip}%
\pgfsetbuttcap%
\pgfsetroundjoin%
\definecolor{currentfill}{rgb}{0.565182,0.699438,0.996635}%
\pgfsetfillcolor{currentfill}%
\pgfsetlinewidth{0.000000pt}%
\definecolor{currentstroke}{rgb}{0.000000,0.000000,0.000000}%
\pgfsetstrokecolor{currentstroke}%
\pgfsetdash{}{0pt}%
\pgfpathmoveto{\pgfqpoint{2.361011in}{2.234938in}}%
\pgfpathlineto{\pgfqpoint{2.402482in}{2.377750in}}%
\pgfpathlineto{\pgfqpoint{2.429742in}{2.382636in}}%
\pgfpathlineto{\pgfqpoint{2.388164in}{2.243398in}}%
\pgfpathlineto{\pgfqpoint{2.361011in}{2.234938in}}%
\pgfpathclose%
\pgfusepath{fill}%
\end{pgfscope}%
\begin{pgfscope}%
\pgfpathrectangle{\pgfqpoint{1.072000in}{0.528000in}}{\pgfqpoint{3.696000in}{3.696000in}}%
\pgfusepath{clip}%
\pgfsetbuttcap%
\pgfsetroundjoin%
\definecolor{currentfill}{rgb}{0.343278,0.459354,0.884122}%
\pgfsetfillcolor{currentfill}%
\pgfsetlinewidth{0.000000pt}%
\definecolor{currentstroke}{rgb}{0.000000,0.000000,0.000000}%
\pgfsetstrokecolor{currentstroke}%
\pgfsetdash{}{0pt}%
\pgfpathmoveto{\pgfqpoint{3.764671in}{2.076282in}}%
\pgfpathlineto{\pgfqpoint{3.808581in}{1.985941in}}%
\pgfpathlineto{\pgfqpoint{3.831042in}{1.940501in}}%
\pgfpathlineto{\pgfqpoint{3.786839in}{2.009888in}}%
\pgfpathlineto{\pgfqpoint{3.764671in}{2.076282in}}%
\pgfpathclose%
\pgfusepath{fill}%
\end{pgfscope}%
\begin{pgfscope}%
\pgfpathrectangle{\pgfqpoint{1.072000in}{0.528000in}}{\pgfqpoint{3.696000in}{3.696000in}}%
\pgfusepath{clip}%
\pgfsetbuttcap%
\pgfsetroundjoin%
\definecolor{currentfill}{rgb}{0.839351,0.861167,0.894494}%
\pgfsetfillcolor{currentfill}%
\pgfsetlinewidth{0.000000pt}%
\definecolor{currentstroke}{rgb}{0.000000,0.000000,0.000000}%
\pgfsetstrokecolor{currentstroke}%
\pgfsetdash{}{0pt}%
\pgfpathmoveto{\pgfqpoint{2.834972in}{2.719632in}}%
\pgfpathlineto{\pgfqpoint{2.879315in}{2.830657in}}%
\pgfpathlineto{\pgfqpoint{2.905660in}{2.714794in}}%
\pgfpathlineto{\pgfqpoint{2.861429in}{2.612884in}}%
\pgfpathlineto{\pgfqpoint{2.834972in}{2.719632in}}%
\pgfpathclose%
\pgfusepath{fill}%
\end{pgfscope}%
\begin{pgfscope}%
\pgfpathrectangle{\pgfqpoint{1.072000in}{0.528000in}}{\pgfqpoint{3.696000in}{3.696000in}}%
\pgfusepath{clip}%
\pgfsetbuttcap%
\pgfsetroundjoin%
\definecolor{currentfill}{rgb}{0.661968,0.775491,0.993937}%
\pgfsetfillcolor{currentfill}%
\pgfsetlinewidth{0.000000pt}%
\definecolor{currentstroke}{rgb}{0.000000,0.000000,0.000000}%
\pgfsetstrokecolor{currentstroke}%
\pgfsetdash{}{0pt}%
\pgfpathmoveto{\pgfqpoint{2.457032in}{2.376504in}}%
\pgfpathlineto{\pgfqpoint{2.498945in}{2.528893in}}%
\pgfpathlineto{\pgfqpoint{2.526340in}{2.505800in}}%
\pgfpathlineto{\pgfqpoint{2.484309in}{2.360562in}}%
\pgfpathlineto{\pgfqpoint{2.457032in}{2.376504in}}%
\pgfpathclose%
\pgfusepath{fill}%
\end{pgfscope}%
\begin{pgfscope}%
\pgfpathrectangle{\pgfqpoint{1.072000in}{0.528000in}}{\pgfqpoint{3.696000in}{3.696000in}}%
\pgfusepath{clip}%
\pgfsetbuttcap%
\pgfsetroundjoin%
\definecolor{currentfill}{rgb}{0.777378,0.840921,0.946149}%
\pgfsetfillcolor{currentfill}%
\pgfsetlinewidth{0.000000pt}%
\definecolor{currentstroke}{rgb}{0.000000,0.000000,0.000000}%
\pgfsetstrokecolor{currentstroke}%
\pgfsetdash{}{0pt}%
\pgfpathmoveto{\pgfqpoint{2.623506in}{2.569503in}}%
\pgfpathlineto{\pgfqpoint{2.666510in}{2.714938in}}%
\pgfpathlineto{\pgfqpoint{2.693739in}{2.645564in}}%
\pgfpathlineto{\pgfqpoint{2.650696in}{2.509899in}}%
\pgfpathlineto{\pgfqpoint{2.623506in}{2.569503in}}%
\pgfpathclose%
\pgfusepath{fill}%
\end{pgfscope}%
\begin{pgfscope}%
\pgfpathrectangle{\pgfqpoint{1.072000in}{0.528000in}}{\pgfqpoint{3.696000in}{3.696000in}}%
\pgfusepath{clip}%
\pgfsetbuttcap%
\pgfsetroundjoin%
\definecolor{currentfill}{rgb}{0.425199,0.559058,0.946061}%
\pgfsetfillcolor{currentfill}%
\pgfsetlinewidth{0.000000pt}%
\definecolor{currentstroke}{rgb}{0.000000,0.000000,0.000000}%
\pgfsetstrokecolor{currentstroke}%
\pgfsetdash{}{0pt}%
\pgfpathmoveto{\pgfqpoint{2.238542in}{2.039452in}}%
\pgfpathlineto{\pgfqpoint{2.279930in}{2.152166in}}%
\pgfpathlineto{\pgfqpoint{2.306871in}{2.189359in}}%
\pgfpathlineto{\pgfqpoint{2.265493in}{2.072979in}}%
\pgfpathlineto{\pgfqpoint{2.238542in}{2.039452in}}%
\pgfpathclose%
\pgfusepath{fill}%
\end{pgfscope}%
\begin{pgfscope}%
\pgfpathrectangle{\pgfqpoint{1.072000in}{0.528000in}}{\pgfqpoint{3.696000in}{3.696000in}}%
\pgfusepath{clip}%
\pgfsetbuttcap%
\pgfsetroundjoin%
\definecolor{currentfill}{rgb}{0.229806,0.298718,0.753683}%
\pgfsetfillcolor{currentfill}%
\pgfsetlinewidth{0.000000pt}%
\definecolor{currentstroke}{rgb}{0.000000,0.000000,0.000000}%
\pgfsetstrokecolor{currentstroke}%
\pgfsetdash{}{0pt}%
\pgfpathmoveto{\pgfqpoint{3.942376in}{1.808901in}}%
\pgfpathlineto{\pgfqpoint{3.988269in}{1.794414in}}%
\pgfpathlineto{\pgfqpoint{4.012706in}{1.836948in}}%
\pgfpathlineto{\pgfqpoint{3.966228in}{1.830187in}}%
\pgfpathlineto{\pgfqpoint{3.942376in}{1.808901in}}%
\pgfpathclose%
\pgfusepath{fill}%
\end{pgfscope}%
\begin{pgfscope}%
\pgfpathrectangle{\pgfqpoint{1.072000in}{0.528000in}}{\pgfqpoint{3.696000in}{3.696000in}}%
\pgfusepath{clip}%
\pgfsetbuttcap%
\pgfsetroundjoin%
\definecolor{currentfill}{rgb}{0.280550,0.373423,0.818011}%
\pgfsetfillcolor{currentfill}%
\pgfsetlinewidth{0.000000pt}%
\definecolor{currentstroke}{rgb}{0.000000,0.000000,0.000000}%
\pgfsetstrokecolor{currentstroke}%
\pgfsetdash{}{0pt}%
\pgfpathmoveto{\pgfqpoint{2.115664in}{1.854089in}}%
\pgfpathlineto{\pgfqpoint{2.157925in}{1.906131in}}%
\pgfpathlineto{\pgfqpoint{2.184778in}{1.954628in}}%
\pgfpathlineto{\pgfqpoint{2.142756in}{1.891360in}}%
\pgfpathlineto{\pgfqpoint{2.115664in}{1.854089in}}%
\pgfpathclose%
\pgfusepath{fill}%
\end{pgfscope}%
\begin{pgfscope}%
\pgfpathrectangle{\pgfqpoint{1.072000in}{0.528000in}}{\pgfqpoint{3.696000in}{3.696000in}}%
\pgfusepath{clip}%
\pgfsetbuttcap%
\pgfsetroundjoin%
\definecolor{currentfill}{rgb}{0.238948,0.312365,0.765676}%
\pgfsetfillcolor{currentfill}%
\pgfsetlinewidth{0.000000pt}%
\definecolor{currentstroke}{rgb}{0.000000,0.000000,0.000000}%
\pgfsetstrokecolor{currentstroke}%
\pgfsetdash{}{0pt}%
\pgfpathmoveto{\pgfqpoint{1.972485in}{1.848901in}}%
\pgfpathlineto{\pgfqpoint{2.017298in}{1.797526in}}%
\pgfpathlineto{\pgfqpoint{2.045094in}{1.816753in}}%
\pgfpathlineto{\pgfqpoint{2.000740in}{1.855136in}}%
\pgfpathlineto{\pgfqpoint{1.972485in}{1.848901in}}%
\pgfpathclose%
\pgfusepath{fill}%
\end{pgfscope}%
\begin{pgfscope}%
\pgfpathrectangle{\pgfqpoint{1.072000in}{0.528000in}}{\pgfqpoint{3.696000in}{3.696000in}}%
\pgfusepath{clip}%
\pgfsetbuttcap%
\pgfsetroundjoin%
\definecolor{currentfill}{rgb}{0.839351,0.861167,0.894494}%
\pgfsetfillcolor{currentfill}%
\pgfsetlinewidth{0.000000pt}%
\definecolor{currentstroke}{rgb}{0.000000,0.000000,0.000000}%
\pgfsetstrokecolor{currentstroke}%
\pgfsetdash{}{0pt}%
\pgfpathmoveto{\pgfqpoint{2.764276in}{2.696340in}}%
\pgfpathlineto{\pgfqpoint{2.808242in}{2.821479in}}%
\pgfpathlineto{\pgfqpoint{2.834972in}{2.719632in}}%
\pgfpathlineto{\pgfqpoint{2.791069in}{2.603622in}}%
\pgfpathlineto{\pgfqpoint{2.764276in}{2.696340in}}%
\pgfpathclose%
\pgfusepath{fill}%
\end{pgfscope}%
\begin{pgfscope}%
\pgfpathrectangle{\pgfqpoint{1.072000in}{0.528000in}}{\pgfqpoint{3.696000in}{3.696000in}}%
\pgfusepath{clip}%
\pgfsetbuttcap%
\pgfsetroundjoin%
\definecolor{currentfill}{rgb}{0.818056,0.855590,0.914638}%
\pgfsetfillcolor{currentfill}%
\pgfsetlinewidth{0.000000pt}%
\definecolor{currentstroke}{rgb}{0.000000,0.000000,0.000000}%
\pgfsetstrokecolor{currentstroke}%
\pgfsetdash{}{0pt}%
\pgfpathmoveto{\pgfqpoint{2.693739in}{2.645564in}}%
\pgfpathlineto{\pgfqpoint{2.737250in}{2.782573in}}%
\pgfpathlineto{\pgfqpoint{2.764276in}{2.696340in}}%
\pgfpathlineto{\pgfqpoint{2.720778in}{2.568746in}}%
\pgfpathlineto{\pgfqpoint{2.693739in}{2.645564in}}%
\pgfpathclose%
\pgfusepath{fill}%
\end{pgfscope}%
\begin{pgfscope}%
\pgfpathrectangle{\pgfqpoint{1.072000in}{0.528000in}}{\pgfqpoint{3.696000in}{3.696000in}}%
\pgfusepath{clip}%
\pgfsetbuttcap%
\pgfsetroundjoin%
\definecolor{currentfill}{rgb}{0.409611,0.540759,0.935545}%
\pgfsetfillcolor{currentfill}%
\pgfsetlinewidth{0.000000pt}%
\definecolor{currentstroke}{rgb}{0.000000,0.000000,0.000000}%
\pgfsetstrokecolor{currentstroke}%
\pgfsetdash{}{0pt}%
\pgfpathmoveto{\pgfqpoint{3.720830in}{2.174570in}}%
\pgfpathlineto{\pgfqpoint{3.764671in}{2.076282in}}%
\pgfpathlineto{\pgfqpoint{3.786839in}{2.009888in}}%
\pgfpathlineto{\pgfqpoint{3.742804in}{2.088868in}}%
\pgfpathlineto{\pgfqpoint{3.720830in}{2.174570in}}%
\pgfpathclose%
\pgfusepath{fill}%
\end{pgfscope}%
\begin{pgfscope}%
\pgfpathrectangle{\pgfqpoint{1.072000in}{0.528000in}}{\pgfqpoint{3.696000in}{3.696000in}}%
\pgfusepath{clip}%
\pgfsetbuttcap%
\pgfsetroundjoin%
\definecolor{currentfill}{rgb}{0.839351,0.861167,0.894494}%
\pgfsetfillcolor{currentfill}%
\pgfsetlinewidth{0.000000pt}%
\definecolor{currentstroke}{rgb}{0.000000,0.000000,0.000000}%
\pgfsetstrokecolor{currentstroke}%
\pgfsetdash{}{0pt}%
\pgfpathmoveto{\pgfqpoint{3.181679in}{2.774102in}}%
\pgfpathlineto{\pgfqpoint{3.227126in}{2.798275in}}%
\pgfpathlineto{\pgfqpoint{3.251117in}{2.642852in}}%
\pgfpathlineto{\pgfqpoint{3.205957in}{2.622259in}}%
\pgfpathlineto{\pgfqpoint{3.181679in}{2.774102in}}%
\pgfpathclose%
\pgfusepath{fill}%
\end{pgfscope}%
\begin{pgfscope}%
\pgfpathrectangle{\pgfqpoint{1.072000in}{0.528000in}}{\pgfqpoint{3.696000in}{3.696000in}}%
\pgfusepath{clip}%
\pgfsetbuttcap%
\pgfsetroundjoin%
\definecolor{currentfill}{rgb}{0.280550,0.373423,0.818011}%
\pgfsetfillcolor{currentfill}%
\pgfsetlinewidth{0.000000pt}%
\definecolor{currentstroke}{rgb}{0.000000,0.000000,0.000000}%
\pgfsetstrokecolor{currentstroke}%
\pgfsetdash{}{0pt}%
\pgfpathmoveto{\pgfqpoint{1.926554in}{1.936834in}}%
\pgfpathlineto{\pgfqpoint{1.972485in}{1.848901in}}%
\pgfpathlineto{\pgfqpoint{2.000740in}{1.855136in}}%
\pgfpathlineto{\pgfqpoint{1.955276in}{1.931957in}}%
\pgfpathlineto{\pgfqpoint{1.926554in}{1.936834in}}%
\pgfpathclose%
\pgfusepath{fill}%
\end{pgfscope}%
\begin{pgfscope}%
\pgfpathrectangle{\pgfqpoint{1.072000in}{0.528000in}}{\pgfqpoint{3.696000in}{3.696000in}}%
\pgfusepath{clip}%
\pgfsetbuttcap%
\pgfsetroundjoin%
\definecolor{currentfill}{rgb}{0.624703,0.748318,0.998719}%
\pgfsetfillcolor{currentfill}%
\pgfsetlinewidth{0.000000pt}%
\definecolor{currentstroke}{rgb}{0.000000,0.000000,0.000000}%
\pgfsetstrokecolor{currentstroke}%
\pgfsetdash{}{0pt}%
\pgfpathmoveto{\pgfqpoint{4.094703in}{2.260877in}}%
\pgfpathlineto{\pgfqpoint{4.143993in}{2.345262in}}%
\pgfpathlineto{\pgfqpoint{4.172101in}{2.498893in}}%
\pgfpathlineto{\pgfqpoint{4.122437in}{2.408593in}}%
\pgfpathlineto{\pgfqpoint{4.094703in}{2.260877in}}%
\pgfpathclose%
\pgfusepath{fill}%
\end{pgfscope}%
\begin{pgfscope}%
\pgfpathrectangle{\pgfqpoint{1.072000in}{0.528000in}}{\pgfqpoint{3.696000in}{3.696000in}}%
\pgfusepath{clip}%
\pgfsetbuttcap%
\pgfsetroundjoin%
\definecolor{currentfill}{rgb}{0.333490,0.446265,0.874452}%
\pgfsetfillcolor{currentfill}%
\pgfsetlinewidth{0.000000pt}%
\definecolor{currentstroke}{rgb}{0.000000,0.000000,0.000000}%
\pgfsetstrokecolor{currentstroke}%
\pgfsetdash{}{0pt}%
\pgfpathmoveto{\pgfqpoint{4.037682in}{1.898752in}}%
\pgfpathlineto{\pgfqpoint{4.085736in}{1.948834in}}%
\pgfpathlineto{\pgfqpoint{4.111923in}{2.047117in}}%
\pgfpathlineto{\pgfqpoint{4.063272in}{1.980914in}}%
\pgfpathlineto{\pgfqpoint{4.037682in}{1.898752in}}%
\pgfpathclose%
\pgfusepath{fill}%
\end{pgfscope}%
\begin{pgfscope}%
\pgfpathrectangle{\pgfqpoint{1.072000in}{0.528000in}}{\pgfqpoint{3.696000in}{3.696000in}}%
\pgfusepath{clip}%
\pgfsetbuttcap%
\pgfsetroundjoin%
\definecolor{currentfill}{rgb}{0.280550,0.373423,0.818011}%
\pgfsetfillcolor{currentfill}%
\pgfsetlinewidth{0.000000pt}%
\definecolor{currentstroke}{rgb}{0.000000,0.000000,0.000000}%
\pgfsetstrokecolor{currentstroke}%
\pgfsetdash{}{0pt}%
\pgfpathmoveto{\pgfqpoint{4.012706in}{1.836948in}}%
\pgfpathlineto{\pgfqpoint{4.060147in}{1.868865in}}%
\pgfpathlineto{\pgfqpoint{4.085736in}{1.948834in}}%
\pgfpathlineto{\pgfqpoint{4.037682in}{1.898752in}}%
\pgfpathlineto{\pgfqpoint{4.012706in}{1.836948in}}%
\pgfpathclose%
\pgfusepath{fill}%
\end{pgfscope}%
\begin{pgfscope}%
\pgfpathrectangle{\pgfqpoint{1.072000in}{0.528000in}}{\pgfqpoint{3.696000in}{3.696000in}}%
\pgfusepath{clip}%
\pgfsetbuttcap%
\pgfsetroundjoin%
\definecolor{currentfill}{rgb}{0.229806,0.298718,0.753683}%
\pgfsetfillcolor{currentfill}%
\pgfsetlinewidth{0.000000pt}%
\definecolor{currentstroke}{rgb}{0.000000,0.000000,0.000000}%
\pgfsetstrokecolor{currentstroke}%
\pgfsetdash{}{0pt}%
\pgfpathmoveto{\pgfqpoint{2.017298in}{1.797526in}}%
\pgfpathlineto{\pgfqpoint{2.061103in}{1.783889in}}%
\pgfpathlineto{\pgfqpoint{2.088466in}{1.817543in}}%
\pgfpathlineto{\pgfqpoint{2.045094in}{1.816753in}}%
\pgfpathlineto{\pgfqpoint{2.017298in}{1.797526in}}%
\pgfpathclose%
\pgfusepath{fill}%
\end{pgfscope}%
\begin{pgfscope}%
\pgfpathrectangle{\pgfqpoint{1.072000in}{0.528000in}}{\pgfqpoint{3.696000in}{3.696000in}}%
\pgfusepath{clip}%
\pgfsetbuttcap%
\pgfsetroundjoin%
\definecolor{currentfill}{rgb}{0.404421,0.534643,0.932002}%
\pgfsetfillcolor{currentfill}%
\pgfsetlinewidth{0.000000pt}%
\definecolor{currentstroke}{rgb}{0.000000,0.000000,0.000000}%
\pgfsetstrokecolor{currentstroke}%
\pgfsetdash{}{0pt}%
\pgfpathmoveto{\pgfqpoint{4.063272in}{1.980914in}}%
\pgfpathlineto{\pgfqpoint{4.111923in}{2.047117in}}%
\pgfpathlineto{\pgfqpoint{4.138728in}{2.163241in}}%
\pgfpathlineto{\pgfqpoint{4.089522in}{2.083558in}}%
\pgfpathlineto{\pgfqpoint{4.063272in}{1.980914in}}%
\pgfpathclose%
\pgfusepath{fill}%
\end{pgfscope}%
\begin{pgfscope}%
\pgfpathrectangle{\pgfqpoint{1.072000in}{0.528000in}}{\pgfqpoint{3.696000in}{3.696000in}}%
\pgfusepath{clip}%
\pgfsetbuttcap%
\pgfsetroundjoin%
\definecolor{currentfill}{rgb}{0.409611,0.540759,0.935545}%
\pgfsetfillcolor{currentfill}%
\pgfsetlinewidth{0.000000pt}%
\definecolor{currentstroke}{rgb}{0.000000,0.000000,0.000000}%
\pgfsetstrokecolor{currentstroke}%
\pgfsetdash{}{0pt}%
\pgfpathmoveto{\pgfqpoint{2.211641in}{1.999611in}}%
\pgfpathlineto{\pgfqpoint{2.253089in}{2.106134in}}%
\pgfpathlineto{\pgfqpoint{2.279930in}{2.152166in}}%
\pgfpathlineto{\pgfqpoint{2.238542in}{2.039452in}}%
\pgfpathlineto{\pgfqpoint{2.211641in}{1.999611in}}%
\pgfpathclose%
\pgfusepath{fill}%
\end{pgfscope}%
\begin{pgfscope}%
\pgfpathrectangle{\pgfqpoint{1.072000in}{0.528000in}}{\pgfqpoint{3.696000in}{3.696000in}}%
\pgfusepath{clip}%
\pgfsetbuttcap%
\pgfsetroundjoin%
\definecolor{currentfill}{rgb}{0.483854,0.622050,0.974808}%
\pgfsetfillcolor{currentfill}%
\pgfsetlinewidth{0.000000pt}%
\definecolor{currentstroke}{rgb}{0.000000,0.000000,0.000000}%
\pgfsetstrokecolor{currentstroke}%
\pgfsetdash{}{0pt}%
\pgfpathmoveto{\pgfqpoint{3.676931in}{2.276087in}}%
\pgfpathlineto{\pgfqpoint{3.720830in}{2.174570in}}%
\pgfpathlineto{\pgfqpoint{3.742804in}{2.088868in}}%
\pgfpathlineto{\pgfqpoint{3.698809in}{2.173173in}}%
\pgfpathlineto{\pgfqpoint{3.676931in}{2.276087in}}%
\pgfpathclose%
\pgfusepath{fill}%
\end{pgfscope}%
\begin{pgfscope}%
\pgfpathrectangle{\pgfqpoint{1.072000in}{0.528000in}}{\pgfqpoint{3.696000in}{3.696000in}}%
\pgfusepath{clip}%
\pgfsetbuttcap%
\pgfsetroundjoin%
\definecolor{currentfill}{rgb}{0.266381,0.353304,0.801637}%
\pgfsetfillcolor{currentfill}%
\pgfsetlinewidth{0.000000pt}%
\definecolor{currentstroke}{rgb}{0.000000,0.000000,0.000000}%
\pgfsetstrokecolor{currentstroke}%
\pgfsetdash{}{0pt}%
\pgfpathmoveto{\pgfqpoint{2.088466in}{1.817543in}}%
\pgfpathlineto{\pgfqpoint{2.131039in}{1.856177in}}%
\pgfpathlineto{\pgfqpoint{2.157925in}{1.906131in}}%
\pgfpathlineto{\pgfqpoint{2.115664in}{1.854089in}}%
\pgfpathlineto{\pgfqpoint{2.088466in}{1.817543in}}%
\pgfpathclose%
\pgfusepath{fill}%
\end{pgfscope}%
\begin{pgfscope}%
\pgfpathrectangle{\pgfqpoint{1.072000in}{0.528000in}}{\pgfqpoint{3.696000in}{3.696000in}}%
\pgfusepath{clip}%
\pgfsetbuttcap%
\pgfsetroundjoin%
\definecolor{currentfill}{rgb}{0.248091,0.326013,0.777669}%
\pgfsetfillcolor{currentfill}%
\pgfsetlinewidth{0.000000pt}%
\definecolor{currentstroke}{rgb}{0.000000,0.000000,0.000000}%
\pgfsetstrokecolor{currentstroke}%
\pgfsetdash{}{0pt}%
\pgfpathmoveto{\pgfqpoint{3.874287in}{1.866691in}}%
\pgfpathlineto{\pgfqpoint{3.918927in}{1.805393in}}%
\pgfpathlineto{\pgfqpoint{3.942376in}{1.808901in}}%
\pgfpathlineto{\pgfqpoint{3.897254in}{1.847957in}}%
\pgfpathlineto{\pgfqpoint{3.874287in}{1.866691in}}%
\pgfpathclose%
\pgfusepath{fill}%
\end{pgfscope}%
\begin{pgfscope}%
\pgfpathrectangle{\pgfqpoint{1.072000in}{0.528000in}}{\pgfqpoint{3.696000in}{3.696000in}}%
\pgfusepath{clip}%
\pgfsetbuttcap%
\pgfsetroundjoin%
\definecolor{currentfill}{rgb}{0.576051,0.708780,0.997755}%
\pgfsetfillcolor{currentfill}%
\pgfsetlinewidth{0.000000pt}%
\definecolor{currentstroke}{rgb}{0.000000,0.000000,0.000000}%
\pgfsetstrokecolor{currentstroke}%
\pgfsetdash{}{0pt}%
\pgfpathmoveto{\pgfqpoint{2.333905in}{2.217018in}}%
\pgfpathlineto{\pgfqpoint{2.375290in}{2.361119in}}%
\pgfpathlineto{\pgfqpoint{2.402482in}{2.377750in}}%
\pgfpathlineto{\pgfqpoint{2.361011in}{2.234938in}}%
\pgfpathlineto{\pgfqpoint{2.333905in}{2.217018in}}%
\pgfpathclose%
\pgfusepath{fill}%
\end{pgfscope}%
\begin{pgfscope}%
\pgfpathrectangle{\pgfqpoint{1.072000in}{0.528000in}}{\pgfqpoint{3.696000in}{3.696000in}}%
\pgfusepath{clip}%
\pgfsetbuttcap%
\pgfsetroundjoin%
\definecolor{currentfill}{rgb}{0.248091,0.326013,0.777669}%
\pgfsetfillcolor{currentfill}%
\pgfsetlinewidth{0.000000pt}%
\definecolor{currentstroke}{rgb}{0.000000,0.000000,0.000000}%
\pgfsetstrokecolor{currentstroke}%
\pgfsetdash{}{0pt}%
\pgfpathmoveto{\pgfqpoint{3.988269in}{1.794414in}}%
\pgfpathlineto{\pgfqpoint{4.035105in}{1.806748in}}%
\pgfpathlineto{\pgfqpoint{4.060147in}{1.868865in}}%
\pgfpathlineto{\pgfqpoint{4.012706in}{1.836948in}}%
\pgfpathlineto{\pgfqpoint{3.988269in}{1.794414in}}%
\pgfpathclose%
\pgfusepath{fill}%
\end{pgfscope}%
\begin{pgfscope}%
\pgfpathrectangle{\pgfqpoint{1.072000in}{0.528000in}}{\pgfqpoint{3.696000in}{3.696000in}}%
\pgfusepath{clip}%
\pgfsetbuttcap%
\pgfsetroundjoin%
\definecolor{currentfill}{rgb}{0.289996,0.386836,0.828926}%
\pgfsetfillcolor{currentfill}%
\pgfsetlinewidth{0.000000pt}%
\definecolor{currentstroke}{rgb}{0.000000,0.000000,0.000000}%
\pgfsetstrokecolor{currentstroke}%
\pgfsetdash{}{0pt}%
\pgfpathmoveto{\pgfqpoint{3.830154in}{1.949142in}}%
\pgfpathlineto{\pgfqpoint{3.874287in}{1.866691in}}%
\pgfpathlineto{\pgfqpoint{3.897254in}{1.847957in}}%
\pgfpathlineto{\pgfqpoint{3.852717in}{1.908317in}}%
\pgfpathlineto{\pgfqpoint{3.830154in}{1.949142in}}%
\pgfpathclose%
\pgfusepath{fill}%
\end{pgfscope}%
\begin{pgfscope}%
\pgfpathrectangle{\pgfqpoint{1.072000in}{0.528000in}}{\pgfqpoint{3.696000in}{3.696000in}}%
\pgfusepath{clip}%
\pgfsetbuttcap%
\pgfsetroundjoin%
\definecolor{currentfill}{rgb}{0.891817,0.851973,0.829085}%
\pgfsetfillcolor{currentfill}%
\pgfsetlinewidth{0.000000pt}%
\definecolor{currentstroke}{rgb}{0.000000,0.000000,0.000000}%
\pgfsetstrokecolor{currentstroke}%
\pgfsetdash{}{0pt}%
\pgfpathmoveto{\pgfqpoint{2.950288in}{2.810158in}}%
\pgfpathlineto{\pgfqpoint{2.995286in}{2.895118in}}%
\pgfpathlineto{\pgfqpoint{3.020982in}{2.761076in}}%
\pgfpathlineto{\pgfqpoint{2.976172in}{2.682563in}}%
\pgfpathlineto{\pgfqpoint{2.950288in}{2.810158in}}%
\pgfpathclose%
\pgfusepath{fill}%
\end{pgfscope}%
\begin{pgfscope}%
\pgfpathrectangle{\pgfqpoint{1.072000in}{0.528000in}}{\pgfqpoint{3.696000in}{3.696000in}}%
\pgfusepath{clip}%
\pgfsetbuttcap%
\pgfsetroundjoin%
\definecolor{currentfill}{rgb}{0.494638,0.633022,0.978983}%
\pgfsetfillcolor{currentfill}%
\pgfsetlinewidth{0.000000pt}%
\definecolor{currentstroke}{rgb}{0.000000,0.000000,0.000000}%
\pgfsetstrokecolor{currentstroke}%
\pgfsetdash{}{0pt}%
\pgfpathmoveto{\pgfqpoint{4.089522in}{2.083558in}}%
\pgfpathlineto{\pgfqpoint{4.138728in}{2.163241in}}%
\pgfpathlineto{\pgfqpoint{4.166136in}{2.295708in}}%
\pgfpathlineto{\pgfqpoint{4.116440in}{2.205716in}}%
\pgfpathlineto{\pgfqpoint{4.089522in}{2.083558in}}%
\pgfpathclose%
\pgfusepath{fill}%
\end{pgfscope}%
\begin{pgfscope}%
\pgfpathrectangle{\pgfqpoint{1.072000in}{0.528000in}}{\pgfqpoint{3.696000in}{3.696000in}}%
\pgfusepath{clip}%
\pgfsetbuttcap%
\pgfsetroundjoin%
\definecolor{currentfill}{rgb}{0.229806,0.298718,0.753683}%
\pgfsetfillcolor{currentfill}%
\pgfsetlinewidth{0.000000pt}%
\definecolor{currentstroke}{rgb}{0.000000,0.000000,0.000000}%
\pgfsetstrokecolor{currentstroke}%
\pgfsetdash{}{0pt}%
\pgfpathmoveto{\pgfqpoint{3.918927in}{1.805393in}}%
\pgfpathlineto{\pgfqpoint{3.964277in}{1.769244in}}%
\pgfpathlineto{\pgfqpoint{3.988269in}{1.794414in}}%
\pgfpathlineto{\pgfqpoint{3.942376in}{1.808901in}}%
\pgfpathlineto{\pgfqpoint{3.918927in}{1.805393in}}%
\pgfpathclose%
\pgfusepath{fill}%
\end{pgfscope}%
\begin{pgfscope}%
\pgfpathrectangle{\pgfqpoint{1.072000in}{0.528000in}}{\pgfqpoint{3.696000in}{3.696000in}}%
\pgfusepath{clip}%
\pgfsetbuttcap%
\pgfsetroundjoin%
\definecolor{currentfill}{rgb}{0.763363,0.835092,0.955658}%
\pgfsetfillcolor{currentfill}%
\pgfsetlinewidth{0.000000pt}%
\definecolor{currentstroke}{rgb}{0.000000,0.000000,0.000000}%
\pgfsetstrokecolor{currentstroke}%
\pgfsetdash{}{0pt}%
\pgfpathmoveto{\pgfqpoint{2.526340in}{2.505800in}}%
\pgfpathlineto{\pgfqpoint{2.568744in}{2.662122in}}%
\pgfpathlineto{\pgfqpoint{2.596174in}{2.621099in}}%
\pgfpathlineto{\pgfqpoint{2.553684in}{2.472329in}}%
\pgfpathlineto{\pgfqpoint{2.526340in}{2.505800in}}%
\pgfpathclose%
\pgfusepath{fill}%
\end{pgfscope}%
\begin{pgfscope}%
\pgfpathrectangle{\pgfqpoint{1.072000in}{0.528000in}}{\pgfqpoint{3.696000in}{3.696000in}}%
\pgfusepath{clip}%
\pgfsetbuttcap%
\pgfsetroundjoin%
\definecolor{currentfill}{rgb}{0.891817,0.851973,0.829085}%
\pgfsetfillcolor{currentfill}%
\pgfsetlinewidth{0.000000pt}%
\definecolor{currentstroke}{rgb}{0.000000,0.000000,0.000000}%
\pgfsetstrokecolor{currentstroke}%
\pgfsetdash{}{0pt}%
\pgfpathmoveto{\pgfqpoint{3.066099in}{2.829021in}}%
\pgfpathlineto{\pgfqpoint{3.111472in}{2.884175in}}%
\pgfpathlineto{\pgfqpoint{3.136363in}{2.736266in}}%
\pgfpathlineto{\pgfqpoint{3.091235in}{2.686021in}}%
\pgfpathlineto{\pgfqpoint{3.066099in}{2.829021in}}%
\pgfpathclose%
\pgfusepath{fill}%
\end{pgfscope}%
\begin{pgfscope}%
\pgfpathrectangle{\pgfqpoint{1.072000in}{0.528000in}}{\pgfqpoint{3.696000in}{3.696000in}}%
\pgfusepath{clip}%
\pgfsetbuttcap%
\pgfsetroundjoin%
\definecolor{currentfill}{rgb}{0.859385,0.864431,0.872111}%
\pgfsetfillcolor{currentfill}%
\pgfsetlinewidth{0.000000pt}%
\definecolor{currentstroke}{rgb}{0.000000,0.000000,0.000000}%
\pgfsetstrokecolor{currentstroke}%
\pgfsetdash{}{0pt}%
\pgfpathmoveto{\pgfqpoint{3.227126in}{2.798275in}}%
\pgfpathlineto{\pgfqpoint{3.272646in}{2.808013in}}%
\pgfpathlineto{\pgfqpoint{3.296344in}{2.650535in}}%
\pgfpathlineto{\pgfqpoint{3.251117in}{2.642852in}}%
\pgfpathlineto{\pgfqpoint{3.227126in}{2.798275in}}%
\pgfpathclose%
\pgfusepath{fill}%
\end{pgfscope}%
\begin{pgfscope}%
\pgfpathrectangle{\pgfqpoint{1.072000in}{0.528000in}}{\pgfqpoint{3.696000in}{3.696000in}}%
\pgfusepath{clip}%
\pgfsetbuttcap%
\pgfsetroundjoin%
\definecolor{currentfill}{rgb}{0.698454,0.799450,0.984577}%
\pgfsetfillcolor{currentfill}%
\pgfsetlinewidth{0.000000pt}%
\definecolor{currentstroke}{rgb}{0.000000,0.000000,0.000000}%
\pgfsetstrokecolor{currentstroke}%
\pgfsetdash{}{0pt}%
\pgfpathmoveto{\pgfqpoint{2.429742in}{2.382636in}}%
\pgfpathlineto{\pgfqpoint{2.471545in}{2.540037in}}%
\pgfpathlineto{\pgfqpoint{2.498945in}{2.528893in}}%
\pgfpathlineto{\pgfqpoint{2.457032in}{2.376504in}}%
\pgfpathlineto{\pgfqpoint{2.429742in}{2.382636in}}%
\pgfpathclose%
\pgfusepath{fill}%
\end{pgfscope}%
\begin{pgfscope}%
\pgfpathrectangle{\pgfqpoint{1.072000in}{0.528000in}}{\pgfqpoint{3.696000in}{3.696000in}}%
\pgfusepath{clip}%
\pgfsetbuttcap%
\pgfsetroundjoin%
\definecolor{currentfill}{rgb}{0.248091,0.326013,0.777669}%
\pgfsetfillcolor{currentfill}%
\pgfsetlinewidth{0.000000pt}%
\definecolor{currentstroke}{rgb}{0.000000,0.000000,0.000000}%
\pgfsetstrokecolor{currentstroke}%
\pgfsetdash{}{0pt}%
\pgfpathmoveto{\pgfqpoint{1.943864in}{1.850190in}}%
\pgfpathlineto{\pgfqpoint{1.989196in}{1.784860in}}%
\pgfpathlineto{\pgfqpoint{2.017298in}{1.797526in}}%
\pgfpathlineto{\pgfqpoint{1.972485in}{1.848901in}}%
\pgfpathlineto{\pgfqpoint{1.943864in}{1.850190in}}%
\pgfpathclose%
\pgfusepath{fill}%
\end{pgfscope}%
\begin{pgfscope}%
\pgfpathrectangle{\pgfqpoint{1.072000in}{0.528000in}}{\pgfqpoint{3.696000in}{3.696000in}}%
\pgfusepath{clip}%
\pgfsetbuttcap%
\pgfsetroundjoin%
\definecolor{currentfill}{rgb}{0.348323,0.465711,0.888346}%
\pgfsetfillcolor{currentfill}%
\pgfsetlinewidth{0.000000pt}%
\definecolor{currentstroke}{rgb}{0.000000,0.000000,0.000000}%
\pgfsetstrokecolor{currentstroke}%
\pgfsetdash{}{0pt}%
\pgfpathmoveto{\pgfqpoint{3.786330in}{2.047995in}}%
\pgfpathlineto{\pgfqpoint{3.830154in}{1.949142in}}%
\pgfpathlineto{\pgfqpoint{3.852717in}{1.908317in}}%
\pgfpathlineto{\pgfqpoint{3.808581in}{1.985941in}}%
\pgfpathlineto{\pgfqpoint{3.786330in}{2.047995in}}%
\pgfpathclose%
\pgfusepath{fill}%
\end{pgfscope}%
\begin{pgfscope}%
\pgfpathrectangle{\pgfqpoint{1.072000in}{0.528000in}}{\pgfqpoint{3.696000in}{3.696000in}}%
\pgfusepath{clip}%
\pgfsetbuttcap%
\pgfsetroundjoin%
\definecolor{currentfill}{rgb}{0.565182,0.699438,0.996635}%
\pgfsetfillcolor{currentfill}%
\pgfsetlinewidth{0.000000pt}%
\definecolor{currentstroke}{rgb}{0.000000,0.000000,0.000000}%
\pgfsetstrokecolor{currentstroke}%
\pgfsetdash{}{0pt}%
\pgfpathmoveto{\pgfqpoint{3.632877in}{2.376424in}}%
\pgfpathlineto{\pgfqpoint{3.676931in}{2.276087in}}%
\pgfpathlineto{\pgfqpoint{3.698809in}{2.173173in}}%
\pgfpathlineto{\pgfqpoint{3.654751in}{2.258697in}}%
\pgfpathlineto{\pgfqpoint{3.632877in}{2.376424in}}%
\pgfpathclose%
\pgfusepath{fill}%
\end{pgfscope}%
\begin{pgfscope}%
\pgfpathrectangle{\pgfqpoint{1.072000in}{0.528000in}}{\pgfqpoint{3.696000in}{3.696000in}}%
\pgfusepath{clip}%
\pgfsetbuttcap%
\pgfsetroundjoin%
\definecolor{currentfill}{rgb}{0.388852,0.516298,0.921373}%
\pgfsetfillcolor{currentfill}%
\pgfsetlinewidth{0.000000pt}%
\definecolor{currentstroke}{rgb}{0.000000,0.000000,0.000000}%
\pgfsetstrokecolor{currentstroke}%
\pgfsetdash{}{0pt}%
\pgfpathmoveto{\pgfqpoint{2.184778in}{1.954628in}}%
\pgfpathlineto{\pgfqpoint{2.226342in}{2.052452in}}%
\pgfpathlineto{\pgfqpoint{2.253089in}{2.106134in}}%
\pgfpathlineto{\pgfqpoint{2.211641in}{1.999611in}}%
\pgfpathlineto{\pgfqpoint{2.184778in}{1.954628in}}%
\pgfpathclose%
\pgfusepath{fill}%
\end{pgfscope}%
\begin{pgfscope}%
\pgfpathrectangle{\pgfqpoint{1.072000in}{0.528000in}}{\pgfqpoint{3.696000in}{3.696000in}}%
\pgfusepath{clip}%
\pgfsetbuttcap%
\pgfsetroundjoin%
\definecolor{currentfill}{rgb}{0.248091,0.326013,0.777669}%
\pgfsetfillcolor{currentfill}%
\pgfsetlinewidth{0.000000pt}%
\definecolor{currentstroke}{rgb}{0.000000,0.000000,0.000000}%
\pgfsetstrokecolor{currentstroke}%
\pgfsetdash{}{0pt}%
\pgfpathmoveto{\pgfqpoint{2.061103in}{1.783889in}}%
\pgfpathlineto{\pgfqpoint{2.104064in}{1.807187in}}%
\pgfpathlineto{\pgfqpoint{2.131039in}{1.856177in}}%
\pgfpathlineto{\pgfqpoint{2.088466in}{1.817543in}}%
\pgfpathlineto{\pgfqpoint{2.061103in}{1.783889in}}%
\pgfpathclose%
\pgfusepath{fill}%
\end{pgfscope}%
\begin{pgfscope}%
\pgfpathrectangle{\pgfqpoint{1.072000in}{0.528000in}}{\pgfqpoint{3.696000in}{3.696000in}}%
\pgfusepath{clip}%
\pgfsetbuttcap%
\pgfsetroundjoin%
\definecolor{currentfill}{rgb}{0.229806,0.298718,0.753683}%
\pgfsetfillcolor{currentfill}%
\pgfsetlinewidth{0.000000pt}%
\definecolor{currentstroke}{rgb}{0.000000,0.000000,0.000000}%
\pgfsetstrokecolor{currentstroke}%
\pgfsetdash{}{0pt}%
\pgfpathmoveto{\pgfqpoint{1.989196in}{1.784860in}}%
\pgfpathlineto{\pgfqpoint{2.033501in}{1.755503in}}%
\pgfpathlineto{\pgfqpoint{2.061103in}{1.783889in}}%
\pgfpathlineto{\pgfqpoint{2.017298in}{1.797526in}}%
\pgfpathlineto{\pgfqpoint{1.989196in}{1.784860in}}%
\pgfpathclose%
\pgfusepath{fill}%
\end{pgfscope}%
\begin{pgfscope}%
\pgfpathrectangle{\pgfqpoint{1.072000in}{0.528000in}}{\pgfqpoint{3.696000in}{3.696000in}}%
\pgfusepath{clip}%
\pgfsetbuttcap%
\pgfsetroundjoin%
\definecolor{currentfill}{rgb}{0.229806,0.298718,0.753683}%
\pgfsetfillcolor{currentfill}%
\pgfsetlinewidth{0.000000pt}%
\definecolor{currentstroke}{rgb}{0.000000,0.000000,0.000000}%
\pgfsetstrokecolor{currentstroke}%
\pgfsetdash{}{0pt}%
\pgfpathmoveto{\pgfqpoint{3.964277in}{1.769244in}}%
\pgfpathlineto{\pgfqpoint{4.010533in}{1.761214in}}%
\pgfpathlineto{\pgfqpoint{4.035105in}{1.806748in}}%
\pgfpathlineto{\pgfqpoint{3.988269in}{1.794414in}}%
\pgfpathlineto{\pgfqpoint{3.964277in}{1.769244in}}%
\pgfpathclose%
\pgfusepath{fill}%
\end{pgfscope}%
\begin{pgfscope}%
\pgfpathrectangle{\pgfqpoint{1.072000in}{0.528000in}}{\pgfqpoint{3.696000in}{3.696000in}}%
\pgfusepath{clip}%
\pgfsetbuttcap%
\pgfsetroundjoin%
\definecolor{currentfill}{rgb}{0.603162,0.731527,0.999565}%
\pgfsetfillcolor{currentfill}%
\pgfsetlinewidth{0.000000pt}%
\definecolor{currentstroke}{rgb}{0.000000,0.000000,0.000000}%
\pgfsetstrokecolor{currentstroke}%
\pgfsetdash{}{0pt}%
\pgfpathmoveto{\pgfqpoint{4.116440in}{2.205716in}}%
\pgfpathlineto{\pgfqpoint{4.166136in}{2.295708in}}%
\pgfpathlineto{\pgfqpoint{4.194087in}{2.441947in}}%
\pgfpathlineto{\pgfqpoint{4.143993in}{2.345262in}}%
\pgfpathlineto{\pgfqpoint{4.116440in}{2.205716in}}%
\pgfpathclose%
\pgfusepath{fill}%
\end{pgfscope}%
\begin{pgfscope}%
\pgfpathrectangle{\pgfqpoint{1.072000in}{0.528000in}}{\pgfqpoint{3.696000in}{3.696000in}}%
\pgfusepath{clip}%
\pgfsetbuttcap%
\pgfsetroundjoin%
\definecolor{currentfill}{rgb}{0.304174,0.406945,0.845263}%
\pgfsetfillcolor{currentfill}%
\pgfsetlinewidth{0.000000pt}%
\definecolor{currentstroke}{rgb}{0.000000,0.000000,0.000000}%
\pgfsetstrokecolor{currentstroke}%
\pgfsetdash{}{0pt}%
\pgfpathmoveto{\pgfqpoint{1.897418in}{1.949850in}}%
\pgfpathlineto{\pgfqpoint{1.943864in}{1.850190in}}%
\pgfpathlineto{\pgfqpoint{1.972485in}{1.848901in}}%
\pgfpathlineto{\pgfqpoint{1.926554in}{1.936834in}}%
\pgfpathlineto{\pgfqpoint{1.897418in}{1.949850in}}%
\pgfpathclose%
\pgfusepath{fill}%
\end{pgfscope}%
\begin{pgfscope}%
\pgfpathrectangle{\pgfqpoint{1.072000in}{0.528000in}}{\pgfqpoint{3.696000in}{3.696000in}}%
\pgfusepath{clip}%
\pgfsetbuttcap%
\pgfsetroundjoin%
\definecolor{currentfill}{rgb}{0.576051,0.708780,0.997755}%
\pgfsetfillcolor{currentfill}%
\pgfsetlinewidth{0.000000pt}%
\definecolor{currentstroke}{rgb}{0.000000,0.000000,0.000000}%
\pgfsetstrokecolor{currentstroke}%
\pgfsetdash{}{0pt}%
\pgfpathmoveto{\pgfqpoint{2.306871in}{2.189359in}}%
\pgfpathlineto{\pgfqpoint{2.348197in}{2.332471in}}%
\pgfpathlineto{\pgfqpoint{2.375290in}{2.361119in}}%
\pgfpathlineto{\pgfqpoint{2.333905in}{2.217018in}}%
\pgfpathlineto{\pgfqpoint{2.306871in}{2.189359in}}%
\pgfpathclose%
\pgfusepath{fill}%
\end{pgfscope}%
\begin{pgfscope}%
\pgfpathrectangle{\pgfqpoint{1.072000in}{0.528000in}}{\pgfqpoint{3.696000in}{3.696000in}}%
\pgfusepath{clip}%
\pgfsetbuttcap%
\pgfsetroundjoin%
\definecolor{currentfill}{rgb}{0.640828,0.760752,0.997846}%
\pgfsetfillcolor{currentfill}%
\pgfsetlinewidth{0.000000pt}%
\definecolor{currentstroke}{rgb}{0.000000,0.000000,0.000000}%
\pgfsetstrokecolor{currentstroke}%
\pgfsetdash{}{0pt}%
\pgfpathmoveto{\pgfqpoint{3.588604in}{2.471685in}}%
\pgfpathlineto{\pgfqpoint{3.632877in}{2.376424in}}%
\pgfpathlineto{\pgfqpoint{3.654751in}{2.258697in}}%
\pgfpathlineto{\pgfqpoint{3.610557in}{2.341691in}}%
\pgfpathlineto{\pgfqpoint{3.588604in}{2.471685in}}%
\pgfpathclose%
\pgfusepath{fill}%
\end{pgfscope}%
\begin{pgfscope}%
\pgfpathrectangle{\pgfqpoint{1.072000in}{0.528000in}}{\pgfqpoint{3.696000in}{3.696000in}}%
\pgfusepath{clip}%
\pgfsetbuttcap%
\pgfsetroundjoin%
\definecolor{currentfill}{rgb}{0.363461,0.484784,0.901019}%
\pgfsetfillcolor{currentfill}%
\pgfsetlinewidth{0.000000pt}%
\definecolor{currentstroke}{rgb}{0.000000,0.000000,0.000000}%
\pgfsetstrokecolor{currentstroke}%
\pgfsetdash{}{0pt}%
\pgfpathmoveto{\pgfqpoint{2.157925in}{1.906131in}}%
\pgfpathlineto{\pgfqpoint{2.199671in}{1.992805in}}%
\pgfpathlineto{\pgfqpoint{2.226342in}{2.052452in}}%
\pgfpathlineto{\pgfqpoint{2.184778in}{1.954628in}}%
\pgfpathlineto{\pgfqpoint{2.157925in}{1.906131in}}%
\pgfpathclose%
\pgfusepath{fill}%
\end{pgfscope}%
\begin{pgfscope}%
\pgfpathrectangle{\pgfqpoint{1.072000in}{0.528000in}}{\pgfqpoint{3.696000in}{3.696000in}}%
\pgfusepath{clip}%
\pgfsetbuttcap%
\pgfsetroundjoin%
\definecolor{currentfill}{rgb}{0.871493,0.862309,0.857016}%
\pgfsetfillcolor{currentfill}%
\pgfsetlinewidth{0.000000pt}%
\definecolor{currentstroke}{rgb}{0.000000,0.000000,0.000000}%
\pgfsetstrokecolor{currentstroke}%
\pgfsetdash{}{0pt}%
\pgfpathmoveto{\pgfqpoint{3.272646in}{2.808013in}}%
\pgfpathlineto{\pgfqpoint{3.318179in}{2.802905in}}%
\pgfpathlineto{\pgfqpoint{3.341581in}{2.644705in}}%
\pgfpathlineto{\pgfqpoint{3.296344in}{2.650535in}}%
\pgfpathlineto{\pgfqpoint{3.272646in}{2.808013in}}%
\pgfpathclose%
\pgfusepath{fill}%
\end{pgfscope}%
\begin{pgfscope}%
\pgfpathrectangle{\pgfqpoint{1.072000in}{0.528000in}}{\pgfqpoint{3.696000in}{3.696000in}}%
\pgfusepath{clip}%
\pgfsetbuttcap%
\pgfsetroundjoin%
\definecolor{currentfill}{rgb}{0.425199,0.559058,0.946061}%
\pgfsetfillcolor{currentfill}%
\pgfsetlinewidth{0.000000pt}%
\definecolor{currentstroke}{rgb}{0.000000,0.000000,0.000000}%
\pgfsetstrokecolor{currentstroke}%
\pgfsetdash{}{0pt}%
\pgfpathmoveto{\pgfqpoint{3.742632in}{2.158050in}}%
\pgfpathlineto{\pgfqpoint{3.786330in}{2.047995in}}%
\pgfpathlineto{\pgfqpoint{3.808581in}{1.985941in}}%
\pgfpathlineto{\pgfqpoint{3.764671in}{2.076282in}}%
\pgfpathlineto{\pgfqpoint{3.742632in}{2.158050in}}%
\pgfpathclose%
\pgfusepath{fill}%
\end{pgfscope}%
\begin{pgfscope}%
\pgfpathrectangle{\pgfqpoint{1.072000in}{0.528000in}}{\pgfqpoint{3.696000in}{3.696000in}}%
\pgfusepath{clip}%
\pgfsetbuttcap%
\pgfsetroundjoin%
\definecolor{currentfill}{rgb}{0.839351,0.861167,0.894494}%
\pgfsetfillcolor{currentfill}%
\pgfsetlinewidth{0.000000pt}%
\definecolor{currentstroke}{rgb}{0.000000,0.000000,0.000000}%
\pgfsetstrokecolor{currentstroke}%
\pgfsetdash{}{0pt}%
\pgfpathmoveto{\pgfqpoint{2.596174in}{2.621099in}}%
\pgfpathlineto{\pgfqpoint{2.639138in}{2.773757in}}%
\pgfpathlineto{\pgfqpoint{2.666510in}{2.714938in}}%
\pgfpathlineto{\pgfqpoint{2.623506in}{2.569503in}}%
\pgfpathlineto{\pgfqpoint{2.596174in}{2.621099in}}%
\pgfpathclose%
\pgfusepath{fill}%
\end{pgfscope}%
\begin{pgfscope}%
\pgfpathrectangle{\pgfqpoint{1.072000in}{0.528000in}}{\pgfqpoint{3.696000in}{3.696000in}}%
\pgfusepath{clip}%
\pgfsetbuttcap%
\pgfsetroundjoin%
\definecolor{currentfill}{rgb}{0.919376,0.831273,0.782874}%
\pgfsetfillcolor{currentfill}%
\pgfsetlinewidth{0.000000pt}%
\definecolor{currentstroke}{rgb}{0.000000,0.000000,0.000000}%
\pgfsetstrokecolor{currentstroke}%
\pgfsetdash{}{0pt}%
\pgfpathmoveto{\pgfqpoint{2.879315in}{2.830657in}}%
\pgfpathlineto{\pgfqpoint{2.924090in}{2.931803in}}%
\pgfpathlineto{\pgfqpoint{2.950288in}{2.810158in}}%
\pgfpathlineto{\pgfqpoint{2.905660in}{2.714794in}}%
\pgfpathlineto{\pgfqpoint{2.879315in}{2.830657in}}%
\pgfpathclose%
\pgfusepath{fill}%
\end{pgfscope}%
\begin{pgfscope}%
\pgfpathrectangle{\pgfqpoint{1.072000in}{0.528000in}}{\pgfqpoint{3.696000in}{3.696000in}}%
\pgfusepath{clip}%
\pgfsetbuttcap%
\pgfsetroundjoin%
\definecolor{currentfill}{rgb}{0.338377,0.452819,0.879317}%
\pgfsetfillcolor{currentfill}%
\pgfsetlinewidth{0.000000pt}%
\definecolor{currentstroke}{rgb}{0.000000,0.000000,0.000000}%
\pgfsetstrokecolor{currentstroke}%
\pgfsetdash{}{0pt}%
\pgfpathmoveto{\pgfqpoint{4.060147in}{1.868865in}}%
\pgfpathlineto{\pgfqpoint{4.108671in}{1.926312in}}%
\pgfpathlineto{\pgfqpoint{4.134849in}{2.021811in}}%
\pgfpathlineto{\pgfqpoint{4.085736in}{1.948834in}}%
\pgfpathlineto{\pgfqpoint{4.060147in}{1.868865in}}%
\pgfpathclose%
\pgfusepath{fill}%
\end{pgfscope}%
\begin{pgfscope}%
\pgfpathrectangle{\pgfqpoint{1.072000in}{0.528000in}}{\pgfqpoint{3.696000in}{3.696000in}}%
\pgfusepath{clip}%
\pgfsetbuttcap%
\pgfsetroundjoin%
\definecolor{currentfill}{rgb}{0.409611,0.540759,0.935545}%
\pgfsetfillcolor{currentfill}%
\pgfsetlinewidth{0.000000pt}%
\definecolor{currentstroke}{rgb}{0.000000,0.000000,0.000000}%
\pgfsetstrokecolor{currentstroke}%
\pgfsetdash{}{0pt}%
\pgfpathmoveto{\pgfqpoint{4.085736in}{1.948834in}}%
\pgfpathlineto{\pgfqpoint{4.134849in}{2.021811in}}%
\pgfpathlineto{\pgfqpoint{4.161589in}{2.133292in}}%
\pgfpathlineto{\pgfqpoint{4.111923in}{2.047117in}}%
\pgfpathlineto{\pgfqpoint{4.085736in}{1.948834in}}%
\pgfpathclose%
\pgfusepath{fill}%
\end{pgfscope}%
\begin{pgfscope}%
\pgfpathrectangle{\pgfqpoint{1.072000in}{0.528000in}}{\pgfqpoint{3.696000in}{3.696000in}}%
\pgfusepath{clip}%
\pgfsetbuttcap%
\pgfsetroundjoin%
\definecolor{currentfill}{rgb}{0.289996,0.386836,0.828926}%
\pgfsetfillcolor{currentfill}%
\pgfsetlinewidth{0.000000pt}%
\definecolor{currentstroke}{rgb}{0.000000,0.000000,0.000000}%
\pgfsetstrokecolor{currentstroke}%
\pgfsetdash{}{0pt}%
\pgfpathmoveto{\pgfqpoint{4.035105in}{1.806748in}}%
\pgfpathlineto{\pgfqpoint{4.083028in}{1.846931in}}%
\pgfpathlineto{\pgfqpoint{4.108671in}{1.926312in}}%
\pgfpathlineto{\pgfqpoint{4.060147in}{1.868865in}}%
\pgfpathlineto{\pgfqpoint{4.035105in}{1.806748in}}%
\pgfpathclose%
\pgfusepath{fill}%
\end{pgfscope}%
\begin{pgfscope}%
\pgfpathrectangle{\pgfqpoint{1.072000in}{0.528000in}}{\pgfqpoint{3.696000in}{3.696000in}}%
\pgfusepath{clip}%
\pgfsetbuttcap%
\pgfsetroundjoin%
\definecolor{currentfill}{rgb}{0.238948,0.312365,0.765676}%
\pgfsetfillcolor{currentfill}%
\pgfsetlinewidth{0.000000pt}%
\definecolor{currentstroke}{rgb}{0.000000,0.000000,0.000000}%
\pgfsetstrokecolor{currentstroke}%
\pgfsetdash{}{0pt}%
\pgfpathmoveto{\pgfqpoint{2.033501in}{1.755503in}}%
\pgfpathlineto{\pgfqpoint{2.076924in}{1.761869in}}%
\pgfpathlineto{\pgfqpoint{2.104064in}{1.807187in}}%
\pgfpathlineto{\pgfqpoint{2.061103in}{1.783889in}}%
\pgfpathlineto{\pgfqpoint{2.033501in}{1.755503in}}%
\pgfpathclose%
\pgfusepath{fill}%
\end{pgfscope}%
\begin{pgfscope}%
\pgfpathrectangle{\pgfqpoint{1.072000in}{0.528000in}}{\pgfqpoint{3.696000in}{3.696000in}}%
\pgfusepath{clip}%
\pgfsetbuttcap%
\pgfsetroundjoin%
\definecolor{currentfill}{rgb}{0.243520,0.319189,0.771672}%
\pgfsetfillcolor{currentfill}%
\pgfsetlinewidth{0.000000pt}%
\definecolor{currentstroke}{rgb}{0.000000,0.000000,0.000000}%
\pgfsetstrokecolor{currentstroke}%
\pgfsetdash{}{0pt}%
\pgfpathmoveto{\pgfqpoint{3.895756in}{1.816485in}}%
\pgfpathlineto{\pgfqpoint{3.940620in}{1.758891in}}%
\pgfpathlineto{\pgfqpoint{3.964277in}{1.769244in}}%
\pgfpathlineto{\pgfqpoint{3.918927in}{1.805393in}}%
\pgfpathlineto{\pgfqpoint{3.895756in}{1.816485in}}%
\pgfpathclose%
\pgfusepath{fill}%
\end{pgfscope}%
\begin{pgfscope}%
\pgfpathrectangle{\pgfqpoint{1.072000in}{0.528000in}}{\pgfqpoint{3.696000in}{3.696000in}}%
\pgfusepath{clip}%
\pgfsetbuttcap%
\pgfsetroundjoin%
\definecolor{currentfill}{rgb}{0.280550,0.373423,0.818011}%
\pgfsetfillcolor{currentfill}%
\pgfsetlinewidth{0.000000pt}%
\definecolor{currentstroke}{rgb}{0.000000,0.000000,0.000000}%
\pgfsetstrokecolor{currentstroke}%
\pgfsetdash{}{0pt}%
\pgfpathmoveto{\pgfqpoint{3.851531in}{1.899415in}}%
\pgfpathlineto{\pgfqpoint{3.895756in}{1.816485in}}%
\pgfpathlineto{\pgfqpoint{3.918927in}{1.805393in}}%
\pgfpathlineto{\pgfqpoint{3.874287in}{1.866691in}}%
\pgfpathlineto{\pgfqpoint{3.851531in}{1.899415in}}%
\pgfpathclose%
\pgfusepath{fill}%
\end{pgfscope}%
\begin{pgfscope}%
\pgfpathrectangle{\pgfqpoint{1.072000in}{0.528000in}}{\pgfqpoint{3.696000in}{3.696000in}}%
\pgfusepath{clip}%
\pgfsetbuttcap%
\pgfsetroundjoin%
\definecolor{currentfill}{rgb}{0.708720,0.805721,0.981117}%
\pgfsetfillcolor{currentfill}%
\pgfsetlinewidth{0.000000pt}%
\definecolor{currentstroke}{rgb}{0.000000,0.000000,0.000000}%
\pgfsetstrokecolor{currentstroke}%
\pgfsetdash{}{0pt}%
\pgfpathmoveto{\pgfqpoint{3.544078in}{2.558631in}}%
\pgfpathlineto{\pgfqpoint{3.588604in}{2.471685in}}%
\pgfpathlineto{\pgfqpoint{3.610557in}{2.341691in}}%
\pgfpathlineto{\pgfqpoint{3.566179in}{2.418898in}}%
\pgfpathlineto{\pgfqpoint{3.544078in}{2.558631in}}%
\pgfpathclose%
\pgfusepath{fill}%
\end{pgfscope}%
\begin{pgfscope}%
\pgfpathrectangle{\pgfqpoint{1.072000in}{0.528000in}}{\pgfqpoint{3.696000in}{3.696000in}}%
\pgfusepath{clip}%
\pgfsetbuttcap%
\pgfsetroundjoin%
\definecolor{currentfill}{rgb}{0.333490,0.446265,0.874452}%
\pgfsetfillcolor{currentfill}%
\pgfsetlinewidth{0.000000pt}%
\definecolor{currentstroke}{rgb}{0.000000,0.000000,0.000000}%
\pgfsetstrokecolor{currentstroke}%
\pgfsetdash{}{0pt}%
\pgfpathmoveto{\pgfqpoint{2.131039in}{1.856177in}}%
\pgfpathlineto{\pgfqpoint{2.173038in}{1.929362in}}%
\pgfpathlineto{\pgfqpoint{2.199671in}{1.992805in}}%
\pgfpathlineto{\pgfqpoint{2.157925in}{1.906131in}}%
\pgfpathlineto{\pgfqpoint{2.131039in}{1.856177in}}%
\pgfpathclose%
\pgfusepath{fill}%
\end{pgfscope}%
\begin{pgfscope}%
\pgfpathrectangle{\pgfqpoint{1.072000in}{0.528000in}}{\pgfqpoint{3.696000in}{3.696000in}}%
\pgfusepath{clip}%
\pgfsetbuttcap%
\pgfsetroundjoin%
\definecolor{currentfill}{rgb}{0.724041,0.814910,0.975651}%
\pgfsetfillcolor{currentfill}%
\pgfsetlinewidth{0.000000pt}%
\definecolor{currentstroke}{rgb}{0.000000,0.000000,0.000000}%
\pgfsetstrokecolor{currentstroke}%
\pgfsetdash{}{0pt}%
\pgfpathmoveto{\pgfqpoint{4.143993in}{2.345262in}}%
\pgfpathlineto{\pgfqpoint{4.194087in}{2.441947in}}%
\pgfpathlineto{\pgfqpoint{4.222476in}{2.598321in}}%
\pgfpathlineto{\pgfqpoint{4.172101in}{2.498893in}}%
\pgfpathlineto{\pgfqpoint{4.143993in}{2.345262in}}%
\pgfpathclose%
\pgfusepath{fill}%
\end{pgfscope}%
\begin{pgfscope}%
\pgfpathrectangle{\pgfqpoint{1.072000in}{0.528000in}}{\pgfqpoint{3.696000in}{3.696000in}}%
\pgfusepath{clip}%
\pgfsetbuttcap%
\pgfsetroundjoin%
\definecolor{currentfill}{rgb}{0.494638,0.633022,0.978983}%
\pgfsetfillcolor{currentfill}%
\pgfsetlinewidth{0.000000pt}%
\definecolor{currentstroke}{rgb}{0.000000,0.000000,0.000000}%
\pgfsetstrokecolor{currentstroke}%
\pgfsetdash{}{0pt}%
\pgfpathmoveto{\pgfqpoint{4.111923in}{2.047117in}}%
\pgfpathlineto{\pgfqpoint{4.161589in}{2.133292in}}%
\pgfpathlineto{\pgfqpoint{4.188887in}{2.259735in}}%
\pgfpathlineto{\pgfqpoint{4.138728in}{2.163241in}}%
\pgfpathlineto{\pgfqpoint{4.111923in}{2.047117in}}%
\pgfpathclose%
\pgfusepath{fill}%
\end{pgfscope}%
\begin{pgfscope}%
\pgfpathrectangle{\pgfqpoint{1.072000in}{0.528000in}}{\pgfqpoint{3.696000in}{3.696000in}}%
\pgfusepath{clip}%
\pgfsetbuttcap%
\pgfsetroundjoin%
\definecolor{currentfill}{rgb}{0.928116,0.822197,0.765141}%
\pgfsetfillcolor{currentfill}%
\pgfsetlinewidth{0.000000pt}%
\definecolor{currentstroke}{rgb}{0.000000,0.000000,0.000000}%
\pgfsetstrokecolor{currentstroke}%
\pgfsetdash{}{0pt}%
\pgfpathmoveto{\pgfqpoint{3.111472in}{2.884175in}}%
\pgfpathlineto{\pgfqpoint{3.157046in}{2.925117in}}%
\pgfpathlineto{\pgfqpoint{3.181679in}{2.774102in}}%
\pgfpathlineto{\pgfqpoint{3.136363in}{2.736266in}}%
\pgfpathlineto{\pgfqpoint{3.111472in}{2.884175in}}%
\pgfpathclose%
\pgfusepath{fill}%
\end{pgfscope}%
\begin{pgfscope}%
\pgfpathrectangle{\pgfqpoint{1.072000in}{0.528000in}}{\pgfqpoint{3.696000in}{3.696000in}}%
\pgfusepath{clip}%
\pgfsetbuttcap%
\pgfsetroundjoin%
\definecolor{currentfill}{rgb}{0.871493,0.862309,0.857016}%
\pgfsetfillcolor{currentfill}%
\pgfsetlinewidth{0.000000pt}%
\definecolor{currentstroke}{rgb}{0.000000,0.000000,0.000000}%
\pgfsetstrokecolor{currentstroke}%
\pgfsetdash{}{0pt}%
\pgfpathmoveto{\pgfqpoint{3.318179in}{2.802905in}}%
\pgfpathlineto{\pgfqpoint{3.363665in}{2.782800in}}%
\pgfpathlineto{\pgfqpoint{3.386772in}{2.625146in}}%
\pgfpathlineto{\pgfqpoint{3.341581in}{2.644705in}}%
\pgfpathlineto{\pgfqpoint{3.318179in}{2.802905in}}%
\pgfpathclose%
\pgfusepath{fill}%
\end{pgfscope}%
\begin{pgfscope}%
\pgfpathrectangle{\pgfqpoint{1.072000in}{0.528000in}}{\pgfqpoint{3.696000in}{3.696000in}}%
\pgfusepath{clip}%
\pgfsetbuttcap%
\pgfsetroundjoin%
\definecolor{currentfill}{rgb}{0.229806,0.298718,0.753683}%
\pgfsetfillcolor{currentfill}%
\pgfsetlinewidth{0.000000pt}%
\definecolor{currentstroke}{rgb}{0.000000,0.000000,0.000000}%
\pgfsetstrokecolor{currentstroke}%
\pgfsetdash{}{0pt}%
\pgfpathmoveto{\pgfqpoint{3.940620in}{1.758891in}}%
\pgfpathlineto{\pgfqpoint{3.986340in}{1.730342in}}%
\pgfpathlineto{\pgfqpoint{4.010533in}{1.761214in}}%
\pgfpathlineto{\pgfqpoint{3.964277in}{1.769244in}}%
\pgfpathlineto{\pgfqpoint{3.940620in}{1.758891in}}%
\pgfpathclose%
\pgfusepath{fill}%
\end{pgfscope}%
\begin{pgfscope}%
\pgfpathrectangle{\pgfqpoint{1.072000in}{0.528000in}}{\pgfqpoint{3.696000in}{3.696000in}}%
\pgfusepath{clip}%
\pgfsetbuttcap%
\pgfsetroundjoin%
\definecolor{currentfill}{rgb}{0.234377,0.305542,0.759680}%
\pgfsetfillcolor{currentfill}%
\pgfsetlinewidth{0.000000pt}%
\definecolor{currentstroke}{rgb}{0.000000,0.000000,0.000000}%
\pgfsetstrokecolor{currentstroke}%
\pgfsetdash{}{0pt}%
\pgfpathmoveto{\pgfqpoint{1.960708in}{1.780790in}}%
\pgfpathlineto{\pgfqpoint{2.005575in}{1.734858in}}%
\pgfpathlineto{\pgfqpoint{2.033501in}{1.755503in}}%
\pgfpathlineto{\pgfqpoint{1.989196in}{1.784860in}}%
\pgfpathlineto{\pgfqpoint{1.960708in}{1.780790in}}%
\pgfpathclose%
\pgfusepath{fill}%
\end{pgfscope}%
\begin{pgfscope}%
\pgfpathrectangle{\pgfqpoint{1.072000in}{0.528000in}}{\pgfqpoint{3.696000in}{3.696000in}}%
\pgfusepath{clip}%
\pgfsetbuttcap%
\pgfsetroundjoin%
\definecolor{currentfill}{rgb}{0.891817,0.851973,0.829085}%
\pgfsetfillcolor{currentfill}%
\pgfsetlinewidth{0.000000pt}%
\definecolor{currentstroke}{rgb}{0.000000,0.000000,0.000000}%
\pgfsetstrokecolor{currentstroke}%
\pgfsetdash{}{0pt}%
\pgfpathmoveto{\pgfqpoint{2.666510in}{2.714938in}}%
\pgfpathlineto{\pgfqpoint{2.710032in}{2.858445in}}%
\pgfpathlineto{\pgfqpoint{2.737250in}{2.782573in}}%
\pgfpathlineto{\pgfqpoint{2.693739in}{2.645564in}}%
\pgfpathlineto{\pgfqpoint{2.666510in}{2.714938in}}%
\pgfpathclose%
\pgfusepath{fill}%
\end{pgfscope}%
\begin{pgfscope}%
\pgfpathrectangle{\pgfqpoint{1.072000in}{0.528000in}}{\pgfqpoint{3.696000in}{3.696000in}}%
\pgfusepath{clip}%
\pgfsetbuttcap%
\pgfsetroundjoin%
\definecolor{currentfill}{rgb}{0.928116,0.822197,0.765141}%
\pgfsetfillcolor{currentfill}%
\pgfsetlinewidth{0.000000pt}%
\definecolor{currentstroke}{rgb}{0.000000,0.000000,0.000000}%
\pgfsetstrokecolor{currentstroke}%
\pgfsetdash{}{0pt}%
\pgfpathmoveto{\pgfqpoint{2.808242in}{2.821479in}}%
\pgfpathlineto{\pgfqpoint{2.852691in}{2.938024in}}%
\pgfpathlineto{\pgfqpoint{2.879315in}{2.830657in}}%
\pgfpathlineto{\pgfqpoint{2.834972in}{2.719632in}}%
\pgfpathlineto{\pgfqpoint{2.808242in}{2.821479in}}%
\pgfpathclose%
\pgfusepath{fill}%
\end{pgfscope}%
\begin{pgfscope}%
\pgfpathrectangle{\pgfqpoint{1.072000in}{0.528000in}}{\pgfqpoint{3.696000in}{3.696000in}}%
\pgfusepath{clip}%
\pgfsetbuttcap%
\pgfsetroundjoin%
\definecolor{currentfill}{rgb}{0.257234,0.339661,0.789661}%
\pgfsetfillcolor{currentfill}%
\pgfsetlinewidth{0.000000pt}%
\definecolor{currentstroke}{rgb}{0.000000,0.000000,0.000000}%
\pgfsetstrokecolor{currentstroke}%
\pgfsetdash{}{0pt}%
\pgfpathmoveto{\pgfqpoint{4.010533in}{1.761214in}}%
\pgfpathlineto{\pgfqpoint{4.057865in}{1.783013in}}%
\pgfpathlineto{\pgfqpoint{4.083028in}{1.846931in}}%
\pgfpathlineto{\pgfqpoint{4.035105in}{1.806748in}}%
\pgfpathlineto{\pgfqpoint{4.010533in}{1.761214in}}%
\pgfpathclose%
\pgfusepath{fill}%
\end{pgfscope}%
\begin{pgfscope}%
\pgfpathrectangle{\pgfqpoint{1.072000in}{0.528000in}}{\pgfqpoint{3.696000in}{3.696000in}}%
\pgfusepath{clip}%
\pgfsetbuttcap%
\pgfsetroundjoin%
\definecolor{currentfill}{rgb}{0.271104,0.360011,0.807095}%
\pgfsetfillcolor{currentfill}%
\pgfsetlinewidth{0.000000pt}%
\definecolor{currentstroke}{rgb}{0.000000,0.000000,0.000000}%
\pgfsetstrokecolor{currentstroke}%
\pgfsetdash{}{0pt}%
\pgfpathmoveto{\pgfqpoint{1.914806in}{1.860588in}}%
\pgfpathlineto{\pgfqpoint{1.960708in}{1.780790in}}%
\pgfpathlineto{\pgfqpoint{1.989196in}{1.784860in}}%
\pgfpathlineto{\pgfqpoint{1.943864in}{1.850190in}}%
\pgfpathlineto{\pgfqpoint{1.914806in}{1.860588in}}%
\pgfpathclose%
\pgfusepath{fill}%
\end{pgfscope}%
\begin{pgfscope}%
\pgfpathrectangle{\pgfqpoint{1.072000in}{0.528000in}}{\pgfqpoint{3.696000in}{3.696000in}}%
\pgfusepath{clip}%
\pgfsetbuttcap%
\pgfsetroundjoin%
\definecolor{currentfill}{rgb}{0.565182,0.699438,0.996635}%
\pgfsetfillcolor{currentfill}%
\pgfsetlinewidth{0.000000pt}%
\definecolor{currentstroke}{rgb}{0.000000,0.000000,0.000000}%
\pgfsetstrokecolor{currentstroke}%
\pgfsetdash{}{0pt}%
\pgfpathmoveto{\pgfqpoint{2.279930in}{2.152166in}}%
\pgfpathlineto{\pgfqpoint{2.321228in}{2.291984in}}%
\pgfpathlineto{\pgfqpoint{2.348197in}{2.332471in}}%
\pgfpathlineto{\pgfqpoint{2.306871in}{2.189359in}}%
\pgfpathlineto{\pgfqpoint{2.279930in}{2.152166in}}%
\pgfpathclose%
\pgfusepath{fill}%
\end{pgfscope}%
\begin{pgfscope}%
\pgfpathrectangle{\pgfqpoint{1.072000in}{0.528000in}}{\pgfqpoint{3.696000in}{3.696000in}}%
\pgfusepath{clip}%
\pgfsetbuttcap%
\pgfsetroundjoin%
\definecolor{currentfill}{rgb}{0.718985,0.811993,0.977656}%
\pgfsetfillcolor{currentfill}%
\pgfsetlinewidth{0.000000pt}%
\definecolor{currentstroke}{rgb}{0.000000,0.000000,0.000000}%
\pgfsetstrokecolor{currentstroke}%
\pgfsetdash{}{0pt}%
\pgfpathmoveto{\pgfqpoint{2.402482in}{2.377750in}}%
\pgfpathlineto{\pgfqpoint{2.444183in}{2.538164in}}%
\pgfpathlineto{\pgfqpoint{2.471545in}{2.540037in}}%
\pgfpathlineto{\pgfqpoint{2.429742in}{2.382636in}}%
\pgfpathlineto{\pgfqpoint{2.402482in}{2.377750in}}%
\pgfpathclose%
\pgfusepath{fill}%
\end{pgfscope}%
\begin{pgfscope}%
\pgfpathrectangle{\pgfqpoint{1.072000in}{0.528000in}}{\pgfqpoint{3.696000in}{3.696000in}}%
\pgfusepath{clip}%
\pgfsetbuttcap%
\pgfsetroundjoin%
\definecolor{currentfill}{rgb}{0.510824,0.649397,0.985079}%
\pgfsetfillcolor{currentfill}%
\pgfsetlinewidth{0.000000pt}%
\definecolor{currentstroke}{rgb}{0.000000,0.000000,0.000000}%
\pgfsetstrokecolor{currentstroke}%
\pgfsetdash{}{0pt}%
\pgfpathmoveto{\pgfqpoint{3.698901in}{2.273983in}}%
\pgfpathlineto{\pgfqpoint{3.742632in}{2.158050in}}%
\pgfpathlineto{\pgfqpoint{3.764671in}{2.076282in}}%
\pgfpathlineto{\pgfqpoint{3.720830in}{2.174570in}}%
\pgfpathlineto{\pgfqpoint{3.698901in}{2.273983in}}%
\pgfpathclose%
\pgfusepath{fill}%
\end{pgfscope}%
\begin{pgfscope}%
\pgfpathrectangle{\pgfqpoint{1.072000in}{0.528000in}}{\pgfqpoint{3.696000in}{3.696000in}}%
\pgfusepath{clip}%
\pgfsetbuttcap%
\pgfsetroundjoin%
\definecolor{currentfill}{rgb}{0.940879,0.805596,0.735167}%
\pgfsetfillcolor{currentfill}%
\pgfsetlinewidth{0.000000pt}%
\definecolor{currentstroke}{rgb}{0.000000,0.000000,0.000000}%
\pgfsetstrokecolor{currentstroke}%
\pgfsetdash{}{0pt}%
\pgfpathmoveto{\pgfqpoint{2.995286in}{2.895118in}}%
\pgfpathlineto{\pgfqpoint{3.040610in}{2.966879in}}%
\pgfpathlineto{\pgfqpoint{3.066099in}{2.829021in}}%
\pgfpathlineto{\pgfqpoint{3.020982in}{2.761076in}}%
\pgfpathlineto{\pgfqpoint{2.995286in}{2.895118in}}%
\pgfpathclose%
\pgfusepath{fill}%
\end{pgfscope}%
\begin{pgfscope}%
\pgfpathrectangle{\pgfqpoint{1.072000in}{0.528000in}}{\pgfqpoint{3.696000in}{3.696000in}}%
\pgfusepath{clip}%
\pgfsetbuttcap%
\pgfsetroundjoin%
\definecolor{currentfill}{rgb}{0.809329,0.852974,0.922323}%
\pgfsetfillcolor{currentfill}%
\pgfsetlinewidth{0.000000pt}%
\definecolor{currentstroke}{rgb}{0.000000,0.000000,0.000000}%
\pgfsetstrokecolor{currentstroke}%
\pgfsetdash{}{0pt}%
\pgfpathmoveto{\pgfqpoint{2.498945in}{2.528893in}}%
\pgfpathlineto{\pgfqpoint{2.541264in}{2.690688in}}%
\pgfpathlineto{\pgfqpoint{2.568744in}{2.662122in}}%
\pgfpathlineto{\pgfqpoint{2.526340in}{2.505800in}}%
\pgfpathlineto{\pgfqpoint{2.498945in}{2.528893in}}%
\pgfpathclose%
\pgfusepath{fill}%
\end{pgfscope}%
\begin{pgfscope}%
\pgfpathrectangle{\pgfqpoint{1.072000in}{0.528000in}}{\pgfqpoint{3.696000in}{3.696000in}}%
\pgfusepath{clip}%
\pgfsetbuttcap%
\pgfsetroundjoin%
\definecolor{currentfill}{rgb}{0.763363,0.835092,0.955658}%
\pgfsetfillcolor{currentfill}%
\pgfsetlinewidth{0.000000pt}%
\definecolor{currentstroke}{rgb}{0.000000,0.000000,0.000000}%
\pgfsetstrokecolor{currentstroke}%
\pgfsetdash{}{0pt}%
\pgfpathmoveto{\pgfqpoint{3.499295in}{2.634738in}}%
\pgfpathlineto{\pgfqpoint{3.544078in}{2.558631in}}%
\pgfpathlineto{\pgfqpoint{3.566179in}{2.418898in}}%
\pgfpathlineto{\pgfqpoint{3.521600in}{2.487643in}}%
\pgfpathlineto{\pgfqpoint{3.499295in}{2.634738in}}%
\pgfpathclose%
\pgfusepath{fill}%
\end{pgfscope}%
\begin{pgfscope}%
\pgfpathrectangle{\pgfqpoint{1.072000in}{0.528000in}}{\pgfqpoint{3.696000in}{3.696000in}}%
\pgfusepath{clip}%
\pgfsetbuttcap%
\pgfsetroundjoin%
\definecolor{currentfill}{rgb}{0.343278,0.459354,0.884122}%
\pgfsetfillcolor{currentfill}%
\pgfsetlinewidth{0.000000pt}%
\definecolor{currentstroke}{rgb}{0.000000,0.000000,0.000000}%
\pgfsetstrokecolor{currentstroke}%
\pgfsetdash{}{0pt}%
\pgfpathmoveto{\pgfqpoint{3.807727in}{2.002932in}}%
\pgfpathlineto{\pgfqpoint{3.851531in}{1.899415in}}%
\pgfpathlineto{\pgfqpoint{3.874287in}{1.866691in}}%
\pgfpathlineto{\pgfqpoint{3.830154in}{1.949142in}}%
\pgfpathlineto{\pgfqpoint{3.807727in}{2.002932in}}%
\pgfpathclose%
\pgfusepath{fill}%
\end{pgfscope}%
\begin{pgfscope}%
\pgfpathrectangle{\pgfqpoint{1.072000in}{0.528000in}}{\pgfqpoint{3.696000in}{3.696000in}}%
\pgfusepath{clip}%
\pgfsetbuttcap%
\pgfsetroundjoin%
\definecolor{currentfill}{rgb}{0.919376,0.831273,0.782874}%
\pgfsetfillcolor{currentfill}%
\pgfsetlinewidth{0.000000pt}%
\definecolor{currentstroke}{rgb}{0.000000,0.000000,0.000000}%
\pgfsetstrokecolor{currentstroke}%
\pgfsetdash{}{0pt}%
\pgfpathmoveto{\pgfqpoint{2.737250in}{2.782573in}}%
\pgfpathlineto{\pgfqpoint{2.781276in}{2.913519in}}%
\pgfpathlineto{\pgfqpoint{2.808242in}{2.821479in}}%
\pgfpathlineto{\pgfqpoint{2.764276in}{2.696340in}}%
\pgfpathlineto{\pgfqpoint{2.737250in}{2.782573in}}%
\pgfpathclose%
\pgfusepath{fill}%
\end{pgfscope}%
\begin{pgfscope}%
\pgfpathrectangle{\pgfqpoint{1.072000in}{0.528000in}}{\pgfqpoint{3.696000in}{3.696000in}}%
\pgfusepath{clip}%
\pgfsetbuttcap%
\pgfsetroundjoin%
\definecolor{currentfill}{rgb}{0.859385,0.864431,0.872111}%
\pgfsetfillcolor{currentfill}%
\pgfsetlinewidth{0.000000pt}%
\definecolor{currentstroke}{rgb}{0.000000,0.000000,0.000000}%
\pgfsetstrokecolor{currentstroke}%
\pgfsetdash{}{0pt}%
\pgfpathmoveto{\pgfqpoint{3.363665in}{2.782800in}}%
\pgfpathlineto{\pgfqpoint{3.409048in}{2.747779in}}%
\pgfpathlineto{\pgfqpoint{3.431868in}{2.592012in}}%
\pgfpathlineto{\pgfqpoint{3.386772in}{2.625146in}}%
\pgfpathlineto{\pgfqpoint{3.363665in}{2.782800in}}%
\pgfpathclose%
\pgfusepath{fill}%
\end{pgfscope}%
\begin{pgfscope}%
\pgfpathrectangle{\pgfqpoint{1.072000in}{0.528000in}}{\pgfqpoint{3.696000in}{3.696000in}}%
\pgfusepath{clip}%
\pgfsetbuttcap%
\pgfsetroundjoin%
\definecolor{currentfill}{rgb}{0.304174,0.406945,0.845263}%
\pgfsetfillcolor{currentfill}%
\pgfsetlinewidth{0.000000pt}%
\definecolor{currentstroke}{rgb}{0.000000,0.000000,0.000000}%
\pgfsetstrokecolor{currentstroke}%
\pgfsetdash{}{0pt}%
\pgfpathmoveto{\pgfqpoint{2.104064in}{1.807187in}}%
\pgfpathlineto{\pgfqpoint{2.146390in}{1.864732in}}%
\pgfpathlineto{\pgfqpoint{2.173038in}{1.929362in}}%
\pgfpathlineto{\pgfqpoint{2.131039in}{1.856177in}}%
\pgfpathlineto{\pgfqpoint{2.104064in}{1.807187in}}%
\pgfpathclose%
\pgfusepath{fill}%
\end{pgfscope}%
\begin{pgfscope}%
\pgfpathrectangle{\pgfqpoint{1.072000in}{0.528000in}}{\pgfqpoint{3.696000in}{3.696000in}}%
\pgfusepath{clip}%
\pgfsetbuttcap%
\pgfsetroundjoin%
\definecolor{currentfill}{rgb}{0.809329,0.852974,0.922323}%
\pgfsetfillcolor{currentfill}%
\pgfsetlinewidth{0.000000pt}%
\definecolor{currentstroke}{rgb}{0.000000,0.000000,0.000000}%
\pgfsetstrokecolor{currentstroke}%
\pgfsetdash{}{0pt}%
\pgfpathmoveto{\pgfqpoint{3.454274in}{2.698188in}}%
\pgfpathlineto{\pgfqpoint{3.499295in}{2.634738in}}%
\pgfpathlineto{\pgfqpoint{3.521600in}{2.487643in}}%
\pgfpathlineto{\pgfqpoint{3.476823in}{2.545849in}}%
\pgfpathlineto{\pgfqpoint{3.454274in}{2.698188in}}%
\pgfpathclose%
\pgfusepath{fill}%
\end{pgfscope}%
\begin{pgfscope}%
\pgfpathrectangle{\pgfqpoint{1.072000in}{0.528000in}}{\pgfqpoint{3.696000in}{3.696000in}}%
\pgfusepath{clip}%
\pgfsetbuttcap%
\pgfsetroundjoin%
\definecolor{currentfill}{rgb}{0.839351,0.861167,0.894494}%
\pgfsetfillcolor{currentfill}%
\pgfsetlinewidth{0.000000pt}%
\definecolor{currentstroke}{rgb}{0.000000,0.000000,0.000000}%
\pgfsetstrokecolor{currentstroke}%
\pgfsetdash{}{0pt}%
\pgfpathmoveto{\pgfqpoint{3.409048in}{2.747779in}}%
\pgfpathlineto{\pgfqpoint{3.454274in}{2.698188in}}%
\pgfpathlineto{\pgfqpoint{3.476823in}{2.545849in}}%
\pgfpathlineto{\pgfqpoint{3.431868in}{2.592012in}}%
\pgfpathlineto{\pgfqpoint{3.409048in}{2.747779in}}%
\pgfpathclose%
\pgfusepath{fill}%
\end{pgfscope}%
\begin{pgfscope}%
\pgfpathrectangle{\pgfqpoint{1.072000in}{0.528000in}}{\pgfqpoint{3.696000in}{3.696000in}}%
\pgfusepath{clip}%
\pgfsetbuttcap%
\pgfsetroundjoin%
\definecolor{currentfill}{rgb}{0.597777,0.727330,0.999777}%
\pgfsetfillcolor{currentfill}%
\pgfsetlinewidth{0.000000pt}%
\definecolor{currentstroke}{rgb}{0.000000,0.000000,0.000000}%
\pgfsetstrokecolor{currentstroke}%
\pgfsetdash{}{0pt}%
\pgfpathmoveto{\pgfqpoint{4.138728in}{2.163241in}}%
\pgfpathlineto{\pgfqpoint{4.188887in}{2.259735in}}%
\pgfpathlineto{\pgfqpoint{4.216702in}{2.399163in}}%
\pgfpathlineto{\pgfqpoint{4.166136in}{2.295708in}}%
\pgfpathlineto{\pgfqpoint{4.138728in}{2.163241in}}%
\pgfpathclose%
\pgfusepath{fill}%
\end{pgfscope}%
\begin{pgfscope}%
\pgfpathrectangle{\pgfqpoint{1.072000in}{0.528000in}}{\pgfqpoint{3.696000in}{3.696000in}}%
\pgfusepath{clip}%
\pgfsetbuttcap%
\pgfsetroundjoin%
\definecolor{currentfill}{rgb}{0.229806,0.298718,0.753683}%
\pgfsetfillcolor{currentfill}%
\pgfsetlinewidth{0.000000pt}%
\definecolor{currentstroke}{rgb}{0.000000,0.000000,0.000000}%
\pgfsetstrokecolor{currentstroke}%
\pgfsetdash{}{0pt}%
\pgfpathmoveto{\pgfqpoint{2.005575in}{1.734858in}}%
\pgfpathlineto{\pgfqpoint{2.049531in}{1.723105in}}%
\pgfpathlineto{\pgfqpoint{2.076924in}{1.761869in}}%
\pgfpathlineto{\pgfqpoint{2.033501in}{1.755503in}}%
\pgfpathlineto{\pgfqpoint{2.005575in}{1.734858in}}%
\pgfpathclose%
\pgfusepath{fill}%
\end{pgfscope}%
\begin{pgfscope}%
\pgfpathrectangle{\pgfqpoint{1.072000in}{0.528000in}}{\pgfqpoint{3.696000in}{3.696000in}}%
\pgfusepath{clip}%
\pgfsetbuttcap%
\pgfsetroundjoin%
\definecolor{currentfill}{rgb}{0.333490,0.446265,0.874452}%
\pgfsetfillcolor{currentfill}%
\pgfsetlinewidth{0.000000pt}%
\definecolor{currentstroke}{rgb}{0.000000,0.000000,0.000000}%
\pgfsetstrokecolor{currentstroke}%
\pgfsetdash{}{0pt}%
\pgfpathmoveto{\pgfqpoint{1.867809in}{1.972136in}}%
\pgfpathlineto{\pgfqpoint{1.914806in}{1.860588in}}%
\pgfpathlineto{\pgfqpoint{1.943864in}{1.850190in}}%
\pgfpathlineto{\pgfqpoint{1.897418in}{1.949850in}}%
\pgfpathlineto{\pgfqpoint{1.867809in}{1.972136in}}%
\pgfpathclose%
\pgfusepath{fill}%
\end{pgfscope}%
\begin{pgfscope}%
\pgfpathrectangle{\pgfqpoint{1.072000in}{0.528000in}}{\pgfqpoint{3.696000in}{3.696000in}}%
\pgfusepath{clip}%
\pgfsetbuttcap%
\pgfsetroundjoin%
\definecolor{currentfill}{rgb}{0.238948,0.312365,0.765676}%
\pgfsetfillcolor{currentfill}%
\pgfsetlinewidth{0.000000pt}%
\definecolor{currentstroke}{rgb}{0.000000,0.000000,0.000000}%
\pgfsetstrokecolor{currentstroke}%
\pgfsetdash{}{0pt}%
\pgfpathmoveto{\pgfqpoint{3.986340in}{1.730342in}}%
\pgfpathlineto{\pgfqpoint{4.033111in}{1.733250in}}%
\pgfpathlineto{\pgfqpoint{4.057865in}{1.783013in}}%
\pgfpathlineto{\pgfqpoint{4.010533in}{1.761214in}}%
\pgfpathlineto{\pgfqpoint{3.986340in}{1.730342in}}%
\pgfpathclose%
\pgfusepath{fill}%
\end{pgfscope}%
\begin{pgfscope}%
\pgfpathrectangle{\pgfqpoint{1.072000in}{0.528000in}}{\pgfqpoint{3.696000in}{3.696000in}}%
\pgfusepath{clip}%
\pgfsetbuttcap%
\pgfsetroundjoin%
\definecolor{currentfill}{rgb}{0.543440,0.680003,0.993051}%
\pgfsetfillcolor{currentfill}%
\pgfsetlinewidth{0.000000pt}%
\definecolor{currentstroke}{rgb}{0.000000,0.000000,0.000000}%
\pgfsetstrokecolor{currentstroke}%
\pgfsetdash{}{0pt}%
\pgfpathmoveto{\pgfqpoint{2.253089in}{2.106134in}}%
\pgfpathlineto{\pgfqpoint{2.294399in}{2.240302in}}%
\pgfpathlineto{\pgfqpoint{2.321228in}{2.291984in}}%
\pgfpathlineto{\pgfqpoint{2.279930in}{2.152166in}}%
\pgfpathlineto{\pgfqpoint{2.253089in}{2.106134in}}%
\pgfpathclose%
\pgfusepath{fill}%
\end{pgfscope}%
\begin{pgfscope}%
\pgfpathrectangle{\pgfqpoint{1.072000in}{0.528000in}}{\pgfqpoint{3.696000in}{3.696000in}}%
\pgfusepath{clip}%
\pgfsetbuttcap%
\pgfsetroundjoin%
\definecolor{currentfill}{rgb}{0.275827,0.366717,0.812553}%
\pgfsetfillcolor{currentfill}%
\pgfsetlinewidth{0.000000pt}%
\definecolor{currentstroke}{rgb}{0.000000,0.000000,0.000000}%
\pgfsetstrokecolor{currentstroke}%
\pgfsetdash{}{0pt}%
\pgfpathmoveto{\pgfqpoint{2.076924in}{1.761869in}}%
\pgfpathlineto{\pgfqpoint{2.119654in}{1.801900in}}%
\pgfpathlineto{\pgfqpoint{2.146390in}{1.864732in}}%
\pgfpathlineto{\pgfqpoint{2.104064in}{1.807187in}}%
\pgfpathlineto{\pgfqpoint{2.076924in}{1.761869in}}%
\pgfpathclose%
\pgfusepath{fill}%
\end{pgfscope}%
\begin{pgfscope}%
\pgfpathrectangle{\pgfqpoint{1.072000in}{0.528000in}}{\pgfqpoint{3.696000in}{3.696000in}}%
\pgfusepath{clip}%
\pgfsetbuttcap%
\pgfsetroundjoin%
\definecolor{currentfill}{rgb}{0.425199,0.559058,0.946061}%
\pgfsetfillcolor{currentfill}%
\pgfsetlinewidth{0.000000pt}%
\definecolor{currentstroke}{rgb}{0.000000,0.000000,0.000000}%
\pgfsetstrokecolor{currentstroke}%
\pgfsetdash{}{0pt}%
\pgfpathmoveto{\pgfqpoint{3.764134in}{2.121557in}}%
\pgfpathlineto{\pgfqpoint{3.807727in}{2.002932in}}%
\pgfpathlineto{\pgfqpoint{3.830154in}{1.949142in}}%
\pgfpathlineto{\pgfqpoint{3.786330in}{2.047995in}}%
\pgfpathlineto{\pgfqpoint{3.764134in}{2.121557in}}%
\pgfpathclose%
\pgfusepath{fill}%
\end{pgfscope}%
\begin{pgfscope}%
\pgfpathrectangle{\pgfqpoint{1.072000in}{0.528000in}}{\pgfqpoint{3.696000in}{3.696000in}}%
\pgfusepath{clip}%
\pgfsetbuttcap%
\pgfsetroundjoin%
\definecolor{currentfill}{rgb}{0.603162,0.731527,0.999565}%
\pgfsetfillcolor{currentfill}%
\pgfsetlinewidth{0.000000pt}%
\definecolor{currentstroke}{rgb}{0.000000,0.000000,0.000000}%
\pgfsetstrokecolor{currentstroke}%
\pgfsetdash{}{0pt}%
\pgfpathmoveto{\pgfqpoint{3.655014in}{2.390664in}}%
\pgfpathlineto{\pgfqpoint{3.698901in}{2.273983in}}%
\pgfpathlineto{\pgfqpoint{3.720830in}{2.174570in}}%
\pgfpathlineto{\pgfqpoint{3.676931in}{2.276087in}}%
\pgfpathlineto{\pgfqpoint{3.655014in}{2.390664in}}%
\pgfpathclose%
\pgfusepath{fill}%
\end{pgfscope}%
\begin{pgfscope}%
\pgfpathrectangle{\pgfqpoint{1.072000in}{0.528000in}}{\pgfqpoint{3.696000in}{3.696000in}}%
\pgfusepath{clip}%
\pgfsetbuttcap%
\pgfsetroundjoin%
\definecolor{currentfill}{rgb}{0.234377,0.305542,0.759680}%
\pgfsetfillcolor{currentfill}%
\pgfsetlinewidth{0.000000pt}%
\definecolor{currentstroke}{rgb}{0.000000,0.000000,0.000000}%
\pgfsetstrokecolor{currentstroke}%
\pgfsetdash{}{0pt}%
\pgfpathmoveto{\pgfqpoint{3.917184in}{1.760358in}}%
\pgfpathlineto{\pgfqpoint{3.962425in}{1.711725in}}%
\pgfpathlineto{\pgfqpoint{3.986340in}{1.730342in}}%
\pgfpathlineto{\pgfqpoint{3.940620in}{1.758891in}}%
\pgfpathlineto{\pgfqpoint{3.917184in}{1.760358in}}%
\pgfpathclose%
\pgfusepath{fill}%
\end{pgfscope}%
\begin{pgfscope}%
\pgfpathrectangle{\pgfqpoint{1.072000in}{0.528000in}}{\pgfqpoint{3.696000in}{3.696000in}}%
\pgfusepath{clip}%
\pgfsetbuttcap%
\pgfsetroundjoin%
\definecolor{currentfill}{rgb}{0.266381,0.353304,0.801637}%
\pgfsetfillcolor{currentfill}%
\pgfsetlinewidth{0.000000pt}%
\definecolor{currentstroke}{rgb}{0.000000,0.000000,0.000000}%
\pgfsetstrokecolor{currentstroke}%
\pgfsetdash{}{0pt}%
\pgfpathmoveto{\pgfqpoint{3.872738in}{1.838596in}}%
\pgfpathlineto{\pgfqpoint{3.917184in}{1.760358in}}%
\pgfpathlineto{\pgfqpoint{3.940620in}{1.758891in}}%
\pgfpathlineto{\pgfqpoint{3.895756in}{1.816485in}}%
\pgfpathlineto{\pgfqpoint{3.872738in}{1.838596in}}%
\pgfpathclose%
\pgfusepath{fill}%
\end{pgfscope}%
\begin{pgfscope}%
\pgfpathrectangle{\pgfqpoint{1.072000in}{0.528000in}}{\pgfqpoint{3.696000in}{3.696000in}}%
\pgfusepath{clip}%
\pgfsetbuttcap%
\pgfsetroundjoin%
\definecolor{currentfill}{rgb}{0.947345,0.794696,0.716991}%
\pgfsetfillcolor{currentfill}%
\pgfsetlinewidth{0.000000pt}%
\definecolor{currentstroke}{rgb}{0.000000,0.000000,0.000000}%
\pgfsetstrokecolor{currentstroke}%
\pgfsetdash{}{0pt}%
\pgfpathmoveto{\pgfqpoint{3.157046in}{2.925117in}}%
\pgfpathlineto{\pgfqpoint{3.202762in}{2.951076in}}%
\pgfpathlineto{\pgfqpoint{3.227126in}{2.798275in}}%
\pgfpathlineto{\pgfqpoint{3.181679in}{2.774102in}}%
\pgfpathlineto{\pgfqpoint{3.157046in}{2.925117in}}%
\pgfpathclose%
\pgfusepath{fill}%
\end{pgfscope}%
\begin{pgfscope}%
\pgfpathrectangle{\pgfqpoint{1.072000in}{0.528000in}}{\pgfqpoint{3.696000in}{3.696000in}}%
\pgfusepath{clip}%
\pgfsetbuttcap%
\pgfsetroundjoin%
\definecolor{currentfill}{rgb}{0.728970,0.817464,0.973188}%
\pgfsetfillcolor{currentfill}%
\pgfsetlinewidth{0.000000pt}%
\definecolor{currentstroke}{rgb}{0.000000,0.000000,0.000000}%
\pgfsetstrokecolor{currentstroke}%
\pgfsetdash{}{0pt}%
\pgfpathmoveto{\pgfqpoint{2.375290in}{2.361119in}}%
\pgfpathlineto{\pgfqpoint{2.416898in}{2.522643in}}%
\pgfpathlineto{\pgfqpoint{2.444183in}{2.538164in}}%
\pgfpathlineto{\pgfqpoint{2.402482in}{2.377750in}}%
\pgfpathlineto{\pgfqpoint{2.375290in}{2.361119in}}%
\pgfpathclose%
\pgfusepath{fill}%
\end{pgfscope}%
\begin{pgfscope}%
\pgfpathrectangle{\pgfqpoint{1.072000in}{0.528000in}}{\pgfqpoint{3.696000in}{3.696000in}}%
\pgfusepath{clip}%
\pgfsetbuttcap%
\pgfsetroundjoin%
\definecolor{currentfill}{rgb}{0.713852,0.808857,0.979386}%
\pgfsetfillcolor{currentfill}%
\pgfsetlinewidth{0.000000pt}%
\definecolor{currentstroke}{rgb}{0.000000,0.000000,0.000000}%
\pgfsetstrokecolor{currentstroke}%
\pgfsetdash{}{0pt}%
\pgfpathmoveto{\pgfqpoint{4.166136in}{2.295708in}}%
\pgfpathlineto{\pgfqpoint{4.216702in}{2.399163in}}%
\pgfpathlineto{\pgfqpoint{4.244949in}{2.548621in}}%
\pgfpathlineto{\pgfqpoint{4.194087in}{2.441947in}}%
\pgfpathlineto{\pgfqpoint{4.166136in}{2.295708in}}%
\pgfpathclose%
\pgfusepath{fill}%
\end{pgfscope}%
\begin{pgfscope}%
\pgfpathrectangle{\pgfqpoint{1.072000in}{0.528000in}}{\pgfqpoint{3.696000in}{3.696000in}}%
\pgfusepath{clip}%
\pgfsetbuttcap%
\pgfsetroundjoin%
\definecolor{currentfill}{rgb}{0.248091,0.326013,0.777669}%
\pgfsetfillcolor{currentfill}%
\pgfsetlinewidth{0.000000pt}%
\definecolor{currentstroke}{rgb}{0.000000,0.000000,0.000000}%
\pgfsetstrokecolor{currentstroke}%
\pgfsetdash{}{0pt}%
\pgfpathmoveto{\pgfqpoint{1.931753in}{1.787255in}}%
\pgfpathlineto{\pgfqpoint{1.977235in}{1.724393in}}%
\pgfpathlineto{\pgfqpoint{2.005575in}{1.734858in}}%
\pgfpathlineto{\pgfqpoint{1.960708in}{1.780790in}}%
\pgfpathlineto{\pgfqpoint{1.931753in}{1.787255in}}%
\pgfpathclose%
\pgfusepath{fill}%
\end{pgfscope}%
\begin{pgfscope}%
\pgfpathrectangle{\pgfqpoint{1.072000in}{0.528000in}}{\pgfqpoint{3.696000in}{3.696000in}}%
\pgfusepath{clip}%
\pgfsetbuttcap%
\pgfsetroundjoin%
\definecolor{currentfill}{rgb}{0.510824,0.649397,0.985079}%
\pgfsetfillcolor{currentfill}%
\pgfsetlinewidth{0.000000pt}%
\definecolor{currentstroke}{rgb}{0.000000,0.000000,0.000000}%
\pgfsetstrokecolor{currentstroke}%
\pgfsetdash{}{0pt}%
\pgfpathmoveto{\pgfqpoint{2.226342in}{2.052452in}}%
\pgfpathlineto{\pgfqpoint{2.267713in}{2.178557in}}%
\pgfpathlineto{\pgfqpoint{2.294399in}{2.240302in}}%
\pgfpathlineto{\pgfqpoint{2.253089in}{2.106134in}}%
\pgfpathlineto{\pgfqpoint{2.226342in}{2.052452in}}%
\pgfpathclose%
\pgfusepath{fill}%
\end{pgfscope}%
\begin{pgfscope}%
\pgfpathrectangle{\pgfqpoint{1.072000in}{0.528000in}}{\pgfqpoint{3.696000in}{3.696000in}}%
\pgfusepath{clip}%
\pgfsetbuttcap%
\pgfsetroundjoin%
\definecolor{currentfill}{rgb}{0.358415,0.478426,0.896795}%
\pgfsetfillcolor{currentfill}%
\pgfsetlinewidth{0.000000pt}%
\definecolor{currentstroke}{rgb}{0.000000,0.000000,0.000000}%
\pgfsetstrokecolor{currentstroke}%
\pgfsetdash{}{0pt}%
\pgfpathmoveto{\pgfqpoint{4.083028in}{1.846931in}}%
\pgfpathlineto{\pgfqpoint{4.132141in}{1.914616in}}%
\pgfpathlineto{\pgfqpoint{4.158349in}{2.008349in}}%
\pgfpathlineto{\pgfqpoint{4.108671in}{1.926312in}}%
\pgfpathlineto{\pgfqpoint{4.083028in}{1.846931in}}%
\pgfpathclose%
\pgfusepath{fill}%
\end{pgfscope}%
\begin{pgfscope}%
\pgfpathrectangle{\pgfqpoint{1.072000in}{0.528000in}}{\pgfqpoint{3.696000in}{3.696000in}}%
\pgfusepath{clip}%
\pgfsetbuttcap%
\pgfsetroundjoin%
\definecolor{currentfill}{rgb}{0.425199,0.559058,0.946061}%
\pgfsetfillcolor{currentfill}%
\pgfsetlinewidth{0.000000pt}%
\definecolor{currentstroke}{rgb}{0.000000,0.000000,0.000000}%
\pgfsetstrokecolor{currentstroke}%
\pgfsetdash{}{0pt}%
\pgfpathmoveto{\pgfqpoint{4.108671in}{1.926312in}}%
\pgfpathlineto{\pgfqpoint{4.158349in}{2.008349in}}%
\pgfpathlineto{\pgfqpoint{4.185061in}{2.116182in}}%
\pgfpathlineto{\pgfqpoint{4.134849in}{2.021811in}}%
\pgfpathlineto{\pgfqpoint{4.108671in}{1.926312in}}%
\pgfpathclose%
\pgfusepath{fill}%
\end{pgfscope}%
\begin{pgfscope}%
\pgfpathrectangle{\pgfqpoint{1.072000in}{0.528000in}}{\pgfqpoint{3.696000in}{3.696000in}}%
\pgfusepath{clip}%
\pgfsetbuttcap%
\pgfsetroundjoin%
\definecolor{currentfill}{rgb}{0.891817,0.851973,0.829085}%
\pgfsetfillcolor{currentfill}%
\pgfsetlinewidth{0.000000pt}%
\definecolor{currentstroke}{rgb}{0.000000,0.000000,0.000000}%
\pgfsetstrokecolor{currentstroke}%
\pgfsetdash{}{0pt}%
\pgfpathmoveto{\pgfqpoint{2.568744in}{2.662122in}}%
\pgfpathlineto{\pgfqpoint{2.611667in}{2.819768in}}%
\pgfpathlineto{\pgfqpoint{2.639138in}{2.773757in}}%
\pgfpathlineto{\pgfqpoint{2.596174in}{2.621099in}}%
\pgfpathlineto{\pgfqpoint{2.568744in}{2.662122in}}%
\pgfpathclose%
\pgfusepath{fill}%
\end{pgfscope}%
\begin{pgfscope}%
\pgfpathrectangle{\pgfqpoint{1.072000in}{0.528000in}}{\pgfqpoint{3.696000in}{3.696000in}}%
\pgfusepath{clip}%
\pgfsetbuttcap%
\pgfsetroundjoin%
\definecolor{currentfill}{rgb}{0.229806,0.298718,0.753683}%
\pgfsetfillcolor{currentfill}%
\pgfsetlinewidth{0.000000pt}%
\definecolor{currentstroke}{rgb}{0.000000,0.000000,0.000000}%
\pgfsetstrokecolor{currentstroke}%
\pgfsetdash{}{0pt}%
\pgfpathmoveto{\pgfqpoint{1.977235in}{1.724393in}}%
\pgfpathlineto{\pgfqpoint{2.021786in}{1.693817in}}%
\pgfpathlineto{\pgfqpoint{2.049531in}{1.723105in}}%
\pgfpathlineto{\pgfqpoint{2.005575in}{1.734858in}}%
\pgfpathlineto{\pgfqpoint{1.977235in}{1.724393in}}%
\pgfpathclose%
\pgfusepath{fill}%
\end{pgfscope}%
\begin{pgfscope}%
\pgfpathrectangle{\pgfqpoint{1.072000in}{0.528000in}}{\pgfqpoint{3.696000in}{3.696000in}}%
\pgfusepath{clip}%
\pgfsetbuttcap%
\pgfsetroundjoin%
\definecolor{currentfill}{rgb}{0.252663,0.332837,0.783665}%
\pgfsetfillcolor{currentfill}%
\pgfsetlinewidth{0.000000pt}%
\definecolor{currentstroke}{rgb}{0.000000,0.000000,0.000000}%
\pgfsetstrokecolor{currentstroke}%
\pgfsetdash{}{0pt}%
\pgfpathmoveto{\pgfqpoint{2.049531in}{1.723105in}}%
\pgfpathlineto{\pgfqpoint{2.092742in}{1.744116in}}%
\pgfpathlineto{\pgfqpoint{2.119654in}{1.801900in}}%
\pgfpathlineto{\pgfqpoint{2.076924in}{1.761869in}}%
\pgfpathlineto{\pgfqpoint{2.049531in}{1.723105in}}%
\pgfpathclose%
\pgfusepath{fill}%
\end{pgfscope}%
\begin{pgfscope}%
\pgfpathrectangle{\pgfqpoint{1.072000in}{0.528000in}}{\pgfqpoint{3.696000in}{3.696000in}}%
\pgfusepath{clip}%
\pgfsetbuttcap%
\pgfsetroundjoin%
\definecolor{currentfill}{rgb}{0.960581,0.762501,0.667964}%
\pgfsetfillcolor{currentfill}%
\pgfsetlinewidth{0.000000pt}%
\definecolor{currentstroke}{rgb}{0.000000,0.000000,0.000000}%
\pgfsetstrokecolor{currentstroke}%
\pgfsetdash{}{0pt}%
\pgfpathmoveto{\pgfqpoint{2.924090in}{2.931803in}}%
\pgfpathlineto{\pgfqpoint{2.969258in}{3.019369in}}%
\pgfpathlineto{\pgfqpoint{2.995286in}{2.895118in}}%
\pgfpathlineto{\pgfqpoint{2.950288in}{2.810158in}}%
\pgfpathlineto{\pgfqpoint{2.924090in}{2.931803in}}%
\pgfpathclose%
\pgfusepath{fill}%
\end{pgfscope}%
\begin{pgfscope}%
\pgfpathrectangle{\pgfqpoint{1.072000in}{0.528000in}}{\pgfqpoint{3.696000in}{3.696000in}}%
\pgfusepath{clip}%
\pgfsetbuttcap%
\pgfsetroundjoin%
\definecolor{currentfill}{rgb}{0.309060,0.413498,0.850128}%
\pgfsetfillcolor{currentfill}%
\pgfsetlinewidth{0.000000pt}%
\definecolor{currentstroke}{rgb}{0.000000,0.000000,0.000000}%
\pgfsetstrokecolor{currentstroke}%
\pgfsetdash{}{0pt}%
\pgfpathmoveto{\pgfqpoint{4.057865in}{1.783013in}}%
\pgfpathlineto{\pgfqpoint{4.106406in}{1.834913in}}%
\pgfpathlineto{\pgfqpoint{4.132141in}{1.914616in}}%
\pgfpathlineto{\pgfqpoint{4.083028in}{1.846931in}}%
\pgfpathlineto{\pgfqpoint{4.057865in}{1.783013in}}%
\pgfpathclose%
\pgfusepath{fill}%
\end{pgfscope}%
\begin{pgfscope}%
\pgfpathrectangle{\pgfqpoint{1.072000in}{0.528000in}}{\pgfqpoint{3.696000in}{3.696000in}}%
\pgfusepath{clip}%
\pgfsetbuttcap%
\pgfsetroundjoin%
\definecolor{currentfill}{rgb}{0.229806,0.298718,0.753683}%
\pgfsetfillcolor{currentfill}%
\pgfsetlinewidth{0.000000pt}%
\definecolor{currentstroke}{rgb}{0.000000,0.000000,0.000000}%
\pgfsetstrokecolor{currentstroke}%
\pgfsetdash{}{0pt}%
\pgfpathmoveto{\pgfqpoint{3.962425in}{1.711725in}}%
\pgfpathlineto{\pgfqpoint{4.008680in}{1.695822in}}%
\pgfpathlineto{\pgfqpoint{4.033111in}{1.733250in}}%
\pgfpathlineto{\pgfqpoint{3.986340in}{1.730342in}}%
\pgfpathlineto{\pgfqpoint{3.962425in}{1.711725in}}%
\pgfpathclose%
\pgfusepath{fill}%
\end{pgfscope}%
\begin{pgfscope}%
\pgfpathrectangle{\pgfqpoint{1.072000in}{0.528000in}}{\pgfqpoint{3.696000in}{3.696000in}}%
\pgfusepath{clip}%
\pgfsetbuttcap%
\pgfsetroundjoin%
\definecolor{currentfill}{rgb}{0.328604,0.439712,0.869587}%
\pgfsetfillcolor{currentfill}%
\pgfsetlinewidth{0.000000pt}%
\definecolor{currentstroke}{rgb}{0.000000,0.000000,0.000000}%
\pgfsetstrokecolor{currentstroke}%
\pgfsetdash{}{0pt}%
\pgfpathmoveto{\pgfqpoint{3.828853in}{1.941973in}}%
\pgfpathlineto{\pgfqpoint{3.872738in}{1.838596in}}%
\pgfpathlineto{\pgfqpoint{3.895756in}{1.816485in}}%
\pgfpathlineto{\pgfqpoint{3.851531in}{1.899415in}}%
\pgfpathlineto{\pgfqpoint{3.828853in}{1.941973in}}%
\pgfpathclose%
\pgfusepath{fill}%
\end{pgfscope}%
\begin{pgfscope}%
\pgfpathrectangle{\pgfqpoint{1.072000in}{0.528000in}}{\pgfqpoint{3.696000in}{3.696000in}}%
\pgfusepath{clip}%
\pgfsetbuttcap%
\pgfsetroundjoin%
\definecolor{currentfill}{rgb}{0.294718,0.393542,0.834384}%
\pgfsetfillcolor{currentfill}%
\pgfsetlinewidth{0.000000pt}%
\definecolor{currentstroke}{rgb}{0.000000,0.000000,0.000000}%
\pgfsetstrokecolor{currentstroke}%
\pgfsetdash{}{0pt}%
\pgfpathmoveto{\pgfqpoint{1.885243in}{1.881530in}}%
\pgfpathlineto{\pgfqpoint{1.931753in}{1.787255in}}%
\pgfpathlineto{\pgfqpoint{1.960708in}{1.780790in}}%
\pgfpathlineto{\pgfqpoint{1.914806in}{1.860588in}}%
\pgfpathlineto{\pgfqpoint{1.885243in}{1.881530in}}%
\pgfpathclose%
\pgfusepath{fill}%
\end{pgfscope}%
\begin{pgfscope}%
\pgfpathrectangle{\pgfqpoint{1.072000in}{0.528000in}}{\pgfqpoint{3.696000in}{3.696000in}}%
\pgfusepath{clip}%
\pgfsetbuttcap%
\pgfsetroundjoin%
\definecolor{currentfill}{rgb}{0.505423,0.643995,0.983157}%
\pgfsetfillcolor{currentfill}%
\pgfsetlinewidth{0.000000pt}%
\definecolor{currentstroke}{rgb}{0.000000,0.000000,0.000000}%
\pgfsetstrokecolor{currentstroke}%
\pgfsetdash{}{0pt}%
\pgfpathmoveto{\pgfqpoint{4.134849in}{2.021811in}}%
\pgfpathlineto{\pgfqpoint{4.185061in}{2.116182in}}%
\pgfpathlineto{\pgfqpoint{4.212280in}{2.237427in}}%
\pgfpathlineto{\pgfqpoint{4.161589in}{2.133292in}}%
\pgfpathlineto{\pgfqpoint{4.134849in}{2.021811in}}%
\pgfpathclose%
\pgfusepath{fill}%
\end{pgfscope}%
\begin{pgfscope}%
\pgfpathrectangle{\pgfqpoint{1.072000in}{0.528000in}}{\pgfqpoint{3.696000in}{3.696000in}}%
\pgfusepath{clip}%
\pgfsetbuttcap%
\pgfsetroundjoin%
\definecolor{currentfill}{rgb}{0.473070,0.611077,0.970634}%
\pgfsetfillcolor{currentfill}%
\pgfsetlinewidth{0.000000pt}%
\definecolor{currentstroke}{rgb}{0.000000,0.000000,0.000000}%
\pgfsetstrokecolor{currentstroke}%
\pgfsetdash{}{0pt}%
\pgfpathmoveto{\pgfqpoint{2.199671in}{1.992805in}}%
\pgfpathlineto{\pgfqpoint{2.241158in}{2.108401in}}%
\pgfpathlineto{\pgfqpoint{2.267713in}{2.178557in}}%
\pgfpathlineto{\pgfqpoint{2.226342in}{2.052452in}}%
\pgfpathlineto{\pgfqpoint{2.199671in}{1.992805in}}%
\pgfpathclose%
\pgfusepath{fill}%
\end{pgfscope}%
\begin{pgfscope}%
\pgfpathrectangle{\pgfqpoint{1.072000in}{0.528000in}}{\pgfqpoint{3.696000in}{3.696000in}}%
\pgfusepath{clip}%
\pgfsetbuttcap%
\pgfsetroundjoin%
\definecolor{currentfill}{rgb}{0.839351,0.861167,0.894494}%
\pgfsetfillcolor{currentfill}%
\pgfsetlinewidth{0.000000pt}%
\definecolor{currentstroke}{rgb}{0.000000,0.000000,0.000000}%
\pgfsetstrokecolor{currentstroke}%
\pgfsetdash{}{0pt}%
\pgfpathmoveto{\pgfqpoint{2.471545in}{2.540037in}}%
\pgfpathlineto{\pgfqpoint{2.513778in}{2.705512in}}%
\pgfpathlineto{\pgfqpoint{2.541264in}{2.690688in}}%
\pgfpathlineto{\pgfqpoint{2.498945in}{2.528893in}}%
\pgfpathlineto{\pgfqpoint{2.471545in}{2.540037in}}%
\pgfpathclose%
\pgfusepath{fill}%
\end{pgfscope}%
\begin{pgfscope}%
\pgfpathrectangle{\pgfqpoint{1.072000in}{0.528000in}}{\pgfqpoint{3.696000in}{3.696000in}}%
\pgfusepath{clip}%
\pgfsetbuttcap%
\pgfsetroundjoin%
\definecolor{currentfill}{rgb}{0.698454,0.799450,0.984577}%
\pgfsetfillcolor{currentfill}%
\pgfsetlinewidth{0.000000pt}%
\definecolor{currentstroke}{rgb}{0.000000,0.000000,0.000000}%
\pgfsetstrokecolor{currentstroke}%
\pgfsetdash{}{0pt}%
\pgfpathmoveto{\pgfqpoint{3.610882in}{2.503442in}}%
\pgfpathlineto{\pgfqpoint{3.655014in}{2.390664in}}%
\pgfpathlineto{\pgfqpoint{3.676931in}{2.276087in}}%
\pgfpathlineto{\pgfqpoint{3.632877in}{2.376424in}}%
\pgfpathlineto{\pgfqpoint{3.610882in}{2.503442in}}%
\pgfpathclose%
\pgfusepath{fill}%
\end{pgfscope}%
\begin{pgfscope}%
\pgfpathrectangle{\pgfqpoint{1.072000in}{0.528000in}}{\pgfqpoint{3.696000in}{3.696000in}}%
\pgfusepath{clip}%
\pgfsetbuttcap%
\pgfsetroundjoin%
\definecolor{currentfill}{rgb}{0.962708,0.753557,0.655601}%
\pgfsetfillcolor{currentfill}%
\pgfsetlinewidth{0.000000pt}%
\definecolor{currentstroke}{rgb}{0.000000,0.000000,0.000000}%
\pgfsetstrokecolor{currentstroke}%
\pgfsetdash{}{0pt}%
\pgfpathmoveto{\pgfqpoint{3.040610in}{2.966879in}}%
\pgfpathlineto{\pgfqpoint{3.086204in}{3.023731in}}%
\pgfpathlineto{\pgfqpoint{3.111472in}{2.884175in}}%
\pgfpathlineto{\pgfqpoint{3.066099in}{2.829021in}}%
\pgfpathlineto{\pgfqpoint{3.040610in}{2.966879in}}%
\pgfpathclose%
\pgfusepath{fill}%
\end{pgfscope}%
\begin{pgfscope}%
\pgfpathrectangle{\pgfqpoint{1.072000in}{0.528000in}}{\pgfqpoint{3.696000in}{3.696000in}}%
\pgfusepath{clip}%
\pgfsetbuttcap%
\pgfsetroundjoin%
\definecolor{currentfill}{rgb}{0.521696,0.659599,0.987736}%
\pgfsetfillcolor{currentfill}%
\pgfsetlinewidth{0.000000pt}%
\definecolor{currentstroke}{rgb}{0.000000,0.000000,0.000000}%
\pgfsetstrokecolor{currentstroke}%
\pgfsetdash{}{0pt}%
\pgfpathmoveto{\pgfqpoint{3.720562in}{2.249428in}}%
\pgfpathlineto{\pgfqpoint{3.764134in}{2.121557in}}%
\pgfpathlineto{\pgfqpoint{3.786330in}{2.047995in}}%
\pgfpathlineto{\pgfqpoint{3.742632in}{2.158050in}}%
\pgfpathlineto{\pgfqpoint{3.720562in}{2.249428in}}%
\pgfpathclose%
\pgfusepath{fill}%
\end{pgfscope}%
\begin{pgfscope}%
\pgfpathrectangle{\pgfqpoint{1.072000in}{0.528000in}}{\pgfqpoint{3.696000in}{3.696000in}}%
\pgfusepath{clip}%
\pgfsetbuttcap%
\pgfsetroundjoin%
\definecolor{currentfill}{rgb}{0.271104,0.360011,0.807095}%
\pgfsetfillcolor{currentfill}%
\pgfsetlinewidth{0.000000pt}%
\definecolor{currentstroke}{rgb}{0.000000,0.000000,0.000000}%
\pgfsetstrokecolor{currentstroke}%
\pgfsetdash{}{0pt}%
\pgfpathmoveto{\pgfqpoint{4.033111in}{1.733250in}}%
\pgfpathlineto{\pgfqpoint{4.081093in}{1.768525in}}%
\pgfpathlineto{\pgfqpoint{4.106406in}{1.834913in}}%
\pgfpathlineto{\pgfqpoint{4.057865in}{1.783013in}}%
\pgfpathlineto{\pgfqpoint{4.033111in}{1.733250in}}%
\pgfpathclose%
\pgfusepath{fill}%
\end{pgfscope}%
\begin{pgfscope}%
\pgfpathrectangle{\pgfqpoint{1.072000in}{0.528000in}}{\pgfqpoint{3.696000in}{3.696000in}}%
\pgfusepath{clip}%
\pgfsetbuttcap%
\pgfsetroundjoin%
\definecolor{currentfill}{rgb}{0.430507,0.564883,0.948889}%
\pgfsetfillcolor{currentfill}%
\pgfsetlinewidth{0.000000pt}%
\definecolor{currentstroke}{rgb}{0.000000,0.000000,0.000000}%
\pgfsetstrokecolor{currentstroke}%
\pgfsetdash{}{0pt}%
\pgfpathmoveto{\pgfqpoint{2.173038in}{1.929362in}}%
\pgfpathlineto{\pgfqpoint{2.214708in}{2.032021in}}%
\pgfpathlineto{\pgfqpoint{2.241158in}{2.108401in}}%
\pgfpathlineto{\pgfqpoint{2.199671in}{1.992805in}}%
\pgfpathlineto{\pgfqpoint{2.173038in}{1.929362in}}%
\pgfpathclose%
\pgfusepath{fill}%
\end{pgfscope}%
\begin{pgfscope}%
\pgfpathrectangle{\pgfqpoint{1.072000in}{0.528000in}}{\pgfqpoint{3.696000in}{3.696000in}}%
\pgfusepath{clip}%
\pgfsetbuttcap%
\pgfsetroundjoin%
\definecolor{currentfill}{rgb}{0.822420,0.856898,0.910795}%
\pgfsetfillcolor{currentfill}%
\pgfsetlinewidth{0.000000pt}%
\definecolor{currentstroke}{rgb}{0.000000,0.000000,0.000000}%
\pgfsetstrokecolor{currentstroke}%
\pgfsetdash{}{0pt}%
\pgfpathmoveto{\pgfqpoint{4.194087in}{2.441947in}}%
\pgfpathlineto{\pgfqpoint{4.244949in}{2.548621in}}%
\pgfpathlineto{\pgfqpoint{4.273504in}{2.704221in}}%
\pgfpathlineto{\pgfqpoint{4.222476in}{2.598321in}}%
\pgfpathlineto{\pgfqpoint{4.194087in}{2.441947in}}%
\pgfpathclose%
\pgfusepath{fill}%
\end{pgfscope}%
\begin{pgfscope}%
\pgfpathrectangle{\pgfqpoint{1.072000in}{0.528000in}}{\pgfqpoint{3.696000in}{3.696000in}}%
\pgfusepath{clip}%
\pgfsetbuttcap%
\pgfsetroundjoin%
\definecolor{currentfill}{rgb}{0.956371,0.775144,0.686416}%
\pgfsetfillcolor{currentfill}%
\pgfsetlinewidth{0.000000pt}%
\definecolor{currentstroke}{rgb}{0.000000,0.000000,0.000000}%
\pgfsetstrokecolor{currentstroke}%
\pgfsetdash{}{0pt}%
\pgfpathmoveto{\pgfqpoint{3.202762in}{2.951076in}}%
\pgfpathlineto{\pgfqpoint{3.248561in}{2.961716in}}%
\pgfpathlineto{\pgfqpoint{3.272646in}{2.808013in}}%
\pgfpathlineto{\pgfqpoint{3.227126in}{2.798275in}}%
\pgfpathlineto{\pgfqpoint{3.202762in}{2.951076in}}%
\pgfpathclose%
\pgfusepath{fill}%
\end{pgfscope}%
\begin{pgfscope}%
\pgfpathrectangle{\pgfqpoint{1.072000in}{0.528000in}}{\pgfqpoint{3.696000in}{3.696000in}}%
\pgfusepath{clip}%
\pgfsetbuttcap%
\pgfsetroundjoin%
\definecolor{currentfill}{rgb}{0.603162,0.731527,0.999565}%
\pgfsetfillcolor{currentfill}%
\pgfsetlinewidth{0.000000pt}%
\definecolor{currentstroke}{rgb}{0.000000,0.000000,0.000000}%
\pgfsetstrokecolor{currentstroke}%
\pgfsetdash{}{0pt}%
\pgfpathmoveto{\pgfqpoint{4.161589in}{2.133292in}}%
\pgfpathlineto{\pgfqpoint{4.212280in}{2.237427in}}%
\pgfpathlineto{\pgfqpoint{4.239978in}{2.370567in}}%
\pgfpathlineto{\pgfqpoint{4.188887in}{2.259735in}}%
\pgfpathlineto{\pgfqpoint{4.161589in}{2.133292in}}%
\pgfpathclose%
\pgfusepath{fill}%
\end{pgfscope}%
\begin{pgfscope}%
\pgfpathrectangle{\pgfqpoint{1.072000in}{0.528000in}}{\pgfqpoint{3.696000in}{3.696000in}}%
\pgfusepath{clip}%
\pgfsetbuttcap%
\pgfsetroundjoin%
\definecolor{currentfill}{rgb}{0.234377,0.305542,0.759680}%
\pgfsetfillcolor{currentfill}%
\pgfsetlinewidth{0.000000pt}%
\definecolor{currentstroke}{rgb}{0.000000,0.000000,0.000000}%
\pgfsetstrokecolor{currentstroke}%
\pgfsetdash{}{0pt}%
\pgfpathmoveto{\pgfqpoint{2.021786in}{1.693817in}}%
\pgfpathlineto{\pgfqpoint{2.065551in}{1.694755in}}%
\pgfpathlineto{\pgfqpoint{2.092742in}{1.744116in}}%
\pgfpathlineto{\pgfqpoint{2.049531in}{1.723105in}}%
\pgfpathlineto{\pgfqpoint{2.021786in}{1.693817in}}%
\pgfpathclose%
\pgfusepath{fill}%
\end{pgfscope}%
\begin{pgfscope}%
\pgfpathrectangle{\pgfqpoint{1.072000in}{0.528000in}}{\pgfqpoint{3.696000in}{3.696000in}}%
\pgfusepath{clip}%
\pgfsetbuttcap%
\pgfsetroundjoin%
\definecolor{currentfill}{rgb}{0.728970,0.817464,0.973188}%
\pgfsetfillcolor{currentfill}%
\pgfsetlinewidth{0.000000pt}%
\definecolor{currentstroke}{rgb}{0.000000,0.000000,0.000000}%
\pgfsetstrokecolor{currentstroke}%
\pgfsetdash{}{0pt}%
\pgfpathmoveto{\pgfqpoint{2.348197in}{2.332471in}}%
\pgfpathlineto{\pgfqpoint{2.389726in}{2.493235in}}%
\pgfpathlineto{\pgfqpoint{2.416898in}{2.522643in}}%
\pgfpathlineto{\pgfqpoint{2.375290in}{2.361119in}}%
\pgfpathlineto{\pgfqpoint{2.348197in}{2.332471in}}%
\pgfpathclose%
\pgfusepath{fill}%
\end{pgfscope}%
\begin{pgfscope}%
\pgfpathrectangle{\pgfqpoint{1.072000in}{0.528000in}}{\pgfqpoint{3.696000in}{3.696000in}}%
\pgfusepath{clip}%
\pgfsetbuttcap%
\pgfsetroundjoin%
\definecolor{currentfill}{rgb}{0.378598,0.503856,0.913692}%
\pgfsetfillcolor{currentfill}%
\pgfsetlinewidth{0.000000pt}%
\definecolor{currentstroke}{rgb}{0.000000,0.000000,0.000000}%
\pgfsetstrokecolor{currentstroke}%
\pgfsetdash{}{0pt}%
\pgfpathmoveto{\pgfqpoint{1.837670in}{2.004632in}}%
\pgfpathlineto{\pgfqpoint{1.885243in}{1.881530in}}%
\pgfpathlineto{\pgfqpoint{1.914806in}{1.860588in}}%
\pgfpathlineto{\pgfqpoint{1.867809in}{1.972136in}}%
\pgfpathlineto{\pgfqpoint{1.837670in}{2.004632in}}%
\pgfpathclose%
\pgfusepath{fill}%
\end{pgfscope}%
\begin{pgfscope}%
\pgfpathrectangle{\pgfqpoint{1.072000in}{0.528000in}}{\pgfqpoint{3.696000in}{3.696000in}}%
\pgfusepath{clip}%
\pgfsetbuttcap%
\pgfsetroundjoin%
\definecolor{currentfill}{rgb}{0.409611,0.540759,0.935545}%
\pgfsetfillcolor{currentfill}%
\pgfsetlinewidth{0.000000pt}%
\definecolor{currentstroke}{rgb}{0.000000,0.000000,0.000000}%
\pgfsetstrokecolor{currentstroke}%
\pgfsetdash{}{0pt}%
\pgfpathmoveto{\pgfqpoint{3.785296in}{2.064967in}}%
\pgfpathlineto{\pgfqpoint{3.828853in}{1.941973in}}%
\pgfpathlineto{\pgfqpoint{3.851531in}{1.899415in}}%
\pgfpathlineto{\pgfqpoint{3.807727in}{2.002932in}}%
\pgfpathlineto{\pgfqpoint{3.785296in}{2.064967in}}%
\pgfpathclose%
\pgfusepath{fill}%
\end{pgfscope}%
\begin{pgfscope}%
\pgfpathrectangle{\pgfqpoint{1.072000in}{0.528000in}}{\pgfqpoint{3.696000in}{3.696000in}}%
\pgfusepath{clip}%
\pgfsetbuttcap%
\pgfsetroundjoin%
\definecolor{currentfill}{rgb}{0.940879,0.805596,0.735167}%
\pgfsetfillcolor{currentfill}%
\pgfsetlinewidth{0.000000pt}%
\definecolor{currentstroke}{rgb}{0.000000,0.000000,0.000000}%
\pgfsetstrokecolor{currentstroke}%
\pgfsetdash{}{0pt}%
\pgfpathmoveto{\pgfqpoint{2.639138in}{2.773757in}}%
\pgfpathlineto{\pgfqpoint{2.682669in}{2.921135in}}%
\pgfpathlineto{\pgfqpoint{2.710032in}{2.858445in}}%
\pgfpathlineto{\pgfqpoint{2.666510in}{2.714938in}}%
\pgfpathlineto{\pgfqpoint{2.639138in}{2.773757in}}%
\pgfpathclose%
\pgfusepath{fill}%
\end{pgfscope}%
\begin{pgfscope}%
\pgfpathrectangle{\pgfqpoint{1.072000in}{0.528000in}}{\pgfqpoint{3.696000in}{3.696000in}}%
\pgfusepath{clip}%
\pgfsetbuttcap%
\pgfsetroundjoin%
\definecolor{currentfill}{rgb}{0.388852,0.516298,0.921373}%
\pgfsetfillcolor{currentfill}%
\pgfsetlinewidth{0.000000pt}%
\definecolor{currentstroke}{rgb}{0.000000,0.000000,0.000000}%
\pgfsetstrokecolor{currentstroke}%
\pgfsetdash{}{0pt}%
\pgfpathmoveto{\pgfqpoint{2.146390in}{1.864732in}}%
\pgfpathlineto{\pgfqpoint{2.188315in}{1.952126in}}%
\pgfpathlineto{\pgfqpoint{2.214708in}{2.032021in}}%
\pgfpathlineto{\pgfqpoint{2.173038in}{1.929362in}}%
\pgfpathlineto{\pgfqpoint{2.146390in}{1.864732in}}%
\pgfpathclose%
\pgfusepath{fill}%
\end{pgfscope}%
\begin{pgfscope}%
\pgfpathrectangle{\pgfqpoint{1.072000in}{0.528000in}}{\pgfqpoint{3.696000in}{3.696000in}}%
\pgfusepath{clip}%
\pgfsetbuttcap%
\pgfsetroundjoin%
\definecolor{currentfill}{rgb}{0.965899,0.740142,0.637058}%
\pgfsetfillcolor{currentfill}%
\pgfsetlinewidth{0.000000pt}%
\definecolor{currentstroke}{rgb}{0.000000,0.000000,0.000000}%
\pgfsetstrokecolor{currentstroke}%
\pgfsetdash{}{0pt}%
\pgfpathmoveto{\pgfqpoint{2.852691in}{2.938024in}}%
\pgfpathlineto{\pgfqpoint{2.897600in}{3.040999in}}%
\pgfpathlineto{\pgfqpoint{2.924090in}{2.931803in}}%
\pgfpathlineto{\pgfqpoint{2.879315in}{2.830657in}}%
\pgfpathlineto{\pgfqpoint{2.852691in}{2.938024in}}%
\pgfpathclose%
\pgfusepath{fill}%
\end{pgfscope}%
\begin{pgfscope}%
\pgfpathrectangle{\pgfqpoint{1.072000in}{0.528000in}}{\pgfqpoint{3.696000in}{3.696000in}}%
\pgfusepath{clip}%
\pgfsetbuttcap%
\pgfsetroundjoin%
\definecolor{currentfill}{rgb}{0.252663,0.332837,0.783665}%
\pgfsetfillcolor{currentfill}%
\pgfsetlinewidth{0.000000pt}%
\definecolor{currentstroke}{rgb}{0.000000,0.000000,0.000000}%
\pgfsetstrokecolor{currentstroke}%
\pgfsetdash{}{0pt}%
\pgfpathmoveto{\pgfqpoint{3.893852in}{1.770391in}}%
\pgfpathlineto{\pgfqpoint{3.938683in}{1.702640in}}%
\pgfpathlineto{\pgfqpoint{3.962425in}{1.711725in}}%
\pgfpathlineto{\pgfqpoint{3.917184in}{1.760358in}}%
\pgfpathlineto{\pgfqpoint{3.893852in}{1.770391in}}%
\pgfpathclose%
\pgfusepath{fill}%
\end{pgfscope}%
\begin{pgfscope}%
\pgfpathrectangle{\pgfqpoint{1.072000in}{0.528000in}}{\pgfqpoint{3.696000in}{3.696000in}}%
\pgfusepath{clip}%
\pgfsetbuttcap%
\pgfsetroundjoin%
\definecolor{currentfill}{rgb}{0.248091,0.326013,0.777669}%
\pgfsetfillcolor{currentfill}%
\pgfsetlinewidth{0.000000pt}%
\definecolor{currentstroke}{rgb}{0.000000,0.000000,0.000000}%
\pgfsetstrokecolor{currentstroke}%
\pgfsetdash{}{0pt}%
\pgfpathmoveto{\pgfqpoint{4.008680in}{1.695822in}}%
\pgfpathlineto{\pgfqpoint{4.056133in}{1.714211in}}%
\pgfpathlineto{\pgfqpoint{4.081093in}{1.768525in}}%
\pgfpathlineto{\pgfqpoint{4.033111in}{1.733250in}}%
\pgfpathlineto{\pgfqpoint{4.008680in}{1.695822in}}%
\pgfpathclose%
\pgfusepath{fill}%
\end{pgfscope}%
\begin{pgfscope}%
\pgfpathrectangle{\pgfqpoint{1.072000in}{0.528000in}}{\pgfqpoint{3.696000in}{3.696000in}}%
\pgfusepath{clip}%
\pgfsetbuttcap%
\pgfsetroundjoin%
\definecolor{currentfill}{rgb}{0.229806,0.298718,0.753683}%
\pgfsetfillcolor{currentfill}%
\pgfsetlinewidth{0.000000pt}%
\definecolor{currentstroke}{rgb}{0.000000,0.000000,0.000000}%
\pgfsetstrokecolor{currentstroke}%
\pgfsetdash{}{0pt}%
\pgfpathmoveto{\pgfqpoint{3.938683in}{1.702640in}}%
\pgfpathlineto{\pgfqpoint{3.984480in}{1.668552in}}%
\pgfpathlineto{\pgfqpoint{4.008680in}{1.695822in}}%
\pgfpathlineto{\pgfqpoint{3.962425in}{1.711725in}}%
\pgfpathlineto{\pgfqpoint{3.938683in}{1.702640in}}%
\pgfpathclose%
\pgfusepath{fill}%
\end{pgfscope}%
\begin{pgfscope}%
\pgfpathrectangle{\pgfqpoint{1.072000in}{0.528000in}}{\pgfqpoint{3.696000in}{3.696000in}}%
\pgfusepath{clip}%
\pgfsetbuttcap%
\pgfsetroundjoin%
\definecolor{currentfill}{rgb}{0.777378,0.840921,0.946149}%
\pgfsetfillcolor{currentfill}%
\pgfsetlinewidth{0.000000pt}%
\definecolor{currentstroke}{rgb}{0.000000,0.000000,0.000000}%
\pgfsetstrokecolor{currentstroke}%
\pgfsetdash{}{0pt}%
\pgfpathmoveto{\pgfqpoint{3.566455in}{2.608382in}}%
\pgfpathlineto{\pgfqpoint{3.610882in}{2.503442in}}%
\pgfpathlineto{\pgfqpoint{3.632877in}{2.376424in}}%
\pgfpathlineto{\pgfqpoint{3.588604in}{2.471685in}}%
\pgfpathlineto{\pgfqpoint{3.566455in}{2.608382in}}%
\pgfpathclose%
\pgfusepath{fill}%
\end{pgfscope}%
\begin{pgfscope}%
\pgfpathrectangle{\pgfqpoint{1.072000in}{0.528000in}}{\pgfqpoint{3.696000in}{3.696000in}}%
\pgfusepath{clip}%
\pgfsetbuttcap%
\pgfsetroundjoin%
\definecolor{currentfill}{rgb}{0.238948,0.312365,0.765676}%
\pgfsetfillcolor{currentfill}%
\pgfsetlinewidth{0.000000pt}%
\definecolor{currentstroke}{rgb}{0.000000,0.000000,0.000000}%
\pgfsetstrokecolor{currentstroke}%
\pgfsetdash{}{0pt}%
\pgfpathmoveto{\pgfqpoint{1.948385in}{1.726366in}}%
\pgfpathlineto{\pgfqpoint{1.993584in}{1.676811in}}%
\pgfpathlineto{\pgfqpoint{2.021786in}{1.693817in}}%
\pgfpathlineto{\pgfqpoint{1.977235in}{1.724393in}}%
\pgfpathlineto{\pgfqpoint{1.948385in}{1.726366in}}%
\pgfpathclose%
\pgfusepath{fill}%
\end{pgfscope}%
\begin{pgfscope}%
\pgfpathrectangle{\pgfqpoint{1.072000in}{0.528000in}}{\pgfqpoint{3.696000in}{3.696000in}}%
\pgfusepath{clip}%
\pgfsetbuttcap%
\pgfsetroundjoin%
\definecolor{currentfill}{rgb}{0.343278,0.459354,0.884122}%
\pgfsetfillcolor{currentfill}%
\pgfsetlinewidth{0.000000pt}%
\definecolor{currentstroke}{rgb}{0.000000,0.000000,0.000000}%
\pgfsetstrokecolor{currentstroke}%
\pgfsetdash{}{0pt}%
\pgfpathmoveto{\pgfqpoint{2.119654in}{1.801900in}}%
\pgfpathlineto{\pgfqpoint{2.161912in}{1.871900in}}%
\pgfpathlineto{\pgfqpoint{2.188315in}{1.952126in}}%
\pgfpathlineto{\pgfqpoint{2.146390in}{1.864732in}}%
\pgfpathlineto{\pgfqpoint{2.119654in}{1.801900in}}%
\pgfpathclose%
\pgfusepath{fill}%
\end{pgfscope}%
\begin{pgfscope}%
\pgfpathrectangle{\pgfqpoint{1.072000in}{0.528000in}}{\pgfqpoint{3.696000in}{3.696000in}}%
\pgfusepath{clip}%
\pgfsetbuttcap%
\pgfsetroundjoin%
\definecolor{currentfill}{rgb}{0.271104,0.360011,0.807095}%
\pgfsetfillcolor{currentfill}%
\pgfsetlinewidth{0.000000pt}%
\definecolor{currentstroke}{rgb}{0.000000,0.000000,0.000000}%
\pgfsetstrokecolor{currentstroke}%
\pgfsetdash{}{0pt}%
\pgfpathmoveto{\pgfqpoint{1.902249in}{1.805965in}}%
\pgfpathlineto{\pgfqpoint{1.948385in}{1.726366in}}%
\pgfpathlineto{\pgfqpoint{1.977235in}{1.724393in}}%
\pgfpathlineto{\pgfqpoint{1.931753in}{1.787255in}}%
\pgfpathlineto{\pgfqpoint{1.902249in}{1.805965in}}%
\pgfpathclose%
\pgfusepath{fill}%
\end{pgfscope}%
\begin{pgfscope}%
\pgfpathrectangle{\pgfqpoint{1.072000in}{0.528000in}}{\pgfqpoint{3.696000in}{3.696000in}}%
\pgfusepath{clip}%
\pgfsetbuttcap%
\pgfsetroundjoin%
\definecolor{currentfill}{rgb}{0.304174,0.406945,0.845263}%
\pgfsetfillcolor{currentfill}%
\pgfsetlinewidth{0.000000pt}%
\definecolor{currentstroke}{rgb}{0.000000,0.000000,0.000000}%
\pgfsetstrokecolor{currentstroke}%
\pgfsetdash{}{0pt}%
\pgfpathmoveto{\pgfqpoint{3.849750in}{1.867961in}}%
\pgfpathlineto{\pgfqpoint{3.893852in}{1.770391in}}%
\pgfpathlineto{\pgfqpoint{3.917184in}{1.760358in}}%
\pgfpathlineto{\pgfqpoint{3.872738in}{1.838596in}}%
\pgfpathlineto{\pgfqpoint{3.849750in}{1.867961in}}%
\pgfpathclose%
\pgfusepath{fill}%
\end{pgfscope}%
\begin{pgfscope}%
\pgfpathrectangle{\pgfqpoint{1.072000in}{0.528000in}}{\pgfqpoint{3.696000in}{3.696000in}}%
\pgfusepath{clip}%
\pgfsetbuttcap%
\pgfsetroundjoin%
\definecolor{currentfill}{rgb}{0.960581,0.762501,0.667964}%
\pgfsetfillcolor{currentfill}%
\pgfsetlinewidth{0.000000pt}%
\definecolor{currentstroke}{rgb}{0.000000,0.000000,0.000000}%
\pgfsetstrokecolor{currentstroke}%
\pgfsetdash{}{0pt}%
\pgfpathmoveto{\pgfqpoint{2.710032in}{2.858445in}}%
\pgfpathlineto{\pgfqpoint{2.754116in}{2.992029in}}%
\pgfpathlineto{\pgfqpoint{2.781276in}{2.913519in}}%
\pgfpathlineto{\pgfqpoint{2.737250in}{2.782573in}}%
\pgfpathlineto{\pgfqpoint{2.710032in}{2.858445in}}%
\pgfpathclose%
\pgfusepath{fill}%
\end{pgfscope}%
\begin{pgfscope}%
\pgfpathrectangle{\pgfqpoint{1.072000in}{0.528000in}}{\pgfqpoint{3.696000in}{3.696000in}}%
\pgfusepath{clip}%
\pgfsetbuttcap%
\pgfsetroundjoin%
\definecolor{currentfill}{rgb}{0.965899,0.740142,0.637058}%
\pgfsetfillcolor{currentfill}%
\pgfsetlinewidth{0.000000pt}%
\definecolor{currentstroke}{rgb}{0.000000,0.000000,0.000000}%
\pgfsetstrokecolor{currentstroke}%
\pgfsetdash{}{0pt}%
\pgfpathmoveto{\pgfqpoint{2.781276in}{2.913519in}}%
\pgfpathlineto{\pgfqpoint{2.825824in}{3.031903in}}%
\pgfpathlineto{\pgfqpoint{2.852691in}{2.938024in}}%
\pgfpathlineto{\pgfqpoint{2.808242in}{2.821479in}}%
\pgfpathlineto{\pgfqpoint{2.781276in}{2.913519in}}%
\pgfpathclose%
\pgfusepath{fill}%
\end{pgfscope}%
\begin{pgfscope}%
\pgfpathrectangle{\pgfqpoint{1.072000in}{0.528000in}}{\pgfqpoint{3.696000in}{3.696000in}}%
\pgfusepath{clip}%
\pgfsetbuttcap%
\pgfsetroundjoin%
\definecolor{currentfill}{rgb}{0.713852,0.808857,0.979386}%
\pgfsetfillcolor{currentfill}%
\pgfsetlinewidth{0.000000pt}%
\definecolor{currentstroke}{rgb}{0.000000,0.000000,0.000000}%
\pgfsetstrokecolor{currentstroke}%
\pgfsetdash{}{0pt}%
\pgfpathmoveto{\pgfqpoint{4.188887in}{2.259735in}}%
\pgfpathlineto{\pgfqpoint{4.239978in}{2.370567in}}%
\pgfpathlineto{\pgfqpoint{4.268087in}{2.513209in}}%
\pgfpathlineto{\pgfqpoint{4.216702in}{2.399163in}}%
\pgfpathlineto{\pgfqpoint{4.188887in}{2.259735in}}%
\pgfpathclose%
\pgfusepath{fill}%
\end{pgfscope}%
\begin{pgfscope}%
\pgfpathrectangle{\pgfqpoint{1.072000in}{0.528000in}}{\pgfqpoint{3.696000in}{3.696000in}}%
\pgfusepath{clip}%
\pgfsetbuttcap%
\pgfsetroundjoin%
\definecolor{currentfill}{rgb}{0.630089,0.752516,0.998508}%
\pgfsetfillcolor{currentfill}%
\pgfsetlinewidth{0.000000pt}%
\definecolor{currentstroke}{rgb}{0.000000,0.000000,0.000000}%
\pgfsetstrokecolor{currentstroke}%
\pgfsetdash{}{0pt}%
\pgfpathmoveto{\pgfqpoint{3.676855in}{2.380675in}}%
\pgfpathlineto{\pgfqpoint{3.720562in}{2.249428in}}%
\pgfpathlineto{\pgfqpoint{3.742632in}{2.158050in}}%
\pgfpathlineto{\pgfqpoint{3.698901in}{2.273983in}}%
\pgfpathlineto{\pgfqpoint{3.676855in}{2.380675in}}%
\pgfpathclose%
\pgfusepath{fill}%
\end{pgfscope}%
\begin{pgfscope}%
\pgfpathrectangle{\pgfqpoint{1.072000in}{0.528000in}}{\pgfqpoint{3.696000in}{3.696000in}}%
\pgfusepath{clip}%
\pgfsetbuttcap%
\pgfsetroundjoin%
\definecolor{currentfill}{rgb}{0.304174,0.406945,0.845263}%
\pgfsetfillcolor{currentfill}%
\pgfsetlinewidth{0.000000pt}%
\definecolor{currentstroke}{rgb}{0.000000,0.000000,0.000000}%
\pgfsetstrokecolor{currentstroke}%
\pgfsetdash{}{0pt}%
\pgfpathmoveto{\pgfqpoint{2.092742in}{1.744116in}}%
\pgfpathlineto{\pgfqpoint{2.135412in}{1.794901in}}%
\pgfpathlineto{\pgfqpoint{2.161912in}{1.871900in}}%
\pgfpathlineto{\pgfqpoint{2.119654in}{1.801900in}}%
\pgfpathlineto{\pgfqpoint{2.092742in}{1.744116in}}%
\pgfpathclose%
\pgfusepath{fill}%
\end{pgfscope}%
\begin{pgfscope}%
\pgfpathrectangle{\pgfqpoint{1.072000in}{0.528000in}}{\pgfqpoint{3.696000in}{3.696000in}}%
\pgfusepath{clip}%
\pgfsetbuttcap%
\pgfsetroundjoin%
\definecolor{currentfill}{rgb}{0.960581,0.762501,0.667964}%
\pgfsetfillcolor{currentfill}%
\pgfsetlinewidth{0.000000pt}%
\definecolor{currentstroke}{rgb}{0.000000,0.000000,0.000000}%
\pgfsetstrokecolor{currentstroke}%
\pgfsetdash{}{0pt}%
\pgfpathmoveto{\pgfqpoint{3.248561in}{2.961716in}}%
\pgfpathlineto{\pgfqpoint{3.294381in}{2.956929in}}%
\pgfpathlineto{\pgfqpoint{3.318179in}{2.802905in}}%
\pgfpathlineto{\pgfqpoint{3.272646in}{2.808013in}}%
\pgfpathlineto{\pgfqpoint{3.248561in}{2.961716in}}%
\pgfpathclose%
\pgfusepath{fill}%
\end{pgfscope}%
\begin{pgfscope}%
\pgfpathrectangle{\pgfqpoint{1.072000in}{0.528000in}}{\pgfqpoint{3.696000in}{3.696000in}}%
\pgfusepath{clip}%
\pgfsetbuttcap%
\pgfsetroundjoin%
\definecolor{currentfill}{rgb}{0.229806,0.298718,0.753683}%
\pgfsetfillcolor{currentfill}%
\pgfsetlinewidth{0.000000pt}%
\definecolor{currentstroke}{rgb}{0.000000,0.000000,0.000000}%
\pgfsetstrokecolor{currentstroke}%
\pgfsetdash{}{0pt}%
\pgfpathmoveto{\pgfqpoint{1.993584in}{1.676811in}}%
\pgfpathlineto{\pgfqpoint{2.037966in}{1.657157in}}%
\pgfpathlineto{\pgfqpoint{2.065551in}{1.694755in}}%
\pgfpathlineto{\pgfqpoint{2.021786in}{1.693817in}}%
\pgfpathlineto{\pgfqpoint{1.993584in}{1.676811in}}%
\pgfpathclose%
\pgfusepath{fill}%
\end{pgfscope}%
\begin{pgfscope}%
\pgfpathrectangle{\pgfqpoint{1.072000in}{0.528000in}}{\pgfqpoint{3.696000in}{3.696000in}}%
\pgfusepath{clip}%
\pgfsetbuttcap%
\pgfsetroundjoin%
\definecolor{currentfill}{rgb}{0.718985,0.811993,0.977656}%
\pgfsetfillcolor{currentfill}%
\pgfsetlinewidth{0.000000pt}%
\definecolor{currentstroke}{rgb}{0.000000,0.000000,0.000000}%
\pgfsetstrokecolor{currentstroke}%
\pgfsetdash{}{0pt}%
\pgfpathmoveto{\pgfqpoint{2.321228in}{2.291984in}}%
\pgfpathlineto{\pgfqpoint{2.362698in}{2.450072in}}%
\pgfpathlineto{\pgfqpoint{2.389726in}{2.493235in}}%
\pgfpathlineto{\pgfqpoint{2.348197in}{2.332471in}}%
\pgfpathlineto{\pgfqpoint{2.321228in}{2.291984in}}%
\pgfpathclose%
\pgfusepath{fill}%
\end{pgfscope}%
\begin{pgfscope}%
\pgfpathrectangle{\pgfqpoint{1.072000in}{0.528000in}}{\pgfqpoint{3.696000in}{3.696000in}}%
\pgfusepath{clip}%
\pgfsetbuttcap%
\pgfsetroundjoin%
\definecolor{currentfill}{rgb}{0.333490,0.446265,0.874452}%
\pgfsetfillcolor{currentfill}%
\pgfsetlinewidth{0.000000pt}%
\definecolor{currentstroke}{rgb}{0.000000,0.000000,0.000000}%
\pgfsetstrokecolor{currentstroke}%
\pgfsetdash{}{0pt}%
\pgfpathmoveto{\pgfqpoint{1.855107in}{1.914183in}}%
\pgfpathlineto{\pgfqpoint{1.902249in}{1.805965in}}%
\pgfpathlineto{\pgfqpoint{1.931753in}{1.787255in}}%
\pgfpathlineto{\pgfqpoint{1.885243in}{1.881530in}}%
\pgfpathlineto{\pgfqpoint{1.855107in}{1.914183in}}%
\pgfpathclose%
\pgfusepath{fill}%
\end{pgfscope}%
\begin{pgfscope}%
\pgfpathrectangle{\pgfqpoint{1.072000in}{0.528000in}}{\pgfqpoint{3.696000in}{3.696000in}}%
\pgfusepath{clip}%
\pgfsetbuttcap%
\pgfsetroundjoin%
\definecolor{currentfill}{rgb}{0.839351,0.861167,0.894494}%
\pgfsetfillcolor{currentfill}%
\pgfsetlinewidth{0.000000pt}%
\definecolor{currentstroke}{rgb}{0.000000,0.000000,0.000000}%
\pgfsetstrokecolor{currentstroke}%
\pgfsetdash{}{0pt}%
\pgfpathmoveto{\pgfqpoint{3.521715in}{2.702397in}}%
\pgfpathlineto{\pgfqpoint{3.566455in}{2.608382in}}%
\pgfpathlineto{\pgfqpoint{3.588604in}{2.471685in}}%
\pgfpathlineto{\pgfqpoint{3.544078in}{2.558631in}}%
\pgfpathlineto{\pgfqpoint{3.521715in}{2.702397in}}%
\pgfpathclose%
\pgfusepath{fill}%
\end{pgfscope}%
\begin{pgfscope}%
\pgfpathrectangle{\pgfqpoint{1.072000in}{0.528000in}}{\pgfqpoint{3.696000in}{3.696000in}}%
\pgfusepath{clip}%
\pgfsetbuttcap%
\pgfsetroundjoin%
\definecolor{currentfill}{rgb}{0.234377,0.305542,0.759680}%
\pgfsetfillcolor{currentfill}%
\pgfsetlinewidth{0.000000pt}%
\definecolor{currentstroke}{rgb}{0.000000,0.000000,0.000000}%
\pgfsetstrokecolor{currentstroke}%
\pgfsetdash{}{0pt}%
\pgfpathmoveto{\pgfqpoint{3.984480in}{1.668552in}}%
\pgfpathlineto{\pgfqpoint{4.031452in}{1.670342in}}%
\pgfpathlineto{\pgfqpoint{4.056133in}{1.714211in}}%
\pgfpathlineto{\pgfqpoint{4.008680in}{1.695822in}}%
\pgfpathlineto{\pgfqpoint{3.984480in}{1.668552in}}%
\pgfpathclose%
\pgfusepath{fill}%
\end{pgfscope}%
\begin{pgfscope}%
\pgfpathrectangle{\pgfqpoint{1.072000in}{0.528000in}}{\pgfqpoint{3.696000in}{3.696000in}}%
\pgfusepath{clip}%
\pgfsetbuttcap%
\pgfsetroundjoin%
\definecolor{currentfill}{rgb}{0.859385,0.864431,0.872111}%
\pgfsetfillcolor{currentfill}%
\pgfsetlinewidth{0.000000pt}%
\definecolor{currentstroke}{rgb}{0.000000,0.000000,0.000000}%
\pgfsetstrokecolor{currentstroke}%
\pgfsetdash{}{0pt}%
\pgfpathmoveto{\pgfqpoint{2.444183in}{2.538164in}}%
\pgfpathlineto{\pgfqpoint{2.486328in}{2.705797in}}%
\pgfpathlineto{\pgfqpoint{2.513778in}{2.705512in}}%
\pgfpathlineto{\pgfqpoint{2.471545in}{2.540037in}}%
\pgfpathlineto{\pgfqpoint{2.444183in}{2.538164in}}%
\pgfpathclose%
\pgfusepath{fill}%
\end{pgfscope}%
\begin{pgfscope}%
\pgfpathrectangle{\pgfqpoint{1.072000in}{0.528000in}}{\pgfqpoint{3.696000in}{3.696000in}}%
\pgfusepath{clip}%
\pgfsetbuttcap%
\pgfsetroundjoin%
\definecolor{currentfill}{rgb}{0.928116,0.822197,0.765141}%
\pgfsetfillcolor{currentfill}%
\pgfsetlinewidth{0.000000pt}%
\definecolor{currentstroke}{rgb}{0.000000,0.000000,0.000000}%
\pgfsetstrokecolor{currentstroke}%
\pgfsetdash{}{0pt}%
\pgfpathmoveto{\pgfqpoint{2.541264in}{2.690688in}}%
\pgfpathlineto{\pgfqpoint{2.584140in}{2.851524in}}%
\pgfpathlineto{\pgfqpoint{2.611667in}{2.819768in}}%
\pgfpathlineto{\pgfqpoint{2.568744in}{2.662122in}}%
\pgfpathlineto{\pgfqpoint{2.541264in}{2.690688in}}%
\pgfpathclose%
\pgfusepath{fill}%
\end{pgfscope}%
\begin{pgfscope}%
\pgfpathrectangle{\pgfqpoint{1.072000in}{0.528000in}}{\pgfqpoint{3.696000in}{3.696000in}}%
\pgfusepath{clip}%
\pgfsetbuttcap%
\pgfsetroundjoin%
\definecolor{currentfill}{rgb}{0.510824,0.649397,0.985079}%
\pgfsetfillcolor{currentfill}%
\pgfsetlinewidth{0.000000pt}%
\definecolor{currentstroke}{rgb}{0.000000,0.000000,0.000000}%
\pgfsetstrokecolor{currentstroke}%
\pgfsetdash{}{0pt}%
\pgfpathmoveto{\pgfqpoint{3.741847in}{2.201354in}}%
\pgfpathlineto{\pgfqpoint{3.785296in}{2.064967in}}%
\pgfpathlineto{\pgfqpoint{3.807727in}{2.002932in}}%
\pgfpathlineto{\pgfqpoint{3.764134in}{2.121557in}}%
\pgfpathlineto{\pgfqpoint{3.741847in}{2.201354in}}%
\pgfpathclose%
\pgfusepath{fill}%
\end{pgfscope}%
\begin{pgfscope}%
\pgfpathrectangle{\pgfqpoint{1.072000in}{0.528000in}}{\pgfqpoint{3.696000in}{3.696000in}}%
\pgfusepath{clip}%
\pgfsetbuttcap%
\pgfsetroundjoin%
\definecolor{currentfill}{rgb}{0.969522,0.700833,0.587508}%
\pgfsetfillcolor{currentfill}%
\pgfsetlinewidth{0.000000pt}%
\definecolor{currentstroke}{rgb}{0.000000,0.000000,0.000000}%
\pgfsetstrokecolor{currentstroke}%
\pgfsetdash{}{0pt}%
\pgfpathmoveto{\pgfqpoint{3.086204in}{3.023731in}}%
\pgfpathlineto{\pgfqpoint{3.132009in}{3.064920in}}%
\pgfpathlineto{\pgfqpoint{3.157046in}{2.925117in}}%
\pgfpathlineto{\pgfqpoint{3.111472in}{2.884175in}}%
\pgfpathlineto{\pgfqpoint{3.086204in}{3.023731in}}%
\pgfpathclose%
\pgfusepath{fill}%
\end{pgfscope}%
\begin{pgfscope}%
\pgfpathrectangle{\pgfqpoint{1.072000in}{0.528000in}}{\pgfqpoint{3.696000in}{3.696000in}}%
\pgfusepath{clip}%
\pgfsetbuttcap%
\pgfsetroundjoin%
\definecolor{currentfill}{rgb}{0.266381,0.353304,0.801637}%
\pgfsetfillcolor{currentfill}%
\pgfsetlinewidth{0.000000pt}%
\definecolor{currentstroke}{rgb}{0.000000,0.000000,0.000000}%
\pgfsetstrokecolor{currentstroke}%
\pgfsetdash{}{0pt}%
\pgfpathmoveto{\pgfqpoint{2.065551in}{1.694755in}}%
\pgfpathlineto{\pgfqpoint{2.108711in}{1.724922in}}%
\pgfpathlineto{\pgfqpoint{2.135412in}{1.794901in}}%
\pgfpathlineto{\pgfqpoint{2.092742in}{1.744116in}}%
\pgfpathlineto{\pgfqpoint{2.065551in}{1.694755in}}%
\pgfpathclose%
\pgfusepath{fill}%
\end{pgfscope}%
\begin{pgfscope}%
\pgfpathrectangle{\pgfqpoint{1.072000in}{0.528000in}}{\pgfqpoint{3.696000in}{3.696000in}}%
\pgfusepath{clip}%
\pgfsetbuttcap%
\pgfsetroundjoin%
\definecolor{currentfill}{rgb}{0.388852,0.516298,0.921373}%
\pgfsetfillcolor{currentfill}%
\pgfsetlinewidth{0.000000pt}%
\definecolor{currentstroke}{rgb}{0.000000,0.000000,0.000000}%
\pgfsetstrokecolor{currentstroke}%
\pgfsetdash{}{0pt}%
\pgfpathmoveto{\pgfqpoint{4.106406in}{1.834913in}}%
\pgfpathlineto{\pgfqpoint{4.156237in}{1.915621in}}%
\pgfpathlineto{\pgfqpoint{4.182495in}{2.007972in}}%
\pgfpathlineto{\pgfqpoint{4.132141in}{1.914616in}}%
\pgfpathlineto{\pgfqpoint{4.106406in}{1.834913in}}%
\pgfpathclose%
\pgfusepath{fill}%
\end{pgfscope}%
\begin{pgfscope}%
\pgfpathrectangle{\pgfqpoint{1.072000in}{0.528000in}}{\pgfqpoint{3.696000in}{3.696000in}}%
\pgfusepath{clip}%
\pgfsetbuttcap%
\pgfsetroundjoin%
\definecolor{currentfill}{rgb}{0.969289,0.684982,0.568975}%
\pgfsetfillcolor{currentfill}%
\pgfsetlinewidth{0.000000pt}%
\definecolor{currentstroke}{rgb}{0.000000,0.000000,0.000000}%
\pgfsetstrokecolor{currentstroke}%
\pgfsetdash{}{0pt}%
\pgfpathmoveto{\pgfqpoint{2.969258in}{3.019369in}}%
\pgfpathlineto{\pgfqpoint{3.014767in}{3.091015in}}%
\pgfpathlineto{\pgfqpoint{3.040610in}{2.966879in}}%
\pgfpathlineto{\pgfqpoint{2.995286in}{2.895118in}}%
\pgfpathlineto{\pgfqpoint{2.969258in}{3.019369in}}%
\pgfpathclose%
\pgfusepath{fill}%
\end{pgfscope}%
\begin{pgfscope}%
\pgfpathrectangle{\pgfqpoint{1.072000in}{0.528000in}}{\pgfqpoint{3.696000in}{3.696000in}}%
\pgfusepath{clip}%
\pgfsetbuttcap%
\pgfsetroundjoin%
\definecolor{currentfill}{rgb}{0.378598,0.503856,0.913692}%
\pgfsetfillcolor{currentfill}%
\pgfsetlinewidth{0.000000pt}%
\definecolor{currentstroke}{rgb}{0.000000,0.000000,0.000000}%
\pgfsetstrokecolor{currentstroke}%
\pgfsetdash{}{0pt}%
\pgfpathmoveto{\pgfqpoint{3.806125in}{1.990115in}}%
\pgfpathlineto{\pgfqpoint{3.849750in}{1.867961in}}%
\pgfpathlineto{\pgfqpoint{3.872738in}{1.838596in}}%
\pgfpathlineto{\pgfqpoint{3.828853in}{1.941973in}}%
\pgfpathlineto{\pgfqpoint{3.806125in}{1.990115in}}%
\pgfpathclose%
\pgfusepath{fill}%
\end{pgfscope}%
\begin{pgfscope}%
\pgfpathrectangle{\pgfqpoint{1.072000in}{0.528000in}}{\pgfqpoint{3.696000in}{3.696000in}}%
\pgfusepath{clip}%
\pgfsetbuttcap%
\pgfsetroundjoin%
\definecolor{currentfill}{rgb}{0.451739,0.588181,0.960201}%
\pgfsetfillcolor{currentfill}%
\pgfsetlinewidth{0.000000pt}%
\definecolor{currentstroke}{rgb}{0.000000,0.000000,0.000000}%
\pgfsetstrokecolor{currentstroke}%
\pgfsetdash{}{0pt}%
\pgfpathmoveto{\pgfqpoint{4.132141in}{1.914616in}}%
\pgfpathlineto{\pgfqpoint{4.182495in}{2.007972in}}%
\pgfpathlineto{\pgfqpoint{4.209200in}{2.112638in}}%
\pgfpathlineto{\pgfqpoint{4.158349in}{2.008349in}}%
\pgfpathlineto{\pgfqpoint{4.132141in}{1.914616in}}%
\pgfpathclose%
\pgfusepath{fill}%
\end{pgfscope}%
\begin{pgfscope}%
\pgfpathrectangle{\pgfqpoint{1.072000in}{0.528000in}}{\pgfqpoint{3.696000in}{3.696000in}}%
\pgfusepath{clip}%
\pgfsetbuttcap%
\pgfsetroundjoin%
\definecolor{currentfill}{rgb}{0.960581,0.762501,0.667964}%
\pgfsetfillcolor{currentfill}%
\pgfsetlinewidth{0.000000pt}%
\definecolor{currentstroke}{rgb}{0.000000,0.000000,0.000000}%
\pgfsetstrokecolor{currentstroke}%
\pgfsetdash{}{0pt}%
\pgfpathmoveto{\pgfqpoint{3.294381in}{2.956929in}}%
\pgfpathlineto{\pgfqpoint{3.340162in}{2.936675in}}%
\pgfpathlineto{\pgfqpoint{3.363665in}{2.782800in}}%
\pgfpathlineto{\pgfqpoint{3.318179in}{2.802905in}}%
\pgfpathlineto{\pgfqpoint{3.294381in}{2.956929in}}%
\pgfpathclose%
\pgfusepath{fill}%
\end{pgfscope}%
\begin{pgfscope}%
\pgfpathrectangle{\pgfqpoint{1.072000in}{0.528000in}}{\pgfqpoint{3.696000in}{3.696000in}}%
\pgfusepath{clip}%
\pgfsetbuttcap%
\pgfsetroundjoin%
\definecolor{currentfill}{rgb}{0.333490,0.446265,0.874452}%
\pgfsetfillcolor{currentfill}%
\pgfsetlinewidth{0.000000pt}%
\definecolor{currentstroke}{rgb}{0.000000,0.000000,0.000000}%
\pgfsetstrokecolor{currentstroke}%
\pgfsetdash{}{0pt}%
\pgfpathmoveto{\pgfqpoint{4.081093in}{1.768525in}}%
\pgfpathlineto{\pgfqpoint{4.130396in}{1.835433in}}%
\pgfpathlineto{\pgfqpoint{4.156237in}{1.915621in}}%
\pgfpathlineto{\pgfqpoint{4.106406in}{1.834913in}}%
\pgfpathlineto{\pgfqpoint{4.081093in}{1.768525in}}%
\pgfpathclose%
\pgfusepath{fill}%
\end{pgfscope}%
\begin{pgfscope}%
\pgfpathrectangle{\pgfqpoint{1.072000in}{0.528000in}}{\pgfqpoint{3.696000in}{3.696000in}}%
\pgfusepath{clip}%
\pgfsetbuttcap%
\pgfsetroundjoin%
\definecolor{currentfill}{rgb}{0.813693,0.854282,0.918480}%
\pgfsetfillcolor{currentfill}%
\pgfsetlinewidth{0.000000pt}%
\definecolor{currentstroke}{rgb}{0.000000,0.000000,0.000000}%
\pgfsetstrokecolor{currentstroke}%
\pgfsetdash{}{0pt}%
\pgfpathmoveto{\pgfqpoint{4.216702in}{2.399163in}}%
\pgfpathlineto{\pgfqpoint{4.268087in}{2.513209in}}%
\pgfpathlineto{\pgfqpoint{4.296502in}{2.662096in}}%
\pgfpathlineto{\pgfqpoint{4.244949in}{2.548621in}}%
\pgfpathlineto{\pgfqpoint{4.216702in}{2.399163in}}%
\pgfpathclose%
\pgfusepath{fill}%
\end{pgfscope}%
\begin{pgfscope}%
\pgfpathrectangle{\pgfqpoint{1.072000in}{0.528000in}}{\pgfqpoint{3.696000in}{3.696000in}}%
\pgfusepath{clip}%
\pgfsetbuttcap%
\pgfsetroundjoin%
\definecolor{currentfill}{rgb}{0.532568,0.669801,0.990393}%
\pgfsetfillcolor{currentfill}%
\pgfsetlinewidth{0.000000pt}%
\definecolor{currentstroke}{rgb}{0.000000,0.000000,0.000000}%
\pgfsetstrokecolor{currentstroke}%
\pgfsetdash{}{0pt}%
\pgfpathmoveto{\pgfqpoint{4.158349in}{2.008349in}}%
\pgfpathlineto{\pgfqpoint{4.209200in}{2.112638in}}%
\pgfpathlineto{\pgfqpoint{4.236360in}{2.229146in}}%
\pgfpathlineto{\pgfqpoint{4.185061in}{2.116182in}}%
\pgfpathlineto{\pgfqpoint{4.158349in}{2.008349in}}%
\pgfpathclose%
\pgfusepath{fill}%
\end{pgfscope}%
\begin{pgfscope}%
\pgfpathrectangle{\pgfqpoint{1.072000in}{0.528000in}}{\pgfqpoint{3.696000in}{3.696000in}}%
\pgfusepath{clip}%
\pgfsetbuttcap%
\pgfsetroundjoin%
\definecolor{currentfill}{rgb}{0.891817,0.851973,0.829085}%
\pgfsetfillcolor{currentfill}%
\pgfsetlinewidth{0.000000pt}%
\definecolor{currentstroke}{rgb}{0.000000,0.000000,0.000000}%
\pgfsetstrokecolor{currentstroke}%
\pgfsetdash{}{0pt}%
\pgfpathmoveto{\pgfqpoint{3.476674in}{2.783296in}}%
\pgfpathlineto{\pgfqpoint{3.521715in}{2.702397in}}%
\pgfpathlineto{\pgfqpoint{3.544078in}{2.558631in}}%
\pgfpathlineto{\pgfqpoint{3.499295in}{2.634738in}}%
\pgfpathlineto{\pgfqpoint{3.476674in}{2.783296in}}%
\pgfpathclose%
\pgfusepath{fill}%
\end{pgfscope}%
\begin{pgfscope}%
\pgfpathrectangle{\pgfqpoint{1.072000in}{0.528000in}}{\pgfqpoint{3.696000in}{3.696000in}}%
\pgfusepath{clip}%
\pgfsetbuttcap%
\pgfsetroundjoin%
\definecolor{currentfill}{rgb}{0.238948,0.312365,0.765676}%
\pgfsetfillcolor{currentfill}%
\pgfsetlinewidth{0.000000pt}%
\definecolor{currentstroke}{rgb}{0.000000,0.000000,0.000000}%
\pgfsetstrokecolor{currentstroke}%
\pgfsetdash{}{0pt}%
\pgfpathmoveto{\pgfqpoint{3.915014in}{1.700220in}}%
\pgfpathlineto{\pgfqpoint{3.960420in}{1.649060in}}%
\pgfpathlineto{\pgfqpoint{3.984480in}{1.668552in}}%
\pgfpathlineto{\pgfqpoint{3.938683in}{1.702640in}}%
\pgfpathlineto{\pgfqpoint{3.915014in}{1.700220in}}%
\pgfpathclose%
\pgfusepath{fill}%
\end{pgfscope}%
\begin{pgfscope}%
\pgfpathrectangle{\pgfqpoint{1.072000in}{0.528000in}}{\pgfqpoint{3.696000in}{3.696000in}}%
\pgfusepath{clip}%
\pgfsetbuttcap%
\pgfsetroundjoin%
\definecolor{currentfill}{rgb}{0.698454,0.799450,0.984577}%
\pgfsetfillcolor{currentfill}%
\pgfsetlinewidth{0.000000pt}%
\definecolor{currentstroke}{rgb}{0.000000,0.000000,0.000000}%
\pgfsetstrokecolor{currentstroke}%
\pgfsetdash{}{0pt}%
\pgfpathmoveto{\pgfqpoint{2.294399in}{2.240302in}}%
\pgfpathlineto{\pgfqpoint{2.335835in}{2.393678in}}%
\pgfpathlineto{\pgfqpoint{2.362698in}{2.450072in}}%
\pgfpathlineto{\pgfqpoint{2.321228in}{2.291984in}}%
\pgfpathlineto{\pgfqpoint{2.294399in}{2.240302in}}%
\pgfpathclose%
\pgfusepath{fill}%
\end{pgfscope}%
\begin{pgfscope}%
\pgfpathrectangle{\pgfqpoint{1.072000in}{0.528000in}}{\pgfqpoint{3.696000in}{3.696000in}}%
\pgfusepath{clip}%
\pgfsetbuttcap%
\pgfsetroundjoin%
\definecolor{currentfill}{rgb}{0.275827,0.366717,0.812553}%
\pgfsetfillcolor{currentfill}%
\pgfsetlinewidth{0.000000pt}%
\definecolor{currentstroke}{rgb}{0.000000,0.000000,0.000000}%
\pgfsetstrokecolor{currentstroke}%
\pgfsetdash{}{0pt}%
\pgfpathmoveto{\pgfqpoint{3.870520in}{1.785664in}}%
\pgfpathlineto{\pgfqpoint{3.915014in}{1.700220in}}%
\pgfpathlineto{\pgfqpoint{3.938683in}{1.702640in}}%
\pgfpathlineto{\pgfqpoint{3.893852in}{1.770391in}}%
\pgfpathlineto{\pgfqpoint{3.870520in}{1.785664in}}%
\pgfpathclose%
\pgfusepath{fill}%
\end{pgfscope}%
\begin{pgfscope}%
\pgfpathrectangle{\pgfqpoint{1.072000in}{0.528000in}}{\pgfqpoint{3.696000in}{3.696000in}}%
\pgfusepath{clip}%
\pgfsetbuttcap%
\pgfsetroundjoin%
\definecolor{currentfill}{rgb}{0.956371,0.775144,0.686416}%
\pgfsetfillcolor{currentfill}%
\pgfsetlinewidth{0.000000pt}%
\definecolor{currentstroke}{rgb}{0.000000,0.000000,0.000000}%
\pgfsetstrokecolor{currentstroke}%
\pgfsetdash{}{0pt}%
\pgfpathmoveto{\pgfqpoint{3.340162in}{2.936675in}}%
\pgfpathlineto{\pgfqpoint{3.385845in}{2.900923in}}%
\pgfpathlineto{\pgfqpoint{3.409048in}{2.747779in}}%
\pgfpathlineto{\pgfqpoint{3.363665in}{2.782800in}}%
\pgfpathlineto{\pgfqpoint{3.340162in}{2.936675in}}%
\pgfpathclose%
\pgfusepath{fill}%
\end{pgfscope}%
\begin{pgfscope}%
\pgfpathrectangle{\pgfqpoint{1.072000in}{0.528000in}}{\pgfqpoint{3.696000in}{3.696000in}}%
\pgfusepath{clip}%
\pgfsetbuttcap%
\pgfsetroundjoin%
\definecolor{currentfill}{rgb}{0.294718,0.393542,0.834384}%
\pgfsetfillcolor{currentfill}%
\pgfsetlinewidth{0.000000pt}%
\definecolor{currentstroke}{rgb}{0.000000,0.000000,0.000000}%
\pgfsetstrokecolor{currentstroke}%
\pgfsetdash{}{0pt}%
\pgfpathmoveto{\pgfqpoint{4.056133in}{1.714211in}}%
\pgfpathlineto{\pgfqpoint{4.104924in}{1.766727in}}%
\pgfpathlineto{\pgfqpoint{4.130396in}{1.835433in}}%
\pgfpathlineto{\pgfqpoint{4.081093in}{1.768525in}}%
\pgfpathlineto{\pgfqpoint{4.056133in}{1.714211in}}%
\pgfpathclose%
\pgfusepath{fill}%
\end{pgfscope}%
\begin{pgfscope}%
\pgfpathrectangle{\pgfqpoint{1.072000in}{0.528000in}}{\pgfqpoint{3.696000in}{3.696000in}}%
\pgfusepath{clip}%
\pgfsetbuttcap%
\pgfsetroundjoin%
\definecolor{currentfill}{rgb}{0.243520,0.319189,0.771672}%
\pgfsetfillcolor{currentfill}%
\pgfsetlinewidth{0.000000pt}%
\definecolor{currentstroke}{rgb}{0.000000,0.000000,0.000000}%
\pgfsetstrokecolor{currentstroke}%
\pgfsetdash{}{0pt}%
\pgfpathmoveto{\pgfqpoint{2.037966in}{1.657157in}}%
\pgfpathlineto{\pgfqpoint{2.081690in}{1.665811in}}%
\pgfpathlineto{\pgfqpoint{2.108711in}{1.724922in}}%
\pgfpathlineto{\pgfqpoint{2.065551in}{1.694755in}}%
\pgfpathlineto{\pgfqpoint{2.037966in}{1.657157in}}%
\pgfpathclose%
\pgfusepath{fill}%
\end{pgfscope}%
\begin{pgfscope}%
\pgfpathrectangle{\pgfqpoint{1.072000in}{0.528000in}}{\pgfqpoint{3.696000in}{3.696000in}}%
\pgfusepath{clip}%
\pgfsetbuttcap%
\pgfsetroundjoin%
\definecolor{currentfill}{rgb}{0.728970,0.817464,0.973188}%
\pgfsetfillcolor{currentfill}%
\pgfsetlinewidth{0.000000pt}%
\definecolor{currentstroke}{rgb}{0.000000,0.000000,0.000000}%
\pgfsetstrokecolor{currentstroke}%
\pgfsetdash{}{0pt}%
\pgfpathmoveto{\pgfqpoint{3.632895in}{2.509787in}}%
\pgfpathlineto{\pgfqpoint{3.676855in}{2.380675in}}%
\pgfpathlineto{\pgfqpoint{3.698901in}{2.273983in}}%
\pgfpathlineto{\pgfqpoint{3.655014in}{2.390664in}}%
\pgfpathlineto{\pgfqpoint{3.632895in}{2.509787in}}%
\pgfpathclose%
\pgfusepath{fill}%
\end{pgfscope}%
\begin{pgfscope}%
\pgfpathrectangle{\pgfqpoint{1.072000in}{0.528000in}}{\pgfqpoint{3.696000in}{3.696000in}}%
\pgfusepath{clip}%
\pgfsetbuttcap%
\pgfsetroundjoin%
\definecolor{currentfill}{rgb}{0.928116,0.822197,0.765141}%
\pgfsetfillcolor{currentfill}%
\pgfsetlinewidth{0.000000pt}%
\definecolor{currentstroke}{rgb}{0.000000,0.000000,0.000000}%
\pgfsetstrokecolor{currentstroke}%
\pgfsetdash{}{0pt}%
\pgfpathmoveto{\pgfqpoint{3.431368in}{2.849708in}}%
\pgfpathlineto{\pgfqpoint{3.476674in}{2.783296in}}%
\pgfpathlineto{\pgfqpoint{3.499295in}{2.634738in}}%
\pgfpathlineto{\pgfqpoint{3.454274in}{2.698188in}}%
\pgfpathlineto{\pgfqpoint{3.431368in}{2.849708in}}%
\pgfpathclose%
\pgfusepath{fill}%
\end{pgfscope}%
\begin{pgfscope}%
\pgfpathrectangle{\pgfqpoint{1.072000in}{0.528000in}}{\pgfqpoint{3.696000in}{3.696000in}}%
\pgfusepath{clip}%
\pgfsetbuttcap%
\pgfsetroundjoin%
\definecolor{currentfill}{rgb}{0.257234,0.339661,0.789661}%
\pgfsetfillcolor{currentfill}%
\pgfsetlinewidth{0.000000pt}%
\definecolor{currentstroke}{rgb}{0.000000,0.000000,0.000000}%
\pgfsetstrokecolor{currentstroke}%
\pgfsetdash{}{0pt}%
\pgfpathmoveto{\pgfqpoint{1.918932in}{1.742703in}}%
\pgfpathlineto{\pgfqpoint{1.964817in}{1.674603in}}%
\pgfpathlineto{\pgfqpoint{1.993584in}{1.676811in}}%
\pgfpathlineto{\pgfqpoint{1.948385in}{1.726366in}}%
\pgfpathlineto{\pgfqpoint{1.918932in}{1.742703in}}%
\pgfpathclose%
\pgfusepath{fill}%
\end{pgfscope}%
\begin{pgfscope}%
\pgfpathrectangle{\pgfqpoint{1.072000in}{0.528000in}}{\pgfqpoint{3.696000in}{3.696000in}}%
\pgfusepath{clip}%
\pgfsetbuttcap%
\pgfsetroundjoin%
\definecolor{currentfill}{rgb}{0.947345,0.794696,0.716991}%
\pgfsetfillcolor{currentfill}%
\pgfsetlinewidth{0.000000pt}%
\definecolor{currentstroke}{rgb}{0.000000,0.000000,0.000000}%
\pgfsetstrokecolor{currentstroke}%
\pgfsetdash{}{0pt}%
\pgfpathmoveto{\pgfqpoint{3.385845in}{2.900923in}}%
\pgfpathlineto{\pgfqpoint{3.431368in}{2.849708in}}%
\pgfpathlineto{\pgfqpoint{3.454274in}{2.698188in}}%
\pgfpathlineto{\pgfqpoint{3.409048in}{2.747779in}}%
\pgfpathlineto{\pgfqpoint{3.385845in}{2.900923in}}%
\pgfpathclose%
\pgfusepath{fill}%
\end{pgfscope}%
\begin{pgfscope}%
\pgfpathrectangle{\pgfqpoint{1.072000in}{0.528000in}}{\pgfqpoint{3.696000in}{3.696000in}}%
\pgfusepath{clip}%
\pgfsetbuttcap%
\pgfsetroundjoin%
\definecolor{currentfill}{rgb}{0.624703,0.748318,0.998719}%
\pgfsetfillcolor{currentfill}%
\pgfsetlinewidth{0.000000pt}%
\definecolor{currentstroke}{rgb}{0.000000,0.000000,0.000000}%
\pgfsetstrokecolor{currentstroke}%
\pgfsetdash{}{0pt}%
\pgfpathmoveto{\pgfqpoint{4.185061in}{2.116182in}}%
\pgfpathlineto{\pgfqpoint{4.236360in}{2.229146in}}%
\pgfpathlineto{\pgfqpoint{4.263953in}{2.356308in}}%
\pgfpathlineto{\pgfqpoint{4.212280in}{2.237427in}}%
\pgfpathlineto{\pgfqpoint{4.185061in}{2.116182in}}%
\pgfpathclose%
\pgfusepath{fill}%
\end{pgfscope}%
\begin{pgfscope}%
\pgfpathrectangle{\pgfqpoint{1.072000in}{0.528000in}}{\pgfqpoint{3.696000in}{3.696000in}}%
\pgfusepath{clip}%
\pgfsetbuttcap%
\pgfsetroundjoin%
\definecolor{currentfill}{rgb}{0.229806,0.298718,0.753683}%
\pgfsetfillcolor{currentfill}%
\pgfsetlinewidth{0.000000pt}%
\definecolor{currentstroke}{rgb}{0.000000,0.000000,0.000000}%
\pgfsetstrokecolor{currentstroke}%
\pgfsetdash{}{0pt}%
\pgfpathmoveto{\pgfqpoint{3.960420in}{1.649060in}}%
\pgfpathlineto{\pgfqpoint{4.006969in}{1.635041in}}%
\pgfpathlineto{\pgfqpoint{4.031452in}{1.670342in}}%
\pgfpathlineto{\pgfqpoint{3.984480in}{1.668552in}}%
\pgfpathlineto{\pgfqpoint{3.960420in}{1.649060in}}%
\pgfpathclose%
\pgfusepath{fill}%
\end{pgfscope}%
\begin{pgfscope}%
\pgfpathrectangle{\pgfqpoint{1.072000in}{0.528000in}}{\pgfqpoint{3.696000in}{3.696000in}}%
\pgfusepath{clip}%
\pgfsetbuttcap%
\pgfsetroundjoin%
\definecolor{currentfill}{rgb}{0.229806,0.298718,0.753683}%
\pgfsetfillcolor{currentfill}%
\pgfsetlinewidth{0.000000pt}%
\definecolor{currentstroke}{rgb}{0.000000,0.000000,0.000000}%
\pgfsetstrokecolor{currentstroke}%
\pgfsetdash{}{0pt}%
\pgfpathmoveto{\pgfqpoint{1.964817in}{1.674603in}}%
\pgfpathlineto{\pgfqpoint{2.009869in}{1.634431in}}%
\pgfpathlineto{\pgfqpoint{2.037966in}{1.657157in}}%
\pgfpathlineto{\pgfqpoint{1.993584in}{1.676811in}}%
\pgfpathlineto{\pgfqpoint{1.964817in}{1.674603in}}%
\pgfpathclose%
\pgfusepath{fill}%
\end{pgfscope}%
\begin{pgfscope}%
\pgfpathrectangle{\pgfqpoint{1.072000in}{0.528000in}}{\pgfqpoint{3.696000in}{3.696000in}}%
\pgfusepath{clip}%
\pgfsetbuttcap%
\pgfsetroundjoin%
\definecolor{currentfill}{rgb}{0.430507,0.564883,0.948889}%
\pgfsetfillcolor{currentfill}%
\pgfsetlinewidth{0.000000pt}%
\definecolor{currentstroke}{rgb}{0.000000,0.000000,0.000000}%
\pgfsetstrokecolor{currentstroke}%
\pgfsetdash{}{0pt}%
\pgfpathmoveto{\pgfqpoint{1.806953in}{2.047979in}}%
\pgfpathlineto{\pgfqpoint{1.855107in}{1.914183in}}%
\pgfpathlineto{\pgfqpoint{1.885243in}{1.881530in}}%
\pgfpathlineto{\pgfqpoint{1.837670in}{2.004632in}}%
\pgfpathlineto{\pgfqpoint{1.806953in}{2.047979in}}%
\pgfpathclose%
\pgfusepath{fill}%
\end{pgfscope}%
\begin{pgfscope}%
\pgfpathrectangle{\pgfqpoint{1.072000in}{0.528000in}}{\pgfqpoint{3.696000in}{3.696000in}}%
\pgfusepath{clip}%
\pgfsetbuttcap%
\pgfsetroundjoin%
\definecolor{currentfill}{rgb}{0.661968,0.775491,0.993937}%
\pgfsetfillcolor{currentfill}%
\pgfsetlinewidth{0.000000pt}%
\definecolor{currentstroke}{rgb}{0.000000,0.000000,0.000000}%
\pgfsetstrokecolor{currentstroke}%
\pgfsetdash{}{0pt}%
\pgfpathmoveto{\pgfqpoint{2.267713in}{2.178557in}}%
\pgfpathlineto{\pgfqpoint{2.309152in}{2.325033in}}%
\pgfpathlineto{\pgfqpoint{2.335835in}{2.393678in}}%
\pgfpathlineto{\pgfqpoint{2.294399in}{2.240302in}}%
\pgfpathlineto{\pgfqpoint{2.267713in}{2.178557in}}%
\pgfpathclose%
\pgfusepath{fill}%
\end{pgfscope}%
\begin{pgfscope}%
\pgfpathrectangle{\pgfqpoint{1.072000in}{0.528000in}}{\pgfqpoint{3.696000in}{3.696000in}}%
\pgfusepath{clip}%
\pgfsetbuttcap%
\pgfsetroundjoin%
\definecolor{currentfill}{rgb}{0.309060,0.413498,0.850128}%
\pgfsetfillcolor{currentfill}%
\pgfsetlinewidth{0.000000pt}%
\definecolor{currentstroke}{rgb}{0.000000,0.000000,0.000000}%
\pgfsetstrokecolor{currentstroke}%
\pgfsetdash{}{0pt}%
\pgfpathmoveto{\pgfqpoint{1.872119in}{1.838259in}}%
\pgfpathlineto{\pgfqpoint{1.918932in}{1.742703in}}%
\pgfpathlineto{\pgfqpoint{1.948385in}{1.726366in}}%
\pgfpathlineto{\pgfqpoint{1.902249in}{1.805965in}}%
\pgfpathlineto{\pgfqpoint{1.872119in}{1.838259in}}%
\pgfpathclose%
\pgfusepath{fill}%
\end{pgfscope}%
\begin{pgfscope}%
\pgfpathrectangle{\pgfqpoint{1.072000in}{0.528000in}}{\pgfqpoint{3.696000in}{3.696000in}}%
\pgfusepath{clip}%
\pgfsetbuttcap%
\pgfsetroundjoin%
\definecolor{currentfill}{rgb}{0.343278,0.459354,0.884122}%
\pgfsetfillcolor{currentfill}%
\pgfsetlinewidth{0.000000pt}%
\definecolor{currentstroke}{rgb}{0.000000,0.000000,0.000000}%
\pgfsetstrokecolor{currentstroke}%
\pgfsetdash{}{0pt}%
\pgfpathmoveto{\pgfqpoint{3.826683in}{1.900832in}}%
\pgfpathlineto{\pgfqpoint{3.870520in}{1.785664in}}%
\pgfpathlineto{\pgfqpoint{3.893852in}{1.770391in}}%
\pgfpathlineto{\pgfqpoint{3.849750in}{1.867961in}}%
\pgfpathlineto{\pgfqpoint{3.826683in}{1.900832in}}%
\pgfpathclose%
\pgfusepath{fill}%
\end{pgfscope}%
\begin{pgfscope}%
\pgfpathrectangle{\pgfqpoint{1.072000in}{0.528000in}}{\pgfqpoint{3.696000in}{3.696000in}}%
\pgfusepath{clip}%
\pgfsetbuttcap%
\pgfsetroundjoin%
\definecolor{currentfill}{rgb}{0.962708,0.753557,0.655601}%
\pgfsetfillcolor{currentfill}%
\pgfsetlinewidth{0.000000pt}%
\definecolor{currentstroke}{rgb}{0.000000,0.000000,0.000000}%
\pgfsetstrokecolor{currentstroke}%
\pgfsetdash{}{0pt}%
\pgfpathmoveto{\pgfqpoint{2.611667in}{2.819768in}}%
\pgfpathlineto{\pgfqpoint{2.655203in}{2.968906in}}%
\pgfpathlineto{\pgfqpoint{2.682669in}{2.921135in}}%
\pgfpathlineto{\pgfqpoint{2.639138in}{2.773757in}}%
\pgfpathlineto{\pgfqpoint{2.611667in}{2.819768in}}%
\pgfpathclose%
\pgfusepath{fill}%
\end{pgfscope}%
\begin{pgfscope}%
\pgfpathrectangle{\pgfqpoint{1.072000in}{0.528000in}}{\pgfqpoint{3.696000in}{3.696000in}}%
\pgfusepath{clip}%
\pgfsetbuttcap%
\pgfsetroundjoin%
\definecolor{currentfill}{rgb}{0.271104,0.360011,0.807095}%
\pgfsetfillcolor{currentfill}%
\pgfsetlinewidth{0.000000pt}%
\definecolor{currentstroke}{rgb}{0.000000,0.000000,0.000000}%
\pgfsetstrokecolor{currentstroke}%
\pgfsetdash{}{0pt}%
\pgfpathmoveto{\pgfqpoint{4.031452in}{1.670342in}}%
\pgfpathlineto{\pgfqpoint{4.079763in}{1.708419in}}%
\pgfpathlineto{\pgfqpoint{4.104924in}{1.766727in}}%
\pgfpathlineto{\pgfqpoint{4.056133in}{1.714211in}}%
\pgfpathlineto{\pgfqpoint{4.031452in}{1.670342in}}%
\pgfpathclose%
\pgfusepath{fill}%
\end{pgfscope}%
\begin{pgfscope}%
\pgfpathrectangle{\pgfqpoint{1.072000in}{0.528000in}}{\pgfqpoint{3.696000in}{3.696000in}}%
\pgfusepath{clip}%
\pgfsetbuttcap%
\pgfsetroundjoin%
\definecolor{currentfill}{rgb}{0.871493,0.862309,0.857016}%
\pgfsetfillcolor{currentfill}%
\pgfsetlinewidth{0.000000pt}%
\definecolor{currentstroke}{rgb}{0.000000,0.000000,0.000000}%
\pgfsetstrokecolor{currentstroke}%
\pgfsetdash{}{0pt}%
\pgfpathmoveto{\pgfqpoint{2.416898in}{2.522643in}}%
\pgfpathlineto{\pgfqpoint{2.458953in}{2.691100in}}%
\pgfpathlineto{\pgfqpoint{2.486328in}{2.705797in}}%
\pgfpathlineto{\pgfqpoint{2.444183in}{2.538164in}}%
\pgfpathlineto{\pgfqpoint{2.416898in}{2.522643in}}%
\pgfpathclose%
\pgfusepath{fill}%
\end{pgfscope}%
\begin{pgfscope}%
\pgfpathrectangle{\pgfqpoint{1.072000in}{0.528000in}}{\pgfqpoint{3.696000in}{3.696000in}}%
\pgfusepath{clip}%
\pgfsetbuttcap%
\pgfsetroundjoin%
\definecolor{currentfill}{rgb}{0.619318,0.744121,0.998931}%
\pgfsetfillcolor{currentfill}%
\pgfsetlinewidth{0.000000pt}%
\definecolor{currentstroke}{rgb}{0.000000,0.000000,0.000000}%
\pgfsetstrokecolor{currentstroke}%
\pgfsetdash{}{0pt}%
\pgfpathmoveto{\pgfqpoint{2.241158in}{2.108401in}}%
\pgfpathlineto{\pgfqpoint{2.282646in}{2.245632in}}%
\pgfpathlineto{\pgfqpoint{2.309152in}{2.325033in}}%
\pgfpathlineto{\pgfqpoint{2.267713in}{2.178557in}}%
\pgfpathlineto{\pgfqpoint{2.241158in}{2.108401in}}%
\pgfpathclose%
\pgfusepath{fill}%
\end{pgfscope}%
\begin{pgfscope}%
\pgfpathrectangle{\pgfqpoint{1.072000in}{0.528000in}}{\pgfqpoint{3.696000in}{3.696000in}}%
\pgfusepath{clip}%
\pgfsetbuttcap%
\pgfsetroundjoin%
\definecolor{currentfill}{rgb}{0.483854,0.622050,0.974808}%
\pgfsetfillcolor{currentfill}%
\pgfsetlinewidth{0.000000pt}%
\definecolor{currentstroke}{rgb}{0.000000,0.000000,0.000000}%
\pgfsetstrokecolor{currentstroke}%
\pgfsetdash{}{0pt}%
\pgfpathmoveto{\pgfqpoint{3.762732in}{2.130547in}}%
\pgfpathlineto{\pgfqpoint{3.806125in}{1.990115in}}%
\pgfpathlineto{\pgfqpoint{3.828853in}{1.941973in}}%
\pgfpathlineto{\pgfqpoint{3.785296in}{2.064967in}}%
\pgfpathlineto{\pgfqpoint{3.762732in}{2.130547in}}%
\pgfpathclose%
\pgfusepath{fill}%
\end{pgfscope}%
\begin{pgfscope}%
\pgfpathrectangle{\pgfqpoint{1.072000in}{0.528000in}}{\pgfqpoint{3.696000in}{3.696000in}}%
\pgfusepath{clip}%
\pgfsetbuttcap%
\pgfsetroundjoin%
\definecolor{currentfill}{rgb}{0.630089,0.752516,0.998508}%
\pgfsetfillcolor{currentfill}%
\pgfsetlinewidth{0.000000pt}%
\definecolor{currentstroke}{rgb}{0.000000,0.000000,0.000000}%
\pgfsetstrokecolor{currentstroke}%
\pgfsetdash{}{0pt}%
\pgfpathmoveto{\pgfqpoint{3.698316in}{2.344597in}}%
\pgfpathlineto{\pgfqpoint{3.741847in}{2.201354in}}%
\pgfpathlineto{\pgfqpoint{3.764134in}{2.121557in}}%
\pgfpathlineto{\pgfqpoint{3.720562in}{2.249428in}}%
\pgfpathlineto{\pgfqpoint{3.698316in}{2.344597in}}%
\pgfpathclose%
\pgfusepath{fill}%
\end{pgfscope}%
\begin{pgfscope}%
\pgfpathrectangle{\pgfqpoint{1.072000in}{0.528000in}}{\pgfqpoint{3.696000in}{3.696000in}}%
\pgfusepath{clip}%
\pgfsetbuttcap%
\pgfsetroundjoin%
\definecolor{currentfill}{rgb}{0.964911,0.640159,0.519806}%
\pgfsetfillcolor{currentfill}%
\pgfsetlinewidth{0.000000pt}%
\definecolor{currentstroke}{rgb}{0.000000,0.000000,0.000000}%
\pgfsetstrokecolor{currentstroke}%
\pgfsetdash{}{0pt}%
\pgfpathmoveto{\pgfqpoint{2.897600in}{3.040999in}}%
\pgfpathlineto{\pgfqpoint{2.942923in}{3.126942in}}%
\pgfpathlineto{\pgfqpoint{2.969258in}{3.019369in}}%
\pgfpathlineto{\pgfqpoint{2.924090in}{2.931803in}}%
\pgfpathlineto{\pgfqpoint{2.897600in}{3.040999in}}%
\pgfpathclose%
\pgfusepath{fill}%
\end{pgfscope}%
\begin{pgfscope}%
\pgfpathrectangle{\pgfqpoint{1.072000in}{0.528000in}}{\pgfqpoint{3.696000in}{3.696000in}}%
\pgfusepath{clip}%
\pgfsetbuttcap%
\pgfsetroundjoin%
\definecolor{currentfill}{rgb}{0.906154,0.842091,0.806151}%
\pgfsetfillcolor{currentfill}%
\pgfsetlinewidth{0.000000pt}%
\definecolor{currentstroke}{rgb}{0.000000,0.000000,0.000000}%
\pgfsetstrokecolor{currentstroke}%
\pgfsetdash{}{0pt}%
\pgfpathmoveto{\pgfqpoint{4.244949in}{2.548621in}}%
\pgfpathlineto{\pgfqpoint{4.296502in}{2.662096in}}%
\pgfpathlineto{\pgfqpoint{4.325078in}{2.813175in}}%
\pgfpathlineto{\pgfqpoint{4.273504in}{2.704221in}}%
\pgfpathlineto{\pgfqpoint{4.244949in}{2.548621in}}%
\pgfpathclose%
\pgfusepath{fill}%
\end{pgfscope}%
\begin{pgfscope}%
\pgfpathrectangle{\pgfqpoint{1.072000in}{0.528000in}}{\pgfqpoint{3.696000in}{3.696000in}}%
\pgfusepath{clip}%
\pgfsetbuttcap%
\pgfsetroundjoin%
\definecolor{currentfill}{rgb}{0.967317,0.657471,0.538160}%
\pgfsetfillcolor{currentfill}%
\pgfsetlinewidth{0.000000pt}%
\definecolor{currentstroke}{rgb}{0.000000,0.000000,0.000000}%
\pgfsetstrokecolor{currentstroke}%
\pgfsetdash{}{0pt}%
\pgfpathmoveto{\pgfqpoint{3.132009in}{3.064920in}}%
\pgfpathlineto{\pgfqpoint{3.177965in}{3.090395in}}%
\pgfpathlineto{\pgfqpoint{3.202762in}{2.951076in}}%
\pgfpathlineto{\pgfqpoint{3.157046in}{2.925117in}}%
\pgfpathlineto{\pgfqpoint{3.132009in}{3.064920in}}%
\pgfpathclose%
\pgfusepath{fill}%
\end{pgfscope}%
\begin{pgfscope}%
\pgfpathrectangle{\pgfqpoint{1.072000in}{0.528000in}}{\pgfqpoint{3.696000in}{3.696000in}}%
\pgfusepath{clip}%
\pgfsetbuttcap%
\pgfsetroundjoin%
\definecolor{currentfill}{rgb}{0.724041,0.814910,0.975651}%
\pgfsetfillcolor{currentfill}%
\pgfsetlinewidth{0.000000pt}%
\definecolor{currentstroke}{rgb}{0.000000,0.000000,0.000000}%
\pgfsetstrokecolor{currentstroke}%
\pgfsetdash{}{0pt}%
\pgfpathmoveto{\pgfqpoint{4.212280in}{2.237427in}}%
\pgfpathlineto{\pgfqpoint{4.263953in}{2.356308in}}%
\pgfpathlineto{\pgfqpoint{4.291926in}{2.492173in}}%
\pgfpathlineto{\pgfqpoint{4.239978in}{2.370567in}}%
\pgfpathlineto{\pgfqpoint{4.212280in}{2.237427in}}%
\pgfpathclose%
\pgfusepath{fill}%
\end{pgfscope}%
\begin{pgfscope}%
\pgfpathrectangle{\pgfqpoint{1.072000in}{0.528000in}}{\pgfqpoint{3.696000in}{3.696000in}}%
\pgfusepath{clip}%
\pgfsetbuttcap%
\pgfsetroundjoin%
\definecolor{currentfill}{rgb}{0.565182,0.699438,0.996635}%
\pgfsetfillcolor{currentfill}%
\pgfsetlinewidth{0.000000pt}%
\definecolor{currentstroke}{rgb}{0.000000,0.000000,0.000000}%
\pgfsetstrokecolor{currentstroke}%
\pgfsetdash{}{0pt}%
\pgfpathmoveto{\pgfqpoint{2.214708in}{2.032021in}}%
\pgfpathlineto{\pgfqpoint{2.256302in}{2.157559in}}%
\pgfpathlineto{\pgfqpoint{2.282646in}{2.245632in}}%
\pgfpathlineto{\pgfqpoint{2.241158in}{2.108401in}}%
\pgfpathlineto{\pgfqpoint{2.214708in}{2.032021in}}%
\pgfpathclose%
\pgfusepath{fill}%
\end{pgfscope}%
\begin{pgfscope}%
\pgfpathrectangle{\pgfqpoint{1.072000in}{0.528000in}}{\pgfqpoint{3.696000in}{3.696000in}}%
\pgfusepath{clip}%
\pgfsetbuttcap%
\pgfsetroundjoin%
\definecolor{currentfill}{rgb}{0.229806,0.298718,0.753683}%
\pgfsetfillcolor{currentfill}%
\pgfsetlinewidth{0.000000pt}%
\definecolor{currentstroke}{rgb}{0.000000,0.000000,0.000000}%
\pgfsetstrokecolor{currentstroke}%
\pgfsetdash{}{0pt}%
\pgfpathmoveto{\pgfqpoint{2.009869in}{1.634431in}}%
\pgfpathlineto{\pgfqpoint{2.054221in}{1.621266in}}%
\pgfpathlineto{\pgfqpoint{2.081690in}{1.665811in}}%
\pgfpathlineto{\pgfqpoint{2.037966in}{1.657157in}}%
\pgfpathlineto{\pgfqpoint{2.009869in}{1.634431in}}%
\pgfpathclose%
\pgfusepath{fill}%
\end{pgfscope}%
\begin{pgfscope}%
\pgfpathrectangle{\pgfqpoint{1.072000in}{0.528000in}}{\pgfqpoint{3.696000in}{3.696000in}}%
\pgfusepath{clip}%
\pgfsetbuttcap%
\pgfsetroundjoin%
\definecolor{currentfill}{rgb}{0.505423,0.643995,0.983157}%
\pgfsetfillcolor{currentfill}%
\pgfsetlinewidth{0.000000pt}%
\definecolor{currentstroke}{rgb}{0.000000,0.000000,0.000000}%
\pgfsetstrokecolor{currentstroke}%
\pgfsetdash{}{0pt}%
\pgfpathmoveto{\pgfqpoint{2.188315in}{1.952126in}}%
\pgfpathlineto{\pgfqpoint{2.230082in}{2.063507in}}%
\pgfpathlineto{\pgfqpoint{2.256302in}{2.157559in}}%
\pgfpathlineto{\pgfqpoint{2.214708in}{2.032021in}}%
\pgfpathlineto{\pgfqpoint{2.188315in}{1.952126in}}%
\pgfpathclose%
\pgfusepath{fill}%
\end{pgfscope}%
\begin{pgfscope}%
\pgfpathrectangle{\pgfqpoint{1.072000in}{0.528000in}}{\pgfqpoint{3.696000in}{3.696000in}}%
\pgfusepath{clip}%
\pgfsetbuttcap%
\pgfsetroundjoin%
\definecolor{currentfill}{rgb}{0.441123,0.576532,0.954545}%
\pgfsetfillcolor{currentfill}%
\pgfsetlinewidth{0.000000pt}%
\definecolor{currentstroke}{rgb}{0.000000,0.000000,0.000000}%
\pgfsetstrokecolor{currentstroke}%
\pgfsetdash{}{0pt}%
\pgfpathmoveto{\pgfqpoint{2.161912in}{1.871900in}}%
\pgfpathlineto{\pgfqpoint{2.203928in}{1.966760in}}%
\pgfpathlineto{\pgfqpoint{2.230082in}{2.063507in}}%
\pgfpathlineto{\pgfqpoint{2.188315in}{1.952126in}}%
\pgfpathlineto{\pgfqpoint{2.161912in}{1.871900in}}%
\pgfpathclose%
\pgfusepath{fill}%
\end{pgfscope}%
\begin{pgfscope}%
\pgfpathrectangle{\pgfqpoint{1.072000in}{0.528000in}}{\pgfqpoint{3.696000in}{3.696000in}}%
\pgfusepath{clip}%
\pgfsetbuttcap%
\pgfsetroundjoin%
\definecolor{currentfill}{rgb}{0.378598,0.503856,0.913692}%
\pgfsetfillcolor{currentfill}%
\pgfsetlinewidth{0.000000pt}%
\definecolor{currentstroke}{rgb}{0.000000,0.000000,0.000000}%
\pgfsetstrokecolor{currentstroke}%
\pgfsetdash{}{0pt}%
\pgfpathmoveto{\pgfqpoint{2.135412in}{1.794901in}}%
\pgfpathlineto{\pgfqpoint{2.177758in}{1.871109in}}%
\pgfpathlineto{\pgfqpoint{2.203928in}{1.966760in}}%
\pgfpathlineto{\pgfqpoint{2.161912in}{1.871900in}}%
\pgfpathlineto{\pgfqpoint{2.135412in}{1.794901in}}%
\pgfpathclose%
\pgfusepath{fill}%
\end{pgfscope}%
\begin{pgfscope}%
\pgfpathrectangle{\pgfqpoint{1.072000in}{0.528000in}}{\pgfqpoint{3.696000in}{3.696000in}}%
\pgfusepath{clip}%
\pgfsetbuttcap%
\pgfsetroundjoin%
\definecolor{currentfill}{rgb}{0.248091,0.326013,0.777669}%
\pgfsetfillcolor{currentfill}%
\pgfsetlinewidth{0.000000pt}%
\definecolor{currentstroke}{rgb}{0.000000,0.000000,0.000000}%
\pgfsetstrokecolor{currentstroke}%
\pgfsetdash{}{0pt}%
\pgfpathmoveto{\pgfqpoint{3.891325in}{1.701610in}}%
\pgfpathlineto{\pgfqpoint{3.936414in}{1.634908in}}%
\pgfpathlineto{\pgfqpoint{3.960420in}{1.649060in}}%
\pgfpathlineto{\pgfqpoint{3.915014in}{1.700220in}}%
\pgfpathlineto{\pgfqpoint{3.891325in}{1.701610in}}%
\pgfpathclose%
\pgfusepath{fill}%
\end{pgfscope}%
\begin{pgfscope}%
\pgfpathrectangle{\pgfqpoint{1.072000in}{0.528000in}}{\pgfqpoint{3.696000in}{3.696000in}}%
\pgfusepath{clip}%
\pgfsetbuttcap%
\pgfsetroundjoin%
\definecolor{currentfill}{rgb}{0.328604,0.439712,0.869587}%
\pgfsetfillcolor{currentfill}%
\pgfsetlinewidth{0.000000pt}%
\definecolor{currentstroke}{rgb}{0.000000,0.000000,0.000000}%
\pgfsetstrokecolor{currentstroke}%
\pgfsetdash{}{0pt}%
\pgfpathmoveto{\pgfqpoint{2.108711in}{1.724922in}}%
\pgfpathlineto{\pgfqpoint{2.151468in}{1.780712in}}%
\pgfpathlineto{\pgfqpoint{2.177758in}{1.871109in}}%
\pgfpathlineto{\pgfqpoint{2.135412in}{1.794901in}}%
\pgfpathlineto{\pgfqpoint{2.108711in}{1.724922in}}%
\pgfpathclose%
\pgfusepath{fill}%
\end{pgfscope}%
\begin{pgfscope}%
\pgfpathrectangle{\pgfqpoint{1.072000in}{0.528000in}}{\pgfqpoint{3.696000in}{3.696000in}}%
\pgfusepath{clip}%
\pgfsetbuttcap%
\pgfsetroundjoin%
\definecolor{currentfill}{rgb}{0.818056,0.855590,0.914638}%
\pgfsetfillcolor{currentfill}%
\pgfsetlinewidth{0.000000pt}%
\definecolor{currentstroke}{rgb}{0.000000,0.000000,0.000000}%
\pgfsetstrokecolor{currentstroke}%
\pgfsetdash{}{0pt}%
\pgfpathmoveto{\pgfqpoint{3.588606in}{2.631942in}}%
\pgfpathlineto{\pgfqpoint{3.632895in}{2.509787in}}%
\pgfpathlineto{\pgfqpoint{3.655014in}{2.390664in}}%
\pgfpathlineto{\pgfqpoint{3.610882in}{2.503442in}}%
\pgfpathlineto{\pgfqpoint{3.588606in}{2.631942in}}%
\pgfpathclose%
\pgfusepath{fill}%
\end{pgfscope}%
\begin{pgfscope}%
\pgfpathrectangle{\pgfqpoint{1.072000in}{0.528000in}}{\pgfqpoint{3.696000in}{3.696000in}}%
\pgfusepath{clip}%
\pgfsetbuttcap%
\pgfsetroundjoin%
\definecolor{currentfill}{rgb}{0.947345,0.794696,0.716991}%
\pgfsetfillcolor{currentfill}%
\pgfsetlinewidth{0.000000pt}%
\definecolor{currentstroke}{rgb}{0.000000,0.000000,0.000000}%
\pgfsetstrokecolor{currentstroke}%
\pgfsetdash{}{0pt}%
\pgfpathmoveto{\pgfqpoint{2.513778in}{2.705512in}}%
\pgfpathlineto{\pgfqpoint{2.556601in}{2.868224in}}%
\pgfpathlineto{\pgfqpoint{2.584140in}{2.851524in}}%
\pgfpathlineto{\pgfqpoint{2.541264in}{2.690688in}}%
\pgfpathlineto{\pgfqpoint{2.513778in}{2.705512in}}%
\pgfpathclose%
\pgfusepath{fill}%
\end{pgfscope}%
\begin{pgfscope}%
\pgfpathrectangle{\pgfqpoint{1.072000in}{0.528000in}}{\pgfqpoint{3.696000in}{3.696000in}}%
\pgfusepath{clip}%
\pgfsetbuttcap%
\pgfsetroundjoin%
\definecolor{currentfill}{rgb}{0.388852,0.516298,0.921373}%
\pgfsetfillcolor{currentfill}%
\pgfsetlinewidth{0.000000pt}%
\definecolor{currentstroke}{rgb}{0.000000,0.000000,0.000000}%
\pgfsetstrokecolor{currentstroke}%
\pgfsetdash{}{0pt}%
\pgfpathmoveto{\pgfqpoint{1.824341in}{1.959318in}}%
\pgfpathlineto{\pgfqpoint{1.872119in}{1.838259in}}%
\pgfpathlineto{\pgfqpoint{1.902249in}{1.805965in}}%
\pgfpathlineto{\pgfqpoint{1.855107in}{1.914183in}}%
\pgfpathlineto{\pgfqpoint{1.824341in}{1.959318in}}%
\pgfpathclose%
\pgfusepath{fill}%
\end{pgfscope}%
\begin{pgfscope}%
\pgfpathrectangle{\pgfqpoint{1.072000in}{0.528000in}}{\pgfqpoint{3.696000in}{3.696000in}}%
\pgfusepath{clip}%
\pgfsetbuttcap%
\pgfsetroundjoin%
\definecolor{currentfill}{rgb}{0.248091,0.326013,0.777669}%
\pgfsetfillcolor{currentfill}%
\pgfsetlinewidth{0.000000pt}%
\definecolor{currentstroke}{rgb}{0.000000,0.000000,0.000000}%
\pgfsetstrokecolor{currentstroke}%
\pgfsetdash{}{0pt}%
\pgfpathmoveto{\pgfqpoint{4.006969in}{1.635041in}}%
\pgfpathlineto{\pgfqpoint{4.054846in}{1.659135in}}%
\pgfpathlineto{\pgfqpoint{4.079763in}{1.708419in}}%
\pgfpathlineto{\pgfqpoint{4.031452in}{1.670342in}}%
\pgfpathlineto{\pgfqpoint{4.006969in}{1.635041in}}%
\pgfpathclose%
\pgfusepath{fill}%
\end{pgfscope}%
\begin{pgfscope}%
\pgfpathrectangle{\pgfqpoint{1.072000in}{0.528000in}}{\pgfqpoint{3.696000in}{3.696000in}}%
\pgfusepath{clip}%
\pgfsetbuttcap%
\pgfsetroundjoin%
\definecolor{currentfill}{rgb}{0.959385,0.610306,0.489382}%
\pgfsetfillcolor{currentfill}%
\pgfsetlinewidth{0.000000pt}%
\definecolor{currentstroke}{rgb}{0.000000,0.000000,0.000000}%
\pgfsetstrokecolor{currentstroke}%
\pgfsetdash{}{0pt}%
\pgfpathmoveto{\pgfqpoint{3.014767in}{3.091015in}}%
\pgfpathlineto{\pgfqpoint{3.060553in}{3.145761in}}%
\pgfpathlineto{\pgfqpoint{3.086204in}{3.023731in}}%
\pgfpathlineto{\pgfqpoint{3.040610in}{2.966879in}}%
\pgfpathlineto{\pgfqpoint{3.014767in}{3.091015in}}%
\pgfpathclose%
\pgfusepath{fill}%
\end{pgfscope}%
\begin{pgfscope}%
\pgfpathrectangle{\pgfqpoint{1.072000in}{0.528000in}}{\pgfqpoint{3.696000in}{3.696000in}}%
\pgfusepath{clip}%
\pgfsetbuttcap%
\pgfsetroundjoin%
\definecolor{currentfill}{rgb}{0.969289,0.684982,0.568975}%
\pgfsetfillcolor{currentfill}%
\pgfsetlinewidth{0.000000pt}%
\definecolor{currentstroke}{rgb}{0.000000,0.000000,0.000000}%
\pgfsetstrokecolor{currentstroke}%
\pgfsetdash{}{0pt}%
\pgfpathmoveto{\pgfqpoint{2.682669in}{2.921135in}}%
\pgfpathlineto{\pgfqpoint{2.726808in}{3.054642in}}%
\pgfpathlineto{\pgfqpoint{2.754116in}{2.992029in}}%
\pgfpathlineto{\pgfqpoint{2.710032in}{2.858445in}}%
\pgfpathlineto{\pgfqpoint{2.682669in}{2.921135in}}%
\pgfpathclose%
\pgfusepath{fill}%
\end{pgfscope}%
\begin{pgfscope}%
\pgfpathrectangle{\pgfqpoint{1.072000in}{0.528000in}}{\pgfqpoint{3.696000in}{3.696000in}}%
\pgfusepath{clip}%
\pgfsetbuttcap%
\pgfsetroundjoin%
\definecolor{currentfill}{rgb}{0.229806,0.298718,0.753683}%
\pgfsetfillcolor{currentfill}%
\pgfsetlinewidth{0.000000pt}%
\definecolor{currentstroke}{rgb}{0.000000,0.000000,0.000000}%
\pgfsetstrokecolor{currentstroke}%
\pgfsetdash{}{0pt}%
\pgfpathmoveto{\pgfqpoint{3.936414in}{1.634908in}}%
\pgfpathlineto{\pgfqpoint{3.982606in}{1.606315in}}%
\pgfpathlineto{\pgfqpoint{4.006969in}{1.635041in}}%
\pgfpathlineto{\pgfqpoint{3.960420in}{1.649060in}}%
\pgfpathlineto{\pgfqpoint{3.936414in}{1.634908in}}%
\pgfpathclose%
\pgfusepath{fill}%
\end{pgfscope}%
\begin{pgfscope}%
\pgfpathrectangle{\pgfqpoint{1.072000in}{0.528000in}}{\pgfqpoint{3.696000in}{3.696000in}}%
\pgfusepath{clip}%
\pgfsetbuttcap%
\pgfsetroundjoin%
\definecolor{currentfill}{rgb}{0.962701,0.628218,0.507636}%
\pgfsetfillcolor{currentfill}%
\pgfsetlinewidth{0.000000pt}%
\definecolor{currentstroke}{rgb}{0.000000,0.000000,0.000000}%
\pgfsetstrokecolor{currentstroke}%
\pgfsetdash{}{0pt}%
\pgfpathmoveto{\pgfqpoint{2.825824in}{3.031903in}}%
\pgfpathlineto{\pgfqpoint{2.870860in}{3.132675in}}%
\pgfpathlineto{\pgfqpoint{2.897600in}{3.040999in}}%
\pgfpathlineto{\pgfqpoint{2.852691in}{2.938024in}}%
\pgfpathlineto{\pgfqpoint{2.825824in}{3.031903in}}%
\pgfpathclose%
\pgfusepath{fill}%
\end{pgfscope}%
\begin{pgfscope}%
\pgfpathrectangle{\pgfqpoint{1.072000in}{0.528000in}}{\pgfqpoint{3.696000in}{3.696000in}}%
\pgfusepath{clip}%
\pgfsetbuttcap%
\pgfsetroundjoin%
\definecolor{currentfill}{rgb}{0.280550,0.373423,0.818011}%
\pgfsetfillcolor{currentfill}%
\pgfsetlinewidth{0.000000pt}%
\definecolor{currentstroke}{rgb}{0.000000,0.000000,0.000000}%
\pgfsetstrokecolor{currentstroke}%
\pgfsetdash{}{0pt}%
\pgfpathmoveto{\pgfqpoint{2.081690in}{1.665811in}}%
\pgfpathlineto{\pgfqpoint{2.124939in}{1.699900in}}%
\pgfpathlineto{\pgfqpoint{2.151468in}{1.780712in}}%
\pgfpathlineto{\pgfqpoint{2.108711in}{1.724922in}}%
\pgfpathlineto{\pgfqpoint{2.081690in}{1.665811in}}%
\pgfpathclose%
\pgfusepath{fill}%
\end{pgfscope}%
\begin{pgfscope}%
\pgfpathrectangle{\pgfqpoint{1.072000in}{0.528000in}}{\pgfqpoint{3.696000in}{3.696000in}}%
\pgfusepath{clip}%
\pgfsetbuttcap%
\pgfsetroundjoin%
\definecolor{currentfill}{rgb}{0.304174,0.406945,0.845263}%
\pgfsetfillcolor{currentfill}%
\pgfsetlinewidth{0.000000pt}%
\definecolor{currentstroke}{rgb}{0.000000,0.000000,0.000000}%
\pgfsetstrokecolor{currentstroke}%
\pgfsetdash{}{0pt}%
\pgfpathmoveto{\pgfqpoint{3.847091in}{1.802944in}}%
\pgfpathlineto{\pgfqpoint{3.891325in}{1.701610in}}%
\pgfpathlineto{\pgfqpoint{3.915014in}{1.700220in}}%
\pgfpathlineto{\pgfqpoint{3.870520in}{1.785664in}}%
\pgfpathlineto{\pgfqpoint{3.847091in}{1.802944in}}%
\pgfpathclose%
\pgfusepath{fill}%
\end{pgfscope}%
\begin{pgfscope}%
\pgfpathrectangle{\pgfqpoint{1.072000in}{0.528000in}}{\pgfqpoint{3.696000in}{3.696000in}}%
\pgfusepath{clip}%
\pgfsetbuttcap%
\pgfsetroundjoin%
\definecolor{currentfill}{rgb}{0.822420,0.856898,0.910795}%
\pgfsetfillcolor{currentfill}%
\pgfsetlinewidth{0.000000pt}%
\definecolor{currentstroke}{rgb}{0.000000,0.000000,0.000000}%
\pgfsetstrokecolor{currentstroke}%
\pgfsetdash{}{0pt}%
\pgfpathmoveto{\pgfqpoint{4.239978in}{2.370567in}}%
\pgfpathlineto{\pgfqpoint{4.291926in}{2.492173in}}%
\pgfpathlineto{\pgfqpoint{4.320189in}{2.634012in}}%
\pgfpathlineto{\pgfqpoint{4.268087in}{2.513209in}}%
\pgfpathlineto{\pgfqpoint{4.239978in}{2.370567in}}%
\pgfpathclose%
\pgfusepath{fill}%
\end{pgfscope}%
\begin{pgfscope}%
\pgfpathrectangle{\pgfqpoint{1.072000in}{0.528000in}}{\pgfqpoint{3.696000in}{3.696000in}}%
\pgfusepath{clip}%
\pgfsetbuttcap%
\pgfsetroundjoin%
\definecolor{currentfill}{rgb}{0.964911,0.640159,0.519806}%
\pgfsetfillcolor{currentfill}%
\pgfsetlinewidth{0.000000pt}%
\definecolor{currentstroke}{rgb}{0.000000,0.000000,0.000000}%
\pgfsetstrokecolor{currentstroke}%
\pgfsetdash{}{0pt}%
\pgfpathmoveto{\pgfqpoint{2.754116in}{2.992029in}}%
\pgfpathlineto{\pgfqpoint{2.798760in}{3.108808in}}%
\pgfpathlineto{\pgfqpoint{2.825824in}{3.031903in}}%
\pgfpathlineto{\pgfqpoint{2.781276in}{2.913519in}}%
\pgfpathlineto{\pgfqpoint{2.754116in}{2.992029in}}%
\pgfpathclose%
\pgfusepath{fill}%
\end{pgfscope}%
\begin{pgfscope}%
\pgfpathrectangle{\pgfqpoint{1.072000in}{0.528000in}}{\pgfqpoint{3.696000in}{3.696000in}}%
\pgfusepath{clip}%
\pgfsetbuttcap%
\pgfsetroundjoin%
\definecolor{currentfill}{rgb}{0.441123,0.576532,0.954545}%
\pgfsetfillcolor{currentfill}%
\pgfsetlinewidth{0.000000pt}%
\definecolor{currentstroke}{rgb}{0.000000,0.000000,0.000000}%
\pgfsetstrokecolor{currentstroke}%
\pgfsetdash{}{0pt}%
\pgfpathmoveto{\pgfqpoint{3.783237in}{2.039723in}}%
\pgfpathlineto{\pgfqpoint{3.826683in}{1.900832in}}%
\pgfpathlineto{\pgfqpoint{3.849750in}{1.867961in}}%
\pgfpathlineto{\pgfqpoint{3.806125in}{1.990115in}}%
\pgfpathlineto{\pgfqpoint{3.783237in}{2.039723in}}%
\pgfpathclose%
\pgfusepath{fill}%
\end{pgfscope}%
\begin{pgfscope}%
\pgfpathrectangle{\pgfqpoint{1.072000in}{0.528000in}}{\pgfqpoint{3.696000in}{3.696000in}}%
\pgfusepath{clip}%
\pgfsetbuttcap%
\pgfsetroundjoin%
\definecolor{currentfill}{rgb}{0.248091,0.326013,0.777669}%
\pgfsetfillcolor{currentfill}%
\pgfsetlinewidth{0.000000pt}%
\definecolor{currentstroke}{rgb}{0.000000,0.000000,0.000000}%
\pgfsetstrokecolor{currentstroke}%
\pgfsetdash{}{0pt}%
\pgfpathmoveto{\pgfqpoint{1.935384in}{1.689245in}}%
\pgfpathlineto{\pgfqpoint{1.981142in}{1.629263in}}%
\pgfpathlineto{\pgfqpoint{2.009869in}{1.634431in}}%
\pgfpathlineto{\pgfqpoint{1.964817in}{1.674603in}}%
\pgfpathlineto{\pgfqpoint{1.935384in}{1.689245in}}%
\pgfpathclose%
\pgfusepath{fill}%
\end{pgfscope}%
\begin{pgfscope}%
\pgfpathrectangle{\pgfqpoint{1.072000in}{0.528000in}}{\pgfqpoint{3.696000in}{3.696000in}}%
\pgfusepath{clip}%
\pgfsetbuttcap%
\pgfsetroundjoin%
\definecolor{currentfill}{rgb}{0.871493,0.862309,0.857016}%
\pgfsetfillcolor{currentfill}%
\pgfsetlinewidth{0.000000pt}%
\definecolor{currentstroke}{rgb}{0.000000,0.000000,0.000000}%
\pgfsetstrokecolor{currentstroke}%
\pgfsetdash{}{0pt}%
\pgfpathmoveto{\pgfqpoint{2.389726in}{2.493235in}}%
\pgfpathlineto{\pgfqpoint{2.431691in}{2.661246in}}%
\pgfpathlineto{\pgfqpoint{2.458953in}{2.691100in}}%
\pgfpathlineto{\pgfqpoint{2.416898in}{2.522643in}}%
\pgfpathlineto{\pgfqpoint{2.389726in}{2.493235in}}%
\pgfpathclose%
\pgfusepath{fill}%
\end{pgfscope}%
\begin{pgfscope}%
\pgfpathrectangle{\pgfqpoint{1.072000in}{0.528000in}}{\pgfqpoint{3.696000in}{3.696000in}}%
\pgfusepath{clip}%
\pgfsetbuttcap%
\pgfsetroundjoin%
\definecolor{currentfill}{rgb}{0.289996,0.386836,0.828926}%
\pgfsetfillcolor{currentfill}%
\pgfsetlinewidth{0.000000pt}%
\definecolor{currentstroke}{rgb}{0.000000,0.000000,0.000000}%
\pgfsetstrokecolor{currentstroke}%
\pgfsetdash{}{0pt}%
\pgfpathmoveto{\pgfqpoint{1.888792in}{1.774830in}}%
\pgfpathlineto{\pgfqpoint{1.935384in}{1.689245in}}%
\pgfpathlineto{\pgfqpoint{1.964817in}{1.674603in}}%
\pgfpathlineto{\pgfqpoint{1.918932in}{1.742703in}}%
\pgfpathlineto{\pgfqpoint{1.888792in}{1.774830in}}%
\pgfpathclose%
\pgfusepath{fill}%
\end{pgfscope}%
\begin{pgfscope}%
\pgfpathrectangle{\pgfqpoint{1.072000in}{0.528000in}}{\pgfqpoint{3.696000in}{3.696000in}}%
\pgfusepath{clip}%
\pgfsetbuttcap%
\pgfsetroundjoin%
\definecolor{currentfill}{rgb}{0.962701,0.628218,0.507636}%
\pgfsetfillcolor{currentfill}%
\pgfsetlinewidth{0.000000pt}%
\definecolor{currentstroke}{rgb}{0.000000,0.000000,0.000000}%
\pgfsetstrokecolor{currentstroke}%
\pgfsetdash{}{0pt}%
\pgfpathmoveto{\pgfqpoint{3.177965in}{3.090395in}}%
\pgfpathlineto{\pgfqpoint{3.224013in}{3.100471in}}%
\pgfpathlineto{\pgfqpoint{3.248561in}{2.961716in}}%
\pgfpathlineto{\pgfqpoint{3.202762in}{2.951076in}}%
\pgfpathlineto{\pgfqpoint{3.177965in}{3.090395in}}%
\pgfpathclose%
\pgfusepath{fill}%
\end{pgfscope}%
\begin{pgfscope}%
\pgfpathrectangle{\pgfqpoint{1.072000in}{0.528000in}}{\pgfqpoint{3.696000in}{3.696000in}}%
\pgfusepath{clip}%
\pgfsetbuttcap%
\pgfsetroundjoin%
\definecolor{currentfill}{rgb}{0.248091,0.326013,0.777669}%
\pgfsetfillcolor{currentfill}%
\pgfsetlinewidth{0.000000pt}%
\definecolor{currentstroke}{rgb}{0.000000,0.000000,0.000000}%
\pgfsetstrokecolor{currentstroke}%
\pgfsetdash{}{0pt}%
\pgfpathmoveto{\pgfqpoint{2.054221in}{1.621266in}}%
\pgfpathlineto{\pgfqpoint{2.098035in}{1.632950in}}%
\pgfpathlineto{\pgfqpoint{2.124939in}{1.699900in}}%
\pgfpathlineto{\pgfqpoint{2.081690in}{1.665811in}}%
\pgfpathlineto{\pgfqpoint{2.054221in}{1.621266in}}%
\pgfpathclose%
\pgfusepath{fill}%
\end{pgfscope}%
\begin{pgfscope}%
\pgfpathrectangle{\pgfqpoint{1.072000in}{0.528000in}}{\pgfqpoint{3.696000in}{3.696000in}}%
\pgfusepath{clip}%
\pgfsetbuttcap%
\pgfsetroundjoin%
\definecolor{currentfill}{rgb}{0.229806,0.298718,0.753683}%
\pgfsetfillcolor{currentfill}%
\pgfsetlinewidth{0.000000pt}%
\definecolor{currentstroke}{rgb}{0.000000,0.000000,0.000000}%
\pgfsetstrokecolor{currentstroke}%
\pgfsetdash{}{0pt}%
\pgfpathmoveto{\pgfqpoint{1.981142in}{1.629263in}}%
\pgfpathlineto{\pgfqpoint{2.026173in}{1.594615in}}%
\pgfpathlineto{\pgfqpoint{2.054221in}{1.621266in}}%
\pgfpathlineto{\pgfqpoint{2.009869in}{1.634431in}}%
\pgfpathlineto{\pgfqpoint{1.981142in}{1.629263in}}%
\pgfpathclose%
\pgfusepath{fill}%
\end{pgfscope}%
\begin{pgfscope}%
\pgfpathrectangle{\pgfqpoint{1.072000in}{0.528000in}}{\pgfqpoint{3.696000in}{3.696000in}}%
\pgfusepath{clip}%
\pgfsetbuttcap%
\pgfsetroundjoin%
\definecolor{currentfill}{rgb}{0.743754,0.825125,0.965798}%
\pgfsetfillcolor{currentfill}%
\pgfsetlinewidth{0.000000pt}%
\definecolor{currentstroke}{rgb}{0.000000,0.000000,0.000000}%
\pgfsetstrokecolor{currentstroke}%
\pgfsetdash{}{0pt}%
\pgfpathmoveto{\pgfqpoint{3.654548in}{2.488283in}}%
\pgfpathlineto{\pgfqpoint{3.698316in}{2.344597in}}%
\pgfpathlineto{\pgfqpoint{3.720562in}{2.249428in}}%
\pgfpathlineto{\pgfqpoint{3.676855in}{2.380675in}}%
\pgfpathlineto{\pgfqpoint{3.654548in}{2.488283in}}%
\pgfpathclose%
\pgfusepath{fill}%
\end{pgfscope}%
\begin{pgfscope}%
\pgfpathrectangle{\pgfqpoint{1.072000in}{0.528000in}}{\pgfqpoint{3.696000in}{3.696000in}}%
\pgfusepath{clip}%
\pgfsetbuttcap%
\pgfsetroundjoin%
\definecolor{currentfill}{rgb}{0.603162,0.731527,0.999565}%
\pgfsetfillcolor{currentfill}%
\pgfsetlinewidth{0.000000pt}%
\definecolor{currentstroke}{rgb}{0.000000,0.000000,0.000000}%
\pgfsetstrokecolor{currentstroke}%
\pgfsetdash{}{0pt}%
\pgfpathmoveto{\pgfqpoint{3.719345in}{2.282267in}}%
\pgfpathlineto{\pgfqpoint{3.762732in}{2.130547in}}%
\pgfpathlineto{\pgfqpoint{3.785296in}{2.064967in}}%
\pgfpathlineto{\pgfqpoint{3.741847in}{2.201354in}}%
\pgfpathlineto{\pgfqpoint{3.719345in}{2.282267in}}%
\pgfpathclose%
\pgfusepath{fill}%
\end{pgfscope}%
\begin{pgfscope}%
\pgfpathrectangle{\pgfqpoint{1.072000in}{0.528000in}}{\pgfqpoint{3.696000in}{3.696000in}}%
\pgfusepath{clip}%
\pgfsetbuttcap%
\pgfsetroundjoin%
\definecolor{currentfill}{rgb}{0.891817,0.851973,0.829085}%
\pgfsetfillcolor{currentfill}%
\pgfsetlinewidth{0.000000pt}%
\definecolor{currentstroke}{rgb}{0.000000,0.000000,0.000000}%
\pgfsetstrokecolor{currentstroke}%
\pgfsetdash{}{0pt}%
\pgfpathmoveto{\pgfqpoint{3.543954in}{2.743270in}}%
\pgfpathlineto{\pgfqpoint{3.588606in}{2.631942in}}%
\pgfpathlineto{\pgfqpoint{3.610882in}{2.503442in}}%
\pgfpathlineto{\pgfqpoint{3.566455in}{2.608382in}}%
\pgfpathlineto{\pgfqpoint{3.543954in}{2.743270in}}%
\pgfpathclose%
\pgfusepath{fill}%
\end{pgfscope}%
\begin{pgfscope}%
\pgfpathrectangle{\pgfqpoint{1.072000in}{0.528000in}}{\pgfqpoint{3.696000in}{3.696000in}}%
\pgfusepath{clip}%
\pgfsetbuttcap%
\pgfsetroundjoin%
\definecolor{currentfill}{rgb}{0.238948,0.312365,0.765676}%
\pgfsetfillcolor{currentfill}%
\pgfsetlinewidth{0.000000pt}%
\definecolor{currentstroke}{rgb}{0.000000,0.000000,0.000000}%
\pgfsetstrokecolor{currentstroke}%
\pgfsetdash{}{0pt}%
\pgfpathmoveto{\pgfqpoint{3.982606in}{1.606315in}}%
\pgfpathlineto{\pgfqpoint{4.030107in}{1.617343in}}%
\pgfpathlineto{\pgfqpoint{4.054846in}{1.659135in}}%
\pgfpathlineto{\pgfqpoint{4.006969in}{1.635041in}}%
\pgfpathlineto{\pgfqpoint{3.982606in}{1.606315in}}%
\pgfpathclose%
\pgfusepath{fill}%
\end{pgfscope}%
\begin{pgfscope}%
\pgfpathrectangle{\pgfqpoint{1.072000in}{0.528000in}}{\pgfqpoint{3.696000in}{3.696000in}}%
\pgfusepath{clip}%
\pgfsetbuttcap%
\pgfsetroundjoin%
\definecolor{currentfill}{rgb}{0.906154,0.842091,0.806151}%
\pgfsetfillcolor{currentfill}%
\pgfsetlinewidth{0.000000pt}%
\definecolor{currentstroke}{rgb}{0.000000,0.000000,0.000000}%
\pgfsetstrokecolor{currentstroke}%
\pgfsetdash{}{0pt}%
\pgfpathmoveto{\pgfqpoint{4.268087in}{2.513209in}}%
\pgfpathlineto{\pgfqpoint{4.320189in}{2.634012in}}%
\pgfpathlineto{\pgfqpoint{4.348618in}{2.778357in}}%
\pgfpathlineto{\pgfqpoint{4.296502in}{2.662096in}}%
\pgfpathlineto{\pgfqpoint{4.268087in}{2.513209in}}%
\pgfpathclose%
\pgfusepath{fill}%
\end{pgfscope}%
\begin{pgfscope}%
\pgfpathrectangle{\pgfqpoint{1.072000in}{0.528000in}}{\pgfqpoint{3.696000in}{3.696000in}}%
\pgfusepath{clip}%
\pgfsetbuttcap%
\pgfsetroundjoin%
\definecolor{currentfill}{rgb}{0.388852,0.516298,0.921373}%
\pgfsetfillcolor{currentfill}%
\pgfsetlinewidth{0.000000pt}%
\definecolor{currentstroke}{rgb}{0.000000,0.000000,0.000000}%
\pgfsetstrokecolor{currentstroke}%
\pgfsetdash{}{0pt}%
\pgfpathmoveto{\pgfqpoint{3.803440in}{1.933659in}}%
\pgfpathlineto{\pgfqpoint{3.847091in}{1.802944in}}%
\pgfpathlineto{\pgfqpoint{3.870520in}{1.785664in}}%
\pgfpathlineto{\pgfqpoint{3.826683in}{1.900832in}}%
\pgfpathlineto{\pgfqpoint{3.803440in}{1.933659in}}%
\pgfpathclose%
\pgfusepath{fill}%
\end{pgfscope}%
\begin{pgfscope}%
\pgfpathrectangle{\pgfqpoint{1.072000in}{0.528000in}}{\pgfqpoint{3.696000in}{3.696000in}}%
\pgfusepath{clip}%
\pgfsetbuttcap%
\pgfsetroundjoin%
\definecolor{currentfill}{rgb}{0.494638,0.633022,0.978983}%
\pgfsetfillcolor{currentfill}%
\pgfsetlinewidth{0.000000pt}%
\definecolor{currentstroke}{rgb}{0.000000,0.000000,0.000000}%
\pgfsetstrokecolor{currentstroke}%
\pgfsetdash{}{0pt}%
\pgfpathmoveto{\pgfqpoint{1.775619in}{2.102408in}}%
\pgfpathlineto{\pgfqpoint{1.824341in}{1.959318in}}%
\pgfpathlineto{\pgfqpoint{1.855107in}{1.914183in}}%
\pgfpathlineto{\pgfqpoint{1.806953in}{2.047979in}}%
\pgfpathlineto{\pgfqpoint{1.775619in}{2.102408in}}%
\pgfpathclose%
\pgfusepath{fill}%
\end{pgfscope}%
\begin{pgfscope}%
\pgfpathrectangle{\pgfqpoint{1.072000in}{0.528000in}}{\pgfqpoint{3.696000in}{3.696000in}}%
\pgfusepath{clip}%
\pgfsetbuttcap%
\pgfsetroundjoin%
\definecolor{currentfill}{rgb}{0.358415,0.478426,0.896795}%
\pgfsetfillcolor{currentfill}%
\pgfsetlinewidth{0.000000pt}%
\definecolor{currentstroke}{rgb}{0.000000,0.000000,0.000000}%
\pgfsetstrokecolor{currentstroke}%
\pgfsetdash{}{0pt}%
\pgfpathmoveto{\pgfqpoint{1.841301in}{1.884963in}}%
\pgfpathlineto{\pgfqpoint{1.888792in}{1.774830in}}%
\pgfpathlineto{\pgfqpoint{1.918932in}{1.742703in}}%
\pgfpathlineto{\pgfqpoint{1.872119in}{1.838259in}}%
\pgfpathlineto{\pgfqpoint{1.841301in}{1.884963in}}%
\pgfpathclose%
\pgfusepath{fill}%
\end{pgfscope}%
\begin{pgfscope}%
\pgfpathrectangle{\pgfqpoint{1.072000in}{0.528000in}}{\pgfqpoint{3.696000in}{3.696000in}}%
\pgfusepath{clip}%
\pgfsetbuttcap%
\pgfsetroundjoin%
\definecolor{currentfill}{rgb}{0.229806,0.298718,0.753683}%
\pgfsetfillcolor{currentfill}%
\pgfsetlinewidth{0.000000pt}%
\definecolor{currentstroke}{rgb}{0.000000,0.000000,0.000000}%
\pgfsetstrokecolor{currentstroke}%
\pgfsetdash{}{0pt}%
\pgfpathmoveto{\pgfqpoint{3.912383in}{1.623731in}}%
\pgfpathlineto{\pgfqpoint{3.958290in}{1.582181in}}%
\pgfpathlineto{\pgfqpoint{3.982606in}{1.606315in}}%
\pgfpathlineto{\pgfqpoint{3.936414in}{1.634908in}}%
\pgfpathlineto{\pgfqpoint{3.912383in}{1.623731in}}%
\pgfpathclose%
\pgfusepath{fill}%
\end{pgfscope}%
\begin{pgfscope}%
\pgfpathrectangle{\pgfqpoint{1.072000in}{0.528000in}}{\pgfqpoint{3.696000in}{3.696000in}}%
\pgfusepath{clip}%
\pgfsetbuttcap%
\pgfsetroundjoin%
\definecolor{currentfill}{rgb}{0.266381,0.353304,0.801637}%
\pgfsetfillcolor{currentfill}%
\pgfsetlinewidth{0.000000pt}%
\definecolor{currentstroke}{rgb}{0.000000,0.000000,0.000000}%
\pgfsetstrokecolor{currentstroke}%
\pgfsetdash{}{0pt}%
\pgfpathmoveto{\pgfqpoint{3.867535in}{1.704100in}}%
\pgfpathlineto{\pgfqpoint{3.912383in}{1.623731in}}%
\pgfpathlineto{\pgfqpoint{3.936414in}{1.634908in}}%
\pgfpathlineto{\pgfqpoint{3.891325in}{1.701610in}}%
\pgfpathlineto{\pgfqpoint{3.867535in}{1.704100in}}%
\pgfpathclose%
\pgfusepath{fill}%
\end{pgfscope}%
\begin{pgfscope}%
\pgfpathrectangle{\pgfqpoint{1.072000in}{0.528000in}}{\pgfqpoint{3.696000in}{3.696000in}}%
\pgfusepath{clip}%
\pgfsetbuttcap%
\pgfsetroundjoin%
\definecolor{currentfill}{rgb}{0.859385,0.864431,0.872111}%
\pgfsetfillcolor{currentfill}%
\pgfsetlinewidth{0.000000pt}%
\definecolor{currentstroke}{rgb}{0.000000,0.000000,0.000000}%
\pgfsetstrokecolor{currentstroke}%
\pgfsetdash{}{0pt}%
\pgfpathmoveto{\pgfqpoint{2.362698in}{2.450072in}}%
\pgfpathlineto{\pgfqpoint{2.404577in}{2.616289in}}%
\pgfpathlineto{\pgfqpoint{2.431691in}{2.661246in}}%
\pgfpathlineto{\pgfqpoint{2.389726in}{2.493235in}}%
\pgfpathlineto{\pgfqpoint{2.362698in}{2.450072in}}%
\pgfpathclose%
\pgfusepath{fill}%
\end{pgfscope}%
\begin{pgfscope}%
\pgfpathrectangle{\pgfqpoint{1.072000in}{0.528000in}}{\pgfqpoint{3.696000in}{3.696000in}}%
\pgfusepath{clip}%
\pgfsetbuttcap%
\pgfsetroundjoin%
\definecolor{currentfill}{rgb}{0.229806,0.298718,0.753683}%
\pgfsetfillcolor{currentfill}%
\pgfsetlinewidth{0.000000pt}%
\definecolor{currentstroke}{rgb}{0.000000,0.000000,0.000000}%
\pgfsetstrokecolor{currentstroke}%
\pgfsetdash{}{0pt}%
\pgfpathmoveto{\pgfqpoint{2.026173in}{1.594615in}}%
\pgfpathlineto{\pgfqpoint{2.070615in}{1.583835in}}%
\pgfpathlineto{\pgfqpoint{2.098035in}{1.632950in}}%
\pgfpathlineto{\pgfqpoint{2.054221in}{1.621266in}}%
\pgfpathlineto{\pgfqpoint{2.026173in}{1.594615in}}%
\pgfpathclose%
\pgfusepath{fill}%
\end{pgfscope}%
\begin{pgfscope}%
\pgfpathrectangle{\pgfqpoint{1.072000in}{0.528000in}}{\pgfqpoint{3.696000in}{3.696000in}}%
\pgfusepath{clip}%
\pgfsetbuttcap%
\pgfsetroundjoin%
\definecolor{currentfill}{rgb}{0.969522,0.700833,0.587508}%
\pgfsetfillcolor{currentfill}%
\pgfsetlinewidth{0.000000pt}%
\definecolor{currentstroke}{rgb}{0.000000,0.000000,0.000000}%
\pgfsetstrokecolor{currentstroke}%
\pgfsetdash{}{0pt}%
\pgfpathmoveto{\pgfqpoint{2.584140in}{2.851524in}}%
\pgfpathlineto{\pgfqpoint{2.627675in}{3.000975in}}%
\pgfpathlineto{\pgfqpoint{2.655203in}{2.968906in}}%
\pgfpathlineto{\pgfqpoint{2.611667in}{2.819768in}}%
\pgfpathlineto{\pgfqpoint{2.584140in}{2.851524in}}%
\pgfpathclose%
\pgfusepath{fill}%
\end{pgfscope}%
\begin{pgfscope}%
\pgfpathrectangle{\pgfqpoint{1.072000in}{0.528000in}}{\pgfqpoint{3.696000in}{3.696000in}}%
\pgfusepath{clip}%
\pgfsetbuttcap%
\pgfsetroundjoin%
\definecolor{currentfill}{rgb}{0.939254,0.539581,0.423900}%
\pgfsetfillcolor{currentfill}%
\pgfsetlinewidth{0.000000pt}%
\definecolor{currentstroke}{rgb}{0.000000,0.000000,0.000000}%
\pgfsetstrokecolor{currentstroke}%
\pgfsetdash{}{0pt}%
\pgfpathmoveto{\pgfqpoint{2.942923in}{3.126942in}}%
\pgfpathlineto{\pgfqpoint{2.988593in}{3.194113in}}%
\pgfpathlineto{\pgfqpoint{3.014767in}{3.091015in}}%
\pgfpathlineto{\pgfqpoint{2.969258in}{3.019369in}}%
\pgfpathlineto{\pgfqpoint{2.942923in}{3.126942in}}%
\pgfpathclose%
\pgfusepath{fill}%
\end{pgfscope}%
\begin{pgfscope}%
\pgfpathrectangle{\pgfqpoint{1.072000in}{0.528000in}}{\pgfqpoint{3.696000in}{3.696000in}}%
\pgfusepath{clip}%
\pgfsetbuttcap%
\pgfsetroundjoin%
\definecolor{currentfill}{rgb}{0.956371,0.775144,0.686416}%
\pgfsetfillcolor{currentfill}%
\pgfsetlinewidth{0.000000pt}%
\definecolor{currentstroke}{rgb}{0.000000,0.000000,0.000000}%
\pgfsetstrokecolor{currentstroke}%
\pgfsetdash{}{0pt}%
\pgfpathmoveto{\pgfqpoint{2.486328in}{2.705797in}}%
\pgfpathlineto{\pgfqpoint{2.529086in}{2.869503in}}%
\pgfpathlineto{\pgfqpoint{2.556601in}{2.868224in}}%
\pgfpathlineto{\pgfqpoint{2.513778in}{2.705512in}}%
\pgfpathlineto{\pgfqpoint{2.486328in}{2.705797in}}%
\pgfpathclose%
\pgfusepath{fill}%
\end{pgfscope}%
\begin{pgfscope}%
\pgfpathrectangle{\pgfqpoint{1.072000in}{0.528000in}}{\pgfqpoint{3.696000in}{3.696000in}}%
\pgfusepath{clip}%
\pgfsetbuttcap%
\pgfsetroundjoin%
\definecolor{currentfill}{rgb}{0.959385,0.610306,0.489382}%
\pgfsetfillcolor{currentfill}%
\pgfsetlinewidth{0.000000pt}%
\definecolor{currentstroke}{rgb}{0.000000,0.000000,0.000000}%
\pgfsetstrokecolor{currentstroke}%
\pgfsetdash{}{0pt}%
\pgfpathmoveto{\pgfqpoint{3.224013in}{3.100471in}}%
\pgfpathlineto{\pgfqpoint{3.270097in}{3.095494in}}%
\pgfpathlineto{\pgfqpoint{3.294381in}{2.956929in}}%
\pgfpathlineto{\pgfqpoint{3.248561in}{2.961716in}}%
\pgfpathlineto{\pgfqpoint{3.224013in}{3.100471in}}%
\pgfpathclose%
\pgfusepath{fill}%
\end{pgfscope}%
\begin{pgfscope}%
\pgfpathrectangle{\pgfqpoint{1.072000in}{0.528000in}}{\pgfqpoint{3.696000in}{3.696000in}}%
\pgfusepath{clip}%
\pgfsetbuttcap%
\pgfsetroundjoin%
\definecolor{currentfill}{rgb}{0.939254,0.539581,0.423900}%
\pgfsetfillcolor{currentfill}%
\pgfsetlinewidth{0.000000pt}%
\definecolor{currentstroke}{rgb}{0.000000,0.000000,0.000000}%
\pgfsetstrokecolor{currentstroke}%
\pgfsetdash{}{0pt}%
\pgfpathmoveto{\pgfqpoint{3.060553in}{3.145761in}}%
\pgfpathlineto{\pgfqpoint{3.106553in}{3.183785in}}%
\pgfpathlineto{\pgfqpoint{3.132009in}{3.064920in}}%
\pgfpathlineto{\pgfqpoint{3.086204in}{3.023731in}}%
\pgfpathlineto{\pgfqpoint{3.060553in}{3.145761in}}%
\pgfpathclose%
\pgfusepath{fill}%
\end{pgfscope}%
\begin{pgfscope}%
\pgfpathrectangle{\pgfqpoint{1.072000in}{0.528000in}}{\pgfqpoint{3.696000in}{3.696000in}}%
\pgfusepath{clip}%
\pgfsetbuttcap%
\pgfsetroundjoin%
\definecolor{currentfill}{rgb}{0.940879,0.805596,0.735167}%
\pgfsetfillcolor{currentfill}%
\pgfsetlinewidth{0.000000pt}%
\definecolor{currentstroke}{rgb}{0.000000,0.000000,0.000000}%
\pgfsetstrokecolor{currentstroke}%
\pgfsetdash{}{0pt}%
\pgfpathmoveto{\pgfqpoint{3.498940in}{2.840998in}}%
\pgfpathlineto{\pgfqpoint{3.543954in}{2.743270in}}%
\pgfpathlineto{\pgfqpoint{3.566455in}{2.608382in}}%
\pgfpathlineto{\pgfqpoint{3.521715in}{2.702397in}}%
\pgfpathlineto{\pgfqpoint{3.498940in}{2.840998in}}%
\pgfpathclose%
\pgfusepath{fill}%
\end{pgfscope}%
\begin{pgfscope}%
\pgfpathrectangle{\pgfqpoint{1.072000in}{0.528000in}}{\pgfqpoint{3.696000in}{3.696000in}}%
\pgfusepath{clip}%
\pgfsetbuttcap%
\pgfsetroundjoin%
\definecolor{currentfill}{rgb}{0.565182,0.699438,0.996635}%
\pgfsetfillcolor{currentfill}%
\pgfsetlinewidth{0.000000pt}%
\definecolor{currentstroke}{rgb}{0.000000,0.000000,0.000000}%
\pgfsetstrokecolor{currentstroke}%
\pgfsetdash{}{0pt}%
\pgfpathmoveto{\pgfqpoint{3.739927in}{2.195240in}}%
\pgfpathlineto{\pgfqpoint{3.783237in}{2.039723in}}%
\pgfpathlineto{\pgfqpoint{3.806125in}{1.990115in}}%
\pgfpathlineto{\pgfqpoint{3.762732in}{2.130547in}}%
\pgfpathlineto{\pgfqpoint{3.739927in}{2.195240in}}%
\pgfpathclose%
\pgfusepath{fill}%
\end{pgfscope}%
\begin{pgfscope}%
\pgfpathrectangle{\pgfqpoint{1.072000in}{0.528000in}}{\pgfqpoint{3.696000in}{3.696000in}}%
\pgfusepath{clip}%
\pgfsetbuttcap%
\pgfsetroundjoin%
\definecolor{currentfill}{rgb}{0.229806,0.298718,0.753683}%
\pgfsetfillcolor{currentfill}%
\pgfsetlinewidth{0.000000pt}%
\definecolor{currentstroke}{rgb}{0.000000,0.000000,0.000000}%
\pgfsetstrokecolor{currentstroke}%
\pgfsetdash{}{0pt}%
\pgfpathmoveto{\pgfqpoint{3.958290in}{1.582181in}}%
\pgfpathlineto{\pgfqpoint{4.005480in}{1.581459in}}%
\pgfpathlineto{\pgfqpoint{4.030107in}{1.617343in}}%
\pgfpathlineto{\pgfqpoint{3.982606in}{1.606315in}}%
\pgfpathlineto{\pgfqpoint{3.958290in}{1.582181in}}%
\pgfpathclose%
\pgfusepath{fill}%
\end{pgfscope}%
\begin{pgfscope}%
\pgfpathrectangle{\pgfqpoint{1.072000in}{0.528000in}}{\pgfqpoint{3.696000in}{3.696000in}}%
\pgfusepath{clip}%
\pgfsetbuttcap%
\pgfsetroundjoin%
\definecolor{currentfill}{rgb}{0.333490,0.446265,0.874452}%
\pgfsetfillcolor{currentfill}%
\pgfsetlinewidth{0.000000pt}%
\definecolor{currentstroke}{rgb}{0.000000,0.000000,0.000000}%
\pgfsetstrokecolor{currentstroke}%
\pgfsetdash{}{0pt}%
\pgfpathmoveto{\pgfqpoint{3.823484in}{1.819236in}}%
\pgfpathlineto{\pgfqpoint{3.867535in}{1.704100in}}%
\pgfpathlineto{\pgfqpoint{3.891325in}{1.701610in}}%
\pgfpathlineto{\pgfqpoint{3.847091in}{1.802944in}}%
\pgfpathlineto{\pgfqpoint{3.823484in}{1.819236in}}%
\pgfpathclose%
\pgfusepath{fill}%
\end{pgfscope}%
\begin{pgfscope}%
\pgfpathrectangle{\pgfqpoint{1.072000in}{0.528000in}}{\pgfqpoint{3.696000in}{3.696000in}}%
\pgfusepath{clip}%
\pgfsetbuttcap%
\pgfsetroundjoin%
\definecolor{currentfill}{rgb}{0.343278,0.459354,0.884122}%
\pgfsetfillcolor{currentfill}%
\pgfsetlinewidth{0.000000pt}%
\definecolor{currentstroke}{rgb}{0.000000,0.000000,0.000000}%
\pgfsetstrokecolor{currentstroke}%
\pgfsetdash{}{0pt}%
\pgfpathmoveto{\pgfqpoint{2.124939in}{1.699900in}}%
\pgfpathlineto{\pgfqpoint{2.167905in}{1.755502in}}%
\pgfpathlineto{\pgfqpoint{2.194030in}{1.857457in}}%
\pgfpathlineto{\pgfqpoint{2.151468in}{1.780712in}}%
\pgfpathlineto{\pgfqpoint{2.124939in}{1.699900in}}%
\pgfpathclose%
\pgfusepath{fill}%
\end{pgfscope}%
\begin{pgfscope}%
\pgfpathrectangle{\pgfqpoint{1.072000in}{0.528000in}}{\pgfqpoint{3.696000in}{3.696000in}}%
\pgfusepath{clip}%
\pgfsetbuttcap%
\pgfsetroundjoin%
\definecolor{currentfill}{rgb}{0.839351,0.861167,0.894494}%
\pgfsetfillcolor{currentfill}%
\pgfsetlinewidth{0.000000pt}%
\definecolor{currentstroke}{rgb}{0.000000,0.000000,0.000000}%
\pgfsetstrokecolor{currentstroke}%
\pgfsetdash{}{0pt}%
\pgfpathmoveto{\pgfqpoint{2.335835in}{2.393678in}}%
\pgfpathlineto{\pgfqpoint{2.377640in}{2.556555in}}%
\pgfpathlineto{\pgfqpoint{2.404577in}{2.616289in}}%
\pgfpathlineto{\pgfqpoint{2.362698in}{2.450072in}}%
\pgfpathlineto{\pgfqpoint{2.335835in}{2.393678in}}%
\pgfpathclose%
\pgfusepath{fill}%
\end{pgfscope}%
\begin{pgfscope}%
\pgfpathrectangle{\pgfqpoint{1.072000in}{0.528000in}}{\pgfqpoint{3.696000in}{3.696000in}}%
\pgfusepath{clip}%
\pgfsetbuttcap%
\pgfsetroundjoin%
\definecolor{currentfill}{rgb}{0.409611,0.540759,0.935545}%
\pgfsetfillcolor{currentfill}%
\pgfsetlinewidth{0.000000pt}%
\definecolor{currentstroke}{rgb}{0.000000,0.000000,0.000000}%
\pgfsetstrokecolor{currentstroke}%
\pgfsetdash{}{0pt}%
\pgfpathmoveto{\pgfqpoint{2.151468in}{1.780712in}}%
\pgfpathlineto{\pgfqpoint{2.194030in}{1.857457in}}%
\pgfpathlineto{\pgfqpoint{2.219999in}{1.967307in}}%
\pgfpathlineto{\pgfqpoint{2.177758in}{1.871109in}}%
\pgfpathlineto{\pgfqpoint{2.151468in}{1.780712in}}%
\pgfpathclose%
\pgfusepath{fill}%
\end{pgfscope}%
\begin{pgfscope}%
\pgfpathrectangle{\pgfqpoint{1.072000in}{0.528000in}}{\pgfqpoint{3.696000in}{3.696000in}}%
\pgfusepath{clip}%
\pgfsetbuttcap%
\pgfsetroundjoin%
\definecolor{currentfill}{rgb}{0.289996,0.386836,0.828926}%
\pgfsetfillcolor{currentfill}%
\pgfsetlinewidth{0.000000pt}%
\definecolor{currentstroke}{rgb}{0.000000,0.000000,0.000000}%
\pgfsetstrokecolor{currentstroke}%
\pgfsetdash{}{0pt}%
\pgfpathmoveto{\pgfqpoint{2.098035in}{1.632950in}}%
\pgfpathlineto{\pgfqpoint{2.141487in}{1.666285in}}%
\pgfpathlineto{\pgfqpoint{2.167905in}{1.755502in}}%
\pgfpathlineto{\pgfqpoint{2.124939in}{1.699900in}}%
\pgfpathlineto{\pgfqpoint{2.098035in}{1.632950in}}%
\pgfpathclose%
\pgfusepath{fill}%
\end{pgfscope}%
\begin{pgfscope}%
\pgfpathrectangle{\pgfqpoint{1.072000in}{0.528000in}}{\pgfqpoint{3.696000in}{3.696000in}}%
\pgfusepath{clip}%
\pgfsetbuttcap%
\pgfsetroundjoin%
\definecolor{currentfill}{rgb}{0.959518,0.766973,0.674145}%
\pgfsetfillcolor{currentfill}%
\pgfsetlinewidth{0.000000pt}%
\definecolor{currentstroke}{rgb}{0.000000,0.000000,0.000000}%
\pgfsetstrokecolor{currentstroke}%
\pgfsetdash{}{0pt}%
\pgfpathmoveto{\pgfqpoint{4.296502in}{2.662096in}}%
\pgfpathlineto{\pgfqpoint{4.348618in}{2.778357in}}%
\pgfpathlineto{\pgfqpoint{4.377051in}{2.921088in}}%
\pgfpathlineto{\pgfqpoint{4.325078in}{2.813175in}}%
\pgfpathlineto{\pgfqpoint{4.296502in}{2.662096in}}%
\pgfpathclose%
\pgfusepath{fill}%
\end{pgfscope}%
\begin{pgfscope}%
\pgfpathrectangle{\pgfqpoint{1.072000in}{0.528000in}}{\pgfqpoint{3.696000in}{3.696000in}}%
\pgfusepath{clip}%
\pgfsetbuttcap%
\pgfsetroundjoin%
\definecolor{currentfill}{rgb}{0.243520,0.319189,0.771672}%
\pgfsetfillcolor{currentfill}%
\pgfsetlinewidth{0.000000pt}%
\definecolor{currentstroke}{rgb}{0.000000,0.000000,0.000000}%
\pgfsetstrokecolor{currentstroke}%
\pgfsetdash{}{0pt}%
\pgfpathmoveto{\pgfqpoint{1.951678in}{1.643717in}}%
\pgfpathlineto{\pgfqpoint{1.997420in}{1.588593in}}%
\pgfpathlineto{\pgfqpoint{2.026173in}{1.594615in}}%
\pgfpathlineto{\pgfqpoint{1.981142in}{1.629263in}}%
\pgfpathlineto{\pgfqpoint{1.951678in}{1.643717in}}%
\pgfpathclose%
\pgfusepath{fill}%
\end{pgfscope}%
\begin{pgfscope}%
\pgfpathrectangle{\pgfqpoint{1.072000in}{0.528000in}}{\pgfqpoint{3.696000in}{3.696000in}}%
\pgfusepath{clip}%
\pgfsetbuttcap%
\pgfsetroundjoin%
\definecolor{currentfill}{rgb}{0.839351,0.861167,0.894494}%
\pgfsetfillcolor{currentfill}%
\pgfsetlinewidth{0.000000pt}%
\definecolor{currentstroke}{rgb}{0.000000,0.000000,0.000000}%
\pgfsetstrokecolor{currentstroke}%
\pgfsetdash{}{0pt}%
\pgfpathmoveto{\pgfqpoint{3.610435in}{2.626543in}}%
\pgfpathlineto{\pgfqpoint{3.654548in}{2.488283in}}%
\pgfpathlineto{\pgfqpoint{3.676855in}{2.380675in}}%
\pgfpathlineto{\pgfqpoint{3.632895in}{2.509787in}}%
\pgfpathlineto{\pgfqpoint{3.610435in}{2.626543in}}%
\pgfpathclose%
\pgfusepath{fill}%
\end{pgfscope}%
\begin{pgfscope}%
\pgfpathrectangle{\pgfqpoint{1.072000in}{0.528000in}}{\pgfqpoint{3.696000in}{3.696000in}}%
\pgfusepath{clip}%
\pgfsetbuttcap%
\pgfsetroundjoin%
\definecolor{currentfill}{rgb}{0.959385,0.610306,0.489382}%
\pgfsetfillcolor{currentfill}%
\pgfsetlinewidth{0.000000pt}%
\definecolor{currentstroke}{rgb}{0.000000,0.000000,0.000000}%
\pgfsetstrokecolor{currentstroke}%
\pgfsetdash{}{0pt}%
\pgfpathmoveto{\pgfqpoint{3.270097in}{3.095494in}}%
\pgfpathlineto{\pgfqpoint{3.316160in}{3.075585in}}%
\pgfpathlineto{\pgfqpoint{3.340162in}{2.936675in}}%
\pgfpathlineto{\pgfqpoint{3.294381in}{2.956929in}}%
\pgfpathlineto{\pgfqpoint{3.270097in}{3.095494in}}%
\pgfpathclose%
\pgfusepath{fill}%
\end{pgfscope}%
\begin{pgfscope}%
\pgfpathrectangle{\pgfqpoint{1.072000in}{0.528000in}}{\pgfqpoint{3.696000in}{3.696000in}}%
\pgfusepath{clip}%
\pgfsetbuttcap%
\pgfsetroundjoin%
\definecolor{currentfill}{rgb}{0.280550,0.373423,0.818011}%
\pgfsetfillcolor{currentfill}%
\pgfsetlinewidth{0.000000pt}%
\definecolor{currentstroke}{rgb}{0.000000,0.000000,0.000000}%
\pgfsetstrokecolor{currentstroke}%
\pgfsetdash{}{0pt}%
\pgfpathmoveto{\pgfqpoint{1.905197in}{1.722147in}}%
\pgfpathlineto{\pgfqpoint{1.951678in}{1.643717in}}%
\pgfpathlineto{\pgfqpoint{1.981142in}{1.629263in}}%
\pgfpathlineto{\pgfqpoint{1.935384in}{1.689245in}}%
\pgfpathlineto{\pgfqpoint{1.905197in}{1.722147in}}%
\pgfpathclose%
\pgfusepath{fill}%
\end{pgfscope}%
\begin{pgfscope}%
\pgfpathrectangle{\pgfqpoint{1.072000in}{0.528000in}}{\pgfqpoint{3.696000in}{3.696000in}}%
\pgfusepath{clip}%
\pgfsetbuttcap%
\pgfsetroundjoin%
\definecolor{currentfill}{rgb}{0.483854,0.622050,0.974808}%
\pgfsetfillcolor{currentfill}%
\pgfsetlinewidth{0.000000pt}%
\definecolor{currentstroke}{rgb}{0.000000,0.000000,0.000000}%
\pgfsetstrokecolor{currentstroke}%
\pgfsetdash{}{0pt}%
\pgfpathmoveto{\pgfqpoint{2.177758in}{1.871109in}}%
\pgfpathlineto{\pgfqpoint{2.219999in}{1.967307in}}%
\pgfpathlineto{\pgfqpoint{2.245933in}{2.080271in}}%
\pgfpathlineto{\pgfqpoint{2.203928in}{1.966760in}}%
\pgfpathlineto{\pgfqpoint{2.177758in}{1.871109in}}%
\pgfpathclose%
\pgfusepath{fill}%
\end{pgfscope}%
\begin{pgfscope}%
\pgfpathrectangle{\pgfqpoint{1.072000in}{0.528000in}}{\pgfqpoint{3.696000in}{3.696000in}}%
\pgfusepath{clip}%
\pgfsetbuttcap%
\pgfsetroundjoin%
\definecolor{currentfill}{rgb}{0.451739,0.588181,0.960201}%
\pgfsetfillcolor{currentfill}%
\pgfsetlinewidth{0.000000pt}%
\definecolor{currentstroke}{rgb}{0.000000,0.000000,0.000000}%
\pgfsetstrokecolor{currentstroke}%
\pgfsetdash{}{0pt}%
\pgfpathmoveto{\pgfqpoint{1.792902in}{2.017187in}}%
\pgfpathlineto{\pgfqpoint{1.841301in}{1.884963in}}%
\pgfpathlineto{\pgfqpoint{1.872119in}{1.838259in}}%
\pgfpathlineto{\pgfqpoint{1.824341in}{1.959318in}}%
\pgfpathlineto{\pgfqpoint{1.792902in}{2.017187in}}%
\pgfpathclose%
\pgfusepath{fill}%
\end{pgfscope}%
\begin{pgfscope}%
\pgfpathrectangle{\pgfqpoint{1.072000in}{0.528000in}}{\pgfqpoint{3.696000in}{3.696000in}}%
\pgfusepath{clip}%
\pgfsetbuttcap%
\pgfsetroundjoin%
\definecolor{currentfill}{rgb}{0.962708,0.753557,0.655601}%
\pgfsetfillcolor{currentfill}%
\pgfsetlinewidth{0.000000pt}%
\definecolor{currentstroke}{rgb}{0.000000,0.000000,0.000000}%
\pgfsetstrokecolor{currentstroke}%
\pgfsetdash{}{0pt}%
\pgfpathmoveto{\pgfqpoint{3.453596in}{2.923462in}}%
\pgfpathlineto{\pgfqpoint{3.498940in}{2.840998in}}%
\pgfpathlineto{\pgfqpoint{3.521715in}{2.702397in}}%
\pgfpathlineto{\pgfqpoint{3.476674in}{2.783296in}}%
\pgfpathlineto{\pgfqpoint{3.453596in}{2.923462in}}%
\pgfpathclose%
\pgfusepath{fill}%
\end{pgfscope}%
\begin{pgfscope}%
\pgfpathrectangle{\pgfqpoint{1.072000in}{0.528000in}}{\pgfqpoint{3.696000in}{3.696000in}}%
\pgfusepath{clip}%
\pgfsetbuttcap%
\pgfsetroundjoin%
\definecolor{currentfill}{rgb}{0.728970,0.817464,0.973188}%
\pgfsetfillcolor{currentfill}%
\pgfsetlinewidth{0.000000pt}%
\definecolor{currentstroke}{rgb}{0.000000,0.000000,0.000000}%
\pgfsetstrokecolor{currentstroke}%
\pgfsetdash{}{0pt}%
\pgfpathmoveto{\pgfqpoint{3.675772in}{2.438053in}}%
\pgfpathlineto{\pgfqpoint{3.719345in}{2.282267in}}%
\pgfpathlineto{\pgfqpoint{3.741847in}{2.201354in}}%
\pgfpathlineto{\pgfqpoint{3.698316in}{2.344597in}}%
\pgfpathlineto{\pgfqpoint{3.675772in}{2.438053in}}%
\pgfpathclose%
\pgfusepath{fill}%
\end{pgfscope}%
\begin{pgfscope}%
\pgfpathrectangle{\pgfqpoint{1.072000in}{0.528000in}}{\pgfqpoint{3.696000in}{3.696000in}}%
\pgfusepath{clip}%
\pgfsetbuttcap%
\pgfsetroundjoin%
\definecolor{currentfill}{rgb}{0.959385,0.610306,0.489382}%
\pgfsetfillcolor{currentfill}%
\pgfsetlinewidth{0.000000pt}%
\definecolor{currentstroke}{rgb}{0.000000,0.000000,0.000000}%
\pgfsetstrokecolor{currentstroke}%
\pgfsetdash{}{0pt}%
\pgfpathmoveto{\pgfqpoint{2.655203in}{2.968906in}}%
\pgfpathlineto{\pgfqpoint{2.699393in}{3.100346in}}%
\pgfpathlineto{\pgfqpoint{2.726808in}{3.054642in}}%
\pgfpathlineto{\pgfqpoint{2.682669in}{2.921135in}}%
\pgfpathlineto{\pgfqpoint{2.655203in}{2.968906in}}%
\pgfpathclose%
\pgfusepath{fill}%
\end{pgfscope}%
\begin{pgfscope}%
\pgfpathrectangle{\pgfqpoint{1.072000in}{0.528000in}}{\pgfqpoint{3.696000in}{3.696000in}}%
\pgfusepath{clip}%
\pgfsetbuttcap%
\pgfsetroundjoin%
\definecolor{currentfill}{rgb}{0.809329,0.852974,0.922323}%
\pgfsetfillcolor{currentfill}%
\pgfsetlinewidth{0.000000pt}%
\definecolor{currentstroke}{rgb}{0.000000,0.000000,0.000000}%
\pgfsetstrokecolor{currentstroke}%
\pgfsetdash{}{0pt}%
\pgfpathmoveto{\pgfqpoint{2.309152in}{2.325033in}}%
\pgfpathlineto{\pgfqpoint{2.350903in}{2.482733in}}%
\pgfpathlineto{\pgfqpoint{2.377640in}{2.556555in}}%
\pgfpathlineto{\pgfqpoint{2.335835in}{2.393678in}}%
\pgfpathlineto{\pgfqpoint{2.309152in}{2.325033in}}%
\pgfpathclose%
\pgfusepath{fill}%
\end{pgfscope}%
\begin{pgfscope}%
\pgfpathrectangle{\pgfqpoint{1.072000in}{0.528000in}}{\pgfqpoint{3.696000in}{3.696000in}}%
\pgfusepath{clip}%
\pgfsetbuttcap%
\pgfsetroundjoin%
\definecolor{currentfill}{rgb}{0.926883,0.505422,0.394866}%
\pgfsetfillcolor{currentfill}%
\pgfsetlinewidth{0.000000pt}%
\definecolor{currentstroke}{rgb}{0.000000,0.000000,0.000000}%
\pgfsetstrokecolor{currentstroke}%
\pgfsetdash{}{0pt}%
\pgfpathmoveto{\pgfqpoint{2.870860in}{3.132675in}}%
\pgfpathlineto{\pgfqpoint{2.916323in}{3.212657in}}%
\pgfpathlineto{\pgfqpoint{2.942923in}{3.126942in}}%
\pgfpathlineto{\pgfqpoint{2.897600in}{3.040999in}}%
\pgfpathlineto{\pgfqpoint{2.870860in}{3.132675in}}%
\pgfpathclose%
\pgfusepath{fill}%
\end{pgfscope}%
\begin{pgfscope}%
\pgfpathrectangle{\pgfqpoint{1.072000in}{0.528000in}}{\pgfqpoint{3.696000in}{3.696000in}}%
\pgfusepath{clip}%
\pgfsetbuttcap%
\pgfsetroundjoin%
\definecolor{currentfill}{rgb}{0.565182,0.699438,0.996635}%
\pgfsetfillcolor{currentfill}%
\pgfsetlinewidth{0.000000pt}%
\definecolor{currentstroke}{rgb}{0.000000,0.000000,0.000000}%
\pgfsetstrokecolor{currentstroke}%
\pgfsetdash{}{0pt}%
\pgfpathmoveto{\pgfqpoint{2.203928in}{1.966760in}}%
\pgfpathlineto{\pgfqpoint{2.245933in}{2.080271in}}%
\pgfpathlineto{\pgfqpoint{2.271930in}{2.191888in}}%
\pgfpathlineto{\pgfqpoint{2.230082in}{2.063507in}}%
\pgfpathlineto{\pgfqpoint{2.203928in}{1.966760in}}%
\pgfpathclose%
\pgfusepath{fill}%
\end{pgfscope}%
\begin{pgfscope}%
\pgfpathrectangle{\pgfqpoint{1.072000in}{0.528000in}}{\pgfqpoint{3.696000in}{3.696000in}}%
\pgfusepath{clip}%
\pgfsetbuttcap%
\pgfsetroundjoin%
\definecolor{currentfill}{rgb}{0.229806,0.298718,0.753683}%
\pgfsetfillcolor{currentfill}%
\pgfsetlinewidth{0.000000pt}%
\definecolor{currentstroke}{rgb}{0.000000,0.000000,0.000000}%
\pgfsetstrokecolor{currentstroke}%
\pgfsetdash{}{0pt}%
\pgfpathmoveto{\pgfqpoint{1.997420in}{1.588593in}}%
\pgfpathlineto{\pgfqpoint{2.042538in}{1.555977in}}%
\pgfpathlineto{\pgfqpoint{2.070615in}{1.583835in}}%
\pgfpathlineto{\pgfqpoint{2.026173in}{1.594615in}}%
\pgfpathlineto{\pgfqpoint{1.997420in}{1.588593in}}%
\pgfpathclose%
\pgfusepath{fill}%
\end{pgfscope}%
\begin{pgfscope}%
\pgfpathrectangle{\pgfqpoint{1.072000in}{0.528000in}}{\pgfqpoint{3.696000in}{3.696000in}}%
\pgfusepath{clip}%
\pgfsetbuttcap%
\pgfsetroundjoin%
\definecolor{currentfill}{rgb}{0.252663,0.332837,0.783665}%
\pgfsetfillcolor{currentfill}%
\pgfsetlinewidth{0.000000pt}%
\definecolor{currentstroke}{rgb}{0.000000,0.000000,0.000000}%
\pgfsetstrokecolor{currentstroke}%
\pgfsetdash{}{0pt}%
\pgfpathmoveto{\pgfqpoint{2.070615in}{1.583835in}}%
\pgfpathlineto{\pgfqpoint{2.114625in}{1.594432in}}%
\pgfpathlineto{\pgfqpoint{2.141487in}{1.666285in}}%
\pgfpathlineto{\pgfqpoint{2.098035in}{1.632950in}}%
\pgfpathlineto{\pgfqpoint{2.070615in}{1.583835in}}%
\pgfpathclose%
\pgfusepath{fill}%
\end{pgfscope}%
\begin{pgfscope}%
\pgfpathrectangle{\pgfqpoint{1.072000in}{0.528000in}}{\pgfqpoint{3.696000in}{3.696000in}}%
\pgfusepath{clip}%
\pgfsetbuttcap%
\pgfsetroundjoin%
\definecolor{currentfill}{rgb}{0.238948,0.312365,0.765676}%
\pgfsetfillcolor{currentfill}%
\pgfsetlinewidth{0.000000pt}%
\definecolor{currentstroke}{rgb}{0.000000,0.000000,0.000000}%
\pgfsetstrokecolor{currentstroke}%
\pgfsetdash{}{0pt}%
\pgfpathmoveto{\pgfqpoint{3.888261in}{1.613350in}}%
\pgfpathlineto{\pgfqpoint{3.933958in}{1.560773in}}%
\pgfpathlineto{\pgfqpoint{3.958290in}{1.582181in}}%
\pgfpathlineto{\pgfqpoint{3.912383in}{1.623731in}}%
\pgfpathlineto{\pgfqpoint{3.888261in}{1.613350in}}%
\pgfpathclose%
\pgfusepath{fill}%
\end{pgfscope}%
\begin{pgfscope}%
\pgfpathrectangle{\pgfqpoint{1.072000in}{0.528000in}}{\pgfqpoint{3.696000in}{3.696000in}}%
\pgfusepath{clip}%
\pgfsetbuttcap%
\pgfsetroundjoin%
\definecolor{currentfill}{rgb}{0.505423,0.643995,0.983157}%
\pgfsetfillcolor{currentfill}%
\pgfsetlinewidth{0.000000pt}%
\definecolor{currentstroke}{rgb}{0.000000,0.000000,0.000000}%
\pgfsetstrokecolor{currentstroke}%
\pgfsetdash{}{0pt}%
\pgfpathmoveto{\pgfqpoint{3.760096in}{2.087003in}}%
\pgfpathlineto{\pgfqpoint{3.803440in}{1.933659in}}%
\pgfpathlineto{\pgfqpoint{3.826683in}{1.900832in}}%
\pgfpathlineto{\pgfqpoint{3.783237in}{2.039723in}}%
\pgfpathlineto{\pgfqpoint{3.760096in}{2.087003in}}%
\pgfpathclose%
\pgfusepath{fill}%
\end{pgfscope}%
\begin{pgfscope}%
\pgfpathrectangle{\pgfqpoint{1.072000in}{0.528000in}}{\pgfqpoint{3.696000in}{3.696000in}}%
\pgfusepath{clip}%
\pgfsetbuttcap%
\pgfsetroundjoin%
\definecolor{currentfill}{rgb}{0.962701,0.628218,0.507636}%
\pgfsetfillcolor{currentfill}%
\pgfsetlinewidth{0.000000pt}%
\definecolor{currentstroke}{rgb}{0.000000,0.000000,0.000000}%
\pgfsetstrokecolor{currentstroke}%
\pgfsetdash{}{0pt}%
\pgfpathmoveto{\pgfqpoint{3.316160in}{3.075585in}}%
\pgfpathlineto{\pgfqpoint{3.362141in}{3.040551in}}%
\pgfpathlineto{\pgfqpoint{3.385845in}{2.900923in}}%
\pgfpathlineto{\pgfqpoint{3.340162in}{2.936675in}}%
\pgfpathlineto{\pgfqpoint{3.316160in}{3.075585in}}%
\pgfpathclose%
\pgfusepath{fill}%
\end{pgfscope}%
\begin{pgfscope}%
\pgfpathrectangle{\pgfqpoint{1.072000in}{0.528000in}}{\pgfqpoint{3.696000in}{3.696000in}}%
\pgfusepath{clip}%
\pgfsetbuttcap%
\pgfsetroundjoin%
\definecolor{currentfill}{rgb}{0.640828,0.760752,0.997846}%
\pgfsetfillcolor{currentfill}%
\pgfsetlinewidth{0.000000pt}%
\definecolor{currentstroke}{rgb}{0.000000,0.000000,0.000000}%
\pgfsetstrokecolor{currentstroke}%
\pgfsetdash{}{0pt}%
\pgfpathmoveto{\pgfqpoint{2.230082in}{2.063507in}}%
\pgfpathlineto{\pgfqpoint{2.271930in}{2.191888in}}%
\pgfpathlineto{\pgfqpoint{2.298063in}{2.298225in}}%
\pgfpathlineto{\pgfqpoint{2.256302in}{2.157559in}}%
\pgfpathlineto{\pgfqpoint{2.230082in}{2.063507in}}%
\pgfpathclose%
\pgfusepath{fill}%
\end{pgfscope}%
\begin{pgfscope}%
\pgfpathrectangle{\pgfqpoint{1.072000in}{0.528000in}}{\pgfqpoint{3.696000in}{3.696000in}}%
\pgfusepath{clip}%
\pgfsetbuttcap%
\pgfsetroundjoin%
\definecolor{currentfill}{rgb}{0.969522,0.700833,0.587508}%
\pgfsetfillcolor{currentfill}%
\pgfsetlinewidth{0.000000pt}%
\definecolor{currentstroke}{rgb}{0.000000,0.000000,0.000000}%
\pgfsetstrokecolor{currentstroke}%
\pgfsetdash{}{0pt}%
\pgfpathmoveto{\pgfqpoint{3.407975in}{2.989973in}}%
\pgfpathlineto{\pgfqpoint{3.453596in}{2.923462in}}%
\pgfpathlineto{\pgfqpoint{3.476674in}{2.783296in}}%
\pgfpathlineto{\pgfqpoint{3.431368in}{2.849708in}}%
\pgfpathlineto{\pgfqpoint{3.407975in}{2.989973in}}%
\pgfpathclose%
\pgfusepath{fill}%
\end{pgfscope}%
\begin{pgfscope}%
\pgfpathrectangle{\pgfqpoint{1.072000in}{0.528000in}}{\pgfqpoint{3.696000in}{3.696000in}}%
\pgfusepath{clip}%
\pgfsetbuttcap%
\pgfsetroundjoin%
\definecolor{currentfill}{rgb}{0.763363,0.835092,0.955658}%
\pgfsetfillcolor{currentfill}%
\pgfsetlinewidth{0.000000pt}%
\definecolor{currentstroke}{rgb}{0.000000,0.000000,0.000000}%
\pgfsetstrokecolor{currentstroke}%
\pgfsetdash{}{0pt}%
\pgfpathmoveto{\pgfqpoint{2.282646in}{2.245632in}}%
\pgfpathlineto{\pgfqpoint{2.324379in}{2.396017in}}%
\pgfpathlineto{\pgfqpoint{2.350903in}{2.482733in}}%
\pgfpathlineto{\pgfqpoint{2.309152in}{2.325033in}}%
\pgfpathlineto{\pgfqpoint{2.282646in}{2.245632in}}%
\pgfpathclose%
\pgfusepath{fill}%
\end{pgfscope}%
\begin{pgfscope}%
\pgfpathrectangle{\pgfqpoint{1.072000in}{0.528000in}}{\pgfqpoint{3.696000in}{3.696000in}}%
\pgfusepath{clip}%
\pgfsetbuttcap%
\pgfsetroundjoin%
\definecolor{currentfill}{rgb}{0.338377,0.452819,0.879317}%
\pgfsetfillcolor{currentfill}%
\pgfsetlinewidth{0.000000pt}%
\definecolor{currentstroke}{rgb}{0.000000,0.000000,0.000000}%
\pgfsetstrokecolor{currentstroke}%
\pgfsetdash{}{0pt}%
\pgfpathmoveto{\pgfqpoint{1.857898in}{1.823532in}}%
\pgfpathlineto{\pgfqpoint{1.905197in}{1.722147in}}%
\pgfpathlineto{\pgfqpoint{1.935384in}{1.689245in}}%
\pgfpathlineto{\pgfqpoint{1.888792in}{1.774830in}}%
\pgfpathlineto{\pgfqpoint{1.857898in}{1.823532in}}%
\pgfpathclose%
\pgfusepath{fill}%
\end{pgfscope}%
\begin{pgfscope}%
\pgfpathrectangle{\pgfqpoint{1.072000in}{0.528000in}}{\pgfqpoint{3.696000in}{3.696000in}}%
\pgfusepath{clip}%
\pgfsetbuttcap%
\pgfsetroundjoin%
\definecolor{currentfill}{rgb}{0.708720,0.805721,0.981117}%
\pgfsetfillcolor{currentfill}%
\pgfsetlinewidth{0.000000pt}%
\definecolor{currentstroke}{rgb}{0.000000,0.000000,0.000000}%
\pgfsetstrokecolor{currentstroke}%
\pgfsetdash{}{0pt}%
\pgfpathmoveto{\pgfqpoint{2.256302in}{2.157559in}}%
\pgfpathlineto{\pgfqpoint{2.298063in}{2.298225in}}%
\pgfpathlineto{\pgfqpoint{2.324379in}{2.396017in}}%
\pgfpathlineto{\pgfqpoint{2.282646in}{2.245632in}}%
\pgfpathlineto{\pgfqpoint{2.256302in}{2.157559in}}%
\pgfpathclose%
\pgfusepath{fill}%
\end{pgfscope}%
\begin{pgfscope}%
\pgfpathrectangle{\pgfqpoint{1.072000in}{0.528000in}}{\pgfqpoint{3.696000in}{3.696000in}}%
\pgfusepath{clip}%
\pgfsetbuttcap%
\pgfsetroundjoin%
\definecolor{currentfill}{rgb}{0.967317,0.657471,0.538160}%
\pgfsetfillcolor{currentfill}%
\pgfsetlinewidth{0.000000pt}%
\definecolor{currentstroke}{rgb}{0.000000,0.000000,0.000000}%
\pgfsetstrokecolor{currentstroke}%
\pgfsetdash{}{0pt}%
\pgfpathmoveto{\pgfqpoint{3.362141in}{3.040551in}}%
\pgfpathlineto{\pgfqpoint{3.407975in}{2.989973in}}%
\pgfpathlineto{\pgfqpoint{3.431368in}{2.849708in}}%
\pgfpathlineto{\pgfqpoint{3.385845in}{2.900923in}}%
\pgfpathlineto{\pgfqpoint{3.362141in}{3.040551in}}%
\pgfpathclose%
\pgfusepath{fill}%
\end{pgfscope}%
\begin{pgfscope}%
\pgfpathrectangle{\pgfqpoint{1.072000in}{0.528000in}}{\pgfqpoint{3.696000in}{3.696000in}}%
\pgfusepath{clip}%
\pgfsetbuttcap%
\pgfsetroundjoin%
\definecolor{currentfill}{rgb}{0.280550,0.373423,0.818011}%
\pgfsetfillcolor{currentfill}%
\pgfsetlinewidth{0.000000pt}%
\definecolor{currentstroke}{rgb}{0.000000,0.000000,0.000000}%
\pgfsetstrokecolor{currentstroke}%
\pgfsetdash{}{0pt}%
\pgfpathmoveto{\pgfqpoint{3.843579in}{1.705246in}}%
\pgfpathlineto{\pgfqpoint{3.888261in}{1.613350in}}%
\pgfpathlineto{\pgfqpoint{3.912383in}{1.623731in}}%
\pgfpathlineto{\pgfqpoint{3.867535in}{1.704100in}}%
\pgfpathlineto{\pgfqpoint{3.843579in}{1.705246in}}%
\pgfpathclose%
\pgfusepath{fill}%
\end{pgfscope}%
\begin{pgfscope}%
\pgfpathrectangle{\pgfqpoint{1.072000in}{0.528000in}}{\pgfqpoint{3.696000in}{3.696000in}}%
\pgfusepath{clip}%
\pgfsetbuttcap%
\pgfsetroundjoin%
\definecolor{currentfill}{rgb}{0.960581,0.762501,0.667964}%
\pgfsetfillcolor{currentfill}%
\pgfsetlinewidth{0.000000pt}%
\definecolor{currentstroke}{rgb}{0.000000,0.000000,0.000000}%
\pgfsetstrokecolor{currentstroke}%
\pgfsetdash{}{0pt}%
\pgfpathmoveto{\pgfqpoint{2.458953in}{2.691100in}}%
\pgfpathlineto{\pgfqpoint{2.501634in}{2.855227in}}%
\pgfpathlineto{\pgfqpoint{2.529086in}{2.869503in}}%
\pgfpathlineto{\pgfqpoint{2.486328in}{2.705797in}}%
\pgfpathlineto{\pgfqpoint{2.458953in}{2.691100in}}%
\pgfpathclose%
\pgfusepath{fill}%
\end{pgfscope}%
\begin{pgfscope}%
\pgfpathrectangle{\pgfqpoint{1.072000in}{0.528000in}}{\pgfqpoint{3.696000in}{3.696000in}}%
\pgfusepath{clip}%
\pgfsetbuttcap%
\pgfsetroundjoin%
\definecolor{currentfill}{rgb}{0.939254,0.539581,0.423900}%
\pgfsetfillcolor{currentfill}%
\pgfsetlinewidth{0.000000pt}%
\definecolor{currentstroke}{rgb}{0.000000,0.000000,0.000000}%
\pgfsetstrokecolor{currentstroke}%
\pgfsetdash{}{0pt}%
\pgfpathmoveto{\pgfqpoint{2.726808in}{3.054642in}}%
\pgfpathlineto{\pgfqpoint{2.771545in}{3.166964in}}%
\pgfpathlineto{\pgfqpoint{2.798760in}{3.108808in}}%
\pgfpathlineto{\pgfqpoint{2.754116in}{2.992029in}}%
\pgfpathlineto{\pgfqpoint{2.726808in}{3.054642in}}%
\pgfpathclose%
\pgfusepath{fill}%
\end{pgfscope}%
\begin{pgfscope}%
\pgfpathrectangle{\pgfqpoint{1.072000in}{0.528000in}}{\pgfqpoint{3.696000in}{3.696000in}}%
\pgfusepath{clip}%
\pgfsetbuttcap%
\pgfsetroundjoin%
\definecolor{currentfill}{rgb}{0.926883,0.505422,0.394866}%
\pgfsetfillcolor{currentfill}%
\pgfsetlinewidth{0.000000pt}%
\definecolor{currentstroke}{rgb}{0.000000,0.000000,0.000000}%
\pgfsetstrokecolor{currentstroke}%
\pgfsetdash{}{0pt}%
\pgfpathmoveto{\pgfqpoint{2.798760in}{3.108808in}}%
\pgfpathlineto{\pgfqpoint{2.843917in}{3.203623in}}%
\pgfpathlineto{\pgfqpoint{2.870860in}{3.132675in}}%
\pgfpathlineto{\pgfqpoint{2.825824in}{3.031903in}}%
\pgfpathlineto{\pgfqpoint{2.798760in}{3.108808in}}%
\pgfpathclose%
\pgfusepath{fill}%
\end{pgfscope}%
\begin{pgfscope}%
\pgfpathrectangle{\pgfqpoint{1.072000in}{0.528000in}}{\pgfqpoint{3.696000in}{3.696000in}}%
\pgfusepath{clip}%
\pgfsetbuttcap%
\pgfsetroundjoin%
\definecolor{currentfill}{rgb}{0.229806,0.298718,0.753683}%
\pgfsetfillcolor{currentfill}%
\pgfsetlinewidth{0.000000pt}%
\definecolor{currentstroke}{rgb}{0.000000,0.000000,0.000000}%
\pgfsetstrokecolor{currentstroke}%
\pgfsetdash{}{0pt}%
\pgfpathmoveto{\pgfqpoint{3.933958in}{1.560773in}}%
\pgfpathlineto{\pgfqpoint{3.980908in}{1.549962in}}%
\pgfpathlineto{\pgfqpoint{4.005480in}{1.581459in}}%
\pgfpathlineto{\pgfqpoint{3.958290in}{1.582181in}}%
\pgfpathlineto{\pgfqpoint{3.933958in}{1.560773in}}%
\pgfpathclose%
\pgfusepath{fill}%
\end{pgfscope}%
\begin{pgfscope}%
\pgfpathrectangle{\pgfqpoint{1.072000in}{0.528000in}}{\pgfqpoint{3.696000in}{3.696000in}}%
\pgfusepath{clip}%
\pgfsetbuttcap%
\pgfsetroundjoin%
\definecolor{currentfill}{rgb}{0.921406,0.491420,0.383408}%
\pgfsetfillcolor{currentfill}%
\pgfsetlinewidth{0.000000pt}%
\definecolor{currentstroke}{rgb}{0.000000,0.000000,0.000000}%
\pgfsetstrokecolor{currentstroke}%
\pgfsetdash{}{0pt}%
\pgfpathmoveto{\pgfqpoint{3.106553in}{3.183785in}}%
\pgfpathlineto{\pgfqpoint{3.152708in}{3.206045in}}%
\pgfpathlineto{\pgfqpoint{3.177965in}{3.090395in}}%
\pgfpathlineto{\pgfqpoint{3.132009in}{3.064920in}}%
\pgfpathlineto{\pgfqpoint{3.106553in}{3.183785in}}%
\pgfpathclose%
\pgfusepath{fill}%
\end{pgfscope}%
\begin{pgfscope}%
\pgfpathrectangle{\pgfqpoint{1.072000in}{0.528000in}}{\pgfqpoint{3.696000in}{3.696000in}}%
\pgfusepath{clip}%
\pgfsetbuttcap%
\pgfsetroundjoin%
\definecolor{currentfill}{rgb}{0.430507,0.564883,0.948889}%
\pgfsetfillcolor{currentfill}%
\pgfsetlinewidth{0.000000pt}%
\definecolor{currentstroke}{rgb}{0.000000,0.000000,0.000000}%
\pgfsetstrokecolor{currentstroke}%
\pgfsetdash{}{0pt}%
\pgfpathmoveto{\pgfqpoint{3.779944in}{1.963234in}}%
\pgfpathlineto{\pgfqpoint{3.823484in}{1.819236in}}%
\pgfpathlineto{\pgfqpoint{3.847091in}{1.802944in}}%
\pgfpathlineto{\pgfqpoint{3.803440in}{1.933659in}}%
\pgfpathlineto{\pgfqpoint{3.779944in}{1.963234in}}%
\pgfpathclose%
\pgfusepath{fill}%
\end{pgfscope}%
\begin{pgfscope}%
\pgfpathrectangle{\pgfqpoint{1.072000in}{0.528000in}}{\pgfqpoint{3.696000in}{3.696000in}}%
\pgfusepath{clip}%
\pgfsetbuttcap%
\pgfsetroundjoin%
\definecolor{currentfill}{rgb}{0.229806,0.298718,0.753683}%
\pgfsetfillcolor{currentfill}%
\pgfsetlinewidth{0.000000pt}%
\definecolor{currentstroke}{rgb}{0.000000,0.000000,0.000000}%
\pgfsetstrokecolor{currentstroke}%
\pgfsetdash{}{0pt}%
\pgfpathmoveto{\pgfqpoint{2.042538in}{1.555977in}}%
\pgfpathlineto{\pgfqpoint{2.087167in}{1.544062in}}%
\pgfpathlineto{\pgfqpoint{2.114625in}{1.594432in}}%
\pgfpathlineto{\pgfqpoint{2.070615in}{1.583835in}}%
\pgfpathlineto{\pgfqpoint{2.042538in}{1.555977in}}%
\pgfpathclose%
\pgfusepath{fill}%
\end{pgfscope}%
\begin{pgfscope}%
\pgfpathrectangle{\pgfqpoint{1.072000in}{0.528000in}}{\pgfqpoint{3.696000in}{3.696000in}}%
\pgfusepath{clip}%
\pgfsetbuttcap%
\pgfsetroundjoin%
\definecolor{currentfill}{rgb}{0.967317,0.657471,0.538160}%
\pgfsetfillcolor{currentfill}%
\pgfsetlinewidth{0.000000pt}%
\definecolor{currentstroke}{rgb}{0.000000,0.000000,0.000000}%
\pgfsetstrokecolor{currentstroke}%
\pgfsetdash{}{0pt}%
\pgfpathmoveto{\pgfqpoint{2.556601in}{2.868224in}}%
\pgfpathlineto{\pgfqpoint{2.600121in}{3.017262in}}%
\pgfpathlineto{\pgfqpoint{2.627675in}{3.000975in}}%
\pgfpathlineto{\pgfqpoint{2.584140in}{2.851524in}}%
\pgfpathlineto{\pgfqpoint{2.556601in}{2.868224in}}%
\pgfpathclose%
\pgfusepath{fill}%
\end{pgfscope}%
\begin{pgfscope}%
\pgfpathrectangle{\pgfqpoint{1.072000in}{0.528000in}}{\pgfqpoint{3.696000in}{3.696000in}}%
\pgfusepath{clip}%
\pgfsetbuttcap%
\pgfsetroundjoin%
\definecolor{currentfill}{rgb}{0.905783,0.455186,0.355336}%
\pgfsetfillcolor{currentfill}%
\pgfsetlinewidth{0.000000pt}%
\definecolor{currentstroke}{rgb}{0.000000,0.000000,0.000000}%
\pgfsetstrokecolor{currentstroke}%
\pgfsetdash{}{0pt}%
\pgfpathmoveto{\pgfqpoint{2.988593in}{3.194113in}}%
\pgfpathlineto{\pgfqpoint{3.034540in}{3.242475in}}%
\pgfpathlineto{\pgfqpoint{3.060553in}{3.145761in}}%
\pgfpathlineto{\pgfqpoint{3.014767in}{3.091015in}}%
\pgfpathlineto{\pgfqpoint{2.988593in}{3.194113in}}%
\pgfpathclose%
\pgfusepath{fill}%
\end{pgfscope}%
\begin{pgfscope}%
\pgfpathrectangle{\pgfqpoint{1.072000in}{0.528000in}}{\pgfqpoint{3.696000in}{3.696000in}}%
\pgfusepath{clip}%
\pgfsetbuttcap%
\pgfsetroundjoin%
\definecolor{currentfill}{rgb}{0.919376,0.831273,0.782874}%
\pgfsetfillcolor{currentfill}%
\pgfsetlinewidth{0.000000pt}%
\definecolor{currentstroke}{rgb}{0.000000,0.000000,0.000000}%
\pgfsetstrokecolor{currentstroke}%
\pgfsetdash{}{0pt}%
\pgfpathmoveto{\pgfqpoint{3.565916in}{2.754446in}}%
\pgfpathlineto{\pgfqpoint{3.610435in}{2.626543in}}%
\pgfpathlineto{\pgfqpoint{3.632895in}{2.509787in}}%
\pgfpathlineto{\pgfqpoint{3.588606in}{2.631942in}}%
\pgfpathlineto{\pgfqpoint{3.565916in}{2.754446in}}%
\pgfpathclose%
\pgfusepath{fill}%
\end{pgfscope}%
\begin{pgfscope}%
\pgfpathrectangle{\pgfqpoint{1.072000in}{0.528000in}}{\pgfqpoint{3.696000in}{3.696000in}}%
\pgfusepath{clip}%
\pgfsetbuttcap%
\pgfsetroundjoin%
\definecolor{currentfill}{rgb}{0.698454,0.799450,0.984577}%
\pgfsetfillcolor{currentfill}%
\pgfsetlinewidth{0.000000pt}%
\definecolor{currentstroke}{rgb}{0.000000,0.000000,0.000000}%
\pgfsetstrokecolor{currentstroke}%
\pgfsetdash{}{0pt}%
\pgfpathmoveto{\pgfqpoint{3.696524in}{2.359619in}}%
\pgfpathlineto{\pgfqpoint{3.739927in}{2.195240in}}%
\pgfpathlineto{\pgfqpoint{3.762732in}{2.130547in}}%
\pgfpathlineto{\pgfqpoint{3.719345in}{2.282267in}}%
\pgfpathlineto{\pgfqpoint{3.696524in}{2.359619in}}%
\pgfpathclose%
\pgfusepath{fill}%
\end{pgfscope}%
\begin{pgfscope}%
\pgfpathrectangle{\pgfqpoint{1.072000in}{0.528000in}}{\pgfqpoint{3.696000in}{3.696000in}}%
\pgfusepath{clip}%
\pgfsetbuttcap%
\pgfsetroundjoin%
\definecolor{currentfill}{rgb}{0.425199,0.559058,0.946061}%
\pgfsetfillcolor{currentfill}%
\pgfsetlinewidth{0.000000pt}%
\definecolor{currentstroke}{rgb}{0.000000,0.000000,0.000000}%
\pgfsetstrokecolor{currentstroke}%
\pgfsetdash{}{0pt}%
\pgfpathmoveto{\pgfqpoint{1.809749in}{1.946260in}}%
\pgfpathlineto{\pgfqpoint{1.857898in}{1.823532in}}%
\pgfpathlineto{\pgfqpoint{1.888792in}{1.774830in}}%
\pgfpathlineto{\pgfqpoint{1.841301in}{1.884963in}}%
\pgfpathlineto{\pgfqpoint{1.809749in}{1.946260in}}%
\pgfpathclose%
\pgfusepath{fill}%
\end{pgfscope}%
\begin{pgfscope}%
\pgfpathrectangle{\pgfqpoint{1.072000in}{0.528000in}}{\pgfqpoint{3.696000in}{3.696000in}}%
\pgfusepath{clip}%
\pgfsetbuttcap%
\pgfsetroundjoin%
\definecolor{currentfill}{rgb}{0.363461,0.484784,0.901019}%
\pgfsetfillcolor{currentfill}%
\pgfsetlinewidth{0.000000pt}%
\definecolor{currentstroke}{rgb}{0.000000,0.000000,0.000000}%
\pgfsetstrokecolor{currentstroke}%
\pgfsetdash{}{0pt}%
\pgfpathmoveto{\pgfqpoint{3.799635in}{1.831889in}}%
\pgfpathlineto{\pgfqpoint{3.843579in}{1.705246in}}%
\pgfpathlineto{\pgfqpoint{3.867535in}{1.704100in}}%
\pgfpathlineto{\pgfqpoint{3.823484in}{1.819236in}}%
\pgfpathlineto{\pgfqpoint{3.799635in}{1.831889in}}%
\pgfpathclose%
\pgfusepath{fill}%
\end{pgfscope}%
\begin{pgfscope}%
\pgfpathrectangle{\pgfqpoint{1.072000in}{0.528000in}}{\pgfqpoint{3.696000in}{3.696000in}}%
\pgfusepath{clip}%
\pgfsetbuttcap%
\pgfsetroundjoin%
\definecolor{currentfill}{rgb}{0.570616,0.704109,0.997195}%
\pgfsetfillcolor{currentfill}%
\pgfsetlinewidth{0.000000pt}%
\definecolor{currentstroke}{rgb}{0.000000,0.000000,0.000000}%
\pgfsetstrokecolor{currentstroke}%
\pgfsetdash{}{0pt}%
\pgfpathmoveto{\pgfqpoint{1.743648in}{2.167640in}}%
\pgfpathlineto{\pgfqpoint{1.792902in}{2.017187in}}%
\pgfpathlineto{\pgfqpoint{1.824341in}{1.959318in}}%
\pgfpathlineto{\pgfqpoint{1.775619in}{2.102408in}}%
\pgfpathlineto{\pgfqpoint{1.743648in}{2.167640in}}%
\pgfpathclose%
\pgfusepath{fill}%
\end{pgfscope}%
\begin{pgfscope}%
\pgfpathrectangle{\pgfqpoint{1.072000in}{0.528000in}}{\pgfqpoint{3.696000in}{3.696000in}}%
\pgfusepath{clip}%
\pgfsetbuttcap%
\pgfsetroundjoin%
\definecolor{currentfill}{rgb}{0.243520,0.319189,0.771672}%
\pgfsetfillcolor{currentfill}%
\pgfsetlinewidth{0.000000pt}%
\definecolor{currentstroke}{rgb}{0.000000,0.000000,0.000000}%
\pgfsetstrokecolor{currentstroke}%
\pgfsetdash{}{0pt}%
\pgfpathmoveto{\pgfqpoint{3.863996in}{1.601868in}}%
\pgfpathlineto{\pgfqpoint{3.909558in}{1.540439in}}%
\pgfpathlineto{\pgfqpoint{3.933958in}{1.560773in}}%
\pgfpathlineto{\pgfqpoint{3.888261in}{1.613350in}}%
\pgfpathlineto{\pgfqpoint{3.863996in}{1.601868in}}%
\pgfpathclose%
\pgfusepath{fill}%
\end{pgfscope}%
\begin{pgfscope}%
\pgfpathrectangle{\pgfqpoint{1.072000in}{0.528000in}}{\pgfqpoint{3.696000in}{3.696000in}}%
\pgfusepath{clip}%
\pgfsetbuttcap%
\pgfsetroundjoin%
\definecolor{currentfill}{rgb}{0.348323,0.465711,0.888346}%
\pgfsetfillcolor{currentfill}%
\pgfsetlinewidth{0.000000pt}%
\definecolor{currentstroke}{rgb}{0.000000,0.000000,0.000000}%
\pgfsetstrokecolor{currentstroke}%
\pgfsetdash{}{0pt}%
\pgfpathmoveto{\pgfqpoint{2.141487in}{1.666285in}}%
\pgfpathlineto{\pgfqpoint{2.184755in}{1.717272in}}%
\pgfpathlineto{\pgfqpoint{2.210780in}{1.827908in}}%
\pgfpathlineto{\pgfqpoint{2.167905in}{1.755502in}}%
\pgfpathlineto{\pgfqpoint{2.141487in}{1.666285in}}%
\pgfpathclose%
\pgfusepath{fill}%
\end{pgfscope}%
\begin{pgfscope}%
\pgfpathrectangle{\pgfqpoint{1.072000in}{0.528000in}}{\pgfqpoint{3.696000in}{3.696000in}}%
\pgfusepath{clip}%
\pgfsetbuttcap%
\pgfsetroundjoin%
\definecolor{currentfill}{rgb}{0.289996,0.386836,0.828926}%
\pgfsetfillcolor{currentfill}%
\pgfsetlinewidth{0.000000pt}%
\definecolor{currentstroke}{rgb}{0.000000,0.000000,0.000000}%
\pgfsetstrokecolor{currentstroke}%
\pgfsetdash{}{0pt}%
\pgfpathmoveto{\pgfqpoint{2.114625in}{1.594432in}}%
\pgfpathlineto{\pgfqpoint{2.158367in}{1.623097in}}%
\pgfpathlineto{\pgfqpoint{2.184755in}{1.717272in}}%
\pgfpathlineto{\pgfqpoint{2.141487in}{1.666285in}}%
\pgfpathlineto{\pgfqpoint{2.114625in}{1.594432in}}%
\pgfpathclose%
\pgfusepath{fill}%
\end{pgfscope}%
\begin{pgfscope}%
\pgfpathrectangle{\pgfqpoint{1.072000in}{0.528000in}}{\pgfqpoint{3.696000in}{3.696000in}}%
\pgfusepath{clip}%
\pgfsetbuttcap%
\pgfsetroundjoin%
\definecolor{currentfill}{rgb}{0.243520,0.319189,0.771672}%
\pgfsetfillcolor{currentfill}%
\pgfsetlinewidth{0.000000pt}%
\definecolor{currentstroke}{rgb}{0.000000,0.000000,0.000000}%
\pgfsetstrokecolor{currentstroke}%
\pgfsetdash{}{0pt}%
\pgfpathmoveto{\pgfqpoint{1.967852in}{1.605131in}}%
\pgfpathlineto{\pgfqpoint{2.013676in}{1.552002in}}%
\pgfpathlineto{\pgfqpoint{2.042538in}{1.555977in}}%
\pgfpathlineto{\pgfqpoint{1.997420in}{1.588593in}}%
\pgfpathlineto{\pgfqpoint{1.967852in}{1.605131in}}%
\pgfpathclose%
\pgfusepath{fill}%
\end{pgfscope}%
\begin{pgfscope}%
\pgfpathrectangle{\pgfqpoint{1.072000in}{0.528000in}}{\pgfqpoint{3.696000in}{3.696000in}}%
\pgfusepath{clip}%
\pgfsetbuttcap%
\pgfsetroundjoin%
\definecolor{currentfill}{rgb}{0.839351,0.861167,0.894494}%
\pgfsetfillcolor{currentfill}%
\pgfsetlinewidth{0.000000pt}%
\definecolor{currentstroke}{rgb}{0.000000,0.000000,0.000000}%
\pgfsetstrokecolor{currentstroke}%
\pgfsetdash{}{0pt}%
\pgfpathmoveto{\pgfqpoint{3.631865in}{2.590939in}}%
\pgfpathlineto{\pgfqpoint{3.675772in}{2.438053in}}%
\pgfpathlineto{\pgfqpoint{3.698316in}{2.344597in}}%
\pgfpathlineto{\pgfqpoint{3.654548in}{2.488283in}}%
\pgfpathlineto{\pgfqpoint{3.631865in}{2.590939in}}%
\pgfpathclose%
\pgfusepath{fill}%
\end{pgfscope}%
\begin{pgfscope}%
\pgfpathrectangle{\pgfqpoint{1.072000in}{0.528000in}}{\pgfqpoint{3.696000in}{3.696000in}}%
\pgfusepath{clip}%
\pgfsetbuttcap%
\pgfsetroundjoin%
\definecolor{currentfill}{rgb}{0.275827,0.366717,0.812553}%
\pgfsetfillcolor{currentfill}%
\pgfsetlinewidth{0.000000pt}%
\definecolor{currentstroke}{rgb}{0.000000,0.000000,0.000000}%
\pgfsetstrokecolor{currentstroke}%
\pgfsetdash{}{0pt}%
\pgfpathmoveto{\pgfqpoint{1.921384in}{1.679051in}}%
\pgfpathlineto{\pgfqpoint{1.967852in}{1.605131in}}%
\pgfpathlineto{\pgfqpoint{1.997420in}{1.588593in}}%
\pgfpathlineto{\pgfqpoint{1.951678in}{1.643717in}}%
\pgfpathlineto{\pgfqpoint{1.921384in}{1.679051in}}%
\pgfpathclose%
\pgfusepath{fill}%
\end{pgfscope}%
\begin{pgfscope}%
\pgfpathrectangle{\pgfqpoint{1.072000in}{0.528000in}}{\pgfqpoint{3.696000in}{3.696000in}}%
\pgfusepath{clip}%
\pgfsetbuttcap%
\pgfsetroundjoin%
\definecolor{currentfill}{rgb}{0.960581,0.762501,0.667964}%
\pgfsetfillcolor{currentfill}%
\pgfsetlinewidth{0.000000pt}%
\definecolor{currentstroke}{rgb}{0.000000,0.000000,0.000000}%
\pgfsetstrokecolor{currentstroke}%
\pgfsetdash{}{0pt}%
\pgfpathmoveto{\pgfqpoint{2.431691in}{2.661246in}}%
\pgfpathlineto{\pgfqpoint{2.474280in}{2.825327in}}%
\pgfpathlineto{\pgfqpoint{2.501634in}{2.855227in}}%
\pgfpathlineto{\pgfqpoint{2.458953in}{2.691100in}}%
\pgfpathlineto{\pgfqpoint{2.431691in}{2.661246in}}%
\pgfpathclose%
\pgfusepath{fill}%
\end{pgfscope}%
\begin{pgfscope}%
\pgfpathrectangle{\pgfqpoint{1.072000in}{0.528000in}}{\pgfqpoint{3.696000in}{3.696000in}}%
\pgfusepath{clip}%
\pgfsetbuttcap%
\pgfsetroundjoin%
\definecolor{currentfill}{rgb}{0.229806,0.298718,0.753683}%
\pgfsetfillcolor{currentfill}%
\pgfsetlinewidth{0.000000pt}%
\definecolor{currentstroke}{rgb}{0.000000,0.000000,0.000000}%
\pgfsetstrokecolor{currentstroke}%
\pgfsetdash{}{0pt}%
\pgfpathmoveto{\pgfqpoint{3.909558in}{1.540439in}}%
\pgfpathlineto{\pgfqpoint{3.956339in}{1.521477in}}%
\pgfpathlineto{\pgfqpoint{3.980908in}{1.549962in}}%
\pgfpathlineto{\pgfqpoint{3.933958in}{1.560773in}}%
\pgfpathlineto{\pgfqpoint{3.909558in}{1.540439in}}%
\pgfpathclose%
\pgfusepath{fill}%
\end{pgfscope}%
\begin{pgfscope}%
\pgfpathrectangle{\pgfqpoint{1.072000in}{0.528000in}}{\pgfqpoint{3.696000in}{3.696000in}}%
\pgfusepath{clip}%
\pgfsetbuttcap%
\pgfsetroundjoin%
\definecolor{currentfill}{rgb}{0.425199,0.559058,0.946061}%
\pgfsetfillcolor{currentfill}%
\pgfsetlinewidth{0.000000pt}%
\definecolor{currentstroke}{rgb}{0.000000,0.000000,0.000000}%
\pgfsetstrokecolor{currentstroke}%
\pgfsetdash{}{0pt}%
\pgfpathmoveto{\pgfqpoint{2.167905in}{1.755502in}}%
\pgfpathlineto{\pgfqpoint{2.210780in}{1.827908in}}%
\pgfpathlineto{\pgfqpoint{2.236597in}{1.949724in}}%
\pgfpathlineto{\pgfqpoint{2.194030in}{1.857457in}}%
\pgfpathlineto{\pgfqpoint{2.167905in}{1.755502in}}%
\pgfpathclose%
\pgfusepath{fill}%
\end{pgfscope}%
\begin{pgfscope}%
\pgfpathrectangle{\pgfqpoint{1.072000in}{0.528000in}}{\pgfqpoint{3.696000in}{3.696000in}}%
\pgfusepath{clip}%
\pgfsetbuttcap%
\pgfsetroundjoin%
\definecolor{currentfill}{rgb}{0.905783,0.455186,0.355336}%
\pgfsetfillcolor{currentfill}%
\pgfsetlinewidth{0.000000pt}%
\definecolor{currentstroke}{rgb}{0.000000,0.000000,0.000000}%
\pgfsetstrokecolor{currentstroke}%
\pgfsetdash{}{0pt}%
\pgfpathmoveto{\pgfqpoint{3.152708in}{3.206045in}}%
\pgfpathlineto{\pgfqpoint{3.198965in}{3.213779in}}%
\pgfpathlineto{\pgfqpoint{3.224013in}{3.100471in}}%
\pgfpathlineto{\pgfqpoint{3.177965in}{3.090395in}}%
\pgfpathlineto{\pgfqpoint{3.152708in}{3.206045in}}%
\pgfpathclose%
\pgfusepath{fill}%
\end{pgfscope}%
\begin{pgfscope}%
\pgfpathrectangle{\pgfqpoint{1.072000in}{0.528000in}}{\pgfqpoint{3.696000in}{3.696000in}}%
\pgfusepath{clip}%
\pgfsetbuttcap%
\pgfsetroundjoin%
\definecolor{currentfill}{rgb}{0.248091,0.326013,0.777669}%
\pgfsetfillcolor{currentfill}%
\pgfsetlinewidth{0.000000pt}%
\definecolor{currentstroke}{rgb}{0.000000,0.000000,0.000000}%
\pgfsetstrokecolor{currentstroke}%
\pgfsetdash{}{0pt}%
\pgfpathmoveto{\pgfqpoint{2.087167in}{1.544062in}}%
\pgfpathlineto{\pgfqpoint{2.131454in}{1.550220in}}%
\pgfpathlineto{\pgfqpoint{2.158367in}{1.623097in}}%
\pgfpathlineto{\pgfqpoint{2.114625in}{1.594432in}}%
\pgfpathlineto{\pgfqpoint{2.087167in}{1.544062in}}%
\pgfpathclose%
\pgfusepath{fill}%
\end{pgfscope}%
\begin{pgfscope}%
\pgfpathrectangle{\pgfqpoint{1.072000in}{0.528000in}}{\pgfqpoint{3.696000in}{3.696000in}}%
\pgfusepath{clip}%
\pgfsetbuttcap%
\pgfsetroundjoin%
\definecolor{currentfill}{rgb}{0.299441,0.400248,0.839842}%
\pgfsetfillcolor{currentfill}%
\pgfsetlinewidth{0.000000pt}%
\definecolor{currentstroke}{rgb}{0.000000,0.000000,0.000000}%
\pgfsetstrokecolor{currentstroke}%
\pgfsetdash{}{0pt}%
\pgfpathmoveto{\pgfqpoint{3.819404in}{1.702948in}}%
\pgfpathlineto{\pgfqpoint{3.863996in}{1.601868in}}%
\pgfpathlineto{\pgfqpoint{3.888261in}{1.613350in}}%
\pgfpathlineto{\pgfqpoint{3.843579in}{1.705246in}}%
\pgfpathlineto{\pgfqpoint{3.819404in}{1.702948in}}%
\pgfpathclose%
\pgfusepath{fill}%
\end{pgfscope}%
\begin{pgfscope}%
\pgfpathrectangle{\pgfqpoint{1.072000in}{0.528000in}}{\pgfqpoint{3.696000in}{3.696000in}}%
\pgfusepath{clip}%
\pgfsetbuttcap%
\pgfsetroundjoin%
\definecolor{currentfill}{rgb}{0.229806,0.298718,0.753683}%
\pgfsetfillcolor{currentfill}%
\pgfsetlinewidth{0.000000pt}%
\definecolor{currentstroke}{rgb}{0.000000,0.000000,0.000000}%
\pgfsetstrokecolor{currentstroke}%
\pgfsetdash{}{0pt}%
\pgfpathmoveto{\pgfqpoint{2.013676in}{1.552002in}}%
\pgfpathlineto{\pgfqpoint{2.058968in}{1.518522in}}%
\pgfpathlineto{\pgfqpoint{2.087167in}{1.544062in}}%
\pgfpathlineto{\pgfqpoint{2.042538in}{1.555977in}}%
\pgfpathlineto{\pgfqpoint{2.013676in}{1.552002in}}%
\pgfpathclose%
\pgfusepath{fill}%
\end{pgfscope}%
\begin{pgfscope}%
\pgfpathrectangle{\pgfqpoint{1.072000in}{0.528000in}}{\pgfqpoint{3.696000in}{3.696000in}}%
\pgfusepath{clip}%
\pgfsetbuttcap%
\pgfsetroundjoin%
\definecolor{currentfill}{rgb}{0.640828,0.760752,0.997846}%
\pgfsetfillcolor{currentfill}%
\pgfsetlinewidth{0.000000pt}%
\definecolor{currentstroke}{rgb}{0.000000,0.000000,0.000000}%
\pgfsetstrokecolor{currentstroke}%
\pgfsetdash{}{0pt}%
\pgfpathmoveto{\pgfqpoint{3.716797in}{2.255100in}}%
\pgfpathlineto{\pgfqpoint{3.760096in}{2.087003in}}%
\pgfpathlineto{\pgfqpoint{3.783237in}{2.039723in}}%
\pgfpathlineto{\pgfqpoint{3.739927in}{2.195240in}}%
\pgfpathlineto{\pgfqpoint{3.716797in}{2.255100in}}%
\pgfpathclose%
\pgfusepath{fill}%
\end{pgfscope}%
\begin{pgfscope}%
\pgfpathrectangle{\pgfqpoint{1.072000in}{0.528000in}}{\pgfqpoint{3.696000in}{3.696000in}}%
\pgfusepath{clip}%
\pgfsetbuttcap%
\pgfsetroundjoin%
\definecolor{currentfill}{rgb}{0.333490,0.446265,0.874452}%
\pgfsetfillcolor{currentfill}%
\pgfsetlinewidth{0.000000pt}%
\definecolor{currentstroke}{rgb}{0.000000,0.000000,0.000000}%
\pgfsetstrokecolor{currentstroke}%
\pgfsetdash{}{0pt}%
\pgfpathmoveto{\pgfqpoint{1.874185in}{1.773922in}}%
\pgfpathlineto{\pgfqpoint{1.921384in}{1.679051in}}%
\pgfpathlineto{\pgfqpoint{1.951678in}{1.643717in}}%
\pgfpathlineto{\pgfqpoint{1.905197in}{1.722147in}}%
\pgfpathlineto{\pgfqpoint{1.874185in}{1.773922in}}%
\pgfpathclose%
\pgfusepath{fill}%
\end{pgfscope}%
\begin{pgfscope}%
\pgfpathrectangle{\pgfqpoint{1.072000in}{0.528000in}}{\pgfqpoint{3.696000in}{3.696000in}}%
\pgfusepath{clip}%
\pgfsetbuttcap%
\pgfsetroundjoin%
\definecolor{currentfill}{rgb}{0.960581,0.762501,0.667964}%
\pgfsetfillcolor{currentfill}%
\pgfsetlinewidth{0.000000pt}%
\definecolor{currentstroke}{rgb}{0.000000,0.000000,0.000000}%
\pgfsetstrokecolor{currentstroke}%
\pgfsetdash{}{0pt}%
\pgfpathmoveto{\pgfqpoint{3.520976in}{2.868293in}}%
\pgfpathlineto{\pgfqpoint{3.565916in}{2.754446in}}%
\pgfpathlineto{\pgfqpoint{3.588606in}{2.631942in}}%
\pgfpathlineto{\pgfqpoint{3.543954in}{2.743270in}}%
\pgfpathlineto{\pgfqpoint{3.520976in}{2.868293in}}%
\pgfpathclose%
\pgfusepath{fill}%
\end{pgfscope}%
\begin{pgfscope}%
\pgfpathrectangle{\pgfqpoint{1.072000in}{0.528000in}}{\pgfqpoint{3.696000in}{3.696000in}}%
\pgfusepath{clip}%
\pgfsetbuttcap%
\pgfsetroundjoin%
\definecolor{currentfill}{rgb}{0.877149,0.394645,0.311724}%
\pgfsetfillcolor{currentfill}%
\pgfsetlinewidth{0.000000pt}%
\definecolor{currentstroke}{rgb}{0.000000,0.000000,0.000000}%
\pgfsetstrokecolor{currentstroke}%
\pgfsetdash{}{0pt}%
\pgfpathmoveto{\pgfqpoint{2.916323in}{3.212657in}}%
\pgfpathlineto{\pgfqpoint{2.962135in}{3.270803in}}%
\pgfpathlineto{\pgfqpoint{2.988593in}{3.194113in}}%
\pgfpathlineto{\pgfqpoint{2.942923in}{3.126942in}}%
\pgfpathlineto{\pgfqpoint{2.916323in}{3.212657in}}%
\pgfpathclose%
\pgfusepath{fill}%
\end{pgfscope}%
\begin{pgfscope}%
\pgfpathrectangle{\pgfqpoint{1.072000in}{0.528000in}}{\pgfqpoint{3.696000in}{3.696000in}}%
\pgfusepath{clip}%
\pgfsetbuttcap%
\pgfsetroundjoin%
\definecolor{currentfill}{rgb}{0.510824,0.649397,0.985079}%
\pgfsetfillcolor{currentfill}%
\pgfsetlinewidth{0.000000pt}%
\definecolor{currentstroke}{rgb}{0.000000,0.000000,0.000000}%
\pgfsetstrokecolor{currentstroke}%
\pgfsetdash{}{0pt}%
\pgfpathmoveto{\pgfqpoint{2.194030in}{1.857457in}}%
\pgfpathlineto{\pgfqpoint{2.236597in}{1.949724in}}%
\pgfpathlineto{\pgfqpoint{2.262344in}{2.077320in}}%
\pgfpathlineto{\pgfqpoint{2.219999in}{1.967307in}}%
\pgfpathlineto{\pgfqpoint{2.194030in}{1.857457in}}%
\pgfpathclose%
\pgfusepath{fill}%
\end{pgfscope}%
\begin{pgfscope}%
\pgfpathrectangle{\pgfqpoint{1.072000in}{0.528000in}}{\pgfqpoint{3.696000in}{3.696000in}}%
\pgfusepath{clip}%
\pgfsetbuttcap%
\pgfsetroundjoin%
\definecolor{currentfill}{rgb}{0.939254,0.539581,0.423900}%
\pgfsetfillcolor{currentfill}%
\pgfsetlinewidth{0.000000pt}%
\definecolor{currentstroke}{rgb}{0.000000,0.000000,0.000000}%
\pgfsetstrokecolor{currentstroke}%
\pgfsetdash{}{0pt}%
\pgfpathmoveto{\pgfqpoint{2.627675in}{3.000975in}}%
\pgfpathlineto{\pgfqpoint{2.671908in}{3.129293in}}%
\pgfpathlineto{\pgfqpoint{2.699393in}{3.100346in}}%
\pgfpathlineto{\pgfqpoint{2.655203in}{2.968906in}}%
\pgfpathlineto{\pgfqpoint{2.627675in}{3.000975in}}%
\pgfpathclose%
\pgfusepath{fill}%
\end{pgfscope}%
\begin{pgfscope}%
\pgfpathrectangle{\pgfqpoint{1.072000in}{0.528000in}}{\pgfqpoint{3.696000in}{3.696000in}}%
\pgfusepath{clip}%
\pgfsetbuttcap%
\pgfsetroundjoin%
\definecolor{currentfill}{rgb}{0.565182,0.699438,0.996635}%
\pgfsetfillcolor{currentfill}%
\pgfsetlinewidth{0.000000pt}%
\definecolor{currentstroke}{rgb}{0.000000,0.000000,0.000000}%
\pgfsetstrokecolor{currentstroke}%
\pgfsetdash{}{0pt}%
\pgfpathmoveto{\pgfqpoint{3.736632in}{2.128634in}}%
\pgfpathlineto{\pgfqpoint{3.779944in}{1.963234in}}%
\pgfpathlineto{\pgfqpoint{3.803440in}{1.933659in}}%
\pgfpathlineto{\pgfqpoint{3.760096in}{2.087003in}}%
\pgfpathlineto{\pgfqpoint{3.736632in}{2.128634in}}%
\pgfpathclose%
\pgfusepath{fill}%
\end{pgfscope}%
\begin{pgfscope}%
\pgfpathrectangle{\pgfqpoint{1.072000in}{0.528000in}}{\pgfqpoint{3.696000in}{3.696000in}}%
\pgfusepath{clip}%
\pgfsetbuttcap%
\pgfsetroundjoin%
\definecolor{currentfill}{rgb}{0.532568,0.669801,0.990393}%
\pgfsetfillcolor{currentfill}%
\pgfsetlinewidth{0.000000pt}%
\definecolor{currentstroke}{rgb}{0.000000,0.000000,0.000000}%
\pgfsetstrokecolor{currentstroke}%
\pgfsetdash{}{0pt}%
\pgfpathmoveto{\pgfqpoint{1.760768in}{2.087414in}}%
\pgfpathlineto{\pgfqpoint{1.809749in}{1.946260in}}%
\pgfpathlineto{\pgfqpoint{1.841301in}{1.884963in}}%
\pgfpathlineto{\pgfqpoint{1.792902in}{2.017187in}}%
\pgfpathlineto{\pgfqpoint{1.760768in}{2.087414in}}%
\pgfpathclose%
\pgfusepath{fill}%
\end{pgfscope}%
\begin{pgfscope}%
\pgfpathrectangle{\pgfqpoint{1.072000in}{0.528000in}}{\pgfqpoint{3.696000in}{3.696000in}}%
\pgfusepath{clip}%
\pgfsetbuttcap%
\pgfsetroundjoin%
\definecolor{currentfill}{rgb}{0.869655,0.379274,0.300941}%
\pgfsetfillcolor{currentfill}%
\pgfsetlinewidth{0.000000pt}%
\definecolor{currentstroke}{rgb}{0.000000,0.000000,0.000000}%
\pgfsetstrokecolor{currentstroke}%
\pgfsetdash{}{0pt}%
\pgfpathmoveto{\pgfqpoint{3.034540in}{3.242475in}}%
\pgfpathlineto{\pgfqpoint{3.080695in}{3.273434in}}%
\pgfpathlineto{\pgfqpoint{3.106553in}{3.183785in}}%
\pgfpathlineto{\pgfqpoint{3.060553in}{3.145761in}}%
\pgfpathlineto{\pgfqpoint{3.034540in}{3.242475in}}%
\pgfpathclose%
\pgfusepath{fill}%
\end{pgfscope}%
\begin{pgfscope}%
\pgfpathrectangle{\pgfqpoint{1.072000in}{0.528000in}}{\pgfqpoint{3.696000in}{3.696000in}}%
\pgfusepath{clip}%
\pgfsetbuttcap%
\pgfsetroundjoin%
\definecolor{currentfill}{rgb}{0.229806,0.298718,0.753683}%
\pgfsetfillcolor{currentfill}%
\pgfsetlinewidth{0.000000pt}%
\definecolor{currentstroke}{rgb}{0.000000,0.000000,0.000000}%
\pgfsetstrokecolor{currentstroke}%
\pgfsetdash{}{0pt}%
\pgfpathmoveto{\pgfqpoint{2.058968in}{1.518522in}}%
\pgfpathlineto{\pgfqpoint{2.103856in}{1.502726in}}%
\pgfpathlineto{\pgfqpoint{2.131454in}{1.550220in}}%
\pgfpathlineto{\pgfqpoint{2.087167in}{1.544062in}}%
\pgfpathlineto{\pgfqpoint{2.058968in}{1.518522in}}%
\pgfpathclose%
\pgfusepath{fill}%
\end{pgfscope}%
\begin{pgfscope}%
\pgfpathrectangle{\pgfqpoint{1.072000in}{0.528000in}}{\pgfqpoint{3.696000in}{3.696000in}}%
\pgfusepath{clip}%
\pgfsetbuttcap%
\pgfsetroundjoin%
\definecolor{currentfill}{rgb}{0.956371,0.775144,0.686416}%
\pgfsetfillcolor{currentfill}%
\pgfsetlinewidth{0.000000pt}%
\definecolor{currentstroke}{rgb}{0.000000,0.000000,0.000000}%
\pgfsetstrokecolor{currentstroke}%
\pgfsetdash{}{0pt}%
\pgfpathmoveto{\pgfqpoint{2.404577in}{2.616289in}}%
\pgfpathlineto{\pgfqpoint{2.447062in}{2.779743in}}%
\pgfpathlineto{\pgfqpoint{2.474280in}{2.825327in}}%
\pgfpathlineto{\pgfqpoint{2.431691in}{2.661246in}}%
\pgfpathlineto{\pgfqpoint{2.404577in}{2.616289in}}%
\pgfpathclose%
\pgfusepath{fill}%
\end{pgfscope}%
\begin{pgfscope}%
\pgfpathrectangle{\pgfqpoint{1.072000in}{0.528000in}}{\pgfqpoint{3.696000in}{3.696000in}}%
\pgfusepath{clip}%
\pgfsetbuttcap%
\pgfsetroundjoin%
\definecolor{currentfill}{rgb}{0.962701,0.628218,0.507636}%
\pgfsetfillcolor{currentfill}%
\pgfsetlinewidth{0.000000pt}%
\definecolor{currentstroke}{rgb}{0.000000,0.000000,0.000000}%
\pgfsetstrokecolor{currentstroke}%
\pgfsetdash{}{0pt}%
\pgfpathmoveto{\pgfqpoint{2.529086in}{2.869503in}}%
\pgfpathlineto{\pgfqpoint{2.572573in}{3.018056in}}%
\pgfpathlineto{\pgfqpoint{2.600121in}{3.017262in}}%
\pgfpathlineto{\pgfqpoint{2.556601in}{2.868224in}}%
\pgfpathlineto{\pgfqpoint{2.529086in}{2.869503in}}%
\pgfpathclose%
\pgfusepath{fill}%
\end{pgfscope}%
\begin{pgfscope}%
\pgfpathrectangle{\pgfqpoint{1.072000in}{0.528000in}}{\pgfqpoint{3.696000in}{3.696000in}}%
\pgfusepath{clip}%
\pgfsetbuttcap%
\pgfsetroundjoin%
\definecolor{currentfill}{rgb}{0.603162,0.731527,0.999565}%
\pgfsetfillcolor{currentfill}%
\pgfsetlinewidth{0.000000pt}%
\definecolor{currentstroke}{rgb}{0.000000,0.000000,0.000000}%
\pgfsetstrokecolor{currentstroke}%
\pgfsetdash{}{0pt}%
\pgfpathmoveto{\pgfqpoint{2.219999in}{1.967307in}}%
\pgfpathlineto{\pgfqpoint{2.262344in}{2.077320in}}%
\pgfpathlineto{\pgfqpoint{2.288139in}{2.205504in}}%
\pgfpathlineto{\pgfqpoint{2.245933in}{2.080271in}}%
\pgfpathlineto{\pgfqpoint{2.219999in}{1.967307in}}%
\pgfpathclose%
\pgfusepath{fill}%
\end{pgfscope}%
\begin{pgfscope}%
\pgfpathrectangle{\pgfqpoint{1.072000in}{0.528000in}}{\pgfqpoint{3.696000in}{3.696000in}}%
\pgfusepath{clip}%
\pgfsetbuttcap%
\pgfsetroundjoin%
\definecolor{currentfill}{rgb}{0.473070,0.611077,0.970634}%
\pgfsetfillcolor{currentfill}%
\pgfsetlinewidth{0.000000pt}%
\definecolor{currentstroke}{rgb}{0.000000,0.000000,0.000000}%
\pgfsetstrokecolor{currentstroke}%
\pgfsetdash{}{0pt}%
\pgfpathmoveto{\pgfqpoint{3.756138in}{1.986793in}}%
\pgfpathlineto{\pgfqpoint{3.799635in}{1.831889in}}%
\pgfpathlineto{\pgfqpoint{3.823484in}{1.819236in}}%
\pgfpathlineto{\pgfqpoint{3.779944in}{1.963234in}}%
\pgfpathlineto{\pgfqpoint{3.756138in}{1.986793in}}%
\pgfpathclose%
\pgfusepath{fill}%
\end{pgfscope}%
\begin{pgfscope}%
\pgfpathrectangle{\pgfqpoint{1.072000in}{0.528000in}}{\pgfqpoint{3.696000in}{3.696000in}}%
\pgfusepath{clip}%
\pgfsetbuttcap%
\pgfsetroundjoin%
\definecolor{currentfill}{rgb}{0.895885,0.433075,0.338681}%
\pgfsetfillcolor{currentfill}%
\pgfsetlinewidth{0.000000pt}%
\definecolor{currentstroke}{rgb}{0.000000,0.000000,0.000000}%
\pgfsetstrokecolor{currentstroke}%
\pgfsetdash{}{0pt}%
\pgfpathmoveto{\pgfqpoint{3.198965in}{3.213779in}}%
\pgfpathlineto{\pgfqpoint{3.245276in}{3.207980in}}%
\pgfpathlineto{\pgfqpoint{3.270097in}{3.095494in}}%
\pgfpathlineto{\pgfqpoint{3.224013in}{3.100471in}}%
\pgfpathlineto{\pgfqpoint{3.198965in}{3.213779in}}%
\pgfpathclose%
\pgfusepath{fill}%
\end{pgfscope}%
\begin{pgfscope}%
\pgfpathrectangle{\pgfqpoint{1.072000in}{0.528000in}}{\pgfqpoint{3.696000in}{3.696000in}}%
\pgfusepath{clip}%
\pgfsetbuttcap%
\pgfsetroundjoin%
\definecolor{currentfill}{rgb}{0.409611,0.540759,0.935545}%
\pgfsetfillcolor{currentfill}%
\pgfsetlinewidth{0.000000pt}%
\definecolor{currentstroke}{rgb}{0.000000,0.000000,0.000000}%
\pgfsetstrokecolor{currentstroke}%
\pgfsetdash{}{0pt}%
\pgfpathmoveto{\pgfqpoint{1.826206in}{1.888804in}}%
\pgfpathlineto{\pgfqpoint{1.874185in}{1.773922in}}%
\pgfpathlineto{\pgfqpoint{1.905197in}{1.722147in}}%
\pgfpathlineto{\pgfqpoint{1.857898in}{1.823532in}}%
\pgfpathlineto{\pgfqpoint{1.826206in}{1.888804in}}%
\pgfpathclose%
\pgfusepath{fill}%
\end{pgfscope}%
\begin{pgfscope}%
\pgfpathrectangle{\pgfqpoint{1.072000in}{0.528000in}}{\pgfqpoint{3.696000in}{3.696000in}}%
\pgfusepath{clip}%
\pgfsetbuttcap%
\pgfsetroundjoin%
\definecolor{currentfill}{rgb}{0.252663,0.332837,0.783665}%
\pgfsetfillcolor{currentfill}%
\pgfsetlinewidth{0.000000pt}%
\definecolor{currentstroke}{rgb}{0.000000,0.000000,0.000000}%
\pgfsetstrokecolor{currentstroke}%
\pgfsetdash{}{0pt}%
\pgfpathmoveto{\pgfqpoint{3.839548in}{1.587736in}}%
\pgfpathlineto{\pgfqpoint{3.885049in}{1.519817in}}%
\pgfpathlineto{\pgfqpoint{3.909558in}{1.540439in}}%
\pgfpathlineto{\pgfqpoint{3.863996in}{1.601868in}}%
\pgfpathlineto{\pgfqpoint{3.839548in}{1.587736in}}%
\pgfpathclose%
\pgfusepath{fill}%
\end{pgfscope}%
\begin{pgfscope}%
\pgfpathrectangle{\pgfqpoint{1.072000in}{0.528000in}}{\pgfqpoint{3.696000in}{3.696000in}}%
\pgfusepath{clip}%
\pgfsetbuttcap%
\pgfsetroundjoin%
\definecolor{currentfill}{rgb}{0.869655,0.379274,0.300941}%
\pgfsetfillcolor{currentfill}%
\pgfsetlinewidth{0.000000pt}%
\definecolor{currentstroke}{rgb}{0.000000,0.000000,0.000000}%
\pgfsetstrokecolor{currentstroke}%
\pgfsetdash{}{0pt}%
\pgfpathmoveto{\pgfqpoint{2.843917in}{3.203623in}}%
\pgfpathlineto{\pgfqpoint{2.889513in}{3.273558in}}%
\pgfpathlineto{\pgfqpoint{2.916323in}{3.212657in}}%
\pgfpathlineto{\pgfqpoint{2.870860in}{3.132675in}}%
\pgfpathlineto{\pgfqpoint{2.843917in}{3.203623in}}%
\pgfpathclose%
\pgfusepath{fill}%
\end{pgfscope}%
\begin{pgfscope}%
\pgfpathrectangle{\pgfqpoint{1.072000in}{0.528000in}}{\pgfqpoint{3.696000in}{3.696000in}}%
\pgfusepath{clip}%
\pgfsetbuttcap%
\pgfsetroundjoin%
\definecolor{currentfill}{rgb}{0.905783,0.455186,0.355336}%
\pgfsetfillcolor{currentfill}%
\pgfsetlinewidth{0.000000pt}%
\definecolor{currentstroke}{rgb}{0.000000,0.000000,0.000000}%
\pgfsetstrokecolor{currentstroke}%
\pgfsetdash{}{0pt}%
\pgfpathmoveto{\pgfqpoint{2.699393in}{3.100346in}}%
\pgfpathlineto{\pgfqpoint{2.744221in}{3.206306in}}%
\pgfpathlineto{\pgfqpoint{2.771545in}{3.166964in}}%
\pgfpathlineto{\pgfqpoint{2.726808in}{3.054642in}}%
\pgfpathlineto{\pgfqpoint{2.699393in}{3.100346in}}%
\pgfpathclose%
\pgfusepath{fill}%
\end{pgfscope}%
\begin{pgfscope}%
\pgfpathrectangle{\pgfqpoint{1.072000in}{0.528000in}}{\pgfqpoint{3.696000in}{3.696000in}}%
\pgfusepath{clip}%
\pgfsetbuttcap%
\pgfsetroundjoin%
\definecolor{currentfill}{rgb}{0.818056,0.855590,0.914638}%
\pgfsetfillcolor{currentfill}%
\pgfsetlinewidth{0.000000pt}%
\definecolor{currentstroke}{rgb}{0.000000,0.000000,0.000000}%
\pgfsetstrokecolor{currentstroke}%
\pgfsetdash{}{0pt}%
\pgfpathmoveto{\pgfqpoint{3.652836in}{2.524938in}}%
\pgfpathlineto{\pgfqpoint{3.696524in}{2.359619in}}%
\pgfpathlineto{\pgfqpoint{3.719345in}{2.282267in}}%
\pgfpathlineto{\pgfqpoint{3.675772in}{2.438053in}}%
\pgfpathlineto{\pgfqpoint{3.652836in}{2.524938in}}%
\pgfpathclose%
\pgfusepath{fill}%
\end{pgfscope}%
\begin{pgfscope}%
\pgfpathrectangle{\pgfqpoint{1.072000in}{0.528000in}}{\pgfqpoint{3.696000in}{3.696000in}}%
\pgfusepath{clip}%
\pgfsetbuttcap%
\pgfsetroundjoin%
\definecolor{currentfill}{rgb}{0.388852,0.516298,0.921373}%
\pgfsetfillcolor{currentfill}%
\pgfsetlinewidth{0.000000pt}%
\definecolor{currentstroke}{rgb}{0.000000,0.000000,0.000000}%
\pgfsetstrokecolor{currentstroke}%
\pgfsetdash{}{0pt}%
\pgfpathmoveto{\pgfqpoint{3.775499in}{1.838676in}}%
\pgfpathlineto{\pgfqpoint{3.819404in}{1.702948in}}%
\pgfpathlineto{\pgfqpoint{3.843579in}{1.705246in}}%
\pgfpathlineto{\pgfqpoint{3.799635in}{1.831889in}}%
\pgfpathlineto{\pgfqpoint{3.775499in}{1.838676in}}%
\pgfpathclose%
\pgfusepath{fill}%
\end{pgfscope}%
\begin{pgfscope}%
\pgfpathrectangle{\pgfqpoint{1.072000in}{0.528000in}}{\pgfqpoint{3.696000in}{3.696000in}}%
\pgfusepath{clip}%
\pgfsetbuttcap%
\pgfsetroundjoin%
\definecolor{currentfill}{rgb}{0.229806,0.298718,0.753683}%
\pgfsetfillcolor{currentfill}%
\pgfsetlinewidth{0.000000pt}%
\definecolor{currentstroke}{rgb}{0.000000,0.000000,0.000000}%
\pgfsetstrokecolor{currentstroke}%
\pgfsetdash{}{0pt}%
\pgfpathmoveto{\pgfqpoint{3.885049in}{1.519817in}}%
\pgfpathlineto{\pgfqpoint{3.931734in}{1.494855in}}%
\pgfpathlineto{\pgfqpoint{3.956339in}{1.521477in}}%
\pgfpathlineto{\pgfqpoint{3.909558in}{1.540439in}}%
\pgfpathlineto{\pgfqpoint{3.885049in}{1.519817in}}%
\pgfpathclose%
\pgfusepath{fill}%
\end{pgfscope}%
\begin{pgfscope}%
\pgfpathrectangle{\pgfqpoint{1.072000in}{0.528000in}}{\pgfqpoint{3.696000in}{3.696000in}}%
\pgfusepath{clip}%
\pgfsetbuttcap%
\pgfsetroundjoin%
\definecolor{currentfill}{rgb}{0.969289,0.684982,0.568975}%
\pgfsetfillcolor{currentfill}%
\pgfsetlinewidth{0.000000pt}%
\definecolor{currentstroke}{rgb}{0.000000,0.000000,0.000000}%
\pgfsetstrokecolor{currentstroke}%
\pgfsetdash{}{0pt}%
\pgfpathmoveto{\pgfqpoint{3.475637in}{2.965785in}}%
\pgfpathlineto{\pgfqpoint{3.520976in}{2.868293in}}%
\pgfpathlineto{\pgfqpoint{3.543954in}{2.743270in}}%
\pgfpathlineto{\pgfqpoint{3.498940in}{2.840998in}}%
\pgfpathlineto{\pgfqpoint{3.475637in}{2.965785in}}%
\pgfpathclose%
\pgfusepath{fill}%
\end{pgfscope}%
\begin{pgfscope}%
\pgfpathrectangle{\pgfqpoint{1.072000in}{0.528000in}}{\pgfqpoint{3.696000in}{3.696000in}}%
\pgfusepath{clip}%
\pgfsetbuttcap%
\pgfsetroundjoin%
\definecolor{currentfill}{rgb}{0.928116,0.822197,0.765141}%
\pgfsetfillcolor{currentfill}%
\pgfsetlinewidth{0.000000pt}%
\definecolor{currentstroke}{rgb}{0.000000,0.000000,0.000000}%
\pgfsetstrokecolor{currentstroke}%
\pgfsetdash{}{0pt}%
\pgfpathmoveto{\pgfqpoint{3.587527in}{2.734716in}}%
\pgfpathlineto{\pgfqpoint{3.631865in}{2.590939in}}%
\pgfpathlineto{\pgfqpoint{3.654548in}{2.488283in}}%
\pgfpathlineto{\pgfqpoint{3.610435in}{2.626543in}}%
\pgfpathlineto{\pgfqpoint{3.587527in}{2.734716in}}%
\pgfpathclose%
\pgfusepath{fill}%
\end{pgfscope}%
\begin{pgfscope}%
\pgfpathrectangle{\pgfqpoint{1.072000in}{0.528000in}}{\pgfqpoint{3.696000in}{3.696000in}}%
\pgfusepath{clip}%
\pgfsetbuttcap%
\pgfsetroundjoin%
\definecolor{currentfill}{rgb}{0.698454,0.799450,0.984577}%
\pgfsetfillcolor{currentfill}%
\pgfsetlinewidth{0.000000pt}%
\definecolor{currentstroke}{rgb}{0.000000,0.000000,0.000000}%
\pgfsetstrokecolor{currentstroke}%
\pgfsetdash{}{0pt}%
\pgfpathmoveto{\pgfqpoint{2.245933in}{2.080271in}}%
\pgfpathlineto{\pgfqpoint{2.288139in}{2.205504in}}%
\pgfpathlineto{\pgfqpoint{2.314072in}{2.329576in}}%
\pgfpathlineto{\pgfqpoint{2.271930in}{2.191888in}}%
\pgfpathlineto{\pgfqpoint{2.245933in}{2.080271in}}%
\pgfpathclose%
\pgfusepath{fill}%
\end{pgfscope}%
\begin{pgfscope}%
\pgfpathrectangle{\pgfqpoint{1.072000in}{0.528000in}}{\pgfqpoint{3.696000in}{3.696000in}}%
\pgfusepath{clip}%
\pgfsetbuttcap%
\pgfsetroundjoin%
\definecolor{currentfill}{rgb}{0.877149,0.394645,0.311724}%
\pgfsetfillcolor{currentfill}%
\pgfsetlinewidth{0.000000pt}%
\definecolor{currentstroke}{rgb}{0.000000,0.000000,0.000000}%
\pgfsetstrokecolor{currentstroke}%
\pgfsetdash{}{0pt}%
\pgfpathmoveto{\pgfqpoint{2.771545in}{3.166964in}}%
\pgfpathlineto{\pgfqpoint{2.816823in}{3.252762in}}%
\pgfpathlineto{\pgfqpoint{2.843917in}{3.203623in}}%
\pgfpathlineto{\pgfqpoint{2.798760in}{3.108808in}}%
\pgfpathlineto{\pgfqpoint{2.771545in}{3.166964in}}%
\pgfpathclose%
\pgfusepath{fill}%
\end{pgfscope}%
\begin{pgfscope}%
\pgfpathrectangle{\pgfqpoint{1.072000in}{0.528000in}}{\pgfqpoint{3.696000in}{3.696000in}}%
\pgfusepath{clip}%
\pgfsetbuttcap%
\pgfsetroundjoin%
\definecolor{currentfill}{rgb}{0.280550,0.373423,0.818011}%
\pgfsetfillcolor{currentfill}%
\pgfsetlinewidth{0.000000pt}%
\definecolor{currentstroke}{rgb}{0.000000,0.000000,0.000000}%
\pgfsetstrokecolor{currentstroke}%
\pgfsetdash{}{0pt}%
\pgfpathmoveto{\pgfqpoint{2.131454in}{1.550220in}}%
\pgfpathlineto{\pgfqpoint{2.175548in}{1.571206in}}%
\pgfpathlineto{\pgfqpoint{2.202003in}{1.665941in}}%
\pgfpathlineto{\pgfqpoint{2.158367in}{1.623097in}}%
\pgfpathlineto{\pgfqpoint{2.131454in}{1.550220in}}%
\pgfpathclose%
\pgfusepath{fill}%
\end{pgfscope}%
\begin{pgfscope}%
\pgfpathrectangle{\pgfqpoint{1.072000in}{0.528000in}}{\pgfqpoint{3.696000in}{3.696000in}}%
\pgfusepath{clip}%
\pgfsetbuttcap%
\pgfsetroundjoin%
\definecolor{currentfill}{rgb}{0.343278,0.459354,0.884122}%
\pgfsetfillcolor{currentfill}%
\pgfsetlinewidth{0.000000pt}%
\definecolor{currentstroke}{rgb}{0.000000,0.000000,0.000000}%
\pgfsetstrokecolor{currentstroke}%
\pgfsetdash{}{0pt}%
\pgfpathmoveto{\pgfqpoint{2.158367in}{1.623097in}}%
\pgfpathlineto{\pgfqpoint{2.202003in}{1.665941in}}%
\pgfpathlineto{\pgfqpoint{2.228011in}{1.781373in}}%
\pgfpathlineto{\pgfqpoint{2.184755in}{1.717272in}}%
\pgfpathlineto{\pgfqpoint{2.158367in}{1.623097in}}%
\pgfpathclose%
\pgfusepath{fill}%
\end{pgfscope}%
\begin{pgfscope}%
\pgfpathrectangle{\pgfqpoint{1.072000in}{0.528000in}}{\pgfqpoint{3.696000in}{3.696000in}}%
\pgfusepath{clip}%
\pgfsetbuttcap%
\pgfsetroundjoin%
\definecolor{currentfill}{rgb}{0.947345,0.794696,0.716991}%
\pgfsetfillcolor{currentfill}%
\pgfsetlinewidth{0.000000pt}%
\definecolor{currentstroke}{rgb}{0.000000,0.000000,0.000000}%
\pgfsetstrokecolor{currentstroke}%
\pgfsetdash{}{0pt}%
\pgfpathmoveto{\pgfqpoint{2.377640in}{2.556555in}}%
\pgfpathlineto{\pgfqpoint{2.420017in}{2.718487in}}%
\pgfpathlineto{\pgfqpoint{2.447062in}{2.779743in}}%
\pgfpathlineto{\pgfqpoint{2.404577in}{2.616289in}}%
\pgfpathlineto{\pgfqpoint{2.377640in}{2.556555in}}%
\pgfpathclose%
\pgfusepath{fill}%
\end{pgfscope}%
\begin{pgfscope}%
\pgfpathrectangle{\pgfqpoint{1.072000in}{0.528000in}}{\pgfqpoint{3.696000in}{3.696000in}}%
\pgfusepath{clip}%
\pgfsetbuttcap%
\pgfsetroundjoin%
\definecolor{currentfill}{rgb}{0.313946,0.420052,0.854993}%
\pgfsetfillcolor{currentfill}%
\pgfsetlinewidth{0.000000pt}%
\definecolor{currentstroke}{rgb}{0.000000,0.000000,0.000000}%
\pgfsetstrokecolor{currentstroke}%
\pgfsetdash{}{0pt}%
\pgfpathmoveto{\pgfqpoint{3.794977in}{1.695519in}}%
\pgfpathlineto{\pgfqpoint{3.839548in}{1.587736in}}%
\pgfpathlineto{\pgfqpoint{3.863996in}{1.601868in}}%
\pgfpathlineto{\pgfqpoint{3.819404in}{1.702948in}}%
\pgfpathlineto{\pgfqpoint{3.794977in}{1.695519in}}%
\pgfpathclose%
\pgfusepath{fill}%
\end{pgfscope}%
\begin{pgfscope}%
\pgfpathrectangle{\pgfqpoint{1.072000in}{0.528000in}}{\pgfqpoint{3.696000in}{3.696000in}}%
\pgfusepath{clip}%
\pgfsetbuttcap%
\pgfsetroundjoin%
\definecolor{currentfill}{rgb}{0.895885,0.433075,0.338681}%
\pgfsetfillcolor{currentfill}%
\pgfsetlinewidth{0.000000pt}%
\definecolor{currentstroke}{rgb}{0.000000,0.000000,0.000000}%
\pgfsetstrokecolor{currentstroke}%
\pgfsetdash{}{0pt}%
\pgfpathmoveto{\pgfqpoint{3.245276in}{3.207980in}}%
\pgfpathlineto{\pgfqpoint{3.291592in}{3.189008in}}%
\pgfpathlineto{\pgfqpoint{3.316160in}{3.075585in}}%
\pgfpathlineto{\pgfqpoint{3.270097in}{3.095494in}}%
\pgfpathlineto{\pgfqpoint{3.245276in}{3.207980in}}%
\pgfpathclose%
\pgfusepath{fill}%
\end{pgfscope}%
\begin{pgfscope}%
\pgfpathrectangle{\pgfqpoint{1.072000in}{0.528000in}}{\pgfqpoint{3.696000in}{3.696000in}}%
\pgfusepath{clip}%
\pgfsetbuttcap%
\pgfsetroundjoin%
\definecolor{currentfill}{rgb}{0.280550,0.373423,0.818011}%
\pgfsetfillcolor{currentfill}%
\pgfsetlinewidth{0.000000pt}%
\definecolor{currentstroke}{rgb}{0.000000,0.000000,0.000000}%
\pgfsetstrokecolor{currentstroke}%
\pgfsetdash{}{0pt}%
\pgfpathmoveto{\pgfqpoint{1.937380in}{1.645168in}}%
\pgfpathlineto{\pgfqpoint{1.983918in}{1.573529in}}%
\pgfpathlineto{\pgfqpoint{2.013676in}{1.552002in}}%
\pgfpathlineto{\pgfqpoint{1.967852in}{1.605131in}}%
\pgfpathlineto{\pgfqpoint{1.937380in}{1.645168in}}%
\pgfpathclose%
\pgfusepath{fill}%
\end{pgfscope}%
\begin{pgfscope}%
\pgfpathrectangle{\pgfqpoint{1.072000in}{0.528000in}}{\pgfqpoint{3.696000in}{3.696000in}}%
\pgfusepath{clip}%
\pgfsetbuttcap%
\pgfsetroundjoin%
\definecolor{currentfill}{rgb}{0.248091,0.326013,0.777669}%
\pgfsetfillcolor{currentfill}%
\pgfsetlinewidth{0.000000pt}%
\definecolor{currentstroke}{rgb}{0.000000,0.000000,0.000000}%
\pgfsetstrokecolor{currentstroke}%
\pgfsetdash{}{0pt}%
\pgfpathmoveto{\pgfqpoint{1.983918in}{1.573529in}}%
\pgfpathlineto{\pgfqpoint{2.029897in}{1.520133in}}%
\pgfpathlineto{\pgfqpoint{2.058968in}{1.518522in}}%
\pgfpathlineto{\pgfqpoint{2.013676in}{1.552002in}}%
\pgfpathlineto{\pgfqpoint{1.983918in}{1.573529in}}%
\pgfpathclose%
\pgfusepath{fill}%
\end{pgfscope}%
\begin{pgfscope}%
\pgfpathrectangle{\pgfqpoint{1.072000in}{0.528000in}}{\pgfqpoint{3.696000in}{3.696000in}}%
\pgfusepath{clip}%
\pgfsetbuttcap%
\pgfsetroundjoin%
\definecolor{currentfill}{rgb}{0.777378,0.840921,0.946149}%
\pgfsetfillcolor{currentfill}%
\pgfsetlinewidth{0.000000pt}%
\definecolor{currentstroke}{rgb}{0.000000,0.000000,0.000000}%
\pgfsetstrokecolor{currentstroke}%
\pgfsetdash{}{0pt}%
\pgfpathmoveto{\pgfqpoint{2.271930in}{2.191888in}}%
\pgfpathlineto{\pgfqpoint{2.314072in}{2.329576in}}%
\pgfpathlineto{\pgfqpoint{2.340204in}{2.445566in}}%
\pgfpathlineto{\pgfqpoint{2.298063in}{2.298225in}}%
\pgfpathlineto{\pgfqpoint{2.271930in}{2.191888in}}%
\pgfpathclose%
\pgfusepath{fill}%
\end{pgfscope}%
\begin{pgfscope}%
\pgfpathrectangle{\pgfqpoint{1.072000in}{0.528000in}}{\pgfqpoint{3.696000in}{3.696000in}}%
\pgfusepath{clip}%
\pgfsetbuttcap%
\pgfsetroundjoin%
\definecolor{currentfill}{rgb}{0.243520,0.319189,0.771672}%
\pgfsetfillcolor{currentfill}%
\pgfsetlinewidth{0.000000pt}%
\definecolor{currentstroke}{rgb}{0.000000,0.000000,0.000000}%
\pgfsetstrokecolor{currentstroke}%
\pgfsetdash{}{0pt}%
\pgfpathmoveto{\pgfqpoint{2.103856in}{1.502726in}}%
\pgfpathlineto{\pgfqpoint{2.148475in}{1.502007in}}%
\pgfpathlineto{\pgfqpoint{2.175548in}{1.571206in}}%
\pgfpathlineto{\pgfqpoint{2.131454in}{1.550220in}}%
\pgfpathlineto{\pgfqpoint{2.103856in}{1.502726in}}%
\pgfpathclose%
\pgfusepath{fill}%
\end{pgfscope}%
\begin{pgfscope}%
\pgfpathrectangle{\pgfqpoint{1.072000in}{0.528000in}}{\pgfqpoint{3.696000in}{3.696000in}}%
\pgfusepath{clip}%
\pgfsetbuttcap%
\pgfsetroundjoin%
\definecolor{currentfill}{rgb}{0.425199,0.559058,0.946061}%
\pgfsetfillcolor{currentfill}%
\pgfsetlinewidth{0.000000pt}%
\definecolor{currentstroke}{rgb}{0.000000,0.000000,0.000000}%
\pgfsetstrokecolor{currentstroke}%
\pgfsetdash{}{0pt}%
\pgfpathmoveto{\pgfqpoint{2.184755in}{1.717272in}}%
\pgfpathlineto{\pgfqpoint{2.228011in}{1.781373in}}%
\pgfpathlineto{\pgfqpoint{2.253740in}{1.911929in}}%
\pgfpathlineto{\pgfqpoint{2.210780in}{1.827908in}}%
\pgfpathlineto{\pgfqpoint{2.184755in}{1.717272in}}%
\pgfpathclose%
\pgfusepath{fill}%
\end{pgfscope}%
\begin{pgfscope}%
\pgfpathrectangle{\pgfqpoint{1.072000in}{0.528000in}}{\pgfqpoint{3.696000in}{3.696000in}}%
\pgfusepath{clip}%
\pgfsetbuttcap%
\pgfsetroundjoin%
\definecolor{currentfill}{rgb}{0.959385,0.610306,0.489382}%
\pgfsetfillcolor{currentfill}%
\pgfsetlinewidth{0.000000pt}%
\definecolor{currentstroke}{rgb}{0.000000,0.000000,0.000000}%
\pgfsetstrokecolor{currentstroke}%
\pgfsetdash{}{0pt}%
\pgfpathmoveto{\pgfqpoint{3.429955in}{3.046004in}}%
\pgfpathlineto{\pgfqpoint{3.475637in}{2.965785in}}%
\pgfpathlineto{\pgfqpoint{3.498940in}{2.840998in}}%
\pgfpathlineto{\pgfqpoint{3.453596in}{2.923462in}}%
\pgfpathlineto{\pgfqpoint{3.429955in}{3.046004in}}%
\pgfpathclose%
\pgfusepath{fill}%
\end{pgfscope}%
\begin{pgfscope}%
\pgfpathrectangle{\pgfqpoint{1.072000in}{0.528000in}}{\pgfqpoint{3.696000in}{3.696000in}}%
\pgfusepath{clip}%
\pgfsetbuttcap%
\pgfsetroundjoin%
\definecolor{currentfill}{rgb}{0.928116,0.822197,0.765141}%
\pgfsetfillcolor{currentfill}%
\pgfsetlinewidth{0.000000pt}%
\definecolor{currentstroke}{rgb}{0.000000,0.000000,0.000000}%
\pgfsetstrokecolor{currentstroke}%
\pgfsetdash{}{0pt}%
\pgfpathmoveto{\pgfqpoint{2.350903in}{2.482733in}}%
\pgfpathlineto{\pgfqpoint{2.393178in}{2.641799in}}%
\pgfpathlineto{\pgfqpoint{2.420017in}{2.718487in}}%
\pgfpathlineto{\pgfqpoint{2.377640in}{2.556555in}}%
\pgfpathlineto{\pgfqpoint{2.350903in}{2.482733in}}%
\pgfpathclose%
\pgfusepath{fill}%
\end{pgfscope}%
\begin{pgfscope}%
\pgfpathrectangle{\pgfqpoint{1.072000in}{0.528000in}}{\pgfqpoint{3.696000in}{3.696000in}}%
\pgfusepath{clip}%
\pgfsetbuttcap%
\pgfsetroundjoin%
\definecolor{currentfill}{rgb}{0.839351,0.861167,0.894494}%
\pgfsetfillcolor{currentfill}%
\pgfsetlinewidth{0.000000pt}%
\definecolor{currentstroke}{rgb}{0.000000,0.000000,0.000000}%
\pgfsetstrokecolor{currentstroke}%
\pgfsetdash{}{0pt}%
\pgfpathmoveto{\pgfqpoint{2.298063in}{2.298225in}}%
\pgfpathlineto{\pgfqpoint{2.340204in}{2.445566in}}%
\pgfpathlineto{\pgfqpoint{2.366570in}{2.550372in}}%
\pgfpathlineto{\pgfqpoint{2.324379in}{2.396017in}}%
\pgfpathlineto{\pgfqpoint{2.298063in}{2.298225in}}%
\pgfpathclose%
\pgfusepath{fill}%
\end{pgfscope}%
\begin{pgfscope}%
\pgfpathrectangle{\pgfqpoint{1.072000in}{0.528000in}}{\pgfqpoint{3.696000in}{3.696000in}}%
\pgfusepath{clip}%
\pgfsetbuttcap%
\pgfsetroundjoin%
\definecolor{currentfill}{rgb}{0.229806,0.298718,0.753683}%
\pgfsetfillcolor{currentfill}%
\pgfsetlinewidth{0.000000pt}%
\definecolor{currentstroke}{rgb}{0.000000,0.000000,0.000000}%
\pgfsetstrokecolor{currentstroke}%
\pgfsetdash{}{0pt}%
\pgfpathmoveto{\pgfqpoint{2.029897in}{1.520133in}}%
\pgfpathlineto{\pgfqpoint{2.075426in}{1.483663in}}%
\pgfpathlineto{\pgfqpoint{2.103856in}{1.502726in}}%
\pgfpathlineto{\pgfqpoint{2.058968in}{1.518522in}}%
\pgfpathlineto{\pgfqpoint{2.029897in}{1.520133in}}%
\pgfpathclose%
\pgfusepath{fill}%
\end{pgfscope}%
\begin{pgfscope}%
\pgfpathrectangle{\pgfqpoint{1.072000in}{0.528000in}}{\pgfqpoint{3.696000in}{3.696000in}}%
\pgfusepath{clip}%
\pgfsetbuttcap%
\pgfsetroundjoin%
\definecolor{currentfill}{rgb}{0.333490,0.446265,0.874452}%
\pgfsetfillcolor{currentfill}%
\pgfsetlinewidth{0.000000pt}%
\definecolor{currentstroke}{rgb}{0.000000,0.000000,0.000000}%
\pgfsetstrokecolor{currentstroke}%
\pgfsetdash{}{0pt}%
\pgfpathmoveto{\pgfqpoint{1.890197in}{1.735553in}}%
\pgfpathlineto{\pgfqpoint{1.937380in}{1.645168in}}%
\pgfpathlineto{\pgfqpoint{1.967852in}{1.605131in}}%
\pgfpathlineto{\pgfqpoint{1.921384in}{1.679051in}}%
\pgfpathlineto{\pgfqpoint{1.890197in}{1.735553in}}%
\pgfpathclose%
\pgfusepath{fill}%
\end{pgfscope}%
\begin{pgfscope}%
\pgfpathrectangle{\pgfqpoint{1.072000in}{0.528000in}}{\pgfqpoint{3.696000in}{3.696000in}}%
\pgfusepath{clip}%
\pgfsetbuttcap%
\pgfsetroundjoin%
\definecolor{currentfill}{rgb}{0.891817,0.851973,0.829085}%
\pgfsetfillcolor{currentfill}%
\pgfsetlinewidth{0.000000pt}%
\definecolor{currentstroke}{rgb}{0.000000,0.000000,0.000000}%
\pgfsetstrokecolor{currentstroke}%
\pgfsetdash{}{0pt}%
\pgfpathmoveto{\pgfqpoint{2.324379in}{2.396017in}}%
\pgfpathlineto{\pgfqpoint{2.366570in}{2.550372in}}%
\pgfpathlineto{\pgfqpoint{2.393178in}{2.641799in}}%
\pgfpathlineto{\pgfqpoint{2.350903in}{2.482733in}}%
\pgfpathlineto{\pgfqpoint{2.324379in}{2.396017in}}%
\pgfpathclose%
\pgfusepath{fill}%
\end{pgfscope}%
\begin{pgfscope}%
\pgfpathrectangle{\pgfqpoint{1.072000in}{0.528000in}}{\pgfqpoint{3.696000in}{3.696000in}}%
\pgfusepath{clip}%
\pgfsetbuttcap%
\pgfsetroundjoin%
\definecolor{currentfill}{rgb}{0.905783,0.455186,0.355336}%
\pgfsetfillcolor{currentfill}%
\pgfsetlinewidth{0.000000pt}%
\definecolor{currentstroke}{rgb}{0.000000,0.000000,0.000000}%
\pgfsetstrokecolor{currentstroke}%
\pgfsetdash{}{0pt}%
\pgfpathmoveto{\pgfqpoint{3.291592in}{3.189008in}}%
\pgfpathlineto{\pgfqpoint{3.337856in}{3.156440in}}%
\pgfpathlineto{\pgfqpoint{3.362141in}{3.040551in}}%
\pgfpathlineto{\pgfqpoint{3.316160in}{3.075585in}}%
\pgfpathlineto{\pgfqpoint{3.291592in}{3.189008in}}%
\pgfpathclose%
\pgfusepath{fill}%
\end{pgfscope}%
\begin{pgfscope}%
\pgfpathrectangle{\pgfqpoint{1.072000in}{0.528000in}}{\pgfqpoint{3.696000in}{3.696000in}}%
\pgfusepath{clip}%
\pgfsetbuttcap%
\pgfsetroundjoin%
\definecolor{currentfill}{rgb}{0.777378,0.840921,0.946149}%
\pgfsetfillcolor{currentfill}%
\pgfsetlinewidth{0.000000pt}%
\definecolor{currentstroke}{rgb}{0.000000,0.000000,0.000000}%
\pgfsetstrokecolor{currentstroke}%
\pgfsetdash{}{0pt}%
\pgfpathmoveto{\pgfqpoint{3.673309in}{2.429414in}}%
\pgfpathlineto{\pgfqpoint{3.716797in}{2.255100in}}%
\pgfpathlineto{\pgfqpoint{3.739927in}{2.195240in}}%
\pgfpathlineto{\pgfqpoint{3.696524in}{2.359619in}}%
\pgfpathlineto{\pgfqpoint{3.673309in}{2.429414in}}%
\pgfpathclose%
\pgfusepath{fill}%
\end{pgfscope}%
\begin{pgfscope}%
\pgfpathrectangle{\pgfqpoint{1.072000in}{0.528000in}}{\pgfqpoint{3.696000in}{3.696000in}}%
\pgfusepath{clip}%
\pgfsetbuttcap%
\pgfsetroundjoin%
\definecolor{currentfill}{rgb}{0.505423,0.643995,0.983157}%
\pgfsetfillcolor{currentfill}%
\pgfsetlinewidth{0.000000pt}%
\definecolor{currentstroke}{rgb}{0.000000,0.000000,0.000000}%
\pgfsetstrokecolor{currentstroke}%
\pgfsetdash{}{0pt}%
\pgfpathmoveto{\pgfqpoint{1.777442in}{2.021573in}}%
\pgfpathlineto{\pgfqpoint{1.826206in}{1.888804in}}%
\pgfpathlineto{\pgfqpoint{1.857898in}{1.823532in}}%
\pgfpathlineto{\pgfqpoint{1.809749in}{1.946260in}}%
\pgfpathlineto{\pgfqpoint{1.777442in}{2.021573in}}%
\pgfpathclose%
\pgfusepath{fill}%
\end{pgfscope}%
\begin{pgfscope}%
\pgfpathrectangle{\pgfqpoint{1.072000in}{0.528000in}}{\pgfqpoint{3.696000in}{3.696000in}}%
\pgfusepath{clip}%
\pgfsetbuttcap%
\pgfsetroundjoin%
\definecolor{currentfill}{rgb}{0.839365,0.321856,0.264924}%
\pgfsetfillcolor{currentfill}%
\pgfsetlinewidth{0.000000pt}%
\definecolor{currentstroke}{rgb}{0.000000,0.000000,0.000000}%
\pgfsetstrokecolor{currentstroke}%
\pgfsetdash{}{0pt}%
\pgfpathmoveto{\pgfqpoint{3.080695in}{3.273434in}}%
\pgfpathlineto{\pgfqpoint{3.127003in}{3.289309in}}%
\pgfpathlineto{\pgfqpoint{3.152708in}{3.206045in}}%
\pgfpathlineto{\pgfqpoint{3.106553in}{3.183785in}}%
\pgfpathlineto{\pgfqpoint{3.080695in}{3.273434in}}%
\pgfpathclose%
\pgfusepath{fill}%
\end{pgfscope}%
\begin{pgfscope}%
\pgfpathrectangle{\pgfqpoint{1.072000in}{0.528000in}}{\pgfqpoint{3.696000in}{3.696000in}}%
\pgfusepath{clip}%
\pgfsetbuttcap%
\pgfsetroundjoin%
\definecolor{currentfill}{rgb}{0.830187,0.304733,0.254891}%
\pgfsetfillcolor{currentfill}%
\pgfsetlinewidth{0.000000pt}%
\definecolor{currentstroke}{rgb}{0.000000,0.000000,0.000000}%
\pgfsetstrokecolor{currentstroke}%
\pgfsetdash{}{0pt}%
\pgfpathmoveto{\pgfqpoint{2.962135in}{3.270803in}}%
\pgfpathlineto{\pgfqpoint{3.008213in}{3.308212in}}%
\pgfpathlineto{\pgfqpoint{3.034540in}{3.242475in}}%
\pgfpathlineto{\pgfqpoint{2.988593in}{3.194113in}}%
\pgfpathlineto{\pgfqpoint{2.962135in}{3.270803in}}%
\pgfpathclose%
\pgfusepath{fill}%
\end{pgfscope}%
\begin{pgfscope}%
\pgfpathrectangle{\pgfqpoint{1.072000in}{0.528000in}}{\pgfqpoint{3.696000in}{3.696000in}}%
\pgfusepath{clip}%
\pgfsetbuttcap%
\pgfsetroundjoin%
\definecolor{currentfill}{rgb}{0.939254,0.539581,0.423900}%
\pgfsetfillcolor{currentfill}%
\pgfsetlinewidth{0.000000pt}%
\definecolor{currentstroke}{rgb}{0.000000,0.000000,0.000000}%
\pgfsetstrokecolor{currentstroke}%
\pgfsetdash{}{0pt}%
\pgfpathmoveto{\pgfqpoint{3.384002in}{3.109209in}}%
\pgfpathlineto{\pgfqpoint{3.429955in}{3.046004in}}%
\pgfpathlineto{\pgfqpoint{3.453596in}{2.923462in}}%
\pgfpathlineto{\pgfqpoint{3.407975in}{2.989973in}}%
\pgfpathlineto{\pgfqpoint{3.384002in}{3.109209in}}%
\pgfpathclose%
\pgfusepath{fill}%
\end{pgfscope}%
\begin{pgfscope}%
\pgfpathrectangle{\pgfqpoint{1.072000in}{0.528000in}}{\pgfqpoint{3.696000in}{3.696000in}}%
\pgfusepath{clip}%
\pgfsetbuttcap%
\pgfsetroundjoin%
\definecolor{currentfill}{rgb}{0.959385,0.610306,0.489382}%
\pgfsetfillcolor{currentfill}%
\pgfsetlinewidth{0.000000pt}%
\definecolor{currentstroke}{rgb}{0.000000,0.000000,0.000000}%
\pgfsetstrokecolor{currentstroke}%
\pgfsetdash{}{0pt}%
\pgfpathmoveto{\pgfqpoint{2.501634in}{2.855227in}}%
\pgfpathlineto{\pgfqpoint{2.545063in}{3.003676in}}%
\pgfpathlineto{\pgfqpoint{2.572573in}{3.018056in}}%
\pgfpathlineto{\pgfqpoint{2.529086in}{2.869503in}}%
\pgfpathlineto{\pgfqpoint{2.501634in}{2.855227in}}%
\pgfpathclose%
\pgfusepath{fill}%
\end{pgfscope}%
\begin{pgfscope}%
\pgfpathrectangle{\pgfqpoint{1.072000in}{0.528000in}}{\pgfqpoint{3.696000in}{3.696000in}}%
\pgfusepath{clip}%
\pgfsetbuttcap%
\pgfsetroundjoin%
\definecolor{currentfill}{rgb}{0.921406,0.491420,0.383408}%
\pgfsetfillcolor{currentfill}%
\pgfsetlinewidth{0.000000pt}%
\definecolor{currentstroke}{rgb}{0.000000,0.000000,0.000000}%
\pgfsetstrokecolor{currentstroke}%
\pgfsetdash{}{0pt}%
\pgfpathmoveto{\pgfqpoint{3.337856in}{3.156440in}}%
\pgfpathlineto{\pgfqpoint{3.384002in}{3.109209in}}%
\pgfpathlineto{\pgfqpoint{3.407975in}{2.989973in}}%
\pgfpathlineto{\pgfqpoint{3.362141in}{3.040551in}}%
\pgfpathlineto{\pgfqpoint{3.337856in}{3.156440in}}%
\pgfpathclose%
\pgfusepath{fill}%
\end{pgfscope}%
\begin{pgfscope}%
\pgfpathrectangle{\pgfqpoint{1.072000in}{0.528000in}}{\pgfqpoint{3.696000in}{3.696000in}}%
\pgfusepath{clip}%
\pgfsetbuttcap%
\pgfsetroundjoin%
\definecolor{currentfill}{rgb}{0.257234,0.339661,0.789661}%
\pgfsetfillcolor{currentfill}%
\pgfsetlinewidth{0.000000pt}%
\definecolor{currentstroke}{rgb}{0.000000,0.000000,0.000000}%
\pgfsetstrokecolor{currentstroke}%
\pgfsetdash{}{0pt}%
\pgfpathmoveto{\pgfqpoint{3.814895in}{1.569809in}}%
\pgfpathlineto{\pgfqpoint{3.860404in}{1.497890in}}%
\pgfpathlineto{\pgfqpoint{3.885049in}{1.519817in}}%
\pgfpathlineto{\pgfqpoint{3.839548in}{1.587736in}}%
\pgfpathlineto{\pgfqpoint{3.814895in}{1.569809in}}%
\pgfpathclose%
\pgfusepath{fill}%
\end{pgfscope}%
\begin{pgfscope}%
\pgfpathrectangle{\pgfqpoint{1.072000in}{0.528000in}}{\pgfqpoint{3.696000in}{3.696000in}}%
\pgfusepath{clip}%
\pgfsetbuttcap%
\pgfsetroundjoin%
\definecolor{currentfill}{rgb}{0.656683,0.771806,0.994914}%
\pgfsetfillcolor{currentfill}%
\pgfsetlinewidth{0.000000pt}%
\definecolor{currentstroke}{rgb}{0.000000,0.000000,0.000000}%
\pgfsetstrokecolor{currentstroke}%
\pgfsetdash{}{0pt}%
\pgfpathmoveto{\pgfqpoint{1.711041in}{2.242789in}}%
\pgfpathlineto{\pgfqpoint{1.760768in}{2.087414in}}%
\pgfpathlineto{\pgfqpoint{1.792902in}{2.017187in}}%
\pgfpathlineto{\pgfqpoint{1.743648in}{2.167640in}}%
\pgfpathlineto{\pgfqpoint{1.711041in}{2.242789in}}%
\pgfpathclose%
\pgfusepath{fill}%
\end{pgfscope}%
\begin{pgfscope}%
\pgfpathrectangle{\pgfqpoint{1.072000in}{0.528000in}}{\pgfqpoint{3.696000in}{3.696000in}}%
\pgfusepath{clip}%
\pgfsetbuttcap%
\pgfsetroundjoin%
\definecolor{currentfill}{rgb}{0.921406,0.491420,0.383408}%
\pgfsetfillcolor{currentfill}%
\pgfsetlinewidth{0.000000pt}%
\definecolor{currentstroke}{rgb}{0.000000,0.000000,0.000000}%
\pgfsetstrokecolor{currentstroke}%
\pgfsetdash{}{0pt}%
\pgfpathmoveto{\pgfqpoint{2.600121in}{3.017262in}}%
\pgfpathlineto{\pgfqpoint{2.644382in}{3.142413in}}%
\pgfpathlineto{\pgfqpoint{2.671908in}{3.129293in}}%
\pgfpathlineto{\pgfqpoint{2.627675in}{3.000975in}}%
\pgfpathlineto{\pgfqpoint{2.600121in}{3.017262in}}%
\pgfpathclose%
\pgfusepath{fill}%
\end{pgfscope}%
\begin{pgfscope}%
\pgfpathrectangle{\pgfqpoint{1.072000in}{0.528000in}}{\pgfqpoint{3.696000in}{3.696000in}}%
\pgfusepath{clip}%
\pgfsetbuttcap%
\pgfsetroundjoin%
\definecolor{currentfill}{rgb}{0.229806,0.298718,0.753683}%
\pgfsetfillcolor{currentfill}%
\pgfsetlinewidth{0.000000pt}%
\definecolor{currentstroke}{rgb}{0.000000,0.000000,0.000000}%
\pgfsetstrokecolor{currentstroke}%
\pgfsetdash{}{0pt}%
\pgfpathmoveto{\pgfqpoint{3.860404in}{1.497890in}}%
\pgfpathlineto{\pgfqpoint{3.907065in}{1.469234in}}%
\pgfpathlineto{\pgfqpoint{3.931734in}{1.494855in}}%
\pgfpathlineto{\pgfqpoint{3.885049in}{1.519817in}}%
\pgfpathlineto{\pgfqpoint{3.860404in}{1.497890in}}%
\pgfpathclose%
\pgfusepath{fill}%
\end{pgfscope}%
\begin{pgfscope}%
\pgfpathrectangle{\pgfqpoint{1.072000in}{0.528000in}}{\pgfqpoint{3.696000in}{3.696000in}}%
\pgfusepath{clip}%
\pgfsetbuttcap%
\pgfsetroundjoin%
\definecolor{currentfill}{rgb}{0.521696,0.659599,0.987736}%
\pgfsetfillcolor{currentfill}%
\pgfsetlinewidth{0.000000pt}%
\definecolor{currentstroke}{rgb}{0.000000,0.000000,0.000000}%
\pgfsetstrokecolor{currentstroke}%
\pgfsetdash{}{0pt}%
\pgfpathmoveto{\pgfqpoint{2.210780in}{1.827908in}}%
\pgfpathlineto{\pgfqpoint{2.253740in}{1.911929in}}%
\pgfpathlineto{\pgfqpoint{2.279349in}{2.051663in}}%
\pgfpathlineto{\pgfqpoint{2.236597in}{1.949724in}}%
\pgfpathlineto{\pgfqpoint{2.210780in}{1.827908in}}%
\pgfpathclose%
\pgfusepath{fill}%
\end{pgfscope}%
\begin{pgfscope}%
\pgfpathrectangle{\pgfqpoint{1.072000in}{0.528000in}}{\pgfqpoint{3.696000in}{3.696000in}}%
\pgfusepath{clip}%
\pgfsetbuttcap%
\pgfsetroundjoin%
\definecolor{currentfill}{rgb}{0.708720,0.805721,0.981117}%
\pgfsetfillcolor{currentfill}%
\pgfsetlinewidth{0.000000pt}%
\definecolor{currentstroke}{rgb}{0.000000,0.000000,0.000000}%
\pgfsetstrokecolor{currentstroke}%
\pgfsetdash{}{0pt}%
\pgfpathmoveto{\pgfqpoint{3.693278in}{2.306811in}}%
\pgfpathlineto{\pgfqpoint{3.736632in}{2.128634in}}%
\pgfpathlineto{\pgfqpoint{3.760096in}{2.087003in}}%
\pgfpathlineto{\pgfqpoint{3.716797in}{2.255100in}}%
\pgfpathlineto{\pgfqpoint{3.693278in}{2.306811in}}%
\pgfpathclose%
\pgfusepath{fill}%
\end{pgfscope}%
\begin{pgfscope}%
\pgfpathrectangle{\pgfqpoint{1.072000in}{0.528000in}}{\pgfqpoint{3.696000in}{3.696000in}}%
\pgfusepath{clip}%
\pgfsetbuttcap%
\pgfsetroundjoin%
\definecolor{currentfill}{rgb}{0.229806,0.298718,0.753683}%
\pgfsetfillcolor{currentfill}%
\pgfsetlinewidth{0.000000pt}%
\definecolor{currentstroke}{rgb}{0.000000,0.000000,0.000000}%
\pgfsetstrokecolor{currentstroke}%
\pgfsetdash{}{0pt}%
\pgfpathmoveto{\pgfqpoint{2.075426in}{1.483663in}}%
\pgfpathlineto{\pgfqpoint{2.120622in}{1.462143in}}%
\pgfpathlineto{\pgfqpoint{2.148475in}{1.502007in}}%
\pgfpathlineto{\pgfqpoint{2.103856in}{1.502726in}}%
\pgfpathlineto{\pgfqpoint{2.075426in}{1.483663in}}%
\pgfpathclose%
\pgfusepath{fill}%
\end{pgfscope}%
\begin{pgfscope}%
\pgfpathrectangle{\pgfqpoint{1.072000in}{0.528000in}}{\pgfqpoint{3.696000in}{3.696000in}}%
\pgfusepath{clip}%
\pgfsetbuttcap%
\pgfsetroundjoin%
\definecolor{currentfill}{rgb}{0.404421,0.534643,0.932002}%
\pgfsetfillcolor{currentfill}%
\pgfsetlinewidth{0.000000pt}%
\definecolor{currentstroke}{rgb}{0.000000,0.000000,0.000000}%
\pgfsetstrokecolor{currentstroke}%
\pgfsetdash{}{0pt}%
\pgfpathmoveto{\pgfqpoint{1.842310in}{1.844241in}}%
\pgfpathlineto{\pgfqpoint{1.890197in}{1.735553in}}%
\pgfpathlineto{\pgfqpoint{1.921384in}{1.679051in}}%
\pgfpathlineto{\pgfqpoint{1.874185in}{1.773922in}}%
\pgfpathlineto{\pgfqpoint{1.842310in}{1.844241in}}%
\pgfpathclose%
\pgfusepath{fill}%
\end{pgfscope}%
\begin{pgfscope}%
\pgfpathrectangle{\pgfqpoint{1.072000in}{0.528000in}}{\pgfqpoint{3.696000in}{3.696000in}}%
\pgfusepath{clip}%
\pgfsetbuttcap%
\pgfsetroundjoin%
\definecolor{currentfill}{rgb}{0.965899,0.740142,0.637058}%
\pgfsetfillcolor{currentfill}%
\pgfsetlinewidth{0.000000pt}%
\definecolor{currentstroke}{rgb}{0.000000,0.000000,0.000000}%
\pgfsetstrokecolor{currentstroke}%
\pgfsetdash{}{0pt}%
\pgfpathmoveto{\pgfqpoint{3.542715in}{2.864381in}}%
\pgfpathlineto{\pgfqpoint{3.587527in}{2.734716in}}%
\pgfpathlineto{\pgfqpoint{3.610435in}{2.626543in}}%
\pgfpathlineto{\pgfqpoint{3.565916in}{2.754446in}}%
\pgfpathlineto{\pgfqpoint{3.542715in}{2.864381in}}%
\pgfpathclose%
\pgfusepath{fill}%
\end{pgfscope}%
\begin{pgfscope}%
\pgfpathrectangle{\pgfqpoint{1.072000in}{0.528000in}}{\pgfqpoint{3.696000in}{3.696000in}}%
\pgfusepath{clip}%
\pgfsetbuttcap%
\pgfsetroundjoin%
\definecolor{currentfill}{rgb}{0.409611,0.540759,0.935545}%
\pgfsetfillcolor{currentfill}%
\pgfsetlinewidth{0.000000pt}%
\definecolor{currentstroke}{rgb}{0.000000,0.000000,0.000000}%
\pgfsetstrokecolor{currentstroke}%
\pgfsetdash{}{0pt}%
\pgfpathmoveto{\pgfqpoint{3.751046in}{1.837837in}}%
\pgfpathlineto{\pgfqpoint{3.794977in}{1.695519in}}%
\pgfpathlineto{\pgfqpoint{3.819404in}{1.702948in}}%
\pgfpathlineto{\pgfqpoint{3.775499in}{1.838676in}}%
\pgfpathlineto{\pgfqpoint{3.751046in}{1.837837in}}%
\pgfpathclose%
\pgfusepath{fill}%
\end{pgfscope}%
\begin{pgfscope}%
\pgfpathrectangle{\pgfqpoint{1.072000in}{0.528000in}}{\pgfqpoint{3.696000in}{3.696000in}}%
\pgfusepath{clip}%
\pgfsetbuttcap%
\pgfsetroundjoin%
\definecolor{currentfill}{rgb}{0.510824,0.649397,0.985079}%
\pgfsetfillcolor{currentfill}%
\pgfsetlinewidth{0.000000pt}%
\definecolor{currentstroke}{rgb}{0.000000,0.000000,0.000000}%
\pgfsetstrokecolor{currentstroke}%
\pgfsetdash{}{0pt}%
\pgfpathmoveto{\pgfqpoint{3.731981in}{2.002074in}}%
\pgfpathlineto{\pgfqpoint{3.775499in}{1.838676in}}%
\pgfpathlineto{\pgfqpoint{3.799635in}{1.831889in}}%
\pgfpathlineto{\pgfqpoint{3.756138in}{1.986793in}}%
\pgfpathlineto{\pgfqpoint{3.731981in}{2.002074in}}%
\pgfpathclose%
\pgfusepath{fill}%
\end{pgfscope}%
\begin{pgfscope}%
\pgfpathrectangle{\pgfqpoint{1.072000in}{0.528000in}}{\pgfqpoint{3.696000in}{3.696000in}}%
\pgfusepath{clip}%
\pgfsetbuttcap%
\pgfsetroundjoin%
\definecolor{currentfill}{rgb}{0.619318,0.744121,0.998931}%
\pgfsetfillcolor{currentfill}%
\pgfsetlinewidth{0.000000pt}%
\definecolor{currentstroke}{rgb}{0.000000,0.000000,0.000000}%
\pgfsetstrokecolor{currentstroke}%
\pgfsetdash{}{0pt}%
\pgfpathmoveto{\pgfqpoint{3.712795in}{2.161853in}}%
\pgfpathlineto{\pgfqpoint{3.756138in}{1.986793in}}%
\pgfpathlineto{\pgfqpoint{3.779944in}{1.963234in}}%
\pgfpathlineto{\pgfqpoint{3.736632in}{2.128634in}}%
\pgfpathlineto{\pgfqpoint{3.712795in}{2.161853in}}%
\pgfpathclose%
\pgfusepath{fill}%
\end{pgfscope}%
\begin{pgfscope}%
\pgfpathrectangle{\pgfqpoint{1.072000in}{0.528000in}}{\pgfqpoint{3.696000in}{3.696000in}}%
\pgfusepath{clip}%
\pgfsetbuttcap%
\pgfsetroundjoin%
\definecolor{currentfill}{rgb}{0.323718,0.433158,0.864722}%
\pgfsetfillcolor{currentfill}%
\pgfsetlinewidth{0.000000pt}%
\definecolor{currentstroke}{rgb}{0.000000,0.000000,0.000000}%
\pgfsetstrokecolor{currentstroke}%
\pgfsetdash{}{0pt}%
\pgfpathmoveto{\pgfqpoint{3.770278in}{1.681722in}}%
\pgfpathlineto{\pgfqpoint{3.814895in}{1.569809in}}%
\pgfpathlineto{\pgfqpoint{3.839548in}{1.587736in}}%
\pgfpathlineto{\pgfqpoint{3.794977in}{1.695519in}}%
\pgfpathlineto{\pgfqpoint{3.770278in}{1.681722in}}%
\pgfpathclose%
\pgfusepath{fill}%
\end{pgfscope}%
\begin{pgfscope}%
\pgfpathrectangle{\pgfqpoint{1.072000in}{0.528000in}}{\pgfqpoint{3.696000in}{3.696000in}}%
\pgfusepath{clip}%
\pgfsetbuttcap%
\pgfsetroundjoin%
\definecolor{currentfill}{rgb}{0.919376,0.831273,0.782874}%
\pgfsetfillcolor{currentfill}%
\pgfsetlinewidth{0.000000pt}%
\definecolor{currentstroke}{rgb}{0.000000,0.000000,0.000000}%
\pgfsetstrokecolor{currentstroke}%
\pgfsetdash{}{0pt}%
\pgfpathmoveto{\pgfqpoint{3.608723in}{2.683667in}}%
\pgfpathlineto{\pgfqpoint{3.652836in}{2.524938in}}%
\pgfpathlineto{\pgfqpoint{3.675772in}{2.438053in}}%
\pgfpathlineto{\pgfqpoint{3.631865in}{2.590939in}}%
\pgfpathlineto{\pgfqpoint{3.608723in}{2.683667in}}%
\pgfpathclose%
\pgfusepath{fill}%
\end{pgfscope}%
\begin{pgfscope}%
\pgfpathrectangle{\pgfqpoint{1.072000in}{0.528000in}}{\pgfqpoint{3.696000in}{3.696000in}}%
\pgfusepath{clip}%
\pgfsetbuttcap%
\pgfsetroundjoin%
\definecolor{currentfill}{rgb}{0.266381,0.353304,0.801637}%
\pgfsetfillcolor{currentfill}%
\pgfsetlinewidth{0.000000pt}%
\definecolor{currentstroke}{rgb}{0.000000,0.000000,0.000000}%
\pgfsetstrokecolor{currentstroke}%
\pgfsetdash{}{0pt}%
\pgfpathmoveto{\pgfqpoint{2.148475in}{1.502007in}}%
\pgfpathlineto{\pgfqpoint{2.192955in}{1.513307in}}%
\pgfpathlineto{\pgfqpoint{2.219589in}{1.603372in}}%
\pgfpathlineto{\pgfqpoint{2.175548in}{1.571206in}}%
\pgfpathlineto{\pgfqpoint{2.148475in}{1.502007in}}%
\pgfpathclose%
\pgfusepath{fill}%
\end{pgfscope}%
\begin{pgfscope}%
\pgfpathrectangle{\pgfqpoint{1.072000in}{0.528000in}}{\pgfqpoint{3.696000in}{3.696000in}}%
\pgfusepath{clip}%
\pgfsetbuttcap%
\pgfsetroundjoin%
\definecolor{currentfill}{rgb}{0.328604,0.439712,0.869587}%
\pgfsetfillcolor{currentfill}%
\pgfsetlinewidth{0.000000pt}%
\definecolor{currentstroke}{rgb}{0.000000,0.000000,0.000000}%
\pgfsetstrokecolor{currentstroke}%
\pgfsetdash{}{0pt}%
\pgfpathmoveto{\pgfqpoint{2.175548in}{1.571206in}}%
\pgfpathlineto{\pgfqpoint{2.219589in}{1.603372in}}%
\pgfpathlineto{\pgfqpoint{2.245680in}{1.718741in}}%
\pgfpathlineto{\pgfqpoint{2.202003in}{1.665941in}}%
\pgfpathlineto{\pgfqpoint{2.175548in}{1.571206in}}%
\pgfpathclose%
\pgfusepath{fill}%
\end{pgfscope}%
\begin{pgfscope}%
\pgfpathrectangle{\pgfqpoint{1.072000in}{0.528000in}}{\pgfqpoint{3.696000in}{3.696000in}}%
\pgfusepath{clip}%
\pgfsetbuttcap%
\pgfsetroundjoin%
\definecolor{currentfill}{rgb}{0.810616,0.268797,0.235428}%
\pgfsetfillcolor{currentfill}%
\pgfsetlinewidth{0.000000pt}%
\definecolor{currentstroke}{rgb}{0.000000,0.000000,0.000000}%
\pgfsetstrokecolor{currentstroke}%
\pgfsetdash{}{0pt}%
\pgfpathmoveto{\pgfqpoint{2.889513in}{3.273558in}}%
\pgfpathlineto{\pgfqpoint{2.935453in}{3.318278in}}%
\pgfpathlineto{\pgfqpoint{2.962135in}{3.270803in}}%
\pgfpathlineto{\pgfqpoint{2.916323in}{3.212657in}}%
\pgfpathlineto{\pgfqpoint{2.889513in}{3.273558in}}%
\pgfpathclose%
\pgfusepath{fill}%
\end{pgfscope}%
\begin{pgfscope}%
\pgfpathrectangle{\pgfqpoint{1.072000in}{0.528000in}}{\pgfqpoint{3.696000in}{3.696000in}}%
\pgfusepath{clip}%
\pgfsetbuttcap%
\pgfsetroundjoin%
\definecolor{currentfill}{rgb}{0.630089,0.752516,0.998508}%
\pgfsetfillcolor{currentfill}%
\pgfsetlinewidth{0.000000pt}%
\definecolor{currentstroke}{rgb}{0.000000,0.000000,0.000000}%
\pgfsetstrokecolor{currentstroke}%
\pgfsetdash{}{0pt}%
\pgfpathmoveto{\pgfqpoint{2.236597in}{1.949724in}}%
\pgfpathlineto{\pgfqpoint{2.279349in}{2.051663in}}%
\pgfpathlineto{\pgfqpoint{2.304974in}{2.194629in}}%
\pgfpathlineto{\pgfqpoint{2.262344in}{2.077320in}}%
\pgfpathlineto{\pgfqpoint{2.236597in}{1.949724in}}%
\pgfpathclose%
\pgfusepath{fill}%
\end{pgfscope}%
\begin{pgfscope}%
\pgfpathrectangle{\pgfqpoint{1.072000in}{0.528000in}}{\pgfqpoint{3.696000in}{3.696000in}}%
\pgfusepath{clip}%
\pgfsetbuttcap%
\pgfsetroundjoin%
\definecolor{currentfill}{rgb}{0.820401,0.286765,0.245160}%
\pgfsetfillcolor{currentfill}%
\pgfsetlinewidth{0.000000pt}%
\definecolor{currentstroke}{rgb}{0.000000,0.000000,0.000000}%
\pgfsetstrokecolor{currentstroke}%
\pgfsetdash{}{0pt}%
\pgfpathmoveto{\pgfqpoint{3.127003in}{3.289309in}}%
\pgfpathlineto{\pgfqpoint{3.173419in}{3.292607in}}%
\pgfpathlineto{\pgfqpoint{3.198965in}{3.213779in}}%
\pgfpathlineto{\pgfqpoint{3.152708in}{3.206045in}}%
\pgfpathlineto{\pgfqpoint{3.127003in}{3.289309in}}%
\pgfpathclose%
\pgfusepath{fill}%
\end{pgfscope}%
\begin{pgfscope}%
\pgfpathrectangle{\pgfqpoint{1.072000in}{0.528000in}}{\pgfqpoint{3.696000in}{3.696000in}}%
\pgfusepath{clip}%
\pgfsetbuttcap%
\pgfsetroundjoin%
\definecolor{currentfill}{rgb}{0.869655,0.379274,0.300941}%
\pgfsetfillcolor{currentfill}%
\pgfsetlinewidth{0.000000pt}%
\definecolor{currentstroke}{rgb}{0.000000,0.000000,0.000000}%
\pgfsetstrokecolor{currentstroke}%
\pgfsetdash{}{0pt}%
\pgfpathmoveto{\pgfqpoint{2.671908in}{3.129293in}}%
\pgfpathlineto{\pgfqpoint{2.716820in}{3.228208in}}%
\pgfpathlineto{\pgfqpoint{2.744221in}{3.206306in}}%
\pgfpathlineto{\pgfqpoint{2.699393in}{3.100346in}}%
\pgfpathlineto{\pgfqpoint{2.671908in}{3.129293in}}%
\pgfpathclose%
\pgfusepath{fill}%
\end{pgfscope}%
\begin{pgfscope}%
\pgfpathrectangle{\pgfqpoint{1.072000in}{0.528000in}}{\pgfqpoint{3.696000in}{3.696000in}}%
\pgfusepath{clip}%
\pgfsetbuttcap%
\pgfsetroundjoin%
\definecolor{currentfill}{rgb}{0.409611,0.540759,0.935545}%
\pgfsetfillcolor{currentfill}%
\pgfsetlinewidth{0.000000pt}%
\definecolor{currentstroke}{rgb}{0.000000,0.000000,0.000000}%
\pgfsetstrokecolor{currentstroke}%
\pgfsetdash{}{0pt}%
\pgfpathmoveto{\pgfqpoint{2.202003in}{1.665941in}}%
\pgfpathlineto{\pgfqpoint{2.245680in}{1.718741in}}%
\pgfpathlineto{\pgfqpoint{2.271405in}{1.853797in}}%
\pgfpathlineto{\pgfqpoint{2.228011in}{1.781373in}}%
\pgfpathlineto{\pgfqpoint{2.202003in}{1.665941in}}%
\pgfpathclose%
\pgfusepath{fill}%
\end{pgfscope}%
\begin{pgfscope}%
\pgfpathrectangle{\pgfqpoint{1.072000in}{0.528000in}}{\pgfqpoint{3.696000in}{3.696000in}}%
\pgfusepath{clip}%
\pgfsetbuttcap%
\pgfsetroundjoin%
\definecolor{currentfill}{rgb}{0.959385,0.610306,0.489382}%
\pgfsetfillcolor{currentfill}%
\pgfsetlinewidth{0.000000pt}%
\definecolor{currentstroke}{rgb}{0.000000,0.000000,0.000000}%
\pgfsetstrokecolor{currentstroke}%
\pgfsetdash{}{0pt}%
\pgfpathmoveto{\pgfqpoint{2.474280in}{2.825327in}}%
\pgfpathlineto{\pgfqpoint{2.517624in}{2.974219in}}%
\pgfpathlineto{\pgfqpoint{2.545063in}{3.003676in}}%
\pgfpathlineto{\pgfqpoint{2.501634in}{2.855227in}}%
\pgfpathlineto{\pgfqpoint{2.474280in}{2.825327in}}%
\pgfpathclose%
\pgfusepath{fill}%
\end{pgfscope}%
\begin{pgfscope}%
\pgfpathrectangle{\pgfqpoint{1.072000in}{0.528000in}}{\pgfqpoint{3.696000in}{3.696000in}}%
\pgfusepath{clip}%
\pgfsetbuttcap%
\pgfsetroundjoin%
\definecolor{currentfill}{rgb}{0.234377,0.305542,0.759680}%
\pgfsetfillcolor{currentfill}%
\pgfsetlinewidth{0.000000pt}%
\definecolor{currentstroke}{rgb}{0.000000,0.000000,0.000000}%
\pgfsetstrokecolor{currentstroke}%
\pgfsetdash{}{0pt}%
\pgfpathmoveto{\pgfqpoint{2.120622in}{1.462143in}}%
\pgfpathlineto{\pgfqpoint{2.165603in}{1.453109in}}%
\pgfpathlineto{\pgfqpoint{2.192955in}{1.513307in}}%
\pgfpathlineto{\pgfqpoint{2.148475in}{1.502007in}}%
\pgfpathlineto{\pgfqpoint{2.120622in}{1.462143in}}%
\pgfpathclose%
\pgfusepath{fill}%
\end{pgfscope}%
\begin{pgfscope}%
\pgfpathrectangle{\pgfqpoint{1.072000in}{0.528000in}}{\pgfqpoint{3.696000in}{3.696000in}}%
\pgfusepath{clip}%
\pgfsetbuttcap%
\pgfsetroundjoin%
\definecolor{currentfill}{rgb}{0.261805,0.346484,0.795658}%
\pgfsetfillcolor{currentfill}%
\pgfsetlinewidth{0.000000pt}%
\definecolor{currentstroke}{rgb}{0.000000,0.000000,0.000000}%
\pgfsetstrokecolor{currentstroke}%
\pgfsetdash{}{0pt}%
\pgfpathmoveto{\pgfqpoint{3.790025in}{1.547377in}}%
\pgfpathlineto{\pgfqpoint{3.835609in}{1.474027in}}%
\pgfpathlineto{\pgfqpoint{3.860404in}{1.497890in}}%
\pgfpathlineto{\pgfqpoint{3.814895in}{1.569809in}}%
\pgfpathlineto{\pgfqpoint{3.790025in}{1.547377in}}%
\pgfpathclose%
\pgfusepath{fill}%
\end{pgfscope}%
\begin{pgfscope}%
\pgfpathrectangle{\pgfqpoint{1.072000in}{0.528000in}}{\pgfqpoint{3.696000in}{3.696000in}}%
\pgfusepath{clip}%
\pgfsetbuttcap%
\pgfsetroundjoin%
\definecolor{currentfill}{rgb}{0.624703,0.748318,0.998719}%
\pgfsetfillcolor{currentfill}%
\pgfsetlinewidth{0.000000pt}%
\definecolor{currentstroke}{rgb}{0.000000,0.000000,0.000000}%
\pgfsetstrokecolor{currentstroke}%
\pgfsetdash{}{0pt}%
\pgfpathmoveto{\pgfqpoint{1.727946in}{2.168901in}}%
\pgfpathlineto{\pgfqpoint{1.777442in}{2.021573in}}%
\pgfpathlineto{\pgfqpoint{1.809749in}{1.946260in}}%
\pgfpathlineto{\pgfqpoint{1.760768in}{2.087414in}}%
\pgfpathlineto{\pgfqpoint{1.727946in}{2.168901in}}%
\pgfpathclose%
\pgfusepath{fill}%
\end{pgfscope}%
\begin{pgfscope}%
\pgfpathrectangle{\pgfqpoint{1.072000in}{0.528000in}}{\pgfqpoint{3.696000in}{3.696000in}}%
\pgfusepath{clip}%
\pgfsetbuttcap%
\pgfsetroundjoin%
\definecolor{currentfill}{rgb}{0.810616,0.268797,0.235428}%
\pgfsetfillcolor{currentfill}%
\pgfsetlinewidth{0.000000pt}%
\definecolor{currentstroke}{rgb}{0.000000,0.000000,0.000000}%
\pgfsetstrokecolor{currentstroke}%
\pgfsetdash{}{0pt}%
\pgfpathmoveto{\pgfqpoint{2.816823in}{3.252762in}}%
\pgfpathlineto{\pgfqpoint{2.862549in}{3.309270in}}%
\pgfpathlineto{\pgfqpoint{2.889513in}{3.273558in}}%
\pgfpathlineto{\pgfqpoint{2.843917in}{3.203623in}}%
\pgfpathlineto{\pgfqpoint{2.816823in}{3.252762in}}%
\pgfpathclose%
\pgfusepath{fill}%
\end{pgfscope}%
\begin{pgfscope}%
\pgfpathrectangle{\pgfqpoint{1.072000in}{0.528000in}}{\pgfqpoint{3.696000in}{3.696000in}}%
\pgfusepath{clip}%
\pgfsetbuttcap%
\pgfsetroundjoin%
\definecolor{currentfill}{rgb}{0.229806,0.298718,0.753683}%
\pgfsetfillcolor{currentfill}%
\pgfsetlinewidth{0.000000pt}%
\definecolor{currentstroke}{rgb}{0.000000,0.000000,0.000000}%
\pgfsetstrokecolor{currentstroke}%
\pgfsetdash{}{0pt}%
\pgfpathmoveto{\pgfqpoint{3.835609in}{1.474027in}}%
\pgfpathlineto{\pgfqpoint{3.882314in}{1.444077in}}%
\pgfpathlineto{\pgfqpoint{3.907065in}{1.469234in}}%
\pgfpathlineto{\pgfqpoint{3.860404in}{1.497890in}}%
\pgfpathlineto{\pgfqpoint{3.835609in}{1.474027in}}%
\pgfpathclose%
\pgfusepath{fill}%
\end{pgfscope}%
\begin{pgfscope}%
\pgfpathrectangle{\pgfqpoint{1.072000in}{0.528000in}}{\pgfqpoint{3.696000in}{3.696000in}}%
\pgfusepath{clip}%
\pgfsetbuttcap%
\pgfsetroundjoin%
\definecolor{currentfill}{rgb}{0.790562,0.231397,0.216242}%
\pgfsetfillcolor{currentfill}%
\pgfsetlinewidth{0.000000pt}%
\definecolor{currentstroke}{rgb}{0.000000,0.000000,0.000000}%
\pgfsetstrokecolor{currentstroke}%
\pgfsetdash{}{0pt}%
\pgfpathmoveto{\pgfqpoint{3.008213in}{3.308212in}}%
\pgfpathlineto{\pgfqpoint{3.054485in}{3.327819in}}%
\pgfpathlineto{\pgfqpoint{3.080695in}{3.273434in}}%
\pgfpathlineto{\pgfqpoint{3.034540in}{3.242475in}}%
\pgfpathlineto{\pgfqpoint{3.008213in}{3.308212in}}%
\pgfpathclose%
\pgfusepath{fill}%
\end{pgfscope}%
\begin{pgfscope}%
\pgfpathrectangle{\pgfqpoint{1.072000in}{0.528000in}}{\pgfqpoint{3.696000in}{3.696000in}}%
\pgfusepath{clip}%
\pgfsetbuttcap%
\pgfsetroundjoin%
\definecolor{currentfill}{rgb}{0.830187,0.304733,0.254891}%
\pgfsetfillcolor{currentfill}%
\pgfsetlinewidth{0.000000pt}%
\definecolor{currentstroke}{rgb}{0.000000,0.000000,0.000000}%
\pgfsetstrokecolor{currentstroke}%
\pgfsetdash{}{0pt}%
\pgfpathmoveto{\pgfqpoint{2.744221in}{3.206306in}}%
\pgfpathlineto{\pgfqpoint{2.789618in}{3.281154in}}%
\pgfpathlineto{\pgfqpoint{2.816823in}{3.252762in}}%
\pgfpathlineto{\pgfqpoint{2.771545in}{3.166964in}}%
\pgfpathlineto{\pgfqpoint{2.744221in}{3.206306in}}%
\pgfpathclose%
\pgfusepath{fill}%
\end{pgfscope}%
\begin{pgfscope}%
\pgfpathrectangle{\pgfqpoint{1.072000in}{0.528000in}}{\pgfqpoint{3.696000in}{3.696000in}}%
\pgfusepath{clip}%
\pgfsetbuttcap%
\pgfsetroundjoin%
\definecolor{currentfill}{rgb}{0.494638,0.633022,0.978983}%
\pgfsetfillcolor{currentfill}%
\pgfsetlinewidth{0.000000pt}%
\definecolor{currentstroke}{rgb}{0.000000,0.000000,0.000000}%
\pgfsetstrokecolor{currentstroke}%
\pgfsetdash{}{0pt}%
\pgfpathmoveto{\pgfqpoint{1.793700in}{1.969744in}}%
\pgfpathlineto{\pgfqpoint{1.842310in}{1.844241in}}%
\pgfpathlineto{\pgfqpoint{1.874185in}{1.773922in}}%
\pgfpathlineto{\pgfqpoint{1.826206in}{1.888804in}}%
\pgfpathlineto{\pgfqpoint{1.793700in}{1.969744in}}%
\pgfpathclose%
\pgfusepath{fill}%
\end{pgfscope}%
\begin{pgfscope}%
\pgfpathrectangle{\pgfqpoint{1.072000in}{0.528000in}}{\pgfqpoint{3.696000in}{3.696000in}}%
\pgfusepath{clip}%
\pgfsetbuttcap%
\pgfsetroundjoin%
\definecolor{currentfill}{rgb}{0.257234,0.339661,0.789661}%
\pgfsetfillcolor{currentfill}%
\pgfsetlinewidth{0.000000pt}%
\definecolor{currentstroke}{rgb}{0.000000,0.000000,0.000000}%
\pgfsetstrokecolor{currentstroke}%
\pgfsetdash{}{0pt}%
\pgfpathmoveto{\pgfqpoint{1.999853in}{1.549989in}}%
\pgfpathlineto{\pgfqpoint{2.046042in}{1.494815in}}%
\pgfpathlineto{\pgfqpoint{2.075426in}{1.483663in}}%
\pgfpathlineto{\pgfqpoint{2.029897in}{1.520133in}}%
\pgfpathlineto{\pgfqpoint{1.999853in}{1.549989in}}%
\pgfpathclose%
\pgfusepath{fill}%
\end{pgfscope}%
\begin{pgfscope}%
\pgfpathrectangle{\pgfqpoint{1.072000in}{0.528000in}}{\pgfqpoint{3.696000in}{3.696000in}}%
\pgfusepath{clip}%
\pgfsetbuttcap%
\pgfsetroundjoin%
\definecolor{currentfill}{rgb}{0.289996,0.386836,0.828926}%
\pgfsetfillcolor{currentfill}%
\pgfsetlinewidth{0.000000pt}%
\definecolor{currentstroke}{rgb}{0.000000,0.000000,0.000000}%
\pgfsetstrokecolor{currentstroke}%
\pgfsetdash{}{0pt}%
\pgfpathmoveto{\pgfqpoint{1.953183in}{1.620984in}}%
\pgfpathlineto{\pgfqpoint{1.999853in}{1.549989in}}%
\pgfpathlineto{\pgfqpoint{2.029897in}{1.520133in}}%
\pgfpathlineto{\pgfqpoint{1.983918in}{1.573529in}}%
\pgfpathlineto{\pgfqpoint{1.953183in}{1.620984in}}%
\pgfpathclose%
\pgfusepath{fill}%
\end{pgfscope}%
\begin{pgfscope}%
\pgfpathrectangle{\pgfqpoint{1.072000in}{0.528000in}}{\pgfqpoint{3.696000in}{3.696000in}}%
\pgfusepath{clip}%
\pgfsetbuttcap%
\pgfsetroundjoin%
\definecolor{currentfill}{rgb}{0.728970,0.817464,0.973188}%
\pgfsetfillcolor{currentfill}%
\pgfsetlinewidth{0.000000pt}%
\definecolor{currentstroke}{rgb}{0.000000,0.000000,0.000000}%
\pgfsetstrokecolor{currentstroke}%
\pgfsetdash{}{0pt}%
\pgfpathmoveto{\pgfqpoint{2.262344in}{2.077320in}}%
\pgfpathlineto{\pgfqpoint{2.304974in}{2.194629in}}%
\pgfpathlineto{\pgfqpoint{2.330727in}{2.335253in}}%
\pgfpathlineto{\pgfqpoint{2.288139in}{2.205504in}}%
\pgfpathlineto{\pgfqpoint{2.262344in}{2.077320in}}%
\pgfpathclose%
\pgfusepath{fill}%
\end{pgfscope}%
\begin{pgfscope}%
\pgfpathrectangle{\pgfqpoint{1.072000in}{0.528000in}}{\pgfqpoint{3.696000in}{3.696000in}}%
\pgfusepath{clip}%
\pgfsetbuttcap%
\pgfsetroundjoin%
\definecolor{currentfill}{rgb}{0.964911,0.640159,0.519806}%
\pgfsetfillcolor{currentfill}%
\pgfsetlinewidth{0.000000pt}%
\definecolor{currentstroke}{rgb}{0.000000,0.000000,0.000000}%
\pgfsetstrokecolor{currentstroke}%
\pgfsetdash{}{0pt}%
\pgfpathmoveto{\pgfqpoint{3.497434in}{2.976495in}}%
\pgfpathlineto{\pgfqpoint{3.542715in}{2.864381in}}%
\pgfpathlineto{\pgfqpoint{3.565916in}{2.754446in}}%
\pgfpathlineto{\pgfqpoint{3.520976in}{2.868293in}}%
\pgfpathlineto{\pgfqpoint{3.497434in}{2.976495in}}%
\pgfpathclose%
\pgfusepath{fill}%
\end{pgfscope}%
\begin{pgfscope}%
\pgfpathrectangle{\pgfqpoint{1.072000in}{0.528000in}}{\pgfqpoint{3.696000in}{3.696000in}}%
\pgfusepath{clip}%
\pgfsetbuttcap%
\pgfsetroundjoin%
\definecolor{currentfill}{rgb}{0.238948,0.312365,0.765676}%
\pgfsetfillcolor{currentfill}%
\pgfsetlinewidth{0.000000pt}%
\definecolor{currentstroke}{rgb}{0.000000,0.000000,0.000000}%
\pgfsetstrokecolor{currentstroke}%
\pgfsetdash{}{0pt}%
\pgfpathmoveto{\pgfqpoint{2.046042in}{1.494815in}}%
\pgfpathlineto{\pgfqpoint{2.091849in}{1.454104in}}%
\pgfpathlineto{\pgfqpoint{2.120622in}{1.462143in}}%
\pgfpathlineto{\pgfqpoint{2.075426in}{1.483663in}}%
\pgfpathlineto{\pgfqpoint{2.046042in}{1.494815in}}%
\pgfpathclose%
\pgfusepath{fill}%
\end{pgfscope}%
\begin{pgfscope}%
\pgfpathrectangle{\pgfqpoint{1.072000in}{0.528000in}}{\pgfqpoint{3.696000in}{3.696000in}}%
\pgfusepath{clip}%
\pgfsetbuttcap%
\pgfsetroundjoin%
\definecolor{currentfill}{rgb}{0.905783,0.455186,0.355336}%
\pgfsetfillcolor{currentfill}%
\pgfsetlinewidth{0.000000pt}%
\definecolor{currentstroke}{rgb}{0.000000,0.000000,0.000000}%
\pgfsetstrokecolor{currentstroke}%
\pgfsetdash{}{0pt}%
\pgfpathmoveto{\pgfqpoint{2.572573in}{3.018056in}}%
\pgfpathlineto{\pgfqpoint{2.616838in}{3.140922in}}%
\pgfpathlineto{\pgfqpoint{2.644382in}{3.142413in}}%
\pgfpathlineto{\pgfqpoint{2.600121in}{3.017262in}}%
\pgfpathlineto{\pgfqpoint{2.572573in}{3.018056in}}%
\pgfpathclose%
\pgfusepath{fill}%
\end{pgfscope}%
\begin{pgfscope}%
\pgfpathrectangle{\pgfqpoint{1.072000in}{0.528000in}}{\pgfqpoint{3.696000in}{3.696000in}}%
\pgfusepath{clip}%
\pgfsetbuttcap%
\pgfsetroundjoin%
\definecolor{currentfill}{rgb}{0.891817,0.851973,0.829085}%
\pgfsetfillcolor{currentfill}%
\pgfsetlinewidth{0.000000pt}%
\definecolor{currentstroke}{rgb}{0.000000,0.000000,0.000000}%
\pgfsetstrokecolor{currentstroke}%
\pgfsetdash{}{0pt}%
\pgfpathmoveto{\pgfqpoint{3.629445in}{2.601312in}}%
\pgfpathlineto{\pgfqpoint{3.673309in}{2.429414in}}%
\pgfpathlineto{\pgfqpoint{3.696524in}{2.359619in}}%
\pgfpathlineto{\pgfqpoint{3.652836in}{2.524938in}}%
\pgfpathlineto{\pgfqpoint{3.629445in}{2.601312in}}%
\pgfpathclose%
\pgfusepath{fill}%
\end{pgfscope}%
\begin{pgfscope}%
\pgfpathrectangle{\pgfqpoint{1.072000in}{0.528000in}}{\pgfqpoint{3.696000in}{3.696000in}}%
\pgfusepath{clip}%
\pgfsetbuttcap%
\pgfsetroundjoin%
\definecolor{currentfill}{rgb}{0.338377,0.452819,0.879317}%
\pgfsetfillcolor{currentfill}%
\pgfsetlinewidth{0.000000pt}%
\definecolor{currentstroke}{rgb}{0.000000,0.000000,0.000000}%
\pgfsetstrokecolor{currentstroke}%
\pgfsetdash{}{0pt}%
\pgfpathmoveto{\pgfqpoint{1.905947in}{1.708492in}}%
\pgfpathlineto{\pgfqpoint{1.953183in}{1.620984in}}%
\pgfpathlineto{\pgfqpoint{1.983918in}{1.573529in}}%
\pgfpathlineto{\pgfqpoint{1.937380in}{1.645168in}}%
\pgfpathlineto{\pgfqpoint{1.905947in}{1.708492in}}%
\pgfpathclose%
\pgfusepath{fill}%
\end{pgfscope}%
\begin{pgfscope}%
\pgfpathrectangle{\pgfqpoint{1.072000in}{0.528000in}}{\pgfqpoint{3.696000in}{3.696000in}}%
\pgfusepath{clip}%
\pgfsetbuttcap%
\pgfsetroundjoin%
\definecolor{currentfill}{rgb}{0.810616,0.268797,0.235428}%
\pgfsetfillcolor{currentfill}%
\pgfsetlinewidth{0.000000pt}%
\definecolor{currentstroke}{rgb}{0.000000,0.000000,0.000000}%
\pgfsetstrokecolor{currentstroke}%
\pgfsetdash{}{0pt}%
\pgfpathmoveto{\pgfqpoint{3.173419in}{3.292607in}}%
\pgfpathlineto{\pgfqpoint{3.219909in}{3.285237in}}%
\pgfpathlineto{\pgfqpoint{3.245276in}{3.207980in}}%
\pgfpathlineto{\pgfqpoint{3.198965in}{3.213779in}}%
\pgfpathlineto{\pgfqpoint{3.173419in}{3.292607in}}%
\pgfpathclose%
\pgfusepath{fill}%
\end{pgfscope}%
\begin{pgfscope}%
\pgfpathrectangle{\pgfqpoint{1.072000in}{0.528000in}}{\pgfqpoint{3.696000in}{3.696000in}}%
\pgfusepath{clip}%
\pgfsetbuttcap%
\pgfsetroundjoin%
\definecolor{currentfill}{rgb}{0.510824,0.649397,0.985079}%
\pgfsetfillcolor{currentfill}%
\pgfsetlinewidth{0.000000pt}%
\definecolor{currentstroke}{rgb}{0.000000,0.000000,0.000000}%
\pgfsetstrokecolor{currentstroke}%
\pgfsetdash{}{0pt}%
\pgfpathmoveto{\pgfqpoint{2.228011in}{1.781373in}}%
\pgfpathlineto{\pgfqpoint{2.271405in}{1.853797in}}%
\pgfpathlineto{\pgfqpoint{2.296938in}{2.002218in}}%
\pgfpathlineto{\pgfqpoint{2.253740in}{1.911929in}}%
\pgfpathlineto{\pgfqpoint{2.228011in}{1.781373in}}%
\pgfpathclose%
\pgfusepath{fill}%
\end{pgfscope}%
\begin{pgfscope}%
\pgfpathrectangle{\pgfqpoint{1.072000in}{0.528000in}}{\pgfqpoint{3.696000in}{3.696000in}}%
\pgfusepath{clip}%
\pgfsetbuttcap%
\pgfsetroundjoin%
\definecolor{currentfill}{rgb}{0.328604,0.439712,0.869587}%
\pgfsetfillcolor{currentfill}%
\pgfsetlinewidth{0.000000pt}%
\definecolor{currentstroke}{rgb}{0.000000,0.000000,0.000000}%
\pgfsetstrokecolor{currentstroke}%
\pgfsetdash{}{0pt}%
\pgfpathmoveto{\pgfqpoint{3.745303in}{1.660796in}}%
\pgfpathlineto{\pgfqpoint{3.790025in}{1.547377in}}%
\pgfpathlineto{\pgfqpoint{3.814895in}{1.569809in}}%
\pgfpathlineto{\pgfqpoint{3.770278in}{1.681722in}}%
\pgfpathlineto{\pgfqpoint{3.745303in}{1.660796in}}%
\pgfpathclose%
\pgfusepath{fill}%
\end{pgfscope}%
\begin{pgfscope}%
\pgfpathrectangle{\pgfqpoint{1.072000in}{0.528000in}}{\pgfqpoint{3.696000in}{3.696000in}}%
\pgfusepath{clip}%
\pgfsetbuttcap%
\pgfsetroundjoin%
\definecolor{currentfill}{rgb}{0.425199,0.559058,0.946061}%
\pgfsetfillcolor{currentfill}%
\pgfsetlinewidth{0.000000pt}%
\definecolor{currentstroke}{rgb}{0.000000,0.000000,0.000000}%
\pgfsetstrokecolor{currentstroke}%
\pgfsetdash{}{0pt}%
\pgfpathmoveto{\pgfqpoint{3.726264in}{1.828097in}}%
\pgfpathlineto{\pgfqpoint{3.770278in}{1.681722in}}%
\pgfpathlineto{\pgfqpoint{3.794977in}{1.695519in}}%
\pgfpathlineto{\pgfqpoint{3.751046in}{1.837837in}}%
\pgfpathlineto{\pgfqpoint{3.726264in}{1.828097in}}%
\pgfpathclose%
\pgfusepath{fill}%
\end{pgfscope}%
\begin{pgfscope}%
\pgfpathrectangle{\pgfqpoint{1.072000in}{0.528000in}}{\pgfqpoint{3.696000in}{3.696000in}}%
\pgfusepath{clip}%
\pgfsetbuttcap%
\pgfsetroundjoin%
\definecolor{currentfill}{rgb}{0.962701,0.628218,0.507636}%
\pgfsetfillcolor{currentfill}%
\pgfsetlinewidth{0.000000pt}%
\definecolor{currentstroke}{rgb}{0.000000,0.000000,0.000000}%
\pgfsetstrokecolor{currentstroke}%
\pgfsetdash{}{0pt}%
\pgfpathmoveto{\pgfqpoint{2.447062in}{2.779743in}}%
\pgfpathlineto{\pgfqpoint{2.490295in}{2.929463in}}%
\pgfpathlineto{\pgfqpoint{2.517624in}{2.974219in}}%
\pgfpathlineto{\pgfqpoint{2.474280in}{2.825327in}}%
\pgfpathlineto{\pgfqpoint{2.447062in}{2.779743in}}%
\pgfpathclose%
\pgfusepath{fill}%
\end{pgfscope}%
\begin{pgfscope}%
\pgfpathrectangle{\pgfqpoint{1.072000in}{0.528000in}}{\pgfqpoint{3.696000in}{3.696000in}}%
\pgfusepath{clip}%
\pgfsetbuttcap%
\pgfsetroundjoin%
\definecolor{currentfill}{rgb}{0.304174,0.406945,0.845263}%
\pgfsetfillcolor{currentfill}%
\pgfsetlinewidth{0.000000pt}%
\definecolor{currentstroke}{rgb}{0.000000,0.000000,0.000000}%
\pgfsetstrokecolor{currentstroke}%
\pgfsetdash{}{0pt}%
\pgfpathmoveto{\pgfqpoint{2.192955in}{1.513307in}}%
\pgfpathlineto{\pgfqpoint{2.237418in}{1.533315in}}%
\pgfpathlineto{\pgfqpoint{2.263704in}{1.642892in}}%
\pgfpathlineto{\pgfqpoint{2.219589in}{1.603372in}}%
\pgfpathlineto{\pgfqpoint{2.192955in}{1.513307in}}%
\pgfpathclose%
\pgfusepath{fill}%
\end{pgfscope}%
\begin{pgfscope}%
\pgfpathrectangle{\pgfqpoint{1.072000in}{0.528000in}}{\pgfqpoint{3.696000in}{3.696000in}}%
\pgfusepath{clip}%
\pgfsetbuttcap%
\pgfsetroundjoin%
\definecolor{currentfill}{rgb}{0.543440,0.680003,0.993051}%
\pgfsetfillcolor{currentfill}%
\pgfsetlinewidth{0.000000pt}%
\definecolor{currentstroke}{rgb}{0.000000,0.000000,0.000000}%
\pgfsetstrokecolor{currentstroke}%
\pgfsetdash{}{0pt}%
\pgfpathmoveto{\pgfqpoint{3.707454in}{2.007333in}}%
\pgfpathlineto{\pgfqpoint{3.751046in}{1.837837in}}%
\pgfpathlineto{\pgfqpoint{3.775499in}{1.838676in}}%
\pgfpathlineto{\pgfqpoint{3.731981in}{2.002074in}}%
\pgfpathlineto{\pgfqpoint{3.707454in}{2.007333in}}%
\pgfpathclose%
\pgfusepath{fill}%
\end{pgfscope}%
\begin{pgfscope}%
\pgfpathrectangle{\pgfqpoint{1.072000in}{0.528000in}}{\pgfqpoint{3.696000in}{3.696000in}}%
\pgfusepath{clip}%
\pgfsetbuttcap%
\pgfsetroundjoin%
\definecolor{currentfill}{rgb}{0.229806,0.298718,0.753683}%
\pgfsetfillcolor{currentfill}%
\pgfsetlinewidth{0.000000pt}%
\definecolor{currentstroke}{rgb}{0.000000,0.000000,0.000000}%
\pgfsetstrokecolor{currentstroke}%
\pgfsetdash{}{0pt}%
\pgfpathmoveto{\pgfqpoint{2.091849in}{1.454104in}}%
\pgfpathlineto{\pgfqpoint{2.137377in}{1.425990in}}%
\pgfpathlineto{\pgfqpoint{2.165603in}{1.453109in}}%
\pgfpathlineto{\pgfqpoint{2.120622in}{1.462143in}}%
\pgfpathlineto{\pgfqpoint{2.091849in}{1.454104in}}%
\pgfpathclose%
\pgfusepath{fill}%
\end{pgfscope}%
\begin{pgfscope}%
\pgfpathrectangle{\pgfqpoint{1.072000in}{0.528000in}}{\pgfqpoint{3.696000in}{3.696000in}}%
\pgfusepath{clip}%
\pgfsetbuttcap%
\pgfsetroundjoin%
\definecolor{currentfill}{rgb}{0.252663,0.332837,0.783665}%
\pgfsetfillcolor{currentfill}%
\pgfsetlinewidth{0.000000pt}%
\definecolor{currentstroke}{rgb}{0.000000,0.000000,0.000000}%
\pgfsetstrokecolor{currentstroke}%
\pgfsetdash{}{0pt}%
\pgfpathmoveto{\pgfqpoint{2.165603in}{1.453109in}}%
\pgfpathlineto{\pgfqpoint{2.210483in}{1.453786in}}%
\pgfpathlineto{\pgfqpoint{2.237418in}{1.533315in}}%
\pgfpathlineto{\pgfqpoint{2.192955in}{1.513307in}}%
\pgfpathlineto{\pgfqpoint{2.165603in}{1.453109in}}%
\pgfpathclose%
\pgfusepath{fill}%
\end{pgfscope}%
\begin{pgfscope}%
\pgfpathrectangle{\pgfqpoint{1.072000in}{0.528000in}}{\pgfqpoint{3.696000in}{3.696000in}}%
\pgfusepath{clip}%
\pgfsetbuttcap%
\pgfsetroundjoin%
\definecolor{currentfill}{rgb}{0.818056,0.855590,0.914638}%
\pgfsetfillcolor{currentfill}%
\pgfsetlinewidth{0.000000pt}%
\definecolor{currentstroke}{rgb}{0.000000,0.000000,0.000000}%
\pgfsetstrokecolor{currentstroke}%
\pgfsetdash{}{0pt}%
\pgfpathmoveto{\pgfqpoint{2.288139in}{2.205504in}}%
\pgfpathlineto{\pgfqpoint{2.330727in}{2.335253in}}%
\pgfpathlineto{\pgfqpoint{2.356686in}{2.468670in}}%
\pgfpathlineto{\pgfqpoint{2.314072in}{2.329576in}}%
\pgfpathlineto{\pgfqpoint{2.288139in}{2.205504in}}%
\pgfpathclose%
\pgfusepath{fill}%
\end{pgfscope}%
\begin{pgfscope}%
\pgfpathrectangle{\pgfqpoint{1.072000in}{0.528000in}}{\pgfqpoint{3.696000in}{3.696000in}}%
\pgfusepath{clip}%
\pgfsetbuttcap%
\pgfsetroundjoin%
\definecolor{currentfill}{rgb}{0.763520,0.178667,0.193396}%
\pgfsetfillcolor{currentfill}%
\pgfsetlinewidth{0.000000pt}%
\definecolor{currentstroke}{rgb}{0.000000,0.000000,0.000000}%
\pgfsetstrokecolor{currentstroke}%
\pgfsetdash{}{0pt}%
\pgfpathmoveto{\pgfqpoint{2.935453in}{3.318278in}}%
\pgfpathlineto{\pgfqpoint{2.981639in}{3.340125in}}%
\pgfpathlineto{\pgfqpoint{3.008213in}{3.308212in}}%
\pgfpathlineto{\pgfqpoint{2.962135in}{3.270803in}}%
\pgfpathlineto{\pgfqpoint{2.935453in}{3.318278in}}%
\pgfpathclose%
\pgfusepath{fill}%
\end{pgfscope}%
\begin{pgfscope}%
\pgfpathrectangle{\pgfqpoint{1.072000in}{0.528000in}}{\pgfqpoint{3.696000in}{3.696000in}}%
\pgfusepath{clip}%
\pgfsetbuttcap%
\pgfsetroundjoin%
\definecolor{currentfill}{rgb}{0.965899,0.740142,0.637058}%
\pgfsetfillcolor{currentfill}%
\pgfsetlinewidth{0.000000pt}%
\definecolor{currentstroke}{rgb}{0.000000,0.000000,0.000000}%
\pgfsetstrokecolor{currentstroke}%
\pgfsetdash{}{0pt}%
\pgfpathmoveto{\pgfqpoint{3.564101in}{2.829246in}}%
\pgfpathlineto{\pgfqpoint{3.608723in}{2.683667in}}%
\pgfpathlineto{\pgfqpoint{3.631865in}{2.590939in}}%
\pgfpathlineto{\pgfqpoint{3.587527in}{2.734716in}}%
\pgfpathlineto{\pgfqpoint{3.564101in}{2.829246in}}%
\pgfpathclose%
\pgfusepath{fill}%
\end{pgfscope}%
\begin{pgfscope}%
\pgfpathrectangle{\pgfqpoint{1.072000in}{0.528000in}}{\pgfqpoint{3.696000in}{3.696000in}}%
\pgfusepath{clip}%
\pgfsetbuttcap%
\pgfsetroundjoin%
\definecolor{currentfill}{rgb}{0.839351,0.861167,0.894494}%
\pgfsetfillcolor{currentfill}%
\pgfsetlinewidth{0.000000pt}%
\definecolor{currentstroke}{rgb}{0.000000,0.000000,0.000000}%
\pgfsetstrokecolor{currentstroke}%
\pgfsetdash{}{0pt}%
\pgfpathmoveto{\pgfqpoint{3.649651in}{2.488481in}}%
\pgfpathlineto{\pgfqpoint{3.693278in}{2.306811in}}%
\pgfpathlineto{\pgfqpoint{3.716797in}{2.255100in}}%
\pgfpathlineto{\pgfqpoint{3.673309in}{2.429414in}}%
\pgfpathlineto{\pgfqpoint{3.649651in}{2.488481in}}%
\pgfpathclose%
\pgfusepath{fill}%
\end{pgfscope}%
\begin{pgfscope}%
\pgfpathrectangle{\pgfqpoint{1.072000in}{0.528000in}}{\pgfqpoint{3.696000in}{3.696000in}}%
\pgfusepath{clip}%
\pgfsetbuttcap%
\pgfsetroundjoin%
\definecolor{currentfill}{rgb}{0.409611,0.540759,0.935545}%
\pgfsetfillcolor{currentfill}%
\pgfsetlinewidth{0.000000pt}%
\definecolor{currentstroke}{rgb}{0.000000,0.000000,0.000000}%
\pgfsetstrokecolor{currentstroke}%
\pgfsetdash{}{0pt}%
\pgfpathmoveto{\pgfqpoint{1.858085in}{1.812410in}}%
\pgfpathlineto{\pgfqpoint{1.905947in}{1.708492in}}%
\pgfpathlineto{\pgfqpoint{1.937380in}{1.645168in}}%
\pgfpathlineto{\pgfqpoint{1.890197in}{1.735553in}}%
\pgfpathlineto{\pgfqpoint{1.858085in}{1.812410in}}%
\pgfpathclose%
\pgfusepath{fill}%
\end{pgfscope}%
\begin{pgfscope}%
\pgfpathrectangle{\pgfqpoint{1.072000in}{0.528000in}}{\pgfqpoint{3.696000in}{3.696000in}}%
\pgfusepath{clip}%
\pgfsetbuttcap%
\pgfsetroundjoin%
\definecolor{currentfill}{rgb}{0.261805,0.346484,0.795658}%
\pgfsetfillcolor{currentfill}%
\pgfsetlinewidth{0.000000pt}%
\definecolor{currentstroke}{rgb}{0.000000,0.000000,0.000000}%
\pgfsetstrokecolor{currentstroke}%
\pgfsetdash{}{0pt}%
\pgfpathmoveto{\pgfqpoint{3.764943in}{1.520190in}}%
\pgfpathlineto{\pgfqpoint{3.810662in}{1.448011in}}%
\pgfpathlineto{\pgfqpoint{3.835609in}{1.474027in}}%
\pgfpathlineto{\pgfqpoint{3.790025in}{1.547377in}}%
\pgfpathlineto{\pgfqpoint{3.764943in}{1.520190in}}%
\pgfpathclose%
\pgfusepath{fill}%
\end{pgfscope}%
\begin{pgfscope}%
\pgfpathrectangle{\pgfqpoint{1.072000in}{0.528000in}}{\pgfqpoint{3.696000in}{3.696000in}}%
\pgfusepath{clip}%
\pgfsetbuttcap%
\pgfsetroundjoin%
\definecolor{currentfill}{rgb}{0.229806,0.298718,0.753683}%
\pgfsetfillcolor{currentfill}%
\pgfsetlinewidth{0.000000pt}%
\definecolor{currentstroke}{rgb}{0.000000,0.000000,0.000000}%
\pgfsetstrokecolor{currentstroke}%
\pgfsetdash{}{0pt}%
\pgfpathmoveto{\pgfqpoint{3.810662in}{1.448011in}}%
\pgfpathlineto{\pgfqpoint{3.857473in}{1.419204in}}%
\pgfpathlineto{\pgfqpoint{3.882314in}{1.444077in}}%
\pgfpathlineto{\pgfqpoint{3.835609in}{1.474027in}}%
\pgfpathlineto{\pgfqpoint{3.810662in}{1.448011in}}%
\pgfpathclose%
\pgfusepath{fill}%
\end{pgfscope}%
\begin{pgfscope}%
\pgfpathrectangle{\pgfqpoint{1.072000in}{0.528000in}}{\pgfqpoint{3.696000in}{3.696000in}}%
\pgfusepath{clip}%
\pgfsetbuttcap%
\pgfsetroundjoin%
\definecolor{currentfill}{rgb}{0.378598,0.503856,0.913692}%
\pgfsetfillcolor{currentfill}%
\pgfsetlinewidth{0.000000pt}%
\definecolor{currentstroke}{rgb}{0.000000,0.000000,0.000000}%
\pgfsetstrokecolor{currentstroke}%
\pgfsetdash{}{0pt}%
\pgfpathmoveto{\pgfqpoint{2.219589in}{1.603372in}}%
\pgfpathlineto{\pgfqpoint{2.263704in}{1.642892in}}%
\pgfpathlineto{\pgfqpoint{2.289524in}{1.777184in}}%
\pgfpathlineto{\pgfqpoint{2.245680in}{1.718741in}}%
\pgfpathlineto{\pgfqpoint{2.219589in}{1.603372in}}%
\pgfpathclose%
\pgfusepath{fill}%
\end{pgfscope}%
\begin{pgfscope}%
\pgfpathrectangle{\pgfqpoint{1.072000in}{0.528000in}}{\pgfqpoint{3.696000in}{3.696000in}}%
\pgfusepath{clip}%
\pgfsetbuttcap%
\pgfsetroundjoin%
\definecolor{currentfill}{rgb}{0.939254,0.539581,0.423900}%
\pgfsetfillcolor{currentfill}%
\pgfsetlinewidth{0.000000pt}%
\definecolor{currentstroke}{rgb}{0.000000,0.000000,0.000000}%
\pgfsetstrokecolor{currentstroke}%
\pgfsetdash{}{0pt}%
\pgfpathmoveto{\pgfqpoint{3.451733in}{3.069378in}}%
\pgfpathlineto{\pgfqpoint{3.497434in}{2.976495in}}%
\pgfpathlineto{\pgfqpoint{3.520976in}{2.868293in}}%
\pgfpathlineto{\pgfqpoint{3.475637in}{2.965785in}}%
\pgfpathlineto{\pgfqpoint{3.451733in}{3.069378in}}%
\pgfpathclose%
\pgfusepath{fill}%
\end{pgfscope}%
\begin{pgfscope}%
\pgfpathrectangle{\pgfqpoint{1.072000in}{0.528000in}}{\pgfqpoint{3.696000in}{3.696000in}}%
\pgfusepath{clip}%
\pgfsetbuttcap%
\pgfsetroundjoin%
\definecolor{currentfill}{rgb}{0.810616,0.268797,0.235428}%
\pgfsetfillcolor{currentfill}%
\pgfsetlinewidth{0.000000pt}%
\definecolor{currentstroke}{rgb}{0.000000,0.000000,0.000000}%
\pgfsetstrokecolor{currentstroke}%
\pgfsetdash{}{0pt}%
\pgfpathmoveto{\pgfqpoint{3.219909in}{3.285237in}}%
\pgfpathlineto{\pgfqpoint{3.266436in}{3.267899in}}%
\pgfpathlineto{\pgfqpoint{3.291592in}{3.189008in}}%
\pgfpathlineto{\pgfqpoint{3.245276in}{3.207980in}}%
\pgfpathlineto{\pgfqpoint{3.219909in}{3.285237in}}%
\pgfpathclose%
\pgfusepath{fill}%
\end{pgfscope}%
\begin{pgfscope}%
\pgfpathrectangle{\pgfqpoint{1.072000in}{0.528000in}}{\pgfqpoint{3.696000in}{3.696000in}}%
\pgfusepath{clip}%
\pgfsetbuttcap%
\pgfsetroundjoin%
\definecolor{currentfill}{rgb}{0.661968,0.775491,0.993937}%
\pgfsetfillcolor{currentfill}%
\pgfsetlinewidth{0.000000pt}%
\definecolor{currentstroke}{rgb}{0.000000,0.000000,0.000000}%
\pgfsetstrokecolor{currentstroke}%
\pgfsetdash{}{0pt}%
\pgfpathmoveto{\pgfqpoint{3.688556in}{2.184487in}}%
\pgfpathlineto{\pgfqpoint{3.731981in}{2.002074in}}%
\pgfpathlineto{\pgfqpoint{3.756138in}{1.986793in}}%
\pgfpathlineto{\pgfqpoint{3.712795in}{2.161853in}}%
\pgfpathlineto{\pgfqpoint{3.688556in}{2.184487in}}%
\pgfpathclose%
\pgfusepath{fill}%
\end{pgfscope}%
\begin{pgfscope}%
\pgfpathrectangle{\pgfqpoint{1.072000in}{0.528000in}}{\pgfqpoint{3.696000in}{3.696000in}}%
\pgfusepath{clip}%
\pgfsetbuttcap%
\pgfsetroundjoin%
\definecolor{currentfill}{rgb}{0.763520,0.178667,0.193396}%
\pgfsetfillcolor{currentfill}%
\pgfsetlinewidth{0.000000pt}%
\definecolor{currentstroke}{rgb}{0.000000,0.000000,0.000000}%
\pgfsetstrokecolor{currentstroke}%
\pgfsetdash{}{0pt}%
\pgfpathmoveto{\pgfqpoint{3.054485in}{3.327819in}}%
\pgfpathlineto{\pgfqpoint{3.100896in}{3.333720in}}%
\pgfpathlineto{\pgfqpoint{3.127003in}{3.289309in}}%
\pgfpathlineto{\pgfqpoint{3.080695in}{3.273434in}}%
\pgfpathlineto{\pgfqpoint{3.054485in}{3.327819in}}%
\pgfpathclose%
\pgfusepath{fill}%
\end{pgfscope}%
\begin{pgfscope}%
\pgfpathrectangle{\pgfqpoint{1.072000in}{0.528000in}}{\pgfqpoint{3.696000in}{3.696000in}}%
\pgfusepath{clip}%
\pgfsetbuttcap%
\pgfsetroundjoin%
\definecolor{currentfill}{rgb}{0.763363,0.835092,0.955658}%
\pgfsetfillcolor{currentfill}%
\pgfsetlinewidth{0.000000pt}%
\definecolor{currentstroke}{rgb}{0.000000,0.000000,0.000000}%
\pgfsetstrokecolor{currentstroke}%
\pgfsetdash{}{0pt}%
\pgfpathmoveto{\pgfqpoint{3.669336in}{2.347758in}}%
\pgfpathlineto{\pgfqpoint{3.712795in}{2.161853in}}%
\pgfpathlineto{\pgfqpoint{3.736632in}{2.128634in}}%
\pgfpathlineto{\pgfqpoint{3.693278in}{2.306811in}}%
\pgfpathlineto{\pgfqpoint{3.669336in}{2.347758in}}%
\pgfpathclose%
\pgfusepath{fill}%
\end{pgfscope}%
\begin{pgfscope}%
\pgfpathrectangle{\pgfqpoint{1.072000in}{0.528000in}}{\pgfqpoint{3.696000in}{3.696000in}}%
\pgfusepath{clip}%
\pgfsetbuttcap%
\pgfsetroundjoin%
\definecolor{currentfill}{rgb}{0.630089,0.752516,0.998508}%
\pgfsetfillcolor{currentfill}%
\pgfsetlinewidth{0.000000pt}%
\definecolor{currentstroke}{rgb}{0.000000,0.000000,0.000000}%
\pgfsetstrokecolor{currentstroke}%
\pgfsetdash{}{0pt}%
\pgfpathmoveto{\pgfqpoint{2.253740in}{1.911929in}}%
\pgfpathlineto{\pgfqpoint{2.296938in}{2.002218in}}%
\pgfpathlineto{\pgfqpoint{2.322438in}{2.157381in}}%
\pgfpathlineto{\pgfqpoint{2.279349in}{2.051663in}}%
\pgfpathlineto{\pgfqpoint{2.253740in}{1.911929in}}%
\pgfpathclose%
\pgfusepath{fill}%
\end{pgfscope}%
\begin{pgfscope}%
\pgfpathrectangle{\pgfqpoint{1.072000in}{0.528000in}}{\pgfqpoint{3.696000in}{3.696000in}}%
\pgfusepath{clip}%
\pgfsetbuttcap%
\pgfsetroundjoin%
\definecolor{currentfill}{rgb}{0.967317,0.657471,0.538160}%
\pgfsetfillcolor{currentfill}%
\pgfsetlinewidth{0.000000pt}%
\definecolor{currentstroke}{rgb}{0.000000,0.000000,0.000000}%
\pgfsetstrokecolor{currentstroke}%
\pgfsetdash{}{0pt}%
\pgfpathmoveto{\pgfqpoint{2.420017in}{2.718487in}}%
\pgfpathlineto{\pgfqpoint{2.463116in}{2.868960in}}%
\pgfpathlineto{\pgfqpoint{2.490295in}{2.929463in}}%
\pgfpathlineto{\pgfqpoint{2.447062in}{2.779743in}}%
\pgfpathlineto{\pgfqpoint{2.420017in}{2.718487in}}%
\pgfpathclose%
\pgfusepath{fill}%
\end{pgfscope}%
\begin{pgfscope}%
\pgfpathrectangle{\pgfqpoint{1.072000in}{0.528000in}}{\pgfqpoint{3.696000in}{3.696000in}}%
\pgfusepath{clip}%
\pgfsetbuttcap%
\pgfsetroundjoin%
\definecolor{currentfill}{rgb}{0.603162,0.731527,0.999565}%
\pgfsetfillcolor{currentfill}%
\pgfsetlinewidth{0.000000pt}%
\definecolor{currentstroke}{rgb}{0.000000,0.000000,0.000000}%
\pgfsetstrokecolor{currentstroke}%
\pgfsetdash{}{0pt}%
\pgfpathmoveto{\pgfqpoint{1.744393in}{2.109478in}}%
\pgfpathlineto{\pgfqpoint{1.793700in}{1.969744in}}%
\pgfpathlineto{\pgfqpoint{1.826206in}{1.888804in}}%
\pgfpathlineto{\pgfqpoint{1.777442in}{2.021573in}}%
\pgfpathlineto{\pgfqpoint{1.744393in}{2.109478in}}%
\pgfpathclose%
\pgfusepath{fill}%
\end{pgfscope}%
\begin{pgfscope}%
\pgfpathrectangle{\pgfqpoint{1.072000in}{0.528000in}}{\pgfqpoint{3.696000in}{3.696000in}}%
\pgfusepath{clip}%
\pgfsetbuttcap%
\pgfsetroundjoin%
\definecolor{currentfill}{rgb}{0.891817,0.851973,0.829085}%
\pgfsetfillcolor{currentfill}%
\pgfsetlinewidth{0.000000pt}%
\definecolor{currentstroke}{rgb}{0.000000,0.000000,0.000000}%
\pgfsetstrokecolor{currentstroke}%
\pgfsetdash{}{0pt}%
\pgfpathmoveto{\pgfqpoint{2.314072in}{2.329576in}}%
\pgfpathlineto{\pgfqpoint{2.356686in}{2.468670in}}%
\pgfpathlineto{\pgfqpoint{2.382900in}{2.590982in}}%
\pgfpathlineto{\pgfqpoint{2.340204in}{2.445566in}}%
\pgfpathlineto{\pgfqpoint{2.314072in}{2.329576in}}%
\pgfpathclose%
\pgfusepath{fill}%
\end{pgfscope}%
\begin{pgfscope}%
\pgfpathrectangle{\pgfqpoint{1.072000in}{0.528000in}}{\pgfqpoint{3.696000in}{3.696000in}}%
\pgfusepath{clip}%
\pgfsetbuttcap%
\pgfsetroundjoin%
\definecolor{currentfill}{rgb}{0.839365,0.321856,0.264924}%
\pgfsetfillcolor{currentfill}%
\pgfsetlinewidth{0.000000pt}%
\definecolor{currentstroke}{rgb}{0.000000,0.000000,0.000000}%
\pgfsetstrokecolor{currentstroke}%
\pgfsetdash{}{0pt}%
\pgfpathmoveto{\pgfqpoint{2.644382in}{3.142413in}}%
\pgfpathlineto{\pgfqpoint{2.689363in}{3.234966in}}%
\pgfpathlineto{\pgfqpoint{2.716820in}{3.228208in}}%
\pgfpathlineto{\pgfqpoint{2.671908in}{3.129293in}}%
\pgfpathlineto{\pgfqpoint{2.644382in}{3.142413in}}%
\pgfpathclose%
\pgfusepath{fill}%
\end{pgfscope}%
\begin{pgfscope}%
\pgfpathrectangle{\pgfqpoint{1.072000in}{0.528000in}}{\pgfqpoint{3.696000in}{3.696000in}}%
\pgfusepath{clip}%
\pgfsetbuttcap%
\pgfsetroundjoin%
\definecolor{currentfill}{rgb}{0.748682,0.827679,0.963334}%
\pgfsetfillcolor{currentfill}%
\pgfsetlinewidth{0.000000pt}%
\definecolor{currentstroke}{rgb}{0.000000,0.000000,0.000000}%
\pgfsetstrokecolor{currentstroke}%
\pgfsetdash{}{0pt}%
\pgfpathmoveto{\pgfqpoint{1.677830in}{2.326304in}}%
\pgfpathlineto{\pgfqpoint{1.727946in}{2.168901in}}%
\pgfpathlineto{\pgfqpoint{1.760768in}{2.087414in}}%
\pgfpathlineto{\pgfqpoint{1.711041in}{2.242789in}}%
\pgfpathlineto{\pgfqpoint{1.677830in}{2.326304in}}%
\pgfpathclose%
\pgfusepath{fill}%
\end{pgfscope}%
\begin{pgfscope}%
\pgfpathrectangle{\pgfqpoint{1.072000in}{0.528000in}}{\pgfqpoint{3.696000in}{3.696000in}}%
\pgfusepath{clip}%
\pgfsetbuttcap%
\pgfsetroundjoin%
\definecolor{currentfill}{rgb}{0.229806,0.298718,0.753683}%
\pgfsetfillcolor{currentfill}%
\pgfsetlinewidth{0.000000pt}%
\definecolor{currentstroke}{rgb}{0.000000,0.000000,0.000000}%
\pgfsetstrokecolor{currentstroke}%
\pgfsetdash{}{0pt}%
\pgfpathmoveto{\pgfqpoint{2.137377in}{1.425990in}}%
\pgfpathlineto{\pgfqpoint{2.182729in}{1.408245in}}%
\pgfpathlineto{\pgfqpoint{2.210483in}{1.453786in}}%
\pgfpathlineto{\pgfqpoint{2.165603in}{1.453109in}}%
\pgfpathlineto{\pgfqpoint{2.137377in}{1.425990in}}%
\pgfpathclose%
\pgfusepath{fill}%
\end{pgfscope}%
\begin{pgfscope}%
\pgfpathrectangle{\pgfqpoint{1.072000in}{0.528000in}}{\pgfqpoint{3.696000in}{3.696000in}}%
\pgfusepath{clip}%
\pgfsetbuttcap%
\pgfsetroundjoin%
\definecolor{currentfill}{rgb}{0.752704,0.157576,0.184258}%
\pgfsetfillcolor{currentfill}%
\pgfsetlinewidth{0.000000pt}%
\definecolor{currentstroke}{rgb}{0.000000,0.000000,0.000000}%
\pgfsetstrokecolor{currentstroke}%
\pgfsetdash{}{0pt}%
\pgfpathmoveto{\pgfqpoint{2.862549in}{3.309270in}}%
\pgfpathlineto{\pgfqpoint{2.908610in}{3.336755in}}%
\pgfpathlineto{\pgfqpoint{2.935453in}{3.318278in}}%
\pgfpathlineto{\pgfqpoint{2.889513in}{3.273558in}}%
\pgfpathlineto{\pgfqpoint{2.862549in}{3.309270in}}%
\pgfpathclose%
\pgfusepath{fill}%
\end{pgfscope}%
\begin{pgfscope}%
\pgfpathrectangle{\pgfqpoint{1.072000in}{0.528000in}}{\pgfqpoint{3.696000in}{3.696000in}}%
\pgfusepath{clip}%
\pgfsetbuttcap%
\pgfsetroundjoin%
\definecolor{currentfill}{rgb}{0.820401,0.286765,0.245160}%
\pgfsetfillcolor{currentfill}%
\pgfsetlinewidth{0.000000pt}%
\definecolor{currentstroke}{rgb}{0.000000,0.000000,0.000000}%
\pgfsetstrokecolor{currentstroke}%
\pgfsetdash{}{0pt}%
\pgfpathmoveto{\pgfqpoint{3.266436in}{3.267899in}}%
\pgfpathlineto{\pgfqpoint{3.312954in}{3.239842in}}%
\pgfpathlineto{\pgfqpoint{3.337856in}{3.156440in}}%
\pgfpathlineto{\pgfqpoint{3.291592in}{3.189008in}}%
\pgfpathlineto{\pgfqpoint{3.266436in}{3.267899in}}%
\pgfpathclose%
\pgfusepath{fill}%
\end{pgfscope}%
\begin{pgfscope}%
\pgfpathrectangle{\pgfqpoint{1.072000in}{0.528000in}}{\pgfqpoint{3.696000in}{3.696000in}}%
\pgfusepath{clip}%
\pgfsetbuttcap%
\pgfsetroundjoin%
\definecolor{currentfill}{rgb}{0.905783,0.455186,0.355336}%
\pgfsetfillcolor{currentfill}%
\pgfsetlinewidth{0.000000pt}%
\definecolor{currentstroke}{rgb}{0.000000,0.000000,0.000000}%
\pgfsetstrokecolor{currentstroke}%
\pgfsetdash{}{0pt}%
\pgfpathmoveto{\pgfqpoint{3.405690in}{3.143078in}}%
\pgfpathlineto{\pgfqpoint{3.451733in}{3.069378in}}%
\pgfpathlineto{\pgfqpoint{3.475637in}{2.965785in}}%
\pgfpathlineto{\pgfqpoint{3.429955in}{3.046004in}}%
\pgfpathlineto{\pgfqpoint{3.405690in}{3.143078in}}%
\pgfpathclose%
\pgfusepath{fill}%
\end{pgfscope}%
\begin{pgfscope}%
\pgfpathrectangle{\pgfqpoint{1.072000in}{0.528000in}}{\pgfqpoint{3.696000in}{3.696000in}}%
\pgfusepath{clip}%
\pgfsetbuttcap%
\pgfsetroundjoin%
\definecolor{currentfill}{rgb}{0.328604,0.439712,0.869587}%
\pgfsetfillcolor{currentfill}%
\pgfsetlinewidth{0.000000pt}%
\definecolor{currentstroke}{rgb}{0.000000,0.000000,0.000000}%
\pgfsetstrokecolor{currentstroke}%
\pgfsetdash{}{0pt}%
\pgfpathmoveto{\pgfqpoint{3.720060in}{1.632466in}}%
\pgfpathlineto{\pgfqpoint{3.764943in}{1.520190in}}%
\pgfpathlineto{\pgfqpoint{3.790025in}{1.547377in}}%
\pgfpathlineto{\pgfqpoint{3.745303in}{1.660796in}}%
\pgfpathlineto{\pgfqpoint{3.720060in}{1.632466in}}%
\pgfpathclose%
\pgfusepath{fill}%
\end{pgfscope}%
\begin{pgfscope}%
\pgfpathrectangle{\pgfqpoint{1.072000in}{0.528000in}}{\pgfqpoint{3.696000in}{3.696000in}}%
\pgfusepath{clip}%
\pgfsetbuttcap%
\pgfsetroundjoin%
\definecolor{currentfill}{rgb}{0.969522,0.700833,0.587508}%
\pgfsetfillcolor{currentfill}%
\pgfsetlinewidth{0.000000pt}%
\definecolor{currentstroke}{rgb}{0.000000,0.000000,0.000000}%
\pgfsetstrokecolor{currentstroke}%
\pgfsetdash{}{0pt}%
\pgfpathmoveto{\pgfqpoint{2.393178in}{2.641799in}}%
\pgfpathlineto{\pgfqpoint{2.436133in}{2.792291in}}%
\pgfpathlineto{\pgfqpoint{2.463116in}{2.868960in}}%
\pgfpathlineto{\pgfqpoint{2.420017in}{2.718487in}}%
\pgfpathlineto{\pgfqpoint{2.393178in}{2.641799in}}%
\pgfpathclose%
\pgfusepath{fill}%
\end{pgfscope}%
\begin{pgfscope}%
\pgfpathrectangle{\pgfqpoint{1.072000in}{0.528000in}}{\pgfqpoint{3.696000in}{3.696000in}}%
\pgfusepath{clip}%
\pgfsetbuttcap%
\pgfsetroundjoin%
\definecolor{currentfill}{rgb}{0.895885,0.433075,0.338681}%
\pgfsetfillcolor{currentfill}%
\pgfsetlinewidth{0.000000pt}%
\definecolor{currentstroke}{rgb}{0.000000,0.000000,0.000000}%
\pgfsetstrokecolor{currentstroke}%
\pgfsetdash{}{0pt}%
\pgfpathmoveto{\pgfqpoint{2.545063in}{3.003676in}}%
\pgfpathlineto{\pgfqpoint{2.589298in}{3.125791in}}%
\pgfpathlineto{\pgfqpoint{2.616838in}{3.140922in}}%
\pgfpathlineto{\pgfqpoint{2.572573in}{3.018056in}}%
\pgfpathlineto{\pgfqpoint{2.545063in}{3.003676in}}%
\pgfpathclose%
\pgfusepath{fill}%
\end{pgfscope}%
\begin{pgfscope}%
\pgfpathrectangle{\pgfqpoint{1.072000in}{0.528000in}}{\pgfqpoint{3.696000in}{3.696000in}}%
\pgfusepath{clip}%
\pgfsetbuttcap%
\pgfsetroundjoin%
\definecolor{currentfill}{rgb}{0.940879,0.805596,0.735167}%
\pgfsetfillcolor{currentfill}%
\pgfsetlinewidth{0.000000pt}%
\definecolor{currentstroke}{rgb}{0.000000,0.000000,0.000000}%
\pgfsetstrokecolor{currentstroke}%
\pgfsetdash{}{0pt}%
\pgfpathmoveto{\pgfqpoint{2.340204in}{2.445566in}}%
\pgfpathlineto{\pgfqpoint{2.382900in}{2.590982in}}%
\pgfpathlineto{\pgfqpoint{2.409385in}{2.699408in}}%
\pgfpathlineto{\pgfqpoint{2.366570in}{2.550372in}}%
\pgfpathlineto{\pgfqpoint{2.340204in}{2.445566in}}%
\pgfpathclose%
\pgfusepath{fill}%
\end{pgfscope}%
\begin{pgfscope}%
\pgfpathrectangle{\pgfqpoint{1.072000in}{0.528000in}}{\pgfqpoint{3.696000in}{3.696000in}}%
\pgfusepath{clip}%
\pgfsetbuttcap%
\pgfsetroundjoin%
\definecolor{currentfill}{rgb}{0.483854,0.622050,0.974808}%
\pgfsetfillcolor{currentfill}%
\pgfsetlinewidth{0.000000pt}%
\definecolor{currentstroke}{rgb}{0.000000,0.000000,0.000000}%
\pgfsetstrokecolor{currentstroke}%
\pgfsetdash{}{0pt}%
\pgfpathmoveto{\pgfqpoint{2.245680in}{1.718741in}}%
\pgfpathlineto{\pgfqpoint{2.289524in}{1.777184in}}%
\pgfpathlineto{\pgfqpoint{2.315063in}{1.929780in}}%
\pgfpathlineto{\pgfqpoint{2.271405in}{1.853797in}}%
\pgfpathlineto{\pgfqpoint{2.245680in}{1.718741in}}%
\pgfpathclose%
\pgfusepath{fill}%
\end{pgfscope}%
\begin{pgfscope}%
\pgfpathrectangle{\pgfqpoint{1.072000in}{0.528000in}}{\pgfqpoint{3.696000in}{3.696000in}}%
\pgfusepath{clip}%
\pgfsetbuttcap%
\pgfsetroundjoin%
\definecolor{currentfill}{rgb}{0.839365,0.321856,0.264924}%
\pgfsetfillcolor{currentfill}%
\pgfsetlinewidth{0.000000pt}%
\definecolor{currentstroke}{rgb}{0.000000,0.000000,0.000000}%
\pgfsetstrokecolor{currentstroke}%
\pgfsetdash{}{0pt}%
\pgfpathmoveto{\pgfqpoint{3.312954in}{3.239842in}}%
\pgfpathlineto{\pgfqpoint{3.359399in}{3.199096in}}%
\pgfpathlineto{\pgfqpoint{3.384002in}{3.109209in}}%
\pgfpathlineto{\pgfqpoint{3.337856in}{3.156440in}}%
\pgfpathlineto{\pgfqpoint{3.312954in}{3.239842in}}%
\pgfpathclose%
\pgfusepath{fill}%
\end{pgfscope}%
\begin{pgfscope}%
\pgfpathrectangle{\pgfqpoint{1.072000in}{0.528000in}}{\pgfqpoint{3.696000in}{3.696000in}}%
\pgfusepath{clip}%
\pgfsetbuttcap%
\pgfsetroundjoin%
\definecolor{currentfill}{rgb}{0.430507,0.564883,0.948889}%
\pgfsetfillcolor{currentfill}%
\pgfsetlinewidth{0.000000pt}%
\definecolor{currentstroke}{rgb}{0.000000,0.000000,0.000000}%
\pgfsetstrokecolor{currentstroke}%
\pgfsetdash{}{0pt}%
\pgfpathmoveto{\pgfqpoint{3.701154in}{1.808678in}}%
\pgfpathlineto{\pgfqpoint{3.745303in}{1.660796in}}%
\pgfpathlineto{\pgfqpoint{3.770278in}{1.681722in}}%
\pgfpathlineto{\pgfqpoint{3.726264in}{1.828097in}}%
\pgfpathlineto{\pgfqpoint{3.701154in}{1.808678in}}%
\pgfpathclose%
\pgfusepath{fill}%
\end{pgfscope}%
\begin{pgfscope}%
\pgfpathrectangle{\pgfqpoint{1.072000in}{0.528000in}}{\pgfqpoint{3.696000in}{3.696000in}}%
\pgfusepath{clip}%
\pgfsetbuttcap%
\pgfsetroundjoin%
\definecolor{currentfill}{rgb}{0.869655,0.379274,0.300941}%
\pgfsetfillcolor{currentfill}%
\pgfsetlinewidth{0.000000pt}%
\definecolor{currentstroke}{rgb}{0.000000,0.000000,0.000000}%
\pgfsetstrokecolor{currentstroke}%
\pgfsetdash{}{0pt}%
\pgfpathmoveto{\pgfqpoint{3.359399in}{3.199096in}}%
\pgfpathlineto{\pgfqpoint{3.405690in}{3.143078in}}%
\pgfpathlineto{\pgfqpoint{3.429955in}{3.046004in}}%
\pgfpathlineto{\pgfqpoint{3.384002in}{3.109209in}}%
\pgfpathlineto{\pgfqpoint{3.359399in}{3.199096in}}%
\pgfpathclose%
\pgfusepath{fill}%
\end{pgfscope}%
\begin{pgfscope}%
\pgfpathrectangle{\pgfqpoint{1.072000in}{0.528000in}}{\pgfqpoint{3.696000in}{3.696000in}}%
\pgfusepath{clip}%
\pgfsetbuttcap%
\pgfsetroundjoin%
\definecolor{currentfill}{rgb}{0.962708,0.753557,0.655601}%
\pgfsetfillcolor{currentfill}%
\pgfsetlinewidth{0.000000pt}%
\definecolor{currentstroke}{rgb}{0.000000,0.000000,0.000000}%
\pgfsetstrokecolor{currentstroke}%
\pgfsetdash{}{0pt}%
\pgfpathmoveto{\pgfqpoint{2.366570in}{2.550372in}}%
\pgfpathlineto{\pgfqpoint{2.409385in}{2.699408in}}%
\pgfpathlineto{\pgfqpoint{2.436133in}{2.792291in}}%
\pgfpathlineto{\pgfqpoint{2.393178in}{2.641799in}}%
\pgfpathlineto{\pgfqpoint{2.366570in}{2.550372in}}%
\pgfpathclose%
\pgfusepath{fill}%
\end{pgfscope}%
\begin{pgfscope}%
\pgfpathrectangle{\pgfqpoint{1.072000in}{0.528000in}}{\pgfqpoint{3.696000in}{3.696000in}}%
\pgfusepath{clip}%
\pgfsetbuttcap%
\pgfsetroundjoin%
\definecolor{currentfill}{rgb}{0.790562,0.231397,0.216242}%
\pgfsetfillcolor{currentfill}%
\pgfsetlinewidth{0.000000pt}%
\definecolor{currentstroke}{rgb}{0.000000,0.000000,0.000000}%
\pgfsetstrokecolor{currentstroke}%
\pgfsetdash{}{0pt}%
\pgfpathmoveto{\pgfqpoint{2.716820in}{3.228208in}}%
\pgfpathlineto{\pgfqpoint{2.762333in}{3.291704in}}%
\pgfpathlineto{\pgfqpoint{2.789618in}{3.281154in}}%
\pgfpathlineto{\pgfqpoint{2.744221in}{3.206306in}}%
\pgfpathlineto{\pgfqpoint{2.716820in}{3.228208in}}%
\pgfpathclose%
\pgfusepath{fill}%
\end{pgfscope}%
\begin{pgfscope}%
\pgfpathrectangle{\pgfqpoint{1.072000in}{0.528000in}}{\pgfqpoint{3.696000in}{3.696000in}}%
\pgfusepath{clip}%
\pgfsetbuttcap%
\pgfsetroundjoin%
\definecolor{currentfill}{rgb}{0.763520,0.178667,0.193396}%
\pgfsetfillcolor{currentfill}%
\pgfsetlinewidth{0.000000pt}%
\definecolor{currentstroke}{rgb}{0.000000,0.000000,0.000000}%
\pgfsetstrokecolor{currentstroke}%
\pgfsetdash{}{0pt}%
\pgfpathmoveto{\pgfqpoint{2.789618in}{3.281154in}}%
\pgfpathlineto{\pgfqpoint{2.835476in}{3.322092in}}%
\pgfpathlineto{\pgfqpoint{2.862549in}{3.309270in}}%
\pgfpathlineto{\pgfqpoint{2.816823in}{3.252762in}}%
\pgfpathlineto{\pgfqpoint{2.789618in}{3.281154in}}%
\pgfpathclose%
\pgfusepath{fill}%
\end{pgfscope}%
\begin{pgfscope}%
\pgfpathrectangle{\pgfqpoint{1.072000in}{0.528000in}}{\pgfqpoint{3.696000in}{3.696000in}}%
\pgfusepath{clip}%
\pgfsetbuttcap%
\pgfsetroundjoin%
\definecolor{currentfill}{rgb}{0.494638,0.633022,0.978983}%
\pgfsetfillcolor{currentfill}%
\pgfsetlinewidth{0.000000pt}%
\definecolor{currentstroke}{rgb}{0.000000,0.000000,0.000000}%
\pgfsetstrokecolor{currentstroke}%
\pgfsetdash{}{0pt}%
\pgfpathmoveto{\pgfqpoint{1.809567in}{1.931730in}}%
\pgfpathlineto{\pgfqpoint{1.858085in}{1.812410in}}%
\pgfpathlineto{\pgfqpoint{1.890197in}{1.735553in}}%
\pgfpathlineto{\pgfqpoint{1.842310in}{1.844241in}}%
\pgfpathlineto{\pgfqpoint{1.809567in}{1.931730in}}%
\pgfpathclose%
\pgfusepath{fill}%
\end{pgfscope}%
\begin{pgfscope}%
\pgfpathrectangle{\pgfqpoint{1.072000in}{0.528000in}}{\pgfqpoint{3.696000in}{3.696000in}}%
\pgfusepath{clip}%
\pgfsetbuttcap%
\pgfsetroundjoin%
\definecolor{currentfill}{rgb}{0.950956,0.786875,0.704761}%
\pgfsetfillcolor{currentfill}%
\pgfsetlinewidth{0.000000pt}%
\definecolor{currentstroke}{rgb}{0.000000,0.000000,0.000000}%
\pgfsetstrokecolor{currentstroke}%
\pgfsetdash{}{0pt}%
\pgfpathmoveto{\pgfqpoint{2.960170in}{2.626493in}}%
\pgfpathlineto{\pgfqpoint{3.006282in}{2.430814in}}%
\pgfpathlineto{\pgfqpoint{3.033087in}{2.636092in}}%
\pgfpathlineto{\pgfqpoint{2.986881in}{2.722478in}}%
\pgfpathlineto{\pgfqpoint{2.960170in}{2.626493in}}%
\pgfpathclose%
\pgfusepath{fill}%
\end{pgfscope}%
\begin{pgfscope}%
\pgfpathrectangle{\pgfqpoint{1.072000in}{0.528000in}}{\pgfqpoint{3.696000in}{3.696000in}}%
\pgfusepath{clip}%
\pgfsetbuttcap%
\pgfsetroundjoin%
\definecolor{currentfill}{rgb}{0.275827,0.366717,0.812553}%
\pgfsetfillcolor{currentfill}%
\pgfsetlinewidth{0.000000pt}%
\definecolor{currentstroke}{rgb}{0.000000,0.000000,0.000000}%
\pgfsetstrokecolor{currentstroke}%
\pgfsetdash{}{0pt}%
\pgfpathmoveto{\pgfqpoint{2.210483in}{1.453786in}}%
\pgfpathlineto{\pgfqpoint{2.255363in}{1.461256in}}%
\pgfpathlineto{\pgfqpoint{2.281968in}{1.558650in}}%
\pgfpathlineto{\pgfqpoint{2.237418in}{1.533315in}}%
\pgfpathlineto{\pgfqpoint{2.210483in}{1.453786in}}%
\pgfpathclose%
\pgfusepath{fill}%
\end{pgfscope}%
\begin{pgfscope}%
\pgfpathrectangle{\pgfqpoint{1.072000in}{0.528000in}}{\pgfqpoint{3.696000in}{3.696000in}}%
\pgfusepath{clip}%
\pgfsetbuttcap%
\pgfsetroundjoin%
\definecolor{currentfill}{rgb}{0.746838,0.140021,0.179996}%
\pgfsetfillcolor{currentfill}%
\pgfsetlinewidth{0.000000pt}%
\definecolor{currentstroke}{rgb}{0.000000,0.000000,0.000000}%
\pgfsetstrokecolor{currentstroke}%
\pgfsetdash{}{0pt}%
\pgfpathmoveto{\pgfqpoint{3.100896in}{3.333720in}}%
\pgfpathlineto{\pgfqpoint{3.147415in}{3.330153in}}%
\pgfpathlineto{\pgfqpoint{3.173419in}{3.292607in}}%
\pgfpathlineto{\pgfqpoint{3.127003in}{3.289309in}}%
\pgfpathlineto{\pgfqpoint{3.100896in}{3.333720in}}%
\pgfpathclose%
\pgfusepath{fill}%
\end{pgfscope}%
\begin{pgfscope}%
\pgfpathrectangle{\pgfqpoint{1.072000in}{0.528000in}}{\pgfqpoint{3.696000in}{3.696000in}}%
\pgfusepath{clip}%
\pgfsetbuttcap%
\pgfsetroundjoin%
\definecolor{currentfill}{rgb}{0.257234,0.339661,0.789661}%
\pgfsetfillcolor{currentfill}%
\pgfsetlinewidth{0.000000pt}%
\definecolor{currentstroke}{rgb}{0.000000,0.000000,0.000000}%
\pgfsetstrokecolor{currentstroke}%
\pgfsetdash{}{0pt}%
\pgfpathmoveto{\pgfqpoint{3.739661in}{1.488464in}}%
\pgfpathlineto{\pgfqpoint{3.785571in}{1.420044in}}%
\pgfpathlineto{\pgfqpoint{3.810662in}{1.448011in}}%
\pgfpathlineto{\pgfqpoint{3.764943in}{1.520190in}}%
\pgfpathlineto{\pgfqpoint{3.739661in}{1.488464in}}%
\pgfpathclose%
\pgfusepath{fill}%
\end{pgfscope}%
\begin{pgfscope}%
\pgfpathrectangle{\pgfqpoint{1.072000in}{0.528000in}}{\pgfqpoint{3.696000in}{3.696000in}}%
\pgfusepath{clip}%
\pgfsetbuttcap%
\pgfsetroundjoin%
\definecolor{currentfill}{rgb}{0.729196,0.086679,0.167240}%
\pgfsetfillcolor{currentfill}%
\pgfsetlinewidth{0.000000pt}%
\definecolor{currentstroke}{rgb}{0.000000,0.000000,0.000000}%
\pgfsetstrokecolor{currentstroke}%
\pgfsetdash{}{0pt}%
\pgfpathmoveto{\pgfqpoint{2.981639in}{3.340125in}}%
\pgfpathlineto{\pgfqpoint{3.027992in}{3.343829in}}%
\pgfpathlineto{\pgfqpoint{3.054485in}{3.327819in}}%
\pgfpathlineto{\pgfqpoint{3.008213in}{3.308212in}}%
\pgfpathlineto{\pgfqpoint{2.981639in}{3.340125in}}%
\pgfpathclose%
\pgfusepath{fill}%
\end{pgfscope}%
\begin{pgfscope}%
\pgfpathrectangle{\pgfqpoint{1.072000in}{0.528000in}}{\pgfqpoint{3.696000in}{3.696000in}}%
\pgfusepath{clip}%
\pgfsetbuttcap%
\pgfsetroundjoin%
\definecolor{currentfill}{rgb}{0.343278,0.459354,0.884122}%
\pgfsetfillcolor{currentfill}%
\pgfsetlinewidth{0.000000pt}%
\definecolor{currentstroke}{rgb}{0.000000,0.000000,0.000000}%
\pgfsetstrokecolor{currentstroke}%
\pgfsetdash{}{0pt}%
\pgfpathmoveto{\pgfqpoint{2.237418in}{1.533315in}}%
\pgfpathlineto{\pgfqpoint{2.281968in}{1.558650in}}%
\pgfpathlineto{\pgfqpoint{2.307997in}{1.685966in}}%
\pgfpathlineto{\pgfqpoint{2.263704in}{1.642892in}}%
\pgfpathlineto{\pgfqpoint{2.237418in}{1.533315in}}%
\pgfpathclose%
\pgfusepath{fill}%
\end{pgfscope}%
\begin{pgfscope}%
\pgfpathrectangle{\pgfqpoint{1.072000in}{0.528000in}}{\pgfqpoint{3.696000in}{3.696000in}}%
\pgfusepath{clip}%
\pgfsetbuttcap%
\pgfsetroundjoin%
\definecolor{currentfill}{rgb}{0.229806,0.298718,0.753683}%
\pgfsetfillcolor{currentfill}%
\pgfsetlinewidth{0.000000pt}%
\definecolor{currentstroke}{rgb}{0.000000,0.000000,0.000000}%
\pgfsetstrokecolor{currentstroke}%
\pgfsetdash{}{0pt}%
\pgfpathmoveto{\pgfqpoint{3.785571in}{1.420044in}}%
\pgfpathlineto{\pgfqpoint{3.832548in}{1.394798in}}%
\pgfpathlineto{\pgfqpoint{3.857473in}{1.419204in}}%
\pgfpathlineto{\pgfqpoint{3.810662in}{1.448011in}}%
\pgfpathlineto{\pgfqpoint{3.785571in}{1.420044in}}%
\pgfpathclose%
\pgfusepath{fill}%
\end{pgfscope}%
\begin{pgfscope}%
\pgfpathrectangle{\pgfqpoint{1.072000in}{0.528000in}}{\pgfqpoint{3.696000in}{3.696000in}}%
\pgfusepath{clip}%
\pgfsetbuttcap%
\pgfsetroundjoin%
\definecolor{currentfill}{rgb}{0.271104,0.360011,0.807095}%
\pgfsetfillcolor{currentfill}%
\pgfsetlinewidth{0.000000pt}%
\definecolor{currentstroke}{rgb}{0.000000,0.000000,0.000000}%
\pgfsetstrokecolor{currentstroke}%
\pgfsetdash{}{0pt}%
\pgfpathmoveto{\pgfqpoint{2.015614in}{1.536508in}}%
\pgfpathlineto{\pgfqpoint{2.062045in}{1.478868in}}%
\pgfpathlineto{\pgfqpoint{2.091849in}{1.454104in}}%
\pgfpathlineto{\pgfqpoint{2.046042in}{1.494815in}}%
\pgfpathlineto{\pgfqpoint{2.015614in}{1.536508in}}%
\pgfpathclose%
\pgfusepath{fill}%
\end{pgfscope}%
\begin{pgfscope}%
\pgfpathrectangle{\pgfqpoint{1.072000in}{0.528000in}}{\pgfqpoint{3.696000in}{3.696000in}}%
\pgfusepath{clip}%
\pgfsetbuttcap%
\pgfsetroundjoin%
\definecolor{currentfill}{rgb}{0.960581,0.762501,0.667964}%
\pgfsetfillcolor{currentfill}%
\pgfsetlinewidth{0.000000pt}%
\definecolor{currentstroke}{rgb}{0.000000,0.000000,0.000000}%
\pgfsetstrokecolor{currentstroke}%
\pgfsetdash{}{0pt}%
\pgfpathmoveto{\pgfqpoint{3.585074in}{2.762686in}}%
\pgfpathlineto{\pgfqpoint{3.629445in}{2.601312in}}%
\pgfpathlineto{\pgfqpoint{3.652836in}{2.524938in}}%
\pgfpathlineto{\pgfqpoint{3.608723in}{2.683667in}}%
\pgfpathlineto{\pgfqpoint{3.585074in}{2.762686in}}%
\pgfpathclose%
\pgfusepath{fill}%
\end{pgfscope}%
\begin{pgfscope}%
\pgfpathrectangle{\pgfqpoint{1.072000in}{0.528000in}}{\pgfqpoint{3.696000in}{3.696000in}}%
\pgfusepath{clip}%
\pgfsetbuttcap%
\pgfsetroundjoin%
\definecolor{currentfill}{rgb}{0.743754,0.825125,0.965798}%
\pgfsetfillcolor{currentfill}%
\pgfsetlinewidth{0.000000pt}%
\definecolor{currentstroke}{rgb}{0.000000,0.000000,0.000000}%
\pgfsetstrokecolor{currentstroke}%
\pgfsetdash{}{0pt}%
\pgfpathmoveto{\pgfqpoint{2.279349in}{2.051663in}}%
\pgfpathlineto{\pgfqpoint{2.322438in}{2.157381in}}%
\pgfpathlineto{\pgfqpoint{2.348037in}{2.312794in}}%
\pgfpathlineto{\pgfqpoint{2.304974in}{2.194629in}}%
\pgfpathlineto{\pgfqpoint{2.279349in}{2.051663in}}%
\pgfpathclose%
\pgfusepath{fill}%
\end{pgfscope}%
\begin{pgfscope}%
\pgfpathrectangle{\pgfqpoint{1.072000in}{0.528000in}}{\pgfqpoint{3.696000in}{3.696000in}}%
\pgfusepath{clip}%
\pgfsetbuttcap%
\pgfsetroundjoin%
\definecolor{currentfill}{rgb}{0.309060,0.413498,0.850128}%
\pgfsetfillcolor{currentfill}%
\pgfsetlinewidth{0.000000pt}%
\definecolor{currentstroke}{rgb}{0.000000,0.000000,0.000000}%
\pgfsetstrokecolor{currentstroke}%
\pgfsetdash{}{0pt}%
\pgfpathmoveto{\pgfqpoint{1.968768in}{1.607788in}}%
\pgfpathlineto{\pgfqpoint{2.015614in}{1.536508in}}%
\pgfpathlineto{\pgfqpoint{2.046042in}{1.494815in}}%
\pgfpathlineto{\pgfqpoint{1.999853in}{1.549989in}}%
\pgfpathlineto{\pgfqpoint{1.968768in}{1.607788in}}%
\pgfpathclose%
\pgfusepath{fill}%
\end{pgfscope}%
\begin{pgfscope}%
\pgfpathrectangle{\pgfqpoint{1.072000in}{0.528000in}}{\pgfqpoint{3.696000in}{3.696000in}}%
\pgfusepath{clip}%
\pgfsetbuttcap%
\pgfsetroundjoin%
\definecolor{currentfill}{rgb}{0.962701,0.628218,0.507636}%
\pgfsetfillcolor{currentfill}%
\pgfsetlinewidth{0.000000pt}%
\definecolor{currentstroke}{rgb}{0.000000,0.000000,0.000000}%
\pgfsetstrokecolor{currentstroke}%
\pgfsetdash{}{0pt}%
\pgfpathmoveto{\pgfqpoint{3.518946in}{2.956615in}}%
\pgfpathlineto{\pgfqpoint{3.564101in}{2.829246in}}%
\pgfpathlineto{\pgfqpoint{3.587527in}{2.734716in}}%
\pgfpathlineto{\pgfqpoint{3.542715in}{2.864381in}}%
\pgfpathlineto{\pgfqpoint{3.518946in}{2.956615in}}%
\pgfpathclose%
\pgfusepath{fill}%
\end{pgfscope}%
\begin{pgfscope}%
\pgfpathrectangle{\pgfqpoint{1.072000in}{0.528000in}}{\pgfqpoint{3.696000in}{3.696000in}}%
\pgfusepath{clip}%
\pgfsetbuttcap%
\pgfsetroundjoin%
\definecolor{currentfill}{rgb}{0.565182,0.699438,0.996635}%
\pgfsetfillcolor{currentfill}%
\pgfsetlinewidth{0.000000pt}%
\definecolor{currentstroke}{rgb}{0.000000,0.000000,0.000000}%
\pgfsetstrokecolor{currentstroke}%
\pgfsetdash{}{0pt}%
\pgfpathmoveto{\pgfqpoint{3.682551in}{2.001337in}}%
\pgfpathlineto{\pgfqpoint{3.726264in}{1.828097in}}%
\pgfpathlineto{\pgfqpoint{3.751046in}{1.837837in}}%
\pgfpathlineto{\pgfqpoint{3.707454in}{2.007333in}}%
\pgfpathlineto{\pgfqpoint{3.682551in}{2.001337in}}%
\pgfpathclose%
\pgfusepath{fill}%
\end{pgfscope}%
\begin{pgfscope}%
\pgfpathrectangle{\pgfqpoint{1.072000in}{0.528000in}}{\pgfqpoint{3.696000in}{3.696000in}}%
\pgfusepath{clip}%
\pgfsetbuttcap%
\pgfsetroundjoin%
\definecolor{currentfill}{rgb}{0.238948,0.312365,0.765676}%
\pgfsetfillcolor{currentfill}%
\pgfsetlinewidth{0.000000pt}%
\definecolor{currentstroke}{rgb}{0.000000,0.000000,0.000000}%
\pgfsetstrokecolor{currentstroke}%
\pgfsetdash{}{0pt}%
\pgfpathmoveto{\pgfqpoint{2.182729in}{1.408245in}}%
\pgfpathlineto{\pgfqpoint{2.228001in}{1.398444in}}%
\pgfpathlineto{\pgfqpoint{2.255363in}{1.461256in}}%
\pgfpathlineto{\pgfqpoint{2.210483in}{1.453786in}}%
\pgfpathlineto{\pgfqpoint{2.182729in}{1.408245in}}%
\pgfpathclose%
\pgfusepath{fill}%
\end{pgfscope}%
\begin{pgfscope}%
\pgfpathrectangle{\pgfqpoint{1.072000in}{0.528000in}}{\pgfqpoint{3.696000in}{3.696000in}}%
\pgfusepath{clip}%
\pgfsetbuttcap%
\pgfsetroundjoin%
\definecolor{currentfill}{rgb}{0.248091,0.326013,0.777669}%
\pgfsetfillcolor{currentfill}%
\pgfsetlinewidth{0.000000pt}%
\definecolor{currentstroke}{rgb}{0.000000,0.000000,0.000000}%
\pgfsetstrokecolor{currentstroke}%
\pgfsetdash{}{0pt}%
\pgfpathmoveto{\pgfqpoint{2.062045in}{1.478868in}}%
\pgfpathlineto{\pgfqpoint{2.108148in}{1.433586in}}%
\pgfpathlineto{\pgfqpoint{2.137377in}{1.425990in}}%
\pgfpathlineto{\pgfqpoint{2.091849in}{1.454104in}}%
\pgfpathlineto{\pgfqpoint{2.062045in}{1.478868in}}%
\pgfpathclose%
\pgfusepath{fill}%
\end{pgfscope}%
\begin{pgfscope}%
\pgfpathrectangle{\pgfqpoint{1.072000in}{0.528000in}}{\pgfqpoint{3.696000in}{3.696000in}}%
\pgfusepath{clip}%
\pgfsetbuttcap%
\pgfsetroundjoin%
\definecolor{currentfill}{rgb}{0.358415,0.478426,0.896795}%
\pgfsetfillcolor{currentfill}%
\pgfsetlinewidth{0.000000pt}%
\definecolor{currentstroke}{rgb}{0.000000,0.000000,0.000000}%
\pgfsetstrokecolor{currentstroke}%
\pgfsetdash{}{0pt}%
\pgfpathmoveto{\pgfqpoint{1.921430in}{1.693462in}}%
\pgfpathlineto{\pgfqpoint{1.968768in}{1.607788in}}%
\pgfpathlineto{\pgfqpoint{1.999853in}{1.549989in}}%
\pgfpathlineto{\pgfqpoint{1.953183in}{1.620984in}}%
\pgfpathlineto{\pgfqpoint{1.921430in}{1.693462in}}%
\pgfpathclose%
\pgfusepath{fill}%
\end{pgfscope}%
\begin{pgfscope}%
\pgfpathrectangle{\pgfqpoint{1.072000in}{0.528000in}}{\pgfqpoint{3.696000in}{3.696000in}}%
\pgfusepath{clip}%
\pgfsetbuttcap%
\pgfsetroundjoin%
\definecolor{currentfill}{rgb}{0.950956,0.786875,0.704761}%
\pgfsetfillcolor{currentfill}%
\pgfsetlinewidth{0.000000pt}%
\definecolor{currentstroke}{rgb}{0.000000,0.000000,0.000000}%
\pgfsetstrokecolor{currentstroke}%
\pgfsetdash{}{0pt}%
\pgfpathmoveto{\pgfqpoint{3.006282in}{2.430814in}}%
\pgfpathlineto{\pgfqpoint{3.052886in}{2.600187in}}%
\pgfpathlineto{\pgfqpoint{3.079668in}{2.696393in}}%
\pgfpathlineto{\pgfqpoint{3.033087in}{2.636092in}}%
\pgfpathlineto{\pgfqpoint{3.006282in}{2.430814in}}%
\pgfpathclose%
\pgfusepath{fill}%
\end{pgfscope}%
\begin{pgfscope}%
\pgfpathrectangle{\pgfqpoint{1.072000in}{0.528000in}}{\pgfqpoint{3.696000in}{3.696000in}}%
\pgfusepath{clip}%
\pgfsetbuttcap%
\pgfsetroundjoin%
\definecolor{currentfill}{rgb}{0.323718,0.433158,0.864722}%
\pgfsetfillcolor{currentfill}%
\pgfsetlinewidth{0.000000pt}%
\definecolor{currentstroke}{rgb}{0.000000,0.000000,0.000000}%
\pgfsetstrokecolor{currentstroke}%
\pgfsetdash{}{0pt}%
\pgfpathmoveto{\pgfqpoint{3.694569in}{1.596950in}}%
\pgfpathlineto{\pgfqpoint{3.739661in}{1.488464in}}%
\pgfpathlineto{\pgfqpoint{3.764943in}{1.520190in}}%
\pgfpathlineto{\pgfqpoint{3.720060in}{1.632466in}}%
\pgfpathlineto{\pgfqpoint{3.694569in}{1.596950in}}%
\pgfpathclose%
\pgfusepath{fill}%
\end{pgfscope}%
\begin{pgfscope}%
\pgfpathrectangle{\pgfqpoint{1.072000in}{0.528000in}}{\pgfqpoint{3.696000in}{3.696000in}}%
\pgfusepath{clip}%
\pgfsetbuttcap%
\pgfsetroundjoin%
\definecolor{currentfill}{rgb}{0.603162,0.731527,0.999565}%
\pgfsetfillcolor{currentfill}%
\pgfsetlinewidth{0.000000pt}%
\definecolor{currentstroke}{rgb}{0.000000,0.000000,0.000000}%
\pgfsetstrokecolor{currentstroke}%
\pgfsetdash{}{0pt}%
\pgfpathmoveto{\pgfqpoint{2.271405in}{1.853797in}}%
\pgfpathlineto{\pgfqpoint{2.315063in}{1.929780in}}%
\pgfpathlineto{\pgfqpoint{2.340496in}{2.093593in}}%
\pgfpathlineto{\pgfqpoint{2.296938in}{2.002218in}}%
\pgfpathlineto{\pgfqpoint{2.271405in}{1.853797in}}%
\pgfpathclose%
\pgfusepath{fill}%
\end{pgfscope}%
\begin{pgfscope}%
\pgfpathrectangle{\pgfqpoint{1.072000in}{0.528000in}}{\pgfqpoint{3.696000in}{3.696000in}}%
\pgfusepath{clip}%
\pgfsetbuttcap%
\pgfsetroundjoin%
\definecolor{currentfill}{rgb}{0.234377,0.305542,0.759680}%
\pgfsetfillcolor{currentfill}%
\pgfsetlinewidth{0.000000pt}%
\definecolor{currentstroke}{rgb}{0.000000,0.000000,0.000000}%
\pgfsetstrokecolor{currentstroke}%
\pgfsetdash{}{0pt}%
\pgfpathmoveto{\pgfqpoint{2.108148in}{1.433586in}}%
\pgfpathlineto{\pgfqpoint{2.154013in}{1.398990in}}%
\pgfpathlineto{\pgfqpoint{2.182729in}{1.408245in}}%
\pgfpathlineto{\pgfqpoint{2.137377in}{1.425990in}}%
\pgfpathlineto{\pgfqpoint{2.108148in}{1.433586in}}%
\pgfpathclose%
\pgfusepath{fill}%
\end{pgfscope}%
\begin{pgfscope}%
\pgfpathrectangle{\pgfqpoint{1.072000in}{0.528000in}}{\pgfqpoint{3.696000in}{3.696000in}}%
\pgfusepath{clip}%
\pgfsetbuttcap%
\pgfsetroundjoin%
\definecolor{currentfill}{rgb}{0.717435,0.051118,0.158737}%
\pgfsetfillcolor{currentfill}%
\pgfsetlinewidth{0.000000pt}%
\definecolor{currentstroke}{rgb}{0.000000,0.000000,0.000000}%
\pgfsetstrokecolor{currentstroke}%
\pgfsetdash{}{0pt}%
\pgfpathmoveto{\pgfqpoint{2.908610in}{3.336755in}}%
\pgfpathlineto{\pgfqpoint{2.954893in}{3.338743in}}%
\pgfpathlineto{\pgfqpoint{2.981639in}{3.340125in}}%
\pgfpathlineto{\pgfqpoint{2.935453in}{3.318278in}}%
\pgfpathlineto{\pgfqpoint{2.908610in}{3.336755in}}%
\pgfpathclose%
\pgfusepath{fill}%
\end{pgfscope}%
\begin{pgfscope}%
\pgfpathrectangle{\pgfqpoint{1.072000in}{0.528000in}}{\pgfqpoint{3.696000in}{3.696000in}}%
\pgfusepath{clip}%
\pgfsetbuttcap%
\pgfsetroundjoin%
\definecolor{currentfill}{rgb}{0.441123,0.576532,0.954545}%
\pgfsetfillcolor{currentfill}%
\pgfsetlinewidth{0.000000pt}%
\definecolor{currentstroke}{rgb}{0.000000,0.000000,0.000000}%
\pgfsetstrokecolor{currentstroke}%
\pgfsetdash{}{0pt}%
\pgfpathmoveto{\pgfqpoint{2.263704in}{1.642892in}}%
\pgfpathlineto{\pgfqpoint{2.307997in}{1.685966in}}%
\pgfpathlineto{\pgfqpoint{2.333639in}{1.837096in}}%
\pgfpathlineto{\pgfqpoint{2.289524in}{1.777184in}}%
\pgfpathlineto{\pgfqpoint{2.263704in}{1.642892in}}%
\pgfpathclose%
\pgfusepath{fill}%
\end{pgfscope}%
\begin{pgfscope}%
\pgfpathrectangle{\pgfqpoint{1.072000in}{0.528000in}}{\pgfqpoint{3.696000in}{3.696000in}}%
\pgfusepath{clip}%
\pgfsetbuttcap%
\pgfsetroundjoin%
\definecolor{currentfill}{rgb}{0.724041,0.814910,0.975651}%
\pgfsetfillcolor{currentfill}%
\pgfsetlinewidth{0.000000pt}%
\definecolor{currentstroke}{rgb}{0.000000,0.000000,0.000000}%
\pgfsetstrokecolor{currentstroke}%
\pgfsetdash{}{0pt}%
\pgfpathmoveto{\pgfqpoint{1.694471in}{2.259769in}}%
\pgfpathlineto{\pgfqpoint{1.744393in}{2.109478in}}%
\pgfpathlineto{\pgfqpoint{1.777442in}{2.021573in}}%
\pgfpathlineto{\pgfqpoint{1.727946in}{2.168901in}}%
\pgfpathlineto{\pgfqpoint{1.694471in}{2.259769in}}%
\pgfpathclose%
\pgfusepath{fill}%
\end{pgfscope}%
\begin{pgfscope}%
\pgfpathrectangle{\pgfqpoint{1.072000in}{0.528000in}}{\pgfqpoint{3.696000in}{3.696000in}}%
\pgfusepath{clip}%
\pgfsetbuttcap%
\pgfsetroundjoin%
\definecolor{currentfill}{rgb}{0.740957,0.122240,0.175744}%
\pgfsetfillcolor{currentfill}%
\pgfsetlinewidth{0.000000pt}%
\definecolor{currentstroke}{rgb}{0.000000,0.000000,0.000000}%
\pgfsetstrokecolor{currentstroke}%
\pgfsetdash{}{0pt}%
\pgfpathmoveto{\pgfqpoint{3.147415in}{3.330153in}}%
\pgfpathlineto{\pgfqpoint{3.194026in}{3.320330in}}%
\pgfpathlineto{\pgfqpoint{3.219909in}{3.285237in}}%
\pgfpathlineto{\pgfqpoint{3.173419in}{3.292607in}}%
\pgfpathlineto{\pgfqpoint{3.147415in}{3.330153in}}%
\pgfpathclose%
\pgfusepath{fill}%
\end{pgfscope}%
\begin{pgfscope}%
\pgfpathrectangle{\pgfqpoint{1.072000in}{0.528000in}}{\pgfqpoint{3.696000in}{3.696000in}}%
\pgfusepath{clip}%
\pgfsetbuttcap%
\pgfsetroundjoin%
\definecolor{currentfill}{rgb}{0.698454,0.799450,0.984577}%
\pgfsetfillcolor{currentfill}%
\pgfsetlinewidth{0.000000pt}%
\definecolor{currentstroke}{rgb}{0.000000,0.000000,0.000000}%
\pgfsetstrokecolor{currentstroke}%
\pgfsetdash{}{0pt}%
\pgfpathmoveto{\pgfqpoint{3.663905in}{2.194933in}}%
\pgfpathlineto{\pgfqpoint{3.707454in}{2.007333in}}%
\pgfpathlineto{\pgfqpoint{3.731981in}{2.002074in}}%
\pgfpathlineto{\pgfqpoint{3.688556in}{2.184487in}}%
\pgfpathlineto{\pgfqpoint{3.663905in}{2.194933in}}%
\pgfpathclose%
\pgfusepath{fill}%
\end{pgfscope}%
\begin{pgfscope}%
\pgfpathrectangle{\pgfqpoint{1.072000in}{0.528000in}}{\pgfqpoint{3.696000in}{3.696000in}}%
\pgfusepath{clip}%
\pgfsetbuttcap%
\pgfsetroundjoin%
\definecolor{currentfill}{rgb}{0.895885,0.433075,0.338681}%
\pgfsetfillcolor{currentfill}%
\pgfsetlinewidth{0.000000pt}%
\definecolor{currentstroke}{rgb}{0.000000,0.000000,0.000000}%
\pgfsetstrokecolor{currentstroke}%
\pgfsetdash{}{0pt}%
\pgfpathmoveto{\pgfqpoint{2.517624in}{2.974219in}}%
\pgfpathlineto{\pgfqpoint{2.561791in}{3.097359in}}%
\pgfpathlineto{\pgfqpoint{2.589298in}{3.125791in}}%
\pgfpathlineto{\pgfqpoint{2.545063in}{3.003676in}}%
\pgfpathlineto{\pgfqpoint{2.517624in}{2.974219in}}%
\pgfpathclose%
\pgfusepath{fill}%
\end{pgfscope}%
\begin{pgfscope}%
\pgfpathrectangle{\pgfqpoint{1.072000in}{0.528000in}}{\pgfqpoint{3.696000in}{3.696000in}}%
\pgfusepath{clip}%
\pgfsetbuttcap%
\pgfsetroundjoin%
\definecolor{currentfill}{rgb}{0.820401,0.286765,0.245160}%
\pgfsetfillcolor{currentfill}%
\pgfsetlinewidth{0.000000pt}%
\definecolor{currentstroke}{rgb}{0.000000,0.000000,0.000000}%
\pgfsetstrokecolor{currentstroke}%
\pgfsetdash{}{0pt}%
\pgfpathmoveto{\pgfqpoint{2.616838in}{3.140922in}}%
\pgfpathlineto{\pgfqpoint{2.661861in}{3.229070in}}%
\pgfpathlineto{\pgfqpoint{2.689363in}{3.234966in}}%
\pgfpathlineto{\pgfqpoint{2.644382in}{3.142413in}}%
\pgfpathlineto{\pgfqpoint{2.616838in}{3.140922in}}%
\pgfpathclose%
\pgfusepath{fill}%
\end{pgfscope}%
\begin{pgfscope}%
\pgfpathrectangle{\pgfqpoint{1.072000in}{0.528000in}}{\pgfqpoint{3.696000in}{3.696000in}}%
\pgfusepath{clip}%
\pgfsetbuttcap%
\pgfsetroundjoin%
\definecolor{currentfill}{rgb}{0.430507,0.564883,0.948889}%
\pgfsetfillcolor{currentfill}%
\pgfsetlinewidth{0.000000pt}%
\definecolor{currentstroke}{rgb}{0.000000,0.000000,0.000000}%
\pgfsetstrokecolor{currentstroke}%
\pgfsetdash{}{0pt}%
\pgfpathmoveto{\pgfqpoint{3.675730in}{1.779293in}}%
\pgfpathlineto{\pgfqpoint{3.720060in}{1.632466in}}%
\pgfpathlineto{\pgfqpoint{3.745303in}{1.660796in}}%
\pgfpathlineto{\pgfqpoint{3.701154in}{1.808678in}}%
\pgfpathlineto{\pgfqpoint{3.675730in}{1.779293in}}%
\pgfpathclose%
\pgfusepath{fill}%
\end{pgfscope}%
\begin{pgfscope}%
\pgfpathrectangle{\pgfqpoint{1.072000in}{0.528000in}}{\pgfqpoint{3.696000in}{3.696000in}}%
\pgfusepath{clip}%
\pgfsetbuttcap%
\pgfsetroundjoin%
\definecolor{currentfill}{rgb}{0.940879,0.805596,0.735167}%
\pgfsetfillcolor{currentfill}%
\pgfsetlinewidth{0.000000pt}%
\definecolor{currentstroke}{rgb}{0.000000,0.000000,0.000000}%
\pgfsetstrokecolor{currentstroke}%
\pgfsetdash{}{0pt}%
\pgfpathmoveto{\pgfqpoint{3.605568in}{2.664308in}}%
\pgfpathlineto{\pgfqpoint{3.649651in}{2.488481in}}%
\pgfpathlineto{\pgfqpoint{3.673309in}{2.429414in}}%
\pgfpathlineto{\pgfqpoint{3.629445in}{2.601312in}}%
\pgfpathlineto{\pgfqpoint{3.605568in}{2.664308in}}%
\pgfpathclose%
\pgfusepath{fill}%
\end{pgfscope}%
\begin{pgfscope}%
\pgfpathrectangle{\pgfqpoint{1.072000in}{0.528000in}}{\pgfqpoint{3.696000in}{3.696000in}}%
\pgfusepath{clip}%
\pgfsetbuttcap%
\pgfsetroundjoin%
\definecolor{currentfill}{rgb}{0.597777,0.727330,0.999777}%
\pgfsetfillcolor{currentfill}%
\pgfsetlinewidth{0.000000pt}%
\definecolor{currentstroke}{rgb}{0.000000,0.000000,0.000000}%
\pgfsetstrokecolor{currentstroke}%
\pgfsetdash{}{0pt}%
\pgfpathmoveto{\pgfqpoint{1.760403in}{2.064470in}}%
\pgfpathlineto{\pgfqpoint{1.809567in}{1.931730in}}%
\pgfpathlineto{\pgfqpoint{1.842310in}{1.844241in}}%
\pgfpathlineto{\pgfqpoint{1.793700in}{1.969744in}}%
\pgfpathlineto{\pgfqpoint{1.760403in}{2.064470in}}%
\pgfpathclose%
\pgfusepath{fill}%
\end{pgfscope}%
\begin{pgfscope}%
\pgfpathrectangle{\pgfqpoint{1.072000in}{0.528000in}}{\pgfqpoint{3.696000in}{3.696000in}}%
\pgfusepath{clip}%
\pgfsetbuttcap%
\pgfsetroundjoin%
\definecolor{currentfill}{rgb}{0.711554,0.033337,0.154485}%
\pgfsetfillcolor{currentfill}%
\pgfsetlinewidth{0.000000pt}%
\definecolor{currentstroke}{rgb}{0.000000,0.000000,0.000000}%
\pgfsetstrokecolor{currentstroke}%
\pgfsetdash{}{0pt}%
\pgfpathmoveto{\pgfqpoint{3.027992in}{3.343829in}}%
\pgfpathlineto{\pgfqpoint{3.074460in}{3.335739in}}%
\pgfpathlineto{\pgfqpoint{3.100896in}{3.333720in}}%
\pgfpathlineto{\pgfqpoint{3.054485in}{3.327819in}}%
\pgfpathlineto{\pgfqpoint{3.027992in}{3.343829in}}%
\pgfpathclose%
\pgfusepath{fill}%
\end{pgfscope}%
\begin{pgfscope}%
\pgfpathrectangle{\pgfqpoint{1.072000in}{0.528000in}}{\pgfqpoint{3.696000in}{3.696000in}}%
\pgfusepath{clip}%
\pgfsetbuttcap%
\pgfsetroundjoin%
\definecolor{currentfill}{rgb}{0.950956,0.786875,0.704761}%
\pgfsetfillcolor{currentfill}%
\pgfsetlinewidth{0.000000pt}%
\definecolor{currentstroke}{rgb}{0.000000,0.000000,0.000000}%
\pgfsetstrokecolor{currentstroke}%
\pgfsetdash{}{0pt}%
\pgfpathmoveto{\pgfqpoint{2.933310in}{2.677220in}}%
\pgfpathlineto{\pgfqpoint{2.979753in}{2.590528in}}%
\pgfpathlineto{\pgfqpoint{3.006282in}{2.430814in}}%
\pgfpathlineto{\pgfqpoint{2.960170in}{2.626493in}}%
\pgfpathlineto{\pgfqpoint{2.933310in}{2.677220in}}%
\pgfpathclose%
\pgfusepath{fill}%
\end{pgfscope}%
\begin{pgfscope}%
\pgfpathrectangle{\pgfqpoint{1.072000in}{0.528000in}}{\pgfqpoint{3.696000in}{3.696000in}}%
\pgfusepath{clip}%
\pgfsetbuttcap%
\pgfsetroundjoin%
\definecolor{currentfill}{rgb}{0.968500,0.673977,0.556649}%
\pgfsetfillcolor{currentfill}%
\pgfsetlinewidth{0.000000pt}%
\definecolor{currentstroke}{rgb}{0.000000,0.000000,0.000000}%
\pgfsetstrokecolor{currentstroke}%
\pgfsetdash{}{0pt}%
\pgfpathmoveto{\pgfqpoint{2.986881in}{2.722478in}}%
\pgfpathlineto{\pgfqpoint{3.033087in}{2.636092in}}%
\pgfpathlineto{\pgfqpoint{3.060038in}{2.831217in}}%
\pgfpathlineto{\pgfqpoint{3.013676in}{2.882171in}}%
\pgfpathlineto{\pgfqpoint{2.986881in}{2.722478in}}%
\pgfpathclose%
\pgfusepath{fill}%
\end{pgfscope}%
\begin{pgfscope}%
\pgfpathrectangle{\pgfqpoint{1.072000in}{0.528000in}}{\pgfqpoint{3.696000in}{3.696000in}}%
\pgfusepath{clip}%
\pgfsetbuttcap%
\pgfsetroundjoin%
\definecolor{currentfill}{rgb}{0.425199,0.559058,0.946061}%
\pgfsetfillcolor{currentfill}%
\pgfsetlinewidth{0.000000pt}%
\definecolor{currentstroke}{rgb}{0.000000,0.000000,0.000000}%
\pgfsetstrokecolor{currentstroke}%
\pgfsetdash{}{0pt}%
\pgfpathmoveto{\pgfqpoint{1.873539in}{1.793627in}}%
\pgfpathlineto{\pgfqpoint{1.921430in}{1.693462in}}%
\pgfpathlineto{\pgfqpoint{1.953183in}{1.620984in}}%
\pgfpathlineto{\pgfqpoint{1.905947in}{1.708492in}}%
\pgfpathlineto{\pgfqpoint{1.873539in}{1.793627in}}%
\pgfpathclose%
\pgfusepath{fill}%
\end{pgfscope}%
\begin{pgfscope}%
\pgfpathrectangle{\pgfqpoint{1.072000in}{0.528000in}}{\pgfqpoint{3.696000in}{3.696000in}}%
\pgfusepath{clip}%
\pgfsetbuttcap%
\pgfsetroundjoin%
\definecolor{currentfill}{rgb}{0.252663,0.332837,0.783665}%
\pgfsetfillcolor{currentfill}%
\pgfsetlinewidth{0.000000pt}%
\definecolor{currentstroke}{rgb}{0.000000,0.000000,0.000000}%
\pgfsetstrokecolor{currentstroke}%
\pgfsetdash{}{0pt}%
\pgfpathmoveto{\pgfqpoint{3.714206in}{1.452877in}}%
\pgfpathlineto{\pgfqpoint{3.760355in}{1.390745in}}%
\pgfpathlineto{\pgfqpoint{3.785571in}{1.420044in}}%
\pgfpathlineto{\pgfqpoint{3.739661in}{1.488464in}}%
\pgfpathlineto{\pgfqpoint{3.714206in}{1.452877in}}%
\pgfpathclose%
\pgfusepath{fill}%
\end{pgfscope}%
\begin{pgfscope}%
\pgfpathrectangle{\pgfqpoint{1.072000in}{0.528000in}}{\pgfqpoint{3.696000in}{3.696000in}}%
\pgfusepath{clip}%
\pgfsetbuttcap%
\pgfsetroundjoin%
\definecolor{currentfill}{rgb}{0.839351,0.861167,0.894494}%
\pgfsetfillcolor{currentfill}%
\pgfsetlinewidth{0.000000pt}%
\definecolor{currentstroke}{rgb}{0.000000,0.000000,0.000000}%
\pgfsetstrokecolor{currentstroke}%
\pgfsetdash{}{0pt}%
\pgfpathmoveto{\pgfqpoint{2.304974in}{2.194629in}}%
\pgfpathlineto{\pgfqpoint{2.348037in}{2.312794in}}%
\pgfpathlineto{\pgfqpoint{2.373838in}{2.462531in}}%
\pgfpathlineto{\pgfqpoint{2.330727in}{2.335253in}}%
\pgfpathlineto{\pgfqpoint{2.304974in}{2.194629in}}%
\pgfpathclose%
\pgfusepath{fill}%
\end{pgfscope}%
\begin{pgfscope}%
\pgfpathrectangle{\pgfqpoint{1.072000in}{0.528000in}}{\pgfqpoint{3.696000in}{3.696000in}}%
\pgfusepath{clip}%
\pgfsetbuttcap%
\pgfsetroundjoin%
\definecolor{currentfill}{rgb}{0.304174,0.406945,0.845263}%
\pgfsetfillcolor{currentfill}%
\pgfsetlinewidth{0.000000pt}%
\definecolor{currentstroke}{rgb}{0.000000,0.000000,0.000000}%
\pgfsetstrokecolor{currentstroke}%
\pgfsetdash{}{0pt}%
\pgfpathmoveto{\pgfqpoint{2.255363in}{1.461256in}}%
\pgfpathlineto{\pgfqpoint{2.300328in}{1.472621in}}%
\pgfpathlineto{\pgfqpoint{2.326690in}{1.586037in}}%
\pgfpathlineto{\pgfqpoint{2.281968in}{1.558650in}}%
\pgfpathlineto{\pgfqpoint{2.255363in}{1.461256in}}%
\pgfpathclose%
\pgfusepath{fill}%
\end{pgfscope}%
\begin{pgfscope}%
\pgfpathrectangle{\pgfqpoint{1.072000in}{0.528000in}}{\pgfqpoint{3.696000in}{3.696000in}}%
\pgfusepath{clip}%
\pgfsetbuttcap%
\pgfsetroundjoin%
\definecolor{currentfill}{rgb}{0.809329,0.852974,0.922323}%
\pgfsetfillcolor{currentfill}%
\pgfsetlinewidth{0.000000pt}%
\definecolor{currentstroke}{rgb}{0.000000,0.000000,0.000000}%
\pgfsetstrokecolor{currentstroke}%
\pgfsetdash{}{0pt}%
\pgfpathmoveto{\pgfqpoint{3.644953in}{2.376017in}}%
\pgfpathlineto{\pgfqpoint{3.688556in}{2.184487in}}%
\pgfpathlineto{\pgfqpoint{3.712795in}{2.161853in}}%
\pgfpathlineto{\pgfqpoint{3.669336in}{2.347758in}}%
\pgfpathlineto{\pgfqpoint{3.644953in}{2.376017in}}%
\pgfpathclose%
\pgfusepath{fill}%
\end{pgfscope}%
\begin{pgfscope}%
\pgfpathrectangle{\pgfqpoint{1.072000in}{0.528000in}}{\pgfqpoint{3.696000in}{3.696000in}}%
\pgfusepath{clip}%
\pgfsetbuttcap%
\pgfsetroundjoin%
\definecolor{currentfill}{rgb}{0.248091,0.326013,0.777669}%
\pgfsetfillcolor{currentfill}%
\pgfsetlinewidth{0.000000pt}%
\definecolor{currentstroke}{rgb}{0.000000,0.000000,0.000000}%
\pgfsetstrokecolor{currentstroke}%
\pgfsetdash{}{0pt}%
\pgfpathmoveto{\pgfqpoint{2.228001in}{1.398444in}}%
\pgfpathlineto{\pgfqpoint{2.273278in}{1.394106in}}%
\pgfpathlineto{\pgfqpoint{2.300328in}{1.472621in}}%
\pgfpathlineto{\pgfqpoint{2.255363in}{1.461256in}}%
\pgfpathlineto{\pgfqpoint{2.228001in}{1.398444in}}%
\pgfpathclose%
\pgfusepath{fill}%
\end{pgfscope}%
\begin{pgfscope}%
\pgfpathrectangle{\pgfqpoint{1.072000in}{0.528000in}}{\pgfqpoint{3.696000in}{3.696000in}}%
\pgfusepath{clip}%
\pgfsetbuttcap%
\pgfsetroundjoin%
\definecolor{currentfill}{rgb}{0.717435,0.051118,0.158737}%
\pgfsetfillcolor{currentfill}%
\pgfsetlinewidth{0.000000pt}%
\definecolor{currentstroke}{rgb}{0.000000,0.000000,0.000000}%
\pgfsetstrokecolor{currentstroke}%
\pgfsetdash{}{0pt}%
\pgfpathmoveto{\pgfqpoint{2.835476in}{3.322092in}}%
\pgfpathlineto{\pgfqpoint{2.881662in}{3.329710in}}%
\pgfpathlineto{\pgfqpoint{2.908610in}{3.336755in}}%
\pgfpathlineto{\pgfqpoint{2.862549in}{3.309270in}}%
\pgfpathlineto{\pgfqpoint{2.835476in}{3.322092in}}%
\pgfpathclose%
\pgfusepath{fill}%
\end{pgfscope}%
\begin{pgfscope}%
\pgfpathrectangle{\pgfqpoint{1.072000in}{0.528000in}}{\pgfqpoint{3.696000in}{3.696000in}}%
\pgfusepath{clip}%
\pgfsetbuttcap%
\pgfsetroundjoin%
\definecolor{currentfill}{rgb}{0.229806,0.298718,0.753683}%
\pgfsetfillcolor{currentfill}%
\pgfsetlinewidth{0.000000pt}%
\definecolor{currentstroke}{rgb}{0.000000,0.000000,0.000000}%
\pgfsetstrokecolor{currentstroke}%
\pgfsetdash{}{0pt}%
\pgfpathmoveto{\pgfqpoint{2.154013in}{1.398990in}}%
\pgfpathlineto{\pgfqpoint{2.199728in}{1.373160in}}%
\pgfpathlineto{\pgfqpoint{2.228001in}{1.398444in}}%
\pgfpathlineto{\pgfqpoint{2.182729in}{1.408245in}}%
\pgfpathlineto{\pgfqpoint{2.154013in}{1.398990in}}%
\pgfpathclose%
\pgfusepath{fill}%
\end{pgfscope}%
\begin{pgfscope}%
\pgfpathrectangle{\pgfqpoint{1.072000in}{0.528000in}}{\pgfqpoint{3.696000in}{3.696000in}}%
\pgfusepath{clip}%
\pgfsetbuttcap%
\pgfsetroundjoin%
\definecolor{currentfill}{rgb}{0.229806,0.298718,0.753683}%
\pgfsetfillcolor{currentfill}%
\pgfsetlinewidth{0.000000pt}%
\definecolor{currentstroke}{rgb}{0.000000,0.000000,0.000000}%
\pgfsetstrokecolor{currentstroke}%
\pgfsetdash{}{0pt}%
\pgfpathmoveto{\pgfqpoint{3.760355in}{1.390745in}}%
\pgfpathlineto{\pgfqpoint{3.807549in}{1.371402in}}%
\pgfpathlineto{\pgfqpoint{3.832548in}{1.394798in}}%
\pgfpathlineto{\pgfqpoint{3.785571in}{1.420044in}}%
\pgfpathlineto{\pgfqpoint{3.760355in}{1.390745in}}%
\pgfpathclose%
\pgfusepath{fill}%
\end{pgfscope}%
\begin{pgfscope}%
\pgfpathrectangle{\pgfqpoint{1.072000in}{0.528000in}}{\pgfqpoint{3.696000in}{3.696000in}}%
\pgfusepath{clip}%
\pgfsetbuttcap%
\pgfsetroundjoin%
\definecolor{currentfill}{rgb}{0.891817,0.851973,0.829085}%
\pgfsetfillcolor{currentfill}%
\pgfsetlinewidth{0.000000pt}%
\definecolor{currentstroke}{rgb}{0.000000,0.000000,0.000000}%
\pgfsetstrokecolor{currentstroke}%
\pgfsetdash{}{0pt}%
\pgfpathmoveto{\pgfqpoint{3.625530in}{2.534526in}}%
\pgfpathlineto{\pgfqpoint{3.669336in}{2.347758in}}%
\pgfpathlineto{\pgfqpoint{3.693278in}{2.306811in}}%
\pgfpathlineto{\pgfqpoint{3.649651in}{2.488481in}}%
\pgfpathlineto{\pgfqpoint{3.625530in}{2.534526in}}%
\pgfpathclose%
\pgfusepath{fill}%
\end{pgfscope}%
\begin{pgfscope}%
\pgfpathrectangle{\pgfqpoint{1.072000in}{0.528000in}}{\pgfqpoint{3.696000in}{3.696000in}}%
\pgfusepath{clip}%
\pgfsetbuttcap%
\pgfsetroundjoin%
\definecolor{currentfill}{rgb}{0.926883,0.505422,0.394866}%
\pgfsetfillcolor{currentfill}%
\pgfsetlinewidth{0.000000pt}%
\definecolor{currentstroke}{rgb}{0.000000,0.000000,0.000000}%
\pgfsetstrokecolor{currentstroke}%
\pgfsetdash{}{0pt}%
\pgfpathmoveto{\pgfqpoint{3.473288in}{3.062640in}}%
\pgfpathlineto{\pgfqpoint{3.518946in}{2.956615in}}%
\pgfpathlineto{\pgfqpoint{3.542715in}{2.864381in}}%
\pgfpathlineto{\pgfqpoint{3.497434in}{2.976495in}}%
\pgfpathlineto{\pgfqpoint{3.473288in}{3.062640in}}%
\pgfpathclose%
\pgfusepath{fill}%
\end{pgfscope}%
\begin{pgfscope}%
\pgfpathrectangle{\pgfqpoint{1.072000in}{0.528000in}}{\pgfqpoint{3.696000in}{3.696000in}}%
\pgfusepath{clip}%
\pgfsetbuttcap%
\pgfsetroundjoin%
\definecolor{currentfill}{rgb}{0.740957,0.122240,0.175744}%
\pgfsetfillcolor{currentfill}%
\pgfsetlinewidth{0.000000pt}%
\definecolor{currentstroke}{rgb}{0.000000,0.000000,0.000000}%
\pgfsetstrokecolor{currentstroke}%
\pgfsetdash{}{0pt}%
\pgfpathmoveto{\pgfqpoint{3.194026in}{3.320330in}}%
\pgfpathlineto{\pgfqpoint{3.240713in}{3.305442in}}%
\pgfpathlineto{\pgfqpoint{3.266436in}{3.267899in}}%
\pgfpathlineto{\pgfqpoint{3.219909in}{3.285237in}}%
\pgfpathlineto{\pgfqpoint{3.194026in}{3.320330in}}%
\pgfpathclose%
\pgfusepath{fill}%
\end{pgfscope}%
\begin{pgfscope}%
\pgfpathrectangle{\pgfqpoint{1.072000in}{0.528000in}}{\pgfqpoint{3.696000in}{3.696000in}}%
\pgfusepath{clip}%
\pgfsetbuttcap%
\pgfsetroundjoin%
\definecolor{currentfill}{rgb}{0.763520,0.178667,0.193396}%
\pgfsetfillcolor{currentfill}%
\pgfsetlinewidth{0.000000pt}%
\definecolor{currentstroke}{rgb}{0.000000,0.000000,0.000000}%
\pgfsetstrokecolor{currentstroke}%
\pgfsetdash{}{0pt}%
\pgfpathmoveto{\pgfqpoint{2.689363in}{3.234966in}}%
\pgfpathlineto{\pgfqpoint{2.734980in}{3.288486in}}%
\pgfpathlineto{\pgfqpoint{2.762333in}{3.291704in}}%
\pgfpathlineto{\pgfqpoint{2.716820in}{3.228208in}}%
\pgfpathlineto{\pgfqpoint{2.689363in}{3.234966in}}%
\pgfpathclose%
\pgfusepath{fill}%
\end{pgfscope}%
\begin{pgfscope}%
\pgfpathrectangle{\pgfqpoint{1.072000in}{0.528000in}}{\pgfqpoint{3.696000in}{3.696000in}}%
\pgfusepath{clip}%
\pgfsetbuttcap%
\pgfsetroundjoin%
\definecolor{currentfill}{rgb}{0.729196,0.086679,0.167240}%
\pgfsetfillcolor{currentfill}%
\pgfsetlinewidth{0.000000pt}%
\definecolor{currentstroke}{rgb}{0.000000,0.000000,0.000000}%
\pgfsetstrokecolor{currentstroke}%
\pgfsetdash{}{0pt}%
\pgfpathmoveto{\pgfqpoint{2.762333in}{3.291704in}}%
\pgfpathlineto{\pgfqpoint{2.808325in}{3.316722in}}%
\pgfpathlineto{\pgfqpoint{2.835476in}{3.322092in}}%
\pgfpathlineto{\pgfqpoint{2.789618in}{3.281154in}}%
\pgfpathlineto{\pgfqpoint{2.762333in}{3.291704in}}%
\pgfpathclose%
\pgfusepath{fill}%
\end{pgfscope}%
\begin{pgfscope}%
\pgfpathrectangle{\pgfqpoint{1.072000in}{0.528000in}}{\pgfqpoint{3.696000in}{3.696000in}}%
\pgfusepath{clip}%
\pgfsetbuttcap%
\pgfsetroundjoin%
\definecolor{currentfill}{rgb}{0.968500,0.673977,0.556649}%
\pgfsetfillcolor{currentfill}%
\pgfsetlinewidth{0.000000pt}%
\definecolor{currentstroke}{rgb}{0.000000,0.000000,0.000000}%
\pgfsetstrokecolor{currentstroke}%
\pgfsetdash{}{0pt}%
\pgfpathmoveto{\pgfqpoint{2.913695in}{2.812279in}}%
\pgfpathlineto{\pgfqpoint{2.960170in}{2.626493in}}%
\pgfpathlineto{\pgfqpoint{2.986881in}{2.722478in}}%
\pgfpathlineto{\pgfqpoint{2.940533in}{2.872755in}}%
\pgfpathlineto{\pgfqpoint{2.913695in}{2.812279in}}%
\pgfpathclose%
\pgfusepath{fill}%
\end{pgfscope}%
\begin{pgfscope}%
\pgfpathrectangle{\pgfqpoint{1.072000in}{0.528000in}}{\pgfqpoint{3.696000in}{3.696000in}}%
\pgfusepath{clip}%
\pgfsetbuttcap%
\pgfsetroundjoin%
\definecolor{currentfill}{rgb}{0.576051,0.708780,0.997755}%
\pgfsetfillcolor{currentfill}%
\pgfsetlinewidth{0.000000pt}%
\definecolor{currentstroke}{rgb}{0.000000,0.000000,0.000000}%
\pgfsetstrokecolor{currentstroke}%
\pgfsetdash{}{0pt}%
\pgfpathmoveto{\pgfqpoint{3.657280in}{1.983338in}}%
\pgfpathlineto{\pgfqpoint{3.701154in}{1.808678in}}%
\pgfpathlineto{\pgfqpoint{3.726264in}{1.828097in}}%
\pgfpathlineto{\pgfqpoint{3.682551in}{2.001337in}}%
\pgfpathlineto{\pgfqpoint{3.657280in}{1.983338in}}%
\pgfpathclose%
\pgfusepath{fill}%
\end{pgfscope}%
\begin{pgfscope}%
\pgfpathrectangle{\pgfqpoint{1.072000in}{0.528000in}}{\pgfqpoint{3.696000in}{3.696000in}}%
\pgfusepath{clip}%
\pgfsetbuttcap%
\pgfsetroundjoin%
\definecolor{currentfill}{rgb}{0.388852,0.516298,0.921373}%
\pgfsetfillcolor{currentfill}%
\pgfsetlinewidth{0.000000pt}%
\definecolor{currentstroke}{rgb}{0.000000,0.000000,0.000000}%
\pgfsetstrokecolor{currentstroke}%
\pgfsetdash{}{0pt}%
\pgfpathmoveto{\pgfqpoint{2.281968in}{1.558650in}}%
\pgfpathlineto{\pgfqpoint{2.326690in}{1.586037in}}%
\pgfpathlineto{\pgfqpoint{2.352549in}{1.728999in}}%
\pgfpathlineto{\pgfqpoint{2.307997in}{1.685966in}}%
\pgfpathlineto{\pgfqpoint{2.281968in}{1.558650in}}%
\pgfpathclose%
\pgfusepath{fill}%
\end{pgfscope}%
\begin{pgfscope}%
\pgfpathrectangle{\pgfqpoint{1.072000in}{0.528000in}}{\pgfqpoint{3.696000in}{3.696000in}}%
\pgfusepath{clip}%
\pgfsetbuttcap%
\pgfsetroundjoin%
\definecolor{currentfill}{rgb}{0.313946,0.420052,0.854993}%
\pgfsetfillcolor{currentfill}%
\pgfsetlinewidth{0.000000pt}%
\definecolor{currentstroke}{rgb}{0.000000,0.000000,0.000000}%
\pgfsetstrokecolor{currentstroke}%
\pgfsetdash{}{0pt}%
\pgfpathmoveto{\pgfqpoint{3.668860in}{1.554962in}}%
\pgfpathlineto{\pgfqpoint{3.714206in}{1.452877in}}%
\pgfpathlineto{\pgfqpoint{3.739661in}{1.488464in}}%
\pgfpathlineto{\pgfqpoint{3.694569in}{1.596950in}}%
\pgfpathlineto{\pgfqpoint{3.668860in}{1.554962in}}%
\pgfpathclose%
\pgfusepath{fill}%
\end{pgfscope}%
\begin{pgfscope}%
\pgfpathrectangle{\pgfqpoint{1.072000in}{0.528000in}}{\pgfqpoint{3.696000in}{3.696000in}}%
\pgfusepath{clip}%
\pgfsetbuttcap%
\pgfsetroundjoin%
\definecolor{currentfill}{rgb}{0.705673,0.015556,0.150233}%
\pgfsetfillcolor{currentfill}%
\pgfsetlinewidth{0.000000pt}%
\definecolor{currentstroke}{rgb}{0.000000,0.000000,0.000000}%
\pgfsetstrokecolor{currentstroke}%
\pgfsetdash{}{0pt}%
\pgfpathmoveto{\pgfqpoint{2.954893in}{3.338743in}}%
\pgfpathlineto{\pgfqpoint{3.001304in}{3.321897in}}%
\pgfpathlineto{\pgfqpoint{3.027992in}{3.343829in}}%
\pgfpathlineto{\pgfqpoint{2.981639in}{3.340125in}}%
\pgfpathlineto{\pgfqpoint{2.954893in}{3.338743in}}%
\pgfpathclose%
\pgfusepath{fill}%
\end{pgfscope}%
\begin{pgfscope}%
\pgfpathrectangle{\pgfqpoint{1.072000in}{0.528000in}}{\pgfqpoint{3.696000in}{3.696000in}}%
\pgfusepath{clip}%
\pgfsetbuttcap%
\pgfsetroundjoin%
\definecolor{currentfill}{rgb}{0.950956,0.786875,0.704761}%
\pgfsetfillcolor{currentfill}%
\pgfsetlinewidth{0.000000pt}%
\definecolor{currentstroke}{rgb}{0.000000,0.000000,0.000000}%
\pgfsetstrokecolor{currentstroke}%
\pgfsetdash{}{0pt}%
\pgfpathmoveto{\pgfqpoint{2.979753in}{2.590528in}}%
\pgfpathlineto{\pgfqpoint{3.026347in}{2.650923in}}%
\pgfpathlineto{\pgfqpoint{3.052886in}{2.600187in}}%
\pgfpathlineto{\pgfqpoint{3.006282in}{2.430814in}}%
\pgfpathlineto{\pgfqpoint{2.979753in}{2.590528in}}%
\pgfpathclose%
\pgfusepath{fill}%
\end{pgfscope}%
\begin{pgfscope}%
\pgfpathrectangle{\pgfqpoint{1.072000in}{0.528000in}}{\pgfqpoint{3.696000in}{3.696000in}}%
\pgfusepath{clip}%
\pgfsetbuttcap%
\pgfsetroundjoin%
\definecolor{currentfill}{rgb}{0.968500,0.673977,0.556649}%
\pgfsetfillcolor{currentfill}%
\pgfsetlinewidth{0.000000pt}%
\definecolor{currentstroke}{rgb}{0.000000,0.000000,0.000000}%
\pgfsetstrokecolor{currentstroke}%
\pgfsetdash{}{0pt}%
\pgfpathmoveto{\pgfqpoint{3.033087in}{2.636092in}}%
\pgfpathlineto{\pgfqpoint{3.079668in}{2.696393in}}%
\pgfpathlineto{\pgfqpoint{3.106704in}{2.856407in}}%
\pgfpathlineto{\pgfqpoint{3.060038in}{2.831217in}}%
\pgfpathlineto{\pgfqpoint{3.033087in}{2.636092in}}%
\pgfpathclose%
\pgfusepath{fill}%
\end{pgfscope}%
\begin{pgfscope}%
\pgfpathrectangle{\pgfqpoint{1.072000in}{0.528000in}}{\pgfqpoint{3.696000in}{3.696000in}}%
\pgfusepath{clip}%
\pgfsetbuttcap%
\pgfsetroundjoin%
\definecolor{currentfill}{rgb}{0.565182,0.699438,0.996635}%
\pgfsetfillcolor{currentfill}%
\pgfsetlinewidth{0.000000pt}%
\definecolor{currentstroke}{rgb}{0.000000,0.000000,0.000000}%
\pgfsetstrokecolor{currentstroke}%
\pgfsetdash{}{0pt}%
\pgfpathmoveto{\pgfqpoint{2.289524in}{1.777184in}}%
\pgfpathlineto{\pgfqpoint{2.333639in}{1.837096in}}%
\pgfpathlineto{\pgfqpoint{2.359082in}{2.004838in}}%
\pgfpathlineto{\pgfqpoint{2.315063in}{1.929780in}}%
\pgfpathlineto{\pgfqpoint{2.289524in}{1.777184in}}%
\pgfpathclose%
\pgfusepath{fill}%
\end{pgfscope}%
\begin{pgfscope}%
\pgfpathrectangle{\pgfqpoint{1.072000in}{0.528000in}}{\pgfqpoint{3.696000in}{3.696000in}}%
\pgfusepath{clip}%
\pgfsetbuttcap%
\pgfsetroundjoin%
\definecolor{currentfill}{rgb}{0.905783,0.455186,0.355336}%
\pgfsetfillcolor{currentfill}%
\pgfsetlinewidth{0.000000pt}%
\definecolor{currentstroke}{rgb}{0.000000,0.000000,0.000000}%
\pgfsetstrokecolor{currentstroke}%
\pgfsetdash{}{0pt}%
\pgfpathmoveto{\pgfqpoint{2.490295in}{2.929463in}}%
\pgfpathlineto{\pgfqpoint{2.534351in}{3.055175in}}%
\pgfpathlineto{\pgfqpoint{2.561791in}{3.097359in}}%
\pgfpathlineto{\pgfqpoint{2.517624in}{2.974219in}}%
\pgfpathlineto{\pgfqpoint{2.490295in}{2.929463in}}%
\pgfpathclose%
\pgfusepath{fill}%
\end{pgfscope}%
\begin{pgfscope}%
\pgfpathrectangle{\pgfqpoint{1.072000in}{0.528000in}}{\pgfqpoint{3.696000in}{3.696000in}}%
\pgfusepath{clip}%
\pgfsetbuttcap%
\pgfsetroundjoin%
\definecolor{currentfill}{rgb}{0.953054,0.585211,0.465373}%
\pgfsetfillcolor{currentfill}%
\pgfsetlinewidth{0.000000pt}%
\definecolor{currentstroke}{rgb}{0.000000,0.000000,0.000000}%
\pgfsetstrokecolor{currentstroke}%
\pgfsetdash{}{0pt}%
\pgfpathmoveto{\pgfqpoint{2.940533in}{2.872755in}}%
\pgfpathlineto{\pgfqpoint{2.986881in}{2.722478in}}%
\pgfpathlineto{\pgfqpoint{3.013676in}{2.882171in}}%
\pgfpathlineto{\pgfqpoint{2.967317in}{2.984400in}}%
\pgfpathlineto{\pgfqpoint{2.940533in}{2.872755in}}%
\pgfpathclose%
\pgfusepath{fill}%
\end{pgfscope}%
\begin{pgfscope}%
\pgfpathrectangle{\pgfqpoint{1.072000in}{0.528000in}}{\pgfqpoint{3.696000in}{3.696000in}}%
\pgfusepath{clip}%
\pgfsetbuttcap%
\pgfsetroundjoin%
\definecolor{currentfill}{rgb}{0.728970,0.817464,0.973188}%
\pgfsetfillcolor{currentfill}%
\pgfsetlinewidth{0.000000pt}%
\definecolor{currentstroke}{rgb}{0.000000,0.000000,0.000000}%
\pgfsetstrokecolor{currentstroke}%
\pgfsetdash{}{0pt}%
\pgfpathmoveto{\pgfqpoint{2.296938in}{2.002218in}}%
\pgfpathlineto{\pgfqpoint{2.340496in}{2.093593in}}%
\pgfpathlineto{\pgfqpoint{2.365979in}{2.261309in}}%
\pgfpathlineto{\pgfqpoint{2.322438in}{2.157381in}}%
\pgfpathlineto{\pgfqpoint{2.296938in}{2.002218in}}%
\pgfpathclose%
\pgfusepath{fill}%
\end{pgfscope}%
\begin{pgfscope}%
\pgfpathrectangle{\pgfqpoint{1.072000in}{0.528000in}}{\pgfqpoint{3.696000in}{3.696000in}}%
\pgfusepath{clip}%
\pgfsetbuttcap%
\pgfsetroundjoin%
\definecolor{currentfill}{rgb}{0.705673,0.015556,0.150233}%
\pgfsetfillcolor{currentfill}%
\pgfsetlinewidth{0.000000pt}%
\definecolor{currentstroke}{rgb}{0.000000,0.000000,0.000000}%
\pgfsetstrokecolor{currentstroke}%
\pgfsetdash{}{0pt}%
\pgfpathmoveto{\pgfqpoint{3.074460in}{3.335739in}}%
\pgfpathlineto{\pgfqpoint{3.121025in}{3.322477in}}%
\pgfpathlineto{\pgfqpoint{3.147415in}{3.330153in}}%
\pgfpathlineto{\pgfqpoint{3.100896in}{3.333720in}}%
\pgfpathlineto{\pgfqpoint{3.074460in}{3.335739in}}%
\pgfpathclose%
\pgfusepath{fill}%
\end{pgfscope}%
\begin{pgfscope}%
\pgfpathrectangle{\pgfqpoint{1.072000in}{0.528000in}}{\pgfqpoint{3.696000in}{3.696000in}}%
\pgfusepath{clip}%
\pgfsetbuttcap%
\pgfsetroundjoin%
\definecolor{currentfill}{rgb}{0.964911,0.640159,0.519806}%
\pgfsetfillcolor{currentfill}%
\pgfsetlinewidth{0.000000pt}%
\definecolor{currentstroke}{rgb}{0.000000,0.000000,0.000000}%
\pgfsetstrokecolor{currentstroke}%
\pgfsetdash{}{0pt}%
\pgfpathmoveto{\pgfqpoint{3.540126in}{2.906597in}}%
\pgfpathlineto{\pgfqpoint{3.585074in}{2.762686in}}%
\pgfpathlineto{\pgfqpoint{3.608723in}{2.683667in}}%
\pgfpathlineto{\pgfqpoint{3.564101in}{2.829246in}}%
\pgfpathlineto{\pgfqpoint{3.540126in}{2.906597in}}%
\pgfpathclose%
\pgfusepath{fill}%
\end{pgfscope}%
\begin{pgfscope}%
\pgfpathrectangle{\pgfqpoint{1.072000in}{0.528000in}}{\pgfqpoint{3.696000in}{3.696000in}}%
\pgfusepath{clip}%
\pgfsetbuttcap%
\pgfsetroundjoin%
\definecolor{currentfill}{rgb}{0.505423,0.643995,0.983157}%
\pgfsetfillcolor{currentfill}%
\pgfsetlinewidth{0.000000pt}%
\definecolor{currentstroke}{rgb}{0.000000,0.000000,0.000000}%
\pgfsetstrokecolor{currentstroke}%
\pgfsetdash{}{0pt}%
\pgfpathmoveto{\pgfqpoint{1.825061in}{1.907614in}}%
\pgfpathlineto{\pgfqpoint{1.873539in}{1.793627in}}%
\pgfpathlineto{\pgfqpoint{1.905947in}{1.708492in}}%
\pgfpathlineto{\pgfqpoint{1.858085in}{1.812410in}}%
\pgfpathlineto{\pgfqpoint{1.825061in}{1.907614in}}%
\pgfpathclose%
\pgfusepath{fill}%
\end{pgfscope}%
\begin{pgfscope}%
\pgfpathrectangle{\pgfqpoint{1.072000in}{0.528000in}}{\pgfqpoint{3.696000in}{3.696000in}}%
\pgfusepath{clip}%
\pgfsetbuttcap%
\pgfsetroundjoin%
\definecolor{currentfill}{rgb}{0.919376,0.831273,0.782874}%
\pgfsetfillcolor{currentfill}%
\pgfsetlinewidth{0.000000pt}%
\definecolor{currentstroke}{rgb}{0.000000,0.000000,0.000000}%
\pgfsetstrokecolor{currentstroke}%
\pgfsetdash{}{0pt}%
\pgfpathmoveto{\pgfqpoint{2.330727in}{2.335253in}}%
\pgfpathlineto{\pgfqpoint{2.373838in}{2.462531in}}%
\pgfpathlineto{\pgfqpoint{2.399905in}{2.601621in}}%
\pgfpathlineto{\pgfqpoint{2.356686in}{2.468670in}}%
\pgfpathlineto{\pgfqpoint{2.330727in}{2.335253in}}%
\pgfpathclose%
\pgfusepath{fill}%
\end{pgfscope}%
\begin{pgfscope}%
\pgfpathrectangle{\pgfqpoint{1.072000in}{0.528000in}}{\pgfqpoint{3.696000in}{3.696000in}}%
\pgfusepath{clip}%
\pgfsetbuttcap%
\pgfsetroundjoin%
\definecolor{currentfill}{rgb}{0.229806,0.298718,0.753683}%
\pgfsetfillcolor{currentfill}%
\pgfsetlinewidth{0.000000pt}%
\definecolor{currentstroke}{rgb}{0.000000,0.000000,0.000000}%
\pgfsetstrokecolor{currentstroke}%
\pgfsetdash{}{0pt}%
\pgfpathmoveto{\pgfqpoint{2.199728in}{1.373160in}}%
\pgfpathlineto{\pgfqpoint{2.245371in}{1.354065in}}%
\pgfpathlineto{\pgfqpoint{2.273278in}{1.394106in}}%
\pgfpathlineto{\pgfqpoint{2.228001in}{1.398444in}}%
\pgfpathlineto{\pgfqpoint{2.199728in}{1.373160in}}%
\pgfpathclose%
\pgfusepath{fill}%
\end{pgfscope}%
\begin{pgfscope}%
\pgfpathrectangle{\pgfqpoint{1.072000in}{0.528000in}}{\pgfqpoint{3.696000in}{3.696000in}}%
\pgfusepath{clip}%
\pgfsetbuttcap%
\pgfsetroundjoin%
\definecolor{currentfill}{rgb}{0.425199,0.559058,0.946061}%
\pgfsetfillcolor{currentfill}%
\pgfsetlinewidth{0.000000pt}%
\definecolor{currentstroke}{rgb}{0.000000,0.000000,0.000000}%
\pgfsetstrokecolor{currentstroke}%
\pgfsetdash{}{0pt}%
\pgfpathmoveto{\pgfqpoint{3.650018in}{1.740153in}}%
\pgfpathlineto{\pgfqpoint{3.694569in}{1.596950in}}%
\pgfpathlineto{\pgfqpoint{3.720060in}{1.632466in}}%
\pgfpathlineto{\pgfqpoint{3.675730in}{1.779293in}}%
\pgfpathlineto{\pgfqpoint{3.650018in}{1.740153in}}%
\pgfpathclose%
\pgfusepath{fill}%
\end{pgfscope}%
\begin{pgfscope}%
\pgfpathrectangle{\pgfqpoint{1.072000in}{0.528000in}}{\pgfqpoint{3.696000in}{3.696000in}}%
\pgfusepath{clip}%
\pgfsetbuttcap%
\pgfsetroundjoin%
\definecolor{currentfill}{rgb}{0.746838,0.140021,0.179996}%
\pgfsetfillcolor{currentfill}%
\pgfsetlinewidth{0.000000pt}%
\definecolor{currentstroke}{rgb}{0.000000,0.000000,0.000000}%
\pgfsetstrokecolor{currentstroke}%
\pgfsetdash{}{0pt}%
\pgfpathmoveto{\pgfqpoint{3.240713in}{3.305442in}}%
\pgfpathlineto{\pgfqpoint{3.287446in}{3.284268in}}%
\pgfpathlineto{\pgfqpoint{3.312954in}{3.239842in}}%
\pgfpathlineto{\pgfqpoint{3.266436in}{3.267899in}}%
\pgfpathlineto{\pgfqpoint{3.240713in}{3.305442in}}%
\pgfpathclose%
\pgfusepath{fill}%
\end{pgfscope}%
\begin{pgfscope}%
\pgfpathrectangle{\pgfqpoint{1.072000in}{0.528000in}}{\pgfqpoint{3.696000in}{3.696000in}}%
\pgfusepath{clip}%
\pgfsetbuttcap%
\pgfsetroundjoin%
\definecolor{currentfill}{rgb}{0.877149,0.394645,0.311724}%
\pgfsetfillcolor{currentfill}%
\pgfsetlinewidth{0.000000pt}%
\definecolor{currentstroke}{rgb}{0.000000,0.000000,0.000000}%
\pgfsetstrokecolor{currentstroke}%
\pgfsetdash{}{0pt}%
\pgfpathmoveto{\pgfqpoint{3.427201in}{3.146359in}}%
\pgfpathlineto{\pgfqpoint{3.473288in}{3.062640in}}%
\pgfpathlineto{\pgfqpoint{3.497434in}{2.976495in}}%
\pgfpathlineto{\pgfqpoint{3.451733in}{3.069378in}}%
\pgfpathlineto{\pgfqpoint{3.427201in}{3.146359in}}%
\pgfpathclose%
\pgfusepath{fill}%
\end{pgfscope}%
\begin{pgfscope}%
\pgfpathrectangle{\pgfqpoint{1.072000in}{0.528000in}}{\pgfqpoint{3.696000in}{3.696000in}}%
\pgfusepath{clip}%
\pgfsetbuttcap%
\pgfsetroundjoin%
\definecolor{currentfill}{rgb}{0.243520,0.319189,0.771672}%
\pgfsetfillcolor{currentfill}%
\pgfsetlinewidth{0.000000pt}%
\definecolor{currentstroke}{rgb}{0.000000,0.000000,0.000000}%
\pgfsetstrokecolor{currentstroke}%
\pgfsetdash{}{0pt}%
\pgfpathmoveto{\pgfqpoint{3.688608in}{1.414558in}}%
\pgfpathlineto{\pgfqpoint{3.735039in}{1.361132in}}%
\pgfpathlineto{\pgfqpoint{3.760355in}{1.390745in}}%
\pgfpathlineto{\pgfqpoint{3.714206in}{1.452877in}}%
\pgfpathlineto{\pgfqpoint{3.688608in}{1.414558in}}%
\pgfpathclose%
\pgfusepath{fill}%
\end{pgfscope}%
\begin{pgfscope}%
\pgfpathrectangle{\pgfqpoint{1.072000in}{0.528000in}}{\pgfqpoint{3.696000in}{3.696000in}}%
\pgfusepath{clip}%
\pgfsetbuttcap%
\pgfsetroundjoin%
\definecolor{currentfill}{rgb}{0.705673,0.015556,0.150233}%
\pgfsetfillcolor{currentfill}%
\pgfsetlinewidth{0.000000pt}%
\definecolor{currentstroke}{rgb}{0.000000,0.000000,0.000000}%
\pgfsetstrokecolor{currentstroke}%
\pgfsetdash{}{0pt}%
\pgfpathmoveto{\pgfqpoint{2.881662in}{3.329710in}}%
\pgfpathlineto{\pgfqpoint{2.928042in}{3.308401in}}%
\pgfpathlineto{\pgfqpoint{2.954893in}{3.338743in}}%
\pgfpathlineto{\pgfqpoint{2.908610in}{3.336755in}}%
\pgfpathlineto{\pgfqpoint{2.881662in}{3.329710in}}%
\pgfpathclose%
\pgfusepath{fill}%
\end{pgfscope}%
\begin{pgfscope}%
\pgfpathrectangle{\pgfqpoint{1.072000in}{0.528000in}}{\pgfqpoint{3.696000in}{3.696000in}}%
\pgfusepath{clip}%
\pgfsetbuttcap%
\pgfsetroundjoin%
\definecolor{currentfill}{rgb}{0.266381,0.353304,0.801637}%
\pgfsetfillcolor{currentfill}%
\pgfsetlinewidth{0.000000pt}%
\definecolor{currentstroke}{rgb}{0.000000,0.000000,0.000000}%
\pgfsetstrokecolor{currentstroke}%
\pgfsetdash{}{0pt}%
\pgfpathmoveto{\pgfqpoint{2.273278in}{1.394106in}}%
\pgfpathlineto{\pgfqpoint{2.318627in}{1.392830in}}%
\pgfpathlineto{\pgfqpoint{2.345447in}{1.485135in}}%
\pgfpathlineto{\pgfqpoint{2.300328in}{1.472621in}}%
\pgfpathlineto{\pgfqpoint{2.273278in}{1.394106in}}%
\pgfpathclose%
\pgfusepath{fill}%
\end{pgfscope}%
\begin{pgfscope}%
\pgfpathrectangle{\pgfqpoint{1.072000in}{0.528000in}}{\pgfqpoint{3.696000in}{3.696000in}}%
\pgfusepath{clip}%
\pgfsetbuttcap%
\pgfsetroundjoin%
\definecolor{currentfill}{rgb}{0.810616,0.268797,0.235428}%
\pgfsetfillcolor{currentfill}%
\pgfsetlinewidth{0.000000pt}%
\definecolor{currentstroke}{rgb}{0.000000,0.000000,0.000000}%
\pgfsetstrokecolor{currentstroke}%
\pgfsetdash{}{0pt}%
\pgfpathmoveto{\pgfqpoint{2.589298in}{3.125791in}}%
\pgfpathlineto{\pgfqpoint{2.634326in}{3.212417in}}%
\pgfpathlineto{\pgfqpoint{2.661861in}{3.229070in}}%
\pgfpathlineto{\pgfqpoint{2.616838in}{3.140922in}}%
\pgfpathlineto{\pgfqpoint{2.589298in}{3.125791in}}%
\pgfpathclose%
\pgfusepath{fill}%
\end{pgfscope}%
\begin{pgfscope}%
\pgfpathrectangle{\pgfqpoint{1.072000in}{0.528000in}}{\pgfqpoint{3.696000in}{3.696000in}}%
\pgfusepath{clip}%
\pgfsetbuttcap%
\pgfsetroundjoin%
\definecolor{currentfill}{rgb}{0.705673,0.015556,0.150233}%
\pgfsetfillcolor{currentfill}%
\pgfsetlinewidth{0.000000pt}%
\definecolor{currentstroke}{rgb}{0.000000,0.000000,0.000000}%
\pgfsetstrokecolor{currentstroke}%
\pgfsetdash{}{0pt}%
\pgfpathmoveto{\pgfqpoint{3.001304in}{3.321897in}}%
\pgfpathlineto{\pgfqpoint{3.047790in}{3.295321in}}%
\pgfpathlineto{\pgfqpoint{3.074460in}{3.335739in}}%
\pgfpathlineto{\pgfqpoint{3.027992in}{3.343829in}}%
\pgfpathlineto{\pgfqpoint{3.001304in}{3.321897in}}%
\pgfpathclose%
\pgfusepath{fill}%
\end{pgfscope}%
\begin{pgfscope}%
\pgfpathrectangle{\pgfqpoint{1.072000in}{0.528000in}}{\pgfqpoint{3.696000in}{3.696000in}}%
\pgfusepath{clip}%
\pgfsetbuttcap%
\pgfsetroundjoin%
\definecolor{currentfill}{rgb}{0.929357,0.512254,0.400673}%
\pgfsetfillcolor{currentfill}%
\pgfsetlinewidth{0.000000pt}%
\definecolor{currentstroke}{rgb}{0.000000,0.000000,0.000000}%
\pgfsetstrokecolor{currentstroke}%
\pgfsetdash{}{0pt}%
\pgfpathmoveto{\pgfqpoint{3.013676in}{2.882171in}}%
\pgfpathlineto{\pgfqpoint{3.060038in}{2.831217in}}%
\pgfpathlineto{\pgfqpoint{3.087074in}{3.005048in}}%
\pgfpathlineto{\pgfqpoint{3.040570in}{3.039678in}}%
\pgfpathlineto{\pgfqpoint{3.013676in}{2.882171in}}%
\pgfpathclose%
\pgfusepath{fill}%
\end{pgfscope}%
\begin{pgfscope}%
\pgfpathrectangle{\pgfqpoint{1.072000in}{0.528000in}}{\pgfqpoint{3.696000in}{3.696000in}}%
\pgfusepath{clip}%
\pgfsetbuttcap%
\pgfsetroundjoin%
\definecolor{currentfill}{rgb}{0.713852,0.808857,0.979386}%
\pgfsetfillcolor{currentfill}%
\pgfsetlinewidth{0.000000pt}%
\definecolor{currentstroke}{rgb}{0.000000,0.000000,0.000000}%
\pgfsetstrokecolor{currentstroke}%
\pgfsetdash{}{0pt}%
\pgfpathmoveto{\pgfqpoint{1.710651in}{2.207651in}}%
\pgfpathlineto{\pgfqpoint{1.760403in}{2.064470in}}%
\pgfpathlineto{\pgfqpoint{1.793700in}{1.969744in}}%
\pgfpathlineto{\pgfqpoint{1.744393in}{2.109478in}}%
\pgfpathlineto{\pgfqpoint{1.710651in}{2.207651in}}%
\pgfpathclose%
\pgfusepath{fill}%
\end{pgfscope}%
\begin{pgfscope}%
\pgfpathrectangle{\pgfqpoint{1.072000in}{0.528000in}}{\pgfqpoint{3.696000in}{3.696000in}}%
\pgfusepath{clip}%
\pgfsetbuttcap%
\pgfsetroundjoin%
\definecolor{currentfill}{rgb}{0.763520,0.178667,0.193396}%
\pgfsetfillcolor{currentfill}%
\pgfsetlinewidth{0.000000pt}%
\definecolor{currentstroke}{rgb}{0.000000,0.000000,0.000000}%
\pgfsetstrokecolor{currentstroke}%
\pgfsetdash{}{0pt}%
\pgfpathmoveto{\pgfqpoint{3.287446in}{3.284268in}}%
\pgfpathlineto{\pgfqpoint{3.334167in}{3.253552in}}%
\pgfpathlineto{\pgfqpoint{3.359399in}{3.199096in}}%
\pgfpathlineto{\pgfqpoint{3.312954in}{3.239842in}}%
\pgfpathlineto{\pgfqpoint{3.287446in}{3.284268in}}%
\pgfpathclose%
\pgfusepath{fill}%
\end{pgfscope}%
\begin{pgfscope}%
\pgfpathrectangle{\pgfqpoint{1.072000in}{0.528000in}}{\pgfqpoint{3.696000in}{3.696000in}}%
\pgfusepath{clip}%
\pgfsetbuttcap%
\pgfsetroundjoin%
\definecolor{currentfill}{rgb}{0.333490,0.446265,0.874452}%
\pgfsetfillcolor{currentfill}%
\pgfsetlinewidth{0.000000pt}%
\definecolor{currentstroke}{rgb}{0.000000,0.000000,0.000000}%
\pgfsetstrokecolor{currentstroke}%
\pgfsetdash{}{0pt}%
\pgfpathmoveto{\pgfqpoint{2.300328in}{1.472621in}}%
\pgfpathlineto{\pgfqpoint{2.345447in}{1.485135in}}%
\pgfpathlineto{\pgfqpoint{2.371646in}{1.612445in}}%
\pgfpathlineto{\pgfqpoint{2.326690in}{1.586037in}}%
\pgfpathlineto{\pgfqpoint{2.300328in}{1.472621in}}%
\pgfpathclose%
\pgfusepath{fill}%
\end{pgfscope}%
\begin{pgfscope}%
\pgfpathrectangle{\pgfqpoint{1.072000in}{0.528000in}}{\pgfqpoint{3.696000in}{3.696000in}}%
\pgfusepath{clip}%
\pgfsetbuttcap%
\pgfsetroundjoin%
\definecolor{currentfill}{rgb}{0.718985,0.811993,0.977656}%
\pgfsetfillcolor{currentfill}%
\pgfsetlinewidth{0.000000pt}%
\definecolor{currentstroke}{rgb}{0.000000,0.000000,0.000000}%
\pgfsetstrokecolor{currentstroke}%
\pgfsetdash{}{0pt}%
\pgfpathmoveto{\pgfqpoint{3.638846in}{2.192096in}}%
\pgfpathlineto{\pgfqpoint{3.682551in}{2.001337in}}%
\pgfpathlineto{\pgfqpoint{3.707454in}{2.007333in}}%
\pgfpathlineto{\pgfqpoint{3.663905in}{2.194933in}}%
\pgfpathlineto{\pgfqpoint{3.638846in}{2.192096in}}%
\pgfpathclose%
\pgfusepath{fill}%
\end{pgfscope}%
\begin{pgfscope}%
\pgfpathrectangle{\pgfqpoint{1.072000in}{0.528000in}}{\pgfqpoint{3.696000in}{3.696000in}}%
\pgfusepath{clip}%
\pgfsetbuttcap%
\pgfsetroundjoin%
\definecolor{currentfill}{rgb}{0.229806,0.298718,0.753683}%
\pgfsetfillcolor{currentfill}%
\pgfsetlinewidth{0.000000pt}%
\definecolor{currentstroke}{rgb}{0.000000,0.000000,0.000000}%
\pgfsetstrokecolor{currentstroke}%
\pgfsetdash{}{0pt}%
\pgfpathmoveto{\pgfqpoint{3.735039in}{1.361132in}}%
\pgfpathlineto{\pgfqpoint{3.782494in}{1.349895in}}%
\pgfpathlineto{\pgfqpoint{3.807549in}{1.371402in}}%
\pgfpathlineto{\pgfqpoint{3.760355in}{1.390745in}}%
\pgfpathlineto{\pgfqpoint{3.735039in}{1.361132in}}%
\pgfpathclose%
\pgfusepath{fill}%
\end{pgfscope}%
\begin{pgfscope}%
\pgfpathrectangle{\pgfqpoint{1.072000in}{0.528000in}}{\pgfqpoint{3.696000in}{3.696000in}}%
\pgfusepath{clip}%
\pgfsetbuttcap%
\pgfsetroundjoin%
\definecolor{currentfill}{rgb}{0.921406,0.491420,0.383408}%
\pgfsetfillcolor{currentfill}%
\pgfsetlinewidth{0.000000pt}%
\definecolor{currentstroke}{rgb}{0.000000,0.000000,0.000000}%
\pgfsetstrokecolor{currentstroke}%
\pgfsetdash{}{0pt}%
\pgfpathmoveto{\pgfqpoint{2.463116in}{2.868960in}}%
\pgfpathlineto{\pgfqpoint{2.507026in}{2.998142in}}%
\pgfpathlineto{\pgfqpoint{2.534351in}{3.055175in}}%
\pgfpathlineto{\pgfqpoint{2.490295in}{2.929463in}}%
\pgfpathlineto{\pgfqpoint{2.463116in}{2.868960in}}%
\pgfpathclose%
\pgfusepath{fill}%
\end{pgfscope}%
\begin{pgfscope}%
\pgfpathrectangle{\pgfqpoint{1.072000in}{0.528000in}}{\pgfqpoint{3.696000in}{3.696000in}}%
\pgfusepath{clip}%
\pgfsetbuttcap%
\pgfsetroundjoin%
\definecolor{currentfill}{rgb}{0.839351,0.861167,0.894494}%
\pgfsetfillcolor{currentfill}%
\pgfsetlinewidth{0.000000pt}%
\definecolor{currentstroke}{rgb}{0.000000,0.000000,0.000000}%
\pgfsetstrokecolor{currentstroke}%
\pgfsetdash{}{0pt}%
\pgfpathmoveto{\pgfqpoint{1.644076in}{2.415926in}}%
\pgfpathlineto{\pgfqpoint{1.694471in}{2.259769in}}%
\pgfpathlineto{\pgfqpoint{1.727946in}{2.168901in}}%
\pgfpathlineto{\pgfqpoint{1.677830in}{2.326304in}}%
\pgfpathlineto{\pgfqpoint{1.644076in}{2.415926in}}%
\pgfpathclose%
\pgfusepath{fill}%
\end{pgfscope}%
\begin{pgfscope}%
\pgfpathrectangle{\pgfqpoint{1.072000in}{0.528000in}}{\pgfqpoint{3.696000in}{3.696000in}}%
\pgfusepath{clip}%
\pgfsetbuttcap%
\pgfsetroundjoin%
\definecolor{currentfill}{rgb}{0.830187,0.304733,0.254891}%
\pgfsetfillcolor{currentfill}%
\pgfsetlinewidth{0.000000pt}%
\definecolor{currentstroke}{rgb}{0.000000,0.000000,0.000000}%
\pgfsetstrokecolor{currentstroke}%
\pgfsetdash{}{0pt}%
\pgfpathmoveto{\pgfqpoint{3.380789in}{3.208980in}}%
\pgfpathlineto{\pgfqpoint{3.427201in}{3.146359in}}%
\pgfpathlineto{\pgfqpoint{3.451733in}{3.069378in}}%
\pgfpathlineto{\pgfqpoint{3.405690in}{3.143078in}}%
\pgfpathlineto{\pgfqpoint{3.380789in}{3.208980in}}%
\pgfpathclose%
\pgfusepath{fill}%
\end{pgfscope}%
\begin{pgfscope}%
\pgfpathrectangle{\pgfqpoint{1.072000in}{0.528000in}}{\pgfqpoint{3.696000in}{3.696000in}}%
\pgfusepath{clip}%
\pgfsetbuttcap%
\pgfsetroundjoin%
\definecolor{currentfill}{rgb}{0.294718,0.393542,0.834384}%
\pgfsetfillcolor{currentfill}%
\pgfsetlinewidth{0.000000pt}%
\definecolor{currentstroke}{rgb}{0.000000,0.000000,0.000000}%
\pgfsetstrokecolor{currentstroke}%
\pgfsetdash{}{0pt}%
\pgfpathmoveto{\pgfqpoint{2.031143in}{1.535748in}}%
\pgfpathlineto{\pgfqpoint{2.077828in}{1.475791in}}%
\pgfpathlineto{\pgfqpoint{2.108148in}{1.433586in}}%
\pgfpathlineto{\pgfqpoint{2.062045in}{1.478868in}}%
\pgfpathlineto{\pgfqpoint{2.031143in}{1.535748in}}%
\pgfpathclose%
\pgfusepath{fill}%
\end{pgfscope}%
\begin{pgfscope}%
\pgfpathrectangle{\pgfqpoint{1.072000in}{0.528000in}}{\pgfqpoint{3.696000in}{3.696000in}}%
\pgfusepath{clip}%
\pgfsetbuttcap%
\pgfsetroundjoin%
\definecolor{currentfill}{rgb}{0.705673,0.015556,0.150233}%
\pgfsetfillcolor{currentfill}%
\pgfsetlinewidth{0.000000pt}%
\definecolor{currentstroke}{rgb}{0.000000,0.000000,0.000000}%
\pgfsetstrokecolor{currentstroke}%
\pgfsetdash{}{0pt}%
\pgfpathmoveto{\pgfqpoint{3.121025in}{3.322477in}}%
\pgfpathlineto{\pgfqpoint{3.167696in}{3.309147in}}%
\pgfpathlineto{\pgfqpoint{3.194026in}{3.320330in}}%
\pgfpathlineto{\pgfqpoint{3.147415in}{3.330153in}}%
\pgfpathlineto{\pgfqpoint{3.121025in}{3.322477in}}%
\pgfpathclose%
\pgfusepath{fill}%
\end{pgfscope}%
\begin{pgfscope}%
\pgfpathrectangle{\pgfqpoint{1.072000in}{0.528000in}}{\pgfqpoint{3.696000in}{3.696000in}}%
\pgfusepath{clip}%
\pgfsetbuttcap%
\pgfsetroundjoin%
\definecolor{currentfill}{rgb}{0.902659,0.447939,0.349721}%
\pgfsetfillcolor{currentfill}%
\pgfsetlinewidth{0.000000pt}%
\definecolor{currentstroke}{rgb}{0.000000,0.000000,0.000000}%
\pgfsetstrokecolor{currentstroke}%
\pgfsetdash{}{0pt}%
\pgfpathmoveto{\pgfqpoint{2.967317in}{2.984400in}}%
\pgfpathlineto{\pgfqpoint{3.013676in}{2.882171in}}%
\pgfpathlineto{\pgfqpoint{3.040570in}{3.039678in}}%
\pgfpathlineto{\pgfqpoint{2.994139in}{3.107684in}}%
\pgfpathlineto{\pgfqpoint{2.967317in}{2.984400in}}%
\pgfpathclose%
\pgfusepath{fill}%
\end{pgfscope}%
\begin{pgfscope}%
\pgfpathrectangle{\pgfqpoint{1.072000in}{0.528000in}}{\pgfqpoint{3.696000in}{3.696000in}}%
\pgfusepath{clip}%
\pgfsetbuttcap%
\pgfsetroundjoin%
\definecolor{currentfill}{rgb}{0.960581,0.762501,0.667964}%
\pgfsetfillcolor{currentfill}%
\pgfsetlinewidth{0.000000pt}%
\definecolor{currentstroke}{rgb}{0.000000,0.000000,0.000000}%
\pgfsetstrokecolor{currentstroke}%
\pgfsetdash{}{0pt}%
\pgfpathmoveto{\pgfqpoint{2.356686in}{2.468670in}}%
\pgfpathlineto{\pgfqpoint{2.399905in}{2.601621in}}%
\pgfpathlineto{\pgfqpoint{2.426272in}{2.726347in}}%
\pgfpathlineto{\pgfqpoint{2.382900in}{2.590982in}}%
\pgfpathlineto{\pgfqpoint{2.356686in}{2.468670in}}%
\pgfpathclose%
\pgfusepath{fill}%
\end{pgfscope}%
\begin{pgfscope}%
\pgfpathrectangle{\pgfqpoint{1.072000in}{0.528000in}}{\pgfqpoint{3.696000in}{3.696000in}}%
\pgfusepath{clip}%
\pgfsetbuttcap%
\pgfsetroundjoin%
\definecolor{currentfill}{rgb}{0.505423,0.643995,0.983157}%
\pgfsetfillcolor{currentfill}%
\pgfsetlinewidth{0.000000pt}%
\definecolor{currentstroke}{rgb}{0.000000,0.000000,0.000000}%
\pgfsetstrokecolor{currentstroke}%
\pgfsetdash{}{0pt}%
\pgfpathmoveto{\pgfqpoint{2.307997in}{1.685966in}}%
\pgfpathlineto{\pgfqpoint{2.352549in}{1.728999in}}%
\pgfpathlineto{\pgfqpoint{2.378097in}{1.894644in}}%
\pgfpathlineto{\pgfqpoint{2.333639in}{1.837096in}}%
\pgfpathlineto{\pgfqpoint{2.307997in}{1.685966in}}%
\pgfpathclose%
\pgfusepath{fill}%
\end{pgfscope}%
\begin{pgfscope}%
\pgfpathrectangle{\pgfqpoint{1.072000in}{0.528000in}}{\pgfqpoint{3.696000in}{3.696000in}}%
\pgfusepath{clip}%
\pgfsetbuttcap%
\pgfsetroundjoin%
\definecolor{currentfill}{rgb}{0.299441,0.400248,0.839842}%
\pgfsetfillcolor{currentfill}%
\pgfsetlinewidth{0.000000pt}%
\definecolor{currentstroke}{rgb}{0.000000,0.000000,0.000000}%
\pgfsetstrokecolor{currentstroke}%
\pgfsetdash{}{0pt}%
\pgfpathmoveto{\pgfqpoint{3.642970in}{1.507701in}}%
\pgfpathlineto{\pgfqpoint{3.688608in}{1.414558in}}%
\pgfpathlineto{\pgfqpoint{3.714206in}{1.452877in}}%
\pgfpathlineto{\pgfqpoint{3.668860in}{1.554962in}}%
\pgfpathlineto{\pgfqpoint{3.642970in}{1.507701in}}%
\pgfpathclose%
\pgfusepath{fill}%
\end{pgfscope}%
\begin{pgfscope}%
\pgfpathrectangle{\pgfqpoint{1.072000in}{0.528000in}}{\pgfqpoint{3.696000in}{3.696000in}}%
\pgfusepath{clip}%
\pgfsetbuttcap%
\pgfsetroundjoin%
\definecolor{currentfill}{rgb}{0.790562,0.231397,0.216242}%
\pgfsetfillcolor{currentfill}%
\pgfsetlinewidth{0.000000pt}%
\definecolor{currentstroke}{rgb}{0.000000,0.000000,0.000000}%
\pgfsetstrokecolor{currentstroke}%
\pgfsetdash{}{0pt}%
\pgfpathmoveto{\pgfqpoint{3.334167in}{3.253552in}}%
\pgfpathlineto{\pgfqpoint{3.380789in}{3.208980in}}%
\pgfpathlineto{\pgfqpoint{3.405690in}{3.143078in}}%
\pgfpathlineto{\pgfqpoint{3.359399in}{3.199096in}}%
\pgfpathlineto{\pgfqpoint{3.334167in}{3.253552in}}%
\pgfpathclose%
\pgfusepath{fill}%
\end{pgfscope}%
\begin{pgfscope}%
\pgfpathrectangle{\pgfqpoint{1.072000in}{0.528000in}}{\pgfqpoint{3.696000in}{3.696000in}}%
\pgfusepath{clip}%
\pgfsetbuttcap%
\pgfsetroundjoin%
\definecolor{currentfill}{rgb}{0.333490,0.446265,0.874452}%
\pgfsetfillcolor{currentfill}%
\pgfsetlinewidth{0.000000pt}%
\definecolor{currentstroke}{rgb}{0.000000,0.000000,0.000000}%
\pgfsetstrokecolor{currentstroke}%
\pgfsetdash{}{0pt}%
\pgfpathmoveto{\pgfqpoint{1.984098in}{1.607489in}}%
\pgfpathlineto{\pgfqpoint{2.031143in}{1.535748in}}%
\pgfpathlineto{\pgfqpoint{2.062045in}{1.478868in}}%
\pgfpathlineto{\pgfqpoint{2.015614in}{1.536508in}}%
\pgfpathlineto{\pgfqpoint{1.984098in}{1.607489in}}%
\pgfpathclose%
\pgfusepath{fill}%
\end{pgfscope}%
\begin{pgfscope}%
\pgfpathrectangle{\pgfqpoint{1.072000in}{0.528000in}}{\pgfqpoint{3.696000in}{3.696000in}}%
\pgfusepath{clip}%
\pgfsetbuttcap%
\pgfsetroundjoin%
\definecolor{currentfill}{rgb}{0.271104,0.360011,0.807095}%
\pgfsetfillcolor{currentfill}%
\pgfsetlinewidth{0.000000pt}%
\definecolor{currentstroke}{rgb}{0.000000,0.000000,0.000000}%
\pgfsetstrokecolor{currentstroke}%
\pgfsetdash{}{0pt}%
\pgfpathmoveto{\pgfqpoint{2.077828in}{1.475791in}}%
\pgfpathlineto{\pgfqpoint{2.124227in}{1.426495in}}%
\pgfpathlineto{\pgfqpoint{2.154013in}{1.398990in}}%
\pgfpathlineto{\pgfqpoint{2.108148in}{1.433586in}}%
\pgfpathlineto{\pgfqpoint{2.077828in}{1.475791in}}%
\pgfpathclose%
\pgfusepath{fill}%
\end{pgfscope}%
\begin{pgfscope}%
\pgfpathrectangle{\pgfqpoint{1.072000in}{0.528000in}}{\pgfqpoint{3.696000in}{3.696000in}}%
\pgfusepath{clip}%
\pgfsetbuttcap%
\pgfsetroundjoin%
\definecolor{currentfill}{rgb}{0.705673,0.015556,0.150233}%
\pgfsetfillcolor{currentfill}%
\pgfsetlinewidth{0.000000pt}%
\definecolor{currentstroke}{rgb}{0.000000,0.000000,0.000000}%
\pgfsetstrokecolor{currentstroke}%
\pgfsetdash{}{0pt}%
\pgfpathmoveto{\pgfqpoint{2.808325in}{3.316722in}}%
\pgfpathlineto{\pgfqpoint{2.854645in}{3.303768in}}%
\pgfpathlineto{\pgfqpoint{2.881662in}{3.329710in}}%
\pgfpathlineto{\pgfqpoint{2.835476in}{3.322092in}}%
\pgfpathlineto{\pgfqpoint{2.808325in}{3.316722in}}%
\pgfpathclose%
\pgfusepath{fill}%
\end{pgfscope}%
\begin{pgfscope}%
\pgfpathrectangle{\pgfqpoint{1.072000in}{0.528000in}}{\pgfqpoint{3.696000in}{3.696000in}}%
\pgfusepath{clip}%
\pgfsetbuttcap%
\pgfsetroundjoin%
\definecolor{currentfill}{rgb}{0.229806,0.298718,0.753683}%
\pgfsetfillcolor{currentfill}%
\pgfsetlinewidth{0.000000pt}%
\definecolor{currentstroke}{rgb}{0.000000,0.000000,0.000000}%
\pgfsetstrokecolor{currentstroke}%
\pgfsetdash{}{0pt}%
\pgfpathmoveto{\pgfqpoint{2.245371in}{1.354065in}}%
\pgfpathlineto{\pgfqpoint{2.291013in}{1.339684in}}%
\pgfpathlineto{\pgfqpoint{2.318627in}{1.392830in}}%
\pgfpathlineto{\pgfqpoint{2.273278in}{1.394106in}}%
\pgfpathlineto{\pgfqpoint{2.245371in}{1.354065in}}%
\pgfpathclose%
\pgfusepath{fill}%
\end{pgfscope}%
\begin{pgfscope}%
\pgfpathrectangle{\pgfqpoint{1.072000in}{0.528000in}}{\pgfqpoint{3.696000in}{3.696000in}}%
\pgfusepath{clip}%
\pgfsetbuttcap%
\pgfsetroundjoin%
\definecolor{currentfill}{rgb}{0.968500,0.673977,0.556649}%
\pgfsetfillcolor{currentfill}%
\pgfsetlinewidth{0.000000pt}%
\definecolor{currentstroke}{rgb}{0.000000,0.000000,0.000000}%
\pgfsetstrokecolor{currentstroke}%
\pgfsetdash{}{0pt}%
\pgfpathmoveto{\pgfqpoint{2.886626in}{2.827977in}}%
\pgfpathlineto{\pgfqpoint{2.933310in}{2.677220in}}%
\pgfpathlineto{\pgfqpoint{2.960170in}{2.626493in}}%
\pgfpathlineto{\pgfqpoint{2.913695in}{2.812279in}}%
\pgfpathlineto{\pgfqpoint{2.886626in}{2.827977in}}%
\pgfpathclose%
\pgfusepath{fill}%
\end{pgfscope}%
\begin{pgfscope}%
\pgfpathrectangle{\pgfqpoint{1.072000in}{0.528000in}}{\pgfqpoint{3.696000in}{3.696000in}}%
\pgfusepath{clip}%
\pgfsetbuttcap%
\pgfsetroundjoin%
\definecolor{currentfill}{rgb}{0.711554,0.033337,0.154485}%
\pgfsetfillcolor{currentfill}%
\pgfsetlinewidth{0.000000pt}%
\definecolor{currentstroke}{rgb}{0.000000,0.000000,0.000000}%
\pgfsetstrokecolor{currentstroke}%
\pgfsetdash{}{0pt}%
\pgfpathmoveto{\pgfqpoint{2.928042in}{3.308401in}}%
\pgfpathlineto{\pgfqpoint{2.974504in}{3.266570in}}%
\pgfpathlineto{\pgfqpoint{3.001304in}{3.321897in}}%
\pgfpathlineto{\pgfqpoint{2.954893in}{3.338743in}}%
\pgfpathlineto{\pgfqpoint{2.928042in}{3.308401in}}%
\pgfpathclose%
\pgfusepath{fill}%
\end{pgfscope}%
\begin{pgfscope}%
\pgfpathrectangle{\pgfqpoint{1.072000in}{0.528000in}}{\pgfqpoint{3.696000in}{3.696000in}}%
\pgfusepath{clip}%
\pgfsetbuttcap%
\pgfsetroundjoin%
\definecolor{currentfill}{rgb}{0.576051,0.708780,0.997755}%
\pgfsetfillcolor{currentfill}%
\pgfsetlinewidth{0.000000pt}%
\definecolor{currentstroke}{rgb}{0.000000,0.000000,0.000000}%
\pgfsetstrokecolor{currentstroke}%
\pgfsetdash{}{0pt}%
\pgfpathmoveto{\pgfqpoint{3.631662in}{1.953060in}}%
\pgfpathlineto{\pgfqpoint{3.675730in}{1.779293in}}%
\pgfpathlineto{\pgfqpoint{3.701154in}{1.808678in}}%
\pgfpathlineto{\pgfqpoint{3.657280in}{1.983338in}}%
\pgfpathlineto{\pgfqpoint{3.631662in}{1.953060in}}%
\pgfpathclose%
\pgfusepath{fill}%
\end{pgfscope}%
\begin{pgfscope}%
\pgfpathrectangle{\pgfqpoint{1.072000in}{0.528000in}}{\pgfqpoint{3.696000in}{3.696000in}}%
\pgfusepath{clip}%
\pgfsetbuttcap%
\pgfsetroundjoin%
\definecolor{currentfill}{rgb}{0.409611,0.540759,0.935545}%
\pgfsetfillcolor{currentfill}%
\pgfsetlinewidth{0.000000pt}%
\definecolor{currentstroke}{rgb}{0.000000,0.000000,0.000000}%
\pgfsetstrokecolor{currentstroke}%
\pgfsetdash{}{0pt}%
\pgfpathmoveto{\pgfqpoint{3.624051in}{1.691968in}}%
\pgfpathlineto{\pgfqpoint{3.668860in}{1.554962in}}%
\pgfpathlineto{\pgfqpoint{3.694569in}{1.596950in}}%
\pgfpathlineto{\pgfqpoint{3.650018in}{1.740153in}}%
\pgfpathlineto{\pgfqpoint{3.624051in}{1.691968in}}%
\pgfpathclose%
\pgfusepath{fill}%
\end{pgfscope}%
\begin{pgfscope}%
\pgfpathrectangle{\pgfqpoint{1.072000in}{0.528000in}}{\pgfqpoint{3.696000in}{3.696000in}}%
\pgfusepath{clip}%
\pgfsetbuttcap%
\pgfsetroundjoin%
\definecolor{currentfill}{rgb}{0.939254,0.539581,0.423900}%
\pgfsetfillcolor{currentfill}%
\pgfsetlinewidth{0.000000pt}%
\definecolor{currentstroke}{rgb}{0.000000,0.000000,0.000000}%
\pgfsetstrokecolor{currentstroke}%
\pgfsetdash{}{0pt}%
\pgfpathmoveto{\pgfqpoint{2.436133in}{2.792291in}}%
\pgfpathlineto{\pgfqpoint{2.479870in}{2.924913in}}%
\pgfpathlineto{\pgfqpoint{2.507026in}{2.998142in}}%
\pgfpathlineto{\pgfqpoint{2.463116in}{2.868960in}}%
\pgfpathlineto{\pgfqpoint{2.436133in}{2.792291in}}%
\pgfpathclose%
\pgfusepath{fill}%
\end{pgfscope}%
\begin{pgfscope}%
\pgfpathrectangle{\pgfqpoint{1.072000in}{0.528000in}}{\pgfqpoint{3.696000in}{3.696000in}}%
\pgfusepath{clip}%
\pgfsetbuttcap%
\pgfsetroundjoin%
\definecolor{currentfill}{rgb}{0.969289,0.684982,0.568975}%
\pgfsetfillcolor{currentfill}%
\pgfsetlinewidth{0.000000pt}%
\definecolor{currentstroke}{rgb}{0.000000,0.000000,0.000000}%
\pgfsetstrokecolor{currentstroke}%
\pgfsetdash{}{0pt}%
\pgfpathmoveto{\pgfqpoint{3.560905in}{2.825589in}}%
\pgfpathlineto{\pgfqpoint{3.605568in}{2.664308in}}%
\pgfpathlineto{\pgfqpoint{3.629445in}{2.601312in}}%
\pgfpathlineto{\pgfqpoint{3.585074in}{2.762686in}}%
\pgfpathlineto{\pgfqpoint{3.560905in}{2.825589in}}%
\pgfpathclose%
\pgfusepath{fill}%
\end{pgfscope}%
\begin{pgfscope}%
\pgfpathrectangle{\pgfqpoint{1.072000in}{0.528000in}}{\pgfqpoint{3.696000in}{3.696000in}}%
\pgfusepath{clip}%
\pgfsetbuttcap%
\pgfsetroundjoin%
\definecolor{currentfill}{rgb}{0.248091,0.326013,0.777669}%
\pgfsetfillcolor{currentfill}%
\pgfsetlinewidth{0.000000pt}%
\definecolor{currentstroke}{rgb}{0.000000,0.000000,0.000000}%
\pgfsetstrokecolor{currentstroke}%
\pgfsetdash{}{0pt}%
\pgfpathmoveto{\pgfqpoint{2.124227in}{1.426495in}}%
\pgfpathlineto{\pgfqpoint{2.170418in}{1.386450in}}%
\pgfpathlineto{\pgfqpoint{2.199728in}{1.373160in}}%
\pgfpathlineto{\pgfqpoint{2.154013in}{1.398990in}}%
\pgfpathlineto{\pgfqpoint{2.124227in}{1.426495in}}%
\pgfpathclose%
\pgfusepath{fill}%
\end{pgfscope}%
\begin{pgfscope}%
\pgfpathrectangle{\pgfqpoint{1.072000in}{0.528000in}}{\pgfqpoint{3.696000in}{3.696000in}}%
\pgfusepath{clip}%
\pgfsetbuttcap%
\pgfsetroundjoin%
\definecolor{currentfill}{rgb}{0.388852,0.516298,0.921373}%
\pgfsetfillcolor{currentfill}%
\pgfsetlinewidth{0.000000pt}%
\definecolor{currentstroke}{rgb}{0.000000,0.000000,0.000000}%
\pgfsetstrokecolor{currentstroke}%
\pgfsetdash{}{0pt}%
\pgfpathmoveto{\pgfqpoint{1.936624in}{1.691725in}}%
\pgfpathlineto{\pgfqpoint{1.984098in}{1.607489in}}%
\pgfpathlineto{\pgfqpoint{2.015614in}{1.536508in}}%
\pgfpathlineto{\pgfqpoint{1.968768in}{1.607788in}}%
\pgfpathlineto{\pgfqpoint{1.936624in}{1.691725in}}%
\pgfpathclose%
\pgfusepath{fill}%
\end{pgfscope}%
\begin{pgfscope}%
\pgfpathrectangle{\pgfqpoint{1.072000in}{0.528000in}}{\pgfqpoint{3.696000in}{3.696000in}}%
\pgfusepath{clip}%
\pgfsetbuttcap%
\pgfsetroundjoin%
\definecolor{currentfill}{rgb}{0.969289,0.684982,0.568975}%
\pgfsetfillcolor{currentfill}%
\pgfsetlinewidth{0.000000pt}%
\definecolor{currentstroke}{rgb}{0.000000,0.000000,0.000000}%
\pgfsetstrokecolor{currentstroke}%
\pgfsetdash{}{0pt}%
\pgfpathmoveto{\pgfqpoint{2.382900in}{2.590982in}}%
\pgfpathlineto{\pgfqpoint{2.426272in}{2.726347in}}%
\pgfpathlineto{\pgfqpoint{2.452936in}{2.834410in}}%
\pgfpathlineto{\pgfqpoint{2.409385in}{2.699408in}}%
\pgfpathlineto{\pgfqpoint{2.382900in}{2.590982in}}%
\pgfpathclose%
\pgfusepath{fill}%
\end{pgfscope}%
\begin{pgfscope}%
\pgfpathrectangle{\pgfqpoint{1.072000in}{0.528000in}}{\pgfqpoint{3.696000in}{3.696000in}}%
\pgfusepath{clip}%
\pgfsetbuttcap%
\pgfsetroundjoin%
\definecolor{currentfill}{rgb}{0.711554,0.033337,0.154485}%
\pgfsetfillcolor{currentfill}%
\pgfsetlinewidth{0.000000pt}%
\definecolor{currentstroke}{rgb}{0.000000,0.000000,0.000000}%
\pgfsetstrokecolor{currentstroke}%
\pgfsetdash{}{0pt}%
\pgfpathmoveto{\pgfqpoint{3.047790in}{3.295321in}}%
\pgfpathlineto{\pgfqpoint{3.094347in}{3.268935in}}%
\pgfpathlineto{\pgfqpoint{3.121025in}{3.322477in}}%
\pgfpathlineto{\pgfqpoint{3.074460in}{3.335739in}}%
\pgfpathlineto{\pgfqpoint{3.047790in}{3.295321in}}%
\pgfpathclose%
\pgfusepath{fill}%
\end{pgfscope}%
\begin{pgfscope}%
\pgfpathrectangle{\pgfqpoint{1.072000in}{0.528000in}}{\pgfqpoint{3.696000in}{3.696000in}}%
\pgfusepath{clip}%
\pgfsetbuttcap%
\pgfsetroundjoin%
\definecolor{currentfill}{rgb}{0.746838,0.140021,0.179996}%
\pgfsetfillcolor{currentfill}%
\pgfsetlinewidth{0.000000pt}%
\definecolor{currentstroke}{rgb}{0.000000,0.000000,0.000000}%
\pgfsetstrokecolor{currentstroke}%
\pgfsetdash{}{0pt}%
\pgfpathmoveto{\pgfqpoint{2.661861in}{3.229070in}}%
\pgfpathlineto{\pgfqpoint{2.707559in}{3.275731in}}%
\pgfpathlineto{\pgfqpoint{2.734980in}{3.288486in}}%
\pgfpathlineto{\pgfqpoint{2.689363in}{3.234966in}}%
\pgfpathlineto{\pgfqpoint{2.661861in}{3.229070in}}%
\pgfpathclose%
\pgfusepath{fill}%
\end{pgfscope}%
\begin{pgfscope}%
\pgfpathrectangle{\pgfqpoint{1.072000in}{0.528000in}}{\pgfqpoint{3.696000in}{3.696000in}}%
\pgfusepath{clip}%
\pgfsetbuttcap%
\pgfsetroundjoin%
\definecolor{currentfill}{rgb}{0.603162,0.731527,0.999565}%
\pgfsetfillcolor{currentfill}%
\pgfsetlinewidth{0.000000pt}%
\definecolor{currentstroke}{rgb}{0.000000,0.000000,0.000000}%
\pgfsetstrokecolor{currentstroke}%
\pgfsetdash{}{0pt}%
\pgfpathmoveto{\pgfqpoint{1.775995in}{2.033905in}}%
\pgfpathlineto{\pgfqpoint{1.825061in}{1.907614in}}%
\pgfpathlineto{\pgfqpoint{1.858085in}{1.812410in}}%
\pgfpathlineto{\pgfqpoint{1.809567in}{1.931730in}}%
\pgfpathlineto{\pgfqpoint{1.775995in}{2.033905in}}%
\pgfpathclose%
\pgfusepath{fill}%
\end{pgfscope}%
\begin{pgfscope}%
\pgfpathrectangle{\pgfqpoint{1.072000in}{0.528000in}}{\pgfqpoint{3.696000in}{3.696000in}}%
\pgfusepath{clip}%
\pgfsetbuttcap%
\pgfsetroundjoin%
\definecolor{currentfill}{rgb}{0.839351,0.861167,0.894494}%
\pgfsetfillcolor{currentfill}%
\pgfsetlinewidth{0.000000pt}%
\definecolor{currentstroke}{rgb}{0.000000,0.000000,0.000000}%
\pgfsetstrokecolor{currentstroke}%
\pgfsetdash{}{0pt}%
\pgfpathmoveto{\pgfqpoint{2.322438in}{2.157381in}}%
\pgfpathlineto{\pgfqpoint{2.365979in}{2.261309in}}%
\pgfpathlineto{\pgfqpoint{2.391637in}{2.425888in}}%
\pgfpathlineto{\pgfqpoint{2.348037in}{2.312794in}}%
\pgfpathlineto{\pgfqpoint{2.322438in}{2.157381in}}%
\pgfpathclose%
\pgfusepath{fill}%
\end{pgfscope}%
\begin{pgfscope}%
\pgfpathrectangle{\pgfqpoint{1.072000in}{0.528000in}}{\pgfqpoint{3.696000in}{3.696000in}}%
\pgfusepath{clip}%
\pgfsetbuttcap%
\pgfsetroundjoin%
\definecolor{currentfill}{rgb}{0.711554,0.033337,0.154485}%
\pgfsetfillcolor{currentfill}%
\pgfsetlinewidth{0.000000pt}%
\definecolor{currentstroke}{rgb}{0.000000,0.000000,0.000000}%
\pgfsetstrokecolor{currentstroke}%
\pgfsetdash{}{0pt}%
\pgfpathmoveto{\pgfqpoint{2.734980in}{3.288486in}}%
\pgfpathlineto{\pgfqpoint{2.781103in}{3.299493in}}%
\pgfpathlineto{\pgfqpoint{2.808325in}{3.316722in}}%
\pgfpathlineto{\pgfqpoint{2.762333in}{3.291704in}}%
\pgfpathlineto{\pgfqpoint{2.734980in}{3.288486in}}%
\pgfpathclose%
\pgfusepath{fill}%
\end{pgfscope}%
\begin{pgfscope}%
\pgfpathrectangle{\pgfqpoint{1.072000in}{0.528000in}}{\pgfqpoint{3.696000in}{3.696000in}}%
\pgfusepath{clip}%
\pgfsetbuttcap%
\pgfsetroundjoin%
\definecolor{currentfill}{rgb}{0.698454,0.799450,0.984577}%
\pgfsetfillcolor{currentfill}%
\pgfsetlinewidth{0.000000pt}%
\definecolor{currentstroke}{rgb}{0.000000,0.000000,0.000000}%
\pgfsetstrokecolor{currentstroke}%
\pgfsetdash{}{0pt}%
\pgfpathmoveto{\pgfqpoint{2.315063in}{1.929780in}}%
\pgfpathlineto{\pgfqpoint{2.359082in}{2.004838in}}%
\pgfpathlineto{\pgfqpoint{2.384500in}{2.181326in}}%
\pgfpathlineto{\pgfqpoint{2.340496in}{2.093593in}}%
\pgfpathlineto{\pgfqpoint{2.315063in}{1.929780in}}%
\pgfpathclose%
\pgfusepath{fill}%
\end{pgfscope}%
\begin{pgfscope}%
\pgfpathrectangle{\pgfqpoint{1.072000in}{0.528000in}}{\pgfqpoint{3.696000in}{3.696000in}}%
\pgfusepath{clip}%
\pgfsetbuttcap%
\pgfsetroundjoin%
\definecolor{currentfill}{rgb}{0.902659,0.447939,0.349721}%
\pgfsetfillcolor{currentfill}%
\pgfsetlinewidth{0.000000pt}%
\definecolor{currentstroke}{rgb}{0.000000,0.000000,0.000000}%
\pgfsetstrokecolor{currentstroke}%
\pgfsetdash{}{0pt}%
\pgfpathmoveto{\pgfqpoint{2.893934in}{3.021063in}}%
\pgfpathlineto{\pgfqpoint{2.940533in}{2.872755in}}%
\pgfpathlineto{\pgfqpoint{2.967317in}{2.984400in}}%
\pgfpathlineto{\pgfqpoint{2.920814in}{3.098438in}}%
\pgfpathlineto{\pgfqpoint{2.893934in}{3.021063in}}%
\pgfpathclose%
\pgfusepath{fill}%
\end{pgfscope}%
\begin{pgfscope}%
\pgfpathrectangle{\pgfqpoint{1.072000in}{0.528000in}}{\pgfqpoint{3.696000in}{3.696000in}}%
\pgfusepath{clip}%
\pgfsetbuttcap%
\pgfsetroundjoin%
\definecolor{currentfill}{rgb}{0.929357,0.512254,0.400673}%
\pgfsetfillcolor{currentfill}%
\pgfsetlinewidth{0.000000pt}%
\definecolor{currentstroke}{rgb}{0.000000,0.000000,0.000000}%
\pgfsetstrokecolor{currentstroke}%
\pgfsetdash{}{0pt}%
\pgfpathmoveto{\pgfqpoint{3.060038in}{2.831217in}}%
\pgfpathlineto{\pgfqpoint{3.106704in}{2.856407in}}%
\pgfpathlineto{\pgfqpoint{3.133832in}{3.014233in}}%
\pgfpathlineto{\pgfqpoint{3.087074in}{3.005048in}}%
\pgfpathlineto{\pgfqpoint{3.060038in}{2.831217in}}%
\pgfpathclose%
\pgfusepath{fill}%
\end{pgfscope}%
\begin{pgfscope}%
\pgfpathrectangle{\pgfqpoint{1.072000in}{0.528000in}}{\pgfqpoint{3.696000in}{3.696000in}}%
\pgfusepath{clip}%
\pgfsetbuttcap%
\pgfsetroundjoin%
\definecolor{currentfill}{rgb}{0.959385,0.610306,0.489382}%
\pgfsetfillcolor{currentfill}%
\pgfsetlinewidth{0.000000pt}%
\definecolor{currentstroke}{rgb}{0.000000,0.000000,0.000000}%
\pgfsetstrokecolor{currentstroke}%
\pgfsetdash{}{0pt}%
\pgfpathmoveto{\pgfqpoint{2.409385in}{2.699408in}}%
\pgfpathlineto{\pgfqpoint{2.452936in}{2.834410in}}%
\pgfpathlineto{\pgfqpoint{2.479870in}{2.924913in}}%
\pgfpathlineto{\pgfqpoint{2.436133in}{2.792291in}}%
\pgfpathlineto{\pgfqpoint{2.409385in}{2.699408in}}%
\pgfpathclose%
\pgfusepath{fill}%
\end{pgfscope}%
\begin{pgfscope}%
\pgfpathrectangle{\pgfqpoint{1.072000in}{0.528000in}}{\pgfqpoint{3.696000in}{3.696000in}}%
\pgfusepath{clip}%
\pgfsetbuttcap%
\pgfsetroundjoin%
\definecolor{currentfill}{rgb}{0.729196,0.086679,0.167240}%
\pgfsetfillcolor{currentfill}%
\pgfsetlinewidth{0.000000pt}%
\definecolor{currentstroke}{rgb}{0.000000,0.000000,0.000000}%
\pgfsetstrokecolor{currentstroke}%
\pgfsetdash{}{0pt}%
\pgfpathmoveto{\pgfqpoint{2.974504in}{3.266570in}}%
\pgfpathlineto{\pgfqpoint{3.020985in}{3.216408in}}%
\pgfpathlineto{\pgfqpoint{3.047790in}{3.295321in}}%
\pgfpathlineto{\pgfqpoint{3.001304in}{3.321897in}}%
\pgfpathlineto{\pgfqpoint{2.974504in}{3.266570in}}%
\pgfpathclose%
\pgfusepath{fill}%
\end{pgfscope}%
\begin{pgfscope}%
\pgfpathrectangle{\pgfqpoint{1.072000in}{0.528000in}}{\pgfqpoint{3.696000in}{3.696000in}}%
\pgfusepath{clip}%
\pgfsetbuttcap%
\pgfsetroundjoin%
\definecolor{currentfill}{rgb}{0.839351,0.861167,0.894494}%
\pgfsetfillcolor{currentfill}%
\pgfsetlinewidth{0.000000pt}%
\definecolor{currentstroke}{rgb}{0.000000,0.000000,0.000000}%
\pgfsetstrokecolor{currentstroke}%
\pgfsetdash{}{0pt}%
\pgfpathmoveto{\pgfqpoint{3.620132in}{2.390277in}}%
\pgfpathlineto{\pgfqpoint{3.663905in}{2.194933in}}%
\pgfpathlineto{\pgfqpoint{3.688556in}{2.184487in}}%
\pgfpathlineto{\pgfqpoint{3.644953in}{2.376017in}}%
\pgfpathlineto{\pgfqpoint{3.620132in}{2.390277in}}%
\pgfpathclose%
\pgfusepath{fill}%
\end{pgfscope}%
\begin{pgfscope}%
\pgfpathrectangle{\pgfqpoint{1.072000in}{0.528000in}}{\pgfqpoint{3.696000in}{3.696000in}}%
\pgfusepath{clip}%
\pgfsetbuttcap%
\pgfsetroundjoin%
\definecolor{currentfill}{rgb}{0.430507,0.564883,0.948889}%
\pgfsetfillcolor{currentfill}%
\pgfsetlinewidth{0.000000pt}%
\definecolor{currentstroke}{rgb}{0.000000,0.000000,0.000000}%
\pgfsetstrokecolor{currentstroke}%
\pgfsetdash{}{0pt}%
\pgfpathmoveto{\pgfqpoint{2.326690in}{1.586037in}}%
\pgfpathlineto{\pgfqpoint{2.371646in}{1.612445in}}%
\pgfpathlineto{\pgfqpoint{2.397415in}{1.768750in}}%
\pgfpathlineto{\pgfqpoint{2.352549in}{1.728999in}}%
\pgfpathlineto{\pgfqpoint{2.326690in}{1.586037in}}%
\pgfpathclose%
\pgfusepath{fill}%
\end{pgfscope}%
\begin{pgfscope}%
\pgfpathrectangle{\pgfqpoint{1.072000in}{0.528000in}}{\pgfqpoint{3.696000in}{3.696000in}}%
\pgfusepath{clip}%
\pgfsetbuttcap%
\pgfsetroundjoin%
\definecolor{currentfill}{rgb}{0.852378,0.346492,0.280346}%
\pgfsetfillcolor{currentfill}%
\pgfsetlinewidth{0.000000pt}%
\definecolor{currentstroke}{rgb}{0.000000,0.000000,0.000000}%
\pgfsetstrokecolor{currentstroke}%
\pgfsetdash{}{0pt}%
\pgfpathmoveto{\pgfqpoint{2.920814in}{3.098438in}}%
\pgfpathlineto{\pgfqpoint{2.967317in}{2.984400in}}%
\pgfpathlineto{\pgfqpoint{2.994139in}{3.107684in}}%
\pgfpathlineto{\pgfqpoint{2.947663in}{3.187135in}}%
\pgfpathlineto{\pgfqpoint{2.920814in}{3.098438in}}%
\pgfpathclose%
\pgfusepath{fill}%
\end{pgfscope}%
\begin{pgfscope}%
\pgfpathrectangle{\pgfqpoint{1.072000in}{0.528000in}}{\pgfqpoint{3.696000in}{3.696000in}}%
\pgfusepath{clip}%
\pgfsetbuttcap%
\pgfsetroundjoin%
\definecolor{currentfill}{rgb}{0.238948,0.312365,0.765676}%
\pgfsetfillcolor{currentfill}%
\pgfsetlinewidth{0.000000pt}%
\definecolor{currentstroke}{rgb}{0.000000,0.000000,0.000000}%
\pgfsetstrokecolor{currentstroke}%
\pgfsetdash{}{0pt}%
\pgfpathmoveto{\pgfqpoint{3.662905in}{1.375054in}}%
\pgfpathlineto{\pgfqpoint{3.709654in}{1.332583in}}%
\pgfpathlineto{\pgfqpoint{3.735039in}{1.361132in}}%
\pgfpathlineto{\pgfqpoint{3.688608in}{1.414558in}}%
\pgfpathlineto{\pgfqpoint{3.662905in}{1.375054in}}%
\pgfpathclose%
\pgfusepath{fill}%
\end{pgfscope}%
\begin{pgfscope}%
\pgfpathrectangle{\pgfqpoint{1.072000in}{0.528000in}}{\pgfqpoint{3.696000in}{3.696000in}}%
\pgfusepath{clip}%
\pgfsetbuttcap%
\pgfsetroundjoin%
\definecolor{currentfill}{rgb}{0.830187,0.304733,0.254891}%
\pgfsetfillcolor{currentfill}%
\pgfsetlinewidth{0.000000pt}%
\definecolor{currentstroke}{rgb}{0.000000,0.000000,0.000000}%
\pgfsetstrokecolor{currentstroke}%
\pgfsetdash{}{0pt}%
\pgfpathmoveto{\pgfqpoint{2.994139in}{3.107684in}}%
\pgfpathlineto{\pgfqpoint{3.040570in}{3.039678in}}%
\pgfpathlineto{\pgfqpoint{3.067492in}{3.172438in}}%
\pgfpathlineto{\pgfqpoint{3.020985in}{3.216408in}}%
\pgfpathlineto{\pgfqpoint{2.994139in}{3.107684in}}%
\pgfpathclose%
\pgfusepath{fill}%
\end{pgfscope}%
\begin{pgfscope}%
\pgfpathrectangle{\pgfqpoint{1.072000in}{0.528000in}}{\pgfqpoint{3.696000in}{3.696000in}}%
\pgfusepath{clip}%
\pgfsetbuttcap%
\pgfsetroundjoin%
\definecolor{currentfill}{rgb}{0.926883,0.505422,0.394866}%
\pgfsetfillcolor{currentfill}%
\pgfsetlinewidth{0.000000pt}%
\definecolor{currentstroke}{rgb}{0.000000,0.000000,0.000000}%
\pgfsetstrokecolor{currentstroke}%
\pgfsetdash{}{0pt}%
\pgfpathmoveto{\pgfqpoint{3.494599in}{3.027898in}}%
\pgfpathlineto{\pgfqpoint{3.540126in}{2.906597in}}%
\pgfpathlineto{\pgfqpoint{3.564101in}{2.829246in}}%
\pgfpathlineto{\pgfqpoint{3.518946in}{2.956615in}}%
\pgfpathlineto{\pgfqpoint{3.494599in}{3.027898in}}%
\pgfpathclose%
\pgfusepath{fill}%
\end{pgfscope}%
\begin{pgfscope}%
\pgfpathrectangle{\pgfqpoint{1.072000in}{0.528000in}}{\pgfqpoint{3.696000in}{3.696000in}}%
\pgfusepath{clip}%
\pgfsetbuttcap%
\pgfsetroundjoin%
\definecolor{currentfill}{rgb}{0.705673,0.015556,0.150233}%
\pgfsetfillcolor{currentfill}%
\pgfsetlinewidth{0.000000pt}%
\definecolor{currentstroke}{rgb}{0.000000,0.000000,0.000000}%
\pgfsetstrokecolor{currentstroke}%
\pgfsetdash{}{0pt}%
\pgfpathmoveto{\pgfqpoint{3.167696in}{3.309147in}}%
\pgfpathlineto{\pgfqpoint{3.214486in}{3.297688in}}%
\pgfpathlineto{\pgfqpoint{3.240713in}{3.305442in}}%
\pgfpathlineto{\pgfqpoint{3.194026in}{3.320330in}}%
\pgfpathlineto{\pgfqpoint{3.167696in}{3.309147in}}%
\pgfpathclose%
\pgfusepath{fill}%
\end{pgfscope}%
\begin{pgfscope}%
\pgfpathrectangle{\pgfqpoint{1.072000in}{0.528000in}}{\pgfqpoint{3.696000in}{3.696000in}}%
\pgfusepath{clip}%
\pgfsetbuttcap%
\pgfsetroundjoin%
\definecolor{currentfill}{rgb}{0.711554,0.033337,0.154485}%
\pgfsetfillcolor{currentfill}%
\pgfsetlinewidth{0.000000pt}%
\definecolor{currentstroke}{rgb}{0.000000,0.000000,0.000000}%
\pgfsetstrokecolor{currentstroke}%
\pgfsetdash{}{0pt}%
\pgfpathmoveto{\pgfqpoint{2.854645in}{3.303768in}}%
\pgfpathlineto{\pgfqpoint{2.901137in}{3.257456in}}%
\pgfpathlineto{\pgfqpoint{2.928042in}{3.308401in}}%
\pgfpathlineto{\pgfqpoint{2.881662in}{3.329710in}}%
\pgfpathlineto{\pgfqpoint{2.854645in}{3.303768in}}%
\pgfpathclose%
\pgfusepath{fill}%
\end{pgfscope}%
\begin{pgfscope}%
\pgfpathrectangle{\pgfqpoint{1.072000in}{0.528000in}}{\pgfqpoint{3.696000in}{3.696000in}}%
\pgfusepath{clip}%
\pgfsetbuttcap%
\pgfsetroundjoin%
\definecolor{currentfill}{rgb}{0.280550,0.373423,0.818011}%
\pgfsetfillcolor{currentfill}%
\pgfsetlinewidth{0.000000pt}%
\definecolor{currentstroke}{rgb}{0.000000,0.000000,0.000000}%
\pgfsetstrokecolor{currentstroke}%
\pgfsetdash{}{0pt}%
\pgfpathmoveto{\pgfqpoint{2.318627in}{1.392830in}}%
\pgfpathlineto{\pgfqpoint{2.364103in}{1.392410in}}%
\pgfpathlineto{\pgfqpoint{2.390767in}{1.496324in}}%
\pgfpathlineto{\pgfqpoint{2.345447in}{1.485135in}}%
\pgfpathlineto{\pgfqpoint{2.318627in}{1.392830in}}%
\pgfpathclose%
\pgfusepath{fill}%
\end{pgfscope}%
\begin{pgfscope}%
\pgfpathrectangle{\pgfqpoint{1.072000in}{0.528000in}}{\pgfqpoint{3.696000in}{3.696000in}}%
\pgfusepath{clip}%
\pgfsetbuttcap%
\pgfsetroundjoin%
\definecolor{currentfill}{rgb}{0.785153,0.220851,0.211673}%
\pgfsetfillcolor{currentfill}%
\pgfsetlinewidth{0.000000pt}%
\definecolor{currentstroke}{rgb}{0.000000,0.000000,0.000000}%
\pgfsetstrokecolor{currentstroke}%
\pgfsetdash{}{0pt}%
\pgfpathmoveto{\pgfqpoint{2.947663in}{3.187135in}}%
\pgfpathlineto{\pgfqpoint{2.994139in}{3.107684in}}%
\pgfpathlineto{\pgfqpoint{3.020985in}{3.216408in}}%
\pgfpathlineto{\pgfqpoint{2.974504in}{3.266570in}}%
\pgfpathlineto{\pgfqpoint{2.947663in}{3.187135in}}%
\pgfpathclose%
\pgfusepath{fill}%
\end{pgfscope}%
\begin{pgfscope}%
\pgfpathrectangle{\pgfqpoint{1.072000in}{0.528000in}}{\pgfqpoint{3.696000in}{3.696000in}}%
\pgfusepath{clip}%
\pgfsetbuttcap%
\pgfsetroundjoin%
\definecolor{currentfill}{rgb}{0.238948,0.312365,0.765676}%
\pgfsetfillcolor{currentfill}%
\pgfsetlinewidth{0.000000pt}%
\definecolor{currentstroke}{rgb}{0.000000,0.000000,0.000000}%
\pgfsetstrokecolor{currentstroke}%
\pgfsetdash{}{0pt}%
\pgfpathmoveto{\pgfqpoint{2.170418in}{1.386450in}}%
\pgfpathlineto{\pgfqpoint{2.216474in}{1.354087in}}%
\pgfpathlineto{\pgfqpoint{2.245371in}{1.354065in}}%
\pgfpathlineto{\pgfqpoint{2.199728in}{1.373160in}}%
\pgfpathlineto{\pgfqpoint{2.170418in}{1.386450in}}%
\pgfpathclose%
\pgfusepath{fill}%
\end{pgfscope}%
\begin{pgfscope}%
\pgfpathrectangle{\pgfqpoint{1.072000in}{0.528000in}}{\pgfqpoint{3.696000in}{3.696000in}}%
\pgfusepath{clip}%
\pgfsetbuttcap%
\pgfsetroundjoin%
\definecolor{currentfill}{rgb}{0.852378,0.346492,0.280346}%
\pgfsetfillcolor{currentfill}%
\pgfsetlinewidth{0.000000pt}%
\definecolor{currentstroke}{rgb}{0.000000,0.000000,0.000000}%
\pgfsetstrokecolor{currentstroke}%
\pgfsetdash{}{0pt}%
\pgfpathmoveto{\pgfqpoint{3.040570in}{3.039678in}}%
\pgfpathlineto{\pgfqpoint{3.087074in}{3.005048in}}%
\pgfpathlineto{\pgfqpoint{3.114103in}{3.147589in}}%
\pgfpathlineto{\pgfqpoint{3.067492in}{3.172438in}}%
\pgfpathlineto{\pgfqpoint{3.040570in}{3.039678in}}%
\pgfpathclose%
\pgfusepath{fill}%
\end{pgfscope}%
\begin{pgfscope}%
\pgfpathrectangle{\pgfqpoint{1.072000in}{0.528000in}}{\pgfqpoint{3.696000in}{3.696000in}}%
\pgfusepath{clip}%
\pgfsetbuttcap%
\pgfsetroundjoin%
\definecolor{currentfill}{rgb}{0.758112,0.168122,0.188827}%
\pgfsetfillcolor{currentfill}%
\pgfsetlinewidth{0.000000pt}%
\definecolor{currentstroke}{rgb}{0.000000,0.000000,0.000000}%
\pgfsetstrokecolor{currentstroke}%
\pgfsetdash{}{0pt}%
\pgfpathmoveto{\pgfqpoint{3.020985in}{3.216408in}}%
\pgfpathlineto{\pgfqpoint{3.067492in}{3.172438in}}%
\pgfpathlineto{\pgfqpoint{3.094347in}{3.268935in}}%
\pgfpathlineto{\pgfqpoint{3.047790in}{3.295321in}}%
\pgfpathlineto{\pgfqpoint{3.020985in}{3.216408in}}%
\pgfpathclose%
\pgfusepath{fill}%
\end{pgfscope}%
\begin{pgfscope}%
\pgfpathrectangle{\pgfqpoint{1.072000in}{0.528000in}}{\pgfqpoint{3.696000in}{3.696000in}}%
\pgfusepath{clip}%
\pgfsetbuttcap%
\pgfsetroundjoin%
\definecolor{currentfill}{rgb}{0.929357,0.512254,0.400673}%
\pgfsetfillcolor{currentfill}%
\pgfsetlinewidth{0.000000pt}%
\definecolor{currentstroke}{rgb}{0.000000,0.000000,0.000000}%
\pgfsetstrokecolor{currentstroke}%
\pgfsetdash{}{0pt}%
\pgfpathmoveto{\pgfqpoint{2.866932in}{2.976997in}}%
\pgfpathlineto{\pgfqpoint{2.913695in}{2.812279in}}%
\pgfpathlineto{\pgfqpoint{2.940533in}{2.872755in}}%
\pgfpathlineto{\pgfqpoint{2.893934in}{3.021063in}}%
\pgfpathlineto{\pgfqpoint{2.866932in}{2.976997in}}%
\pgfpathclose%
\pgfusepath{fill}%
\end{pgfscope}%
\begin{pgfscope}%
\pgfpathrectangle{\pgfqpoint{1.072000in}{0.528000in}}{\pgfqpoint{3.696000in}{3.696000in}}%
\pgfusepath{clip}%
\pgfsetbuttcap%
\pgfsetroundjoin%
\definecolor{currentfill}{rgb}{0.740957,0.122240,0.175744}%
\pgfsetfillcolor{currentfill}%
\pgfsetlinewidth{0.000000pt}%
\definecolor{currentstroke}{rgb}{0.000000,0.000000,0.000000}%
\pgfsetstrokecolor{currentstroke}%
\pgfsetdash{}{0pt}%
\pgfpathmoveto{\pgfqpoint{2.901137in}{3.257456in}}%
\pgfpathlineto{\pgfqpoint{2.947663in}{3.187135in}}%
\pgfpathlineto{\pgfqpoint{2.974504in}{3.266570in}}%
\pgfpathlineto{\pgfqpoint{2.928042in}{3.308401in}}%
\pgfpathlineto{\pgfqpoint{2.901137in}{3.257456in}}%
\pgfpathclose%
\pgfusepath{fill}%
\end{pgfscope}%
\begin{pgfscope}%
\pgfpathrectangle{\pgfqpoint{1.072000in}{0.528000in}}{\pgfqpoint{3.696000in}{3.696000in}}%
\pgfusepath{clip}%
\pgfsetbuttcap%
\pgfsetroundjoin%
\definecolor{currentfill}{rgb}{0.280550,0.373423,0.818011}%
\pgfsetfillcolor{currentfill}%
\pgfsetlinewidth{0.000000pt}%
\definecolor{currentstroke}{rgb}{0.000000,0.000000,0.000000}%
\pgfsetstrokecolor{currentstroke}%
\pgfsetdash{}{0pt}%
\pgfpathmoveto{\pgfqpoint{3.616944in}{1.456837in}}%
\pgfpathlineto{\pgfqpoint{3.662905in}{1.375054in}}%
\pgfpathlineto{\pgfqpoint{3.688608in}{1.414558in}}%
\pgfpathlineto{\pgfqpoint{3.642970in}{1.507701in}}%
\pgfpathlineto{\pgfqpoint{3.616944in}{1.456837in}}%
\pgfpathclose%
\pgfusepath{fill}%
\end{pgfscope}%
\begin{pgfscope}%
\pgfpathrectangle{\pgfqpoint{1.072000in}{0.528000in}}{\pgfqpoint{3.696000in}{3.696000in}}%
\pgfusepath{clip}%
\pgfsetbuttcap%
\pgfsetroundjoin%
\definecolor{currentfill}{rgb}{0.451739,0.588181,0.960201}%
\pgfsetfillcolor{currentfill}%
\pgfsetlinewidth{0.000000pt}%
\definecolor{currentstroke}{rgb}{0.000000,0.000000,0.000000}%
\pgfsetstrokecolor{currentstroke}%
\pgfsetdash{}{0pt}%
\pgfpathmoveto{\pgfqpoint{1.888665in}{1.788637in}}%
\pgfpathlineto{\pgfqpoint{1.936624in}{1.691725in}}%
\pgfpathlineto{\pgfqpoint{1.968768in}{1.607788in}}%
\pgfpathlineto{\pgfqpoint{1.921430in}{1.693462in}}%
\pgfpathlineto{\pgfqpoint{1.888665in}{1.788637in}}%
\pgfpathclose%
\pgfusepath{fill}%
\end{pgfscope}%
\begin{pgfscope}%
\pgfpathrectangle{\pgfqpoint{1.072000in}{0.528000in}}{\pgfqpoint{3.696000in}{3.696000in}}%
\pgfusepath{clip}%
\pgfsetbuttcap%
\pgfsetroundjoin%
\definecolor{currentfill}{rgb}{0.962708,0.753557,0.655601}%
\pgfsetfillcolor{currentfill}%
\pgfsetlinewidth{0.000000pt}%
\definecolor{currentstroke}{rgb}{0.000000,0.000000,0.000000}%
\pgfsetstrokecolor{currentstroke}%
\pgfsetdash{}{0pt}%
\pgfpathmoveto{\pgfqpoint{3.581200in}{2.712154in}}%
\pgfpathlineto{\pgfqpoint{3.625530in}{2.534526in}}%
\pgfpathlineto{\pgfqpoint{3.649651in}{2.488481in}}%
\pgfpathlineto{\pgfqpoint{3.605568in}{2.664308in}}%
\pgfpathlineto{\pgfqpoint{3.581200in}{2.712154in}}%
\pgfpathclose%
\pgfusepath{fill}%
\end{pgfscope}%
\begin{pgfscope}%
\pgfpathrectangle{\pgfqpoint{1.072000in}{0.528000in}}{\pgfqpoint{3.696000in}{3.696000in}}%
\pgfusepath{clip}%
\pgfsetbuttcap%
\pgfsetroundjoin%
\definecolor{currentfill}{rgb}{0.810616,0.268797,0.235428}%
\pgfsetfillcolor{currentfill}%
\pgfsetlinewidth{0.000000pt}%
\definecolor{currentstroke}{rgb}{0.000000,0.000000,0.000000}%
\pgfsetstrokecolor{currentstroke}%
\pgfsetdash{}{0pt}%
\pgfpathmoveto{\pgfqpoint{2.561791in}{3.097359in}}%
\pgfpathlineto{\pgfqpoint{2.606774in}{3.185689in}}%
\pgfpathlineto{\pgfqpoint{2.634326in}{3.212417in}}%
\pgfpathlineto{\pgfqpoint{2.589298in}{3.125791in}}%
\pgfpathlineto{\pgfqpoint{2.561791in}{3.097359in}}%
\pgfpathclose%
\pgfusepath{fill}%
\end{pgfscope}%
\begin{pgfscope}%
\pgfpathrectangle{\pgfqpoint{1.072000in}{0.528000in}}{\pgfqpoint{3.696000in}{3.696000in}}%
\pgfusepath{clip}%
\pgfsetbuttcap%
\pgfsetroundjoin%
\definecolor{currentfill}{rgb}{0.928116,0.822197,0.765141}%
\pgfsetfillcolor{currentfill}%
\pgfsetlinewidth{0.000000pt}%
\definecolor{currentstroke}{rgb}{0.000000,0.000000,0.000000}%
\pgfsetstrokecolor{currentstroke}%
\pgfsetdash{}{0pt}%
\pgfpathmoveto{\pgfqpoint{3.600944in}{2.566072in}}%
\pgfpathlineto{\pgfqpoint{3.644953in}{2.376017in}}%
\pgfpathlineto{\pgfqpoint{3.669336in}{2.347758in}}%
\pgfpathlineto{\pgfqpoint{3.625530in}{2.534526in}}%
\pgfpathlineto{\pgfqpoint{3.600944in}{2.566072in}}%
\pgfpathclose%
\pgfusepath{fill}%
\end{pgfscope}%
\begin{pgfscope}%
\pgfpathrectangle{\pgfqpoint{1.072000in}{0.528000in}}{\pgfqpoint{3.696000in}{3.696000in}}%
\pgfusepath{clip}%
\pgfsetbuttcap%
\pgfsetroundjoin%
\definecolor{currentfill}{rgb}{0.238948,0.312365,0.765676}%
\pgfsetfillcolor{currentfill}%
\pgfsetlinewidth{0.000000pt}%
\definecolor{currentstroke}{rgb}{0.000000,0.000000,0.000000}%
\pgfsetstrokecolor{currentstroke}%
\pgfsetdash{}{0pt}%
\pgfpathmoveto{\pgfqpoint{2.291013in}{1.339684in}}%
\pgfpathlineto{\pgfqpoint{2.336707in}{1.328126in}}%
\pgfpathlineto{\pgfqpoint{2.364103in}{1.392410in}}%
\pgfpathlineto{\pgfqpoint{2.318627in}{1.392830in}}%
\pgfpathlineto{\pgfqpoint{2.291013in}{1.339684in}}%
\pgfpathclose%
\pgfusepath{fill}%
\end{pgfscope}%
\begin{pgfscope}%
\pgfpathrectangle{\pgfqpoint{1.072000in}{0.528000in}}{\pgfqpoint{3.696000in}{3.696000in}}%
\pgfusepath{clip}%
\pgfsetbuttcap%
\pgfsetroundjoin%
\definecolor{currentfill}{rgb}{0.717435,0.051118,0.158737}%
\pgfsetfillcolor{currentfill}%
\pgfsetlinewidth{0.000000pt}%
\definecolor{currentstroke}{rgb}{0.000000,0.000000,0.000000}%
\pgfsetstrokecolor{currentstroke}%
\pgfsetdash{}{0pt}%
\pgfpathmoveto{\pgfqpoint{3.094347in}{3.268935in}}%
\pgfpathlineto{\pgfqpoint{3.141017in}{3.250775in}}%
\pgfpathlineto{\pgfqpoint{3.167696in}{3.309147in}}%
\pgfpathlineto{\pgfqpoint{3.121025in}{3.322477in}}%
\pgfpathlineto{\pgfqpoint{3.094347in}{3.268935in}}%
\pgfpathclose%
\pgfusepath{fill}%
\end{pgfscope}%
\begin{pgfscope}%
\pgfpathrectangle{\pgfqpoint{1.072000in}{0.528000in}}{\pgfqpoint{3.696000in}{3.696000in}}%
\pgfusepath{clip}%
\pgfsetbuttcap%
\pgfsetroundjoin%
\definecolor{currentfill}{rgb}{0.968500,0.673977,0.556649}%
\pgfsetfillcolor{currentfill}%
\pgfsetlinewidth{0.000000pt}%
\definecolor{currentstroke}{rgb}{0.000000,0.000000,0.000000}%
\pgfsetstrokecolor{currentstroke}%
\pgfsetdash{}{0pt}%
\pgfpathmoveto{\pgfqpoint{3.052886in}{2.600187in}}%
\pgfpathlineto{\pgfqpoint{3.100050in}{2.760212in}}%
\pgfpathlineto{\pgfqpoint{3.126840in}{2.821017in}}%
\pgfpathlineto{\pgfqpoint{3.079668in}{2.696393in}}%
\pgfpathlineto{\pgfqpoint{3.052886in}{2.600187in}}%
\pgfpathclose%
\pgfusepath{fill}%
\end{pgfscope}%
\begin{pgfscope}%
\pgfpathrectangle{\pgfqpoint{1.072000in}{0.528000in}}{\pgfqpoint{3.696000in}{3.696000in}}%
\pgfusepath{clip}%
\pgfsetbuttcap%
\pgfsetroundjoin%
\definecolor{currentfill}{rgb}{0.388852,0.516298,0.921373}%
\pgfsetfillcolor{currentfill}%
\pgfsetlinewidth{0.000000pt}%
\definecolor{currentstroke}{rgb}{0.000000,0.000000,0.000000}%
\pgfsetstrokecolor{currentstroke}%
\pgfsetdash{}{0pt}%
\pgfpathmoveto{\pgfqpoint{3.597873in}{1.635954in}}%
\pgfpathlineto{\pgfqpoint{3.642970in}{1.507701in}}%
\pgfpathlineto{\pgfqpoint{3.668860in}{1.554962in}}%
\pgfpathlineto{\pgfqpoint{3.624051in}{1.691968in}}%
\pgfpathlineto{\pgfqpoint{3.597873in}{1.635954in}}%
\pgfpathclose%
\pgfusepath{fill}%
\end{pgfscope}%
\begin{pgfscope}%
\pgfpathrectangle{\pgfqpoint{1.072000in}{0.528000in}}{\pgfqpoint{3.696000in}{3.696000in}}%
\pgfusepath{clip}%
\pgfsetbuttcap%
\pgfsetroundjoin%
\definecolor{currentfill}{rgb}{0.229806,0.298718,0.753683}%
\pgfsetfillcolor{currentfill}%
\pgfsetlinewidth{0.000000pt}%
\definecolor{currentstroke}{rgb}{0.000000,0.000000,0.000000}%
\pgfsetstrokecolor{currentstroke}%
\pgfsetdash{}{0pt}%
\pgfpathmoveto{\pgfqpoint{3.709654in}{1.332583in}}%
\pgfpathlineto{\pgfqpoint{3.757406in}{1.331454in}}%
\pgfpathlineto{\pgfqpoint{3.782494in}{1.349895in}}%
\pgfpathlineto{\pgfqpoint{3.735039in}{1.361132in}}%
\pgfpathlineto{\pgfqpoint{3.709654in}{1.332583in}}%
\pgfpathclose%
\pgfusepath{fill}%
\end{pgfscope}%
\begin{pgfscope}%
\pgfpathrectangle{\pgfqpoint{1.072000in}{0.528000in}}{\pgfqpoint{3.696000in}{3.696000in}}%
\pgfusepath{clip}%
\pgfsetbuttcap%
\pgfsetroundjoin%
\definecolor{currentfill}{rgb}{0.363461,0.484784,0.901019}%
\pgfsetfillcolor{currentfill}%
\pgfsetlinewidth{0.000000pt}%
\definecolor{currentstroke}{rgb}{0.000000,0.000000,0.000000}%
\pgfsetstrokecolor{currentstroke}%
\pgfsetdash{}{0pt}%
\pgfpathmoveto{\pgfqpoint{2.345447in}{1.485135in}}%
\pgfpathlineto{\pgfqpoint{2.390767in}{1.496324in}}%
\pgfpathlineto{\pgfqpoint{2.416879in}{1.635198in}}%
\pgfpathlineto{\pgfqpoint{2.371646in}{1.612445in}}%
\pgfpathlineto{\pgfqpoint{2.345447in}{1.485135in}}%
\pgfpathclose%
\pgfusepath{fill}%
\end{pgfscope}%
\begin{pgfscope}%
\pgfpathrectangle{\pgfqpoint{1.072000in}{0.528000in}}{\pgfqpoint{3.696000in}{3.696000in}}%
\pgfusepath{clip}%
\pgfsetbuttcap%
\pgfsetroundjoin%
\definecolor{currentfill}{rgb}{0.565182,0.699438,0.996635}%
\pgfsetfillcolor{currentfill}%
\pgfsetlinewidth{0.000000pt}%
\definecolor{currentstroke}{rgb}{0.000000,0.000000,0.000000}%
\pgfsetstrokecolor{currentstroke}%
\pgfsetdash{}{0pt}%
\pgfpathmoveto{\pgfqpoint{3.605725in}{1.910684in}}%
\pgfpathlineto{\pgfqpoint{3.650018in}{1.740153in}}%
\pgfpathlineto{\pgfqpoint{3.675730in}{1.779293in}}%
\pgfpathlineto{\pgfqpoint{3.631662in}{1.953060in}}%
\pgfpathlineto{\pgfqpoint{3.605725in}{1.910684in}}%
\pgfpathclose%
\pgfusepath{fill}%
\end{pgfscope}%
\begin{pgfscope}%
\pgfpathrectangle{\pgfqpoint{1.072000in}{0.528000in}}{\pgfqpoint{3.696000in}{3.696000in}}%
\pgfusepath{clip}%
\pgfsetbuttcap%
\pgfsetroundjoin%
\definecolor{currentfill}{rgb}{0.768929,0.189213,0.197965}%
\pgfsetfillcolor{currentfill}%
\pgfsetlinewidth{0.000000pt}%
\definecolor{currentstroke}{rgb}{0.000000,0.000000,0.000000}%
\pgfsetstrokecolor{currentstroke}%
\pgfsetdash{}{0pt}%
\pgfpathmoveto{\pgfqpoint{3.067492in}{3.172438in}}%
\pgfpathlineto{\pgfqpoint{3.114103in}{3.147589in}}%
\pgfpathlineto{\pgfqpoint{3.141017in}{3.250775in}}%
\pgfpathlineto{\pgfqpoint{3.094347in}{3.268935in}}%
\pgfpathlineto{\pgfqpoint{3.067492in}{3.172438in}}%
\pgfpathclose%
\pgfusepath{fill}%
\end{pgfscope}%
\begin{pgfscope}%
\pgfpathrectangle{\pgfqpoint{1.072000in}{0.528000in}}{\pgfqpoint{3.696000in}{3.696000in}}%
\pgfusepath{clip}%
\pgfsetbuttcap%
\pgfsetroundjoin%
\definecolor{currentfill}{rgb}{0.785153,0.220851,0.211673}%
\pgfsetfillcolor{currentfill}%
\pgfsetlinewidth{0.000000pt}%
\definecolor{currentstroke}{rgb}{0.000000,0.000000,0.000000}%
\pgfsetstrokecolor{currentstroke}%
\pgfsetdash{}{0pt}%
\pgfpathmoveto{\pgfqpoint{2.874190in}{3.198078in}}%
\pgfpathlineto{\pgfqpoint{2.920814in}{3.098438in}}%
\pgfpathlineto{\pgfqpoint{2.947663in}{3.187135in}}%
\pgfpathlineto{\pgfqpoint{2.901137in}{3.257456in}}%
\pgfpathlineto{\pgfqpoint{2.874190in}{3.198078in}}%
\pgfpathclose%
\pgfusepath{fill}%
\end{pgfscope}%
\begin{pgfscope}%
\pgfpathrectangle{\pgfqpoint{1.072000in}{0.528000in}}{\pgfqpoint{3.696000in}{3.696000in}}%
\pgfusepath{clip}%
\pgfsetbuttcap%
\pgfsetroundjoin%
\definecolor{currentfill}{rgb}{0.728970,0.817464,0.973188}%
\pgfsetfillcolor{currentfill}%
\pgfsetlinewidth{0.000000pt}%
\definecolor{currentstroke}{rgb}{0.000000,0.000000,0.000000}%
\pgfsetstrokecolor{currentstroke}%
\pgfsetdash{}{0pt}%
\pgfpathmoveto{\pgfqpoint{3.613394in}{2.175333in}}%
\pgfpathlineto{\pgfqpoint{3.657280in}{1.983338in}}%
\pgfpathlineto{\pgfqpoint{3.682551in}{2.001337in}}%
\pgfpathlineto{\pgfqpoint{3.638846in}{2.192096in}}%
\pgfpathlineto{\pgfqpoint{3.613394in}{2.175333in}}%
\pgfpathclose%
\pgfusepath{fill}%
\end{pgfscope}%
\begin{pgfscope}%
\pgfpathrectangle{\pgfqpoint{1.072000in}{0.528000in}}{\pgfqpoint{3.696000in}{3.696000in}}%
\pgfusepath{clip}%
\pgfsetbuttcap%
\pgfsetroundjoin%
\definecolor{currentfill}{rgb}{0.640828,0.760752,0.997846}%
\pgfsetfillcolor{currentfill}%
\pgfsetlinewidth{0.000000pt}%
\definecolor{currentstroke}{rgb}{0.000000,0.000000,0.000000}%
\pgfsetstrokecolor{currentstroke}%
\pgfsetdash{}{0pt}%
\pgfpathmoveto{\pgfqpoint{2.333639in}{1.837096in}}%
\pgfpathlineto{\pgfqpoint{2.378097in}{1.894644in}}%
\pgfpathlineto{\pgfqpoint{2.403524in}{2.074986in}}%
\pgfpathlineto{\pgfqpoint{2.359082in}{2.004838in}}%
\pgfpathlineto{\pgfqpoint{2.333639in}{1.837096in}}%
\pgfpathclose%
\pgfusepath{fill}%
\end{pgfscope}%
\begin{pgfscope}%
\pgfpathrectangle{\pgfqpoint{1.072000in}{0.528000in}}{\pgfqpoint{3.696000in}{3.696000in}}%
\pgfusepath{clip}%
\pgfsetbuttcap%
\pgfsetroundjoin%
\definecolor{currentfill}{rgb}{0.953054,0.585211,0.465373}%
\pgfsetfillcolor{currentfill}%
\pgfsetlinewidth{0.000000pt}%
\definecolor{currentstroke}{rgb}{0.000000,0.000000,0.000000}%
\pgfsetstrokecolor{currentstroke}%
\pgfsetdash{}{0pt}%
\pgfpathmoveto{\pgfqpoint{3.079668in}{2.696393in}}%
\pgfpathlineto{\pgfqpoint{3.126840in}{2.821017in}}%
\pgfpathlineto{\pgfqpoint{3.153851in}{2.933157in}}%
\pgfpathlineto{\pgfqpoint{3.106704in}{2.856407in}}%
\pgfpathlineto{\pgfqpoint{3.079668in}{2.696393in}}%
\pgfpathclose%
\pgfusepath{fill}%
\end{pgfscope}%
\begin{pgfscope}%
\pgfpathrectangle{\pgfqpoint{1.072000in}{0.528000in}}{\pgfqpoint{3.696000in}{3.696000in}}%
\pgfusepath{clip}%
\pgfsetbuttcap%
\pgfsetroundjoin%
\definecolor{currentfill}{rgb}{0.705673,0.015556,0.150233}%
\pgfsetfillcolor{currentfill}%
\pgfsetlinewidth{0.000000pt}%
\definecolor{currentstroke}{rgb}{0.000000,0.000000,0.000000}%
\pgfsetstrokecolor{currentstroke}%
\pgfsetdash{}{0pt}%
\pgfpathmoveto{\pgfqpoint{2.781103in}{3.299493in}}%
\pgfpathlineto{\pgfqpoint{2.827565in}{3.268015in}}%
\pgfpathlineto{\pgfqpoint{2.854645in}{3.303768in}}%
\pgfpathlineto{\pgfqpoint{2.808325in}{3.316722in}}%
\pgfpathlineto{\pgfqpoint{2.781103in}{3.299493in}}%
\pgfpathclose%
\pgfusepath{fill}%
\end{pgfscope}%
\begin{pgfscope}%
\pgfpathrectangle{\pgfqpoint{1.072000in}{0.528000in}}{\pgfqpoint{3.696000in}{3.696000in}}%
\pgfusepath{clip}%
\pgfsetbuttcap%
\pgfsetroundjoin%
\definecolor{currentfill}{rgb}{0.705673,0.015556,0.150233}%
\pgfsetfillcolor{currentfill}%
\pgfsetlinewidth{0.000000pt}%
\definecolor{currentstroke}{rgb}{0.000000,0.000000,0.000000}%
\pgfsetstrokecolor{currentstroke}%
\pgfsetdash{}{0pt}%
\pgfpathmoveto{\pgfqpoint{3.214486in}{3.297688in}}%
\pgfpathlineto{\pgfqpoint{3.261387in}{3.286164in}}%
\pgfpathlineto{\pgfqpoint{3.287446in}{3.284268in}}%
\pgfpathlineto{\pgfqpoint{3.240713in}{3.305442in}}%
\pgfpathlineto{\pgfqpoint{3.214486in}{3.297688in}}%
\pgfpathclose%
\pgfusepath{fill}%
\end{pgfscope}%
\begin{pgfscope}%
\pgfpathrectangle{\pgfqpoint{1.072000in}{0.528000in}}{\pgfqpoint{3.696000in}{3.696000in}}%
\pgfusepath{clip}%
\pgfsetbuttcap%
\pgfsetroundjoin%
\definecolor{currentfill}{rgb}{0.229806,0.298718,0.753683}%
\pgfsetfillcolor{currentfill}%
\pgfsetlinewidth{0.000000pt}%
\definecolor{currentstroke}{rgb}{0.000000,0.000000,0.000000}%
\pgfsetstrokecolor{currentstroke}%
\pgfsetdash{}{0pt}%
\pgfpathmoveto{\pgfqpoint{2.216474in}{1.354087in}}%
\pgfpathlineto{\pgfqpoint{2.262459in}{1.327791in}}%
\pgfpathlineto{\pgfqpoint{2.291013in}{1.339684in}}%
\pgfpathlineto{\pgfqpoint{2.245371in}{1.354065in}}%
\pgfpathlineto{\pgfqpoint{2.216474in}{1.354087in}}%
\pgfpathclose%
\pgfusepath{fill}%
\end{pgfscope}%
\begin{pgfscope}%
\pgfpathrectangle{\pgfqpoint{1.072000in}{0.528000in}}{\pgfqpoint{3.696000in}{3.696000in}}%
\pgfusepath{clip}%
\pgfsetbuttcap%
\pgfsetroundjoin%
\definecolor{currentfill}{rgb}{0.822420,0.856898,0.910795}%
\pgfsetfillcolor{currentfill}%
\pgfsetlinewidth{0.000000pt}%
\definecolor{currentstroke}{rgb}{0.000000,0.000000,0.000000}%
\pgfsetstrokecolor{currentstroke}%
\pgfsetdash{}{0pt}%
\pgfpathmoveto{\pgfqpoint{1.660420in}{2.357328in}}%
\pgfpathlineto{\pgfqpoint{1.710651in}{2.207651in}}%
\pgfpathlineto{\pgfqpoint{1.744393in}{2.109478in}}%
\pgfpathlineto{\pgfqpoint{1.694471in}{2.259769in}}%
\pgfpathlineto{\pgfqpoint{1.660420in}{2.357328in}}%
\pgfpathclose%
\pgfusepath{fill}%
\end{pgfscope}%
\begin{pgfscope}%
\pgfpathrectangle{\pgfqpoint{1.072000in}{0.528000in}}{\pgfqpoint{3.696000in}{3.696000in}}%
\pgfusepath{clip}%
\pgfsetbuttcap%
\pgfsetroundjoin%
\definecolor{currentfill}{rgb}{0.852378,0.346492,0.280346}%
\pgfsetfillcolor{currentfill}%
\pgfsetlinewidth{0.000000pt}%
\definecolor{currentstroke}{rgb}{0.000000,0.000000,0.000000}%
\pgfsetstrokecolor{currentstroke}%
\pgfsetdash{}{0pt}%
\pgfpathmoveto{\pgfqpoint{3.087074in}{3.005048in}}%
\pgfpathlineto{\pgfqpoint{3.133832in}{3.014233in}}%
\pgfpathlineto{\pgfqpoint{3.160923in}{3.147274in}}%
\pgfpathlineto{\pgfqpoint{3.114103in}{3.147589in}}%
\pgfpathlineto{\pgfqpoint{3.087074in}{3.005048in}}%
\pgfpathclose%
\pgfusepath{fill}%
\end{pgfscope}%
\begin{pgfscope}%
\pgfpathrectangle{\pgfqpoint{1.072000in}{0.528000in}}{\pgfqpoint{3.696000in}{3.696000in}}%
\pgfusepath{clip}%
\pgfsetbuttcap%
\pgfsetroundjoin%
\definecolor{currentfill}{rgb}{0.928116,0.822197,0.765141}%
\pgfsetfillcolor{currentfill}%
\pgfsetlinewidth{0.000000pt}%
\definecolor{currentstroke}{rgb}{0.000000,0.000000,0.000000}%
\pgfsetstrokecolor{currentstroke}%
\pgfsetdash{}{0pt}%
\pgfpathmoveto{\pgfqpoint{2.348037in}{2.312794in}}%
\pgfpathlineto{\pgfqpoint{2.391637in}{2.425888in}}%
\pgfpathlineto{\pgfqpoint{2.417563in}{2.581064in}}%
\pgfpathlineto{\pgfqpoint{2.373838in}{2.462531in}}%
\pgfpathlineto{\pgfqpoint{2.348037in}{2.312794in}}%
\pgfpathclose%
\pgfusepath{fill}%
\end{pgfscope}%
\begin{pgfscope}%
\pgfpathrectangle{\pgfqpoint{1.072000in}{0.528000in}}{\pgfqpoint{3.696000in}{3.696000in}}%
\pgfusepath{clip}%
\pgfsetbuttcap%
\pgfsetroundjoin%
\definecolor{currentfill}{rgb}{0.729196,0.086679,0.167240}%
\pgfsetfillcolor{currentfill}%
\pgfsetlinewidth{0.000000pt}%
\definecolor{currentstroke}{rgb}{0.000000,0.000000,0.000000}%
\pgfsetstrokecolor{currentstroke}%
\pgfsetdash{}{0pt}%
\pgfpathmoveto{\pgfqpoint{2.827565in}{3.268015in}}%
\pgfpathlineto{\pgfqpoint{2.874190in}{3.198078in}}%
\pgfpathlineto{\pgfqpoint{2.901137in}{3.257456in}}%
\pgfpathlineto{\pgfqpoint{2.854645in}{3.303768in}}%
\pgfpathlineto{\pgfqpoint{2.827565in}{3.268015in}}%
\pgfpathclose%
\pgfusepath{fill}%
\end{pgfscope}%
\begin{pgfscope}%
\pgfpathrectangle{\pgfqpoint{1.072000in}{0.528000in}}{\pgfqpoint{3.696000in}{3.696000in}}%
\pgfusepath{clip}%
\pgfsetbuttcap%
\pgfsetroundjoin%
\definecolor{currentfill}{rgb}{0.869655,0.379274,0.300941}%
\pgfsetfillcolor{currentfill}%
\pgfsetlinewidth{0.000000pt}%
\definecolor{currentstroke}{rgb}{0.000000,0.000000,0.000000}%
\pgfsetstrokecolor{currentstroke}%
\pgfsetdash{}{0pt}%
\pgfpathmoveto{\pgfqpoint{3.448544in}{3.123737in}}%
\pgfpathlineto{\pgfqpoint{3.494599in}{3.027898in}}%
\pgfpathlineto{\pgfqpoint{3.518946in}{2.956615in}}%
\pgfpathlineto{\pgfqpoint{3.473288in}{3.062640in}}%
\pgfpathlineto{\pgfqpoint{3.448544in}{3.123737in}}%
\pgfpathclose%
\pgfusepath{fill}%
\end{pgfscope}%
\begin{pgfscope}%
\pgfpathrectangle{\pgfqpoint{1.072000in}{0.528000in}}{\pgfqpoint{3.696000in}{3.696000in}}%
\pgfusepath{clip}%
\pgfsetbuttcap%
\pgfsetroundjoin%
\definecolor{currentfill}{rgb}{0.830187,0.304733,0.254891}%
\pgfsetfillcolor{currentfill}%
\pgfsetlinewidth{0.000000pt}%
\definecolor{currentstroke}{rgb}{0.000000,0.000000,0.000000}%
\pgfsetstrokecolor{currentstroke}%
\pgfsetdash{}{0pt}%
\pgfpathmoveto{\pgfqpoint{2.847170in}{3.144800in}}%
\pgfpathlineto{\pgfqpoint{2.893934in}{3.021063in}}%
\pgfpathlineto{\pgfqpoint{2.920814in}{3.098438in}}%
\pgfpathlineto{\pgfqpoint{2.874190in}{3.198078in}}%
\pgfpathlineto{\pgfqpoint{2.847170in}{3.144800in}}%
\pgfpathclose%
\pgfusepath{fill}%
\end{pgfscope}%
\begin{pgfscope}%
\pgfpathrectangle{\pgfqpoint{1.072000in}{0.528000in}}{\pgfqpoint{3.696000in}{3.696000in}}%
\pgfusepath{clip}%
\pgfsetbuttcap%
\pgfsetroundjoin%
\definecolor{currentfill}{rgb}{0.532568,0.669801,0.990393}%
\pgfsetfillcolor{currentfill}%
\pgfsetlinewidth{0.000000pt}%
\definecolor{currentstroke}{rgb}{0.000000,0.000000,0.000000}%
\pgfsetstrokecolor{currentstroke}%
\pgfsetdash{}{0pt}%
\pgfpathmoveto{\pgfqpoint{1.840187in}{1.897768in}}%
\pgfpathlineto{\pgfqpoint{1.888665in}{1.788637in}}%
\pgfpathlineto{\pgfqpoint{1.921430in}{1.693462in}}%
\pgfpathlineto{\pgfqpoint{1.873539in}{1.793627in}}%
\pgfpathlineto{\pgfqpoint{1.840187in}{1.897768in}}%
\pgfpathclose%
\pgfusepath{fill}%
\end{pgfscope}%
\begin{pgfscope}%
\pgfpathrectangle{\pgfqpoint{1.072000in}{0.528000in}}{\pgfqpoint{3.696000in}{3.696000in}}%
\pgfusepath{clip}%
\pgfsetbuttcap%
\pgfsetroundjoin%
\definecolor{currentfill}{rgb}{0.717435,0.051118,0.158737}%
\pgfsetfillcolor{currentfill}%
\pgfsetlinewidth{0.000000pt}%
\definecolor{currentstroke}{rgb}{0.000000,0.000000,0.000000}%
\pgfsetstrokecolor{currentstroke}%
\pgfsetdash{}{0pt}%
\pgfpathmoveto{\pgfqpoint{3.141017in}{3.250775in}}%
\pgfpathlineto{\pgfqpoint{3.187850in}{3.243994in}}%
\pgfpathlineto{\pgfqpoint{3.214486in}{3.297688in}}%
\pgfpathlineto{\pgfqpoint{3.167696in}{3.309147in}}%
\pgfpathlineto{\pgfqpoint{3.141017in}{3.250775in}}%
\pgfpathclose%
\pgfusepath{fill}%
\end{pgfscope}%
\begin{pgfscope}%
\pgfpathrectangle{\pgfqpoint{1.072000in}{0.528000in}}{\pgfqpoint{3.696000in}{3.696000in}}%
\pgfusepath{clip}%
\pgfsetbuttcap%
\pgfsetroundjoin%
\definecolor{currentfill}{rgb}{0.229806,0.298718,0.753683}%
\pgfsetfillcolor{currentfill}%
\pgfsetlinewidth{0.000000pt}%
\definecolor{currentstroke}{rgb}{0.000000,0.000000,0.000000}%
\pgfsetstrokecolor{currentstroke}%
\pgfsetdash{}{0pt}%
\pgfpathmoveto{\pgfqpoint{3.637137in}{1.336282in}}%
\pgfpathlineto{\pgfqpoint{3.684231in}{1.306780in}}%
\pgfpathlineto{\pgfqpoint{3.709654in}{1.332583in}}%
\pgfpathlineto{\pgfqpoint{3.662905in}{1.375054in}}%
\pgfpathlineto{\pgfqpoint{3.637137in}{1.336282in}}%
\pgfpathclose%
\pgfusepath{fill}%
\end{pgfscope}%
\begin{pgfscope}%
\pgfpathrectangle{\pgfqpoint{1.072000in}{0.528000in}}{\pgfqpoint{3.696000in}{3.696000in}}%
\pgfusepath{clip}%
\pgfsetbuttcap%
\pgfsetroundjoin%
\definecolor{currentfill}{rgb}{0.266381,0.353304,0.801637}%
\pgfsetfillcolor{currentfill}%
\pgfsetlinewidth{0.000000pt}%
\definecolor{currentstroke}{rgb}{0.000000,0.000000,0.000000}%
\pgfsetstrokecolor{currentstroke}%
\pgfsetdash{}{0pt}%
\pgfpathmoveto{\pgfqpoint{3.590827in}{1.404473in}}%
\pgfpathlineto{\pgfqpoint{3.637137in}{1.336282in}}%
\pgfpathlineto{\pgfqpoint{3.662905in}{1.375054in}}%
\pgfpathlineto{\pgfqpoint{3.616944in}{1.456837in}}%
\pgfpathlineto{\pgfqpoint{3.590827in}{1.404473in}}%
\pgfpathclose%
\pgfusepath{fill}%
\end{pgfscope}%
\begin{pgfscope}%
\pgfpathrectangle{\pgfqpoint{1.072000in}{0.528000in}}{\pgfqpoint{3.696000in}{3.696000in}}%
\pgfusepath{clip}%
\pgfsetbuttcap%
\pgfsetroundjoin%
\definecolor{currentfill}{rgb}{0.713852,0.808857,0.979386}%
\pgfsetfillcolor{currentfill}%
\pgfsetlinewidth{0.000000pt}%
\definecolor{currentstroke}{rgb}{0.000000,0.000000,0.000000}%
\pgfsetstrokecolor{currentstroke}%
\pgfsetdash{}{0pt}%
\pgfpathmoveto{\pgfqpoint{1.726382in}{2.170087in}}%
\pgfpathlineto{\pgfqpoint{1.775995in}{2.033905in}}%
\pgfpathlineto{\pgfqpoint{1.809567in}{1.931730in}}%
\pgfpathlineto{\pgfqpoint{1.760403in}{2.064470in}}%
\pgfpathlineto{\pgfqpoint{1.726382in}{2.170087in}}%
\pgfpathclose%
\pgfusepath{fill}%
\end{pgfscope}%
\begin{pgfscope}%
\pgfpathrectangle{\pgfqpoint{1.072000in}{0.528000in}}{\pgfqpoint{3.696000in}{3.696000in}}%
\pgfusepath{clip}%
\pgfsetbuttcap%
\pgfsetroundjoin%
\definecolor{currentfill}{rgb}{0.740957,0.122240,0.175744}%
\pgfsetfillcolor{currentfill}%
\pgfsetlinewidth{0.000000pt}%
\definecolor{currentstroke}{rgb}{0.000000,0.000000,0.000000}%
\pgfsetstrokecolor{currentstroke}%
\pgfsetdash{}{0pt}%
\pgfpathmoveto{\pgfqpoint{2.634326in}{3.212417in}}%
\pgfpathlineto{\pgfqpoint{2.680065in}{3.256646in}}%
\pgfpathlineto{\pgfqpoint{2.707559in}{3.275731in}}%
\pgfpathlineto{\pgfqpoint{2.661861in}{3.229070in}}%
\pgfpathlineto{\pgfqpoint{2.634326in}{3.212417in}}%
\pgfpathclose%
\pgfusepath{fill}%
\end{pgfscope}%
\begin{pgfscope}%
\pgfpathrectangle{\pgfqpoint{1.072000in}{0.528000in}}{\pgfqpoint{3.696000in}{3.696000in}}%
\pgfusepath{clip}%
\pgfsetbuttcap%
\pgfsetroundjoin%
\definecolor{currentfill}{rgb}{0.299441,0.400248,0.839842}%
\pgfsetfillcolor{currentfill}%
\pgfsetlinewidth{0.000000pt}%
\definecolor{currentstroke}{rgb}{0.000000,0.000000,0.000000}%
\pgfsetstrokecolor{currentstroke}%
\pgfsetdash{}{0pt}%
\pgfpathmoveto{\pgfqpoint{2.364103in}{1.392410in}}%
\pgfpathlineto{\pgfqpoint{2.409740in}{1.390925in}}%
\pgfpathlineto{\pgfqpoint{2.436318in}{1.504072in}}%
\pgfpathlineto{\pgfqpoint{2.390767in}{1.496324in}}%
\pgfpathlineto{\pgfqpoint{2.364103in}{1.392410in}}%
\pgfpathclose%
\pgfusepath{fill}%
\end{pgfscope}%
\begin{pgfscope}%
\pgfpathrectangle{\pgfqpoint{1.072000in}{0.528000in}}{\pgfqpoint{3.696000in}{3.696000in}}%
\pgfusepath{clip}%
\pgfsetbuttcap%
\pgfsetroundjoin%
\definecolor{currentfill}{rgb}{0.363461,0.484784,0.901019}%
\pgfsetfillcolor{currentfill}%
\pgfsetlinewidth{0.000000pt}%
\definecolor{currentstroke}{rgb}{0.000000,0.000000,0.000000}%
\pgfsetstrokecolor{currentstroke}%
\pgfsetdash{}{0pt}%
\pgfpathmoveto{\pgfqpoint{3.571531in}{1.573835in}}%
\pgfpathlineto{\pgfqpoint{3.616944in}{1.456837in}}%
\pgfpathlineto{\pgfqpoint{3.642970in}{1.507701in}}%
\pgfpathlineto{\pgfqpoint{3.597873in}{1.635954in}}%
\pgfpathlineto{\pgfqpoint{3.571531in}{1.573835in}}%
\pgfpathclose%
\pgfusepath{fill}%
\end{pgfscope}%
\begin{pgfscope}%
\pgfpathrectangle{\pgfqpoint{1.072000in}{0.528000in}}{\pgfqpoint{3.696000in}{3.696000in}}%
\pgfusepath{clip}%
\pgfsetbuttcap%
\pgfsetroundjoin%
\definecolor{currentfill}{rgb}{0.565182,0.699438,0.996635}%
\pgfsetfillcolor{currentfill}%
\pgfsetlinewidth{0.000000pt}%
\definecolor{currentstroke}{rgb}{0.000000,0.000000,0.000000}%
\pgfsetstrokecolor{currentstroke}%
\pgfsetdash{}{0pt}%
\pgfpathmoveto{\pgfqpoint{2.352549in}{1.728999in}}%
\pgfpathlineto{\pgfqpoint{2.397415in}{1.768750in}}%
\pgfpathlineto{\pgfqpoint{2.422946in}{1.946475in}}%
\pgfpathlineto{\pgfqpoint{2.378097in}{1.894644in}}%
\pgfpathlineto{\pgfqpoint{2.352549in}{1.728999in}}%
\pgfpathclose%
\pgfusepath{fill}%
\end{pgfscope}%
\begin{pgfscope}%
\pgfpathrectangle{\pgfqpoint{1.072000in}{0.528000in}}{\pgfqpoint{3.696000in}{3.696000in}}%
\pgfusepath{clip}%
\pgfsetbuttcap%
\pgfsetroundjoin%
\definecolor{currentfill}{rgb}{0.705673,0.015556,0.150233}%
\pgfsetfillcolor{currentfill}%
\pgfsetlinewidth{0.000000pt}%
\definecolor{currentstroke}{rgb}{0.000000,0.000000,0.000000}%
\pgfsetstrokecolor{currentstroke}%
\pgfsetdash{}{0pt}%
\pgfpathmoveto{\pgfqpoint{2.707559in}{3.275731in}}%
\pgfpathlineto{\pgfqpoint{2.753796in}{3.277022in}}%
\pgfpathlineto{\pgfqpoint{2.781103in}{3.299493in}}%
\pgfpathlineto{\pgfqpoint{2.734980in}{3.288486in}}%
\pgfpathlineto{\pgfqpoint{2.707559in}{3.275731in}}%
\pgfpathclose%
\pgfusepath{fill}%
\end{pgfscope}%
\begin{pgfscope}%
\pgfpathrectangle{\pgfqpoint{1.072000in}{0.528000in}}{\pgfqpoint{3.696000in}{3.696000in}}%
\pgfusepath{clip}%
\pgfsetbuttcap%
\pgfsetroundjoin%
\definecolor{currentfill}{rgb}{0.818056,0.855590,0.914638}%
\pgfsetfillcolor{currentfill}%
\pgfsetlinewidth{0.000000pt}%
\definecolor{currentstroke}{rgb}{0.000000,0.000000,0.000000}%
\pgfsetstrokecolor{currentstroke}%
\pgfsetdash{}{0pt}%
\pgfpathmoveto{\pgfqpoint{2.340496in}{2.093593in}}%
\pgfpathlineto{\pgfqpoint{2.384500in}{2.181326in}}%
\pgfpathlineto{\pgfqpoint{2.410045in}{2.358545in}}%
\pgfpathlineto{\pgfqpoint{2.365979in}{2.261309in}}%
\pgfpathlineto{\pgfqpoint{2.340496in}{2.093593in}}%
\pgfpathclose%
\pgfusepath{fill}%
\end{pgfscope}%
\begin{pgfscope}%
\pgfpathrectangle{\pgfqpoint{1.072000in}{0.528000in}}{\pgfqpoint{3.696000in}{3.696000in}}%
\pgfusepath{clip}%
\pgfsetbuttcap%
\pgfsetroundjoin%
\definecolor{currentfill}{rgb}{0.243520,0.319189,0.771672}%
\pgfsetfillcolor{currentfill}%
\pgfsetlinewidth{0.000000pt}%
\definecolor{currentstroke}{rgb}{0.000000,0.000000,0.000000}%
\pgfsetstrokecolor{currentstroke}%
\pgfsetdash{}{0pt}%
\pgfpathmoveto{\pgfqpoint{2.336707in}{1.328126in}}%
\pgfpathlineto{\pgfqpoint{2.382496in}{1.317715in}}%
\pgfpathlineto{\pgfqpoint{2.409740in}{1.390925in}}%
\pgfpathlineto{\pgfqpoint{2.364103in}{1.392410in}}%
\pgfpathlineto{\pgfqpoint{2.336707in}{1.328126in}}%
\pgfpathclose%
\pgfusepath{fill}%
\end{pgfscope}%
\begin{pgfscope}%
\pgfpathrectangle{\pgfqpoint{1.072000in}{0.528000in}}{\pgfqpoint{3.696000in}{3.696000in}}%
\pgfusepath{clip}%
\pgfsetbuttcap%
\pgfsetroundjoin%
\definecolor{currentfill}{rgb}{0.711554,0.033337,0.154485}%
\pgfsetfillcolor{currentfill}%
\pgfsetlinewidth{0.000000pt}%
\definecolor{currentstroke}{rgb}{0.000000,0.000000,0.000000}%
\pgfsetstrokecolor{currentstroke}%
\pgfsetdash{}{0pt}%
\pgfpathmoveto{\pgfqpoint{3.261387in}{3.286164in}}%
\pgfpathlineto{\pgfqpoint{3.308352in}{3.269444in}}%
\pgfpathlineto{\pgfqpoint{3.334167in}{3.253552in}}%
\pgfpathlineto{\pgfqpoint{3.287446in}{3.284268in}}%
\pgfpathlineto{\pgfqpoint{3.261387in}{3.286164in}}%
\pgfpathclose%
\pgfusepath{fill}%
\end{pgfscope}%
\begin{pgfscope}%
\pgfpathrectangle{\pgfqpoint{1.072000in}{0.528000in}}{\pgfqpoint{3.696000in}{3.696000in}}%
\pgfusepath{clip}%
\pgfsetbuttcap%
\pgfsetroundjoin%
\definecolor{currentfill}{rgb}{0.820401,0.286765,0.245160}%
\pgfsetfillcolor{currentfill}%
\pgfsetlinewidth{0.000000pt}%
\definecolor{currentstroke}{rgb}{0.000000,0.000000,0.000000}%
\pgfsetstrokecolor{currentstroke}%
\pgfsetdash{}{0pt}%
\pgfpathmoveto{\pgfqpoint{2.534351in}{3.055175in}}%
\pgfpathlineto{\pgfqpoint{2.579237in}{3.148114in}}%
\pgfpathlineto{\pgfqpoint{2.606774in}{3.185689in}}%
\pgfpathlineto{\pgfqpoint{2.561791in}{3.097359in}}%
\pgfpathlineto{\pgfqpoint{2.534351in}{3.055175in}}%
\pgfpathclose%
\pgfusepath{fill}%
\end{pgfscope}%
\begin{pgfscope}%
\pgfpathrectangle{\pgfqpoint{1.072000in}{0.528000in}}{\pgfqpoint{3.696000in}{3.696000in}}%
\pgfusepath{clip}%
\pgfsetbuttcap%
\pgfsetroundjoin%
\definecolor{currentfill}{rgb}{0.768929,0.189213,0.197965}%
\pgfsetfillcolor{currentfill}%
\pgfsetlinewidth{0.000000pt}%
\definecolor{currentstroke}{rgb}{0.000000,0.000000,0.000000}%
\pgfsetstrokecolor{currentstroke}%
\pgfsetdash{}{0pt}%
\pgfpathmoveto{\pgfqpoint{3.114103in}{3.147589in}}%
\pgfpathlineto{\pgfqpoint{3.160923in}{3.147274in}}%
\pgfpathlineto{\pgfqpoint{3.187850in}{3.243994in}}%
\pgfpathlineto{\pgfqpoint{3.141017in}{3.250775in}}%
\pgfpathlineto{\pgfqpoint{3.114103in}{3.147589in}}%
\pgfpathclose%
\pgfusepath{fill}%
\end{pgfscope}%
\begin{pgfscope}%
\pgfpathrectangle{\pgfqpoint{1.072000in}{0.528000in}}{\pgfqpoint{3.696000in}{3.696000in}}%
\pgfusepath{clip}%
\pgfsetbuttcap%
\pgfsetroundjoin%
\definecolor{currentfill}{rgb}{0.229806,0.298718,0.753683}%
\pgfsetfillcolor{currentfill}%
\pgfsetlinewidth{0.000000pt}%
\definecolor{currentstroke}{rgb}{0.000000,0.000000,0.000000}%
\pgfsetstrokecolor{currentstroke}%
\pgfsetdash{}{0pt}%
\pgfpathmoveto{\pgfqpoint{2.262459in}{1.327791in}}%
\pgfpathlineto{\pgfqpoint{2.308430in}{1.306011in}}%
\pgfpathlineto{\pgfqpoint{2.336707in}{1.328126in}}%
\pgfpathlineto{\pgfqpoint{2.291013in}{1.339684in}}%
\pgfpathlineto{\pgfqpoint{2.262459in}{1.327791in}}%
\pgfpathclose%
\pgfusepath{fill}%
\end{pgfscope}%
\begin{pgfscope}%
\pgfpathrectangle{\pgfqpoint{1.072000in}{0.528000in}}{\pgfqpoint{3.696000in}{3.696000in}}%
\pgfusepath{clip}%
\pgfsetbuttcap%
\pgfsetroundjoin%
\definecolor{currentfill}{rgb}{0.939254,0.539581,0.423900}%
\pgfsetfillcolor{currentfill}%
\pgfsetlinewidth{0.000000pt}%
\definecolor{currentstroke}{rgb}{0.000000,0.000000,0.000000}%
\pgfsetstrokecolor{currentstroke}%
\pgfsetdash{}{0pt}%
\pgfpathmoveto{\pgfqpoint{3.515611in}{2.964967in}}%
\pgfpathlineto{\pgfqpoint{3.560905in}{2.825589in}}%
\pgfpathlineto{\pgfqpoint{3.585074in}{2.762686in}}%
\pgfpathlineto{\pgfqpoint{3.540126in}{2.906597in}}%
\pgfpathlineto{\pgfqpoint{3.515611in}{2.964967in}}%
\pgfpathclose%
\pgfusepath{fill}%
\end{pgfscope}%
\begin{pgfscope}%
\pgfpathrectangle{\pgfqpoint{1.072000in}{0.528000in}}{\pgfqpoint{3.696000in}{3.696000in}}%
\pgfusepath{clip}%
\pgfsetbuttcap%
\pgfsetroundjoin%
\definecolor{currentfill}{rgb}{0.929357,0.512254,0.400673}%
\pgfsetfillcolor{currentfill}%
\pgfsetlinewidth{0.000000pt}%
\definecolor{currentstroke}{rgb}{0.000000,0.000000,0.000000}%
\pgfsetstrokecolor{currentstroke}%
\pgfsetdash{}{0pt}%
\pgfpathmoveto{\pgfqpoint{2.839692in}{2.976764in}}%
\pgfpathlineto{\pgfqpoint{2.886626in}{2.827977in}}%
\pgfpathlineto{\pgfqpoint{2.913695in}{2.812279in}}%
\pgfpathlineto{\pgfqpoint{2.866932in}{2.976997in}}%
\pgfpathlineto{\pgfqpoint{2.839692in}{2.976764in}}%
\pgfpathclose%
\pgfusepath{fill}%
\end{pgfscope}%
\begin{pgfscope}%
\pgfpathrectangle{\pgfqpoint{1.072000in}{0.528000in}}{\pgfqpoint{3.696000in}{3.696000in}}%
\pgfusepath{clip}%
\pgfsetbuttcap%
\pgfsetroundjoin%
\definecolor{currentfill}{rgb}{0.543440,0.680003,0.993051}%
\pgfsetfillcolor{currentfill}%
\pgfsetlinewidth{0.000000pt}%
\definecolor{currentstroke}{rgb}{0.000000,0.000000,0.000000}%
\pgfsetstrokecolor{currentstroke}%
\pgfsetdash{}{0pt}%
\pgfpathmoveto{\pgfqpoint{3.579509in}{1.856869in}}%
\pgfpathlineto{\pgfqpoint{3.624051in}{1.691968in}}%
\pgfpathlineto{\pgfqpoint{3.650018in}{1.740153in}}%
\pgfpathlineto{\pgfqpoint{3.605725in}{1.910684in}}%
\pgfpathlineto{\pgfqpoint{3.579509in}{1.856869in}}%
\pgfpathclose%
\pgfusepath{fill}%
\end{pgfscope}%
\begin{pgfscope}%
\pgfpathrectangle{\pgfqpoint{1.072000in}{0.528000in}}{\pgfqpoint{3.696000in}{3.696000in}}%
\pgfusepath{clip}%
\pgfsetbuttcap%
\pgfsetroundjoin%
\definecolor{currentfill}{rgb}{0.902659,0.447939,0.349721}%
\pgfsetfillcolor{currentfill}%
\pgfsetlinewidth{0.000000pt}%
\definecolor{currentstroke}{rgb}{0.000000,0.000000,0.000000}%
\pgfsetstrokecolor{currentstroke}%
\pgfsetdash{}{0pt}%
\pgfpathmoveto{\pgfqpoint{3.106704in}{2.856407in}}%
\pgfpathlineto{\pgfqpoint{3.153851in}{2.933157in}}%
\pgfpathlineto{\pgfqpoint{3.180959in}{3.056971in}}%
\pgfpathlineto{\pgfqpoint{3.133832in}{3.014233in}}%
\pgfpathlineto{\pgfqpoint{3.106704in}{2.856407in}}%
\pgfpathclose%
\pgfusepath{fill}%
\end{pgfscope}%
\begin{pgfscope}%
\pgfpathrectangle{\pgfqpoint{1.072000in}{0.528000in}}{\pgfqpoint{3.696000in}{3.696000in}}%
\pgfusepath{clip}%
\pgfsetbuttcap%
\pgfsetroundjoin%
\definecolor{currentfill}{rgb}{0.968500,0.673977,0.556649}%
\pgfsetfillcolor{currentfill}%
\pgfsetlinewidth{0.000000pt}%
\definecolor{currentstroke}{rgb}{0.000000,0.000000,0.000000}%
\pgfsetstrokecolor{currentstroke}%
\pgfsetdash{}{0pt}%
\pgfpathmoveto{\pgfqpoint{2.906110in}{2.792406in}}%
\pgfpathlineto{\pgfqpoint{2.952840in}{2.741032in}}%
\pgfpathlineto{\pgfqpoint{2.979753in}{2.590528in}}%
\pgfpathlineto{\pgfqpoint{2.933310in}{2.677220in}}%
\pgfpathlineto{\pgfqpoint{2.906110in}{2.792406in}}%
\pgfpathclose%
\pgfusepath{fill}%
\end{pgfscope}%
\begin{pgfscope}%
\pgfpathrectangle{\pgfqpoint{1.072000in}{0.528000in}}{\pgfqpoint{3.696000in}{3.696000in}}%
\pgfusepath{clip}%
\pgfsetbuttcap%
\pgfsetroundjoin%
\definecolor{currentfill}{rgb}{0.473070,0.611077,0.970634}%
\pgfsetfillcolor{currentfill}%
\pgfsetlinewidth{0.000000pt}%
\definecolor{currentstroke}{rgb}{0.000000,0.000000,0.000000}%
\pgfsetstrokecolor{currentstroke}%
\pgfsetdash{}{0pt}%
\pgfpathmoveto{\pgfqpoint{2.371646in}{1.612445in}}%
\pgfpathlineto{\pgfqpoint{2.416879in}{1.635198in}}%
\pgfpathlineto{\pgfqpoint{2.442627in}{1.802433in}}%
\pgfpathlineto{\pgfqpoint{2.397415in}{1.768750in}}%
\pgfpathlineto{\pgfqpoint{2.371646in}{1.612445in}}%
\pgfpathclose%
\pgfusepath{fill}%
\end{pgfscope}%
\begin{pgfscope}%
\pgfpathrectangle{\pgfqpoint{1.072000in}{0.528000in}}{\pgfqpoint{3.696000in}{3.696000in}}%
\pgfusepath{clip}%
\pgfsetbuttcap%
\pgfsetroundjoin%
\definecolor{currentfill}{rgb}{0.810616,0.268797,0.235428}%
\pgfsetfillcolor{currentfill}%
\pgfsetlinewidth{0.000000pt}%
\definecolor{currentstroke}{rgb}{0.000000,0.000000,0.000000}%
\pgfsetstrokecolor{currentstroke}%
\pgfsetdash{}{0pt}%
\pgfpathmoveto{\pgfqpoint{3.402063in}{3.193893in}}%
\pgfpathlineto{\pgfqpoint{3.448544in}{3.123737in}}%
\pgfpathlineto{\pgfqpoint{3.473288in}{3.062640in}}%
\pgfpathlineto{\pgfqpoint{3.427201in}{3.146359in}}%
\pgfpathlineto{\pgfqpoint{3.402063in}{3.193893in}}%
\pgfpathclose%
\pgfusepath{fill}%
\end{pgfscope}%
\begin{pgfscope}%
\pgfpathrectangle{\pgfqpoint{1.072000in}{0.528000in}}{\pgfqpoint{3.696000in}{3.696000in}}%
\pgfusepath{clip}%
\pgfsetbuttcap%
\pgfsetroundjoin%
\definecolor{currentfill}{rgb}{0.968500,0.673977,0.556649}%
\pgfsetfillcolor{currentfill}%
\pgfsetlinewidth{0.000000pt}%
\definecolor{currentstroke}{rgb}{0.000000,0.000000,0.000000}%
\pgfsetstrokecolor{currentstroke}%
\pgfsetdash{}{0pt}%
\pgfpathmoveto{\pgfqpoint{3.026347in}{2.650923in}}%
\pgfpathlineto{\pgfqpoint{3.073437in}{2.775820in}}%
\pgfpathlineto{\pgfqpoint{3.100050in}{2.760212in}}%
\pgfpathlineto{\pgfqpoint{3.052886in}{2.600187in}}%
\pgfpathlineto{\pgfqpoint{3.026347in}{2.650923in}}%
\pgfpathclose%
\pgfusepath{fill}%
\end{pgfscope}%
\begin{pgfscope}%
\pgfpathrectangle{\pgfqpoint{1.072000in}{0.528000in}}{\pgfqpoint{3.696000in}{3.696000in}}%
\pgfusepath{clip}%
\pgfsetbuttcap%
\pgfsetroundjoin%
\definecolor{currentfill}{rgb}{0.859385,0.864431,0.872111}%
\pgfsetfillcolor{currentfill}%
\pgfsetlinewidth{0.000000pt}%
\definecolor{currentstroke}{rgb}{0.000000,0.000000,0.000000}%
\pgfsetstrokecolor{currentstroke}%
\pgfsetdash{}{0pt}%
\pgfpathmoveto{\pgfqpoint{3.594886in}{2.389722in}}%
\pgfpathlineto{\pgfqpoint{3.638846in}{2.192096in}}%
\pgfpathlineto{\pgfqpoint{3.663905in}{2.194933in}}%
\pgfpathlineto{\pgfqpoint{3.620132in}{2.390277in}}%
\pgfpathlineto{\pgfqpoint{3.594886in}{2.389722in}}%
\pgfpathclose%
\pgfusepath{fill}%
\end{pgfscope}%
\begin{pgfscope}%
\pgfpathrectangle{\pgfqpoint{1.072000in}{0.528000in}}{\pgfqpoint{3.696000in}{3.696000in}}%
\pgfusepath{clip}%
\pgfsetbuttcap%
\pgfsetroundjoin%
\definecolor{currentfill}{rgb}{0.729196,0.086679,0.167240}%
\pgfsetfillcolor{currentfill}%
\pgfsetlinewidth{0.000000pt}%
\definecolor{currentstroke}{rgb}{0.000000,0.000000,0.000000}%
\pgfsetstrokecolor{currentstroke}%
\pgfsetdash{}{0pt}%
\pgfpathmoveto{\pgfqpoint{3.308352in}{3.269444in}}%
\pgfpathlineto{\pgfqpoint{3.355285in}{3.240840in}}%
\pgfpathlineto{\pgfqpoint{3.380789in}{3.208980in}}%
\pgfpathlineto{\pgfqpoint{3.334167in}{3.253552in}}%
\pgfpathlineto{\pgfqpoint{3.308352in}{3.269444in}}%
\pgfpathclose%
\pgfusepath{fill}%
\end{pgfscope}%
\begin{pgfscope}%
\pgfpathrectangle{\pgfqpoint{1.072000in}{0.528000in}}{\pgfqpoint{3.696000in}{3.696000in}}%
\pgfusepath{clip}%
\pgfsetbuttcap%
\pgfsetroundjoin%
\definecolor{currentfill}{rgb}{0.388852,0.516298,0.921373}%
\pgfsetfillcolor{currentfill}%
\pgfsetlinewidth{0.000000pt}%
\definecolor{currentstroke}{rgb}{0.000000,0.000000,0.000000}%
\pgfsetstrokecolor{currentstroke}%
\pgfsetdash{}{0pt}%
\pgfpathmoveto{\pgfqpoint{2.390767in}{1.496324in}}%
\pgfpathlineto{\pgfqpoint{2.436318in}{1.504072in}}%
\pgfpathlineto{\pgfqpoint{2.462406in}{1.652052in}}%
\pgfpathlineto{\pgfqpoint{2.416879in}{1.635198in}}%
\pgfpathlineto{\pgfqpoint{2.390767in}{1.496324in}}%
\pgfpathclose%
\pgfusepath{fill}%
\end{pgfscope}%
\begin{pgfscope}%
\pgfpathrectangle{\pgfqpoint{1.072000in}{0.528000in}}{\pgfqpoint{3.696000in}{3.696000in}}%
\pgfusepath{clip}%
\pgfsetbuttcap%
\pgfsetroundjoin%
\definecolor{currentfill}{rgb}{0.852378,0.346492,0.280346}%
\pgfsetfillcolor{currentfill}%
\pgfsetlinewidth{0.000000pt}%
\definecolor{currentstroke}{rgb}{0.000000,0.000000,0.000000}%
\pgfsetstrokecolor{currentstroke}%
\pgfsetdash{}{0pt}%
\pgfpathmoveto{\pgfqpoint{2.820008in}{3.110591in}}%
\pgfpathlineto{\pgfqpoint{2.866932in}{2.976997in}}%
\pgfpathlineto{\pgfqpoint{2.893934in}{3.021063in}}%
\pgfpathlineto{\pgfqpoint{2.847170in}{3.144800in}}%
\pgfpathlineto{\pgfqpoint{2.820008in}{3.110591in}}%
\pgfpathclose%
\pgfusepath{fill}%
\end{pgfscope}%
\begin{pgfscope}%
\pgfpathrectangle{\pgfqpoint{1.072000in}{0.528000in}}{\pgfqpoint{3.696000in}{3.696000in}}%
\pgfusepath{clip}%
\pgfsetbuttcap%
\pgfsetroundjoin%
\definecolor{currentfill}{rgb}{0.758112,0.168122,0.188827}%
\pgfsetfillcolor{currentfill}%
\pgfsetlinewidth{0.000000pt}%
\definecolor{currentstroke}{rgb}{0.000000,0.000000,0.000000}%
\pgfsetstrokecolor{currentstroke}%
\pgfsetdash{}{0pt}%
\pgfpathmoveto{\pgfqpoint{2.800396in}{3.232379in}}%
\pgfpathlineto{\pgfqpoint{2.847170in}{3.144800in}}%
\pgfpathlineto{\pgfqpoint{2.874190in}{3.198078in}}%
\pgfpathlineto{\pgfqpoint{2.827565in}{3.268015in}}%
\pgfpathlineto{\pgfqpoint{2.800396in}{3.232379in}}%
\pgfpathclose%
\pgfusepath{fill}%
\end{pgfscope}%
\begin{pgfscope}%
\pgfpathrectangle{\pgfqpoint{1.072000in}{0.528000in}}{\pgfqpoint{3.696000in}{3.696000in}}%
\pgfusepath{clip}%
\pgfsetbuttcap%
\pgfsetroundjoin%
\definecolor{currentfill}{rgb}{0.229806,0.298718,0.753683}%
\pgfsetfillcolor{currentfill}%
\pgfsetlinewidth{0.000000pt}%
\definecolor{currentstroke}{rgb}{0.000000,0.000000,0.000000}%
\pgfsetstrokecolor{currentstroke}%
\pgfsetdash{}{0pt}%
\pgfpathmoveto{\pgfqpoint{3.684231in}{1.306780in}}%
\pgfpathlineto{\pgfqpoint{3.732311in}{1.317497in}}%
\pgfpathlineto{\pgfqpoint{3.757406in}{1.331454in}}%
\pgfpathlineto{\pgfqpoint{3.709654in}{1.332583in}}%
\pgfpathlineto{\pgfqpoint{3.684231in}{1.306780in}}%
\pgfpathclose%
\pgfusepath{fill}%
\end{pgfscope}%
\begin{pgfscope}%
\pgfpathrectangle{\pgfqpoint{1.072000in}{0.528000in}}{\pgfqpoint{3.696000in}{3.696000in}}%
\pgfusepath{clip}%
\pgfsetbuttcap%
\pgfsetroundjoin%
\definecolor{currentfill}{rgb}{0.333490,0.446265,0.874452}%
\pgfsetfillcolor{currentfill}%
\pgfsetlinewidth{0.000000pt}%
\definecolor{currentstroke}{rgb}{0.000000,0.000000,0.000000}%
\pgfsetstrokecolor{currentstroke}%
\pgfsetdash{}{0pt}%
\pgfpathmoveto{\pgfqpoint{3.545076in}{1.507829in}}%
\pgfpathlineto{\pgfqpoint{3.590827in}{1.404473in}}%
\pgfpathlineto{\pgfqpoint{3.616944in}{1.456837in}}%
\pgfpathlineto{\pgfqpoint{3.571531in}{1.573835in}}%
\pgfpathlineto{\pgfqpoint{3.545076in}{1.507829in}}%
\pgfpathclose%
\pgfusepath{fill}%
\end{pgfscope}%
\begin{pgfscope}%
\pgfpathrectangle{\pgfqpoint{1.072000in}{0.528000in}}{\pgfqpoint{3.696000in}{3.696000in}}%
\pgfusepath{clip}%
\pgfsetbuttcap%
\pgfsetroundjoin%
\definecolor{currentfill}{rgb}{0.711554,0.033337,0.154485}%
\pgfsetfillcolor{currentfill}%
\pgfsetlinewidth{0.000000pt}%
\definecolor{currentstroke}{rgb}{0.000000,0.000000,0.000000}%
\pgfsetstrokecolor{currentstroke}%
\pgfsetdash{}{0pt}%
\pgfpathmoveto{\pgfqpoint{2.753796in}{3.277022in}}%
\pgfpathlineto{\pgfqpoint{2.800396in}{3.232379in}}%
\pgfpathlineto{\pgfqpoint{2.827565in}{3.268015in}}%
\pgfpathlineto{\pgfqpoint{2.781103in}{3.299493in}}%
\pgfpathlineto{\pgfqpoint{2.753796in}{3.277022in}}%
\pgfpathclose%
\pgfusepath{fill}%
\end{pgfscope}%
\begin{pgfscope}%
\pgfpathrectangle{\pgfqpoint{1.072000in}{0.528000in}}{\pgfqpoint{3.696000in}{3.696000in}}%
\pgfusepath{clip}%
\pgfsetbuttcap%
\pgfsetroundjoin%
\definecolor{currentfill}{rgb}{0.965899,0.740142,0.637058}%
\pgfsetfillcolor{currentfill}%
\pgfsetlinewidth{0.000000pt}%
\definecolor{currentstroke}{rgb}{0.000000,0.000000,0.000000}%
\pgfsetstrokecolor{currentstroke}%
\pgfsetdash{}{0pt}%
\pgfpathmoveto{\pgfqpoint{2.373838in}{2.462531in}}%
\pgfpathlineto{\pgfqpoint{2.417563in}{2.581064in}}%
\pgfpathlineto{\pgfqpoint{2.443807in}{2.721803in}}%
\pgfpathlineto{\pgfqpoint{2.399905in}{2.601621in}}%
\pgfpathlineto{\pgfqpoint{2.373838in}{2.462531in}}%
\pgfpathclose%
\pgfusepath{fill}%
\end{pgfscope}%
\begin{pgfscope}%
\pgfpathrectangle{\pgfqpoint{1.072000in}{0.528000in}}{\pgfqpoint{3.696000in}{3.696000in}}%
\pgfusepath{clip}%
\pgfsetbuttcap%
\pgfsetroundjoin%
\definecolor{currentfill}{rgb}{0.763520,0.178667,0.193396}%
\pgfsetfillcolor{currentfill}%
\pgfsetlinewidth{0.000000pt}%
\definecolor{currentstroke}{rgb}{0.000000,0.000000,0.000000}%
\pgfsetstrokecolor{currentstroke}%
\pgfsetdash{}{0pt}%
\pgfpathmoveto{\pgfqpoint{3.355285in}{3.240840in}}%
\pgfpathlineto{\pgfqpoint{3.402063in}{3.193893in}}%
\pgfpathlineto{\pgfqpoint{3.427201in}{3.146359in}}%
\pgfpathlineto{\pgfqpoint{3.380789in}{3.208980in}}%
\pgfpathlineto{\pgfqpoint{3.355285in}{3.240840in}}%
\pgfpathclose%
\pgfusepath{fill}%
\end{pgfscope}%
\begin{pgfscope}%
\pgfpathrectangle{\pgfqpoint{1.072000in}{0.528000in}}{\pgfqpoint{3.696000in}{3.696000in}}%
\pgfusepath{clip}%
\pgfsetbuttcap%
\pgfsetroundjoin%
\definecolor{currentfill}{rgb}{0.711554,0.033337,0.154485}%
\pgfsetfillcolor{currentfill}%
\pgfsetlinewidth{0.000000pt}%
\definecolor{currentstroke}{rgb}{0.000000,0.000000,0.000000}%
\pgfsetstrokecolor{currentstroke}%
\pgfsetdash{}{0pt}%
\pgfpathmoveto{\pgfqpoint{3.187850in}{3.243994in}}%
\pgfpathlineto{\pgfqpoint{3.234870in}{3.245483in}}%
\pgfpathlineto{\pgfqpoint{3.261387in}{3.286164in}}%
\pgfpathlineto{\pgfqpoint{3.214486in}{3.297688in}}%
\pgfpathlineto{\pgfqpoint{3.187850in}{3.243994in}}%
\pgfpathclose%
\pgfusepath{fill}%
\end{pgfscope}%
\begin{pgfscope}%
\pgfpathrectangle{\pgfqpoint{1.072000in}{0.528000in}}{\pgfqpoint{3.696000in}{3.696000in}}%
\pgfusepath{clip}%
\pgfsetbuttcap%
\pgfsetroundjoin%
\definecolor{currentfill}{rgb}{0.728970,0.817464,0.973188}%
\pgfsetfillcolor{currentfill}%
\pgfsetlinewidth{0.000000pt}%
\definecolor{currentstroke}{rgb}{0.000000,0.000000,0.000000}%
\pgfsetstrokecolor{currentstroke}%
\pgfsetdash{}{0pt}%
\pgfpathmoveto{\pgfqpoint{3.587575in}{2.144398in}}%
\pgfpathlineto{\pgfqpoint{3.631662in}{1.953060in}}%
\pgfpathlineto{\pgfqpoint{3.657280in}{1.983338in}}%
\pgfpathlineto{\pgfqpoint{3.613394in}{2.175333in}}%
\pgfpathlineto{\pgfqpoint{3.587575in}{2.144398in}}%
\pgfpathclose%
\pgfusepath{fill}%
\end{pgfscope}%
\begin{pgfscope}%
\pgfpathrectangle{\pgfqpoint{1.072000in}{0.528000in}}{\pgfqpoint{3.696000in}{3.696000in}}%
\pgfusepath{clip}%
\pgfsetbuttcap%
\pgfsetroundjoin%
\definecolor{currentfill}{rgb}{0.248091,0.326013,0.777669}%
\pgfsetfillcolor{currentfill}%
\pgfsetlinewidth{0.000000pt}%
\definecolor{currentstroke}{rgb}{0.000000,0.000000,0.000000}%
\pgfsetstrokecolor{currentstroke}%
\pgfsetdash{}{0pt}%
\pgfpathmoveto{\pgfqpoint{3.564666in}{1.353088in}}%
\pgfpathlineto{\pgfqpoint{3.611344in}{1.300467in}}%
\pgfpathlineto{\pgfqpoint{3.637137in}{1.336282in}}%
\pgfpathlineto{\pgfqpoint{3.590827in}{1.404473in}}%
\pgfpathlineto{\pgfqpoint{3.564666in}{1.353088in}}%
\pgfpathclose%
\pgfusepath{fill}%
\end{pgfscope}%
\begin{pgfscope}%
\pgfpathrectangle{\pgfqpoint{1.072000in}{0.528000in}}{\pgfqpoint{3.696000in}{3.696000in}}%
\pgfusepath{clip}%
\pgfsetbuttcap%
\pgfsetroundjoin%
\definecolor{currentfill}{rgb}{0.333490,0.446265,0.874452}%
\pgfsetfillcolor{currentfill}%
\pgfsetlinewidth{0.000000pt}%
\definecolor{currentstroke}{rgb}{0.000000,0.000000,0.000000}%
\pgfsetstrokecolor{currentstroke}%
\pgfsetdash{}{0pt}%
\pgfpathmoveto{\pgfqpoint{2.046378in}{1.550664in}}%
\pgfpathlineto{\pgfqpoint{2.093313in}{1.489309in}}%
\pgfpathlineto{\pgfqpoint{2.124227in}{1.426495in}}%
\pgfpathlineto{\pgfqpoint{2.077828in}{1.475791in}}%
\pgfpathlineto{\pgfqpoint{2.046378in}{1.550664in}}%
\pgfpathclose%
\pgfusepath{fill}%
\end{pgfscope}%
\begin{pgfscope}%
\pgfpathrectangle{\pgfqpoint{1.072000in}{0.528000in}}{\pgfqpoint{3.696000in}{3.696000in}}%
\pgfusepath{clip}%
\pgfsetbuttcap%
\pgfsetroundjoin%
\definecolor{currentfill}{rgb}{0.229806,0.298718,0.753683}%
\pgfsetfillcolor{currentfill}%
\pgfsetlinewidth{0.000000pt}%
\definecolor{currentstroke}{rgb}{0.000000,0.000000,0.000000}%
\pgfsetstrokecolor{currentstroke}%
\pgfsetdash{}{0pt}%
\pgfpathmoveto{\pgfqpoint{2.308430in}{1.306011in}}%
\pgfpathlineto{\pgfqpoint{2.354430in}{1.287351in}}%
\pgfpathlineto{\pgfqpoint{2.382496in}{1.317715in}}%
\pgfpathlineto{\pgfqpoint{2.336707in}{1.328126in}}%
\pgfpathlineto{\pgfqpoint{2.308430in}{1.306011in}}%
\pgfpathclose%
\pgfusepath{fill}%
\end{pgfscope}%
\begin{pgfscope}%
\pgfpathrectangle{\pgfqpoint{1.072000in}{0.528000in}}{\pgfqpoint{3.696000in}{3.696000in}}%
\pgfusepath{clip}%
\pgfsetbuttcap%
\pgfsetroundjoin%
\definecolor{currentfill}{rgb}{0.252663,0.332837,0.783665}%
\pgfsetfillcolor{currentfill}%
\pgfsetlinewidth{0.000000pt}%
\definecolor{currentstroke}{rgb}{0.000000,0.000000,0.000000}%
\pgfsetstrokecolor{currentstroke}%
\pgfsetdash{}{0pt}%
\pgfpathmoveto{\pgfqpoint{2.382496in}{1.317715in}}%
\pgfpathlineto{\pgfqpoint{2.428408in}{1.307072in}}%
\pgfpathlineto{\pgfqpoint{2.455561in}{1.386814in}}%
\pgfpathlineto{\pgfqpoint{2.409740in}{1.390925in}}%
\pgfpathlineto{\pgfqpoint{2.382496in}{1.317715in}}%
\pgfpathclose%
\pgfusepath{fill}%
\end{pgfscope}%
\begin{pgfscope}%
\pgfpathrectangle{\pgfqpoint{1.072000in}{0.528000in}}{\pgfqpoint{3.696000in}{3.696000in}}%
\pgfusepath{clip}%
\pgfsetbuttcap%
\pgfsetroundjoin%
\definecolor{currentfill}{rgb}{0.624703,0.748318,0.998719}%
\pgfsetfillcolor{currentfill}%
\pgfsetlinewidth{0.000000pt}%
\definecolor{currentstroke}{rgb}{0.000000,0.000000,0.000000}%
\pgfsetstrokecolor{currentstroke}%
\pgfsetdash{}{0pt}%
\pgfpathmoveto{\pgfqpoint{1.791184in}{2.017934in}}%
\pgfpathlineto{\pgfqpoint{1.840187in}{1.897768in}}%
\pgfpathlineto{\pgfqpoint{1.873539in}{1.793627in}}%
\pgfpathlineto{\pgfqpoint{1.825061in}{1.907614in}}%
\pgfpathlineto{\pgfqpoint{1.791184in}{2.017934in}}%
\pgfpathclose%
\pgfusepath{fill}%
\end{pgfscope}%
\begin{pgfscope}%
\pgfpathrectangle{\pgfqpoint{1.072000in}{0.528000in}}{\pgfqpoint{3.696000in}{3.696000in}}%
\pgfusepath{clip}%
\pgfsetbuttcap%
\pgfsetroundjoin%
\definecolor{currentfill}{rgb}{0.304174,0.406945,0.845263}%
\pgfsetfillcolor{currentfill}%
\pgfsetlinewidth{0.000000pt}%
\definecolor{currentstroke}{rgb}{0.000000,0.000000,0.000000}%
\pgfsetstrokecolor{currentstroke}%
\pgfsetdash{}{0pt}%
\pgfpathmoveto{\pgfqpoint{2.093313in}{1.489309in}}%
\pgfpathlineto{\pgfqpoint{2.139997in}{1.437343in}}%
\pgfpathlineto{\pgfqpoint{2.170418in}{1.386450in}}%
\pgfpathlineto{\pgfqpoint{2.124227in}{1.426495in}}%
\pgfpathlineto{\pgfqpoint{2.093313in}{1.489309in}}%
\pgfpathclose%
\pgfusepath{fill}%
\end{pgfscope}%
\begin{pgfscope}%
\pgfpathrectangle{\pgfqpoint{1.072000in}{0.528000in}}{\pgfqpoint{3.696000in}{3.696000in}}%
\pgfusepath{clip}%
\pgfsetbuttcap%
\pgfsetroundjoin%
\definecolor{currentfill}{rgb}{0.839365,0.321856,0.264924}%
\pgfsetfillcolor{currentfill}%
\pgfsetlinewidth{0.000000pt}%
\definecolor{currentstroke}{rgb}{0.000000,0.000000,0.000000}%
\pgfsetstrokecolor{currentstroke}%
\pgfsetdash{}{0pt}%
\pgfpathmoveto{\pgfqpoint{2.507026in}{2.998142in}}%
\pgfpathlineto{\pgfqpoint{2.551765in}{3.097688in}}%
\pgfpathlineto{\pgfqpoint{2.579237in}{3.148114in}}%
\pgfpathlineto{\pgfqpoint{2.534351in}{3.055175in}}%
\pgfpathlineto{\pgfqpoint{2.507026in}{2.998142in}}%
\pgfpathclose%
\pgfusepath{fill}%
\end{pgfscope}%
\begin{pgfscope}%
\pgfpathrectangle{\pgfqpoint{1.072000in}{0.528000in}}{\pgfqpoint{3.696000in}{3.696000in}}%
\pgfusepath{clip}%
\pgfsetbuttcap%
\pgfsetroundjoin%
\definecolor{currentfill}{rgb}{0.830187,0.304733,0.254891}%
\pgfsetfillcolor{currentfill}%
\pgfsetlinewidth{0.000000pt}%
\definecolor{currentstroke}{rgb}{0.000000,0.000000,0.000000}%
\pgfsetstrokecolor{currentstroke}%
\pgfsetdash{}{0pt}%
\pgfpathmoveto{\pgfqpoint{3.133832in}{3.014233in}}%
\pgfpathlineto{\pgfqpoint{3.180959in}{3.056971in}}%
\pgfpathlineto{\pgfqpoint{3.208015in}{3.166182in}}%
\pgfpathlineto{\pgfqpoint{3.160923in}{3.147274in}}%
\pgfpathlineto{\pgfqpoint{3.133832in}{3.014233in}}%
\pgfpathclose%
\pgfusepath{fill}%
\end{pgfscope}%
\begin{pgfscope}%
\pgfpathrectangle{\pgfqpoint{1.072000in}{0.528000in}}{\pgfqpoint{3.696000in}{3.696000in}}%
\pgfusepath{clip}%
\pgfsetbuttcap%
\pgfsetroundjoin%
\definecolor{currentfill}{rgb}{0.313946,0.420052,0.854993}%
\pgfsetfillcolor{currentfill}%
\pgfsetlinewidth{0.000000pt}%
\definecolor{currentstroke}{rgb}{0.000000,0.000000,0.000000}%
\pgfsetstrokecolor{currentstroke}%
\pgfsetdash{}{0pt}%
\pgfpathmoveto{\pgfqpoint{2.409740in}{1.390925in}}%
\pgfpathlineto{\pgfqpoint{2.455561in}{1.386814in}}%
\pgfpathlineto{\pgfqpoint{2.482110in}{1.506680in}}%
\pgfpathlineto{\pgfqpoint{2.436318in}{1.504072in}}%
\pgfpathlineto{\pgfqpoint{2.409740in}{1.390925in}}%
\pgfpathclose%
\pgfusepath{fill}%
\end{pgfscope}%
\begin{pgfscope}%
\pgfpathrectangle{\pgfqpoint{1.072000in}{0.528000in}}{\pgfqpoint{3.696000in}{3.696000in}}%
\pgfusepath{clip}%
\pgfsetbuttcap%
\pgfsetroundjoin%
\definecolor{currentfill}{rgb}{0.510824,0.649397,0.985079}%
\pgfsetfillcolor{currentfill}%
\pgfsetlinewidth{0.000000pt}%
\definecolor{currentstroke}{rgb}{0.000000,0.000000,0.000000}%
\pgfsetstrokecolor{currentstroke}%
\pgfsetdash{}{0pt}%
\pgfpathmoveto{\pgfqpoint{3.553057in}{1.792772in}}%
\pgfpathlineto{\pgfqpoint{3.597873in}{1.635954in}}%
\pgfpathlineto{\pgfqpoint{3.624051in}{1.691968in}}%
\pgfpathlineto{\pgfqpoint{3.579509in}{1.856869in}}%
\pgfpathlineto{\pgfqpoint{3.553057in}{1.792772in}}%
\pgfpathclose%
\pgfusepath{fill}%
\end{pgfscope}%
\begin{pgfscope}%
\pgfpathrectangle{\pgfqpoint{1.072000in}{0.528000in}}{\pgfqpoint{3.696000in}{3.696000in}}%
\pgfusepath{clip}%
\pgfsetbuttcap%
\pgfsetroundjoin%
\definecolor{currentfill}{rgb}{0.777378,0.840921,0.946149}%
\pgfsetfillcolor{currentfill}%
\pgfsetlinewidth{0.000000pt}%
\definecolor{currentstroke}{rgb}{0.000000,0.000000,0.000000}%
\pgfsetstrokecolor{currentstroke}%
\pgfsetdash{}{0pt}%
\pgfpathmoveto{\pgfqpoint{2.359082in}{2.004838in}}%
\pgfpathlineto{\pgfqpoint{2.403524in}{2.074986in}}%
\pgfpathlineto{\pgfqpoint{2.429001in}{2.261386in}}%
\pgfpathlineto{\pgfqpoint{2.384500in}{2.181326in}}%
\pgfpathlineto{\pgfqpoint{2.359082in}{2.004838in}}%
\pgfpathclose%
\pgfusepath{fill}%
\end{pgfscope}%
\begin{pgfscope}%
\pgfpathrectangle{\pgfqpoint{1.072000in}{0.528000in}}{\pgfqpoint{3.696000in}{3.696000in}}%
\pgfusepath{clip}%
\pgfsetbuttcap%
\pgfsetroundjoin%
\definecolor{currentfill}{rgb}{0.378598,0.503856,0.913692}%
\pgfsetfillcolor{currentfill}%
\pgfsetlinewidth{0.000000pt}%
\definecolor{currentstroke}{rgb}{0.000000,0.000000,0.000000}%
\pgfsetstrokecolor{currentstroke}%
\pgfsetdash{}{0pt}%
\pgfpathmoveto{\pgfqpoint{1.999127in}{1.622314in}}%
\pgfpathlineto{\pgfqpoint{2.046378in}{1.550664in}}%
\pgfpathlineto{\pgfqpoint{2.077828in}{1.475791in}}%
\pgfpathlineto{\pgfqpoint{2.031143in}{1.535748in}}%
\pgfpathlineto{\pgfqpoint{1.999127in}{1.622314in}}%
\pgfpathclose%
\pgfusepath{fill}%
\end{pgfscope}%
\begin{pgfscope}%
\pgfpathrectangle{\pgfqpoint{1.072000in}{0.528000in}}{\pgfqpoint{3.696000in}{3.696000in}}%
\pgfusepath{clip}%
\pgfsetbuttcap%
\pgfsetroundjoin%
\definecolor{currentfill}{rgb}{0.968500,0.673977,0.556649}%
\pgfsetfillcolor{currentfill}%
\pgfsetlinewidth{0.000000pt}%
\definecolor{currentstroke}{rgb}{0.000000,0.000000,0.000000}%
\pgfsetstrokecolor{currentstroke}%
\pgfsetdash{}{0pt}%
\pgfpathmoveto{\pgfqpoint{2.952840in}{2.741032in}}%
\pgfpathlineto{\pgfqpoint{2.999642in}{2.766222in}}%
\pgfpathlineto{\pgfqpoint{3.026347in}{2.650923in}}%
\pgfpathlineto{\pgfqpoint{2.979753in}{2.590528in}}%
\pgfpathlineto{\pgfqpoint{2.952840in}{2.741032in}}%
\pgfpathclose%
\pgfusepath{fill}%
\end{pgfscope}%
\begin{pgfscope}%
\pgfpathrectangle{\pgfqpoint{1.072000in}{0.528000in}}{\pgfqpoint{3.696000in}{3.696000in}}%
\pgfusepath{clip}%
\pgfsetbuttcap%
\pgfsetroundjoin%
\definecolor{currentfill}{rgb}{0.229806,0.298718,0.753683}%
\pgfsetfillcolor{currentfill}%
\pgfsetlinewidth{0.000000pt}%
\definecolor{currentstroke}{rgb}{0.000000,0.000000,0.000000}%
\pgfsetstrokecolor{currentstroke}%
\pgfsetdash{}{0pt}%
\pgfpathmoveto{\pgfqpoint{3.611344in}{1.300467in}}%
\pgfpathlineto{\pgfqpoint{3.658805in}{1.285643in}}%
\pgfpathlineto{\pgfqpoint{3.684231in}{1.306780in}}%
\pgfpathlineto{\pgfqpoint{3.637137in}{1.336282in}}%
\pgfpathlineto{\pgfqpoint{3.611344in}{1.300467in}}%
\pgfpathclose%
\pgfusepath{fill}%
\end{pgfscope}%
\begin{pgfscope}%
\pgfpathrectangle{\pgfqpoint{1.072000in}{0.528000in}}{\pgfqpoint{3.696000in}{3.696000in}}%
\pgfusepath{clip}%
\pgfsetbuttcap%
\pgfsetroundjoin%
\definecolor{currentfill}{rgb}{0.953054,0.585211,0.465373}%
\pgfsetfillcolor{currentfill}%
\pgfsetlinewidth{0.000000pt}%
\definecolor{currentstroke}{rgb}{0.000000,0.000000,0.000000}%
\pgfsetstrokecolor{currentstroke}%
\pgfsetdash{}{0pt}%
\pgfpathmoveto{\pgfqpoint{2.859225in}{2.895339in}}%
\pgfpathlineto{\pgfqpoint{2.906110in}{2.792406in}}%
\pgfpathlineto{\pgfqpoint{2.933310in}{2.677220in}}%
\pgfpathlineto{\pgfqpoint{2.886626in}{2.827977in}}%
\pgfpathlineto{\pgfqpoint{2.859225in}{2.895339in}}%
\pgfpathclose%
\pgfusepath{fill}%
\end{pgfscope}%
\begin{pgfscope}%
\pgfpathrectangle{\pgfqpoint{1.072000in}{0.528000in}}{\pgfqpoint{3.696000in}{3.696000in}}%
\pgfusepath{clip}%
\pgfsetbuttcap%
\pgfsetroundjoin%
\definecolor{currentfill}{rgb}{0.959385,0.610306,0.489382}%
\pgfsetfillcolor{currentfill}%
\pgfsetlinewidth{0.000000pt}%
\definecolor{currentstroke}{rgb}{0.000000,0.000000,0.000000}%
\pgfsetstrokecolor{currentstroke}%
\pgfsetdash{}{0pt}%
\pgfpathmoveto{\pgfqpoint{3.536234in}{2.871345in}}%
\pgfpathlineto{\pgfqpoint{3.581200in}{2.712154in}}%
\pgfpathlineto{\pgfqpoint{3.605568in}{2.664308in}}%
\pgfpathlineto{\pgfqpoint{3.560905in}{2.825589in}}%
\pgfpathlineto{\pgfqpoint{3.536234in}{2.871345in}}%
\pgfpathclose%
\pgfusepath{fill}%
\end{pgfscope}%
\begin{pgfscope}%
\pgfpathrectangle{\pgfqpoint{1.072000in}{0.528000in}}{\pgfqpoint{3.696000in}{3.696000in}}%
\pgfusepath{clip}%
\pgfsetbuttcap%
\pgfsetroundjoin%
\definecolor{currentfill}{rgb}{0.758112,0.168122,0.188827}%
\pgfsetfillcolor{currentfill}%
\pgfsetlinewidth{0.000000pt}%
\definecolor{currentstroke}{rgb}{0.000000,0.000000,0.000000}%
\pgfsetstrokecolor{currentstroke}%
\pgfsetdash{}{0pt}%
\pgfpathmoveto{\pgfqpoint{3.160923in}{3.147274in}}%
\pgfpathlineto{\pgfqpoint{3.208015in}{3.166182in}}%
\pgfpathlineto{\pgfqpoint{3.234870in}{3.245483in}}%
\pgfpathlineto{\pgfqpoint{3.187850in}{3.243994in}}%
\pgfpathlineto{\pgfqpoint{3.160923in}{3.147274in}}%
\pgfpathclose%
\pgfusepath{fill}%
\end{pgfscope}%
\begin{pgfscope}%
\pgfpathrectangle{\pgfqpoint{1.072000in}{0.528000in}}{\pgfqpoint{3.696000in}{3.696000in}}%
\pgfusepath{clip}%
\pgfsetbuttcap%
\pgfsetroundjoin%
\definecolor{currentfill}{rgb}{0.280550,0.373423,0.818011}%
\pgfsetfillcolor{currentfill}%
\pgfsetlinewidth{0.000000pt}%
\definecolor{currentstroke}{rgb}{0.000000,0.000000,0.000000}%
\pgfsetstrokecolor{currentstroke}%
\pgfsetdash{}{0pt}%
\pgfpathmoveto{\pgfqpoint{2.139997in}{1.437343in}}%
\pgfpathlineto{\pgfqpoint{2.186492in}{1.393662in}}%
\pgfpathlineto{\pgfqpoint{2.216474in}{1.354087in}}%
\pgfpathlineto{\pgfqpoint{2.170418in}{1.386450in}}%
\pgfpathlineto{\pgfqpoint{2.139997in}{1.437343in}}%
\pgfpathclose%
\pgfusepath{fill}%
\end{pgfscope}%
\begin{pgfscope}%
\pgfpathrectangle{\pgfqpoint{1.072000in}{0.528000in}}{\pgfqpoint{3.696000in}{3.696000in}}%
\pgfusepath{clip}%
\pgfsetbuttcap%
\pgfsetroundjoin%
\definecolor{currentfill}{rgb}{0.740957,0.122240,0.175744}%
\pgfsetfillcolor{currentfill}%
\pgfsetlinewidth{0.000000pt}%
\definecolor{currentstroke}{rgb}{0.000000,0.000000,0.000000}%
\pgfsetstrokecolor{currentstroke}%
\pgfsetdash{}{0pt}%
\pgfpathmoveto{\pgfqpoint{2.606774in}{3.185689in}}%
\pgfpathlineto{\pgfqpoint{2.652501in}{3.232415in}}%
\pgfpathlineto{\pgfqpoint{2.680065in}{3.256646in}}%
\pgfpathlineto{\pgfqpoint{2.634326in}{3.212417in}}%
\pgfpathlineto{\pgfqpoint{2.606774in}{3.185689in}}%
\pgfpathclose%
\pgfusepath{fill}%
\end{pgfscope}%
\begin{pgfscope}%
\pgfpathrectangle{\pgfqpoint{1.072000in}{0.528000in}}{\pgfqpoint{3.696000in}{3.696000in}}%
\pgfusepath{clip}%
\pgfsetbuttcap%
\pgfsetroundjoin%
\definecolor{currentfill}{rgb}{0.947345,0.794696,0.716991}%
\pgfsetfillcolor{currentfill}%
\pgfsetlinewidth{0.000000pt}%
\definecolor{currentstroke}{rgb}{0.000000,0.000000,0.000000}%
\pgfsetstrokecolor{currentstroke}%
\pgfsetdash{}{0pt}%
\pgfpathmoveto{\pgfqpoint{3.575910in}{2.582299in}}%
\pgfpathlineto{\pgfqpoint{3.620132in}{2.390277in}}%
\pgfpathlineto{\pgfqpoint{3.644953in}{2.376017in}}%
\pgfpathlineto{\pgfqpoint{3.600944in}{2.566072in}}%
\pgfpathlineto{\pgfqpoint{3.575910in}{2.582299in}}%
\pgfpathclose%
\pgfusepath{fill}%
\end{pgfscope}%
\begin{pgfscope}%
\pgfpathrectangle{\pgfqpoint{1.072000in}{0.528000in}}{\pgfqpoint{3.696000in}{3.696000in}}%
\pgfusepath{clip}%
\pgfsetbuttcap%
\pgfsetroundjoin%
\definecolor{currentfill}{rgb}{0.304174,0.406945,0.845263}%
\pgfsetfillcolor{currentfill}%
\pgfsetlinewidth{0.000000pt}%
\definecolor{currentstroke}{rgb}{0.000000,0.000000,0.000000}%
\pgfsetstrokecolor{currentstroke}%
\pgfsetdash{}{0pt}%
\pgfpathmoveto{\pgfqpoint{3.518560in}{1.440606in}}%
\pgfpathlineto{\pgfqpoint{3.564666in}{1.353088in}}%
\pgfpathlineto{\pgfqpoint{3.590827in}{1.404473in}}%
\pgfpathlineto{\pgfqpoint{3.545076in}{1.507829in}}%
\pgfpathlineto{\pgfqpoint{3.518560in}{1.440606in}}%
\pgfpathclose%
\pgfusepath{fill}%
\end{pgfscope}%
\begin{pgfscope}%
\pgfpathrectangle{\pgfqpoint{1.072000in}{0.528000in}}{\pgfqpoint{3.696000in}{3.696000in}}%
\pgfusepath{clip}%
\pgfsetbuttcap%
\pgfsetroundjoin%
\definecolor{currentfill}{rgb}{0.705673,0.015556,0.150233}%
\pgfsetfillcolor{currentfill}%
\pgfsetlinewidth{0.000000pt}%
\definecolor{currentstroke}{rgb}{0.000000,0.000000,0.000000}%
\pgfsetstrokecolor{currentstroke}%
\pgfsetdash{}{0pt}%
\pgfpathmoveto{\pgfqpoint{2.680065in}{3.256646in}}%
\pgfpathlineto{\pgfqpoint{2.726380in}{3.254423in}}%
\pgfpathlineto{\pgfqpoint{2.753796in}{3.277022in}}%
\pgfpathlineto{\pgfqpoint{2.707559in}{3.275731in}}%
\pgfpathlineto{\pgfqpoint{2.680065in}{3.256646in}}%
\pgfpathclose%
\pgfusepath{fill}%
\end{pgfscope}%
\begin{pgfscope}%
\pgfpathrectangle{\pgfqpoint{1.072000in}{0.528000in}}{\pgfqpoint{3.696000in}{3.696000in}}%
\pgfusepath{clip}%
\pgfsetbuttcap%
\pgfsetroundjoin%
\definecolor{currentfill}{rgb}{0.430507,0.564883,0.948889}%
\pgfsetfillcolor{currentfill}%
\pgfsetlinewidth{0.000000pt}%
\definecolor{currentstroke}{rgb}{0.000000,0.000000,0.000000}%
\pgfsetstrokecolor{currentstroke}%
\pgfsetdash{}{0pt}%
\pgfpathmoveto{\pgfqpoint{1.951501in}{1.704851in}}%
\pgfpathlineto{\pgfqpoint{1.999127in}{1.622314in}}%
\pgfpathlineto{\pgfqpoint{2.031143in}{1.535748in}}%
\pgfpathlineto{\pgfqpoint{1.984098in}{1.607489in}}%
\pgfpathlineto{\pgfqpoint{1.951501in}{1.704851in}}%
\pgfpathclose%
\pgfusepath{fill}%
\end{pgfscope}%
\begin{pgfscope}%
\pgfpathrectangle{\pgfqpoint{1.072000in}{0.528000in}}{\pgfqpoint{3.696000in}{3.696000in}}%
\pgfusepath{clip}%
\pgfsetbuttcap%
\pgfsetroundjoin%
\definecolor{currentfill}{rgb}{0.964911,0.640159,0.519806}%
\pgfsetfillcolor{currentfill}%
\pgfsetlinewidth{0.000000pt}%
\definecolor{currentstroke}{rgb}{0.000000,0.000000,0.000000}%
\pgfsetstrokecolor{currentstroke}%
\pgfsetdash{}{0pt}%
\pgfpathmoveto{\pgfqpoint{2.399905in}{2.601621in}}%
\pgfpathlineto{\pgfqpoint{2.443807in}{2.721803in}}%
\pgfpathlineto{\pgfqpoint{2.470383in}{2.844657in}}%
\pgfpathlineto{\pgfqpoint{2.426272in}{2.726347in}}%
\pgfpathlineto{\pgfqpoint{2.399905in}{2.601621in}}%
\pgfpathclose%
\pgfusepath{fill}%
\end{pgfscope}%
\begin{pgfscope}%
\pgfpathrectangle{\pgfqpoint{1.072000in}{0.528000in}}{\pgfqpoint{3.696000in}{3.696000in}}%
\pgfusepath{clip}%
\pgfsetbuttcap%
\pgfsetroundjoin%
\definecolor{currentfill}{rgb}{0.916071,0.833977,0.788693}%
\pgfsetfillcolor{currentfill}%
\pgfsetlinewidth{0.000000pt}%
\definecolor{currentstroke}{rgb}{0.000000,0.000000,0.000000}%
\pgfsetstrokecolor{currentstroke}%
\pgfsetdash{}{0pt}%
\pgfpathmoveto{\pgfqpoint{1.609876in}{2.508693in}}%
\pgfpathlineto{\pgfqpoint{1.660420in}{2.357328in}}%
\pgfpathlineto{\pgfqpoint{1.694471in}{2.259769in}}%
\pgfpathlineto{\pgfqpoint{1.644076in}{2.415926in}}%
\pgfpathlineto{\pgfqpoint{1.609876in}{2.508693in}}%
\pgfpathclose%
\pgfusepath{fill}%
\end{pgfscope}%
\begin{pgfscope}%
\pgfpathrectangle{\pgfqpoint{1.072000in}{0.528000in}}{\pgfqpoint{3.696000in}{3.696000in}}%
\pgfusepath{clip}%
\pgfsetbuttcap%
\pgfsetroundjoin%
\definecolor{currentfill}{rgb}{0.919376,0.831273,0.782874}%
\pgfsetfillcolor{currentfill}%
\pgfsetlinewidth{0.000000pt}%
\definecolor{currentstroke}{rgb}{0.000000,0.000000,0.000000}%
\pgfsetstrokecolor{currentstroke}%
\pgfsetdash{}{0pt}%
\pgfpathmoveto{\pgfqpoint{2.365979in}{2.261309in}}%
\pgfpathlineto{\pgfqpoint{2.410045in}{2.358545in}}%
\pgfpathlineto{\pgfqpoint{2.435834in}{2.528892in}}%
\pgfpathlineto{\pgfqpoint{2.391637in}{2.425888in}}%
\pgfpathlineto{\pgfqpoint{2.365979in}{2.261309in}}%
\pgfpathclose%
\pgfusepath{fill}%
\end{pgfscope}%
\begin{pgfscope}%
\pgfpathrectangle{\pgfqpoint{1.072000in}{0.528000in}}{\pgfqpoint{3.696000in}{3.696000in}}%
\pgfusepath{clip}%
\pgfsetbuttcap%
\pgfsetroundjoin%
\definecolor{currentfill}{rgb}{0.969522,0.700833,0.587508}%
\pgfsetfillcolor{currentfill}%
\pgfsetlinewidth{0.000000pt}%
\definecolor{currentstroke}{rgb}{0.000000,0.000000,0.000000}%
\pgfsetstrokecolor{currentstroke}%
\pgfsetdash{}{0pt}%
\pgfpathmoveto{\pgfqpoint{3.556359in}{2.744051in}}%
\pgfpathlineto{\pgfqpoint{3.600944in}{2.566072in}}%
\pgfpathlineto{\pgfqpoint{3.625530in}{2.534526in}}%
\pgfpathlineto{\pgfqpoint{3.581200in}{2.712154in}}%
\pgfpathlineto{\pgfqpoint{3.556359in}{2.744051in}}%
\pgfpathclose%
\pgfusepath{fill}%
\end{pgfscope}%
\begin{pgfscope}%
\pgfpathrectangle{\pgfqpoint{1.072000in}{0.528000in}}{\pgfqpoint{3.696000in}{3.696000in}}%
\pgfusepath{clip}%
\pgfsetbuttcap%
\pgfsetroundjoin%
\definecolor{currentfill}{rgb}{0.813693,0.854282,0.918480}%
\pgfsetfillcolor{currentfill}%
\pgfsetlinewidth{0.000000pt}%
\definecolor{currentstroke}{rgb}{0.000000,0.000000,0.000000}%
\pgfsetstrokecolor{currentstroke}%
\pgfsetdash{}{0pt}%
\pgfpathmoveto{\pgfqpoint{1.676308in}{2.312863in}}%
\pgfpathlineto{\pgfqpoint{1.726382in}{2.170087in}}%
\pgfpathlineto{\pgfqpoint{1.760403in}{2.064470in}}%
\pgfpathlineto{\pgfqpoint{1.710651in}{2.207651in}}%
\pgfpathlineto{\pgfqpoint{1.676308in}{2.312863in}}%
\pgfpathclose%
\pgfusepath{fill}%
\end{pgfscope}%
\begin{pgfscope}%
\pgfpathrectangle{\pgfqpoint{1.072000in}{0.528000in}}{\pgfqpoint{3.696000in}{3.696000in}}%
\pgfusepath{clip}%
\pgfsetbuttcap%
\pgfsetroundjoin%
\definecolor{currentfill}{rgb}{0.869655,0.379274,0.300941}%
\pgfsetfillcolor{currentfill}%
\pgfsetlinewidth{0.000000pt}%
\definecolor{currentstroke}{rgb}{0.000000,0.000000,0.000000}%
\pgfsetstrokecolor{currentstroke}%
\pgfsetdash{}{0pt}%
\pgfpathmoveto{\pgfqpoint{2.479870in}{2.924913in}}%
\pgfpathlineto{\pgfqpoint{2.524419in}{3.031787in}}%
\pgfpathlineto{\pgfqpoint{2.551765in}{3.097688in}}%
\pgfpathlineto{\pgfqpoint{2.507026in}{2.998142in}}%
\pgfpathlineto{\pgfqpoint{2.479870in}{2.924913in}}%
\pgfpathclose%
\pgfusepath{fill}%
\end{pgfscope}%
\begin{pgfscope}%
\pgfpathrectangle{\pgfqpoint{1.072000in}{0.528000in}}{\pgfqpoint{3.696000in}{3.696000in}}%
\pgfusepath{clip}%
\pgfsetbuttcap%
\pgfsetroundjoin%
\definecolor{currentfill}{rgb}{0.261805,0.346484,0.795658}%
\pgfsetfillcolor{currentfill}%
\pgfsetlinewidth{0.000000pt}%
\definecolor{currentstroke}{rgb}{0.000000,0.000000,0.000000}%
\pgfsetstrokecolor{currentstroke}%
\pgfsetdash{}{0pt}%
\pgfpathmoveto{\pgfqpoint{2.186492in}{1.393662in}}%
\pgfpathlineto{\pgfqpoint{2.232860in}{1.357063in}}%
\pgfpathlineto{\pgfqpoint{2.262459in}{1.327791in}}%
\pgfpathlineto{\pgfqpoint{2.216474in}{1.354087in}}%
\pgfpathlineto{\pgfqpoint{2.186492in}{1.393662in}}%
\pgfpathclose%
\pgfusepath{fill}%
\end{pgfscope}%
\begin{pgfscope}%
\pgfpathrectangle{\pgfqpoint{1.072000in}{0.528000in}}{\pgfqpoint{3.696000in}{3.696000in}}%
\pgfusepath{clip}%
\pgfsetbuttcap%
\pgfsetroundjoin%
\definecolor{currentfill}{rgb}{0.473070,0.611077,0.970634}%
\pgfsetfillcolor{currentfill}%
\pgfsetlinewidth{0.000000pt}%
\definecolor{currentstroke}{rgb}{0.000000,0.000000,0.000000}%
\pgfsetstrokecolor{currentstroke}%
\pgfsetdash{}{0pt}%
\pgfpathmoveto{\pgfqpoint{3.526421in}{1.720081in}}%
\pgfpathlineto{\pgfqpoint{3.571531in}{1.573835in}}%
\pgfpathlineto{\pgfqpoint{3.597873in}{1.635954in}}%
\pgfpathlineto{\pgfqpoint{3.553057in}{1.792772in}}%
\pgfpathlineto{\pgfqpoint{3.526421in}{1.720081in}}%
\pgfpathclose%
\pgfusepath{fill}%
\end{pgfscope}%
\begin{pgfscope}%
\pgfpathrectangle{\pgfqpoint{1.072000in}{0.528000in}}{\pgfqpoint{3.696000in}{3.696000in}}%
\pgfusepath{clip}%
\pgfsetbuttcap%
\pgfsetroundjoin%
\definecolor{currentfill}{rgb}{0.705673,0.015556,0.150233}%
\pgfsetfillcolor{currentfill}%
\pgfsetlinewidth{0.000000pt}%
\definecolor{currentstroke}{rgb}{0.000000,0.000000,0.000000}%
\pgfsetstrokecolor{currentstroke}%
\pgfsetdash{}{0pt}%
\pgfpathmoveto{\pgfqpoint{3.234870in}{3.245483in}}%
\pgfpathlineto{\pgfqpoint{3.282039in}{3.247209in}}%
\pgfpathlineto{\pgfqpoint{3.308352in}{3.269444in}}%
\pgfpathlineto{\pgfqpoint{3.261387in}{3.286164in}}%
\pgfpathlineto{\pgfqpoint{3.234870in}{3.245483in}}%
\pgfpathclose%
\pgfusepath{fill}%
\end{pgfscope}%
\begin{pgfscope}%
\pgfpathrectangle{\pgfqpoint{1.072000in}{0.528000in}}{\pgfqpoint{3.696000in}{3.696000in}}%
\pgfusepath{clip}%
\pgfsetbuttcap%
\pgfsetroundjoin%
\definecolor{currentfill}{rgb}{0.708720,0.805721,0.981117}%
\pgfsetfillcolor{currentfill}%
\pgfsetlinewidth{0.000000pt}%
\definecolor{currentstroke}{rgb}{0.000000,0.000000,0.000000}%
\pgfsetstrokecolor{currentstroke}%
\pgfsetdash{}{0pt}%
\pgfpathmoveto{\pgfqpoint{2.378097in}{1.894644in}}%
\pgfpathlineto{\pgfqpoint{2.422946in}{1.946475in}}%
\pgfpathlineto{\pgfqpoint{2.448424in}{2.136882in}}%
\pgfpathlineto{\pgfqpoint{2.403524in}{2.074986in}}%
\pgfpathlineto{\pgfqpoint{2.378097in}{1.894644in}}%
\pgfpathclose%
\pgfusepath{fill}%
\end{pgfscope}%
\begin{pgfscope}%
\pgfpathrectangle{\pgfqpoint{1.072000in}{0.528000in}}{\pgfqpoint{3.696000in}{3.696000in}}%
\pgfusepath{clip}%
\pgfsetbuttcap%
\pgfsetroundjoin%
\definecolor{currentfill}{rgb}{0.768929,0.189213,0.197965}%
\pgfsetfillcolor{currentfill}%
\pgfsetlinewidth{0.000000pt}%
\definecolor{currentstroke}{rgb}{0.000000,0.000000,0.000000}%
\pgfsetstrokecolor{currentstroke}%
\pgfsetdash{}{0pt}%
\pgfpathmoveto{\pgfqpoint{2.773093in}{3.204916in}}%
\pgfpathlineto{\pgfqpoint{2.820008in}{3.110591in}}%
\pgfpathlineto{\pgfqpoint{2.847170in}{3.144800in}}%
\pgfpathlineto{\pgfqpoint{2.800396in}{3.232379in}}%
\pgfpathlineto{\pgfqpoint{2.773093in}{3.204916in}}%
\pgfpathclose%
\pgfusepath{fill}%
\end{pgfscope}%
\begin{pgfscope}%
\pgfpathrectangle{\pgfqpoint{1.072000in}{0.528000in}}{\pgfqpoint{3.696000in}{3.696000in}}%
\pgfusepath{clip}%
\pgfsetbuttcap%
\pgfsetroundjoin%
\definecolor{currentfill}{rgb}{0.877149,0.394645,0.311724}%
\pgfsetfillcolor{currentfill}%
\pgfsetlinewidth{0.000000pt}%
\definecolor{currentstroke}{rgb}{0.000000,0.000000,0.000000}%
\pgfsetstrokecolor{currentstroke}%
\pgfsetdash{}{0pt}%
\pgfpathmoveto{\pgfqpoint{3.469700in}{3.077129in}}%
\pgfpathlineto{\pgfqpoint{3.515611in}{2.964967in}}%
\pgfpathlineto{\pgfqpoint{3.540126in}{2.906597in}}%
\pgfpathlineto{\pgfqpoint{3.494599in}{3.027898in}}%
\pgfpathlineto{\pgfqpoint{3.469700in}{3.077129in}}%
\pgfpathclose%
\pgfusepath{fill}%
\end{pgfscope}%
\begin{pgfscope}%
\pgfpathrectangle{\pgfqpoint{1.072000in}{0.528000in}}{\pgfqpoint{3.696000in}{3.696000in}}%
\pgfusepath{clip}%
\pgfsetbuttcap%
\pgfsetroundjoin%
\definecolor{currentfill}{rgb}{0.229806,0.298718,0.753683}%
\pgfsetfillcolor{currentfill}%
\pgfsetlinewidth{0.000000pt}%
\definecolor{currentstroke}{rgb}{0.000000,0.000000,0.000000}%
\pgfsetstrokecolor{currentstroke}%
\pgfsetdash{}{0pt}%
\pgfpathmoveto{\pgfqpoint{2.354430in}{1.287351in}}%
\pgfpathlineto{\pgfqpoint{2.400494in}{1.270643in}}%
\pgfpathlineto{\pgfqpoint{2.428408in}{1.307072in}}%
\pgfpathlineto{\pgfqpoint{2.382496in}{1.317715in}}%
\pgfpathlineto{\pgfqpoint{2.354430in}{1.287351in}}%
\pgfpathclose%
\pgfusepath{fill}%
\end{pgfscope}%
\begin{pgfscope}%
\pgfpathrectangle{\pgfqpoint{1.072000in}{0.528000in}}{\pgfqpoint{3.696000in}{3.696000in}}%
\pgfusepath{clip}%
\pgfsetbuttcap%
\pgfsetroundjoin%
\definecolor{currentfill}{rgb}{0.409611,0.540759,0.935545}%
\pgfsetfillcolor{currentfill}%
\pgfsetlinewidth{0.000000pt}%
\definecolor{currentstroke}{rgb}{0.000000,0.000000,0.000000}%
\pgfsetstrokecolor{currentstroke}%
\pgfsetdash{}{0pt}%
\pgfpathmoveto{\pgfqpoint{2.436318in}{1.504072in}}%
\pgfpathlineto{\pgfqpoint{2.482110in}{1.506680in}}%
\pgfpathlineto{\pgfqpoint{2.508230in}{1.661239in}}%
\pgfpathlineto{\pgfqpoint{2.462406in}{1.652052in}}%
\pgfpathlineto{\pgfqpoint{2.436318in}{1.504072in}}%
\pgfpathclose%
\pgfusepath{fill}%
\end{pgfscope}%
\begin{pgfscope}%
\pgfpathrectangle{\pgfqpoint{1.072000in}{0.528000in}}{\pgfqpoint{3.696000in}{3.696000in}}%
\pgfusepath{clip}%
\pgfsetbuttcap%
\pgfsetroundjoin%
\definecolor{currentfill}{rgb}{0.939254,0.539581,0.423900}%
\pgfsetfillcolor{currentfill}%
\pgfsetlinewidth{0.000000pt}%
\definecolor{currentstroke}{rgb}{0.000000,0.000000,0.000000}%
\pgfsetstrokecolor{currentstroke}%
\pgfsetdash{}{0pt}%
\pgfpathmoveto{\pgfqpoint{2.426272in}{2.726347in}}%
\pgfpathlineto{\pgfqpoint{2.470383in}{2.844657in}}%
\pgfpathlineto{\pgfqpoint{2.497270in}{2.947958in}}%
\pgfpathlineto{\pgfqpoint{2.452936in}{2.834410in}}%
\pgfpathlineto{\pgfqpoint{2.426272in}{2.726347in}}%
\pgfpathclose%
\pgfusepath{fill}%
\end{pgfscope}%
\begin{pgfscope}%
\pgfpathrectangle{\pgfqpoint{1.072000in}{0.528000in}}{\pgfqpoint{3.696000in}{3.696000in}}%
\pgfusepath{clip}%
\pgfsetbuttcap%
\pgfsetroundjoin%
\definecolor{currentfill}{rgb}{0.510824,0.649397,0.985079}%
\pgfsetfillcolor{currentfill}%
\pgfsetlinewidth{0.000000pt}%
\definecolor{currentstroke}{rgb}{0.000000,0.000000,0.000000}%
\pgfsetstrokecolor{currentstroke}%
\pgfsetdash{}{0pt}%
\pgfpathmoveto{\pgfqpoint{2.416879in}{1.635198in}}%
\pgfpathlineto{\pgfqpoint{2.462406in}{1.652052in}}%
\pgfpathlineto{\pgfqpoint{2.488193in}{1.827772in}}%
\pgfpathlineto{\pgfqpoint{2.442627in}{1.802433in}}%
\pgfpathlineto{\pgfqpoint{2.416879in}{1.635198in}}%
\pgfpathclose%
\pgfusepath{fill}%
\end{pgfscope}%
\begin{pgfscope}%
\pgfpathrectangle{\pgfqpoint{1.072000in}{0.528000in}}{\pgfqpoint{3.696000in}{3.696000in}}%
\pgfusepath{clip}%
\pgfsetbuttcap%
\pgfsetroundjoin%
\definecolor{currentfill}{rgb}{0.718985,0.811993,0.977656}%
\pgfsetfillcolor{currentfill}%
\pgfsetlinewidth{0.000000pt}%
\definecolor{currentstroke}{rgb}{0.000000,0.000000,0.000000}%
\pgfsetstrokecolor{currentstroke}%
\pgfsetdash{}{0pt}%
\pgfpathmoveto{\pgfqpoint{3.561421in}{2.099425in}}%
\pgfpathlineto{\pgfqpoint{3.605725in}{1.910684in}}%
\pgfpathlineto{\pgfqpoint{3.631662in}{1.953060in}}%
\pgfpathlineto{\pgfqpoint{3.587575in}{2.144398in}}%
\pgfpathlineto{\pgfqpoint{3.561421in}{2.099425in}}%
\pgfpathclose%
\pgfusepath{fill}%
\end{pgfscope}%
\begin{pgfscope}%
\pgfpathrectangle{\pgfqpoint{1.072000in}{0.528000in}}{\pgfqpoint{3.696000in}{3.696000in}}%
\pgfusepath{clip}%
\pgfsetbuttcap%
\pgfsetroundjoin%
\definecolor{currentfill}{rgb}{0.619318,0.744121,0.998931}%
\pgfsetfillcolor{currentfill}%
\pgfsetlinewidth{0.000000pt}%
\definecolor{currentstroke}{rgb}{0.000000,0.000000,0.000000}%
\pgfsetstrokecolor{currentstroke}%
\pgfsetdash{}{0pt}%
\pgfpathmoveto{\pgfqpoint{2.397415in}{1.768750in}}%
\pgfpathlineto{\pgfqpoint{2.442627in}{1.802433in}}%
\pgfpathlineto{\pgfqpoint{2.468202in}{1.989807in}}%
\pgfpathlineto{\pgfqpoint{2.422946in}{1.946475in}}%
\pgfpathlineto{\pgfqpoint{2.397415in}{1.768750in}}%
\pgfpathclose%
\pgfusepath{fill}%
\end{pgfscope}%
\begin{pgfscope}%
\pgfpathrectangle{\pgfqpoint{1.072000in}{0.528000in}}{\pgfqpoint{3.696000in}{3.696000in}}%
\pgfusepath{clip}%
\pgfsetbuttcap%
\pgfsetroundjoin%
\definecolor{currentfill}{rgb}{0.257234,0.339661,0.789661}%
\pgfsetfillcolor{currentfill}%
\pgfsetlinewidth{0.000000pt}%
\definecolor{currentstroke}{rgb}{0.000000,0.000000,0.000000}%
\pgfsetstrokecolor{currentstroke}%
\pgfsetdash{}{0pt}%
\pgfpathmoveto{\pgfqpoint{2.428408in}{1.307072in}}%
\pgfpathlineto{\pgfqpoint{2.474456in}{1.295171in}}%
\pgfpathlineto{\pgfqpoint{2.501570in}{1.378923in}}%
\pgfpathlineto{\pgfqpoint{2.455561in}{1.386814in}}%
\pgfpathlineto{\pgfqpoint{2.428408in}{1.307072in}}%
\pgfpathclose%
\pgfusepath{fill}%
\end{pgfscope}%
\begin{pgfscope}%
\pgfpathrectangle{\pgfqpoint{1.072000in}{0.528000in}}{\pgfqpoint{3.696000in}{3.696000in}}%
\pgfusepath{clip}%
\pgfsetbuttcap%
\pgfsetroundjoin%
\definecolor{currentfill}{rgb}{0.905783,0.455186,0.355336}%
\pgfsetfillcolor{currentfill}%
\pgfsetlinewidth{0.000000pt}%
\definecolor{currentstroke}{rgb}{0.000000,0.000000,0.000000}%
\pgfsetstrokecolor{currentstroke}%
\pgfsetdash{}{0pt}%
\pgfpathmoveto{\pgfqpoint{2.452936in}{2.834410in}}%
\pgfpathlineto{\pgfqpoint{2.497270in}{2.947958in}}%
\pgfpathlineto{\pgfqpoint{2.524419in}{3.031787in}}%
\pgfpathlineto{\pgfqpoint{2.479870in}{2.924913in}}%
\pgfpathlineto{\pgfqpoint{2.452936in}{2.834410in}}%
\pgfpathclose%
\pgfusepath{fill}%
\end{pgfscope}%
\begin{pgfscope}%
\pgfpathrectangle{\pgfqpoint{1.072000in}{0.528000in}}{\pgfqpoint{3.696000in}{3.696000in}}%
\pgfusepath{clip}%
\pgfsetbuttcap%
\pgfsetroundjoin%
\definecolor{currentfill}{rgb}{0.238948,0.312365,0.765676}%
\pgfsetfillcolor{currentfill}%
\pgfsetlinewidth{0.000000pt}%
\definecolor{currentstroke}{rgb}{0.000000,0.000000,0.000000}%
\pgfsetstrokecolor{currentstroke}%
\pgfsetdash{}{0pt}%
\pgfpathmoveto{\pgfqpoint{3.538505in}{1.305454in}}%
\pgfpathlineto{\pgfqpoint{3.585565in}{1.270041in}}%
\pgfpathlineto{\pgfqpoint{3.611344in}{1.300467in}}%
\pgfpathlineto{\pgfqpoint{3.564666in}{1.353088in}}%
\pgfpathlineto{\pgfqpoint{3.538505in}{1.305454in}}%
\pgfpathclose%
\pgfusepath{fill}%
\end{pgfscope}%
\begin{pgfscope}%
\pgfpathrectangle{\pgfqpoint{1.072000in}{0.528000in}}{\pgfqpoint{3.696000in}{3.696000in}}%
\pgfusepath{clip}%
\pgfsetbuttcap%
\pgfsetroundjoin%
\definecolor{currentfill}{rgb}{0.717435,0.051118,0.158737}%
\pgfsetfillcolor{currentfill}%
\pgfsetlinewidth{0.000000pt}%
\definecolor{currentstroke}{rgb}{0.000000,0.000000,0.000000}%
\pgfsetstrokecolor{currentstroke}%
\pgfsetdash{}{0pt}%
\pgfpathmoveto{\pgfqpoint{2.726380in}{3.254423in}}%
\pgfpathlineto{\pgfqpoint{2.773093in}{3.204916in}}%
\pgfpathlineto{\pgfqpoint{2.800396in}{3.232379in}}%
\pgfpathlineto{\pgfqpoint{2.753796in}{3.277022in}}%
\pgfpathlineto{\pgfqpoint{2.726380in}{3.254423in}}%
\pgfpathclose%
\pgfusepath{fill}%
\end{pgfscope}%
\begin{pgfscope}%
\pgfpathrectangle{\pgfqpoint{1.072000in}{0.528000in}}{\pgfqpoint{3.696000in}{3.696000in}}%
\pgfusepath{clip}%
\pgfsetbuttcap%
\pgfsetroundjoin%
\definecolor{currentfill}{rgb}{0.852378,0.346492,0.280346}%
\pgfsetfillcolor{currentfill}%
\pgfsetlinewidth{0.000000pt}%
\definecolor{currentstroke}{rgb}{0.000000,0.000000,0.000000}%
\pgfsetstrokecolor{currentstroke}%
\pgfsetdash{}{0pt}%
\pgfpathmoveto{\pgfqpoint{2.792630in}{3.100915in}}%
\pgfpathlineto{\pgfqpoint{2.839692in}{2.976764in}}%
\pgfpathlineto{\pgfqpoint{2.866932in}{2.976997in}}%
\pgfpathlineto{\pgfqpoint{2.820008in}{3.110591in}}%
\pgfpathlineto{\pgfqpoint{2.792630in}{3.100915in}}%
\pgfpathclose%
\pgfusepath{fill}%
\end{pgfscope}%
\begin{pgfscope}%
\pgfpathrectangle{\pgfqpoint{1.072000in}{0.528000in}}{\pgfqpoint{3.696000in}{3.696000in}}%
\pgfusepath{clip}%
\pgfsetbuttcap%
\pgfsetroundjoin%
\definecolor{currentfill}{rgb}{0.494638,0.633022,0.978983}%
\pgfsetfillcolor{currentfill}%
\pgfsetlinewidth{0.000000pt}%
\definecolor{currentstroke}{rgb}{0.000000,0.000000,0.000000}%
\pgfsetstrokecolor{currentstroke}%
\pgfsetdash{}{0pt}%
\pgfpathmoveto{\pgfqpoint{1.903453in}{1.798446in}}%
\pgfpathlineto{\pgfqpoint{1.951501in}{1.704851in}}%
\pgfpathlineto{\pgfqpoint{1.984098in}{1.607489in}}%
\pgfpathlineto{\pgfqpoint{1.936624in}{1.691725in}}%
\pgfpathlineto{\pgfqpoint{1.903453in}{1.798446in}}%
\pgfpathclose%
\pgfusepath{fill}%
\end{pgfscope}%
\begin{pgfscope}%
\pgfpathrectangle{\pgfqpoint{1.072000in}{0.528000in}}{\pgfqpoint{3.696000in}{3.696000in}}%
\pgfusepath{clip}%
\pgfsetbuttcap%
\pgfsetroundjoin%
\definecolor{currentfill}{rgb}{0.323718,0.433158,0.864722}%
\pgfsetfillcolor{currentfill}%
\pgfsetlinewidth{0.000000pt}%
\definecolor{currentstroke}{rgb}{0.000000,0.000000,0.000000}%
\pgfsetstrokecolor{currentstroke}%
\pgfsetdash{}{0pt}%
\pgfpathmoveto{\pgfqpoint{2.455561in}{1.386814in}}%
\pgfpathlineto{\pgfqpoint{2.501570in}{1.378923in}}%
\pgfpathlineto{\pgfqpoint{2.528140in}{1.502908in}}%
\pgfpathlineto{\pgfqpoint{2.482110in}{1.506680in}}%
\pgfpathlineto{\pgfqpoint{2.455561in}{1.386814in}}%
\pgfpathclose%
\pgfusepath{fill}%
\end{pgfscope}%
\begin{pgfscope}%
\pgfpathrectangle{\pgfqpoint{1.072000in}{0.528000in}}{\pgfqpoint{3.696000in}{3.696000in}}%
\pgfusepath{clip}%
\pgfsetbuttcap%
\pgfsetroundjoin%
\definecolor{currentfill}{rgb}{0.275827,0.366717,0.812553}%
\pgfsetfillcolor{currentfill}%
\pgfsetlinewidth{0.000000pt}%
\definecolor{currentstroke}{rgb}{0.000000,0.000000,0.000000}%
\pgfsetstrokecolor{currentstroke}%
\pgfsetdash{}{0pt}%
\pgfpathmoveto{\pgfqpoint{3.492032in}{1.375215in}}%
\pgfpathlineto{\pgfqpoint{3.538505in}{1.305454in}}%
\pgfpathlineto{\pgfqpoint{3.564666in}{1.353088in}}%
\pgfpathlineto{\pgfqpoint{3.518560in}{1.440606in}}%
\pgfpathlineto{\pgfqpoint{3.492032in}{1.375215in}}%
\pgfpathclose%
\pgfusepath{fill}%
\end{pgfscope}%
\begin{pgfscope}%
\pgfpathrectangle{\pgfqpoint{1.072000in}{0.528000in}}{\pgfqpoint{3.696000in}{3.696000in}}%
\pgfusepath{clip}%
\pgfsetbuttcap%
\pgfsetroundjoin%
\definecolor{currentfill}{rgb}{0.430507,0.564883,0.948889}%
\pgfsetfillcolor{currentfill}%
\pgfsetlinewidth{0.000000pt}%
\definecolor{currentstroke}{rgb}{0.000000,0.000000,0.000000}%
\pgfsetstrokecolor{currentstroke}%
\pgfsetdash{}{0pt}%
\pgfpathmoveto{\pgfqpoint{3.499653in}{1.641031in}}%
\pgfpathlineto{\pgfqpoint{3.545076in}{1.507829in}}%
\pgfpathlineto{\pgfqpoint{3.571531in}{1.573835in}}%
\pgfpathlineto{\pgfqpoint{3.526421in}{1.720081in}}%
\pgfpathlineto{\pgfqpoint{3.499653in}{1.641031in}}%
\pgfpathclose%
\pgfusepath{fill}%
\end{pgfscope}%
\begin{pgfscope}%
\pgfpathrectangle{\pgfqpoint{1.072000in}{0.528000in}}{\pgfqpoint{3.696000in}{3.696000in}}%
\pgfusepath{clip}%
\pgfsetbuttcap%
\pgfsetroundjoin%
\definecolor{currentfill}{rgb}{0.252663,0.332837,0.783665}%
\pgfsetfillcolor{currentfill}%
\pgfsetlinewidth{0.000000pt}%
\definecolor{currentstroke}{rgb}{0.000000,0.000000,0.000000}%
\pgfsetstrokecolor{currentstroke}%
\pgfsetdash{}{0pt}%
\pgfpathmoveto{\pgfqpoint{2.232860in}{1.357063in}}%
\pgfpathlineto{\pgfqpoint{2.279154in}{1.326353in}}%
\pgfpathlineto{\pgfqpoint{2.308430in}{1.306011in}}%
\pgfpathlineto{\pgfqpoint{2.262459in}{1.327791in}}%
\pgfpathlineto{\pgfqpoint{2.232860in}{1.357063in}}%
\pgfpathclose%
\pgfusepath{fill}%
\end{pgfscope}%
\begin{pgfscope}%
\pgfpathrectangle{\pgfqpoint{1.072000in}{0.528000in}}{\pgfqpoint{3.696000in}{3.696000in}}%
\pgfusepath{clip}%
\pgfsetbuttcap%
\pgfsetroundjoin%
\definecolor{currentfill}{rgb}{0.238948,0.312365,0.765676}%
\pgfsetfillcolor{currentfill}%
\pgfsetlinewidth{0.000000pt}%
\definecolor{currentstroke}{rgb}{0.000000,0.000000,0.000000}%
\pgfsetstrokecolor{currentstroke}%
\pgfsetdash{}{0pt}%
\pgfpathmoveto{\pgfqpoint{3.658805in}{1.285643in}}%
\pgfpathlineto{\pgfqpoint{3.707231in}{1.309607in}}%
\pgfpathlineto{\pgfqpoint{3.732311in}{1.317497in}}%
\pgfpathlineto{\pgfqpoint{3.684231in}{1.306780in}}%
\pgfpathlineto{\pgfqpoint{3.658805in}{1.285643in}}%
\pgfpathclose%
\pgfusepath{fill}%
\end{pgfscope}%
\begin{pgfscope}%
\pgfpathrectangle{\pgfqpoint{1.072000in}{0.528000in}}{\pgfqpoint{3.696000in}{3.696000in}}%
\pgfusepath{clip}%
\pgfsetbuttcap%
\pgfsetroundjoin%
\definecolor{currentfill}{rgb}{0.871493,0.862309,0.857016}%
\pgfsetfillcolor{currentfill}%
\pgfsetlinewidth{0.000000pt}%
\definecolor{currentstroke}{rgb}{0.000000,0.000000,0.000000}%
\pgfsetstrokecolor{currentstroke}%
\pgfsetdash{}{0pt}%
\pgfpathmoveto{\pgfqpoint{3.569239in}{2.373900in}}%
\pgfpathlineto{\pgfqpoint{3.613394in}{2.175333in}}%
\pgfpathlineto{\pgfqpoint{3.638846in}{2.192096in}}%
\pgfpathlineto{\pgfqpoint{3.594886in}{2.389722in}}%
\pgfpathlineto{\pgfqpoint{3.569239in}{2.373900in}}%
\pgfpathclose%
\pgfusepath{fill}%
\end{pgfscope}%
\begin{pgfscope}%
\pgfpathrectangle{\pgfqpoint{1.072000in}{0.528000in}}{\pgfqpoint{3.696000in}{3.696000in}}%
\pgfusepath{clip}%
\pgfsetbuttcap%
\pgfsetroundjoin%
\definecolor{currentfill}{rgb}{0.724041,0.814910,0.975651}%
\pgfsetfillcolor{currentfill}%
\pgfsetlinewidth{0.000000pt}%
\definecolor{currentstroke}{rgb}{0.000000,0.000000,0.000000}%
\pgfsetstrokecolor{currentstroke}%
\pgfsetdash{}{0pt}%
\pgfpathmoveto{\pgfqpoint{1.741682in}{2.147169in}}%
\pgfpathlineto{\pgfqpoint{1.791184in}{2.017934in}}%
\pgfpathlineto{\pgfqpoint{1.825061in}{1.907614in}}%
\pgfpathlineto{\pgfqpoint{1.775995in}{2.033905in}}%
\pgfpathlineto{\pgfqpoint{1.741682in}{2.147169in}}%
\pgfpathclose%
\pgfusepath{fill}%
\end{pgfscope}%
\begin{pgfscope}%
\pgfpathrectangle{\pgfqpoint{1.072000in}{0.528000in}}{\pgfqpoint{3.696000in}{3.696000in}}%
\pgfusepath{clip}%
\pgfsetbuttcap%
\pgfsetroundjoin%
\definecolor{currentfill}{rgb}{0.705673,0.015556,0.150233}%
\pgfsetfillcolor{currentfill}%
\pgfsetlinewidth{0.000000pt}%
\definecolor{currentstroke}{rgb}{0.000000,0.000000,0.000000}%
\pgfsetstrokecolor{currentstroke}%
\pgfsetdash{}{0pt}%
\pgfpathmoveto{\pgfqpoint{3.282039in}{3.247209in}}%
\pgfpathlineto{\pgfqpoint{3.329258in}{3.239188in}}%
\pgfpathlineto{\pgfqpoint{3.355285in}{3.240840in}}%
\pgfpathlineto{\pgfqpoint{3.308352in}{3.269444in}}%
\pgfpathlineto{\pgfqpoint{3.282039in}{3.247209in}}%
\pgfpathclose%
\pgfusepath{fill}%
\end{pgfscope}%
\begin{pgfscope}%
\pgfpathrectangle{\pgfqpoint{1.072000in}{0.528000in}}{\pgfqpoint{3.696000in}{3.696000in}}%
\pgfusepath{clip}%
\pgfsetbuttcap%
\pgfsetroundjoin%
\definecolor{currentfill}{rgb}{0.902659,0.447939,0.349721}%
\pgfsetfillcolor{currentfill}%
\pgfsetlinewidth{0.000000pt}%
\definecolor{currentstroke}{rgb}{0.000000,0.000000,0.000000}%
\pgfsetstrokecolor{currentstroke}%
\pgfsetdash{}{0pt}%
\pgfpathmoveto{\pgfqpoint{3.126840in}{2.821017in}}%
\pgfpathlineto{\pgfqpoint{3.174482in}{2.944108in}}%
\pgfpathlineto{\pgfqpoint{3.201426in}{3.022061in}}%
\pgfpathlineto{\pgfqpoint{3.153851in}{2.933157in}}%
\pgfpathlineto{\pgfqpoint{3.126840in}{2.821017in}}%
\pgfpathclose%
\pgfusepath{fill}%
\end{pgfscope}%
\begin{pgfscope}%
\pgfpathrectangle{\pgfqpoint{1.072000in}{0.528000in}}{\pgfqpoint{3.696000in}{3.696000in}}%
\pgfusepath{clip}%
\pgfsetbuttcap%
\pgfsetroundjoin%
\definecolor{currentfill}{rgb}{0.729196,0.086679,0.167240}%
\pgfsetfillcolor{currentfill}%
\pgfsetlinewidth{0.000000pt}%
\definecolor{currentstroke}{rgb}{0.000000,0.000000,0.000000}%
\pgfsetstrokecolor{currentstroke}%
\pgfsetdash{}{0pt}%
\pgfpathmoveto{\pgfqpoint{3.208015in}{3.166182in}}%
\pgfpathlineto{\pgfqpoint{3.255351in}{3.191413in}}%
\pgfpathlineto{\pgfqpoint{3.282039in}{3.247209in}}%
\pgfpathlineto{\pgfqpoint{3.234870in}{3.245483in}}%
\pgfpathlineto{\pgfqpoint{3.208015in}{3.166182in}}%
\pgfpathclose%
\pgfusepath{fill}%
\end{pgfscope}%
\begin{pgfscope}%
\pgfpathrectangle{\pgfqpoint{1.072000in}{0.528000in}}{\pgfqpoint{3.696000in}{3.696000in}}%
\pgfusepath{clip}%
\pgfsetbuttcap%
\pgfsetroundjoin%
\definecolor{currentfill}{rgb}{0.388852,0.516298,0.921373}%
\pgfsetfillcolor{currentfill}%
\pgfsetlinewidth{0.000000pt}%
\definecolor{currentstroke}{rgb}{0.000000,0.000000,0.000000}%
\pgfsetstrokecolor{currentstroke}%
\pgfsetdash{}{0pt}%
\pgfpathmoveto{\pgfqpoint{3.472810in}{1.558390in}}%
\pgfpathlineto{\pgfqpoint{3.518560in}{1.440606in}}%
\pgfpathlineto{\pgfqpoint{3.545076in}{1.507829in}}%
\pgfpathlineto{\pgfqpoint{3.499653in}{1.641031in}}%
\pgfpathlineto{\pgfqpoint{3.472810in}{1.558390in}}%
\pgfpathclose%
\pgfusepath{fill}%
\end{pgfscope}%
\begin{pgfscope}%
\pgfpathrectangle{\pgfqpoint{1.072000in}{0.528000in}}{\pgfqpoint{3.696000in}{3.696000in}}%
\pgfusepath{clip}%
\pgfsetbuttcap%
\pgfsetroundjoin%
\definecolor{currentfill}{rgb}{0.229806,0.298718,0.753683}%
\pgfsetfillcolor{currentfill}%
\pgfsetlinewidth{0.000000pt}%
\definecolor{currentstroke}{rgb}{0.000000,0.000000,0.000000}%
\pgfsetstrokecolor{currentstroke}%
\pgfsetdash{}{0pt}%
\pgfpathmoveto{\pgfqpoint{3.585565in}{1.270041in}}%
\pgfpathlineto{\pgfqpoint{3.633403in}{1.271231in}}%
\pgfpathlineto{\pgfqpoint{3.658805in}{1.285643in}}%
\pgfpathlineto{\pgfqpoint{3.611344in}{1.300467in}}%
\pgfpathlineto{\pgfqpoint{3.585565in}{1.270041in}}%
\pgfpathclose%
\pgfusepath{fill}%
\end{pgfscope}%
\begin{pgfscope}%
\pgfpathrectangle{\pgfqpoint{1.072000in}{0.528000in}}{\pgfqpoint{3.696000in}{3.696000in}}%
\pgfusepath{clip}%
\pgfsetbuttcap%
\pgfsetroundjoin%
\definecolor{currentfill}{rgb}{0.746838,0.140021,0.179996}%
\pgfsetfillcolor{currentfill}%
\pgfsetlinewidth{0.000000pt}%
\definecolor{currentstroke}{rgb}{0.000000,0.000000,0.000000}%
\pgfsetstrokecolor{currentstroke}%
\pgfsetdash{}{0pt}%
\pgfpathmoveto{\pgfqpoint{2.579237in}{3.148114in}}%
\pgfpathlineto{\pgfqpoint{2.624890in}{3.201803in}}%
\pgfpathlineto{\pgfqpoint{2.652501in}{3.232415in}}%
\pgfpathlineto{\pgfqpoint{2.606774in}{3.185689in}}%
\pgfpathlineto{\pgfqpoint{2.579237in}{3.148114in}}%
\pgfpathclose%
\pgfusepath{fill}%
\end{pgfscope}%
\begin{pgfscope}%
\pgfpathrectangle{\pgfqpoint{1.072000in}{0.528000in}}{\pgfqpoint{3.696000in}{3.696000in}}%
\pgfusepath{clip}%
\pgfsetbuttcap%
\pgfsetroundjoin%
\definecolor{currentfill}{rgb}{0.229806,0.298718,0.753683}%
\pgfsetfillcolor{currentfill}%
\pgfsetlinewidth{0.000000pt}%
\definecolor{currentstroke}{rgb}{0.000000,0.000000,0.000000}%
\pgfsetstrokecolor{currentstroke}%
\pgfsetdash{}{0pt}%
\pgfpathmoveto{\pgfqpoint{2.400494in}{1.270643in}}%
\pgfpathlineto{\pgfqpoint{2.446643in}{1.255014in}}%
\pgfpathlineto{\pgfqpoint{2.474456in}{1.295171in}}%
\pgfpathlineto{\pgfqpoint{2.428408in}{1.307072in}}%
\pgfpathlineto{\pgfqpoint{2.400494in}{1.270643in}}%
\pgfpathclose%
\pgfusepath{fill}%
\end{pgfscope}%
\begin{pgfscope}%
\pgfpathrectangle{\pgfqpoint{1.072000in}{0.528000in}}{\pgfqpoint{3.696000in}{3.696000in}}%
\pgfusepath{clip}%
\pgfsetbuttcap%
\pgfsetroundjoin%
\definecolor{currentfill}{rgb}{0.852378,0.346492,0.280346}%
\pgfsetfillcolor{currentfill}%
\pgfsetlinewidth{0.000000pt}%
\definecolor{currentstroke}{rgb}{0.000000,0.000000,0.000000}%
\pgfsetstrokecolor{currentstroke}%
\pgfsetdash{}{0pt}%
\pgfpathmoveto{\pgfqpoint{3.153851in}{2.933157in}}%
\pgfpathlineto{\pgfqpoint{3.201426in}{3.022061in}}%
\pgfpathlineto{\pgfqpoint{3.228429in}{3.111391in}}%
\pgfpathlineto{\pgfqpoint{3.180959in}{3.056971in}}%
\pgfpathlineto{\pgfqpoint{3.153851in}{2.933157in}}%
\pgfpathclose%
\pgfusepath{fill}%
\end{pgfscope}%
\begin{pgfscope}%
\pgfpathrectangle{\pgfqpoint{1.072000in}{0.528000in}}{\pgfqpoint{3.696000in}{3.696000in}}%
\pgfusepath{clip}%
\pgfsetbuttcap%
\pgfsetroundjoin%
\definecolor{currentfill}{rgb}{0.902659,0.447939,0.349721}%
\pgfsetfillcolor{currentfill}%
\pgfsetlinewidth{0.000000pt}%
\definecolor{currentstroke}{rgb}{0.000000,0.000000,0.000000}%
\pgfsetstrokecolor{currentstroke}%
\pgfsetdash{}{0pt}%
\pgfpathmoveto{\pgfqpoint{2.812156in}{3.010150in}}%
\pgfpathlineto{\pgfqpoint{2.859225in}{2.895339in}}%
\pgfpathlineto{\pgfqpoint{2.886626in}{2.827977in}}%
\pgfpathlineto{\pgfqpoint{2.839692in}{2.976764in}}%
\pgfpathlineto{\pgfqpoint{2.812156in}{3.010150in}}%
\pgfpathclose%
\pgfusepath{fill}%
\end{pgfscope}%
\begin{pgfscope}%
\pgfpathrectangle{\pgfqpoint{1.072000in}{0.528000in}}{\pgfqpoint{3.696000in}{3.696000in}}%
\pgfusepath{clip}%
\pgfsetbuttcap%
\pgfsetroundjoin%
\definecolor{currentfill}{rgb}{0.810616,0.268797,0.235428}%
\pgfsetfillcolor{currentfill}%
\pgfsetlinewidth{0.000000pt}%
\definecolor{currentstroke}{rgb}{0.000000,0.000000,0.000000}%
\pgfsetstrokecolor{currentstroke}%
\pgfsetdash{}{0pt}%
\pgfpathmoveto{\pgfqpoint{3.423248in}{3.159403in}}%
\pgfpathlineto{\pgfqpoint{3.469700in}{3.077129in}}%
\pgfpathlineto{\pgfqpoint{3.494599in}{3.027898in}}%
\pgfpathlineto{\pgfqpoint{3.448544in}{3.123737in}}%
\pgfpathlineto{\pgfqpoint{3.423248in}{3.159403in}}%
\pgfpathclose%
\pgfusepath{fill}%
\end{pgfscope}%
\begin{pgfscope}%
\pgfpathrectangle{\pgfqpoint{1.072000in}{0.528000in}}{\pgfqpoint{3.696000in}{3.696000in}}%
\pgfusepath{clip}%
\pgfsetbuttcap%
\pgfsetroundjoin%
\definecolor{currentfill}{rgb}{0.785153,0.220851,0.211673}%
\pgfsetfillcolor{currentfill}%
\pgfsetlinewidth{0.000000pt}%
\definecolor{currentstroke}{rgb}{0.000000,0.000000,0.000000}%
\pgfsetstrokecolor{currentstroke}%
\pgfsetdash{}{0pt}%
\pgfpathmoveto{\pgfqpoint{3.180959in}{3.056971in}}%
\pgfpathlineto{\pgfqpoint{3.228429in}{3.111391in}}%
\pgfpathlineto{\pgfqpoint{3.255351in}{3.191413in}}%
\pgfpathlineto{\pgfqpoint{3.208015in}{3.166182in}}%
\pgfpathlineto{\pgfqpoint{3.180959in}{3.056971in}}%
\pgfpathclose%
\pgfusepath{fill}%
\end{pgfscope}%
\begin{pgfscope}%
\pgfpathrectangle{\pgfqpoint{1.072000in}{0.528000in}}{\pgfqpoint{3.696000in}{3.696000in}}%
\pgfusepath{clip}%
\pgfsetbuttcap%
\pgfsetroundjoin%
\definecolor{currentfill}{rgb}{0.243520,0.319189,0.771672}%
\pgfsetfillcolor{currentfill}%
\pgfsetlinewidth{0.000000pt}%
\definecolor{currentstroke}{rgb}{0.000000,0.000000,0.000000}%
\pgfsetstrokecolor{currentstroke}%
\pgfsetdash{}{0pt}%
\pgfpathmoveto{\pgfqpoint{2.279154in}{1.326353in}}%
\pgfpathlineto{\pgfqpoint{2.325420in}{1.300429in}}%
\pgfpathlineto{\pgfqpoint{2.354430in}{1.287351in}}%
\pgfpathlineto{\pgfqpoint{2.308430in}{1.306011in}}%
\pgfpathlineto{\pgfqpoint{2.279154in}{1.326353in}}%
\pgfpathclose%
\pgfusepath{fill}%
\end{pgfscope}%
\begin{pgfscope}%
\pgfpathrectangle{\pgfqpoint{1.072000in}{0.528000in}}{\pgfqpoint{3.696000in}{3.696000in}}%
\pgfusepath{clip}%
\pgfsetbuttcap%
\pgfsetroundjoin%
\definecolor{currentfill}{rgb}{0.698454,0.799450,0.984577}%
\pgfsetfillcolor{currentfill}%
\pgfsetlinewidth{0.000000pt}%
\definecolor{currentstroke}{rgb}{0.000000,0.000000,0.000000}%
\pgfsetstrokecolor{currentstroke}%
\pgfsetdash{}{0pt}%
\pgfpathmoveto{\pgfqpoint{3.534972in}{2.040951in}}%
\pgfpathlineto{\pgfqpoint{3.579509in}{1.856869in}}%
\pgfpathlineto{\pgfqpoint{3.605725in}{1.910684in}}%
\pgfpathlineto{\pgfqpoint{3.561421in}{2.099425in}}%
\pgfpathlineto{\pgfqpoint{3.534972in}{2.040951in}}%
\pgfpathclose%
\pgfusepath{fill}%
\end{pgfscope}%
\begin{pgfscope}%
\pgfpathrectangle{\pgfqpoint{1.072000in}{0.528000in}}{\pgfqpoint{3.696000in}{3.696000in}}%
\pgfusepath{clip}%
\pgfsetbuttcap%
\pgfsetroundjoin%
\definecolor{currentfill}{rgb}{0.261805,0.346484,0.795658}%
\pgfsetfillcolor{currentfill}%
\pgfsetlinewidth{0.000000pt}%
\definecolor{currentstroke}{rgb}{0.000000,0.000000,0.000000}%
\pgfsetstrokecolor{currentstroke}%
\pgfsetdash{}{0pt}%
\pgfpathmoveto{\pgfqpoint{2.474456in}{1.295171in}}%
\pgfpathlineto{\pgfqpoint{2.520645in}{1.281379in}}%
\pgfpathlineto{\pgfqpoint{2.547762in}{1.366546in}}%
\pgfpathlineto{\pgfqpoint{2.501570in}{1.378923in}}%
\pgfpathlineto{\pgfqpoint{2.474456in}{1.295171in}}%
\pgfpathclose%
\pgfusepath{fill}%
\end{pgfscope}%
\begin{pgfscope}%
\pgfpathrectangle{\pgfqpoint{1.072000in}{0.528000in}}{\pgfqpoint{3.696000in}{3.696000in}}%
\pgfusepath{clip}%
\pgfsetbuttcap%
\pgfsetroundjoin%
\definecolor{currentfill}{rgb}{0.705673,0.015556,0.150233}%
\pgfsetfillcolor{currentfill}%
\pgfsetlinewidth{0.000000pt}%
\definecolor{currentstroke}{rgb}{0.000000,0.000000,0.000000}%
\pgfsetstrokecolor{currentstroke}%
\pgfsetdash{}{0pt}%
\pgfpathmoveto{\pgfqpoint{2.652501in}{3.232415in}}%
\pgfpathlineto{\pgfqpoint{2.698839in}{3.233643in}}%
\pgfpathlineto{\pgfqpoint{2.726380in}{3.254423in}}%
\pgfpathlineto{\pgfqpoint{2.680065in}{3.256646in}}%
\pgfpathlineto{\pgfqpoint{2.652501in}{3.232415in}}%
\pgfpathclose%
\pgfusepath{fill}%
\end{pgfscope}%
\begin{pgfscope}%
\pgfpathrectangle{\pgfqpoint{1.072000in}{0.528000in}}{\pgfqpoint{3.696000in}{3.696000in}}%
\pgfusepath{clip}%
\pgfsetbuttcap%
\pgfsetroundjoin%
\definecolor{currentfill}{rgb}{0.891817,0.851973,0.829085}%
\pgfsetfillcolor{currentfill}%
\pgfsetlinewidth{0.000000pt}%
\definecolor{currentstroke}{rgb}{0.000000,0.000000,0.000000}%
\pgfsetstrokecolor{currentstroke}%
\pgfsetdash{}{0pt}%
\pgfpathmoveto{\pgfqpoint{2.384500in}{2.181326in}}%
\pgfpathlineto{\pgfqpoint{2.429001in}{2.261386in}}%
\pgfpathlineto{\pgfqpoint{2.454671in}{2.445119in}}%
\pgfpathlineto{\pgfqpoint{2.410045in}{2.358545in}}%
\pgfpathlineto{\pgfqpoint{2.384500in}{2.181326in}}%
\pgfpathclose%
\pgfusepath{fill}%
\end{pgfscope}%
\begin{pgfscope}%
\pgfpathrectangle{\pgfqpoint{1.072000in}{0.528000in}}{\pgfqpoint{3.696000in}{3.696000in}}%
\pgfusepath{clip}%
\pgfsetbuttcap%
\pgfsetroundjoin%
\definecolor{currentfill}{rgb}{0.570616,0.704109,0.997195}%
\pgfsetfillcolor{currentfill}%
\pgfsetlinewidth{0.000000pt}%
\definecolor{currentstroke}{rgb}{0.000000,0.000000,0.000000}%
\pgfsetstrokecolor{currentstroke}%
\pgfsetdash{}{0pt}%
\pgfpathmoveto{\pgfqpoint{1.854949in}{1.902746in}}%
\pgfpathlineto{\pgfqpoint{1.903453in}{1.798446in}}%
\pgfpathlineto{\pgfqpoint{1.936624in}{1.691725in}}%
\pgfpathlineto{\pgfqpoint{1.888665in}{1.788637in}}%
\pgfpathlineto{\pgfqpoint{1.854949in}{1.902746in}}%
\pgfpathclose%
\pgfusepath{fill}%
\end{pgfscope}%
\begin{pgfscope}%
\pgfpathrectangle{\pgfqpoint{1.072000in}{0.528000in}}{\pgfqpoint{3.696000in}{3.696000in}}%
\pgfusepath{clip}%
\pgfsetbuttcap%
\pgfsetroundjoin%
\definecolor{currentfill}{rgb}{0.929357,0.512254,0.400673}%
\pgfsetfillcolor{currentfill}%
\pgfsetlinewidth{0.000000pt}%
\definecolor{currentstroke}{rgb}{0.000000,0.000000,0.000000}%
\pgfsetstrokecolor{currentstroke}%
\pgfsetdash{}{0pt}%
\pgfpathmoveto{\pgfqpoint{3.100050in}{2.760212in}}%
\pgfpathlineto{\pgfqpoint{3.147684in}{2.899623in}}%
\pgfpathlineto{\pgfqpoint{3.174482in}{2.944108in}}%
\pgfpathlineto{\pgfqpoint{3.126840in}{2.821017in}}%
\pgfpathlineto{\pgfqpoint{3.100050in}{2.760212in}}%
\pgfpathclose%
\pgfusepath{fill}%
\end{pgfscope}%
\begin{pgfscope}%
\pgfpathrectangle{\pgfqpoint{1.072000in}{0.528000in}}{\pgfqpoint{3.696000in}{3.696000in}}%
\pgfusepath{clip}%
\pgfsetbuttcap%
\pgfsetroundjoin%
\definecolor{currentfill}{rgb}{0.965899,0.740142,0.637058}%
\pgfsetfillcolor{currentfill}%
\pgfsetlinewidth{0.000000pt}%
\definecolor{currentstroke}{rgb}{0.000000,0.000000,0.000000}%
\pgfsetstrokecolor{currentstroke}%
\pgfsetdash{}{0pt}%
\pgfpathmoveto{\pgfqpoint{2.391637in}{2.425888in}}%
\pgfpathlineto{\pgfqpoint{2.435834in}{2.528892in}}%
\pgfpathlineto{\pgfqpoint{2.461945in}{2.685758in}}%
\pgfpathlineto{\pgfqpoint{2.417563in}{2.581064in}}%
\pgfpathlineto{\pgfqpoint{2.391637in}{2.425888in}}%
\pgfpathclose%
\pgfusepath{fill}%
\end{pgfscope}%
\begin{pgfscope}%
\pgfpathrectangle{\pgfqpoint{1.072000in}{0.528000in}}{\pgfqpoint{3.696000in}{3.696000in}}%
\pgfusepath{clip}%
\pgfsetbuttcap%
\pgfsetroundjoin%
\definecolor{currentfill}{rgb}{0.343278,0.459354,0.884122}%
\pgfsetfillcolor{currentfill}%
\pgfsetlinewidth{0.000000pt}%
\definecolor{currentstroke}{rgb}{0.000000,0.000000,0.000000}%
\pgfsetstrokecolor{currentstroke}%
\pgfsetdash{}{0pt}%
\pgfpathmoveto{\pgfqpoint{3.445946in}{1.475409in}}%
\pgfpathlineto{\pgfqpoint{3.492032in}{1.375215in}}%
\pgfpathlineto{\pgfqpoint{3.518560in}{1.440606in}}%
\pgfpathlineto{\pgfqpoint{3.472810in}{1.558390in}}%
\pgfpathlineto{\pgfqpoint{3.445946in}{1.475409in}}%
\pgfpathclose%
\pgfusepath{fill}%
\end{pgfscope}%
\begin{pgfscope}%
\pgfpathrectangle{\pgfqpoint{1.072000in}{0.528000in}}{\pgfqpoint{3.696000in}{3.696000in}}%
\pgfusepath{clip}%
\pgfsetbuttcap%
\pgfsetroundjoin%
\definecolor{currentfill}{rgb}{0.252663,0.332837,0.783665}%
\pgfsetfillcolor{currentfill}%
\pgfsetlinewidth{0.000000pt}%
\definecolor{currentstroke}{rgb}{0.000000,0.000000,0.000000}%
\pgfsetstrokecolor{currentstroke}%
\pgfsetdash{}{0pt}%
\pgfpathmoveto{\pgfqpoint{3.465539in}{1.314976in}}%
\pgfpathlineto{\pgfqpoint{3.512384in}{1.264520in}}%
\pgfpathlineto{\pgfqpoint{3.538505in}{1.305454in}}%
\pgfpathlineto{\pgfqpoint{3.492032in}{1.375215in}}%
\pgfpathlineto{\pgfqpoint{3.465539in}{1.314976in}}%
\pgfpathclose%
\pgfusepath{fill}%
\end{pgfscope}%
\begin{pgfscope}%
\pgfpathrectangle{\pgfqpoint{1.072000in}{0.528000in}}{\pgfqpoint{3.696000in}{3.696000in}}%
\pgfusepath{clip}%
\pgfsetbuttcap%
\pgfsetroundjoin%
\definecolor{currentfill}{rgb}{0.717435,0.051118,0.158737}%
\pgfsetfillcolor{currentfill}%
\pgfsetlinewidth{0.000000pt}%
\definecolor{currentstroke}{rgb}{0.000000,0.000000,0.000000}%
\pgfsetstrokecolor{currentstroke}%
\pgfsetdash{}{0pt}%
\pgfpathmoveto{\pgfqpoint{3.329258in}{3.239188in}}%
\pgfpathlineto{\pgfqpoint{3.376382in}{3.212193in}}%
\pgfpathlineto{\pgfqpoint{3.402063in}{3.193893in}}%
\pgfpathlineto{\pgfqpoint{3.355285in}{3.240840in}}%
\pgfpathlineto{\pgfqpoint{3.329258in}{3.239188in}}%
\pgfpathclose%
\pgfusepath{fill}%
\end{pgfscope}%
\begin{pgfscope}%
\pgfpathrectangle{\pgfqpoint{1.072000in}{0.528000in}}{\pgfqpoint{3.696000in}{3.696000in}}%
\pgfusepath{clip}%
\pgfsetbuttcap%
\pgfsetroundjoin%
\definecolor{currentfill}{rgb}{0.328604,0.439712,0.869587}%
\pgfsetfillcolor{currentfill}%
\pgfsetlinewidth{0.000000pt}%
\definecolor{currentstroke}{rgb}{0.000000,0.000000,0.000000}%
\pgfsetstrokecolor{currentstroke}%
\pgfsetdash{}{0pt}%
\pgfpathmoveto{\pgfqpoint{2.501570in}{1.378923in}}%
\pgfpathlineto{\pgfqpoint{2.547762in}{1.366546in}}%
\pgfpathlineto{\pgfqpoint{2.574391in}{1.491997in}}%
\pgfpathlineto{\pgfqpoint{2.528140in}{1.502908in}}%
\pgfpathlineto{\pgfqpoint{2.501570in}{1.378923in}}%
\pgfpathclose%
\pgfusepath{fill}%
\end{pgfscope}%
\begin{pgfscope}%
\pgfpathrectangle{\pgfqpoint{1.072000in}{0.528000in}}{\pgfqpoint{3.696000in}{3.696000in}}%
\pgfusepath{clip}%
\pgfsetbuttcap%
\pgfsetroundjoin%
\definecolor{currentfill}{rgb}{0.425199,0.559058,0.946061}%
\pgfsetfillcolor{currentfill}%
\pgfsetlinewidth{0.000000pt}%
\definecolor{currentstroke}{rgb}{0.000000,0.000000,0.000000}%
\pgfsetstrokecolor{currentstroke}%
\pgfsetdash{}{0pt}%
\pgfpathmoveto{\pgfqpoint{2.482110in}{1.506680in}}%
\pgfpathlineto{\pgfqpoint{2.528140in}{1.502908in}}%
\pgfpathlineto{\pgfqpoint{2.554336in}{1.661487in}}%
\pgfpathlineto{\pgfqpoint{2.508230in}{1.661239in}}%
\pgfpathlineto{\pgfqpoint{2.482110in}{1.506680in}}%
\pgfpathclose%
\pgfusepath{fill}%
\end{pgfscope}%
\begin{pgfscope}%
\pgfpathrectangle{\pgfqpoint{1.072000in}{0.528000in}}{\pgfqpoint{3.696000in}{3.696000in}}%
\pgfusepath{clip}%
\pgfsetbuttcap%
\pgfsetroundjoin%
\definecolor{currentfill}{rgb}{0.905783,0.455186,0.355336}%
\pgfsetfillcolor{currentfill}%
\pgfsetlinewidth{0.000000pt}%
\definecolor{currentstroke}{rgb}{0.000000,0.000000,0.000000}%
\pgfsetstrokecolor{currentstroke}%
\pgfsetdash{}{0pt}%
\pgfpathmoveto{\pgfqpoint{3.490593in}{3.004282in}}%
\pgfpathlineto{\pgfqpoint{3.536234in}{2.871345in}}%
\pgfpathlineto{\pgfqpoint{3.560905in}{2.825589in}}%
\pgfpathlineto{\pgfqpoint{3.515611in}{2.964967in}}%
\pgfpathlineto{\pgfqpoint{3.490593in}{3.004282in}}%
\pgfpathclose%
\pgfusepath{fill}%
\end{pgfscope}%
\begin{pgfscope}%
\pgfpathrectangle{\pgfqpoint{1.072000in}{0.528000in}}{\pgfqpoint{3.696000in}{3.696000in}}%
\pgfusepath{clip}%
\pgfsetbuttcap%
\pgfsetroundjoin%
\definecolor{currentfill}{rgb}{0.752704,0.157576,0.184258}%
\pgfsetfillcolor{currentfill}%
\pgfsetlinewidth{0.000000pt}%
\definecolor{currentstroke}{rgb}{0.000000,0.000000,0.000000}%
\pgfsetstrokecolor{currentstroke}%
\pgfsetdash{}{0pt}%
\pgfpathmoveto{\pgfqpoint{3.376382in}{3.212193in}}%
\pgfpathlineto{\pgfqpoint{3.423248in}{3.159403in}}%
\pgfpathlineto{\pgfqpoint{3.448544in}{3.123737in}}%
\pgfpathlineto{\pgfqpoint{3.402063in}{3.193893in}}%
\pgfpathlineto{\pgfqpoint{3.376382in}{3.212193in}}%
\pgfpathclose%
\pgfusepath{fill}%
\end{pgfscope}%
\begin{pgfscope}%
\pgfpathrectangle{\pgfqpoint{1.072000in}{0.528000in}}{\pgfqpoint{3.696000in}{3.696000in}}%
\pgfusepath{clip}%
\pgfsetbuttcap%
\pgfsetroundjoin%
\definecolor{currentfill}{rgb}{0.768929,0.189213,0.197965}%
\pgfsetfillcolor{currentfill}%
\pgfsetlinewidth{0.000000pt}%
\definecolor{currentstroke}{rgb}{0.000000,0.000000,0.000000}%
\pgfsetstrokecolor{currentstroke}%
\pgfsetdash{}{0pt}%
\pgfpathmoveto{\pgfqpoint{2.745614in}{3.188808in}}%
\pgfpathlineto{\pgfqpoint{2.792630in}{3.100915in}}%
\pgfpathlineto{\pgfqpoint{2.820008in}{3.110591in}}%
\pgfpathlineto{\pgfqpoint{2.773093in}{3.204916in}}%
\pgfpathlineto{\pgfqpoint{2.745614in}{3.188808in}}%
\pgfpathclose%
\pgfusepath{fill}%
\end{pgfscope}%
\begin{pgfscope}%
\pgfpathrectangle{\pgfqpoint{1.072000in}{0.528000in}}{\pgfqpoint{3.696000in}{3.696000in}}%
\pgfusepath{clip}%
\pgfsetbuttcap%
\pgfsetroundjoin%
\definecolor{currentfill}{rgb}{0.229806,0.298718,0.753683}%
\pgfsetfillcolor{currentfill}%
\pgfsetlinewidth{0.000000pt}%
\definecolor{currentstroke}{rgb}{0.000000,0.000000,0.000000}%
\pgfsetstrokecolor{currentstroke}%
\pgfsetdash{}{0pt}%
\pgfpathmoveto{\pgfqpoint{3.512384in}{1.264520in}}%
\pgfpathlineto{\pgfqpoint{3.559829in}{1.247540in}}%
\pgfpathlineto{\pgfqpoint{3.585565in}{1.270041in}}%
\pgfpathlineto{\pgfqpoint{3.538505in}{1.305454in}}%
\pgfpathlineto{\pgfqpoint{3.512384in}{1.264520in}}%
\pgfpathclose%
\pgfusepath{fill}%
\end{pgfscope}%
\begin{pgfscope}%
\pgfpathrectangle{\pgfqpoint{1.072000in}{0.528000in}}{\pgfqpoint{3.696000in}{3.696000in}}%
\pgfusepath{clip}%
\pgfsetbuttcap%
\pgfsetroundjoin%
\definecolor{currentfill}{rgb}{0.956371,0.775144,0.686416}%
\pgfsetfillcolor{currentfill}%
\pgfsetlinewidth{0.000000pt}%
\definecolor{currentstroke}{rgb}{0.000000,0.000000,0.000000}%
\pgfsetstrokecolor{currentstroke}%
\pgfsetdash{}{0pt}%
\pgfpathmoveto{\pgfqpoint{3.550453in}{2.582835in}}%
\pgfpathlineto{\pgfqpoint{3.594886in}{2.389722in}}%
\pgfpathlineto{\pgfqpoint{3.620132in}{2.390277in}}%
\pgfpathlineto{\pgfqpoint{3.575910in}{2.582299in}}%
\pgfpathlineto{\pgfqpoint{3.550453in}{2.582835in}}%
\pgfpathclose%
\pgfusepath{fill}%
\end{pgfscope}%
\begin{pgfscope}%
\pgfpathrectangle{\pgfqpoint{1.072000in}{0.528000in}}{\pgfqpoint{3.696000in}{3.696000in}}%
\pgfusepath{clip}%
\pgfsetbuttcap%
\pgfsetroundjoin%
\definecolor{currentfill}{rgb}{0.543440,0.680003,0.993051}%
\pgfsetfillcolor{currentfill}%
\pgfsetlinewidth{0.000000pt}%
\definecolor{currentstroke}{rgb}{0.000000,0.000000,0.000000}%
\pgfsetstrokecolor{currentstroke}%
\pgfsetdash{}{0pt}%
\pgfpathmoveto{\pgfqpoint{2.462406in}{1.652052in}}%
\pgfpathlineto{\pgfqpoint{2.508230in}{1.661239in}}%
\pgfpathlineto{\pgfqpoint{2.534104in}{1.843021in}}%
\pgfpathlineto{\pgfqpoint{2.488193in}{1.827772in}}%
\pgfpathlineto{\pgfqpoint{2.462406in}{1.652052in}}%
\pgfpathclose%
\pgfusepath{fill}%
\end{pgfscope}%
\begin{pgfscope}%
\pgfpathrectangle{\pgfqpoint{1.072000in}{0.528000in}}{\pgfqpoint{3.696000in}{3.696000in}}%
\pgfusepath{clip}%
\pgfsetbuttcap%
\pgfsetroundjoin%
\definecolor{currentfill}{rgb}{0.304174,0.406945,0.845263}%
\pgfsetfillcolor{currentfill}%
\pgfsetlinewidth{0.000000pt}%
\definecolor{currentstroke}{rgb}{0.000000,0.000000,0.000000}%
\pgfsetstrokecolor{currentstroke}%
\pgfsetdash{}{0pt}%
\pgfpathmoveto{\pgfqpoint{3.419109in}{1.395719in}}%
\pgfpathlineto{\pgfqpoint{3.465539in}{1.314976in}}%
\pgfpathlineto{\pgfqpoint{3.492032in}{1.375215in}}%
\pgfpathlineto{\pgfqpoint{3.445946in}{1.475409in}}%
\pgfpathlineto{\pgfqpoint{3.419109in}{1.395719in}}%
\pgfpathclose%
\pgfusepath{fill}%
\end{pgfscope}%
\begin{pgfscope}%
\pgfpathrectangle{\pgfqpoint{1.072000in}{0.528000in}}{\pgfqpoint{3.696000in}{3.696000in}}%
\pgfusepath{clip}%
\pgfsetbuttcap%
\pgfsetroundjoin%
\definecolor{currentfill}{rgb}{0.717435,0.051118,0.158737}%
\pgfsetfillcolor{currentfill}%
\pgfsetlinewidth{0.000000pt}%
\definecolor{currentstroke}{rgb}{0.000000,0.000000,0.000000}%
\pgfsetstrokecolor{currentstroke}%
\pgfsetdash{}{0pt}%
\pgfpathmoveto{\pgfqpoint{2.698839in}{3.233643in}}%
\pgfpathlineto{\pgfqpoint{2.745614in}{3.188808in}}%
\pgfpathlineto{\pgfqpoint{2.773093in}{3.204916in}}%
\pgfpathlineto{\pgfqpoint{2.726380in}{3.254423in}}%
\pgfpathlineto{\pgfqpoint{2.698839in}{3.233643in}}%
\pgfpathclose%
\pgfusepath{fill}%
\end{pgfscope}%
\begin{pgfscope}%
\pgfpathrectangle{\pgfqpoint{1.072000in}{0.528000in}}{\pgfqpoint{3.696000in}{3.696000in}}%
\pgfusepath{clip}%
\pgfsetbuttcap%
\pgfsetroundjoin%
\definecolor{currentfill}{rgb}{0.661968,0.775491,0.993937}%
\pgfsetfillcolor{currentfill}%
\pgfsetlinewidth{0.000000pt}%
\definecolor{currentstroke}{rgb}{0.000000,0.000000,0.000000}%
\pgfsetstrokecolor{currentstroke}%
\pgfsetdash{}{0pt}%
\pgfpathmoveto{\pgfqpoint{3.508274in}{1.969976in}}%
\pgfpathlineto{\pgfqpoint{3.553057in}{1.792772in}}%
\pgfpathlineto{\pgfqpoint{3.579509in}{1.856869in}}%
\pgfpathlineto{\pgfqpoint{3.534972in}{2.040951in}}%
\pgfpathlineto{\pgfqpoint{3.508274in}{1.969976in}}%
\pgfpathclose%
\pgfusepath{fill}%
\end{pgfscope}%
\begin{pgfscope}%
\pgfpathrectangle{\pgfqpoint{1.072000in}{0.528000in}}{\pgfqpoint{3.696000in}{3.696000in}}%
\pgfusepath{clip}%
\pgfsetbuttcap%
\pgfsetroundjoin%
\definecolor{currentfill}{rgb}{0.906154,0.842091,0.806151}%
\pgfsetfillcolor{currentfill}%
\pgfsetlinewidth{0.000000pt}%
\definecolor{currentstroke}{rgb}{0.000000,0.000000,0.000000}%
\pgfsetstrokecolor{currentstroke}%
\pgfsetdash{}{0pt}%
\pgfpathmoveto{\pgfqpoint{1.625907in}{2.458106in}}%
\pgfpathlineto{\pgfqpoint{1.676308in}{2.312863in}}%
\pgfpathlineto{\pgfqpoint{1.710651in}{2.207651in}}%
\pgfpathlineto{\pgfqpoint{1.660420in}{2.357328in}}%
\pgfpathlineto{\pgfqpoint{1.625907in}{2.458106in}}%
\pgfpathclose%
\pgfusepath{fill}%
\end{pgfscope}%
\begin{pgfscope}%
\pgfpathrectangle{\pgfqpoint{1.072000in}{0.528000in}}{\pgfqpoint{3.696000in}{3.696000in}}%
\pgfusepath{clip}%
\pgfsetbuttcap%
\pgfsetroundjoin%
\definecolor{currentfill}{rgb}{0.229806,0.298718,0.753683}%
\pgfsetfillcolor{currentfill}%
\pgfsetlinewidth{0.000000pt}%
\definecolor{currentstroke}{rgb}{0.000000,0.000000,0.000000}%
\pgfsetstrokecolor{currentstroke}%
\pgfsetdash{}{0pt}%
\pgfpathmoveto{\pgfqpoint{2.446643in}{1.255014in}}%
\pgfpathlineto{\pgfqpoint{2.492890in}{1.239920in}}%
\pgfpathlineto{\pgfqpoint{2.520645in}{1.281379in}}%
\pgfpathlineto{\pgfqpoint{2.474456in}{1.295171in}}%
\pgfpathlineto{\pgfqpoint{2.446643in}{1.255014in}}%
\pgfpathclose%
\pgfusepath{fill}%
\end{pgfscope}%
\begin{pgfscope}%
\pgfpathrectangle{\pgfqpoint{1.072000in}{0.528000in}}{\pgfqpoint{3.696000in}{3.696000in}}%
\pgfusepath{clip}%
\pgfsetbuttcap%
\pgfsetroundjoin%
\definecolor{currentfill}{rgb}{0.238948,0.312365,0.765676}%
\pgfsetfillcolor{currentfill}%
\pgfsetlinewidth{0.000000pt}%
\definecolor{currentstroke}{rgb}{0.000000,0.000000,0.000000}%
\pgfsetstrokecolor{currentstroke}%
\pgfsetdash{}{0pt}%
\pgfpathmoveto{\pgfqpoint{2.325420in}{1.300429in}}%
\pgfpathlineto{\pgfqpoint{2.371694in}{1.278355in}}%
\pgfpathlineto{\pgfqpoint{2.400494in}{1.270643in}}%
\pgfpathlineto{\pgfqpoint{2.354430in}{1.287351in}}%
\pgfpathlineto{\pgfqpoint{2.325420in}{1.300429in}}%
\pgfpathclose%
\pgfusepath{fill}%
\end{pgfscope}%
\begin{pgfscope}%
\pgfpathrectangle{\pgfqpoint{1.072000in}{0.528000in}}{\pgfqpoint{3.696000in}{3.696000in}}%
\pgfusepath{clip}%
\pgfsetbuttcap%
\pgfsetroundjoin%
\definecolor{currentfill}{rgb}{0.839351,0.861167,0.894494}%
\pgfsetfillcolor{currentfill}%
\pgfsetlinewidth{0.000000pt}%
\definecolor{currentstroke}{rgb}{0.000000,0.000000,0.000000}%
\pgfsetstrokecolor{currentstroke}%
\pgfsetdash{}{0pt}%
\pgfpathmoveto{\pgfqpoint{2.403524in}{2.074986in}}%
\pgfpathlineto{\pgfqpoint{2.448424in}{2.136882in}}%
\pgfpathlineto{\pgfqpoint{2.474016in}{2.330583in}}%
\pgfpathlineto{\pgfqpoint{2.429001in}{2.261386in}}%
\pgfpathlineto{\pgfqpoint{2.403524in}{2.074986in}}%
\pgfpathclose%
\pgfusepath{fill}%
\end{pgfscope}%
\begin{pgfscope}%
\pgfpathrectangle{\pgfqpoint{1.072000in}{0.528000in}}{\pgfqpoint{3.696000in}{3.696000in}}%
\pgfusepath{clip}%
\pgfsetbuttcap%
\pgfsetroundjoin%
\definecolor{currentfill}{rgb}{0.953054,0.585211,0.465373}%
\pgfsetfillcolor{currentfill}%
\pgfsetlinewidth{0.000000pt}%
\definecolor{currentstroke}{rgb}{0.000000,0.000000,0.000000}%
\pgfsetstrokecolor{currentstroke}%
\pgfsetdash{}{0pt}%
\pgfpathmoveto{\pgfqpoint{2.999642in}{2.766222in}}%
\pgfpathlineto{\pgfqpoint{3.046770in}{2.843259in}}%
\pgfpathlineto{\pgfqpoint{3.073437in}{2.775820in}}%
\pgfpathlineto{\pgfqpoint{3.026347in}{2.650923in}}%
\pgfpathlineto{\pgfqpoint{2.999642in}{2.766222in}}%
\pgfpathclose%
\pgfusepath{fill}%
\end{pgfscope}%
\begin{pgfscope}%
\pgfpathrectangle{\pgfqpoint{1.072000in}{0.528000in}}{\pgfqpoint{3.696000in}{3.696000in}}%
\pgfusepath{clip}%
\pgfsetbuttcap%
\pgfsetroundjoin%
\definecolor{currentfill}{rgb}{0.261805,0.346484,0.795658}%
\pgfsetfillcolor{currentfill}%
\pgfsetlinewidth{0.000000pt}%
\definecolor{currentstroke}{rgb}{0.000000,0.000000,0.000000}%
\pgfsetstrokecolor{currentstroke}%
\pgfsetdash{}{0pt}%
\pgfpathmoveto{\pgfqpoint{2.520645in}{1.281379in}}%
\pgfpathlineto{\pgfqpoint{2.566970in}{1.265479in}}%
\pgfpathlineto{\pgfqpoint{2.594121in}{1.349435in}}%
\pgfpathlineto{\pgfqpoint{2.547762in}{1.366546in}}%
\pgfpathlineto{\pgfqpoint{2.520645in}{1.281379in}}%
\pgfpathclose%
\pgfusepath{fill}%
\end{pgfscope}%
\begin{pgfscope}%
\pgfpathrectangle{\pgfqpoint{1.072000in}{0.528000in}}{\pgfqpoint{3.696000in}{3.696000in}}%
\pgfusepath{clip}%
\pgfsetbuttcap%
\pgfsetroundjoin%
\definecolor{currentfill}{rgb}{0.763520,0.178667,0.193396}%
\pgfsetfillcolor{currentfill}%
\pgfsetlinewidth{0.000000pt}%
\definecolor{currentstroke}{rgb}{0.000000,0.000000,0.000000}%
\pgfsetstrokecolor{currentstroke}%
\pgfsetdash{}{0pt}%
\pgfpathmoveto{\pgfqpoint{2.551765in}{3.097688in}}%
\pgfpathlineto{\pgfqpoint{2.597281in}{3.161522in}}%
\pgfpathlineto{\pgfqpoint{2.624890in}{3.201803in}}%
\pgfpathlineto{\pgfqpoint{2.579237in}{3.148114in}}%
\pgfpathlineto{\pgfqpoint{2.551765in}{3.097688in}}%
\pgfpathclose%
\pgfusepath{fill}%
\end{pgfscope}%
\begin{pgfscope}%
\pgfpathrectangle{\pgfqpoint{1.072000in}{0.528000in}}{\pgfqpoint{3.696000in}{3.696000in}}%
\pgfusepath{clip}%
\pgfsetbuttcap%
\pgfsetroundjoin%
\definecolor{currentfill}{rgb}{0.871493,0.862309,0.857016}%
\pgfsetfillcolor{currentfill}%
\pgfsetlinewidth{0.000000pt}%
\definecolor{currentstroke}{rgb}{0.000000,0.000000,0.000000}%
\pgfsetstrokecolor{currentstroke}%
\pgfsetdash{}{0pt}%
\pgfpathmoveto{\pgfqpoint{3.543220in}{2.342629in}}%
\pgfpathlineto{\pgfqpoint{3.587575in}{2.144398in}}%
\pgfpathlineto{\pgfqpoint{3.613394in}{2.175333in}}%
\pgfpathlineto{\pgfqpoint{3.569239in}{2.373900in}}%
\pgfpathlineto{\pgfqpoint{3.543220in}{2.342629in}}%
\pgfpathclose%
\pgfusepath{fill}%
\end{pgfscope}%
\begin{pgfscope}%
\pgfpathrectangle{\pgfqpoint{1.072000in}{0.528000in}}{\pgfqpoint{3.696000in}{3.696000in}}%
\pgfusepath{clip}%
\pgfsetbuttcap%
\pgfsetroundjoin%
\definecolor{currentfill}{rgb}{0.661968,0.775491,0.993937}%
\pgfsetfillcolor{currentfill}%
\pgfsetlinewidth{0.000000pt}%
\definecolor{currentstroke}{rgb}{0.000000,0.000000,0.000000}%
\pgfsetstrokecolor{currentstroke}%
\pgfsetdash{}{0pt}%
\pgfpathmoveto{\pgfqpoint{2.442627in}{1.802433in}}%
\pgfpathlineto{\pgfqpoint{2.488193in}{1.827772in}}%
\pgfpathlineto{\pgfqpoint{2.513864in}{2.022462in}}%
\pgfpathlineto{\pgfqpoint{2.468202in}{1.989807in}}%
\pgfpathlineto{\pgfqpoint{2.442627in}{1.802433in}}%
\pgfpathclose%
\pgfusepath{fill}%
\end{pgfscope}%
\begin{pgfscope}%
\pgfpathrectangle{\pgfqpoint{1.072000in}{0.528000in}}{\pgfqpoint{3.696000in}{3.696000in}}%
\pgfusepath{clip}%
\pgfsetbuttcap%
\pgfsetroundjoin%
\definecolor{currentfill}{rgb}{0.711554,0.033337,0.154485}%
\pgfsetfillcolor{currentfill}%
\pgfsetlinewidth{0.000000pt}%
\definecolor{currentstroke}{rgb}{0.000000,0.000000,0.000000}%
\pgfsetstrokecolor{currentstroke}%
\pgfsetdash{}{0pt}%
\pgfpathmoveto{\pgfqpoint{3.255351in}{3.191413in}}%
\pgfpathlineto{\pgfqpoint{3.302822in}{3.208383in}}%
\pgfpathlineto{\pgfqpoint{3.329258in}{3.239188in}}%
\pgfpathlineto{\pgfqpoint{3.282039in}{3.247209in}}%
\pgfpathlineto{\pgfqpoint{3.255351in}{3.191413in}}%
\pgfpathclose%
\pgfusepath{fill}%
\end{pgfscope}%
\begin{pgfscope}%
\pgfpathrectangle{\pgfqpoint{1.072000in}{0.528000in}}{\pgfqpoint{3.696000in}{3.696000in}}%
\pgfusepath{clip}%
\pgfsetbuttcap%
\pgfsetroundjoin%
\definecolor{currentfill}{rgb}{0.763363,0.835092,0.955658}%
\pgfsetfillcolor{currentfill}%
\pgfsetlinewidth{0.000000pt}%
\definecolor{currentstroke}{rgb}{0.000000,0.000000,0.000000}%
\pgfsetstrokecolor{currentstroke}%
\pgfsetdash{}{0pt}%
\pgfpathmoveto{\pgfqpoint{2.422946in}{1.946475in}}%
\pgfpathlineto{\pgfqpoint{2.468202in}{1.989807in}}%
\pgfpathlineto{\pgfqpoint{2.493786in}{2.187897in}}%
\pgfpathlineto{\pgfqpoint{2.448424in}{2.136882in}}%
\pgfpathlineto{\pgfqpoint{2.422946in}{1.946475in}}%
\pgfpathclose%
\pgfusepath{fill}%
\end{pgfscope}%
\begin{pgfscope}%
\pgfpathrectangle{\pgfqpoint{1.072000in}{0.528000in}}{\pgfqpoint{3.696000in}{3.696000in}}%
\pgfusepath{clip}%
\pgfsetbuttcap%
\pgfsetroundjoin%
\definecolor{currentfill}{rgb}{0.328604,0.439712,0.869587}%
\pgfsetfillcolor{currentfill}%
\pgfsetlinewidth{0.000000pt}%
\definecolor{currentstroke}{rgb}{0.000000,0.000000,0.000000}%
\pgfsetstrokecolor{currentstroke}%
\pgfsetdash{}{0pt}%
\pgfpathmoveto{\pgfqpoint{2.547762in}{1.366546in}}%
\pgfpathlineto{\pgfqpoint{2.594121in}{1.349435in}}%
\pgfpathlineto{\pgfqpoint{2.620837in}{1.473684in}}%
\pgfpathlineto{\pgfqpoint{2.574391in}{1.491997in}}%
\pgfpathlineto{\pgfqpoint{2.547762in}{1.366546in}}%
\pgfpathclose%
\pgfusepath{fill}%
\end{pgfscope}%
\begin{pgfscope}%
\pgfpathrectangle{\pgfqpoint{1.072000in}{0.528000in}}{\pgfqpoint{3.696000in}{3.696000in}}%
\pgfusepath{clip}%
\pgfsetbuttcap%
\pgfsetroundjoin%
\definecolor{currentfill}{rgb}{0.939254,0.539581,0.423900}%
\pgfsetfillcolor{currentfill}%
\pgfsetlinewidth{0.000000pt}%
\definecolor{currentstroke}{rgb}{0.000000,0.000000,0.000000}%
\pgfsetstrokecolor{currentstroke}%
\pgfsetdash{}{0pt}%
\pgfpathmoveto{\pgfqpoint{3.511097in}{2.900102in}}%
\pgfpathlineto{\pgfqpoint{3.556359in}{2.744051in}}%
\pgfpathlineto{\pgfqpoint{3.581200in}{2.712154in}}%
\pgfpathlineto{\pgfqpoint{3.536234in}{2.871345in}}%
\pgfpathlineto{\pgfqpoint{3.511097in}{2.900102in}}%
\pgfpathclose%
\pgfusepath{fill}%
\end{pgfscope}%
\begin{pgfscope}%
\pgfpathrectangle{\pgfqpoint{1.072000in}{0.528000in}}{\pgfqpoint{3.696000in}{3.696000in}}%
\pgfusepath{clip}%
\pgfsetbuttcap%
\pgfsetroundjoin%
\definecolor{currentfill}{rgb}{0.619318,0.744121,0.998931}%
\pgfsetfillcolor{currentfill}%
\pgfsetlinewidth{0.000000pt}%
\definecolor{currentstroke}{rgb}{0.000000,0.000000,0.000000}%
\pgfsetstrokecolor{currentstroke}%
\pgfsetdash{}{0pt}%
\pgfpathmoveto{\pgfqpoint{3.481376in}{1.888028in}}%
\pgfpathlineto{\pgfqpoint{3.526421in}{1.720081in}}%
\pgfpathlineto{\pgfqpoint{3.553057in}{1.792772in}}%
\pgfpathlineto{\pgfqpoint{3.508274in}{1.969976in}}%
\pgfpathlineto{\pgfqpoint{3.481376in}{1.888028in}}%
\pgfpathclose%
\pgfusepath{fill}%
\end{pgfscope}%
\begin{pgfscope}%
\pgfpathrectangle{\pgfqpoint{1.072000in}{0.528000in}}{\pgfqpoint{3.696000in}{3.696000in}}%
\pgfusepath{clip}%
\pgfsetbuttcap%
\pgfsetroundjoin%
\definecolor{currentfill}{rgb}{0.967317,0.657471,0.538160}%
\pgfsetfillcolor{currentfill}%
\pgfsetlinewidth{0.000000pt}%
\definecolor{currentstroke}{rgb}{0.000000,0.000000,0.000000}%
\pgfsetstrokecolor{currentstroke}%
\pgfsetdash{}{0pt}%
\pgfpathmoveto{\pgfqpoint{3.531078in}{2.759913in}}%
\pgfpathlineto{\pgfqpoint{3.575910in}{2.582299in}}%
\pgfpathlineto{\pgfqpoint{3.600944in}{2.566072in}}%
\pgfpathlineto{\pgfqpoint{3.556359in}{2.744051in}}%
\pgfpathlineto{\pgfqpoint{3.531078in}{2.759913in}}%
\pgfpathclose%
\pgfusepath{fill}%
\end{pgfscope}%
\begin{pgfscope}%
\pgfpathrectangle{\pgfqpoint{1.072000in}{0.528000in}}{\pgfqpoint{3.696000in}{3.696000in}}%
\pgfusepath{clip}%
\pgfsetbuttcap%
\pgfsetroundjoin%
\definecolor{currentfill}{rgb}{0.248091,0.326013,0.777669}%
\pgfsetfillcolor{currentfill}%
\pgfsetlinewidth{0.000000pt}%
\definecolor{currentstroke}{rgb}{0.000000,0.000000,0.000000}%
\pgfsetstrokecolor{currentstroke}%
\pgfsetdash{}{0pt}%
\pgfpathmoveto{\pgfqpoint{3.633403in}{1.271231in}}%
\pgfpathlineto{\pgfqpoint{3.682188in}{1.309445in}}%
\pgfpathlineto{\pgfqpoint{3.707231in}{1.309607in}}%
\pgfpathlineto{\pgfqpoint{3.658805in}{1.285643in}}%
\pgfpathlineto{\pgfqpoint{3.633403in}{1.271231in}}%
\pgfpathclose%
\pgfusepath{fill}%
\end{pgfscope}%
\begin{pgfscope}%
\pgfpathrectangle{\pgfqpoint{1.072000in}{0.528000in}}{\pgfqpoint{3.696000in}{3.696000in}}%
\pgfusepath{clip}%
\pgfsetbuttcap%
\pgfsetroundjoin%
\definecolor{currentfill}{rgb}{0.234377,0.305542,0.759680}%
\pgfsetfillcolor{currentfill}%
\pgfsetlinewidth{0.000000pt}%
\definecolor{currentstroke}{rgb}{0.000000,0.000000,0.000000}%
\pgfsetstrokecolor{currentstroke}%
\pgfsetdash{}{0pt}%
\pgfpathmoveto{\pgfqpoint{3.439115in}{1.263341in}}%
\pgfpathlineto{\pgfqpoint{3.486332in}{1.233276in}}%
\pgfpathlineto{\pgfqpoint{3.512384in}{1.264520in}}%
\pgfpathlineto{\pgfqpoint{3.465539in}{1.314976in}}%
\pgfpathlineto{\pgfqpoint{3.439115in}{1.263341in}}%
\pgfpathclose%
\pgfusepath{fill}%
\end{pgfscope}%
\begin{pgfscope}%
\pgfpathrectangle{\pgfqpoint{1.072000in}{0.528000in}}{\pgfqpoint{3.696000in}{3.696000in}}%
\pgfusepath{clip}%
\pgfsetbuttcap%
\pgfsetroundjoin%
\definecolor{currentfill}{rgb}{0.656683,0.771806,0.994914}%
\pgfsetfillcolor{currentfill}%
\pgfsetlinewidth{0.000000pt}%
\definecolor{currentstroke}{rgb}{0.000000,0.000000,0.000000}%
\pgfsetstrokecolor{currentstroke}%
\pgfsetdash{}{0pt}%
\pgfpathmoveto{\pgfqpoint{1.805981in}{2.016789in}}%
\pgfpathlineto{\pgfqpoint{1.854949in}{1.902746in}}%
\pgfpathlineto{\pgfqpoint{1.888665in}{1.788637in}}%
\pgfpathlineto{\pgfqpoint{1.840187in}{1.897768in}}%
\pgfpathlineto{\pgfqpoint{1.805981in}{2.016789in}}%
\pgfpathclose%
\pgfusepath{fill}%
\end{pgfscope}%
\begin{pgfscope}%
\pgfpathrectangle{\pgfqpoint{1.072000in}{0.528000in}}{\pgfqpoint{3.696000in}{3.696000in}}%
\pgfusepath{clip}%
\pgfsetbuttcap%
\pgfsetroundjoin%
\definecolor{currentfill}{rgb}{0.266381,0.353304,0.801637}%
\pgfsetfillcolor{currentfill}%
\pgfsetlinewidth{0.000000pt}%
\definecolor{currentstroke}{rgb}{0.000000,0.000000,0.000000}%
\pgfsetstrokecolor{currentstroke}%
\pgfsetdash{}{0pt}%
\pgfpathmoveto{\pgfqpoint{3.392341in}{1.323190in}}%
\pgfpathlineto{\pgfqpoint{3.439115in}{1.263341in}}%
\pgfpathlineto{\pgfqpoint{3.465539in}{1.314976in}}%
\pgfpathlineto{\pgfqpoint{3.419109in}{1.395719in}}%
\pgfpathlineto{\pgfqpoint{3.392341in}{1.323190in}}%
\pgfpathclose%
\pgfusepath{fill}%
\end{pgfscope}%
\begin{pgfscope}%
\pgfpathrectangle{\pgfqpoint{1.072000in}{0.528000in}}{\pgfqpoint{3.696000in}{3.696000in}}%
\pgfusepath{clip}%
\pgfsetbuttcap%
\pgfsetroundjoin%
\definecolor{currentfill}{rgb}{0.822420,0.856898,0.910795}%
\pgfsetfillcolor{currentfill}%
\pgfsetlinewidth{0.000000pt}%
\definecolor{currentstroke}{rgb}{0.000000,0.000000,0.000000}%
\pgfsetstrokecolor{currentstroke}%
\pgfsetdash{}{0pt}%
\pgfpathmoveto{\pgfqpoint{1.691750in}{2.282711in}}%
\pgfpathlineto{\pgfqpoint{1.741682in}{2.147169in}}%
\pgfpathlineto{\pgfqpoint{1.775995in}{2.033905in}}%
\pgfpathlineto{\pgfqpoint{1.726382in}{2.170087in}}%
\pgfpathlineto{\pgfqpoint{1.691750in}{2.282711in}}%
\pgfpathclose%
\pgfusepath{fill}%
\end{pgfscope}%
\begin{pgfscope}%
\pgfpathrectangle{\pgfqpoint{1.072000in}{0.528000in}}{\pgfqpoint{3.696000in}{3.696000in}}%
\pgfusepath{clip}%
\pgfsetbuttcap%
\pgfsetroundjoin%
\definecolor{currentfill}{rgb}{0.962701,0.628218,0.507636}%
\pgfsetfillcolor{currentfill}%
\pgfsetlinewidth{0.000000pt}%
\definecolor{currentstroke}{rgb}{0.000000,0.000000,0.000000}%
\pgfsetstrokecolor{currentstroke}%
\pgfsetdash{}{0pt}%
\pgfpathmoveto{\pgfqpoint{2.417563in}{2.581064in}}%
\pgfpathlineto{\pgfqpoint{2.461945in}{2.685758in}}%
\pgfpathlineto{\pgfqpoint{2.488414in}{2.824057in}}%
\pgfpathlineto{\pgfqpoint{2.443807in}{2.721803in}}%
\pgfpathlineto{\pgfqpoint{2.417563in}{2.581064in}}%
\pgfpathclose%
\pgfusepath{fill}%
\end{pgfscope}%
\begin{pgfscope}%
\pgfpathrectangle{\pgfqpoint{1.072000in}{0.528000in}}{\pgfqpoint{3.696000in}{3.696000in}}%
\pgfusepath{clip}%
\pgfsetbuttcap%
\pgfsetroundjoin%
\definecolor{currentfill}{rgb}{0.830187,0.304733,0.254891}%
\pgfsetfillcolor{currentfill}%
\pgfsetlinewidth{0.000000pt}%
\definecolor{currentstroke}{rgb}{0.000000,0.000000,0.000000}%
\pgfsetstrokecolor{currentstroke}%
\pgfsetdash{}{0pt}%
\pgfpathmoveto{\pgfqpoint{2.765010in}{3.110485in}}%
\pgfpathlineto{\pgfqpoint{2.812156in}{3.010150in}}%
\pgfpathlineto{\pgfqpoint{2.839692in}{2.976764in}}%
\pgfpathlineto{\pgfqpoint{2.792630in}{3.100915in}}%
\pgfpathlineto{\pgfqpoint{2.765010in}{3.110485in}}%
\pgfpathclose%
\pgfusepath{fill}%
\end{pgfscope}%
\begin{pgfscope}%
\pgfpathrectangle{\pgfqpoint{1.072000in}{0.528000in}}{\pgfqpoint{3.696000in}{3.696000in}}%
\pgfusepath{clip}%
\pgfsetbuttcap%
\pgfsetroundjoin%
\definecolor{currentfill}{rgb}{0.430507,0.564883,0.948889}%
\pgfsetfillcolor{currentfill}%
\pgfsetlinewidth{0.000000pt}%
\definecolor{currentstroke}{rgb}{0.000000,0.000000,0.000000}%
\pgfsetstrokecolor{currentstroke}%
\pgfsetdash{}{0pt}%
\pgfpathmoveto{\pgfqpoint{2.528140in}{1.502908in}}%
\pgfpathlineto{\pgfqpoint{2.574391in}{1.491997in}}%
\pgfpathlineto{\pgfqpoint{2.600696in}{1.652024in}}%
\pgfpathlineto{\pgfqpoint{2.554336in}{1.661487in}}%
\pgfpathlineto{\pgfqpoint{2.528140in}{1.502908in}}%
\pgfpathclose%
\pgfusepath{fill}%
\end{pgfscope}%
\begin{pgfscope}%
\pgfpathrectangle{\pgfqpoint{1.072000in}{0.528000in}}{\pgfqpoint{3.696000in}{3.696000in}}%
\pgfusepath{clip}%
\pgfsetbuttcap%
\pgfsetroundjoin%
\definecolor{currentfill}{rgb}{0.234377,0.305542,0.759680}%
\pgfsetfillcolor{currentfill}%
\pgfsetlinewidth{0.000000pt}%
\definecolor{currentstroke}{rgb}{0.000000,0.000000,0.000000}%
\pgfsetstrokecolor{currentstroke}%
\pgfsetdash{}{0pt}%
\pgfpathmoveto{\pgfqpoint{3.559829in}{1.247540in}}%
\pgfpathlineto{\pgfqpoint{3.608051in}{1.265632in}}%
\pgfpathlineto{\pgfqpoint{3.633403in}{1.271231in}}%
\pgfpathlineto{\pgfqpoint{3.585565in}{1.270041in}}%
\pgfpathlineto{\pgfqpoint{3.559829in}{1.247540in}}%
\pgfpathclose%
\pgfusepath{fill}%
\end{pgfscope}%
\begin{pgfscope}%
\pgfpathrectangle{\pgfqpoint{1.072000in}{0.528000in}}{\pgfqpoint{3.696000in}{3.696000in}}%
\pgfusepath{clip}%
\pgfsetbuttcap%
\pgfsetroundjoin%
\definecolor{currentfill}{rgb}{0.565182,0.699438,0.996635}%
\pgfsetfillcolor{currentfill}%
\pgfsetlinewidth{0.000000pt}%
\definecolor{currentstroke}{rgb}{0.000000,0.000000,0.000000}%
\pgfsetstrokecolor{currentstroke}%
\pgfsetdash{}{0pt}%
\pgfpathmoveto{\pgfqpoint{3.454335in}{1.797234in}}%
\pgfpathlineto{\pgfqpoint{3.499653in}{1.641031in}}%
\pgfpathlineto{\pgfqpoint{3.526421in}{1.720081in}}%
\pgfpathlineto{\pgfqpoint{3.481376in}{1.888028in}}%
\pgfpathlineto{\pgfqpoint{3.454335in}{1.797234in}}%
\pgfpathclose%
\pgfusepath{fill}%
\end{pgfscope}%
\begin{pgfscope}%
\pgfpathrectangle{\pgfqpoint{1.072000in}{0.528000in}}{\pgfqpoint{3.696000in}{3.696000in}}%
\pgfusepath{clip}%
\pgfsetbuttcap%
\pgfsetroundjoin%
\definecolor{currentfill}{rgb}{0.705673,0.015556,0.150233}%
\pgfsetfillcolor{currentfill}%
\pgfsetlinewidth{0.000000pt}%
\definecolor{currentstroke}{rgb}{0.000000,0.000000,0.000000}%
\pgfsetstrokecolor{currentstroke}%
\pgfsetdash{}{0pt}%
\pgfpathmoveto{\pgfqpoint{2.624890in}{3.201803in}}%
\pgfpathlineto{\pgfqpoint{2.671185in}{3.212740in}}%
\pgfpathlineto{\pgfqpoint{2.698839in}{3.233643in}}%
\pgfpathlineto{\pgfqpoint{2.652501in}{3.232415in}}%
\pgfpathlineto{\pgfqpoint{2.624890in}{3.201803in}}%
\pgfpathclose%
\pgfusepath{fill}%
\end{pgfscope}%
\begin{pgfscope}%
\pgfpathrectangle{\pgfqpoint{1.072000in}{0.528000in}}{\pgfqpoint{3.696000in}{3.696000in}}%
\pgfusepath{clip}%
\pgfsetbuttcap%
\pgfsetroundjoin%
\definecolor{currentfill}{rgb}{0.238948,0.312365,0.765676}%
\pgfsetfillcolor{currentfill}%
\pgfsetlinewidth{0.000000pt}%
\definecolor{currentstroke}{rgb}{0.000000,0.000000,0.000000}%
\pgfsetstrokecolor{currentstroke}%
\pgfsetdash{}{0pt}%
\pgfpathmoveto{\pgfqpoint{2.371694in}{1.278355in}}%
\pgfpathlineto{\pgfqpoint{2.418003in}{1.259422in}}%
\pgfpathlineto{\pgfqpoint{2.446643in}{1.255014in}}%
\pgfpathlineto{\pgfqpoint{2.400494in}{1.270643in}}%
\pgfpathlineto{\pgfqpoint{2.371694in}{1.278355in}}%
\pgfpathclose%
\pgfusepath{fill}%
\end{pgfscope}%
\begin{pgfscope}%
\pgfpathrectangle{\pgfqpoint{1.072000in}{0.528000in}}{\pgfqpoint{3.696000in}{3.696000in}}%
\pgfusepath{clip}%
\pgfsetbuttcap%
\pgfsetroundjoin%
\definecolor{currentfill}{rgb}{0.229806,0.298718,0.753683}%
\pgfsetfillcolor{currentfill}%
\pgfsetlinewidth{0.000000pt}%
\definecolor{currentstroke}{rgb}{0.000000,0.000000,0.000000}%
\pgfsetstrokecolor{currentstroke}%
\pgfsetdash{}{0pt}%
\pgfpathmoveto{\pgfqpoint{2.492890in}{1.239920in}}%
\pgfpathlineto{\pgfqpoint{2.539240in}{1.225179in}}%
\pgfpathlineto{\pgfqpoint{2.566970in}{1.265479in}}%
\pgfpathlineto{\pgfqpoint{2.520645in}{1.281379in}}%
\pgfpathlineto{\pgfqpoint{2.492890in}{1.239920in}}%
\pgfpathclose%
\pgfusepath{fill}%
\end{pgfscope}%
\begin{pgfscope}%
\pgfpathrectangle{\pgfqpoint{1.072000in}{0.528000in}}{\pgfqpoint{3.696000in}{3.696000in}}%
\pgfusepath{clip}%
\pgfsetbuttcap%
\pgfsetroundjoin%
\definecolor{currentfill}{rgb}{0.505423,0.643995,0.983157}%
\pgfsetfillcolor{currentfill}%
\pgfsetlinewidth{0.000000pt}%
\definecolor{currentstroke}{rgb}{0.000000,0.000000,0.000000}%
\pgfsetstrokecolor{currentstroke}%
\pgfsetdash{}{0pt}%
\pgfpathmoveto{\pgfqpoint{3.427207in}{1.700344in}}%
\pgfpathlineto{\pgfqpoint{3.472810in}{1.558390in}}%
\pgfpathlineto{\pgfqpoint{3.499653in}{1.641031in}}%
\pgfpathlineto{\pgfqpoint{3.454335in}{1.797234in}}%
\pgfpathlineto{\pgfqpoint{3.427207in}{1.700344in}}%
\pgfpathclose%
\pgfusepath{fill}%
\end{pgfscope}%
\begin{pgfscope}%
\pgfpathrectangle{\pgfqpoint{1.072000in}{0.528000in}}{\pgfqpoint{3.696000in}{3.696000in}}%
\pgfusepath{clip}%
\pgfsetbuttcap%
\pgfsetroundjoin%
\definecolor{currentfill}{rgb}{0.929357,0.512254,0.400673}%
\pgfsetfillcolor{currentfill}%
\pgfsetlinewidth{0.000000pt}%
\definecolor{currentstroke}{rgb}{0.000000,0.000000,0.000000}%
\pgfsetstrokecolor{currentstroke}%
\pgfsetdash{}{0pt}%
\pgfpathmoveto{\pgfqpoint{3.073437in}{2.775820in}}%
\pgfpathlineto{\pgfqpoint{3.121001in}{2.899182in}}%
\pgfpathlineto{\pgfqpoint{3.147684in}{2.899623in}}%
\pgfpathlineto{\pgfqpoint{3.100050in}{2.760212in}}%
\pgfpathlineto{\pgfqpoint{3.073437in}{2.775820in}}%
\pgfpathclose%
\pgfusepath{fill}%
\end{pgfscope}%
\begin{pgfscope}%
\pgfpathrectangle{\pgfqpoint{1.072000in}{0.528000in}}{\pgfqpoint{3.696000in}{3.696000in}}%
\pgfusepath{clip}%
\pgfsetbuttcap%
\pgfsetroundjoin%
\definecolor{currentfill}{rgb}{0.441123,0.576532,0.954545}%
\pgfsetfillcolor{currentfill}%
\pgfsetlinewidth{0.000000pt}%
\definecolor{currentstroke}{rgb}{0.000000,0.000000,0.000000}%
\pgfsetstrokecolor{currentstroke}%
\pgfsetdash{}{0pt}%
\pgfpathmoveto{\pgfqpoint{3.400048in}{1.600706in}}%
\pgfpathlineto{\pgfqpoint{3.445946in}{1.475409in}}%
\pgfpathlineto{\pgfqpoint{3.472810in}{1.558390in}}%
\pgfpathlineto{\pgfqpoint{3.427207in}{1.700344in}}%
\pgfpathlineto{\pgfqpoint{3.400048in}{1.600706in}}%
\pgfpathclose%
\pgfusepath{fill}%
\end{pgfscope}%
\begin{pgfscope}%
\pgfpathrectangle{\pgfqpoint{1.072000in}{0.528000in}}{\pgfqpoint{3.696000in}{3.696000in}}%
\pgfusepath{clip}%
\pgfsetbuttcap%
\pgfsetroundjoin%
\definecolor{currentfill}{rgb}{0.257234,0.339661,0.789661}%
\pgfsetfillcolor{currentfill}%
\pgfsetlinewidth{0.000000pt}%
\definecolor{currentstroke}{rgb}{0.000000,0.000000,0.000000}%
\pgfsetstrokecolor{currentstroke}%
\pgfsetdash{}{0pt}%
\pgfpathmoveto{\pgfqpoint{2.566970in}{1.265479in}}%
\pgfpathlineto{\pgfqpoint{2.613421in}{1.247677in}}%
\pgfpathlineto{\pgfqpoint{2.640626in}{1.327817in}}%
\pgfpathlineto{\pgfqpoint{2.594121in}{1.349435in}}%
\pgfpathlineto{\pgfqpoint{2.566970in}{1.265479in}}%
\pgfpathclose%
\pgfusepath{fill}%
\end{pgfscope}%
\begin{pgfscope}%
\pgfpathrectangle{\pgfqpoint{1.072000in}{0.528000in}}{\pgfqpoint{3.696000in}{3.696000in}}%
\pgfusepath{clip}%
\pgfsetbuttcap%
\pgfsetroundjoin%
\definecolor{currentfill}{rgb}{0.378598,0.503856,0.913692}%
\pgfsetfillcolor{currentfill}%
\pgfsetlinewidth{0.000000pt}%
\definecolor{currentstroke}{rgb}{0.000000,0.000000,0.000000}%
\pgfsetstrokecolor{currentstroke}%
\pgfsetdash{}{0pt}%
\pgfpathmoveto{\pgfqpoint{3.372912in}{1.502188in}}%
\pgfpathlineto{\pgfqpoint{3.419109in}{1.395719in}}%
\pgfpathlineto{\pgfqpoint{3.445946in}{1.475409in}}%
\pgfpathlineto{\pgfqpoint{3.400048in}{1.600706in}}%
\pgfpathlineto{\pgfqpoint{3.372912in}{1.502188in}}%
\pgfpathclose%
\pgfusepath{fill}%
\end{pgfscope}%
\begin{pgfscope}%
\pgfpathrectangle{\pgfqpoint{1.072000in}{0.528000in}}{\pgfqpoint{3.696000in}{3.696000in}}%
\pgfusepath{clip}%
\pgfsetbuttcap%
\pgfsetroundjoin%
\definecolor{currentfill}{rgb}{0.790562,0.231397,0.216242}%
\pgfsetfillcolor{currentfill}%
\pgfsetlinewidth{0.000000pt}%
\definecolor{currentstroke}{rgb}{0.000000,0.000000,0.000000}%
\pgfsetstrokecolor{currentstroke}%
\pgfsetdash{}{0pt}%
\pgfpathmoveto{\pgfqpoint{2.524419in}{3.031787in}}%
\pgfpathlineto{\pgfqpoint{2.569746in}{3.107212in}}%
\pgfpathlineto{\pgfqpoint{2.597281in}{3.161522in}}%
\pgfpathlineto{\pgfqpoint{2.551765in}{3.097688in}}%
\pgfpathlineto{\pgfqpoint{2.524419in}{3.031787in}}%
\pgfpathclose%
\pgfusepath{fill}%
\end{pgfscope}%
\begin{pgfscope}%
\pgfpathrectangle{\pgfqpoint{1.072000in}{0.528000in}}{\pgfqpoint{3.696000in}{3.696000in}}%
\pgfusepath{clip}%
\pgfsetbuttcap%
\pgfsetroundjoin%
\definecolor{currentfill}{rgb}{0.830187,0.304733,0.254891}%
\pgfsetfillcolor{currentfill}%
\pgfsetlinewidth{0.000000pt}%
\definecolor{currentstroke}{rgb}{0.000000,0.000000,0.000000}%
\pgfsetstrokecolor{currentstroke}%
\pgfsetdash{}{0pt}%
\pgfpathmoveto{\pgfqpoint{3.444311in}{3.105380in}}%
\pgfpathlineto{\pgfqpoint{3.490593in}{3.004282in}}%
\pgfpathlineto{\pgfqpoint{3.515611in}{2.964967in}}%
\pgfpathlineto{\pgfqpoint{3.469700in}{3.077129in}}%
\pgfpathlineto{\pgfqpoint{3.444311in}{3.105380in}}%
\pgfpathclose%
\pgfusepath{fill}%
\end{pgfscope}%
\begin{pgfscope}%
\pgfpathrectangle{\pgfqpoint{1.072000in}{0.528000in}}{\pgfqpoint{3.696000in}{3.696000in}}%
\pgfusepath{clip}%
\pgfsetbuttcap%
\pgfsetroundjoin%
\definecolor{currentfill}{rgb}{0.960581,0.762501,0.667964}%
\pgfsetfillcolor{currentfill}%
\pgfsetlinewidth{0.000000pt}%
\definecolor{currentstroke}{rgb}{0.000000,0.000000,0.000000}%
\pgfsetstrokecolor{currentstroke}%
\pgfsetdash{}{0pt}%
\pgfpathmoveto{\pgfqpoint{2.410045in}{2.358545in}}%
\pgfpathlineto{\pgfqpoint{2.454671in}{2.445119in}}%
\pgfpathlineto{\pgfqpoint{2.480641in}{2.618006in}}%
\pgfpathlineto{\pgfqpoint{2.435834in}{2.528892in}}%
\pgfpathlineto{\pgfqpoint{2.410045in}{2.358545in}}%
\pgfpathclose%
\pgfusepath{fill}%
\end{pgfscope}%
\begin{pgfscope}%
\pgfpathrectangle{\pgfqpoint{1.072000in}{0.528000in}}{\pgfqpoint{3.696000in}{3.696000in}}%
\pgfusepath{clip}%
\pgfsetbuttcap%
\pgfsetroundjoin%
\definecolor{currentfill}{rgb}{0.323718,0.433158,0.864722}%
\pgfsetfillcolor{currentfill}%
\pgfsetlinewidth{0.000000pt}%
\definecolor{currentstroke}{rgb}{0.000000,0.000000,0.000000}%
\pgfsetstrokecolor{currentstroke}%
\pgfsetdash{}{0pt}%
\pgfpathmoveto{\pgfqpoint{2.594121in}{1.349435in}}%
\pgfpathlineto{\pgfqpoint{2.640626in}{1.327817in}}%
\pgfpathlineto{\pgfqpoint{2.667447in}{1.448200in}}%
\pgfpathlineto{\pgfqpoint{2.620837in}{1.473684in}}%
\pgfpathlineto{\pgfqpoint{2.594121in}{1.349435in}}%
\pgfpathclose%
\pgfusepath{fill}%
\end{pgfscope}%
\begin{pgfscope}%
\pgfpathrectangle{\pgfqpoint{1.072000in}{0.528000in}}{\pgfqpoint{3.696000in}{3.696000in}}%
\pgfusepath{clip}%
\pgfsetbuttcap%
\pgfsetroundjoin%
\definecolor{currentfill}{rgb}{0.740957,0.122240,0.175744}%
\pgfsetfillcolor{currentfill}%
\pgfsetlinewidth{0.000000pt}%
\definecolor{currentstroke}{rgb}{0.000000,0.000000,0.000000}%
\pgfsetstrokecolor{currentstroke}%
\pgfsetdash{}{0pt}%
\pgfpathmoveto{\pgfqpoint{3.228429in}{3.111391in}}%
\pgfpathlineto{\pgfqpoint{3.276112in}{3.156840in}}%
\pgfpathlineto{\pgfqpoint{3.302822in}{3.208383in}}%
\pgfpathlineto{\pgfqpoint{3.255351in}{3.191413in}}%
\pgfpathlineto{\pgfqpoint{3.228429in}{3.111391in}}%
\pgfpathclose%
\pgfusepath{fill}%
\end{pgfscope}%
\begin{pgfscope}%
\pgfpathrectangle{\pgfqpoint{1.072000in}{0.528000in}}{\pgfqpoint{3.696000in}{3.696000in}}%
\pgfusepath{clip}%
\pgfsetbuttcap%
\pgfsetroundjoin%
\definecolor{currentfill}{rgb}{0.328604,0.439712,0.869587}%
\pgfsetfillcolor{currentfill}%
\pgfsetlinewidth{0.000000pt}%
\definecolor{currentstroke}{rgb}{0.000000,0.000000,0.000000}%
\pgfsetstrokecolor{currentstroke}%
\pgfsetdash{}{0pt}%
\pgfpathmoveto{\pgfqpoint{3.345843in}{1.409025in}}%
\pgfpathlineto{\pgfqpoint{3.392341in}{1.323190in}}%
\pgfpathlineto{\pgfqpoint{3.419109in}{1.395719in}}%
\pgfpathlineto{\pgfqpoint{3.372912in}{1.502188in}}%
\pgfpathlineto{\pgfqpoint{3.345843in}{1.409025in}}%
\pgfpathclose%
\pgfusepath{fill}%
\end{pgfscope}%
\begin{pgfscope}%
\pgfpathrectangle{\pgfqpoint{1.072000in}{0.528000in}}{\pgfqpoint{3.696000in}{3.696000in}}%
\pgfusepath{clip}%
\pgfsetbuttcap%
\pgfsetroundjoin%
\definecolor{currentfill}{rgb}{0.229806,0.298718,0.753683}%
\pgfsetfillcolor{currentfill}%
\pgfsetlinewidth{0.000000pt}%
\definecolor{currentstroke}{rgb}{0.000000,0.000000,0.000000}%
\pgfsetstrokecolor{currentstroke}%
\pgfsetdash{}{0pt}%
\pgfpathmoveto{\pgfqpoint{3.486332in}{1.233276in}}%
\pgfpathlineto{\pgfqpoint{3.534162in}{1.235463in}}%
\pgfpathlineto{\pgfqpoint{3.559829in}{1.247540in}}%
\pgfpathlineto{\pgfqpoint{3.512384in}{1.264520in}}%
\pgfpathlineto{\pgfqpoint{3.486332in}{1.233276in}}%
\pgfpathclose%
\pgfusepath{fill}%
\end{pgfscope}%
\begin{pgfscope}%
\pgfpathrectangle{\pgfqpoint{1.072000in}{0.528000in}}{\pgfqpoint{3.696000in}{3.696000in}}%
\pgfusepath{clip}%
\pgfsetbuttcap%
\pgfsetroundjoin%
\definecolor{currentfill}{rgb}{0.565182,0.699438,0.996635}%
\pgfsetfillcolor{currentfill}%
\pgfsetlinewidth{0.000000pt}%
\definecolor{currentstroke}{rgb}{0.000000,0.000000,0.000000}%
\pgfsetstrokecolor{currentstroke}%
\pgfsetdash{}{0pt}%
\pgfpathmoveto{\pgfqpoint{2.508230in}{1.661239in}}%
\pgfpathlineto{\pgfqpoint{2.554336in}{1.661487in}}%
\pgfpathlineto{\pgfqpoint{2.580333in}{1.846953in}}%
\pgfpathlineto{\pgfqpoint{2.534104in}{1.843021in}}%
\pgfpathlineto{\pgfqpoint{2.508230in}{1.661239in}}%
\pgfpathclose%
\pgfusepath{fill}%
\end{pgfscope}%
\begin{pgfscope}%
\pgfpathrectangle{\pgfqpoint{1.072000in}{0.528000in}}{\pgfqpoint{3.696000in}{3.696000in}}%
\pgfusepath{clip}%
\pgfsetbuttcap%
\pgfsetroundjoin%
\definecolor{currentfill}{rgb}{0.929357,0.512254,0.400673}%
\pgfsetfillcolor{currentfill}%
\pgfsetlinewidth{0.000000pt}%
\definecolor{currentstroke}{rgb}{0.000000,0.000000,0.000000}%
\pgfsetstrokecolor{currentstroke}%
\pgfsetdash{}{0pt}%
\pgfpathmoveto{\pgfqpoint{2.878591in}{2.906156in}}%
\pgfpathlineto{\pgfqpoint{2.925575in}{2.871029in}}%
\pgfpathlineto{\pgfqpoint{2.952840in}{2.741032in}}%
\pgfpathlineto{\pgfqpoint{2.906110in}{2.792406in}}%
\pgfpathlineto{\pgfqpoint{2.878591in}{2.906156in}}%
\pgfpathclose%
\pgfusepath{fill}%
\end{pgfscope}%
\begin{pgfscope}%
\pgfpathrectangle{\pgfqpoint{1.072000in}{0.528000in}}{\pgfqpoint{3.696000in}{3.696000in}}%
\pgfusepath{clip}%
\pgfsetbuttcap%
\pgfsetroundjoin%
\definecolor{currentfill}{rgb}{0.926883,0.505422,0.394866}%
\pgfsetfillcolor{currentfill}%
\pgfsetlinewidth{0.000000pt}%
\definecolor{currentstroke}{rgb}{0.000000,0.000000,0.000000}%
\pgfsetstrokecolor{currentstroke}%
\pgfsetdash{}{0pt}%
\pgfpathmoveto{\pgfqpoint{2.443807in}{2.721803in}}%
\pgfpathlineto{\pgfqpoint{2.488414in}{2.824057in}}%
\pgfpathlineto{\pgfqpoint{2.515236in}{2.940657in}}%
\pgfpathlineto{\pgfqpoint{2.470383in}{2.844657in}}%
\pgfpathlineto{\pgfqpoint{2.443807in}{2.721803in}}%
\pgfpathclose%
\pgfusepath{fill}%
\end{pgfscope}%
\begin{pgfscope}%
\pgfpathrectangle{\pgfqpoint{1.072000in}{0.528000in}}{\pgfqpoint{3.696000in}{3.696000in}}%
\pgfusepath{clip}%
\pgfsetbuttcap%
\pgfsetroundjoin%
\definecolor{currentfill}{rgb}{0.243520,0.319189,0.771672}%
\pgfsetfillcolor{currentfill}%
\pgfsetlinewidth{0.000000pt}%
\definecolor{currentstroke}{rgb}{0.000000,0.000000,0.000000}%
\pgfsetstrokecolor{currentstroke}%
\pgfsetdash{}{0pt}%
\pgfpathmoveto{\pgfqpoint{3.365672in}{1.261748in}}%
\pgfpathlineto{\pgfqpoint{3.412785in}{1.223719in}}%
\pgfpathlineto{\pgfqpoint{3.439115in}{1.263341in}}%
\pgfpathlineto{\pgfqpoint{3.392341in}{1.323190in}}%
\pgfpathlineto{\pgfqpoint{3.365672in}{1.261748in}}%
\pgfpathclose%
\pgfusepath{fill}%
\end{pgfscope}%
\begin{pgfscope}%
\pgfpathrectangle{\pgfqpoint{1.072000in}{0.528000in}}{\pgfqpoint{3.696000in}{3.696000in}}%
\pgfusepath{clip}%
\pgfsetbuttcap%
\pgfsetroundjoin%
\definecolor{currentfill}{rgb}{0.705673,0.015556,0.150233}%
\pgfsetfillcolor{currentfill}%
\pgfsetlinewidth{0.000000pt}%
\definecolor{currentstroke}{rgb}{0.000000,0.000000,0.000000}%
\pgfsetstrokecolor{currentstroke}%
\pgfsetdash{}{0pt}%
\pgfpathmoveto{\pgfqpoint{3.302822in}{3.208383in}}%
\pgfpathlineto{\pgfqpoint{3.350261in}{3.204761in}}%
\pgfpathlineto{\pgfqpoint{3.376382in}{3.212193in}}%
\pgfpathlineto{\pgfqpoint{3.329258in}{3.239188in}}%
\pgfpathlineto{\pgfqpoint{3.302822in}{3.208383in}}%
\pgfpathclose%
\pgfusepath{fill}%
\end{pgfscope}%
\begin{pgfscope}%
\pgfpathrectangle{\pgfqpoint{1.072000in}{0.528000in}}{\pgfqpoint{3.696000in}{3.696000in}}%
\pgfusepath{clip}%
\pgfsetbuttcap%
\pgfsetroundjoin%
\definecolor{currentfill}{rgb}{0.859385,0.864431,0.872111}%
\pgfsetfillcolor{currentfill}%
\pgfsetlinewidth{0.000000pt}%
\definecolor{currentstroke}{rgb}{0.000000,0.000000,0.000000}%
\pgfsetstrokecolor{currentstroke}%
\pgfsetdash{}{0pt}%
\pgfpathmoveto{\pgfqpoint{3.516865in}{2.295966in}}%
\pgfpathlineto{\pgfqpoint{3.561421in}{2.099425in}}%
\pgfpathlineto{\pgfqpoint{3.587575in}{2.144398in}}%
\pgfpathlineto{\pgfqpoint{3.543220in}{2.342629in}}%
\pgfpathlineto{\pgfqpoint{3.516865in}{2.295966in}}%
\pgfpathclose%
\pgfusepath{fill}%
\end{pgfscope}%
\begin{pgfscope}%
\pgfpathrectangle{\pgfqpoint{1.072000in}{0.528000in}}{\pgfqpoint{3.696000in}{3.696000in}}%
\pgfusepath{clip}%
\pgfsetbuttcap%
\pgfsetroundjoin%
\definecolor{currentfill}{rgb}{0.758112,0.168122,0.188827}%
\pgfsetfillcolor{currentfill}%
\pgfsetlinewidth{0.000000pt}%
\definecolor{currentstroke}{rgb}{0.000000,0.000000,0.000000}%
\pgfsetstrokecolor{currentstroke}%
\pgfsetdash{}{0pt}%
\pgfpathmoveto{\pgfqpoint{2.717956in}{3.180951in}}%
\pgfpathlineto{\pgfqpoint{2.765010in}{3.110485in}}%
\pgfpathlineto{\pgfqpoint{2.792630in}{3.100915in}}%
\pgfpathlineto{\pgfqpoint{2.745614in}{3.188808in}}%
\pgfpathlineto{\pgfqpoint{2.717956in}{3.180951in}}%
\pgfpathclose%
\pgfusepath{fill}%
\end{pgfscope}%
\begin{pgfscope}%
\pgfpathrectangle{\pgfqpoint{1.072000in}{0.528000in}}{\pgfqpoint{3.696000in}{3.696000in}}%
\pgfusepath{clip}%
\pgfsetbuttcap%
\pgfsetroundjoin%
\definecolor{currentfill}{rgb}{0.430507,0.564883,0.948889}%
\pgfsetfillcolor{currentfill}%
\pgfsetlinewidth{0.000000pt}%
\definecolor{currentstroke}{rgb}{0.000000,0.000000,0.000000}%
\pgfsetstrokecolor{currentstroke}%
\pgfsetdash{}{0pt}%
\pgfpathmoveto{\pgfqpoint{2.574391in}{1.491997in}}%
\pgfpathlineto{\pgfqpoint{2.620837in}{1.473684in}}%
\pgfpathlineto{\pgfqpoint{2.647274in}{1.632579in}}%
\pgfpathlineto{\pgfqpoint{2.600696in}{1.652024in}}%
\pgfpathlineto{\pgfqpoint{2.574391in}{1.491997in}}%
\pgfpathclose%
\pgfusepath{fill}%
\end{pgfscope}%
\begin{pgfscope}%
\pgfpathrectangle{\pgfqpoint{1.072000in}{0.528000in}}{\pgfqpoint{3.696000in}{3.696000in}}%
\pgfusepath{clip}%
\pgfsetbuttcap%
\pgfsetroundjoin%
\definecolor{currentfill}{rgb}{0.902659,0.447939,0.349721}%
\pgfsetfillcolor{currentfill}%
\pgfsetlinewidth{0.000000pt}%
\definecolor{currentstroke}{rgb}{0.000000,0.000000,0.000000}%
\pgfsetstrokecolor{currentstroke}%
\pgfsetdash{}{0pt}%
\pgfpathmoveto{\pgfqpoint{2.831525in}{2.974938in}}%
\pgfpathlineto{\pgfqpoint{2.878591in}{2.906156in}}%
\pgfpathlineto{\pgfqpoint{2.906110in}{2.792406in}}%
\pgfpathlineto{\pgfqpoint{2.859225in}{2.895339in}}%
\pgfpathlineto{\pgfqpoint{2.831525in}{2.974938in}}%
\pgfpathclose%
\pgfusepath{fill}%
\end{pgfscope}%
\begin{pgfscope}%
\pgfpathrectangle{\pgfqpoint{1.072000in}{0.528000in}}{\pgfqpoint{3.696000in}{3.696000in}}%
\pgfusepath{clip}%
\pgfsetbuttcap%
\pgfsetroundjoin%
\definecolor{currentfill}{rgb}{0.280550,0.373423,0.818011}%
\pgfsetfillcolor{currentfill}%
\pgfsetlinewidth{0.000000pt}%
\definecolor{currentstroke}{rgb}{0.000000,0.000000,0.000000}%
\pgfsetstrokecolor{currentstroke}%
\pgfsetdash{}{0pt}%
\pgfpathmoveto{\pgfqpoint{3.318874in}{1.325631in}}%
\pgfpathlineto{\pgfqpoint{3.365672in}{1.261748in}}%
\pgfpathlineto{\pgfqpoint{3.392341in}{1.323190in}}%
\pgfpathlineto{\pgfqpoint{3.345843in}{1.409025in}}%
\pgfpathlineto{\pgfqpoint{3.318874in}{1.325631in}}%
\pgfpathclose%
\pgfusepath{fill}%
\end{pgfscope}%
\begin{pgfscope}%
\pgfpathrectangle{\pgfqpoint{1.072000in}{0.528000in}}{\pgfqpoint{3.696000in}{3.696000in}}%
\pgfusepath{clip}%
\pgfsetbuttcap%
\pgfsetroundjoin%
\definecolor{currentfill}{rgb}{0.830187,0.304733,0.254891}%
\pgfsetfillcolor{currentfill}%
\pgfsetlinewidth{0.000000pt}%
\definecolor{currentstroke}{rgb}{0.000000,0.000000,0.000000}%
\pgfsetstrokecolor{currentstroke}%
\pgfsetdash{}{0pt}%
\pgfpathmoveto{\pgfqpoint{2.497270in}{2.947958in}}%
\pgfpathlineto{\pgfqpoint{2.542370in}{3.034624in}}%
\pgfpathlineto{\pgfqpoint{2.569746in}{3.107212in}}%
\pgfpathlineto{\pgfqpoint{2.524419in}{3.031787in}}%
\pgfpathlineto{\pgfqpoint{2.497270in}{2.947958in}}%
\pgfpathclose%
\pgfusepath{fill}%
\end{pgfscope}%
\begin{pgfscope}%
\pgfpathrectangle{\pgfqpoint{1.072000in}{0.528000in}}{\pgfqpoint{3.696000in}{3.696000in}}%
\pgfusepath{clip}%
\pgfsetbuttcap%
\pgfsetroundjoin%
\definecolor{currentfill}{rgb}{0.711554,0.033337,0.154485}%
\pgfsetfillcolor{currentfill}%
\pgfsetlinewidth{0.000000pt}%
\definecolor{currentstroke}{rgb}{0.000000,0.000000,0.000000}%
\pgfsetstrokecolor{currentstroke}%
\pgfsetdash{}{0pt}%
\pgfpathmoveto{\pgfqpoint{2.671185in}{3.212740in}}%
\pgfpathlineto{\pgfqpoint{2.717956in}{3.180951in}}%
\pgfpathlineto{\pgfqpoint{2.745614in}{3.188808in}}%
\pgfpathlineto{\pgfqpoint{2.698839in}{3.233643in}}%
\pgfpathlineto{\pgfqpoint{2.671185in}{3.212740in}}%
\pgfpathclose%
\pgfusepath{fill}%
\end{pgfscope}%
\begin{pgfscope}%
\pgfpathrectangle{\pgfqpoint{1.072000in}{0.528000in}}{\pgfqpoint{3.696000in}{3.696000in}}%
\pgfusepath{clip}%
\pgfsetbuttcap%
\pgfsetroundjoin%
\definecolor{currentfill}{rgb}{0.234377,0.305542,0.759680}%
\pgfsetfillcolor{currentfill}%
\pgfsetlinewidth{0.000000pt}%
\definecolor{currentstroke}{rgb}{0.000000,0.000000,0.000000}%
\pgfsetstrokecolor{currentstroke}%
\pgfsetdash{}{0pt}%
\pgfpathmoveto{\pgfqpoint{2.418003in}{1.259422in}}%
\pgfpathlineto{\pgfqpoint{2.464370in}{1.243189in}}%
\pgfpathlineto{\pgfqpoint{2.492890in}{1.239920in}}%
\pgfpathlineto{\pgfqpoint{2.446643in}{1.255014in}}%
\pgfpathlineto{\pgfqpoint{2.418003in}{1.259422in}}%
\pgfpathclose%
\pgfusepath{fill}%
\end{pgfscope}%
\begin{pgfscope}%
\pgfpathrectangle{\pgfqpoint{1.072000in}{0.528000in}}{\pgfqpoint{3.696000in}{3.696000in}}%
\pgfusepath{clip}%
\pgfsetbuttcap%
\pgfsetroundjoin%
\definecolor{currentfill}{rgb}{0.877149,0.394645,0.311724}%
\pgfsetfillcolor{currentfill}%
\pgfsetlinewidth{0.000000pt}%
\definecolor{currentstroke}{rgb}{0.000000,0.000000,0.000000}%
\pgfsetstrokecolor{currentstroke}%
\pgfsetdash{}{0pt}%
\pgfpathmoveto{\pgfqpoint{2.470383in}{2.844657in}}%
\pgfpathlineto{\pgfqpoint{2.515236in}{2.940657in}}%
\pgfpathlineto{\pgfqpoint{2.542370in}{3.034624in}}%
\pgfpathlineto{\pgfqpoint{2.497270in}{2.947958in}}%
\pgfpathlineto{\pgfqpoint{2.470383in}{2.844657in}}%
\pgfpathclose%
\pgfusepath{fill}%
\end{pgfscope}%
\begin{pgfscope}%
\pgfpathrectangle{\pgfqpoint{1.072000in}{0.528000in}}{\pgfqpoint{3.696000in}{3.696000in}}%
\pgfusepath{clip}%
\pgfsetbuttcap%
\pgfsetroundjoin%
\definecolor{currentfill}{rgb}{0.785153,0.220851,0.211673}%
\pgfsetfillcolor{currentfill}%
\pgfsetlinewidth{0.000000pt}%
\definecolor{currentstroke}{rgb}{0.000000,0.000000,0.000000}%
\pgfsetstrokecolor{currentstroke}%
\pgfsetdash{}{0pt}%
\pgfpathmoveto{\pgfqpoint{3.201426in}{3.022061in}}%
\pgfpathlineto{\pgfqpoint{3.249272in}{3.096807in}}%
\pgfpathlineto{\pgfqpoint{3.276112in}{3.156840in}}%
\pgfpathlineto{\pgfqpoint{3.228429in}{3.111391in}}%
\pgfpathlineto{\pgfqpoint{3.201426in}{3.022061in}}%
\pgfpathclose%
\pgfusepath{fill}%
\end{pgfscope}%
\begin{pgfscope}%
\pgfpathrectangle{\pgfqpoint{1.072000in}{0.528000in}}{\pgfqpoint{3.696000in}{3.696000in}}%
\pgfusepath{clip}%
\pgfsetbuttcap%
\pgfsetroundjoin%
\definecolor{currentfill}{rgb}{0.252663,0.332837,0.783665}%
\pgfsetfillcolor{currentfill}%
\pgfsetlinewidth{0.000000pt}%
\definecolor{currentstroke}{rgb}{0.000000,0.000000,0.000000}%
\pgfsetstrokecolor{currentstroke}%
\pgfsetdash{}{0pt}%
\pgfpathmoveto{\pgfqpoint{2.613421in}{1.247677in}}%
\pgfpathlineto{\pgfqpoint{2.659984in}{1.228604in}}%
\pgfpathlineto{\pgfqpoint{2.687254in}{1.302384in}}%
\pgfpathlineto{\pgfqpoint{2.640626in}{1.327817in}}%
\pgfpathlineto{\pgfqpoint{2.613421in}{1.247677in}}%
\pgfpathclose%
\pgfusepath{fill}%
\end{pgfscope}%
\begin{pgfscope}%
\pgfpathrectangle{\pgfqpoint{1.072000in}{0.528000in}}{\pgfqpoint{3.696000in}{3.696000in}}%
\pgfusepath{clip}%
\pgfsetbuttcap%
\pgfsetroundjoin%
\definecolor{currentfill}{rgb}{0.229806,0.298718,0.753683}%
\pgfsetfillcolor{currentfill}%
\pgfsetlinewidth{0.000000pt}%
\definecolor{currentstroke}{rgb}{0.000000,0.000000,0.000000}%
\pgfsetstrokecolor{currentstroke}%
\pgfsetdash{}{0pt}%
\pgfpathmoveto{\pgfqpoint{2.539240in}{1.225179in}}%
\pgfpathlineto{\pgfqpoint{2.585693in}{1.210976in}}%
\pgfpathlineto{\pgfqpoint{2.613421in}{1.247677in}}%
\pgfpathlineto{\pgfqpoint{2.566970in}{1.265479in}}%
\pgfpathlineto{\pgfqpoint{2.539240in}{1.225179in}}%
\pgfpathclose%
\pgfusepath{fill}%
\end{pgfscope}%
\begin{pgfscope}%
\pgfpathrectangle{\pgfqpoint{1.072000in}{0.528000in}}{\pgfqpoint{3.696000in}{3.696000in}}%
\pgfusepath{clip}%
\pgfsetbuttcap%
\pgfsetroundjoin%
\definecolor{currentfill}{rgb}{0.960581,0.762501,0.667964}%
\pgfsetfillcolor{currentfill}%
\pgfsetlinewidth{0.000000pt}%
\definecolor{currentstroke}{rgb}{0.000000,0.000000,0.000000}%
\pgfsetstrokecolor{currentstroke}%
\pgfsetdash{}{0pt}%
\pgfpathmoveto{\pgfqpoint{3.524605in}{2.567539in}}%
\pgfpathlineto{\pgfqpoint{3.569239in}{2.373900in}}%
\pgfpathlineto{\pgfqpoint{3.594886in}{2.389722in}}%
\pgfpathlineto{\pgfqpoint{3.550453in}{2.582835in}}%
\pgfpathlineto{\pgfqpoint{3.524605in}{2.567539in}}%
\pgfpathclose%
\pgfusepath{fill}%
\end{pgfscope}%
\begin{pgfscope}%
\pgfpathrectangle{\pgfqpoint{1.072000in}{0.528000in}}{\pgfqpoint{3.696000in}{3.696000in}}%
\pgfusepath{clip}%
\pgfsetbuttcap%
\pgfsetroundjoin%
\definecolor{currentfill}{rgb}{0.313946,0.420052,0.854993}%
\pgfsetfillcolor{currentfill}%
\pgfsetlinewidth{0.000000pt}%
\definecolor{currentstroke}{rgb}{0.000000,0.000000,0.000000}%
\pgfsetstrokecolor{currentstroke}%
\pgfsetdash{}{0pt}%
\pgfpathmoveto{\pgfqpoint{2.640626in}{1.327817in}}%
\pgfpathlineto{\pgfqpoint{2.687254in}{1.302384in}}%
\pgfpathlineto{\pgfqpoint{2.714184in}{1.416276in}}%
\pgfpathlineto{\pgfqpoint{2.667447in}{1.448200in}}%
\pgfpathlineto{\pgfqpoint{2.640626in}{1.327817in}}%
\pgfpathclose%
\pgfusepath{fill}%
\end{pgfscope}%
\begin{pgfscope}%
\pgfpathrectangle{\pgfqpoint{1.072000in}{0.528000in}}{\pgfqpoint{3.696000in}{3.696000in}}%
\pgfusepath{clip}%
\pgfsetbuttcap%
\pgfsetroundjoin%
\definecolor{currentfill}{rgb}{0.229806,0.298718,0.753683}%
\pgfsetfillcolor{currentfill}%
\pgfsetlinewidth{0.000000pt}%
\definecolor{currentstroke}{rgb}{0.000000,0.000000,0.000000}%
\pgfsetstrokecolor{currentstroke}%
\pgfsetdash{}{0pt}%
\pgfpathmoveto{\pgfqpoint{3.412785in}{1.223719in}}%
\pgfpathlineto{\pgfqpoint{3.460368in}{1.214592in}}%
\pgfpathlineto{\pgfqpoint{3.486332in}{1.233276in}}%
\pgfpathlineto{\pgfqpoint{3.439115in}{1.263341in}}%
\pgfpathlineto{\pgfqpoint{3.412785in}{1.223719in}}%
\pgfpathclose%
\pgfusepath{fill}%
\end{pgfscope}%
\begin{pgfscope}%
\pgfpathrectangle{\pgfqpoint{1.072000in}{0.528000in}}{\pgfqpoint{3.696000in}{3.696000in}}%
\pgfusepath{clip}%
\pgfsetbuttcap%
\pgfsetroundjoin%
\definecolor{currentfill}{rgb}{0.763520,0.178667,0.193396}%
\pgfsetfillcolor{currentfill}%
\pgfsetlinewidth{0.000000pt}%
\definecolor{currentstroke}{rgb}{0.000000,0.000000,0.000000}%
\pgfsetstrokecolor{currentstroke}%
\pgfsetdash{}{0pt}%
\pgfpathmoveto{\pgfqpoint{3.397484in}{3.171954in}}%
\pgfpathlineto{\pgfqpoint{3.444311in}{3.105380in}}%
\pgfpathlineto{\pgfqpoint{3.469700in}{3.077129in}}%
\pgfpathlineto{\pgfqpoint{3.423248in}{3.159403in}}%
\pgfpathlineto{\pgfqpoint{3.397484in}{3.171954in}}%
\pgfpathclose%
\pgfusepath{fill}%
\end{pgfscope}%
\begin{pgfscope}%
\pgfpathrectangle{\pgfqpoint{1.072000in}{0.528000in}}{\pgfqpoint{3.696000in}{3.696000in}}%
\pgfusepath{clip}%
\pgfsetbuttcap%
\pgfsetroundjoin%
\definecolor{currentfill}{rgb}{0.748682,0.827679,0.963334}%
\pgfsetfillcolor{currentfill}%
\pgfsetlinewidth{0.000000pt}%
\definecolor{currentstroke}{rgb}{0.000000,0.000000,0.000000}%
\pgfsetstrokecolor{currentstroke}%
\pgfsetdash{}{0pt}%
\pgfpathmoveto{\pgfqpoint{1.756568in}{2.138942in}}%
\pgfpathlineto{\pgfqpoint{1.805981in}{2.016789in}}%
\pgfpathlineto{\pgfqpoint{1.840187in}{1.897768in}}%
\pgfpathlineto{\pgfqpoint{1.791184in}{2.017934in}}%
\pgfpathlineto{\pgfqpoint{1.756568in}{2.138942in}}%
\pgfpathclose%
\pgfusepath{fill}%
\end{pgfscope}%
\begin{pgfscope}%
\pgfpathrectangle{\pgfqpoint{1.072000in}{0.528000in}}{\pgfqpoint{3.696000in}{3.696000in}}%
\pgfusepath{clip}%
\pgfsetbuttcap%
\pgfsetroundjoin%
\definecolor{currentfill}{rgb}{0.717435,0.051118,0.158737}%
\pgfsetfillcolor{currentfill}%
\pgfsetlinewidth{0.000000pt}%
\definecolor{currentstroke}{rgb}{0.000000,0.000000,0.000000}%
\pgfsetstrokecolor{currentstroke}%
\pgfsetdash{}{0pt}%
\pgfpathmoveto{\pgfqpoint{3.350261in}{3.204761in}}%
\pgfpathlineto{\pgfqpoint{3.397484in}{3.171954in}}%
\pgfpathlineto{\pgfqpoint{3.423248in}{3.159403in}}%
\pgfpathlineto{\pgfqpoint{3.376382in}{3.212193in}}%
\pgfpathlineto{\pgfqpoint{3.350261in}{3.204761in}}%
\pgfpathclose%
\pgfusepath{fill}%
\end{pgfscope}%
\begin{pgfscope}%
\pgfpathrectangle{\pgfqpoint{1.072000in}{0.528000in}}{\pgfqpoint{3.696000in}{3.696000in}}%
\pgfusepath{clip}%
\pgfsetbuttcap%
\pgfsetroundjoin%
\definecolor{currentfill}{rgb}{0.698454,0.799450,0.984577}%
\pgfsetfillcolor{currentfill}%
\pgfsetlinewidth{0.000000pt}%
\definecolor{currentstroke}{rgb}{0.000000,0.000000,0.000000}%
\pgfsetstrokecolor{currentstroke}%
\pgfsetdash{}{0pt}%
\pgfpathmoveto{\pgfqpoint{2.488193in}{1.827772in}}%
\pgfpathlineto{\pgfqpoint{2.534104in}{1.843021in}}%
\pgfpathlineto{\pgfqpoint{2.559908in}{2.042837in}}%
\pgfpathlineto{\pgfqpoint{2.513864in}{2.022462in}}%
\pgfpathlineto{\pgfqpoint{2.488193in}{1.827772in}}%
\pgfpathclose%
\pgfusepath{fill}%
\end{pgfscope}%
\begin{pgfscope}%
\pgfpathrectangle{\pgfqpoint{1.072000in}{0.528000in}}{\pgfqpoint{3.696000in}{3.696000in}}%
\pgfusepath{clip}%
\pgfsetbuttcap%
\pgfsetroundjoin%
\definecolor{currentfill}{rgb}{0.960581,0.762501,0.667964}%
\pgfsetfillcolor{currentfill}%
\pgfsetlinewidth{0.000000pt}%
\definecolor{currentstroke}{rgb}{0.000000,0.000000,0.000000}%
\pgfsetstrokecolor{currentstroke}%
\pgfsetdash{}{0pt}%
\pgfpathmoveto{\pgfqpoint{1.575364in}{2.600994in}}%
\pgfpathlineto{\pgfqpoint{1.625907in}{2.458106in}}%
\pgfpathlineto{\pgfqpoint{1.660420in}{2.357328in}}%
\pgfpathlineto{\pgfqpoint{1.609876in}{2.508693in}}%
\pgfpathlineto{\pgfqpoint{1.575364in}{2.600994in}}%
\pgfpathclose%
\pgfusepath{fill}%
\end{pgfscope}%
\begin{pgfscope}%
\pgfpathrectangle{\pgfqpoint{1.072000in}{0.528000in}}{\pgfqpoint{3.696000in}{3.696000in}}%
\pgfusepath{clip}%
\pgfsetbuttcap%
\pgfsetroundjoin%
\definecolor{currentfill}{rgb}{0.711554,0.033337,0.154485}%
\pgfsetfillcolor{currentfill}%
\pgfsetlinewidth{0.000000pt}%
\definecolor{currentstroke}{rgb}{0.000000,0.000000,0.000000}%
\pgfsetstrokecolor{currentstroke}%
\pgfsetdash{}{0pt}%
\pgfpathmoveto{\pgfqpoint{2.597281in}{3.161522in}}%
\pgfpathlineto{\pgfqpoint{2.643465in}{3.186555in}}%
\pgfpathlineto{\pgfqpoint{2.671185in}{3.212740in}}%
\pgfpathlineto{\pgfqpoint{2.624890in}{3.201803in}}%
\pgfpathlineto{\pgfqpoint{2.597281in}{3.161522in}}%
\pgfpathclose%
\pgfusepath{fill}%
\end{pgfscope}%
\begin{pgfscope}%
\pgfpathrectangle{\pgfqpoint{1.072000in}{0.528000in}}{\pgfqpoint{3.696000in}{3.696000in}}%
\pgfusepath{clip}%
\pgfsetbuttcap%
\pgfsetroundjoin%
\definecolor{currentfill}{rgb}{0.830187,0.304733,0.254891}%
\pgfsetfillcolor{currentfill}%
\pgfsetlinewidth{0.000000pt}%
\definecolor{currentstroke}{rgb}{0.000000,0.000000,0.000000}%
\pgfsetstrokecolor{currentstroke}%
\pgfsetdash{}{0pt}%
\pgfpathmoveto{\pgfqpoint{3.174482in}{2.944108in}}%
\pgfpathlineto{\pgfqpoint{3.222425in}{3.042911in}}%
\pgfpathlineto{\pgfqpoint{3.249272in}{3.096807in}}%
\pgfpathlineto{\pgfqpoint{3.201426in}{3.022061in}}%
\pgfpathlineto{\pgfqpoint{3.174482in}{2.944108in}}%
\pgfpathclose%
\pgfusepath{fill}%
\end{pgfscope}%
\begin{pgfscope}%
\pgfpathrectangle{\pgfqpoint{1.072000in}{0.528000in}}{\pgfqpoint{3.696000in}{3.696000in}}%
\pgfusepath{clip}%
\pgfsetbuttcap%
\pgfsetroundjoin%
\definecolor{currentfill}{rgb}{0.425199,0.559058,0.946061}%
\pgfsetfillcolor{currentfill}%
\pgfsetlinewidth{0.000000pt}%
\definecolor{currentstroke}{rgb}{0.000000,0.000000,0.000000}%
\pgfsetstrokecolor{currentstroke}%
\pgfsetdash{}{0pt}%
\pgfpathmoveto{\pgfqpoint{2.620837in}{1.473684in}}%
\pgfpathlineto{\pgfqpoint{2.667447in}{1.448200in}}%
\pgfpathlineto{\pgfqpoint{2.694027in}{1.603380in}}%
\pgfpathlineto{\pgfqpoint{2.647274in}{1.632579in}}%
\pgfpathlineto{\pgfqpoint{2.620837in}{1.473684in}}%
\pgfpathclose%
\pgfusepath{fill}%
\end{pgfscope}%
\begin{pgfscope}%
\pgfpathrectangle{\pgfqpoint{1.072000in}{0.528000in}}{\pgfqpoint{3.696000in}{3.696000in}}%
\pgfusepath{clip}%
\pgfsetbuttcap%
\pgfsetroundjoin%
\definecolor{currentfill}{rgb}{0.248091,0.326013,0.777669}%
\pgfsetfillcolor{currentfill}%
\pgfsetlinewidth{0.000000pt}%
\definecolor{currentstroke}{rgb}{0.000000,0.000000,0.000000}%
\pgfsetstrokecolor{currentstroke}%
\pgfsetdash{}{0pt}%
\pgfpathmoveto{\pgfqpoint{3.292026in}{1.256360in}}%
\pgfpathlineto{\pgfqpoint{3.339117in}{1.215165in}}%
\pgfpathlineto{\pgfqpoint{3.365672in}{1.261748in}}%
\pgfpathlineto{\pgfqpoint{3.318874in}{1.325631in}}%
\pgfpathlineto{\pgfqpoint{3.292026in}{1.256360in}}%
\pgfpathclose%
\pgfusepath{fill}%
\end{pgfscope}%
\begin{pgfscope}%
\pgfpathrectangle{\pgfqpoint{1.072000in}{0.528000in}}{\pgfqpoint{3.696000in}{3.696000in}}%
\pgfusepath{clip}%
\pgfsetbuttcap%
\pgfsetroundjoin%
\definecolor{currentfill}{rgb}{0.940879,0.805596,0.735167}%
\pgfsetfillcolor{currentfill}%
\pgfsetlinewidth{0.000000pt}%
\definecolor{currentstroke}{rgb}{0.000000,0.000000,0.000000}%
\pgfsetstrokecolor{currentstroke}%
\pgfsetdash{}{0pt}%
\pgfpathmoveto{\pgfqpoint{2.429001in}{2.261386in}}%
\pgfpathlineto{\pgfqpoint{2.474016in}{2.330583in}}%
\pgfpathlineto{\pgfqpoint{2.499856in}{2.518153in}}%
\pgfpathlineto{\pgfqpoint{2.454671in}{2.445119in}}%
\pgfpathlineto{\pgfqpoint{2.429001in}{2.261386in}}%
\pgfpathclose%
\pgfusepath{fill}%
\end{pgfscope}%
\begin{pgfscope}%
\pgfpathrectangle{\pgfqpoint{1.072000in}{0.528000in}}{\pgfqpoint{3.696000in}{3.696000in}}%
\pgfusepath{clip}%
\pgfsetbuttcap%
\pgfsetroundjoin%
\definecolor{currentfill}{rgb}{0.929357,0.512254,0.400673}%
\pgfsetfillcolor{currentfill}%
\pgfsetlinewidth{0.000000pt}%
\definecolor{currentstroke}{rgb}{0.000000,0.000000,0.000000}%
\pgfsetstrokecolor{currentstroke}%
\pgfsetdash{}{0pt}%
\pgfpathmoveto{\pgfqpoint{2.925575in}{2.871029in}}%
\pgfpathlineto{\pgfqpoint{2.972614in}{2.880085in}}%
\pgfpathlineto{\pgfqpoint{2.999642in}{2.766222in}}%
\pgfpathlineto{\pgfqpoint{2.952840in}{2.741032in}}%
\pgfpathlineto{\pgfqpoint{2.925575in}{2.871029in}}%
\pgfpathclose%
\pgfusepath{fill}%
\end{pgfscope}%
\begin{pgfscope}%
\pgfpathrectangle{\pgfqpoint{1.072000in}{0.528000in}}{\pgfqpoint{3.696000in}{3.696000in}}%
\pgfusepath{clip}%
\pgfsetbuttcap%
\pgfsetroundjoin%
\definecolor{currentfill}{rgb}{0.343278,0.459354,0.884122}%
\pgfsetfillcolor{currentfill}%
\pgfsetlinewidth{0.000000pt}%
\definecolor{currentstroke}{rgb}{0.000000,0.000000,0.000000}%
\pgfsetstrokecolor{currentstroke}%
\pgfsetdash{}{0pt}%
\pgfpathmoveto{\pgfqpoint{3.272245in}{1.411267in}}%
\pgfpathlineto{\pgfqpoint{3.318874in}{1.325631in}}%
\pgfpathlineto{\pgfqpoint{3.345843in}{1.409025in}}%
\pgfpathlineto{\pgfqpoint{3.299461in}{1.515988in}}%
\pgfpathlineto{\pgfqpoint{3.272245in}{1.411267in}}%
\pgfpathclose%
\pgfusepath{fill}%
\end{pgfscope}%
\begin{pgfscope}%
\pgfpathrectangle{\pgfqpoint{1.072000in}{0.528000in}}{\pgfqpoint{3.696000in}{3.696000in}}%
\pgfusepath{clip}%
\pgfsetbuttcap%
\pgfsetroundjoin%
\definecolor{currentfill}{rgb}{0.839351,0.861167,0.894494}%
\pgfsetfillcolor{currentfill}%
\pgfsetlinewidth{0.000000pt}%
\definecolor{currentstroke}{rgb}{0.000000,0.000000,0.000000}%
\pgfsetstrokecolor{currentstroke}%
\pgfsetdash{}{0pt}%
\pgfpathmoveto{\pgfqpoint{3.490212in}{2.234241in}}%
\pgfpathlineto{\pgfqpoint{3.534972in}{2.040951in}}%
\pgfpathlineto{\pgfqpoint{3.561421in}{2.099425in}}%
\pgfpathlineto{\pgfqpoint{3.516865in}{2.295966in}}%
\pgfpathlineto{\pgfqpoint{3.490212in}{2.234241in}}%
\pgfpathclose%
\pgfusepath{fill}%
\end{pgfscope}%
\begin{pgfscope}%
\pgfpathrectangle{\pgfqpoint{1.072000in}{0.528000in}}{\pgfqpoint{3.696000in}{3.696000in}}%
\pgfusepath{clip}%
\pgfsetbuttcap%
\pgfsetroundjoin%
\definecolor{currentfill}{rgb}{0.299441,0.400248,0.839842}%
\pgfsetfillcolor{currentfill}%
\pgfsetlinewidth{0.000000pt}%
\definecolor{currentstroke}{rgb}{0.000000,0.000000,0.000000}%
\pgfsetstrokecolor{currentstroke}%
\pgfsetdash{}{0pt}%
\pgfpathmoveto{\pgfqpoint{2.687254in}{1.302384in}}%
\pgfpathlineto{\pgfqpoint{2.733980in}{1.274277in}}%
\pgfpathlineto{\pgfqpoint{2.761016in}{1.379130in}}%
\pgfpathlineto{\pgfqpoint{2.714184in}{1.416276in}}%
\pgfpathlineto{\pgfqpoint{2.687254in}{1.302384in}}%
\pgfpathclose%
\pgfusepath{fill}%
\end{pgfscope}%
\begin{pgfscope}%
\pgfpathrectangle{\pgfqpoint{1.072000in}{0.528000in}}{\pgfqpoint{3.696000in}{3.696000in}}%
\pgfusepath{clip}%
\pgfsetbuttcap%
\pgfsetroundjoin%
\definecolor{currentfill}{rgb}{0.869655,0.379274,0.300941}%
\pgfsetfillcolor{currentfill}%
\pgfsetlinewidth{0.000000pt}%
\definecolor{currentstroke}{rgb}{0.000000,0.000000,0.000000}%
\pgfsetstrokecolor{currentstroke}%
\pgfsetdash{}{0pt}%
\pgfpathmoveto{\pgfqpoint{3.465132in}{3.025932in}}%
\pgfpathlineto{\pgfqpoint{3.511097in}{2.900102in}}%
\pgfpathlineto{\pgfqpoint{3.536234in}{2.871345in}}%
\pgfpathlineto{\pgfqpoint{3.490593in}{3.004282in}}%
\pgfpathlineto{\pgfqpoint{3.465132in}{3.025932in}}%
\pgfpathclose%
\pgfusepath{fill}%
\end{pgfscope}%
\begin{pgfscope}%
\pgfpathrectangle{\pgfqpoint{1.072000in}{0.528000in}}{\pgfqpoint{3.696000in}{3.696000in}}%
\pgfusepath{clip}%
\pgfsetbuttcap%
\pgfsetroundjoin%
\definecolor{currentfill}{rgb}{0.852378,0.346492,0.280346}%
\pgfsetfillcolor{currentfill}%
\pgfsetlinewidth{0.000000pt}%
\definecolor{currentstroke}{rgb}{0.000000,0.000000,0.000000}%
\pgfsetstrokecolor{currentstroke}%
\pgfsetdash{}{0pt}%
\pgfpathmoveto{\pgfqpoint{2.784360in}{3.055264in}}%
\pgfpathlineto{\pgfqpoint{2.831525in}{2.974938in}}%
\pgfpathlineto{\pgfqpoint{2.859225in}{2.895339in}}%
\pgfpathlineto{\pgfqpoint{2.812156in}{3.010150in}}%
\pgfpathlineto{\pgfqpoint{2.784360in}{3.055264in}}%
\pgfpathclose%
\pgfusepath{fill}%
\end{pgfscope}%
\begin{pgfscope}%
\pgfpathrectangle{\pgfqpoint{1.072000in}{0.528000in}}{\pgfqpoint{3.696000in}{3.696000in}}%
\pgfusepath{clip}%
\pgfsetbuttcap%
\pgfsetroundjoin%
\definecolor{currentfill}{rgb}{0.409611,0.540759,0.935545}%
\pgfsetfillcolor{currentfill}%
\pgfsetlinewidth{0.000000pt}%
\definecolor{currentstroke}{rgb}{0.000000,0.000000,0.000000}%
\pgfsetstrokecolor{currentstroke}%
\pgfsetdash{}{0pt}%
\pgfpathmoveto{\pgfqpoint{3.299461in}{1.515988in}}%
\pgfpathlineto{\pgfqpoint{3.345843in}{1.409025in}}%
\pgfpathlineto{\pgfqpoint{3.372912in}{1.502188in}}%
\pgfpathlineto{\pgfqpoint{3.326779in}{1.628749in}}%
\pgfpathlineto{\pgfqpoint{3.299461in}{1.515988in}}%
\pgfpathclose%
\pgfusepath{fill}%
\end{pgfscope}%
\begin{pgfscope}%
\pgfpathrectangle{\pgfqpoint{1.072000in}{0.528000in}}{\pgfqpoint{3.696000in}{3.696000in}}%
\pgfusepath{clip}%
\pgfsetbuttcap%
\pgfsetroundjoin%
\definecolor{currentfill}{rgb}{0.576051,0.708780,0.997755}%
\pgfsetfillcolor{currentfill}%
\pgfsetlinewidth{0.000000pt}%
\definecolor{currentstroke}{rgb}{0.000000,0.000000,0.000000}%
\pgfsetstrokecolor{currentstroke}%
\pgfsetdash{}{0pt}%
\pgfpathmoveto{\pgfqpoint{2.554336in}{1.661487in}}%
\pgfpathlineto{\pgfqpoint{2.600696in}{1.652024in}}%
\pgfpathlineto{\pgfqpoint{2.626841in}{1.838835in}}%
\pgfpathlineto{\pgfqpoint{2.580333in}{1.846953in}}%
\pgfpathlineto{\pgfqpoint{2.554336in}{1.661487in}}%
\pgfpathclose%
\pgfusepath{fill}%
\end{pgfscope}%
\begin{pgfscope}%
\pgfpathrectangle{\pgfqpoint{1.072000in}{0.528000in}}{\pgfqpoint{3.696000in}{3.696000in}}%
\pgfusepath{clip}%
\pgfsetbuttcap%
\pgfsetroundjoin%
\definecolor{currentfill}{rgb}{0.906154,0.842091,0.806151}%
\pgfsetfillcolor{currentfill}%
\pgfsetlinewidth{0.000000pt}%
\definecolor{currentstroke}{rgb}{0.000000,0.000000,0.000000}%
\pgfsetstrokecolor{currentstroke}%
\pgfsetdash{}{0pt}%
\pgfpathmoveto{\pgfqpoint{1.641498in}{2.421031in}}%
\pgfpathlineto{\pgfqpoint{1.691750in}{2.282711in}}%
\pgfpathlineto{\pgfqpoint{1.726382in}{2.170087in}}%
\pgfpathlineto{\pgfqpoint{1.676308in}{2.312863in}}%
\pgfpathlineto{\pgfqpoint{1.641498in}{2.421031in}}%
\pgfpathclose%
\pgfusepath{fill}%
\end{pgfscope}%
\begin{pgfscope}%
\pgfpathrectangle{\pgfqpoint{1.072000in}{0.528000in}}{\pgfqpoint{3.696000in}{3.696000in}}%
\pgfusepath{clip}%
\pgfsetbuttcap%
\pgfsetroundjoin%
\definecolor{currentfill}{rgb}{0.289996,0.386836,0.828926}%
\pgfsetfillcolor{currentfill}%
\pgfsetlinewidth{0.000000pt}%
\definecolor{currentstroke}{rgb}{0.000000,0.000000,0.000000}%
\pgfsetstrokecolor{currentstroke}%
\pgfsetdash{}{0pt}%
\pgfpathmoveto{\pgfqpoint{3.245152in}{1.319512in}}%
\pgfpathlineto{\pgfqpoint{3.292026in}{1.256360in}}%
\pgfpathlineto{\pgfqpoint{3.318874in}{1.325631in}}%
\pgfpathlineto{\pgfqpoint{3.272245in}{1.411267in}}%
\pgfpathlineto{\pgfqpoint{3.245152in}{1.319512in}}%
\pgfpathclose%
\pgfusepath{fill}%
\end{pgfscope}%
\begin{pgfscope}%
\pgfpathrectangle{\pgfqpoint{1.072000in}{0.528000in}}{\pgfqpoint{3.696000in}{3.696000in}}%
\pgfusepath{clip}%
\pgfsetbuttcap%
\pgfsetroundjoin%
\definecolor{currentfill}{rgb}{0.243520,0.319189,0.771672}%
\pgfsetfillcolor{currentfill}%
\pgfsetlinewidth{0.000000pt}%
\definecolor{currentstroke}{rgb}{0.000000,0.000000,0.000000}%
\pgfsetstrokecolor{currentstroke}%
\pgfsetdash{}{0pt}%
\pgfpathmoveto{\pgfqpoint{2.659984in}{1.228604in}}%
\pgfpathlineto{\pgfqpoint{2.706647in}{1.209286in}}%
\pgfpathlineto{\pgfqpoint{2.733980in}{1.274277in}}%
\pgfpathlineto{\pgfqpoint{2.687254in}{1.302384in}}%
\pgfpathlineto{\pgfqpoint{2.659984in}{1.228604in}}%
\pgfpathclose%
\pgfusepath{fill}%
\end{pgfscope}%
\begin{pgfscope}%
\pgfpathrectangle{\pgfqpoint{1.072000in}{0.528000in}}{\pgfqpoint{3.696000in}{3.696000in}}%
\pgfusepath{clip}%
\pgfsetbuttcap%
\pgfsetroundjoin%
\definecolor{currentfill}{rgb}{0.809329,0.852974,0.922323}%
\pgfsetfillcolor{currentfill}%
\pgfsetlinewidth{0.000000pt}%
\definecolor{currentstroke}{rgb}{0.000000,0.000000,0.000000}%
\pgfsetstrokecolor{currentstroke}%
\pgfsetdash{}{0pt}%
\pgfpathmoveto{\pgfqpoint{2.468202in}{1.989807in}}%
\pgfpathlineto{\pgfqpoint{2.513864in}{2.022462in}}%
\pgfpathlineto{\pgfqpoint{2.539590in}{2.226110in}}%
\pgfpathlineto{\pgfqpoint{2.493786in}{2.187897in}}%
\pgfpathlineto{\pgfqpoint{2.468202in}{1.989807in}}%
\pgfpathclose%
\pgfusepath{fill}%
\end{pgfscope}%
\begin{pgfscope}%
\pgfpathrectangle{\pgfqpoint{1.072000in}{0.528000in}}{\pgfqpoint{3.696000in}{3.696000in}}%
\pgfusepath{clip}%
\pgfsetbuttcap%
\pgfsetroundjoin%
\definecolor{currentfill}{rgb}{0.229806,0.298718,0.753683}%
\pgfsetfillcolor{currentfill}%
\pgfsetlinewidth{0.000000pt}%
\definecolor{currentstroke}{rgb}{0.000000,0.000000,0.000000}%
\pgfsetstrokecolor{currentstroke}%
\pgfsetdash{}{0pt}%
\pgfpathmoveto{\pgfqpoint{3.339117in}{1.215165in}}%
\pgfpathlineto{\pgfqpoint{3.386561in}{1.199287in}}%
\pgfpathlineto{\pgfqpoint{3.412785in}{1.223719in}}%
\pgfpathlineto{\pgfqpoint{3.365672in}{1.261748in}}%
\pgfpathlineto{\pgfqpoint{3.339117in}{1.215165in}}%
\pgfpathclose%
\pgfusepath{fill}%
\end{pgfscope}%
\begin{pgfscope}%
\pgfpathrectangle{\pgfqpoint{1.072000in}{0.528000in}}{\pgfqpoint{3.696000in}{3.696000in}}%
\pgfusepath{clip}%
\pgfsetbuttcap%
\pgfsetroundjoin%
\definecolor{currentfill}{rgb}{0.271104,0.360011,0.807095}%
\pgfsetfillcolor{currentfill}%
\pgfsetlinewidth{0.000000pt}%
\definecolor{currentstroke}{rgb}{0.000000,0.000000,0.000000}%
\pgfsetstrokecolor{currentstroke}%
\pgfsetdash{}{0pt}%
\pgfpathmoveto{\pgfqpoint{3.608051in}{1.265632in}}%
\pgfpathlineto{\pgfqpoint{3.657196in}{1.318639in}}%
\pgfpathlineto{\pgfqpoint{3.682188in}{1.309445in}}%
\pgfpathlineto{\pgfqpoint{3.633403in}{1.271231in}}%
\pgfpathlineto{\pgfqpoint{3.608051in}{1.265632in}}%
\pgfpathclose%
\pgfusepath{fill}%
\end{pgfscope}%
\begin{pgfscope}%
\pgfpathrectangle{\pgfqpoint{1.072000in}{0.528000in}}{\pgfqpoint{3.696000in}{3.696000in}}%
\pgfusepath{clip}%
\pgfsetbuttcap%
\pgfsetroundjoin%
\definecolor{currentfill}{rgb}{0.234377,0.305542,0.759680}%
\pgfsetfillcolor{currentfill}%
\pgfsetlinewidth{0.000000pt}%
\definecolor{currentstroke}{rgb}{0.000000,0.000000,0.000000}%
\pgfsetstrokecolor{currentstroke}%
\pgfsetdash{}{0pt}%
\pgfpathmoveto{\pgfqpoint{2.464370in}{1.243189in}}%
\pgfpathlineto{\pgfqpoint{2.510806in}{1.229509in}}%
\pgfpathlineto{\pgfqpoint{2.539240in}{1.225179in}}%
\pgfpathlineto{\pgfqpoint{2.492890in}{1.239920in}}%
\pgfpathlineto{\pgfqpoint{2.464370in}{1.243189in}}%
\pgfpathclose%
\pgfusepath{fill}%
\end{pgfscope}%
\begin{pgfscope}%
\pgfpathrectangle{\pgfqpoint{1.072000in}{0.528000in}}{\pgfqpoint{3.696000in}{3.696000in}}%
\pgfusepath{clip}%
\pgfsetbuttcap%
\pgfsetroundjoin%
\definecolor{currentfill}{rgb}{0.229806,0.298718,0.753683}%
\pgfsetfillcolor{currentfill}%
\pgfsetlinewidth{0.000000pt}%
\definecolor{currentstroke}{rgb}{0.000000,0.000000,0.000000}%
\pgfsetstrokecolor{currentstroke}%
\pgfsetdash{}{0pt}%
\pgfpathmoveto{\pgfqpoint{2.585693in}{1.210976in}}%
\pgfpathlineto{\pgfqpoint{2.632246in}{1.197857in}}%
\pgfpathlineto{\pgfqpoint{2.659984in}{1.228604in}}%
\pgfpathlineto{\pgfqpoint{2.613421in}{1.247677in}}%
\pgfpathlineto{\pgfqpoint{2.585693in}{1.210976in}}%
\pgfpathclose%
\pgfusepath{fill}%
\end{pgfscope}%
\begin{pgfscope}%
\pgfpathrectangle{\pgfqpoint{1.072000in}{0.528000in}}{\pgfqpoint{3.696000in}{3.696000in}}%
\pgfusepath{clip}%
\pgfsetbuttcap%
\pgfsetroundjoin%
\definecolor{currentfill}{rgb}{0.483854,0.622050,0.974808}%
\pgfsetfillcolor{currentfill}%
\pgfsetlinewidth{0.000000pt}%
\definecolor{currentstroke}{rgb}{0.000000,0.000000,0.000000}%
\pgfsetstrokecolor{currentstroke}%
\pgfsetdash{}{0pt}%
\pgfpathmoveto{\pgfqpoint{3.326779in}{1.628749in}}%
\pgfpathlineto{\pgfqpoint{3.372912in}{1.502188in}}%
\pgfpathlineto{\pgfqpoint{3.400048in}{1.600706in}}%
\pgfpathlineto{\pgfqpoint{3.354162in}{1.744685in}}%
\pgfpathlineto{\pgfqpoint{3.326779in}{1.628749in}}%
\pgfpathclose%
\pgfusepath{fill}%
\end{pgfscope}%
\begin{pgfscope}%
\pgfpathrectangle{\pgfqpoint{1.072000in}{0.528000in}}{\pgfqpoint{3.696000in}{3.696000in}}%
\pgfusepath{clip}%
\pgfsetbuttcap%
\pgfsetroundjoin%
\definecolor{currentfill}{rgb}{0.248091,0.326013,0.777669}%
\pgfsetfillcolor{currentfill}%
\pgfsetlinewidth{0.000000pt}%
\definecolor{currentstroke}{rgb}{0.000000,0.000000,0.000000}%
\pgfsetstrokecolor{currentstroke}%
\pgfsetdash{}{0pt}%
\pgfpathmoveto{\pgfqpoint{3.534162in}{1.235463in}}%
\pgfpathlineto{\pgfqpoint{3.582762in}{1.270841in}}%
\pgfpathlineto{\pgfqpoint{3.608051in}{1.265632in}}%
\pgfpathlineto{\pgfqpoint{3.559829in}{1.247540in}}%
\pgfpathlineto{\pgfqpoint{3.534162in}{1.235463in}}%
\pgfpathclose%
\pgfusepath{fill}%
\end{pgfscope}%
\begin{pgfscope}%
\pgfpathrectangle{\pgfqpoint{1.072000in}{0.528000in}}{\pgfqpoint{3.696000in}{3.696000in}}%
\pgfusepath{clip}%
\pgfsetbuttcap%
\pgfsetroundjoin%
\definecolor{currentfill}{rgb}{0.891817,0.851973,0.829085}%
\pgfsetfillcolor{currentfill}%
\pgfsetlinewidth{0.000000pt}%
\definecolor{currentstroke}{rgb}{0.000000,0.000000,0.000000}%
\pgfsetstrokecolor{currentstroke}%
\pgfsetdash{}{0pt}%
\pgfpathmoveto{\pgfqpoint{2.448424in}{2.136882in}}%
\pgfpathlineto{\pgfqpoint{2.493786in}{2.187897in}}%
\pgfpathlineto{\pgfqpoint{2.519533in}{2.386619in}}%
\pgfpathlineto{\pgfqpoint{2.474016in}{2.330583in}}%
\pgfpathlineto{\pgfqpoint{2.448424in}{2.136882in}}%
\pgfpathclose%
\pgfusepath{fill}%
\end{pgfscope}%
\begin{pgfscope}%
\pgfpathrectangle{\pgfqpoint{1.072000in}{0.528000in}}{\pgfqpoint{3.696000in}{3.696000in}}%
\pgfusepath{clip}%
\pgfsetbuttcap%
\pgfsetroundjoin%
\definecolor{currentfill}{rgb}{0.964911,0.640159,0.519806}%
\pgfsetfillcolor{currentfill}%
\pgfsetlinewidth{0.000000pt}%
\definecolor{currentstroke}{rgb}{0.000000,0.000000,0.000000}%
\pgfsetstrokecolor{currentstroke}%
\pgfsetdash{}{0pt}%
\pgfpathmoveto{\pgfqpoint{2.435834in}{2.528892in}}%
\pgfpathlineto{\pgfqpoint{2.480641in}{2.618006in}}%
\pgfpathlineto{\pgfqpoint{2.506977in}{2.773065in}}%
\pgfpathlineto{\pgfqpoint{2.461945in}{2.685758in}}%
\pgfpathlineto{\pgfqpoint{2.435834in}{2.528892in}}%
\pgfpathclose%
\pgfusepath{fill}%
\end{pgfscope}%
\begin{pgfscope}%
\pgfpathrectangle{\pgfqpoint{1.072000in}{0.528000in}}{\pgfqpoint{3.696000in}{3.696000in}}%
\pgfusepath{clip}%
\pgfsetbuttcap%
\pgfsetroundjoin%
\definecolor{currentfill}{rgb}{0.409611,0.540759,0.935545}%
\pgfsetfillcolor{currentfill}%
\pgfsetlinewidth{0.000000pt}%
\definecolor{currentstroke}{rgb}{0.000000,0.000000,0.000000}%
\pgfsetstrokecolor{currentstroke}%
\pgfsetdash{}{0pt}%
\pgfpathmoveto{\pgfqpoint{2.667447in}{1.448200in}}%
\pgfpathlineto{\pgfqpoint{2.714184in}{1.416276in}}%
\pgfpathlineto{\pgfqpoint{2.740911in}{1.565157in}}%
\pgfpathlineto{\pgfqpoint{2.694027in}{1.603380in}}%
\pgfpathlineto{\pgfqpoint{2.667447in}{1.448200in}}%
\pgfpathclose%
\pgfusepath{fill}%
\end{pgfscope}%
\begin{pgfscope}%
\pgfpathrectangle{\pgfqpoint{1.072000in}{0.528000in}}{\pgfqpoint{3.696000in}{3.696000in}}%
\pgfusepath{clip}%
\pgfsetbuttcap%
\pgfsetroundjoin%
\definecolor{currentfill}{rgb}{0.962701,0.628218,0.507636}%
\pgfsetfillcolor{currentfill}%
\pgfsetlinewidth{0.000000pt}%
\definecolor{currentstroke}{rgb}{0.000000,0.000000,0.000000}%
\pgfsetstrokecolor{currentstroke}%
\pgfsetdash{}{0pt}%
\pgfpathmoveto{\pgfqpoint{3.505394in}{2.760031in}}%
\pgfpathlineto{\pgfqpoint{3.550453in}{2.582835in}}%
\pgfpathlineto{\pgfqpoint{3.575910in}{2.582299in}}%
\pgfpathlineto{\pgfqpoint{3.531078in}{2.759913in}}%
\pgfpathlineto{\pgfqpoint{3.505394in}{2.760031in}}%
\pgfpathclose%
\pgfusepath{fill}%
\end{pgfscope}%
\begin{pgfscope}%
\pgfpathrectangle{\pgfqpoint{1.072000in}{0.528000in}}{\pgfqpoint{3.696000in}{3.696000in}}%
\pgfusepath{clip}%
\pgfsetbuttcap%
\pgfsetroundjoin%
\definecolor{currentfill}{rgb}{0.711554,0.033337,0.154485}%
\pgfsetfillcolor{currentfill}%
\pgfsetlinewidth{0.000000pt}%
\definecolor{currentstroke}{rgb}{0.000000,0.000000,0.000000}%
\pgfsetstrokecolor{currentstroke}%
\pgfsetdash{}{0pt}%
\pgfpathmoveto{\pgfqpoint{3.276112in}{3.156840in}}%
\pgfpathlineto{\pgfqpoint{3.323822in}{3.178274in}}%
\pgfpathlineto{\pgfqpoint{3.350261in}{3.204761in}}%
\pgfpathlineto{\pgfqpoint{3.302822in}{3.208383in}}%
\pgfpathlineto{\pgfqpoint{3.276112in}{3.156840in}}%
\pgfpathclose%
\pgfusepath{fill}%
\end{pgfscope}%
\begin{pgfscope}%
\pgfpathrectangle{\pgfqpoint{1.072000in}{0.528000in}}{\pgfqpoint{3.696000in}{3.696000in}}%
\pgfusepath{clip}%
\pgfsetbuttcap%
\pgfsetroundjoin%
\definecolor{currentfill}{rgb}{0.280550,0.373423,0.818011}%
\pgfsetfillcolor{currentfill}%
\pgfsetlinewidth{0.000000pt}%
\definecolor{currentstroke}{rgb}{0.000000,0.000000,0.000000}%
\pgfsetstrokecolor{currentstroke}%
\pgfsetdash{}{0pt}%
\pgfpathmoveto{\pgfqpoint{2.733980in}{1.274277in}}%
\pgfpathlineto{\pgfqpoint{2.780785in}{1.245058in}}%
\pgfpathlineto{\pgfqpoint{2.807912in}{1.338451in}}%
\pgfpathlineto{\pgfqpoint{2.761016in}{1.379130in}}%
\pgfpathlineto{\pgfqpoint{2.733980in}{1.274277in}}%
\pgfpathclose%
\pgfusepath{fill}%
\end{pgfscope}%
\begin{pgfscope}%
\pgfpathrectangle{\pgfqpoint{1.072000in}{0.528000in}}{\pgfqpoint{3.696000in}{3.696000in}}%
\pgfusepath{clip}%
\pgfsetbuttcap%
\pgfsetroundjoin%
\definecolor{currentfill}{rgb}{0.809329,0.852974,0.922323}%
\pgfsetfillcolor{currentfill}%
\pgfsetlinewidth{0.000000pt}%
\definecolor{currentstroke}{rgb}{0.000000,0.000000,0.000000}%
\pgfsetstrokecolor{currentstroke}%
\pgfsetdash{}{0pt}%
\pgfpathmoveto{\pgfqpoint{3.463304in}{2.158163in}}%
\pgfpathlineto{\pgfqpoint{3.508274in}{1.969976in}}%
\pgfpathlineto{\pgfqpoint{3.534972in}{2.040951in}}%
\pgfpathlineto{\pgfqpoint{3.490212in}{2.234241in}}%
\pgfpathlineto{\pgfqpoint{3.463304in}{2.158163in}}%
\pgfpathclose%
\pgfusepath{fill}%
\end{pgfscope}%
\begin{pgfscope}%
\pgfpathrectangle{\pgfqpoint{1.072000in}{0.528000in}}{\pgfqpoint{3.696000in}{3.696000in}}%
\pgfusepath{clip}%
\pgfsetbuttcap%
\pgfsetroundjoin%
\definecolor{currentfill}{rgb}{0.565182,0.699438,0.996635}%
\pgfsetfillcolor{currentfill}%
\pgfsetlinewidth{0.000000pt}%
\definecolor{currentstroke}{rgb}{0.000000,0.000000,0.000000}%
\pgfsetstrokecolor{currentstroke}%
\pgfsetdash{}{0pt}%
\pgfpathmoveto{\pgfqpoint{3.354162in}{1.744685in}}%
\pgfpathlineto{\pgfqpoint{3.400048in}{1.600706in}}%
\pgfpathlineto{\pgfqpoint{3.427207in}{1.700344in}}%
\pgfpathlineto{\pgfqpoint{3.381561in}{1.859254in}}%
\pgfpathlineto{\pgfqpoint{3.354162in}{1.744685in}}%
\pgfpathclose%
\pgfusepath{fill}%
\end{pgfscope}%
\begin{pgfscope}%
\pgfpathrectangle{\pgfqpoint{1.072000in}{0.528000in}}{\pgfqpoint{3.696000in}{3.696000in}}%
\pgfusepath{clip}%
\pgfsetbuttcap%
\pgfsetroundjoin%
\definecolor{currentfill}{rgb}{0.252663,0.332837,0.783665}%
\pgfsetfillcolor{currentfill}%
\pgfsetlinewidth{0.000000pt}%
\definecolor{currentstroke}{rgb}{0.000000,0.000000,0.000000}%
\pgfsetstrokecolor{currentstroke}%
\pgfsetdash{}{0pt}%
\pgfpathmoveto{\pgfqpoint{3.218187in}{1.245424in}}%
\pgfpathlineto{\pgfqpoint{3.265299in}{1.205260in}}%
\pgfpathlineto{\pgfqpoint{3.292026in}{1.256360in}}%
\pgfpathlineto{\pgfqpoint{3.245152in}{1.319512in}}%
\pgfpathlineto{\pgfqpoint{3.218187in}{1.245424in}}%
\pgfpathclose%
\pgfusepath{fill}%
\end{pgfscope}%
\begin{pgfscope}%
\pgfpathrectangle{\pgfqpoint{1.072000in}{0.528000in}}{\pgfqpoint{3.696000in}{3.696000in}}%
\pgfusepath{clip}%
\pgfsetbuttcap%
\pgfsetroundjoin%
\definecolor{currentfill}{rgb}{0.921406,0.491420,0.383408}%
\pgfsetfillcolor{currentfill}%
\pgfsetlinewidth{0.000000pt}%
\definecolor{currentstroke}{rgb}{0.000000,0.000000,0.000000}%
\pgfsetstrokecolor{currentstroke}%
\pgfsetdash{}{0pt}%
\pgfpathmoveto{\pgfqpoint{3.485545in}{2.912803in}}%
\pgfpathlineto{\pgfqpoint{3.531078in}{2.759913in}}%
\pgfpathlineto{\pgfqpoint{3.556359in}{2.744051in}}%
\pgfpathlineto{\pgfqpoint{3.511097in}{2.900102in}}%
\pgfpathlineto{\pgfqpoint{3.485545in}{2.912803in}}%
\pgfpathclose%
\pgfusepath{fill}%
\end{pgfscope}%
\begin{pgfscope}%
\pgfpathrectangle{\pgfqpoint{1.072000in}{0.528000in}}{\pgfqpoint{3.696000in}{3.696000in}}%
\pgfusepath{clip}%
\pgfsetbuttcap%
\pgfsetroundjoin%
\definecolor{currentfill}{rgb}{0.238948,0.312365,0.765676}%
\pgfsetfillcolor{currentfill}%
\pgfsetlinewidth{0.000000pt}%
\definecolor{currentstroke}{rgb}{0.000000,0.000000,0.000000}%
\pgfsetstrokecolor{currentstroke}%
\pgfsetdash{}{0pt}%
\pgfpathmoveto{\pgfqpoint{3.460368in}{1.214592in}}%
\pgfpathlineto{\pgfqpoint{3.508575in}{1.236128in}}%
\pgfpathlineto{\pgfqpoint{3.534162in}{1.235463in}}%
\pgfpathlineto{\pgfqpoint{3.486332in}{1.233276in}}%
\pgfpathlineto{\pgfqpoint{3.460368in}{1.214592in}}%
\pgfpathclose%
\pgfusepath{fill}%
\end{pgfscope}%
\begin{pgfscope}%
\pgfpathrectangle{\pgfqpoint{1.072000in}{0.528000in}}{\pgfqpoint{3.696000in}{3.696000in}}%
\pgfusepath{clip}%
\pgfsetbuttcap%
\pgfsetroundjoin%
\definecolor{currentfill}{rgb}{0.348323,0.465711,0.888346}%
\pgfsetfillcolor{currentfill}%
\pgfsetlinewidth{0.000000pt}%
\definecolor{currentstroke}{rgb}{0.000000,0.000000,0.000000}%
\pgfsetstrokecolor{currentstroke}%
\pgfsetdash{}{0pt}%
\pgfpathmoveto{\pgfqpoint{3.198374in}{1.400462in}}%
\pgfpathlineto{\pgfqpoint{3.245152in}{1.319512in}}%
\pgfpathlineto{\pgfqpoint{3.272245in}{1.411267in}}%
\pgfpathlineto{\pgfqpoint{3.225655in}{1.513781in}}%
\pgfpathlineto{\pgfqpoint{3.198374in}{1.400462in}}%
\pgfpathclose%
\pgfusepath{fill}%
\end{pgfscope}%
\begin{pgfscope}%
\pgfpathrectangle{\pgfqpoint{1.072000in}{0.528000in}}{\pgfqpoint{3.696000in}{3.696000in}}%
\pgfusepath{clip}%
\pgfsetbuttcap%
\pgfsetroundjoin%
\definecolor{currentfill}{rgb}{0.289996,0.386836,0.828926}%
\pgfsetfillcolor{currentfill}%
\pgfsetlinewidth{0.000000pt}%
\definecolor{currentstroke}{rgb}{0.000000,0.000000,0.000000}%
\pgfsetstrokecolor{currentstroke}%
\pgfsetdash{}{0pt}%
\pgfpathmoveto{\pgfqpoint{3.171225in}{1.303875in}}%
\pgfpathlineto{\pgfqpoint{3.218187in}{1.245424in}}%
\pgfpathlineto{\pgfqpoint{3.245152in}{1.319512in}}%
\pgfpathlineto{\pgfqpoint{3.198374in}{1.400462in}}%
\pgfpathlineto{\pgfqpoint{3.171225in}{1.303875in}}%
\pgfpathclose%
\pgfusepath{fill}%
\end{pgfscope}%
\begin{pgfscope}%
\pgfpathrectangle{\pgfqpoint{1.072000in}{0.528000in}}{\pgfqpoint{3.696000in}{3.696000in}}%
\pgfusepath{clip}%
\pgfsetbuttcap%
\pgfsetroundjoin%
\definecolor{currentfill}{rgb}{0.388852,0.516298,0.921373}%
\pgfsetfillcolor{currentfill}%
\pgfsetlinewidth{0.000000pt}%
\definecolor{currentstroke}{rgb}{0.000000,0.000000,0.000000}%
\pgfsetstrokecolor{currentstroke}%
\pgfsetdash{}{0pt}%
\pgfpathmoveto{\pgfqpoint{2.714184in}{1.416276in}}%
\pgfpathlineto{\pgfqpoint{2.761016in}{1.379130in}}%
\pgfpathlineto{\pgfqpoint{2.787880in}{1.519151in}}%
\pgfpathlineto{\pgfqpoint{2.740911in}{1.565157in}}%
\pgfpathlineto{\pgfqpoint{2.714184in}{1.416276in}}%
\pgfpathclose%
\pgfusepath{fill}%
\end{pgfscope}%
\begin{pgfscope}%
\pgfpathrectangle{\pgfqpoint{1.072000in}{0.528000in}}{\pgfqpoint{3.696000in}{3.696000in}}%
\pgfusepath{clip}%
\pgfsetbuttcap%
\pgfsetroundjoin%
\definecolor{currentfill}{rgb}{0.852378,0.346492,0.280346}%
\pgfsetfillcolor{currentfill}%
\pgfsetlinewidth{0.000000pt}%
\definecolor{currentstroke}{rgb}{0.000000,0.000000,0.000000}%
\pgfsetstrokecolor{currentstroke}%
\pgfsetdash{}{0pt}%
\pgfpathmoveto{\pgfqpoint{3.147684in}{2.899623in}}%
\pgfpathlineto{\pgfqpoint{3.195641in}{3.008207in}}%
\pgfpathlineto{\pgfqpoint{3.222425in}{3.042911in}}%
\pgfpathlineto{\pgfqpoint{3.174482in}{2.944108in}}%
\pgfpathlineto{\pgfqpoint{3.147684in}{2.899623in}}%
\pgfpathclose%
\pgfusepath{fill}%
\end{pgfscope}%
\begin{pgfscope}%
\pgfpathrectangle{\pgfqpoint{1.072000in}{0.528000in}}{\pgfqpoint{3.696000in}{3.696000in}}%
\pgfusepath{clip}%
\pgfsetbuttcap%
\pgfsetroundjoin%
\definecolor{currentfill}{rgb}{0.238948,0.312365,0.765676}%
\pgfsetfillcolor{currentfill}%
\pgfsetlinewidth{0.000000pt}%
\definecolor{currentstroke}{rgb}{0.000000,0.000000,0.000000}%
\pgfsetstrokecolor{currentstroke}%
\pgfsetdash{}{0pt}%
\pgfpathmoveto{\pgfqpoint{2.706647in}{1.209286in}}%
\pgfpathlineto{\pgfqpoint{2.753400in}{1.191109in}}%
\pgfpathlineto{\pgfqpoint{2.780785in}{1.245058in}}%
\pgfpathlineto{\pgfqpoint{2.733980in}{1.274277in}}%
\pgfpathlineto{\pgfqpoint{2.706647in}{1.209286in}}%
\pgfpathclose%
\pgfusepath{fill}%
\end{pgfscope}%
\begin{pgfscope}%
\pgfpathrectangle{\pgfqpoint{1.072000in}{0.528000in}}{\pgfqpoint{3.696000in}{3.696000in}}%
\pgfusepath{clip}%
\pgfsetbuttcap%
\pgfsetroundjoin%
\definecolor{currentfill}{rgb}{0.640828,0.760752,0.997846}%
\pgfsetfillcolor{currentfill}%
\pgfsetlinewidth{0.000000pt}%
\definecolor{currentstroke}{rgb}{0.000000,0.000000,0.000000}%
\pgfsetstrokecolor{currentstroke}%
\pgfsetdash{}{0pt}%
\pgfpathmoveto{\pgfqpoint{3.381561in}{1.859254in}}%
\pgfpathlineto{\pgfqpoint{3.427207in}{1.700344in}}%
\pgfpathlineto{\pgfqpoint{3.454335in}{1.797234in}}%
\pgfpathlineto{\pgfqpoint{3.408924in}{1.968451in}}%
\pgfpathlineto{\pgfqpoint{3.381561in}{1.859254in}}%
\pgfpathclose%
\pgfusepath{fill}%
\end{pgfscope}%
\begin{pgfscope}%
\pgfpathrectangle{\pgfqpoint{1.072000in}{0.528000in}}{\pgfqpoint{3.696000in}{3.696000in}}%
\pgfusepath{clip}%
\pgfsetbuttcap%
\pgfsetroundjoin%
\definecolor{currentfill}{rgb}{0.785153,0.220851,0.211673}%
\pgfsetfillcolor{currentfill}%
\pgfsetlinewidth{0.000000pt}%
\definecolor{currentstroke}{rgb}{0.000000,0.000000,0.000000}%
\pgfsetstrokecolor{currentstroke}%
\pgfsetdash{}{0pt}%
\pgfpathmoveto{\pgfqpoint{2.737187in}{3.126384in}}%
\pgfpathlineto{\pgfqpoint{2.784360in}{3.055264in}}%
\pgfpathlineto{\pgfqpoint{2.812156in}{3.010150in}}%
\pgfpathlineto{\pgfqpoint{2.765010in}{3.110485in}}%
\pgfpathlineto{\pgfqpoint{2.737187in}{3.126384in}}%
\pgfpathclose%
\pgfusepath{fill}%
\end{pgfscope}%
\begin{pgfscope}%
\pgfpathrectangle{\pgfqpoint{1.072000in}{0.528000in}}{\pgfqpoint{3.696000in}{3.696000in}}%
\pgfusepath{clip}%
\pgfsetbuttcap%
\pgfsetroundjoin%
\definecolor{currentfill}{rgb}{0.763363,0.835092,0.955658}%
\pgfsetfillcolor{currentfill}%
\pgfsetlinewidth{0.000000pt}%
\definecolor{currentstroke}{rgb}{0.000000,0.000000,0.000000}%
\pgfsetstrokecolor{currentstroke}%
\pgfsetdash{}{0pt}%
\pgfpathmoveto{\pgfqpoint{3.436190in}{2.068948in}}%
\pgfpathlineto{\pgfqpoint{3.481376in}{1.888028in}}%
\pgfpathlineto{\pgfqpoint{3.508274in}{1.969976in}}%
\pgfpathlineto{\pgfqpoint{3.463304in}{2.158163in}}%
\pgfpathlineto{\pgfqpoint{3.436190in}{2.068948in}}%
\pgfpathclose%
\pgfusepath{fill}%
\end{pgfscope}%
\begin{pgfscope}%
\pgfpathrectangle{\pgfqpoint{1.072000in}{0.528000in}}{\pgfqpoint{3.696000in}{3.696000in}}%
\pgfusepath{clip}%
\pgfsetbuttcap%
\pgfsetroundjoin%
\definecolor{currentfill}{rgb}{0.229806,0.298718,0.753683}%
\pgfsetfillcolor{currentfill}%
\pgfsetlinewidth{0.000000pt}%
\definecolor{currentstroke}{rgb}{0.000000,0.000000,0.000000}%
\pgfsetstrokecolor{currentstroke}%
\pgfsetdash{}{0pt}%
\pgfpathmoveto{\pgfqpoint{3.265299in}{1.205260in}}%
\pgfpathlineto{\pgfqpoint{3.312675in}{1.186836in}}%
\pgfpathlineto{\pgfqpoint{3.339117in}{1.215165in}}%
\pgfpathlineto{\pgfqpoint{3.292026in}{1.256360in}}%
\pgfpathlineto{\pgfqpoint{3.265299in}{1.205260in}}%
\pgfpathclose%
\pgfusepath{fill}%
\end{pgfscope}%
\begin{pgfscope}%
\pgfpathrectangle{\pgfqpoint{1.072000in}{0.528000in}}{\pgfqpoint{3.696000in}{3.696000in}}%
\pgfusepath{clip}%
\pgfsetbuttcap%
\pgfsetroundjoin%
\definecolor{currentfill}{rgb}{0.266381,0.353304,0.801637}%
\pgfsetfillcolor{currentfill}%
\pgfsetlinewidth{0.000000pt}%
\definecolor{currentstroke}{rgb}{0.000000,0.000000,0.000000}%
\pgfsetstrokecolor{currentstroke}%
\pgfsetdash{}{0pt}%
\pgfpathmoveto{\pgfqpoint{2.780785in}{1.245058in}}%
\pgfpathlineto{\pgfqpoint{2.827656in}{1.216660in}}%
\pgfpathlineto{\pgfqpoint{2.854851in}{1.296359in}}%
\pgfpathlineto{\pgfqpoint{2.807912in}{1.338451in}}%
\pgfpathlineto{\pgfqpoint{2.780785in}{1.245058in}}%
\pgfpathclose%
\pgfusepath{fill}%
\end{pgfscope}%
\begin{pgfscope}%
\pgfpathrectangle{\pgfqpoint{1.072000in}{0.528000in}}{\pgfqpoint{3.696000in}{3.696000in}}%
\pgfusepath{clip}%
\pgfsetbuttcap%
\pgfsetroundjoin%
\definecolor{currentfill}{rgb}{0.705673,0.015556,0.150233}%
\pgfsetfillcolor{currentfill}%
\pgfsetlinewidth{0.000000pt}%
\definecolor{currentstroke}{rgb}{0.000000,0.000000,0.000000}%
\pgfsetstrokecolor{currentstroke}%
\pgfsetdash{}{0pt}%
\pgfpathmoveto{\pgfqpoint{2.643465in}{3.186555in}}%
\pgfpathlineto{\pgfqpoint{2.690161in}{3.173291in}}%
\pgfpathlineto{\pgfqpoint{2.717956in}{3.180951in}}%
\pgfpathlineto{\pgfqpoint{2.671185in}{3.212740in}}%
\pgfpathlineto{\pgfqpoint{2.643465in}{3.186555in}}%
\pgfpathclose%
\pgfusepath{fill}%
\end{pgfscope}%
\begin{pgfscope}%
\pgfpathrectangle{\pgfqpoint{1.072000in}{0.528000in}}{\pgfqpoint{3.696000in}{3.696000in}}%
\pgfusepath{clip}%
\pgfsetbuttcap%
\pgfsetroundjoin%
\definecolor{currentfill}{rgb}{0.576051,0.708780,0.997755}%
\pgfsetfillcolor{currentfill}%
\pgfsetlinewidth{0.000000pt}%
\definecolor{currentstroke}{rgb}{0.000000,0.000000,0.000000}%
\pgfsetstrokecolor{currentstroke}%
\pgfsetdash{}{0pt}%
\pgfpathmoveto{\pgfqpoint{2.600696in}{1.652024in}}%
\pgfpathlineto{\pgfqpoint{2.647274in}{1.632579in}}%
\pgfpathlineto{\pgfqpoint{2.673584in}{1.818405in}}%
\pgfpathlineto{\pgfqpoint{2.626841in}{1.838835in}}%
\pgfpathlineto{\pgfqpoint{2.600696in}{1.652024in}}%
\pgfpathclose%
\pgfusepath{fill}%
\end{pgfscope}%
\begin{pgfscope}%
\pgfpathrectangle{\pgfqpoint{1.072000in}{0.528000in}}{\pgfqpoint{3.696000in}{3.696000in}}%
\pgfusepath{clip}%
\pgfsetbuttcap%
\pgfsetroundjoin%
\definecolor{currentfill}{rgb}{0.708720,0.805721,0.981117}%
\pgfsetfillcolor{currentfill}%
\pgfsetlinewidth{0.000000pt}%
\definecolor{currentstroke}{rgb}{0.000000,0.000000,0.000000}%
\pgfsetstrokecolor{currentstroke}%
\pgfsetdash{}{0pt}%
\pgfpathmoveto{\pgfqpoint{3.408924in}{1.968451in}}%
\pgfpathlineto{\pgfqpoint{3.454335in}{1.797234in}}%
\pgfpathlineto{\pgfqpoint{3.481376in}{1.888028in}}%
\pgfpathlineto{\pgfqpoint{3.436190in}{2.068948in}}%
\pgfpathlineto{\pgfqpoint{3.408924in}{1.968451in}}%
\pgfpathclose%
\pgfusepath{fill}%
\end{pgfscope}%
\begin{pgfscope}%
\pgfpathrectangle{\pgfqpoint{1.072000in}{0.528000in}}{\pgfqpoint{3.696000in}{3.696000in}}%
\pgfusepath{clip}%
\pgfsetbuttcap%
\pgfsetroundjoin%
\definecolor{currentfill}{rgb}{0.960581,0.762501,0.667964}%
\pgfsetfillcolor{currentfill}%
\pgfsetlinewidth{0.000000pt}%
\definecolor{currentstroke}{rgb}{0.000000,0.000000,0.000000}%
\pgfsetstrokecolor{currentstroke}%
\pgfsetdash{}{0pt}%
\pgfpathmoveto{\pgfqpoint{3.498398in}{2.536341in}}%
\pgfpathlineto{\pgfqpoint{3.543220in}{2.342629in}}%
\pgfpathlineto{\pgfqpoint{3.569239in}{2.373900in}}%
\pgfpathlineto{\pgfqpoint{3.524605in}{2.567539in}}%
\pgfpathlineto{\pgfqpoint{3.498398in}{2.536341in}}%
\pgfpathclose%
\pgfusepath{fill}%
\end{pgfscope}%
\begin{pgfscope}%
\pgfpathrectangle{\pgfqpoint{1.072000in}{0.528000in}}{\pgfqpoint{3.696000in}{3.696000in}}%
\pgfusepath{clip}%
\pgfsetbuttcap%
\pgfsetroundjoin%
\definecolor{currentfill}{rgb}{0.363461,0.484784,0.901019}%
\pgfsetfillcolor{currentfill}%
\pgfsetlinewidth{0.000000pt}%
\definecolor{currentstroke}{rgb}{0.000000,0.000000,0.000000}%
\pgfsetstrokecolor{currentstroke}%
\pgfsetdash{}{0pt}%
\pgfpathmoveto{\pgfqpoint{2.761016in}{1.379130in}}%
\pgfpathlineto{\pgfqpoint{2.807912in}{1.338451in}}%
\pgfpathlineto{\pgfqpoint{2.834898in}{1.467108in}}%
\pgfpathlineto{\pgfqpoint{2.787880in}{1.519151in}}%
\pgfpathlineto{\pgfqpoint{2.761016in}{1.379130in}}%
\pgfpathclose%
\pgfusepath{fill}%
\end{pgfscope}%
\begin{pgfscope}%
\pgfpathrectangle{\pgfqpoint{1.072000in}{0.528000in}}{\pgfqpoint{3.696000in}{3.696000in}}%
\pgfusepath{clip}%
\pgfsetbuttcap%
\pgfsetroundjoin%
\definecolor{currentfill}{rgb}{0.729196,0.086679,0.167240}%
\pgfsetfillcolor{currentfill}%
\pgfsetlinewidth{0.000000pt}%
\definecolor{currentstroke}{rgb}{0.000000,0.000000,0.000000}%
\pgfsetstrokecolor{currentstroke}%
\pgfsetdash{}{0pt}%
\pgfpathmoveto{\pgfqpoint{2.569746in}{3.107212in}}%
\pgfpathlineto{\pgfqpoint{2.615758in}{3.148348in}}%
\pgfpathlineto{\pgfqpoint{2.643465in}{3.186555in}}%
\pgfpathlineto{\pgfqpoint{2.597281in}{3.161522in}}%
\pgfpathlineto{\pgfqpoint{2.569746in}{3.107212in}}%
\pgfpathclose%
\pgfusepath{fill}%
\end{pgfscope}%
\begin{pgfscope}%
\pgfpathrectangle{\pgfqpoint{1.072000in}{0.528000in}}{\pgfqpoint{3.696000in}{3.696000in}}%
\pgfusepath{clip}%
\pgfsetbuttcap%
\pgfsetroundjoin%
\definecolor{currentfill}{rgb}{0.425199,0.559058,0.946061}%
\pgfsetfillcolor{currentfill}%
\pgfsetlinewidth{0.000000pt}%
\definecolor{currentstroke}{rgb}{0.000000,0.000000,0.000000}%
\pgfsetstrokecolor{currentstroke}%
\pgfsetdash{}{0pt}%
\pgfpathmoveto{\pgfqpoint{3.225655in}{1.513781in}}%
\pgfpathlineto{\pgfqpoint{3.272245in}{1.411267in}}%
\pgfpathlineto{\pgfqpoint{3.299461in}{1.515988in}}%
\pgfpathlineto{\pgfqpoint{3.253061in}{1.638469in}}%
\pgfpathlineto{\pgfqpoint{3.225655in}{1.513781in}}%
\pgfpathclose%
\pgfusepath{fill}%
\end{pgfscope}%
\begin{pgfscope}%
\pgfpathrectangle{\pgfqpoint{1.072000in}{0.528000in}}{\pgfqpoint{3.696000in}{3.696000in}}%
\pgfusepath{clip}%
\pgfsetbuttcap%
\pgfsetroundjoin%
\definecolor{currentfill}{rgb}{0.718985,0.811993,0.977656}%
\pgfsetfillcolor{currentfill}%
\pgfsetlinewidth{0.000000pt}%
\definecolor{currentstroke}{rgb}{0.000000,0.000000,0.000000}%
\pgfsetstrokecolor{currentstroke}%
\pgfsetdash{}{0pt}%
\pgfpathmoveto{\pgfqpoint{2.534104in}{1.843021in}}%
\pgfpathlineto{\pgfqpoint{2.580333in}{1.846953in}}%
\pgfpathlineto{\pgfqpoint{2.606296in}{2.049851in}}%
\pgfpathlineto{\pgfqpoint{2.559908in}{2.042837in}}%
\pgfpathlineto{\pgfqpoint{2.534104in}{1.843021in}}%
\pgfpathclose%
\pgfusepath{fill}%
\end{pgfscope}%
\begin{pgfscope}%
\pgfpathrectangle{\pgfqpoint{1.072000in}{0.528000in}}{\pgfqpoint{3.696000in}{3.696000in}}%
\pgfusepath{clip}%
\pgfsetbuttcap%
\pgfsetroundjoin%
\definecolor{currentfill}{rgb}{0.333490,0.446265,0.874452}%
\pgfsetfillcolor{currentfill}%
\pgfsetlinewidth{0.000000pt}%
\definecolor{currentstroke}{rgb}{0.000000,0.000000,0.000000}%
\pgfsetstrokecolor{currentstroke}%
\pgfsetdash{}{0pt}%
\pgfpathmoveto{\pgfqpoint{2.807912in}{1.338451in}}%
\pgfpathlineto{\pgfqpoint{2.854851in}{1.296359in}}%
\pgfpathlineto{\pgfqpoint{2.881931in}{1.411268in}}%
\pgfpathlineto{\pgfqpoint{2.834898in}{1.467108in}}%
\pgfpathlineto{\pgfqpoint{2.807912in}{1.338451in}}%
\pgfpathclose%
\pgfusepath{fill}%
\end{pgfscope}%
\begin{pgfscope}%
\pgfpathrectangle{\pgfqpoint{1.072000in}{0.528000in}}{\pgfqpoint{3.696000in}{3.696000in}}%
\pgfusepath{clip}%
\pgfsetbuttcap%
\pgfsetroundjoin%
\definecolor{currentfill}{rgb}{0.229806,0.298718,0.753683}%
\pgfsetfillcolor{currentfill}%
\pgfsetlinewidth{0.000000pt}%
\definecolor{currentstroke}{rgb}{0.000000,0.000000,0.000000}%
\pgfsetstrokecolor{currentstroke}%
\pgfsetdash{}{0pt}%
\pgfpathmoveto{\pgfqpoint{3.386561in}{1.199287in}}%
\pgfpathlineto{\pgfqpoint{3.434500in}{1.211046in}}%
\pgfpathlineto{\pgfqpoint{3.460368in}{1.214592in}}%
\pgfpathlineto{\pgfqpoint{3.412785in}{1.223719in}}%
\pgfpathlineto{\pgfqpoint{3.386561in}{1.199287in}}%
\pgfpathclose%
\pgfusepath{fill}%
\end{pgfscope}%
\begin{pgfscope}%
\pgfpathrectangle{\pgfqpoint{1.072000in}{0.528000in}}{\pgfqpoint{3.696000in}{3.696000in}}%
\pgfusepath{clip}%
\pgfsetbuttcap%
\pgfsetroundjoin%
\definecolor{currentfill}{rgb}{0.839351,0.861167,0.894494}%
\pgfsetfillcolor{currentfill}%
\pgfsetlinewidth{0.000000pt}%
\definecolor{currentstroke}{rgb}{0.000000,0.000000,0.000000}%
\pgfsetstrokecolor{currentstroke}%
\pgfsetdash{}{0pt}%
\pgfpathmoveto{\pgfqpoint{1.706765in}{2.266873in}}%
\pgfpathlineto{\pgfqpoint{1.756568in}{2.138942in}}%
\pgfpathlineto{\pgfqpoint{1.791184in}{2.017934in}}%
\pgfpathlineto{\pgfqpoint{1.741682in}{2.147169in}}%
\pgfpathlineto{\pgfqpoint{1.706765in}{2.266873in}}%
\pgfpathclose%
\pgfusepath{fill}%
\end{pgfscope}%
\begin{pgfscope}%
\pgfpathrectangle{\pgfqpoint{1.072000in}{0.528000in}}{\pgfqpoint{3.696000in}{3.696000in}}%
\pgfusepath{clip}%
\pgfsetbuttcap%
\pgfsetroundjoin%
\definecolor{currentfill}{rgb}{0.229806,0.298718,0.753683}%
\pgfsetfillcolor{currentfill}%
\pgfsetlinewidth{0.000000pt}%
\definecolor{currentstroke}{rgb}{0.000000,0.000000,0.000000}%
\pgfsetstrokecolor{currentstroke}%
\pgfsetdash{}{0pt}%
\pgfpathmoveto{\pgfqpoint{2.632246in}{1.197857in}}%
\pgfpathlineto{\pgfqpoint{2.678896in}{1.186708in}}%
\pgfpathlineto{\pgfqpoint{2.706647in}{1.209286in}}%
\pgfpathlineto{\pgfqpoint{2.659984in}{1.228604in}}%
\pgfpathlineto{\pgfqpoint{2.632246in}{1.197857in}}%
\pgfpathclose%
\pgfusepath{fill}%
\end{pgfscope}%
\begin{pgfscope}%
\pgfpathrectangle{\pgfqpoint{1.072000in}{0.528000in}}{\pgfqpoint{3.696000in}{3.696000in}}%
\pgfusepath{clip}%
\pgfsetbuttcap%
\pgfsetroundjoin%
\definecolor{currentfill}{rgb}{0.238948,0.312365,0.765676}%
\pgfsetfillcolor{currentfill}%
\pgfsetlinewidth{0.000000pt}%
\definecolor{currentstroke}{rgb}{0.000000,0.000000,0.000000}%
\pgfsetstrokecolor{currentstroke}%
\pgfsetdash{}{0pt}%
\pgfpathmoveto{\pgfqpoint{2.510806in}{1.229509in}}%
\pgfpathlineto{\pgfqpoint{2.557321in}{1.218542in}}%
\pgfpathlineto{\pgfqpoint{2.585693in}{1.210976in}}%
\pgfpathlineto{\pgfqpoint{2.539240in}{1.225179in}}%
\pgfpathlineto{\pgfqpoint{2.510806in}{1.229509in}}%
\pgfpathclose%
\pgfusepath{fill}%
\end{pgfscope}%
\begin{pgfscope}%
\pgfpathrectangle{\pgfqpoint{1.072000in}{0.528000in}}{\pgfqpoint{3.696000in}{3.696000in}}%
\pgfusepath{clip}%
\pgfsetbuttcap%
\pgfsetroundjoin%
\definecolor{currentfill}{rgb}{0.248091,0.326013,0.777669}%
\pgfsetfillcolor{currentfill}%
\pgfsetlinewidth{0.000000pt}%
\definecolor{currentstroke}{rgb}{0.000000,0.000000,0.000000}%
\pgfsetstrokecolor{currentstroke}%
\pgfsetdash{}{0pt}%
\pgfpathmoveto{\pgfqpoint{3.144192in}{1.228930in}}%
\pgfpathlineto{\pgfqpoint{3.191336in}{1.193194in}}%
\pgfpathlineto{\pgfqpoint{3.218187in}{1.245424in}}%
\pgfpathlineto{\pgfqpoint{3.171225in}{1.303875in}}%
\pgfpathlineto{\pgfqpoint{3.144192in}{1.228930in}}%
\pgfpathclose%
\pgfusepath{fill}%
\end{pgfscope}%
\begin{pgfscope}%
\pgfpathrectangle{\pgfqpoint{1.072000in}{0.528000in}}{\pgfqpoint{3.696000in}{3.696000in}}%
\pgfusepath{clip}%
\pgfsetbuttcap%
\pgfsetroundjoin%
\definecolor{currentfill}{rgb}{0.280550,0.373423,0.818011}%
\pgfsetfillcolor{currentfill}%
\pgfsetlinewidth{0.000000pt}%
\definecolor{currentstroke}{rgb}{0.000000,0.000000,0.000000}%
\pgfsetstrokecolor{currentstroke}%
\pgfsetdash{}{0pt}%
\pgfpathmoveto{\pgfqpoint{3.097159in}{1.279648in}}%
\pgfpathlineto{\pgfqpoint{3.144192in}{1.228930in}}%
\pgfpathlineto{\pgfqpoint{3.171225in}{1.303875in}}%
\pgfpathlineto{\pgfqpoint{3.124316in}{1.376588in}}%
\pgfpathlineto{\pgfqpoint{3.097159in}{1.279648in}}%
\pgfpathclose%
\pgfusepath{fill}%
\end{pgfscope}%
\begin{pgfscope}%
\pgfpathrectangle{\pgfqpoint{1.072000in}{0.528000in}}{\pgfqpoint{3.696000in}{3.696000in}}%
\pgfusepath{clip}%
\pgfsetbuttcap%
\pgfsetroundjoin%
\definecolor{currentfill}{rgb}{0.729196,0.086679,0.167240}%
\pgfsetfillcolor{currentfill}%
\pgfsetlinewidth{0.000000pt}%
\definecolor{currentstroke}{rgb}{0.000000,0.000000,0.000000}%
\pgfsetstrokecolor{currentstroke}%
\pgfsetdash{}{0pt}%
\pgfpathmoveto{\pgfqpoint{2.690161in}{3.173291in}}%
\pgfpathlineto{\pgfqpoint{2.737187in}{3.126384in}}%
\pgfpathlineto{\pgfqpoint{2.765010in}{3.110485in}}%
\pgfpathlineto{\pgfqpoint{2.717956in}{3.180951in}}%
\pgfpathlineto{\pgfqpoint{2.690161in}{3.173291in}}%
\pgfpathclose%
\pgfusepath{fill}%
\end{pgfscope}%
\begin{pgfscope}%
\pgfpathrectangle{\pgfqpoint{1.072000in}{0.528000in}}{\pgfqpoint{3.696000in}{3.696000in}}%
\pgfusepath{clip}%
\pgfsetbuttcap%
\pgfsetroundjoin%
\definecolor{currentfill}{rgb}{0.343278,0.459354,0.884122}%
\pgfsetfillcolor{currentfill}%
\pgfsetlinewidth{0.000000pt}%
\definecolor{currentstroke}{rgb}{0.000000,0.000000,0.000000}%
\pgfsetstrokecolor{currentstroke}%
\pgfsetdash{}{0pt}%
\pgfpathmoveto{\pgfqpoint{3.124316in}{1.376588in}}%
\pgfpathlineto{\pgfqpoint{3.171225in}{1.303875in}}%
\pgfpathlineto{\pgfqpoint{3.198374in}{1.400462in}}%
\pgfpathlineto{\pgfqpoint{3.151592in}{1.494532in}}%
\pgfpathlineto{\pgfqpoint{3.124316in}{1.376588in}}%
\pgfpathclose%
\pgfusepath{fill}%
\end{pgfscope}%
\begin{pgfscope}%
\pgfpathrectangle{\pgfqpoint{1.072000in}{0.528000in}}{\pgfqpoint{3.696000in}{3.696000in}}%
\pgfusepath{clip}%
\pgfsetbuttcap%
\pgfsetroundjoin%
\definecolor{currentfill}{rgb}{0.902659,0.447939,0.349721}%
\pgfsetfillcolor{currentfill}%
\pgfsetlinewidth{0.000000pt}%
\definecolor{currentstroke}{rgb}{0.000000,0.000000,0.000000}%
\pgfsetstrokecolor{currentstroke}%
\pgfsetdash{}{0pt}%
\pgfpathmoveto{\pgfqpoint{3.046770in}{2.843259in}}%
\pgfpathlineto{\pgfqpoint{3.094295in}{2.932522in}}%
\pgfpathlineto{\pgfqpoint{3.121001in}{2.899182in}}%
\pgfpathlineto{\pgfqpoint{3.073437in}{2.775820in}}%
\pgfpathlineto{\pgfqpoint{3.046770in}{2.843259in}}%
\pgfpathclose%
\pgfusepath{fill}%
\end{pgfscope}%
\begin{pgfscope}%
\pgfpathrectangle{\pgfqpoint{1.072000in}{0.528000in}}{\pgfqpoint{3.696000in}{3.696000in}}%
\pgfusepath{clip}%
\pgfsetbuttcap%
\pgfsetroundjoin%
\definecolor{currentfill}{rgb}{0.304174,0.406945,0.845263}%
\pgfsetfillcolor{currentfill}%
\pgfsetlinewidth{0.000000pt}%
\definecolor{currentstroke}{rgb}{0.000000,0.000000,0.000000}%
\pgfsetstrokecolor{currentstroke}%
\pgfsetdash{}{0pt}%
\pgfpathmoveto{\pgfqpoint{2.854851in}{1.296359in}}%
\pgfpathlineto{\pgfqpoint{2.901822in}{1.255349in}}%
\pgfpathlineto{\pgfqpoint{2.928961in}{1.354317in}}%
\pgfpathlineto{\pgfqpoint{2.881931in}{1.411268in}}%
\pgfpathlineto{\pgfqpoint{2.854851in}{1.296359in}}%
\pgfpathclose%
\pgfusepath{fill}%
\end{pgfscope}%
\begin{pgfscope}%
\pgfpathrectangle{\pgfqpoint{1.072000in}{0.528000in}}{\pgfqpoint{3.696000in}{3.696000in}}%
\pgfusepath{clip}%
\pgfsetbuttcap%
\pgfsetroundjoin%
\definecolor{currentfill}{rgb}{0.790562,0.231397,0.216242}%
\pgfsetfillcolor{currentfill}%
\pgfsetlinewidth{0.000000pt}%
\definecolor{currentstroke}{rgb}{0.000000,0.000000,0.000000}%
\pgfsetstrokecolor{currentstroke}%
\pgfsetdash{}{0pt}%
\pgfpathmoveto{\pgfqpoint{3.418513in}{3.115583in}}%
\pgfpathlineto{\pgfqpoint{3.465132in}{3.025932in}}%
\pgfpathlineto{\pgfqpoint{3.490593in}{3.004282in}}%
\pgfpathlineto{\pgfqpoint{3.444311in}{3.105380in}}%
\pgfpathlineto{\pgfqpoint{3.418513in}{3.115583in}}%
\pgfpathclose%
\pgfusepath{fill}%
\end{pgfscope}%
\begin{pgfscope}%
\pgfpathrectangle{\pgfqpoint{1.072000in}{0.528000in}}{\pgfqpoint{3.696000in}{3.696000in}}%
\pgfusepath{clip}%
\pgfsetbuttcap%
\pgfsetroundjoin%
\definecolor{currentfill}{rgb}{0.248091,0.326013,0.777669}%
\pgfsetfillcolor{currentfill}%
\pgfsetlinewidth{0.000000pt}%
\definecolor{currentstroke}{rgb}{0.000000,0.000000,0.000000}%
\pgfsetstrokecolor{currentstroke}%
\pgfsetdash{}{0pt}%
\pgfpathmoveto{\pgfqpoint{2.827656in}{1.216660in}}%
\pgfpathlineto{\pgfqpoint{2.874592in}{1.191316in}}%
\pgfpathlineto{\pgfqpoint{2.901822in}{1.255349in}}%
\pgfpathlineto{\pgfqpoint{2.854851in}{1.296359in}}%
\pgfpathlineto{\pgfqpoint{2.827656in}{1.216660in}}%
\pgfpathclose%
\pgfusepath{fill}%
\end{pgfscope}%
\begin{pgfscope}%
\pgfpathrectangle{\pgfqpoint{1.072000in}{0.528000in}}{\pgfqpoint{3.696000in}{3.696000in}}%
\pgfusepath{clip}%
\pgfsetbuttcap%
\pgfsetroundjoin%
\definecolor{currentfill}{rgb}{0.275827,0.366717,0.812553}%
\pgfsetfillcolor{currentfill}%
\pgfsetlinewidth{0.000000pt}%
\definecolor{currentstroke}{rgb}{0.000000,0.000000,0.000000}%
\pgfsetstrokecolor{currentstroke}%
\pgfsetdash{}{0pt}%
\pgfpathmoveto{\pgfqpoint{2.901822in}{1.255349in}}%
\pgfpathlineto{\pgfqpoint{2.948828in}{1.218202in}}%
\pgfpathlineto{\pgfqpoint{2.975985in}{1.299319in}}%
\pgfpathlineto{\pgfqpoint{2.928961in}{1.354317in}}%
\pgfpathlineto{\pgfqpoint{2.901822in}{1.255349in}}%
\pgfpathclose%
\pgfusepath{fill}%
\end{pgfscope}%
\begin{pgfscope}%
\pgfpathrectangle{\pgfqpoint{1.072000in}{0.528000in}}{\pgfqpoint{3.696000in}{3.696000in}}%
\pgfusepath{clip}%
\pgfsetbuttcap%
\pgfsetroundjoin%
\definecolor{currentfill}{rgb}{0.926883,0.505422,0.394866}%
\pgfsetfillcolor{currentfill}%
\pgfsetlinewidth{0.000000pt}%
\definecolor{currentstroke}{rgb}{0.000000,0.000000,0.000000}%
\pgfsetstrokecolor{currentstroke}%
\pgfsetdash{}{0pt}%
\pgfpathmoveto{\pgfqpoint{2.461945in}{2.685758in}}%
\pgfpathlineto{\pgfqpoint{2.506977in}{2.773065in}}%
\pgfpathlineto{\pgfqpoint{2.533697in}{2.905132in}}%
\pgfpathlineto{\pgfqpoint{2.488414in}{2.824057in}}%
\pgfpathlineto{\pgfqpoint{2.461945in}{2.685758in}}%
\pgfpathclose%
\pgfusepath{fill}%
\end{pgfscope}%
\begin{pgfscope}%
\pgfpathrectangle{\pgfqpoint{1.072000in}{0.528000in}}{\pgfqpoint{3.696000in}{3.696000in}}%
\pgfusepath{clip}%
\pgfsetbuttcap%
\pgfsetroundjoin%
\definecolor{currentfill}{rgb}{0.266381,0.353304,0.801637}%
\pgfsetfillcolor{currentfill}%
\pgfsetlinewidth{0.000000pt}%
\definecolor{currentstroke}{rgb}{0.000000,0.000000,0.000000}%
\pgfsetstrokecolor{currentstroke}%
\pgfsetdash{}{0pt}%
\pgfpathmoveto{\pgfqpoint{3.023014in}{1.249603in}}%
\pgfpathlineto{\pgfqpoint{3.070083in}{1.208619in}}%
\pgfpathlineto{\pgfqpoint{3.097159in}{1.279648in}}%
\pgfpathlineto{\pgfqpoint{3.050162in}{1.341592in}}%
\pgfpathlineto{\pgfqpoint{3.023014in}{1.249603in}}%
\pgfpathclose%
\pgfusepath{fill}%
\end{pgfscope}%
\begin{pgfscope}%
\pgfpathrectangle{\pgfqpoint{1.072000in}{0.528000in}}{\pgfqpoint{3.696000in}{3.696000in}}%
\pgfusepath{clip}%
\pgfsetbuttcap%
\pgfsetroundjoin%
\definecolor{currentfill}{rgb}{0.705673,0.015556,0.150233}%
\pgfsetfillcolor{currentfill}%
\pgfsetlinewidth{0.000000pt}%
\definecolor{currentstroke}{rgb}{0.000000,0.000000,0.000000}%
\pgfsetstrokecolor{currentstroke}%
\pgfsetdash{}{0pt}%
\pgfpathmoveto{\pgfqpoint{3.323822in}{3.178274in}}%
\pgfpathlineto{\pgfqpoint{3.371354in}{3.166133in}}%
\pgfpathlineto{\pgfqpoint{3.397484in}{3.171954in}}%
\pgfpathlineto{\pgfqpoint{3.350261in}{3.204761in}}%
\pgfpathlineto{\pgfqpoint{3.323822in}{3.178274in}}%
\pgfpathclose%
\pgfusepath{fill}%
\end{pgfscope}%
\begin{pgfscope}%
\pgfpathrectangle{\pgfqpoint{1.072000in}{0.528000in}}{\pgfqpoint{3.696000in}{3.696000in}}%
\pgfusepath{clip}%
\pgfsetbuttcap%
\pgfsetroundjoin%
\definecolor{currentfill}{rgb}{0.328604,0.439712,0.869587}%
\pgfsetfillcolor{currentfill}%
\pgfsetlinewidth{0.000000pt}%
\definecolor{currentstroke}{rgb}{0.000000,0.000000,0.000000}%
\pgfsetstrokecolor{currentstroke}%
\pgfsetdash{}{0pt}%
\pgfpathmoveto{\pgfqpoint{3.050162in}{1.341592in}}%
\pgfpathlineto{\pgfqpoint{3.097159in}{1.279648in}}%
\pgfpathlineto{\pgfqpoint{3.124316in}{1.376588in}}%
\pgfpathlineto{\pgfqpoint{3.077387in}{1.459223in}}%
\pgfpathlineto{\pgfqpoint{3.050162in}{1.341592in}}%
\pgfpathclose%
\pgfusepath{fill}%
\end{pgfscope}%
\begin{pgfscope}%
\pgfpathrectangle{\pgfqpoint{1.072000in}{0.528000in}}{\pgfqpoint{3.696000in}{3.696000in}}%
\pgfusepath{clip}%
\pgfsetbuttcap%
\pgfsetroundjoin%
\definecolor{currentfill}{rgb}{0.304174,0.406945,0.845263}%
\pgfsetfillcolor{currentfill}%
\pgfsetlinewidth{0.000000pt}%
\definecolor{currentstroke}{rgb}{0.000000,0.000000,0.000000}%
\pgfsetstrokecolor{currentstroke}%
\pgfsetdash{}{0pt}%
\pgfpathmoveto{\pgfqpoint{2.975985in}{1.299319in}}%
\pgfpathlineto{\pgfqpoint{3.023014in}{1.249603in}}%
\pgfpathlineto{\pgfqpoint{3.050162in}{1.341592in}}%
\pgfpathlineto{\pgfqpoint{3.003150in}{1.410840in}}%
\pgfpathlineto{\pgfqpoint{2.975985in}{1.299319in}}%
\pgfpathclose%
\pgfusepath{fill}%
\end{pgfscope}%
\begin{pgfscope}%
\pgfpathrectangle{\pgfqpoint{1.072000in}{0.528000in}}{\pgfqpoint{3.696000in}{3.696000in}}%
\pgfusepath{clip}%
\pgfsetbuttcap%
\pgfsetroundjoin%
\definecolor{currentfill}{rgb}{0.229806,0.298718,0.753683}%
\pgfsetfillcolor{currentfill}%
\pgfsetlinewidth{0.000000pt}%
\definecolor{currentstroke}{rgb}{0.000000,0.000000,0.000000}%
\pgfsetstrokecolor{currentstroke}%
\pgfsetdash{}{0pt}%
\pgfpathmoveto{\pgfqpoint{2.753400in}{1.191109in}}%
\pgfpathlineto{\pgfqpoint{2.800240in}{1.175770in}}%
\pgfpathlineto{\pgfqpoint{2.827656in}{1.216660in}}%
\pgfpathlineto{\pgfqpoint{2.780785in}{1.245058in}}%
\pgfpathlineto{\pgfqpoint{2.753400in}{1.191109in}}%
\pgfpathclose%
\pgfusepath{fill}%
\end{pgfscope}%
\begin{pgfscope}%
\pgfpathrectangle{\pgfqpoint{1.072000in}{0.528000in}}{\pgfqpoint{3.696000in}{3.696000in}}%
\pgfusepath{clip}%
\pgfsetbuttcap%
\pgfsetroundjoin%
\definecolor{currentfill}{rgb}{0.729196,0.086679,0.167240}%
\pgfsetfillcolor{currentfill}%
\pgfsetlinewidth{0.000000pt}%
\definecolor{currentstroke}{rgb}{0.000000,0.000000,0.000000}%
\pgfsetstrokecolor{currentstroke}%
\pgfsetdash{}{0pt}%
\pgfpathmoveto{\pgfqpoint{3.249272in}{3.096807in}}%
\pgfpathlineto{\pgfqpoint{3.297190in}{3.141894in}}%
\pgfpathlineto{\pgfqpoint{3.323822in}{3.178274in}}%
\pgfpathlineto{\pgfqpoint{3.276112in}{3.156840in}}%
\pgfpathlineto{\pgfqpoint{3.249272in}{3.096807in}}%
\pgfpathclose%
\pgfusepath{fill}%
\end{pgfscope}%
\begin{pgfscope}%
\pgfpathrectangle{\pgfqpoint{1.072000in}{0.528000in}}{\pgfqpoint{3.696000in}{3.696000in}}%
\pgfusepath{clip}%
\pgfsetbuttcap%
\pgfsetroundjoin%
\definecolor{currentfill}{rgb}{0.229806,0.298718,0.753683}%
\pgfsetfillcolor{currentfill}%
\pgfsetlinewidth{0.000000pt}%
\definecolor{currentstroke}{rgb}{0.000000,0.000000,0.000000}%
\pgfsetstrokecolor{currentstroke}%
\pgfsetdash{}{0pt}%
\pgfpathmoveto{\pgfqpoint{3.191336in}{1.193194in}}%
\pgfpathlineto{\pgfqpoint{3.238680in}{1.175813in}}%
\pgfpathlineto{\pgfqpoint{3.265299in}{1.205260in}}%
\pgfpathlineto{\pgfqpoint{3.218187in}{1.245424in}}%
\pgfpathlineto{\pgfqpoint{3.191336in}{1.193194in}}%
\pgfpathclose%
\pgfusepath{fill}%
\end{pgfscope}%
\begin{pgfscope}%
\pgfpathrectangle{\pgfqpoint{1.072000in}{0.528000in}}{\pgfqpoint{3.696000in}{3.696000in}}%
\pgfusepath{clip}%
\pgfsetbuttcap%
\pgfsetroundjoin%
\definecolor{currentfill}{rgb}{0.565182,0.699438,0.996635}%
\pgfsetfillcolor{currentfill}%
\pgfsetlinewidth{0.000000pt}%
\definecolor{currentstroke}{rgb}{0.000000,0.000000,0.000000}%
\pgfsetstrokecolor{currentstroke}%
\pgfsetdash{}{0pt}%
\pgfpathmoveto{\pgfqpoint{2.647274in}{1.632579in}}%
\pgfpathlineto{\pgfqpoint{2.694027in}{1.603380in}}%
\pgfpathlineto{\pgfqpoint{2.720508in}{1.785868in}}%
\pgfpathlineto{\pgfqpoint{2.673584in}{1.818405in}}%
\pgfpathlineto{\pgfqpoint{2.647274in}{1.632579in}}%
\pgfpathclose%
\pgfusepath{fill}%
\end{pgfscope}%
\begin{pgfscope}%
\pgfpathrectangle{\pgfqpoint{1.072000in}{0.528000in}}{\pgfqpoint{3.696000in}{3.696000in}}%
\pgfusepath{clip}%
\pgfsetbuttcap%
\pgfsetroundjoin%
\definecolor{currentfill}{rgb}{0.510824,0.649397,0.985079}%
\pgfsetfillcolor{currentfill}%
\pgfsetlinewidth{0.000000pt}%
\definecolor{currentstroke}{rgb}{0.000000,0.000000,0.000000}%
\pgfsetstrokecolor{currentstroke}%
\pgfsetdash{}{0pt}%
\pgfpathmoveto{\pgfqpoint{3.253061in}{1.638469in}}%
\pgfpathlineto{\pgfqpoint{3.299461in}{1.515988in}}%
\pgfpathlineto{\pgfqpoint{3.326779in}{1.628749in}}%
\pgfpathlineto{\pgfqpoint{3.280568in}{1.769041in}}%
\pgfpathlineto{\pgfqpoint{3.253061in}{1.638469in}}%
\pgfpathclose%
\pgfusepath{fill}%
\end{pgfscope}%
\begin{pgfscope}%
\pgfpathrectangle{\pgfqpoint{1.072000in}{0.528000in}}{\pgfqpoint{3.696000in}{3.696000in}}%
\pgfusepath{clip}%
\pgfsetbuttcap%
\pgfsetroundjoin%
\definecolor{currentfill}{rgb}{0.343278,0.459354,0.884122}%
\pgfsetfillcolor{currentfill}%
\pgfsetlinewidth{0.000000pt}%
\definecolor{currentstroke}{rgb}{0.000000,0.000000,0.000000}%
\pgfsetstrokecolor{currentstroke}%
\pgfsetdash{}{0pt}%
\pgfpathmoveto{\pgfqpoint{2.928961in}{1.354317in}}%
\pgfpathlineto{\pgfqpoint{2.975985in}{1.299319in}}%
\pgfpathlineto{\pgfqpoint{3.003150in}{1.410840in}}%
\pgfpathlineto{\pgfqpoint{2.956096in}{1.483519in}}%
\pgfpathlineto{\pgfqpoint{2.928961in}{1.354317in}}%
\pgfpathclose%
\pgfusepath{fill}%
\end{pgfscope}%
\begin{pgfscope}%
\pgfpathrectangle{\pgfqpoint{1.072000in}{0.528000in}}{\pgfqpoint{3.696000in}{3.696000in}}%
\pgfusepath{clip}%
\pgfsetbuttcap%
\pgfsetroundjoin%
\definecolor{currentfill}{rgb}{0.252663,0.332837,0.783665}%
\pgfsetfillcolor{currentfill}%
\pgfsetlinewidth{0.000000pt}%
\definecolor{currentstroke}{rgb}{0.000000,0.000000,0.000000}%
\pgfsetstrokecolor{currentstroke}%
\pgfsetdash{}{0pt}%
\pgfpathmoveto{\pgfqpoint{2.948828in}{1.218202in}}%
\pgfpathlineto{\pgfqpoint{2.995887in}{1.187872in}}%
\pgfpathlineto{\pgfqpoint{3.023014in}{1.249603in}}%
\pgfpathlineto{\pgfqpoint{2.975985in}{1.299319in}}%
\pgfpathlineto{\pgfqpoint{2.948828in}{1.218202in}}%
\pgfpathclose%
\pgfusepath{fill}%
\end{pgfscope}%
\begin{pgfscope}%
\pgfpathrectangle{\pgfqpoint{1.072000in}{0.528000in}}{\pgfqpoint{3.696000in}{3.696000in}}%
\pgfusepath{clip}%
\pgfsetbuttcap%
\pgfsetroundjoin%
\definecolor{currentfill}{rgb}{0.243520,0.319189,0.771672}%
\pgfsetfillcolor{currentfill}%
\pgfsetlinewidth{0.000000pt}%
\definecolor{currentstroke}{rgb}{0.000000,0.000000,0.000000}%
\pgfsetstrokecolor{currentstroke}%
\pgfsetdash{}{0pt}%
\pgfpathmoveto{\pgfqpoint{3.070083in}{1.208619in}}%
\pgfpathlineto{\pgfqpoint{3.117243in}{1.179771in}}%
\pgfpathlineto{\pgfqpoint{3.144192in}{1.228930in}}%
\pgfpathlineto{\pgfqpoint{3.097159in}{1.279648in}}%
\pgfpathlineto{\pgfqpoint{3.070083in}{1.208619in}}%
\pgfpathclose%
\pgfusepath{fill}%
\end{pgfscope}%
\begin{pgfscope}%
\pgfpathrectangle{\pgfqpoint{1.072000in}{0.528000in}}{\pgfqpoint{3.696000in}{3.696000in}}%
\pgfusepath{clip}%
\pgfsetbuttcap%
\pgfsetroundjoin%
\definecolor{currentfill}{rgb}{0.425199,0.559058,0.946061}%
\pgfsetfillcolor{currentfill}%
\pgfsetlinewidth{0.000000pt}%
\definecolor{currentstroke}{rgb}{0.000000,0.000000,0.000000}%
\pgfsetstrokecolor{currentstroke}%
\pgfsetdash{}{0pt}%
\pgfpathmoveto{\pgfqpoint{3.151592in}{1.494532in}}%
\pgfpathlineto{\pgfqpoint{3.198374in}{1.400462in}}%
\pgfpathlineto{\pgfqpoint{3.225655in}{1.513781in}}%
\pgfpathlineto{\pgfqpoint{3.179000in}{1.627834in}}%
\pgfpathlineto{\pgfqpoint{3.151592in}{1.494532in}}%
\pgfpathclose%
\pgfusepath{fill}%
\end{pgfscope}%
\begin{pgfscope}%
\pgfpathrectangle{\pgfqpoint{1.072000in}{0.528000in}}{\pgfqpoint{3.696000in}{3.696000in}}%
\pgfusepath{clip}%
\pgfsetbuttcap%
\pgfsetroundjoin%
\definecolor{currentfill}{rgb}{0.229806,0.298718,0.753683}%
\pgfsetfillcolor{currentfill}%
\pgfsetlinewidth{0.000000pt}%
\definecolor{currentstroke}{rgb}{0.000000,0.000000,0.000000}%
\pgfsetstrokecolor{currentstroke}%
\pgfsetdash{}{0pt}%
\pgfpathmoveto{\pgfqpoint{3.312675in}{1.186836in}}%
\pgfpathlineto{\pgfqpoint{3.360439in}{1.192792in}}%
\pgfpathlineto{\pgfqpoint{3.386561in}{1.199287in}}%
\pgfpathlineto{\pgfqpoint{3.339117in}{1.215165in}}%
\pgfpathlineto{\pgfqpoint{3.312675in}{1.186836in}}%
\pgfpathclose%
\pgfusepath{fill}%
\end{pgfscope}%
\begin{pgfscope}%
\pgfpathrectangle{\pgfqpoint{1.072000in}{0.528000in}}{\pgfqpoint{3.696000in}{3.696000in}}%
\pgfusepath{clip}%
\pgfsetbuttcap%
\pgfsetroundjoin%
\definecolor{currentfill}{rgb}{0.969289,0.684982,0.568975}%
\pgfsetfillcolor{currentfill}%
\pgfsetlinewidth{0.000000pt}%
\definecolor{currentstroke}{rgb}{0.000000,0.000000,0.000000}%
\pgfsetstrokecolor{currentstroke}%
\pgfsetdash{}{0pt}%
\pgfpathmoveto{\pgfqpoint{2.454671in}{2.445119in}}%
\pgfpathlineto{\pgfqpoint{2.499856in}{2.518153in}}%
\pgfpathlineto{\pgfqpoint{2.526038in}{2.690821in}}%
\pgfpathlineto{\pgfqpoint{2.480641in}{2.618006in}}%
\pgfpathlineto{\pgfqpoint{2.454671in}{2.445119in}}%
\pgfpathclose%
\pgfusepath{fill}%
\end{pgfscope}%
\begin{pgfscope}%
\pgfpathrectangle{\pgfqpoint{1.072000in}{0.528000in}}{\pgfqpoint{3.696000in}{3.696000in}}%
\pgfusepath{clip}%
\pgfsetbuttcap%
\pgfsetroundjoin%
\definecolor{currentfill}{rgb}{0.729196,0.086679,0.167240}%
\pgfsetfillcolor{currentfill}%
\pgfsetlinewidth{0.000000pt}%
\definecolor{currentstroke}{rgb}{0.000000,0.000000,0.000000}%
\pgfsetstrokecolor{currentstroke}%
\pgfsetdash{}{0pt}%
\pgfpathmoveto{\pgfqpoint{3.371354in}{3.166133in}}%
\pgfpathlineto{\pgfqpoint{3.418513in}{3.115583in}}%
\pgfpathlineto{\pgfqpoint{3.444311in}{3.105380in}}%
\pgfpathlineto{\pgfqpoint{3.397484in}{3.171954in}}%
\pgfpathlineto{\pgfqpoint{3.371354in}{3.166133in}}%
\pgfpathclose%
\pgfusepath{fill}%
\end{pgfscope}%
\begin{pgfscope}%
\pgfpathrectangle{\pgfqpoint{1.072000in}{0.528000in}}{\pgfqpoint{3.696000in}{3.696000in}}%
\pgfusepath{clip}%
\pgfsetbuttcap%
\pgfsetroundjoin%
\definecolor{currentfill}{rgb}{0.763520,0.178667,0.193396}%
\pgfsetfillcolor{currentfill}%
\pgfsetlinewidth{0.000000pt}%
\definecolor{currentstroke}{rgb}{0.000000,0.000000,0.000000}%
\pgfsetstrokecolor{currentstroke}%
\pgfsetdash{}{0pt}%
\pgfpathmoveto{\pgfqpoint{2.542370in}{3.034624in}}%
\pgfpathlineto{\pgfqpoint{2.588162in}{3.091597in}}%
\pgfpathlineto{\pgfqpoint{2.615758in}{3.148348in}}%
\pgfpathlineto{\pgfqpoint{2.569746in}{3.107212in}}%
\pgfpathlineto{\pgfqpoint{2.542370in}{3.034624in}}%
\pgfpathclose%
\pgfusepath{fill}%
\end{pgfscope}%
\begin{pgfscope}%
\pgfpathrectangle{\pgfqpoint{1.072000in}{0.528000in}}{\pgfqpoint{3.696000in}{3.696000in}}%
\pgfusepath{clip}%
\pgfsetbuttcap%
\pgfsetroundjoin%
\definecolor{currentfill}{rgb}{0.388852,0.516298,0.921373}%
\pgfsetfillcolor{currentfill}%
\pgfsetlinewidth{0.000000pt}%
\definecolor{currentstroke}{rgb}{0.000000,0.000000,0.000000}%
\pgfsetstrokecolor{currentstroke}%
\pgfsetdash{}{0pt}%
\pgfpathmoveto{\pgfqpoint{2.881931in}{1.411268in}}%
\pgfpathlineto{\pgfqpoint{2.928961in}{1.354317in}}%
\pgfpathlineto{\pgfqpoint{2.956096in}{1.483519in}}%
\pgfpathlineto{\pgfqpoint{2.908990in}{1.555989in}}%
\pgfpathlineto{\pgfqpoint{2.881931in}{1.411268in}}%
\pgfpathclose%
\pgfusepath{fill}%
\end{pgfscope}%
\begin{pgfscope}%
\pgfpathrectangle{\pgfqpoint{1.072000in}{0.528000in}}{\pgfqpoint{3.696000in}{3.696000in}}%
\pgfusepath{clip}%
\pgfsetbuttcap%
\pgfsetroundjoin%
\definecolor{currentfill}{rgb}{0.238948,0.312365,0.765676}%
\pgfsetfillcolor{currentfill}%
\pgfsetlinewidth{0.000000pt}%
\definecolor{currentstroke}{rgb}{0.000000,0.000000,0.000000}%
\pgfsetstrokecolor{currentstroke}%
\pgfsetdash{}{0pt}%
\pgfpathmoveto{\pgfqpoint{2.874592in}{1.191316in}}%
\pgfpathlineto{\pgfqpoint{2.921600in}{1.171468in}}%
\pgfpathlineto{\pgfqpoint{2.948828in}{1.218202in}}%
\pgfpathlineto{\pgfqpoint{2.901822in}{1.255349in}}%
\pgfpathlineto{\pgfqpoint{2.874592in}{1.191316in}}%
\pgfpathclose%
\pgfusepath{fill}%
\end{pgfscope}%
\begin{pgfscope}%
\pgfpathrectangle{\pgfqpoint{1.072000in}{0.528000in}}{\pgfqpoint{3.696000in}{3.696000in}}%
\pgfusepath{clip}%
\pgfsetbuttcap%
\pgfsetroundjoin%
\definecolor{currentfill}{rgb}{0.902659,0.447939,0.349721}%
\pgfsetfillcolor{currentfill}%
\pgfsetlinewidth{0.000000pt}%
\definecolor{currentstroke}{rgb}{0.000000,0.000000,0.000000}%
\pgfsetstrokecolor{currentstroke}%
\pgfsetdash{}{0pt}%
\pgfpathmoveto{\pgfqpoint{2.972614in}{2.880085in}}%
\pgfpathlineto{\pgfqpoint{3.019872in}{2.922975in}}%
\pgfpathlineto{\pgfqpoint{3.046770in}{2.843259in}}%
\pgfpathlineto{\pgfqpoint{2.999642in}{2.766222in}}%
\pgfpathlineto{\pgfqpoint{2.972614in}{2.880085in}}%
\pgfpathclose%
\pgfusepath{fill}%
\end{pgfscope}%
\begin{pgfscope}%
\pgfpathrectangle{\pgfqpoint{1.072000in}{0.528000in}}{\pgfqpoint{3.696000in}{3.696000in}}%
\pgfusepath{clip}%
\pgfsetbuttcap%
\pgfsetroundjoin%
\definecolor{currentfill}{rgb}{0.378598,0.503856,0.913692}%
\pgfsetfillcolor{currentfill}%
\pgfsetlinewidth{0.000000pt}%
\definecolor{currentstroke}{rgb}{0.000000,0.000000,0.000000}%
\pgfsetstrokecolor{currentstroke}%
\pgfsetdash{}{0pt}%
\pgfpathmoveto{\pgfqpoint{3.003150in}{1.410840in}}%
\pgfpathlineto{\pgfqpoint{3.050162in}{1.341592in}}%
\pgfpathlineto{\pgfqpoint{3.077387in}{1.459223in}}%
\pgfpathlineto{\pgfqpoint{3.030389in}{1.547372in}}%
\pgfpathlineto{\pgfqpoint{3.003150in}{1.410840in}}%
\pgfpathclose%
\pgfusepath{fill}%
\end{pgfscope}%
\begin{pgfscope}%
\pgfpathrectangle{\pgfqpoint{1.072000in}{0.528000in}}{\pgfqpoint{3.696000in}{3.696000in}}%
\pgfusepath{clip}%
\pgfsetbuttcap%
\pgfsetroundjoin%
\definecolor{currentfill}{rgb}{0.959518,0.766973,0.674145}%
\pgfsetfillcolor{currentfill}%
\pgfsetlinewidth{0.000000pt}%
\definecolor{currentstroke}{rgb}{0.000000,0.000000,0.000000}%
\pgfsetstrokecolor{currentstroke}%
\pgfsetdash{}{0pt}%
\pgfpathmoveto{\pgfqpoint{1.591085in}{2.557926in}}%
\pgfpathlineto{\pgfqpoint{1.641498in}{2.421031in}}%
\pgfpathlineto{\pgfqpoint{1.676308in}{2.312863in}}%
\pgfpathlineto{\pgfqpoint{1.625907in}{2.458106in}}%
\pgfpathlineto{\pgfqpoint{1.591085in}{2.557926in}}%
\pgfpathclose%
\pgfusepath{fill}%
\end{pgfscope}%
\begin{pgfscope}%
\pgfpathrectangle{\pgfqpoint{1.072000in}{0.528000in}}{\pgfqpoint{3.696000in}{3.696000in}}%
\pgfusepath{clip}%
\pgfsetbuttcap%
\pgfsetroundjoin%
\definecolor{currentfill}{rgb}{0.409611,0.540759,0.935545}%
\pgfsetfillcolor{currentfill}%
\pgfsetlinewidth{0.000000pt}%
\definecolor{currentstroke}{rgb}{0.000000,0.000000,0.000000}%
\pgfsetstrokecolor{currentstroke}%
\pgfsetdash{}{0pt}%
\pgfpathmoveto{\pgfqpoint{3.077387in}{1.459223in}}%
\pgfpathlineto{\pgfqpoint{3.124316in}{1.376588in}}%
\pgfpathlineto{\pgfqpoint{3.151592in}{1.494532in}}%
\pgfpathlineto{\pgfqpoint{3.104730in}{1.596811in}}%
\pgfpathlineto{\pgfqpoint{3.077387in}{1.459223in}}%
\pgfpathclose%
\pgfusepath{fill}%
\end{pgfscope}%
\begin{pgfscope}%
\pgfpathrectangle{\pgfqpoint{1.072000in}{0.528000in}}{\pgfqpoint{3.696000in}{3.696000in}}%
\pgfusepath{clip}%
\pgfsetbuttcap%
\pgfsetroundjoin%
\definecolor{currentfill}{rgb}{0.543440,0.680003,0.993051}%
\pgfsetfillcolor{currentfill}%
\pgfsetlinewidth{0.000000pt}%
\definecolor{currentstroke}{rgb}{0.000000,0.000000,0.000000}%
\pgfsetstrokecolor{currentstroke}%
\pgfsetdash{}{0pt}%
\pgfpathmoveto{\pgfqpoint{2.694027in}{1.603380in}}%
\pgfpathlineto{\pgfqpoint{2.740911in}{1.565157in}}%
\pgfpathlineto{\pgfqpoint{2.767561in}{1.741908in}}%
\pgfpathlineto{\pgfqpoint{2.720508in}{1.785868in}}%
\pgfpathlineto{\pgfqpoint{2.694027in}{1.603380in}}%
\pgfpathclose%
\pgfusepath{fill}%
\end{pgfscope}%
\begin{pgfscope}%
\pgfpathrectangle{\pgfqpoint{1.072000in}{0.528000in}}{\pgfqpoint{3.696000in}{3.696000in}}%
\pgfusepath{clip}%
\pgfsetbuttcap%
\pgfsetroundjoin%
\definecolor{currentfill}{rgb}{0.839351,0.861167,0.894494}%
\pgfsetfillcolor{currentfill}%
\pgfsetlinewidth{0.000000pt}%
\definecolor{currentstroke}{rgb}{0.000000,0.000000,0.000000}%
\pgfsetstrokecolor{currentstroke}%
\pgfsetdash{}{0pt}%
\pgfpathmoveto{\pgfqpoint{2.513864in}{2.022462in}}%
\pgfpathlineto{\pgfqpoint{2.559908in}{2.042837in}}%
\pgfpathlineto{\pgfqpoint{2.585802in}{2.250217in}}%
\pgfpathlineto{\pgfqpoint{2.539590in}{2.226110in}}%
\pgfpathlineto{\pgfqpoint{2.513864in}{2.022462in}}%
\pgfpathclose%
\pgfusepath{fill}%
\end{pgfscope}%
\begin{pgfscope}%
\pgfpathrectangle{\pgfqpoint{1.072000in}{0.528000in}}{\pgfqpoint{3.696000in}{3.696000in}}%
\pgfusepath{clip}%
\pgfsetbuttcap%
\pgfsetroundjoin%
\definecolor{currentfill}{rgb}{0.229806,0.298718,0.753683}%
\pgfsetfillcolor{currentfill}%
\pgfsetlinewidth{0.000000pt}%
\definecolor{currentstroke}{rgb}{0.000000,0.000000,0.000000}%
\pgfsetstrokecolor{currentstroke}%
\pgfsetdash{}{0pt}%
\pgfpathmoveto{\pgfqpoint{2.678896in}{1.186708in}}%
\pgfpathlineto{\pgfqpoint{2.725645in}{1.178711in}}%
\pgfpathlineto{\pgfqpoint{2.753400in}{1.191109in}}%
\pgfpathlineto{\pgfqpoint{2.706647in}{1.209286in}}%
\pgfpathlineto{\pgfqpoint{2.678896in}{1.186708in}}%
\pgfpathclose%
\pgfusepath{fill}%
\end{pgfscope}%
\begin{pgfscope}%
\pgfpathrectangle{\pgfqpoint{1.072000in}{0.528000in}}{\pgfqpoint{3.696000in}{3.696000in}}%
\pgfusepath{clip}%
\pgfsetbuttcap%
\pgfsetroundjoin%
\definecolor{currentfill}{rgb}{0.430507,0.564883,0.948889}%
\pgfsetfillcolor{currentfill}%
\pgfsetlinewidth{0.000000pt}%
\definecolor{currentstroke}{rgb}{0.000000,0.000000,0.000000}%
\pgfsetstrokecolor{currentstroke}%
\pgfsetdash{}{0pt}%
\pgfpathmoveto{\pgfqpoint{2.834898in}{1.467108in}}%
\pgfpathlineto{\pgfqpoint{2.881931in}{1.411268in}}%
\pgfpathlineto{\pgfqpoint{2.908990in}{1.555989in}}%
\pgfpathlineto{\pgfqpoint{2.861846in}{1.624981in}}%
\pgfpathlineto{\pgfqpoint{2.834898in}{1.467108in}}%
\pgfpathclose%
\pgfusepath{fill}%
\end{pgfscope}%
\begin{pgfscope}%
\pgfpathrectangle{\pgfqpoint{1.072000in}{0.528000in}}{\pgfqpoint{3.696000in}{3.696000in}}%
\pgfusepath{clip}%
\pgfsetbuttcap%
\pgfsetroundjoin%
\definecolor{currentfill}{rgb}{0.234377,0.305542,0.759680}%
\pgfsetfillcolor{currentfill}%
\pgfsetlinewidth{0.000000pt}%
\definecolor{currentstroke}{rgb}{0.000000,0.000000,0.000000}%
\pgfsetstrokecolor{currentstroke}%
\pgfsetdash{}{0pt}%
\pgfpathmoveto{\pgfqpoint{2.995887in}{1.187872in}}%
\pgfpathlineto{\pgfqpoint{3.043036in}{1.167349in}}%
\pgfpathlineto{\pgfqpoint{3.070083in}{1.208619in}}%
\pgfpathlineto{\pgfqpoint{3.023014in}{1.249603in}}%
\pgfpathlineto{\pgfqpoint{2.995887in}{1.187872in}}%
\pgfpathclose%
\pgfusepath{fill}%
\end{pgfscope}%
\begin{pgfscope}%
\pgfpathrectangle{\pgfqpoint{1.072000in}{0.528000in}}{\pgfqpoint{3.696000in}{3.696000in}}%
\pgfusepath{clip}%
\pgfsetbuttcap%
\pgfsetroundjoin%
\definecolor{currentfill}{rgb}{0.869655,0.379274,0.300941}%
\pgfsetfillcolor{currentfill}%
\pgfsetlinewidth{0.000000pt}%
\definecolor{currentstroke}{rgb}{0.000000,0.000000,0.000000}%
\pgfsetstrokecolor{currentstroke}%
\pgfsetdash{}{0pt}%
\pgfpathmoveto{\pgfqpoint{2.488414in}{2.824057in}}%
\pgfpathlineto{\pgfqpoint{2.533697in}{2.905132in}}%
\pgfpathlineto{\pgfqpoint{2.560779in}{3.011369in}}%
\pgfpathlineto{\pgfqpoint{2.515236in}{2.940657in}}%
\pgfpathlineto{\pgfqpoint{2.488414in}{2.824057in}}%
\pgfpathclose%
\pgfusepath{fill}%
\end{pgfscope}%
\begin{pgfscope}%
\pgfpathrectangle{\pgfqpoint{1.072000in}{0.528000in}}{\pgfqpoint{3.696000in}{3.696000in}}%
\pgfusepath{clip}%
\pgfsetbuttcap%
\pgfsetroundjoin%
\definecolor{currentfill}{rgb}{0.238948,0.312365,0.765676}%
\pgfsetfillcolor{currentfill}%
\pgfsetlinewidth{0.000000pt}%
\definecolor{currentstroke}{rgb}{0.000000,0.000000,0.000000}%
\pgfsetstrokecolor{currentstroke}%
\pgfsetdash{}{0pt}%
\pgfpathmoveto{\pgfqpoint{2.557321in}{1.218542in}}%
\pgfpathlineto{\pgfqpoint{2.603922in}{1.210740in}}%
\pgfpathlineto{\pgfqpoint{2.632246in}{1.197857in}}%
\pgfpathlineto{\pgfqpoint{2.585693in}{1.210976in}}%
\pgfpathlineto{\pgfqpoint{2.557321in}{1.218542in}}%
\pgfpathclose%
\pgfusepath{fill}%
\end{pgfscope}%
\begin{pgfscope}%
\pgfpathrectangle{\pgfqpoint{1.072000in}{0.528000in}}{\pgfqpoint{3.696000in}{3.696000in}}%
\pgfusepath{clip}%
\pgfsetbuttcap%
\pgfsetroundjoin%
\definecolor{currentfill}{rgb}{0.229806,0.298718,0.753683}%
\pgfsetfillcolor{currentfill}%
\pgfsetlinewidth{0.000000pt}%
\definecolor{currentstroke}{rgb}{0.000000,0.000000,0.000000}%
\pgfsetstrokecolor{currentstroke}%
\pgfsetdash{}{0pt}%
\pgfpathmoveto{\pgfqpoint{3.117243in}{1.179771in}}%
\pgfpathlineto{\pgfqpoint{3.164565in}{1.166223in}}%
\pgfpathlineto{\pgfqpoint{3.191336in}{1.193194in}}%
\pgfpathlineto{\pgfqpoint{3.144192in}{1.228930in}}%
\pgfpathlineto{\pgfqpoint{3.117243in}{1.179771in}}%
\pgfpathclose%
\pgfusepath{fill}%
\end{pgfscope}%
\begin{pgfscope}%
\pgfpathrectangle{\pgfqpoint{1.072000in}{0.528000in}}{\pgfqpoint{3.696000in}{3.696000in}}%
\pgfusepath{clip}%
\pgfsetbuttcap%
\pgfsetroundjoin%
\definecolor{currentfill}{rgb}{0.956371,0.775144,0.686416}%
\pgfsetfillcolor{currentfill}%
\pgfsetlinewidth{0.000000pt}%
\definecolor{currentstroke}{rgb}{0.000000,0.000000,0.000000}%
\pgfsetstrokecolor{currentstroke}%
\pgfsetdash{}{0pt}%
\pgfpathmoveto{\pgfqpoint{3.471866in}{2.489179in}}%
\pgfpathlineto{\pgfqpoint{3.516865in}{2.295966in}}%
\pgfpathlineto{\pgfqpoint{3.543220in}{2.342629in}}%
\pgfpathlineto{\pgfqpoint{3.498398in}{2.536341in}}%
\pgfpathlineto{\pgfqpoint{3.471866in}{2.489179in}}%
\pgfpathclose%
\pgfusepath{fill}%
\end{pgfscope}%
\begin{pgfscope}%
\pgfpathrectangle{\pgfqpoint{1.072000in}{0.528000in}}{\pgfqpoint{3.696000in}{3.696000in}}%
\pgfusepath{clip}%
\pgfsetbuttcap%
\pgfsetroundjoin%
\definecolor{currentfill}{rgb}{0.510824,0.649397,0.985079}%
\pgfsetfillcolor{currentfill}%
\pgfsetlinewidth{0.000000pt}%
\definecolor{currentstroke}{rgb}{0.000000,0.000000,0.000000}%
\pgfsetstrokecolor{currentstroke}%
\pgfsetdash{}{0pt}%
\pgfpathmoveto{\pgfqpoint{2.740911in}{1.565157in}}%
\pgfpathlineto{\pgfqpoint{2.787880in}{1.519151in}}%
\pgfpathlineto{\pgfqpoint{2.814689in}{1.687707in}}%
\pgfpathlineto{\pgfqpoint{2.767561in}{1.741908in}}%
\pgfpathlineto{\pgfqpoint{2.740911in}{1.565157in}}%
\pgfpathclose%
\pgfusepath{fill}%
\end{pgfscope}%
\begin{pgfscope}%
\pgfpathrectangle{\pgfqpoint{1.072000in}{0.528000in}}{\pgfqpoint{3.696000in}{3.696000in}}%
\pgfusepath{clip}%
\pgfsetbuttcap%
\pgfsetroundjoin%
\definecolor{currentfill}{rgb}{0.810616,0.268797,0.235428}%
\pgfsetfillcolor{currentfill}%
\pgfsetlinewidth{0.000000pt}%
\definecolor{currentstroke}{rgb}{0.000000,0.000000,0.000000}%
\pgfsetstrokecolor{currentstroke}%
\pgfsetdash{}{0pt}%
\pgfpathmoveto{\pgfqpoint{2.515236in}{2.940657in}}%
\pgfpathlineto{\pgfqpoint{2.560779in}{3.011369in}}%
\pgfpathlineto{\pgfqpoint{2.588162in}{3.091597in}}%
\pgfpathlineto{\pgfqpoint{2.542370in}{3.034624in}}%
\pgfpathlineto{\pgfqpoint{2.515236in}{2.940657in}}%
\pgfpathclose%
\pgfusepath{fill}%
\end{pgfscope}%
\begin{pgfscope}%
\pgfpathrectangle{\pgfqpoint{1.072000in}{0.528000in}}{\pgfqpoint{3.696000in}{3.696000in}}%
\pgfusepath{clip}%
\pgfsetbuttcap%
\pgfsetroundjoin%
\definecolor{currentfill}{rgb}{0.473070,0.611077,0.970634}%
\pgfsetfillcolor{currentfill}%
\pgfsetlinewidth{0.000000pt}%
\definecolor{currentstroke}{rgb}{0.000000,0.000000,0.000000}%
\pgfsetstrokecolor{currentstroke}%
\pgfsetdash{}{0pt}%
\pgfpathmoveto{\pgfqpoint{2.787880in}{1.519151in}}%
\pgfpathlineto{\pgfqpoint{2.834898in}{1.467108in}}%
\pgfpathlineto{\pgfqpoint{2.861846in}{1.624981in}}%
\pgfpathlineto{\pgfqpoint{2.814689in}{1.687707in}}%
\pgfpathlineto{\pgfqpoint{2.787880in}{1.519151in}}%
\pgfpathclose%
\pgfusepath{fill}%
\end{pgfscope}%
\begin{pgfscope}%
\pgfpathrectangle{\pgfqpoint{1.072000in}{0.528000in}}{\pgfqpoint{3.696000in}{3.696000in}}%
\pgfusepath{clip}%
\pgfsetbuttcap%
\pgfsetroundjoin%
\definecolor{currentfill}{rgb}{0.603162,0.731527,0.999565}%
\pgfsetfillcolor{currentfill}%
\pgfsetlinewidth{0.000000pt}%
\definecolor{currentstroke}{rgb}{0.000000,0.000000,0.000000}%
\pgfsetstrokecolor{currentstroke}%
\pgfsetdash{}{0pt}%
\pgfpathmoveto{\pgfqpoint{3.280568in}{1.769041in}}%
\pgfpathlineto{\pgfqpoint{3.326779in}{1.628749in}}%
\pgfpathlineto{\pgfqpoint{3.354162in}{1.744685in}}%
\pgfpathlineto{\pgfqpoint{3.308138in}{1.900219in}}%
\pgfpathlineto{\pgfqpoint{3.280568in}{1.769041in}}%
\pgfpathclose%
\pgfusepath{fill}%
\end{pgfscope}%
\begin{pgfscope}%
\pgfpathrectangle{\pgfqpoint{1.072000in}{0.528000in}}{\pgfqpoint{3.696000in}{3.696000in}}%
\pgfusepath{clip}%
\pgfsetbuttcap%
\pgfsetroundjoin%
\definecolor{currentfill}{rgb}{0.959385,0.610306,0.489382}%
\pgfsetfillcolor{currentfill}%
\pgfsetlinewidth{0.000000pt}%
\definecolor{currentstroke}{rgb}{0.000000,0.000000,0.000000}%
\pgfsetstrokecolor{currentstroke}%
\pgfsetdash{}{0pt}%
\pgfpathmoveto{\pgfqpoint{3.479347in}{2.744727in}}%
\pgfpathlineto{\pgfqpoint{3.524605in}{2.567539in}}%
\pgfpathlineto{\pgfqpoint{3.550453in}{2.582835in}}%
\pgfpathlineto{\pgfqpoint{3.505394in}{2.760031in}}%
\pgfpathlineto{\pgfqpoint{3.479347in}{2.744727in}}%
\pgfpathclose%
\pgfusepath{fill}%
\end{pgfscope}%
\begin{pgfscope}%
\pgfpathrectangle{\pgfqpoint{1.072000in}{0.528000in}}{\pgfqpoint{3.696000in}{3.696000in}}%
\pgfusepath{clip}%
\pgfsetbuttcap%
\pgfsetroundjoin%
\definecolor{currentfill}{rgb}{0.229806,0.298718,0.753683}%
\pgfsetfillcolor{currentfill}%
\pgfsetlinewidth{0.000000pt}%
\definecolor{currentstroke}{rgb}{0.000000,0.000000,0.000000}%
\pgfsetstrokecolor{currentstroke}%
\pgfsetdash{}{0pt}%
\pgfpathmoveto{\pgfqpoint{2.800240in}{1.175770in}}%
\pgfpathlineto{\pgfqpoint{2.847172in}{1.165191in}}%
\pgfpathlineto{\pgfqpoint{2.874592in}{1.191316in}}%
\pgfpathlineto{\pgfqpoint{2.827656in}{1.216660in}}%
\pgfpathlineto{\pgfqpoint{2.800240in}{1.175770in}}%
\pgfpathclose%
\pgfusepath{fill}%
\end{pgfscope}%
\begin{pgfscope}%
\pgfpathrectangle{\pgfqpoint{1.072000in}{0.528000in}}{\pgfqpoint{3.696000in}{3.696000in}}%
\pgfusepath{clip}%
\pgfsetbuttcap%
\pgfsetroundjoin%
\definecolor{currentfill}{rgb}{0.728970,0.817464,0.973188}%
\pgfsetfillcolor{currentfill}%
\pgfsetlinewidth{0.000000pt}%
\definecolor{currentstroke}{rgb}{0.000000,0.000000,0.000000}%
\pgfsetstrokecolor{currentstroke}%
\pgfsetdash{}{0pt}%
\pgfpathmoveto{\pgfqpoint{2.580333in}{1.846953in}}%
\pgfpathlineto{\pgfqpoint{2.626841in}{1.838835in}}%
\pgfpathlineto{\pgfqpoint{2.652980in}{2.042876in}}%
\pgfpathlineto{\pgfqpoint{2.606296in}{2.049851in}}%
\pgfpathlineto{\pgfqpoint{2.580333in}{1.846953in}}%
\pgfpathclose%
\pgfusepath{fill}%
\end{pgfscope}%
\begin{pgfscope}%
\pgfpathrectangle{\pgfqpoint{1.072000in}{0.528000in}}{\pgfqpoint{3.696000in}{3.696000in}}%
\pgfusepath{clip}%
\pgfsetbuttcap%
\pgfsetroundjoin%
\definecolor{currentfill}{rgb}{0.852378,0.346492,0.280346}%
\pgfsetfillcolor{currentfill}%
\pgfsetlinewidth{0.000000pt}%
\definecolor{currentstroke}{rgb}{0.000000,0.000000,0.000000}%
\pgfsetstrokecolor{currentstroke}%
\pgfsetdash{}{0pt}%
\pgfpathmoveto{\pgfqpoint{3.121001in}{2.899182in}}%
\pgfpathlineto{\pgfqpoint{3.168906in}{2.998194in}}%
\pgfpathlineto{\pgfqpoint{3.195641in}{3.008207in}}%
\pgfpathlineto{\pgfqpoint{3.147684in}{2.899623in}}%
\pgfpathlineto{\pgfqpoint{3.121001in}{2.899182in}}%
\pgfpathclose%
\pgfusepath{fill}%
\end{pgfscope}%
\begin{pgfscope}%
\pgfpathrectangle{\pgfqpoint{1.072000in}{0.528000in}}{\pgfqpoint{3.696000in}{3.696000in}}%
\pgfusepath{clip}%
\pgfsetbuttcap%
\pgfsetroundjoin%
\definecolor{currentfill}{rgb}{0.229806,0.298718,0.753683}%
\pgfsetfillcolor{currentfill}%
\pgfsetlinewidth{0.000000pt}%
\definecolor{currentstroke}{rgb}{0.000000,0.000000,0.000000}%
\pgfsetstrokecolor{currentstroke}%
\pgfsetdash{}{0pt}%
\pgfpathmoveto{\pgfqpoint{3.238680in}{1.175813in}}%
\pgfpathlineto{\pgfqpoint{3.286330in}{1.179550in}}%
\pgfpathlineto{\pgfqpoint{3.312675in}{1.186836in}}%
\pgfpathlineto{\pgfqpoint{3.265299in}{1.205260in}}%
\pgfpathlineto{\pgfqpoint{3.238680in}{1.175813in}}%
\pgfpathclose%
\pgfusepath{fill}%
\end{pgfscope}%
\begin{pgfscope}%
\pgfpathrectangle{\pgfqpoint{1.072000in}{0.528000in}}{\pgfqpoint{3.696000in}{3.696000in}}%
\pgfusepath{clip}%
\pgfsetbuttcap%
\pgfsetroundjoin%
\definecolor{currentfill}{rgb}{0.839365,0.321856,0.264924}%
\pgfsetfillcolor{currentfill}%
\pgfsetlinewidth{0.000000pt}%
\definecolor{currentstroke}{rgb}{0.000000,0.000000,0.000000}%
\pgfsetstrokecolor{currentstroke}%
\pgfsetdash{}{0pt}%
\pgfpathmoveto{\pgfqpoint{3.439296in}{3.032240in}}%
\pgfpathlineto{\pgfqpoint{3.485545in}{2.912803in}}%
\pgfpathlineto{\pgfqpoint{3.511097in}{2.900102in}}%
\pgfpathlineto{\pgfqpoint{3.465132in}{3.025932in}}%
\pgfpathlineto{\pgfqpoint{3.439296in}{3.032240in}}%
\pgfpathclose%
\pgfusepath{fill}%
\end{pgfscope}%
\begin{pgfscope}%
\pgfpathrectangle{\pgfqpoint{1.072000in}{0.528000in}}{\pgfqpoint{3.696000in}{3.696000in}}%
\pgfusepath{clip}%
\pgfsetbuttcap%
\pgfsetroundjoin%
\definecolor{currentfill}{rgb}{0.705673,0.015556,0.150233}%
\pgfsetfillcolor{currentfill}%
\pgfsetlinewidth{0.000000pt}%
\definecolor{currentstroke}{rgb}{0.000000,0.000000,0.000000}%
\pgfsetstrokecolor{currentstroke}%
\pgfsetdash{}{0pt}%
\pgfpathmoveto{\pgfqpoint{2.615758in}{3.148348in}}%
\pgfpathlineto{\pgfqpoint{2.662314in}{3.155786in}}%
\pgfpathlineto{\pgfqpoint{2.690161in}{3.173291in}}%
\pgfpathlineto{\pgfqpoint{2.643465in}{3.186555in}}%
\pgfpathlineto{\pgfqpoint{2.615758in}{3.148348in}}%
\pgfpathclose%
\pgfusepath{fill}%
\end{pgfscope}%
\begin{pgfscope}%
\pgfpathrectangle{\pgfqpoint{1.072000in}{0.528000in}}{\pgfqpoint{3.696000in}{3.696000in}}%
\pgfusepath{clip}%
\pgfsetbuttcap%
\pgfsetroundjoin%
\definecolor{currentfill}{rgb}{0.962708,0.753557,0.655601}%
\pgfsetfillcolor{currentfill}%
\pgfsetlinewidth{0.000000pt}%
\definecolor{currentstroke}{rgb}{0.000000,0.000000,0.000000}%
\pgfsetstrokecolor{currentstroke}%
\pgfsetdash{}{0pt}%
\pgfpathmoveto{\pgfqpoint{2.474016in}{2.330583in}}%
\pgfpathlineto{\pgfqpoint{2.519533in}{2.386619in}}%
\pgfpathlineto{\pgfqpoint{2.545567in}{2.575881in}}%
\pgfpathlineto{\pgfqpoint{2.499856in}{2.518153in}}%
\pgfpathlineto{\pgfqpoint{2.474016in}{2.330583in}}%
\pgfpathclose%
\pgfusepath{fill}%
\end{pgfscope}%
\begin{pgfscope}%
\pgfpathrectangle{\pgfqpoint{1.072000in}{0.528000in}}{\pgfqpoint{3.696000in}{3.696000in}}%
\pgfusepath{clip}%
\pgfsetbuttcap%
\pgfsetroundjoin%
\definecolor{currentfill}{rgb}{0.521696,0.659599,0.987736}%
\pgfsetfillcolor{currentfill}%
\pgfsetlinewidth{0.000000pt}%
\definecolor{currentstroke}{rgb}{0.000000,0.000000,0.000000}%
\pgfsetstrokecolor{currentstroke}%
\pgfsetdash{}{0pt}%
\pgfpathmoveto{\pgfqpoint{3.179000in}{1.627834in}}%
\pgfpathlineto{\pgfqpoint{3.225655in}{1.513781in}}%
\pgfpathlineto{\pgfqpoint{3.253061in}{1.638469in}}%
\pgfpathlineto{\pgfqpoint{3.206535in}{1.770462in}}%
\pgfpathlineto{\pgfqpoint{3.179000in}{1.627834in}}%
\pgfpathclose%
\pgfusepath{fill}%
\end{pgfscope}%
\begin{pgfscope}%
\pgfpathrectangle{\pgfqpoint{1.072000in}{0.528000in}}{\pgfqpoint{3.696000in}{3.696000in}}%
\pgfusepath{clip}%
\pgfsetbuttcap%
\pgfsetroundjoin%
\definecolor{currentfill}{rgb}{0.271104,0.360011,0.807095}%
\pgfsetfillcolor{currentfill}%
\pgfsetlinewidth{0.000000pt}%
\definecolor{currentstroke}{rgb}{0.000000,0.000000,0.000000}%
\pgfsetstrokecolor{currentstroke}%
\pgfsetdash{}{0pt}%
\pgfpathmoveto{\pgfqpoint{3.508575in}{1.236128in}}%
\pgfpathlineto{\pgfqpoint{3.557543in}{1.288614in}}%
\pgfpathlineto{\pgfqpoint{3.582762in}{1.270841in}}%
\pgfpathlineto{\pgfqpoint{3.534162in}{1.235463in}}%
\pgfpathlineto{\pgfqpoint{3.508575in}{1.236128in}}%
\pgfpathclose%
\pgfusepath{fill}%
\end{pgfscope}%
\begin{pgfscope}%
\pgfpathrectangle{\pgfqpoint{1.072000in}{0.528000in}}{\pgfqpoint{3.696000in}{3.696000in}}%
\pgfusepath{clip}%
\pgfsetbuttcap%
\pgfsetroundjoin%
\definecolor{currentfill}{rgb}{0.441123,0.576532,0.954545}%
\pgfsetfillcolor{currentfill}%
\pgfsetlinewidth{0.000000pt}%
\definecolor{currentstroke}{rgb}{0.000000,0.000000,0.000000}%
\pgfsetstrokecolor{currentstroke}%
\pgfsetdash{}{0pt}%
\pgfpathmoveto{\pgfqpoint{2.956096in}{1.483519in}}%
\pgfpathlineto{\pgfqpoint{3.003150in}{1.410840in}}%
\pgfpathlineto{\pgfqpoint{3.030389in}{1.547372in}}%
\pgfpathlineto{\pgfqpoint{2.983297in}{1.636792in}}%
\pgfpathlineto{\pgfqpoint{2.956096in}{1.483519in}}%
\pgfpathclose%
\pgfusepath{fill}%
\end{pgfscope}%
\begin{pgfscope}%
\pgfpathrectangle{\pgfqpoint{1.072000in}{0.528000in}}{\pgfqpoint{3.696000in}{3.696000in}}%
\pgfusepath{clip}%
\pgfsetbuttcap%
\pgfsetroundjoin%
\definecolor{currentfill}{rgb}{0.294718,0.393542,0.834384}%
\pgfsetfillcolor{currentfill}%
\pgfsetlinewidth{0.000000pt}%
\definecolor{currentstroke}{rgb}{0.000000,0.000000,0.000000}%
\pgfsetstrokecolor{currentstroke}%
\pgfsetdash{}{0pt}%
\pgfpathmoveto{\pgfqpoint{3.582762in}{1.270841in}}%
\pgfpathlineto{\pgfqpoint{3.632262in}{1.338669in}}%
\pgfpathlineto{\pgfqpoint{3.657196in}{1.318639in}}%
\pgfpathlineto{\pgfqpoint{3.608051in}{1.265632in}}%
\pgfpathlineto{\pgfqpoint{3.582762in}{1.270841in}}%
\pgfpathclose%
\pgfusepath{fill}%
\end{pgfscope}%
\begin{pgfscope}%
\pgfpathrectangle{\pgfqpoint{1.072000in}{0.528000in}}{\pgfqpoint{3.696000in}{3.696000in}}%
\pgfusepath{clip}%
\pgfsetbuttcap%
\pgfsetroundjoin%
\definecolor{currentfill}{rgb}{0.928116,0.822197,0.765141}%
\pgfsetfillcolor{currentfill}%
\pgfsetlinewidth{0.000000pt}%
\definecolor{currentstroke}{rgb}{0.000000,0.000000,0.000000}%
\pgfsetstrokecolor{currentstroke}%
\pgfsetdash{}{0pt}%
\pgfpathmoveto{\pgfqpoint{2.493786in}{2.187897in}}%
\pgfpathlineto{\pgfqpoint{2.539590in}{2.226110in}}%
\pgfpathlineto{\pgfqpoint{2.565517in}{2.428025in}}%
\pgfpathlineto{\pgfqpoint{2.519533in}{2.386619in}}%
\pgfpathlineto{\pgfqpoint{2.493786in}{2.187897in}}%
\pgfpathclose%
\pgfusepath{fill}%
\end{pgfscope}%
\begin{pgfscope}%
\pgfpathrectangle{\pgfqpoint{1.072000in}{0.528000in}}{\pgfqpoint{3.696000in}{3.696000in}}%
\pgfusepath{clip}%
\pgfsetbuttcap%
\pgfsetroundjoin%
\definecolor{currentfill}{rgb}{0.758112,0.168122,0.188827}%
\pgfsetfillcolor{currentfill}%
\pgfsetlinewidth{0.000000pt}%
\definecolor{currentstroke}{rgb}{0.000000,0.000000,0.000000}%
\pgfsetstrokecolor{currentstroke}%
\pgfsetdash{}{0pt}%
\pgfpathmoveto{\pgfqpoint{3.222425in}{3.042911in}}%
\pgfpathlineto{\pgfqpoint{3.270474in}{3.105629in}}%
\pgfpathlineto{\pgfqpoint{3.297190in}{3.141894in}}%
\pgfpathlineto{\pgfqpoint{3.249272in}{3.096807in}}%
\pgfpathlineto{\pgfqpoint{3.222425in}{3.042911in}}%
\pgfpathclose%
\pgfusepath{fill}%
\end{pgfscope}%
\begin{pgfscope}%
\pgfpathrectangle{\pgfqpoint{1.072000in}{0.528000in}}{\pgfqpoint{3.696000in}{3.696000in}}%
\pgfusepath{clip}%
\pgfsetbuttcap%
\pgfsetroundjoin%
\definecolor{currentfill}{rgb}{0.229806,0.298718,0.753683}%
\pgfsetfillcolor{currentfill}%
\pgfsetlinewidth{0.000000pt}%
\definecolor{currentstroke}{rgb}{0.000000,0.000000,0.000000}%
\pgfsetstrokecolor{currentstroke}%
\pgfsetdash{}{0pt}%
\pgfpathmoveto{\pgfqpoint{2.921600in}{1.171468in}}%
\pgfpathlineto{\pgfqpoint{2.968706in}{1.159651in}}%
\pgfpathlineto{\pgfqpoint{2.995887in}{1.187872in}}%
\pgfpathlineto{\pgfqpoint{2.948828in}{1.218202in}}%
\pgfpathlineto{\pgfqpoint{2.921600in}{1.171468in}}%
\pgfpathclose%
\pgfusepath{fill}%
\end{pgfscope}%
\begin{pgfscope}%
\pgfpathrectangle{\pgfqpoint{1.072000in}{0.528000in}}{\pgfqpoint{3.696000in}{3.696000in}}%
\pgfusepath{clip}%
\pgfsetbuttcap%
\pgfsetroundjoin%
\definecolor{currentfill}{rgb}{0.257234,0.339661,0.789661}%
\pgfsetfillcolor{currentfill}%
\pgfsetlinewidth{0.000000pt}%
\definecolor{currentstroke}{rgb}{0.000000,0.000000,0.000000}%
\pgfsetstrokecolor{currentstroke}%
\pgfsetdash{}{0pt}%
\pgfpathmoveto{\pgfqpoint{3.434500in}{1.211046in}}%
\pgfpathlineto{\pgfqpoint{3.483068in}{1.251509in}}%
\pgfpathlineto{\pgfqpoint{3.508575in}{1.236128in}}%
\pgfpathlineto{\pgfqpoint{3.460368in}{1.214592in}}%
\pgfpathlineto{\pgfqpoint{3.434500in}{1.211046in}}%
\pgfpathclose%
\pgfusepath{fill}%
\end{pgfscope}%
\begin{pgfscope}%
\pgfpathrectangle{\pgfqpoint{1.072000in}{0.528000in}}{\pgfqpoint{3.696000in}{3.696000in}}%
\pgfusepath{clip}%
\pgfsetbuttcap%
\pgfsetroundjoin%
\definecolor{currentfill}{rgb}{0.229806,0.298718,0.753683}%
\pgfsetfillcolor{currentfill}%
\pgfsetlinewidth{0.000000pt}%
\definecolor{currentstroke}{rgb}{0.000000,0.000000,0.000000}%
\pgfsetstrokecolor{currentstroke}%
\pgfsetdash{}{0pt}%
\pgfpathmoveto{\pgfqpoint{3.043036in}{1.167349in}}%
\pgfpathlineto{\pgfqpoint{3.090326in}{1.159496in}}%
\pgfpathlineto{\pgfqpoint{3.117243in}{1.179771in}}%
\pgfpathlineto{\pgfqpoint{3.070083in}{1.208619in}}%
\pgfpathlineto{\pgfqpoint{3.043036in}{1.167349in}}%
\pgfpathclose%
\pgfusepath{fill}%
\end{pgfscope}%
\begin{pgfscope}%
\pgfpathrectangle{\pgfqpoint{1.072000in}{0.528000in}}{\pgfqpoint{3.696000in}{3.696000in}}%
\pgfusepath{clip}%
\pgfsetbuttcap%
\pgfsetroundjoin%
\definecolor{currentfill}{rgb}{0.905783,0.455186,0.355336}%
\pgfsetfillcolor{currentfill}%
\pgfsetlinewidth{0.000000pt}%
\definecolor{currentstroke}{rgb}{0.000000,0.000000,0.000000}%
\pgfsetstrokecolor{currentstroke}%
\pgfsetdash{}{0pt}%
\pgfpathmoveto{\pgfqpoint{3.459628in}{2.910675in}}%
\pgfpathlineto{\pgfqpoint{3.505394in}{2.760031in}}%
\pgfpathlineto{\pgfqpoint{3.531078in}{2.759913in}}%
\pgfpathlineto{\pgfqpoint{3.485545in}{2.912803in}}%
\pgfpathlineto{\pgfqpoint{3.459628in}{2.910675in}}%
\pgfpathclose%
\pgfusepath{fill}%
\end{pgfscope}%
\begin{pgfscope}%
\pgfpathrectangle{\pgfqpoint{1.072000in}{0.528000in}}{\pgfqpoint{3.696000in}{3.696000in}}%
\pgfusepath{clip}%
\pgfsetbuttcap%
\pgfsetroundjoin%
\definecolor{currentfill}{rgb}{0.916071,0.833977,0.788693}%
\pgfsetfillcolor{currentfill}%
\pgfsetlinewidth{0.000000pt}%
\definecolor{currentstroke}{rgb}{0.000000,0.000000,0.000000}%
\pgfsetstrokecolor{currentstroke}%
\pgfsetdash{}{0pt}%
\pgfpathmoveto{\pgfqpoint{1.656665in}{2.397557in}}%
\pgfpathlineto{\pgfqpoint{1.706765in}{2.266873in}}%
\pgfpathlineto{\pgfqpoint{1.741682in}{2.147169in}}%
\pgfpathlineto{\pgfqpoint{1.691750in}{2.282711in}}%
\pgfpathlineto{\pgfqpoint{1.656665in}{2.397557in}}%
\pgfpathclose%
\pgfusepath{fill}%
\end{pgfscope}%
\begin{pgfscope}%
\pgfpathrectangle{\pgfqpoint{1.072000in}{0.528000in}}{\pgfqpoint{3.696000in}{3.696000in}}%
\pgfusepath{clip}%
\pgfsetbuttcap%
\pgfsetroundjoin%
\definecolor{currentfill}{rgb}{0.698454,0.799450,0.984577}%
\pgfsetfillcolor{currentfill}%
\pgfsetlinewidth{0.000000pt}%
\definecolor{currentstroke}{rgb}{0.000000,0.000000,0.000000}%
\pgfsetstrokecolor{currentstroke}%
\pgfsetdash{}{0pt}%
\pgfpathmoveto{\pgfqpoint{3.308138in}{1.900219in}}%
\pgfpathlineto{\pgfqpoint{3.354162in}{1.744685in}}%
\pgfpathlineto{\pgfqpoint{3.381561in}{1.859254in}}%
\pgfpathlineto{\pgfqpoint{3.335720in}{2.027224in}}%
\pgfpathlineto{\pgfqpoint{3.308138in}{1.900219in}}%
\pgfpathclose%
\pgfusepath{fill}%
\end{pgfscope}%
\begin{pgfscope}%
\pgfpathrectangle{\pgfqpoint{1.072000in}{0.528000in}}{\pgfqpoint{3.696000in}{3.696000in}}%
\pgfusepath{clip}%
\pgfsetbuttcap%
\pgfsetroundjoin%
\definecolor{currentfill}{rgb}{0.483854,0.622050,0.974808}%
\pgfsetfillcolor{currentfill}%
\pgfsetlinewidth{0.000000pt}%
\definecolor{currentstroke}{rgb}{0.000000,0.000000,0.000000}%
\pgfsetstrokecolor{currentstroke}%
\pgfsetdash{}{0pt}%
\pgfpathmoveto{\pgfqpoint{3.030389in}{1.547372in}}%
\pgfpathlineto{\pgfqpoint{3.077387in}{1.459223in}}%
\pgfpathlineto{\pgfqpoint{3.104730in}{1.596811in}}%
\pgfpathlineto{\pgfqpoint{3.057743in}{1.702438in}}%
\pgfpathlineto{\pgfqpoint{3.030389in}{1.547372in}}%
\pgfpathclose%
\pgfusepath{fill}%
\end{pgfscope}%
\begin{pgfscope}%
\pgfpathrectangle{\pgfqpoint{1.072000in}{0.528000in}}{\pgfqpoint{3.696000in}{3.696000in}}%
\pgfusepath{clip}%
\pgfsetbuttcap%
\pgfsetroundjoin%
\definecolor{currentfill}{rgb}{0.510824,0.649397,0.985079}%
\pgfsetfillcolor{currentfill}%
\pgfsetlinewidth{0.000000pt}%
\definecolor{currentstroke}{rgb}{0.000000,0.000000,0.000000}%
\pgfsetstrokecolor{currentstroke}%
\pgfsetdash{}{0pt}%
\pgfpathmoveto{\pgfqpoint{3.104730in}{1.596811in}}%
\pgfpathlineto{\pgfqpoint{3.151592in}{1.494532in}}%
\pgfpathlineto{\pgfqpoint{3.179000in}{1.627834in}}%
\pgfpathlineto{\pgfqpoint{3.132207in}{1.747956in}}%
\pgfpathlineto{\pgfqpoint{3.104730in}{1.596811in}}%
\pgfpathclose%
\pgfusepath{fill}%
\end{pgfscope}%
\begin{pgfscope}%
\pgfpathrectangle{\pgfqpoint{1.072000in}{0.528000in}}{\pgfqpoint{3.696000in}{3.696000in}}%
\pgfusepath{clip}%
\pgfsetbuttcap%
\pgfsetroundjoin%
\definecolor{currentfill}{rgb}{0.705673,0.015556,0.150233}%
\pgfsetfillcolor{currentfill}%
\pgfsetlinewidth{0.000000pt}%
\definecolor{currentstroke}{rgb}{0.000000,0.000000,0.000000}%
\pgfsetstrokecolor{currentstroke}%
\pgfsetdash{}{0pt}%
\pgfpathmoveto{\pgfqpoint{3.297190in}{3.141894in}}%
\pgfpathlineto{\pgfqpoint{3.344967in}{3.148332in}}%
\pgfpathlineto{\pgfqpoint{3.371354in}{3.166133in}}%
\pgfpathlineto{\pgfqpoint{3.323822in}{3.178274in}}%
\pgfpathlineto{\pgfqpoint{3.297190in}{3.141894in}}%
\pgfpathclose%
\pgfusepath{fill}%
\end{pgfscope}%
\begin{pgfscope}%
\pgfpathrectangle{\pgfqpoint{1.072000in}{0.528000in}}{\pgfqpoint{3.696000in}{3.696000in}}%
\pgfusepath{clip}%
\pgfsetbuttcap%
\pgfsetroundjoin%
\definecolor{currentfill}{rgb}{0.947345,0.794696,0.716991}%
\pgfsetfillcolor{currentfill}%
\pgfsetlinewidth{0.000000pt}%
\definecolor{currentstroke}{rgb}{0.000000,0.000000,0.000000}%
\pgfsetstrokecolor{currentstroke}%
\pgfsetdash{}{0pt}%
\pgfpathmoveto{\pgfqpoint{3.445045in}{2.426067in}}%
\pgfpathlineto{\pgfqpoint{3.490212in}{2.234241in}}%
\pgfpathlineto{\pgfqpoint{3.516865in}{2.295966in}}%
\pgfpathlineto{\pgfqpoint{3.471866in}{2.489179in}}%
\pgfpathlineto{\pgfqpoint{3.445045in}{2.426067in}}%
\pgfpathclose%
\pgfusepath{fill}%
\end{pgfscope}%
\begin{pgfscope}%
\pgfpathrectangle{\pgfqpoint{1.072000in}{0.528000in}}{\pgfqpoint{3.696000in}{3.696000in}}%
\pgfusepath{clip}%
\pgfsetbuttcap%
\pgfsetroundjoin%
\definecolor{currentfill}{rgb}{0.229806,0.298718,0.753683}%
\pgfsetfillcolor{currentfill}%
\pgfsetlinewidth{0.000000pt}%
\definecolor{currentstroke}{rgb}{0.000000,0.000000,0.000000}%
\pgfsetstrokecolor{currentstroke}%
\pgfsetdash{}{0pt}%
\pgfpathmoveto{\pgfqpoint{2.725645in}{1.178711in}}%
\pgfpathlineto{\pgfqpoint{2.772496in}{1.175290in}}%
\pgfpathlineto{\pgfqpoint{2.800240in}{1.175770in}}%
\pgfpathlineto{\pgfqpoint{2.753400in}{1.191109in}}%
\pgfpathlineto{\pgfqpoint{2.725645in}{1.178711in}}%
\pgfpathclose%
\pgfusepath{fill}%
\end{pgfscope}%
\begin{pgfscope}%
\pgfpathrectangle{\pgfqpoint{1.072000in}{0.528000in}}{\pgfqpoint{3.696000in}{3.696000in}}%
\pgfusepath{clip}%
\pgfsetbuttcap%
\pgfsetroundjoin%
\definecolor{currentfill}{rgb}{0.229806,0.298718,0.753683}%
\pgfsetfillcolor{currentfill}%
\pgfsetlinewidth{0.000000pt}%
\definecolor{currentstroke}{rgb}{0.000000,0.000000,0.000000}%
\pgfsetstrokecolor{currentstroke}%
\pgfsetdash{}{0pt}%
\pgfpathmoveto{\pgfqpoint{3.164565in}{1.166223in}}%
\pgfpathlineto{\pgfqpoint{3.212134in}{1.170698in}}%
\pgfpathlineto{\pgfqpoint{3.238680in}{1.175813in}}%
\pgfpathlineto{\pgfqpoint{3.191336in}{1.193194in}}%
\pgfpathlineto{\pgfqpoint{3.164565in}{1.166223in}}%
\pgfpathclose%
\pgfusepath{fill}%
\end{pgfscope}%
\begin{pgfscope}%
\pgfpathrectangle{\pgfqpoint{1.072000in}{0.528000in}}{\pgfqpoint{3.696000in}{3.696000in}}%
\pgfusepath{clip}%
\pgfsetbuttcap%
\pgfsetroundjoin%
\definecolor{currentfill}{rgb}{0.505423,0.643995,0.983157}%
\pgfsetfillcolor{currentfill}%
\pgfsetlinewidth{0.000000pt}%
\definecolor{currentstroke}{rgb}{0.000000,0.000000,0.000000}%
\pgfsetstrokecolor{currentstroke}%
\pgfsetdash{}{0pt}%
\pgfpathmoveto{\pgfqpoint{2.908990in}{1.555989in}}%
\pgfpathlineto{\pgfqpoint{2.956096in}{1.483519in}}%
\pgfpathlineto{\pgfqpoint{2.983297in}{1.636792in}}%
\pgfpathlineto{\pgfqpoint{2.936109in}{1.723607in}}%
\pgfpathlineto{\pgfqpoint{2.908990in}{1.555989in}}%
\pgfpathclose%
\pgfusepath{fill}%
\end{pgfscope}%
\begin{pgfscope}%
\pgfpathrectangle{\pgfqpoint{1.072000in}{0.528000in}}{\pgfqpoint{3.696000in}{3.696000in}}%
\pgfusepath{clip}%
\pgfsetbuttcap%
\pgfsetroundjoin%
\definecolor{currentfill}{rgb}{0.830187,0.304733,0.254891}%
\pgfsetfillcolor{currentfill}%
\pgfsetlinewidth{0.000000pt}%
\definecolor{currentstroke}{rgb}{0.000000,0.000000,0.000000}%
\pgfsetstrokecolor{currentstroke}%
\pgfsetdash{}{0pt}%
\pgfpathmoveto{\pgfqpoint{2.803620in}{3.040390in}}%
\pgfpathlineto{\pgfqpoint{2.850829in}{2.995651in}}%
\pgfpathlineto{\pgfqpoint{2.878591in}{2.906156in}}%
\pgfpathlineto{\pgfqpoint{2.831525in}{2.974938in}}%
\pgfpathlineto{\pgfqpoint{2.803620in}{3.040390in}}%
\pgfpathclose%
\pgfusepath{fill}%
\end{pgfscope}%
\begin{pgfscope}%
\pgfpathrectangle{\pgfqpoint{1.072000in}{0.528000in}}{\pgfqpoint{3.696000in}{3.696000in}}%
\pgfusepath{clip}%
\pgfsetbuttcap%
\pgfsetroundjoin%
\definecolor{currentfill}{rgb}{0.248091,0.326013,0.777669}%
\pgfsetfillcolor{currentfill}%
\pgfsetlinewidth{0.000000pt}%
\definecolor{currentstroke}{rgb}{0.000000,0.000000,0.000000}%
\pgfsetstrokecolor{currentstroke}%
\pgfsetdash{}{0pt}%
\pgfpathmoveto{\pgfqpoint{3.360439in}{1.192792in}}%
\pgfpathlineto{\pgfqpoint{3.408717in}{1.224739in}}%
\pgfpathlineto{\pgfqpoint{3.434500in}{1.211046in}}%
\pgfpathlineto{\pgfqpoint{3.386561in}{1.199287in}}%
\pgfpathlineto{\pgfqpoint{3.360439in}{1.192792in}}%
\pgfpathclose%
\pgfusepath{fill}%
\end{pgfscope}%
\begin{pgfscope}%
\pgfpathrectangle{\pgfqpoint{1.072000in}{0.528000in}}{\pgfqpoint{3.696000in}{3.696000in}}%
\pgfusepath{clip}%
\pgfsetbuttcap%
\pgfsetroundjoin%
\definecolor{currentfill}{rgb}{0.243520,0.319189,0.771672}%
\pgfsetfillcolor{currentfill}%
\pgfsetlinewidth{0.000000pt}%
\definecolor{currentstroke}{rgb}{0.000000,0.000000,0.000000}%
\pgfsetstrokecolor{currentstroke}%
\pgfsetdash{}{0pt}%
\pgfpathmoveto{\pgfqpoint{2.603922in}{1.210740in}}%
\pgfpathlineto{\pgfqpoint{2.650617in}{1.206832in}}%
\pgfpathlineto{\pgfqpoint{2.678896in}{1.186708in}}%
\pgfpathlineto{\pgfqpoint{2.632246in}{1.197857in}}%
\pgfpathlineto{\pgfqpoint{2.603922in}{1.210740in}}%
\pgfpathclose%
\pgfusepath{fill}%
\end{pgfscope}%
\begin{pgfscope}%
\pgfpathrectangle{\pgfqpoint{1.072000in}{0.528000in}}{\pgfqpoint{3.696000in}{3.696000in}}%
\pgfusepath{clip}%
\pgfsetbuttcap%
\pgfsetroundjoin%
\definecolor{currentfill}{rgb}{0.728970,0.817464,0.973188}%
\pgfsetfillcolor{currentfill}%
\pgfsetlinewidth{0.000000pt}%
\definecolor{currentstroke}{rgb}{0.000000,0.000000,0.000000}%
\pgfsetstrokecolor{currentstroke}%
\pgfsetdash{}{0pt}%
\pgfpathmoveto{\pgfqpoint{2.626841in}{1.838835in}}%
\pgfpathlineto{\pgfqpoint{2.673584in}{1.818405in}}%
\pgfpathlineto{\pgfqpoint{2.699907in}{2.021690in}}%
\pgfpathlineto{\pgfqpoint{2.652980in}{2.042876in}}%
\pgfpathlineto{\pgfqpoint{2.626841in}{1.838835in}}%
\pgfpathclose%
\pgfusepath{fill}%
\end{pgfscope}%
\begin{pgfscope}%
\pgfpathrectangle{\pgfqpoint{1.072000in}{0.528000in}}{\pgfqpoint{3.696000in}{3.696000in}}%
\pgfusepath{clip}%
\pgfsetbuttcap%
\pgfsetroundjoin%
\definecolor{currentfill}{rgb}{0.229806,0.298718,0.753683}%
\pgfsetfillcolor{currentfill}%
\pgfsetlinewidth{0.000000pt}%
\definecolor{currentstroke}{rgb}{0.000000,0.000000,0.000000}%
\pgfsetstrokecolor{currentstroke}%
\pgfsetdash{}{0pt}%
\pgfpathmoveto{\pgfqpoint{2.847172in}{1.165191in}}%
\pgfpathlineto{\pgfqpoint{2.894212in}{1.161438in}}%
\pgfpathlineto{\pgfqpoint{2.921600in}{1.171468in}}%
\pgfpathlineto{\pgfqpoint{2.874592in}{1.191316in}}%
\pgfpathlineto{\pgfqpoint{2.847172in}{1.165191in}}%
\pgfpathclose%
\pgfusepath{fill}%
\end{pgfscope}%
\begin{pgfscope}%
\pgfpathrectangle{\pgfqpoint{1.072000in}{0.528000in}}{\pgfqpoint{3.696000in}{3.696000in}}%
\pgfusepath{clip}%
\pgfsetbuttcap%
\pgfsetroundjoin%
\definecolor{currentfill}{rgb}{0.711554,0.033337,0.154485}%
\pgfsetfillcolor{currentfill}%
\pgfsetlinewidth{0.000000pt}%
\definecolor{currentstroke}{rgb}{0.000000,0.000000,0.000000}%
\pgfsetstrokecolor{currentstroke}%
\pgfsetdash{}{0pt}%
\pgfpathmoveto{\pgfqpoint{2.662314in}{3.155786in}}%
\pgfpathlineto{\pgfqpoint{2.709251in}{3.133967in}}%
\pgfpathlineto{\pgfqpoint{2.737187in}{3.126384in}}%
\pgfpathlineto{\pgfqpoint{2.690161in}{3.173291in}}%
\pgfpathlineto{\pgfqpoint{2.662314in}{3.155786in}}%
\pgfpathclose%
\pgfusepath{fill}%
\end{pgfscope}%
\begin{pgfscope}%
\pgfpathrectangle{\pgfqpoint{1.072000in}{0.528000in}}{\pgfqpoint{3.696000in}{3.696000in}}%
\pgfusepath{clip}%
\pgfsetbuttcap%
\pgfsetroundjoin%
\definecolor{currentfill}{rgb}{0.939254,0.539581,0.423900}%
\pgfsetfillcolor{currentfill}%
\pgfsetlinewidth{0.000000pt}%
\definecolor{currentstroke}{rgb}{0.000000,0.000000,0.000000}%
\pgfsetstrokecolor{currentstroke}%
\pgfsetdash{}{0pt}%
\pgfpathmoveto{\pgfqpoint{2.480641in}{2.618006in}}%
\pgfpathlineto{\pgfqpoint{2.526038in}{2.690821in}}%
\pgfpathlineto{\pgfqpoint{2.552614in}{2.841195in}}%
\pgfpathlineto{\pgfqpoint{2.506977in}{2.773065in}}%
\pgfpathlineto{\pgfqpoint{2.480641in}{2.618006in}}%
\pgfpathclose%
\pgfusepath{fill}%
\end{pgfscope}%
\begin{pgfscope}%
\pgfpathrectangle{\pgfqpoint{1.072000in}{0.528000in}}{\pgfqpoint{3.696000in}{3.696000in}}%
\pgfusepath{clip}%
\pgfsetbuttcap%
\pgfsetroundjoin%
\definecolor{currentfill}{rgb}{0.785153,0.220851,0.211673}%
\pgfsetfillcolor{currentfill}%
\pgfsetlinewidth{0.000000pt}%
\definecolor{currentstroke}{rgb}{0.000000,0.000000,0.000000}%
\pgfsetstrokecolor{currentstroke}%
\pgfsetdash{}{0pt}%
\pgfpathmoveto{\pgfqpoint{2.756400in}{3.091391in}}%
\pgfpathlineto{\pgfqpoint{2.803620in}{3.040390in}}%
\pgfpathlineto{\pgfqpoint{2.831525in}{2.974938in}}%
\pgfpathlineto{\pgfqpoint{2.784360in}{3.055264in}}%
\pgfpathlineto{\pgfqpoint{2.756400in}{3.091391in}}%
\pgfpathclose%
\pgfusepath{fill}%
\end{pgfscope}%
\begin{pgfscope}%
\pgfpathrectangle{\pgfqpoint{1.072000in}{0.528000in}}{\pgfqpoint{3.696000in}{3.696000in}}%
\pgfusepath{clip}%
\pgfsetbuttcap%
\pgfsetroundjoin%
\definecolor{currentfill}{rgb}{0.852378,0.346492,0.280346}%
\pgfsetfillcolor{currentfill}%
\pgfsetlinewidth{0.000000pt}%
\definecolor{currentstroke}{rgb}{0.000000,0.000000,0.000000}%
\pgfsetstrokecolor{currentstroke}%
\pgfsetdash{}{0pt}%
\pgfpathmoveto{\pgfqpoint{2.850829in}{2.995651in}}%
\pgfpathlineto{\pgfqpoint{2.898024in}{2.970247in}}%
\pgfpathlineto{\pgfqpoint{2.925575in}{2.871029in}}%
\pgfpathlineto{\pgfqpoint{2.878591in}{2.906156in}}%
\pgfpathlineto{\pgfqpoint{2.850829in}{2.995651in}}%
\pgfpathclose%
\pgfusepath{fill}%
\end{pgfscope}%
\begin{pgfscope}%
\pgfpathrectangle{\pgfqpoint{1.072000in}{0.528000in}}{\pgfqpoint{3.696000in}{3.696000in}}%
\pgfusepath{clip}%
\pgfsetbuttcap%
\pgfsetroundjoin%
\definecolor{currentfill}{rgb}{0.777378,0.840921,0.946149}%
\pgfsetfillcolor{currentfill}%
\pgfsetlinewidth{0.000000pt}%
\definecolor{currentstroke}{rgb}{0.000000,0.000000,0.000000}%
\pgfsetstrokecolor{currentstroke}%
\pgfsetdash{}{0pt}%
\pgfpathmoveto{\pgfqpoint{3.335720in}{2.027224in}}%
\pgfpathlineto{\pgfqpoint{3.381561in}{1.859254in}}%
\pgfpathlineto{\pgfqpoint{3.408924in}{1.968451in}}%
\pgfpathlineto{\pgfqpoint{3.363258in}{2.146017in}}%
\pgfpathlineto{\pgfqpoint{3.335720in}{2.027224in}}%
\pgfpathclose%
\pgfusepath{fill}%
\end{pgfscope}%
\begin{pgfscope}%
\pgfpathrectangle{\pgfqpoint{1.072000in}{0.528000in}}{\pgfqpoint{3.696000in}{3.696000in}}%
\pgfusepath{clip}%
\pgfsetbuttcap%
\pgfsetroundjoin%
\definecolor{currentfill}{rgb}{0.630089,0.752516,0.998508}%
\pgfsetfillcolor{currentfill}%
\pgfsetlinewidth{0.000000pt}%
\definecolor{currentstroke}{rgb}{0.000000,0.000000,0.000000}%
\pgfsetstrokecolor{currentstroke}%
\pgfsetdash{}{0pt}%
\pgfpathmoveto{\pgfqpoint{3.206535in}{1.770462in}}%
\pgfpathlineto{\pgfqpoint{3.253061in}{1.638469in}}%
\pgfpathlineto{\pgfqpoint{3.280568in}{1.769041in}}%
\pgfpathlineto{\pgfqpoint{3.234171in}{1.916383in}}%
\pgfpathlineto{\pgfqpoint{3.206535in}{1.770462in}}%
\pgfpathclose%
\pgfusepath{fill}%
\end{pgfscope}%
\begin{pgfscope}%
\pgfpathrectangle{\pgfqpoint{1.072000in}{0.528000in}}{\pgfqpoint{3.696000in}{3.696000in}}%
\pgfusepath{clip}%
\pgfsetbuttcap%
\pgfsetroundjoin%
\definecolor{currentfill}{rgb}{0.740957,0.122240,0.175744}%
\pgfsetfillcolor{currentfill}%
\pgfsetlinewidth{0.000000pt}%
\definecolor{currentstroke}{rgb}{0.000000,0.000000,0.000000}%
\pgfsetstrokecolor{currentstroke}%
\pgfsetdash{}{0pt}%
\pgfpathmoveto{\pgfqpoint{2.709251in}{3.133967in}}%
\pgfpathlineto{\pgfqpoint{2.756400in}{3.091391in}}%
\pgfpathlineto{\pgfqpoint{2.784360in}{3.055264in}}%
\pgfpathlineto{\pgfqpoint{2.737187in}{3.126384in}}%
\pgfpathlineto{\pgfqpoint{2.709251in}{3.133967in}}%
\pgfpathclose%
\pgfusepath{fill}%
\end{pgfscope}%
\begin{pgfscope}%
\pgfpathrectangle{\pgfqpoint{1.072000in}{0.528000in}}{\pgfqpoint{3.696000in}{3.696000in}}%
\pgfusepath{clip}%
\pgfsetbuttcap%
\pgfsetroundjoin%
\definecolor{currentfill}{rgb}{0.229806,0.298718,0.753683}%
\pgfsetfillcolor{currentfill}%
\pgfsetlinewidth{0.000000pt}%
\definecolor{currentstroke}{rgb}{0.000000,0.000000,0.000000}%
\pgfsetstrokecolor{currentstroke}%
\pgfsetdash{}{0pt}%
\pgfpathmoveto{\pgfqpoint{2.968706in}{1.159651in}}%
\pgfpathlineto{\pgfqpoint{3.015948in}{1.158365in}}%
\pgfpathlineto{\pgfqpoint{3.043036in}{1.167349in}}%
\pgfpathlineto{\pgfqpoint{2.995887in}{1.187872in}}%
\pgfpathlineto{\pgfqpoint{2.968706in}{1.159651in}}%
\pgfpathclose%
\pgfusepath{fill}%
\end{pgfscope}%
\begin{pgfscope}%
\pgfpathrectangle{\pgfqpoint{1.072000in}{0.528000in}}{\pgfqpoint{3.696000in}{3.696000in}}%
\pgfusepath{clip}%
\pgfsetbuttcap%
\pgfsetroundjoin%
\definecolor{currentfill}{rgb}{0.763520,0.178667,0.193396}%
\pgfsetfillcolor{currentfill}%
\pgfsetlinewidth{0.000000pt}%
\definecolor{currentstroke}{rgb}{0.000000,0.000000,0.000000}%
\pgfsetstrokecolor{currentstroke}%
\pgfsetdash{}{0pt}%
\pgfpathmoveto{\pgfqpoint{3.392395in}{3.111859in}}%
\pgfpathlineto{\pgfqpoint{3.439296in}{3.032240in}}%
\pgfpathlineto{\pgfqpoint{3.465132in}{3.025932in}}%
\pgfpathlineto{\pgfqpoint{3.418513in}{3.115583in}}%
\pgfpathlineto{\pgfqpoint{3.392395in}{3.111859in}}%
\pgfpathclose%
\pgfusepath{fill}%
\end{pgfscope}%
\begin{pgfscope}%
\pgfpathrectangle{\pgfqpoint{1.072000in}{0.528000in}}{\pgfqpoint{3.696000in}{3.696000in}}%
\pgfusepath{clip}%
\pgfsetbuttcap%
\pgfsetroundjoin%
\definecolor{currentfill}{rgb}{0.928116,0.822197,0.765141}%
\pgfsetfillcolor{currentfill}%
\pgfsetlinewidth{0.000000pt}%
\definecolor{currentstroke}{rgb}{0.000000,0.000000,0.000000}%
\pgfsetstrokecolor{currentstroke}%
\pgfsetdash{}{0pt}%
\pgfpathmoveto{\pgfqpoint{3.417973in}{2.347251in}}%
\pgfpathlineto{\pgfqpoint{3.463304in}{2.158163in}}%
\pgfpathlineto{\pgfqpoint{3.490212in}{2.234241in}}%
\pgfpathlineto{\pgfqpoint{3.445045in}{2.426067in}}%
\pgfpathlineto{\pgfqpoint{3.417973in}{2.347251in}}%
\pgfpathclose%
\pgfusepath{fill}%
\end{pgfscope}%
\begin{pgfscope}%
\pgfpathrectangle{\pgfqpoint{1.072000in}{0.528000in}}{\pgfqpoint{3.696000in}{3.696000in}}%
\pgfusepath{clip}%
\pgfsetbuttcap%
\pgfsetroundjoin%
\definecolor{currentfill}{rgb}{0.859385,0.864431,0.872111}%
\pgfsetfillcolor{currentfill}%
\pgfsetlinewidth{0.000000pt}%
\definecolor{currentstroke}{rgb}{0.000000,0.000000,0.000000}%
\pgfsetstrokecolor{currentstroke}%
\pgfsetdash{}{0pt}%
\pgfpathmoveto{\pgfqpoint{2.559908in}{2.042837in}}%
\pgfpathlineto{\pgfqpoint{2.606296in}{2.049851in}}%
\pgfpathlineto{\pgfqpoint{2.632372in}{2.259419in}}%
\pgfpathlineto{\pgfqpoint{2.585802in}{2.250217in}}%
\pgfpathlineto{\pgfqpoint{2.559908in}{2.042837in}}%
\pgfpathclose%
\pgfusepath{fill}%
\end{pgfscope}%
\begin{pgfscope}%
\pgfpathrectangle{\pgfqpoint{1.072000in}{0.528000in}}{\pgfqpoint{3.696000in}{3.696000in}}%
\pgfusepath{clip}%
\pgfsetbuttcap%
\pgfsetroundjoin%
\definecolor{currentfill}{rgb}{0.565182,0.699438,0.996635}%
\pgfsetfillcolor{currentfill}%
\pgfsetlinewidth{0.000000pt}%
\definecolor{currentstroke}{rgb}{0.000000,0.000000,0.000000}%
\pgfsetstrokecolor{currentstroke}%
\pgfsetdash{}{0pt}%
\pgfpathmoveto{\pgfqpoint{2.861846in}{1.624981in}}%
\pgfpathlineto{\pgfqpoint{2.908990in}{1.555989in}}%
\pgfpathlineto{\pgfqpoint{2.936109in}{1.723607in}}%
\pgfpathlineto{\pgfqpoint{2.888846in}{1.804446in}}%
\pgfpathlineto{\pgfqpoint{2.861846in}{1.624981in}}%
\pgfpathclose%
\pgfusepath{fill}%
\end{pgfscope}%
\begin{pgfscope}%
\pgfpathrectangle{\pgfqpoint{1.072000in}{0.528000in}}{\pgfqpoint{3.696000in}{3.696000in}}%
\pgfusepath{clip}%
\pgfsetbuttcap%
\pgfsetroundjoin%
\definecolor{currentfill}{rgb}{0.711554,0.033337,0.154485}%
\pgfsetfillcolor{currentfill}%
\pgfsetlinewidth{0.000000pt}%
\definecolor{currentstroke}{rgb}{0.000000,0.000000,0.000000}%
\pgfsetstrokecolor{currentstroke}%
\pgfsetdash{}{0pt}%
\pgfpathmoveto{\pgfqpoint{3.344967in}{3.148332in}}%
\pgfpathlineto{\pgfqpoint{3.392395in}{3.111859in}}%
\pgfpathlineto{\pgfqpoint{3.418513in}{3.115583in}}%
\pgfpathlineto{\pgfqpoint{3.371354in}{3.166133in}}%
\pgfpathlineto{\pgfqpoint{3.344967in}{3.148332in}}%
\pgfpathclose%
\pgfusepath{fill}%
\end{pgfscope}%
\begin{pgfscope}%
\pgfpathrectangle{\pgfqpoint{1.072000in}{0.528000in}}{\pgfqpoint{3.696000in}{3.696000in}}%
\pgfusepath{clip}%
\pgfsetbuttcap%
\pgfsetroundjoin%
\definecolor{currentfill}{rgb}{0.717435,0.051118,0.158737}%
\pgfsetfillcolor{currentfill}%
\pgfsetlinewidth{0.000000pt}%
\definecolor{currentstroke}{rgb}{0.000000,0.000000,0.000000}%
\pgfsetstrokecolor{currentstroke}%
\pgfsetdash{}{0pt}%
\pgfpathmoveto{\pgfqpoint{2.588162in}{3.091597in}}%
\pgfpathlineto{\pgfqpoint{2.634528in}{3.119139in}}%
\pgfpathlineto{\pgfqpoint{2.662314in}{3.155786in}}%
\pgfpathlineto{\pgfqpoint{2.615758in}{3.148348in}}%
\pgfpathlineto{\pgfqpoint{2.588162in}{3.091597in}}%
\pgfpathclose%
\pgfusepath{fill}%
\end{pgfscope}%
\begin{pgfscope}%
\pgfpathrectangle{\pgfqpoint{1.072000in}{0.528000in}}{\pgfqpoint{3.696000in}{3.696000in}}%
\pgfusepath{clip}%
\pgfsetbuttcap%
\pgfsetroundjoin%
\definecolor{currentfill}{rgb}{0.839351,0.861167,0.894494}%
\pgfsetfillcolor{currentfill}%
\pgfsetlinewidth{0.000000pt}%
\definecolor{currentstroke}{rgb}{0.000000,0.000000,0.000000}%
\pgfsetstrokecolor{currentstroke}%
\pgfsetdash{}{0pt}%
\pgfpathmoveto{\pgfqpoint{3.363258in}{2.146017in}}%
\pgfpathlineto{\pgfqpoint{3.408924in}{1.968451in}}%
\pgfpathlineto{\pgfqpoint{3.436190in}{2.068948in}}%
\pgfpathlineto{\pgfqpoint{3.390694in}{2.253438in}}%
\pgfpathlineto{\pgfqpoint{3.363258in}{2.146017in}}%
\pgfpathclose%
\pgfusepath{fill}%
\end{pgfscope}%
\begin{pgfscope}%
\pgfpathrectangle{\pgfqpoint{1.072000in}{0.528000in}}{\pgfqpoint{3.696000in}{3.696000in}}%
\pgfusepath{clip}%
\pgfsetbuttcap%
\pgfsetroundjoin%
\definecolor{currentfill}{rgb}{0.243520,0.319189,0.771672}%
\pgfsetfillcolor{currentfill}%
\pgfsetlinewidth{0.000000pt}%
\definecolor{currentstroke}{rgb}{0.000000,0.000000,0.000000}%
\pgfsetstrokecolor{currentstroke}%
\pgfsetdash{}{0pt}%
\pgfpathmoveto{\pgfqpoint{3.286330in}{1.179550in}}%
\pgfpathlineto{\pgfqpoint{3.334398in}{1.206344in}}%
\pgfpathlineto{\pgfqpoint{3.360439in}{1.192792in}}%
\pgfpathlineto{\pgfqpoint{3.312675in}{1.186836in}}%
\pgfpathlineto{\pgfqpoint{3.286330in}{1.179550in}}%
\pgfpathclose%
\pgfusepath{fill}%
\end{pgfscope}%
\begin{pgfscope}%
\pgfpathrectangle{\pgfqpoint{1.072000in}{0.528000in}}{\pgfqpoint{3.696000in}{3.696000in}}%
\pgfusepath{clip}%
\pgfsetbuttcap%
\pgfsetroundjoin%
\definecolor{currentfill}{rgb}{0.768929,0.189213,0.197965}%
\pgfsetfillcolor{currentfill}%
\pgfsetlinewidth{0.000000pt}%
\definecolor{currentstroke}{rgb}{0.000000,0.000000,0.000000}%
\pgfsetstrokecolor{currentstroke}%
\pgfsetdash{}{0pt}%
\pgfpathmoveto{\pgfqpoint{3.195641in}{3.008207in}}%
\pgfpathlineto{\pgfqpoint{3.243734in}{3.077602in}}%
\pgfpathlineto{\pgfqpoint{3.270474in}{3.105629in}}%
\pgfpathlineto{\pgfqpoint{3.222425in}{3.042911in}}%
\pgfpathlineto{\pgfqpoint{3.195641in}{3.008207in}}%
\pgfpathclose%
\pgfusepath{fill}%
\end{pgfscope}%
\begin{pgfscope}%
\pgfpathrectangle{\pgfqpoint{1.072000in}{0.528000in}}{\pgfqpoint{3.696000in}{3.696000in}}%
\pgfusepath{clip}%
\pgfsetbuttcap%
\pgfsetroundjoin%
\definecolor{currentfill}{rgb}{0.229806,0.298718,0.753683}%
\pgfsetfillcolor{currentfill}%
\pgfsetlinewidth{0.000000pt}%
\definecolor{currentstroke}{rgb}{0.000000,0.000000,0.000000}%
\pgfsetstrokecolor{currentstroke}%
\pgfsetdash{}{0pt}%
\pgfpathmoveto{\pgfqpoint{3.090326in}{1.159496in}}%
\pgfpathlineto{\pgfqpoint{3.137826in}{1.166874in}}%
\pgfpathlineto{\pgfqpoint{3.164565in}{1.166223in}}%
\pgfpathlineto{\pgfqpoint{3.117243in}{1.179771in}}%
\pgfpathlineto{\pgfqpoint{3.090326in}{1.159496in}}%
\pgfpathclose%
\pgfusepath{fill}%
\end{pgfscope}%
\begin{pgfscope}%
\pgfpathrectangle{\pgfqpoint{1.072000in}{0.528000in}}{\pgfqpoint{3.696000in}{3.696000in}}%
\pgfusepath{clip}%
\pgfsetbuttcap%
\pgfsetroundjoin%
\definecolor{currentfill}{rgb}{0.891817,0.851973,0.829085}%
\pgfsetfillcolor{currentfill}%
\pgfsetlinewidth{0.000000pt}%
\definecolor{currentstroke}{rgb}{0.000000,0.000000,0.000000}%
\pgfsetstrokecolor{currentstroke}%
\pgfsetdash{}{0pt}%
\pgfpathmoveto{\pgfqpoint{3.390694in}{2.253438in}}%
\pgfpathlineto{\pgfqpoint{3.436190in}{2.068948in}}%
\pgfpathlineto{\pgfqpoint{3.463304in}{2.158163in}}%
\pgfpathlineto{\pgfqpoint{3.417973in}{2.347251in}}%
\pgfpathlineto{\pgfqpoint{3.390694in}{2.253438in}}%
\pgfpathclose%
\pgfusepath{fill}%
\end{pgfscope}%
\begin{pgfscope}%
\pgfpathrectangle{\pgfqpoint{1.072000in}{0.528000in}}{\pgfqpoint{3.696000in}{3.696000in}}%
\pgfusepath{clip}%
\pgfsetbuttcap%
\pgfsetroundjoin%
\definecolor{currentfill}{rgb}{0.718985,0.811993,0.977656}%
\pgfsetfillcolor{currentfill}%
\pgfsetlinewidth{0.000000pt}%
\definecolor{currentstroke}{rgb}{0.000000,0.000000,0.000000}%
\pgfsetstrokecolor{currentstroke}%
\pgfsetdash{}{0pt}%
\pgfpathmoveto{\pgfqpoint{2.673584in}{1.818405in}}%
\pgfpathlineto{\pgfqpoint{2.720508in}{1.785868in}}%
\pgfpathlineto{\pgfqpoint{2.747016in}{1.986450in}}%
\pgfpathlineto{\pgfqpoint{2.699907in}{2.021690in}}%
\pgfpathlineto{\pgfqpoint{2.673584in}{1.818405in}}%
\pgfpathclose%
\pgfusepath{fill}%
\end{pgfscope}%
\begin{pgfscope}%
\pgfpathrectangle{\pgfqpoint{1.072000in}{0.528000in}}{\pgfqpoint{3.696000in}{3.696000in}}%
\pgfusepath{clip}%
\pgfsetbuttcap%
\pgfsetroundjoin%
\definecolor{currentfill}{rgb}{0.967317,0.657471,0.538160}%
\pgfsetfillcolor{currentfill}%
\pgfsetlinewidth{0.000000pt}%
\definecolor{currentstroke}{rgb}{0.000000,0.000000,0.000000}%
\pgfsetstrokecolor{currentstroke}%
\pgfsetdash{}{0pt}%
\pgfpathmoveto{\pgfqpoint{1.540710in}{2.688669in}}%
\pgfpathlineto{\pgfqpoint{1.591085in}{2.557926in}}%
\pgfpathlineto{\pgfqpoint{1.625907in}{2.458106in}}%
\pgfpathlineto{\pgfqpoint{1.575364in}{2.600994in}}%
\pgfpathlineto{\pgfqpoint{1.540710in}{2.688669in}}%
\pgfpathclose%
\pgfusepath{fill}%
\end{pgfscope}%
\begin{pgfscope}%
\pgfpathrectangle{\pgfqpoint{1.072000in}{0.528000in}}{\pgfqpoint{3.696000in}{3.696000in}}%
\pgfusepath{clip}%
\pgfsetbuttcap%
\pgfsetroundjoin%
\definecolor{currentfill}{rgb}{0.959385,0.610306,0.489382}%
\pgfsetfillcolor{currentfill}%
\pgfsetlinewidth{0.000000pt}%
\definecolor{currentstroke}{rgb}{0.000000,0.000000,0.000000}%
\pgfsetstrokecolor{currentstroke}%
\pgfsetdash{}{0pt}%
\pgfpathmoveto{\pgfqpoint{3.452970in}{2.714098in}}%
\pgfpathlineto{\pgfqpoint{3.498398in}{2.536341in}}%
\pgfpathlineto{\pgfqpoint{3.524605in}{2.567539in}}%
\pgfpathlineto{\pgfqpoint{3.479347in}{2.744727in}}%
\pgfpathlineto{\pgfqpoint{3.452970in}{2.714098in}}%
\pgfpathclose%
\pgfusepath{fill}%
\end{pgfscope}%
\begin{pgfscope}%
\pgfpathrectangle{\pgfqpoint{1.072000in}{0.528000in}}{\pgfqpoint{3.696000in}{3.696000in}}%
\pgfusepath{clip}%
\pgfsetbuttcap%
\pgfsetroundjoin%
\definecolor{currentfill}{rgb}{0.565182,0.699438,0.996635}%
\pgfsetfillcolor{currentfill}%
\pgfsetlinewidth{0.000000pt}%
\definecolor{currentstroke}{rgb}{0.000000,0.000000,0.000000}%
\pgfsetstrokecolor{currentstroke}%
\pgfsetdash{}{0pt}%
\pgfpathmoveto{\pgfqpoint{2.983297in}{1.636792in}}%
\pgfpathlineto{\pgfqpoint{3.030389in}{1.547372in}}%
\pgfpathlineto{\pgfqpoint{3.057743in}{1.702438in}}%
\pgfpathlineto{\pgfqpoint{3.010612in}{1.806864in}}%
\pgfpathlineto{\pgfqpoint{2.983297in}{1.636792in}}%
\pgfpathclose%
\pgfusepath{fill}%
\end{pgfscope}%
\begin{pgfscope}%
\pgfpathrectangle{\pgfqpoint{1.072000in}{0.528000in}}{\pgfqpoint{3.696000in}{3.696000in}}%
\pgfusepath{clip}%
\pgfsetbuttcap%
\pgfsetroundjoin%
\definecolor{currentfill}{rgb}{0.619318,0.744121,0.998931}%
\pgfsetfillcolor{currentfill}%
\pgfsetlinewidth{0.000000pt}%
\definecolor{currentstroke}{rgb}{0.000000,0.000000,0.000000}%
\pgfsetstrokecolor{currentstroke}%
\pgfsetdash{}{0pt}%
\pgfpathmoveto{\pgfqpoint{2.814689in}{1.687707in}}%
\pgfpathlineto{\pgfqpoint{2.861846in}{1.624981in}}%
\pgfpathlineto{\pgfqpoint{2.888846in}{1.804446in}}%
\pgfpathlineto{\pgfqpoint{2.841543in}{1.876535in}}%
\pgfpathlineto{\pgfqpoint{2.814689in}{1.687707in}}%
\pgfpathclose%
\pgfusepath{fill}%
\end{pgfscope}%
\begin{pgfscope}%
\pgfpathrectangle{\pgfqpoint{1.072000in}{0.528000in}}{\pgfqpoint{3.696000in}{3.696000in}}%
\pgfusepath{clip}%
\pgfsetbuttcap%
\pgfsetroundjoin%
\definecolor{currentfill}{rgb}{0.630089,0.752516,0.998508}%
\pgfsetfillcolor{currentfill}%
\pgfsetlinewidth{0.000000pt}%
\definecolor{currentstroke}{rgb}{0.000000,0.000000,0.000000}%
\pgfsetstrokecolor{currentstroke}%
\pgfsetdash{}{0pt}%
\pgfpathmoveto{\pgfqpoint{3.132207in}{1.747956in}}%
\pgfpathlineto{\pgfqpoint{3.179000in}{1.627834in}}%
\pgfpathlineto{\pgfqpoint{3.206535in}{1.770462in}}%
\pgfpathlineto{\pgfqpoint{3.159812in}{1.905949in}}%
\pgfpathlineto{\pgfqpoint{3.132207in}{1.747956in}}%
\pgfpathclose%
\pgfusepath{fill}%
\end{pgfscope}%
\begin{pgfscope}%
\pgfpathrectangle{\pgfqpoint{1.072000in}{0.528000in}}{\pgfqpoint{3.696000in}{3.696000in}}%
\pgfusepath{clip}%
\pgfsetbuttcap%
\pgfsetroundjoin%
\definecolor{currentfill}{rgb}{0.698454,0.799450,0.984577}%
\pgfsetfillcolor{currentfill}%
\pgfsetlinewidth{0.000000pt}%
\definecolor{currentstroke}{rgb}{0.000000,0.000000,0.000000}%
\pgfsetstrokecolor{currentstroke}%
\pgfsetdash{}{0pt}%
\pgfpathmoveto{\pgfqpoint{2.720508in}{1.785868in}}%
\pgfpathlineto{\pgfqpoint{2.767561in}{1.741908in}}%
\pgfpathlineto{\pgfqpoint{2.794248in}{1.937723in}}%
\pgfpathlineto{\pgfqpoint{2.747016in}{1.986450in}}%
\pgfpathlineto{\pgfqpoint{2.720508in}{1.785868in}}%
\pgfpathclose%
\pgfusepath{fill}%
\end{pgfscope}%
\begin{pgfscope}%
\pgfpathrectangle{\pgfqpoint{1.072000in}{0.528000in}}{\pgfqpoint{3.696000in}{3.696000in}}%
\pgfusepath{clip}%
\pgfsetbuttcap%
\pgfsetroundjoin%
\definecolor{currentfill}{rgb}{0.238948,0.312365,0.765676}%
\pgfsetfillcolor{currentfill}%
\pgfsetlinewidth{0.000000pt}%
\definecolor{currentstroke}{rgb}{0.000000,0.000000,0.000000}%
\pgfsetstrokecolor{currentstroke}%
\pgfsetdash{}{0pt}%
\pgfpathmoveto{\pgfqpoint{2.772496in}{1.175290in}}%
\pgfpathlineto{\pgfqpoint{2.819463in}{1.178031in}}%
\pgfpathlineto{\pgfqpoint{2.847172in}{1.165191in}}%
\pgfpathlineto{\pgfqpoint{2.800240in}{1.175770in}}%
\pgfpathlineto{\pgfqpoint{2.772496in}{1.175290in}}%
\pgfpathclose%
\pgfusepath{fill}%
\end{pgfscope}%
\begin{pgfscope}%
\pgfpathrectangle{\pgfqpoint{1.072000in}{0.528000in}}{\pgfqpoint{3.696000in}{3.696000in}}%
\pgfusepath{clip}%
\pgfsetbuttcap%
\pgfsetroundjoin%
\definecolor{currentfill}{rgb}{0.661968,0.775491,0.993937}%
\pgfsetfillcolor{currentfill}%
\pgfsetlinewidth{0.000000pt}%
\definecolor{currentstroke}{rgb}{0.000000,0.000000,0.000000}%
\pgfsetstrokecolor{currentstroke}%
\pgfsetdash{}{0pt}%
\pgfpathmoveto{\pgfqpoint{2.767561in}{1.741908in}}%
\pgfpathlineto{\pgfqpoint{2.814689in}{1.687707in}}%
\pgfpathlineto{\pgfqpoint{2.841543in}{1.876535in}}%
\pgfpathlineto{\pgfqpoint{2.794248in}{1.937723in}}%
\pgfpathlineto{\pgfqpoint{2.767561in}{1.741908in}}%
\pgfpathclose%
\pgfusepath{fill}%
\end{pgfscope}%
\begin{pgfscope}%
\pgfpathrectangle{\pgfqpoint{1.072000in}{0.528000in}}{\pgfqpoint{3.696000in}{3.696000in}}%
\pgfusepath{clip}%
\pgfsetbuttcap%
\pgfsetroundjoin%
\definecolor{currentfill}{rgb}{0.603162,0.731527,0.999565}%
\pgfsetfillcolor{currentfill}%
\pgfsetlinewidth{0.000000pt}%
\definecolor{currentstroke}{rgb}{0.000000,0.000000,0.000000}%
\pgfsetstrokecolor{currentstroke}%
\pgfsetdash{}{0pt}%
\pgfpathmoveto{\pgfqpoint{3.057743in}{1.702438in}}%
\pgfpathlineto{\pgfqpoint{3.104730in}{1.596811in}}%
\pgfpathlineto{\pgfqpoint{3.132207in}{1.747956in}}%
\pgfpathlineto{\pgfqpoint{3.085233in}{1.868871in}}%
\pgfpathlineto{\pgfqpoint{3.057743in}{1.702438in}}%
\pgfpathclose%
\pgfusepath{fill}%
\end{pgfscope}%
\begin{pgfscope}%
\pgfpathrectangle{\pgfqpoint{1.072000in}{0.528000in}}{\pgfqpoint{3.696000in}{3.696000in}}%
\pgfusepath{clip}%
\pgfsetbuttcap%
\pgfsetroundjoin%
\definecolor{currentfill}{rgb}{0.877149,0.394645,0.311724}%
\pgfsetfillcolor{currentfill}%
\pgfsetlinewidth{0.000000pt}%
\definecolor{currentstroke}{rgb}{0.000000,0.000000,0.000000}%
\pgfsetstrokecolor{currentstroke}%
\pgfsetdash{}{0pt}%
\pgfpathmoveto{\pgfqpoint{2.506977in}{2.773065in}}%
\pgfpathlineto{\pgfqpoint{2.552614in}{2.841195in}}%
\pgfpathlineto{\pgfqpoint{2.579587in}{2.963952in}}%
\pgfpathlineto{\pgfqpoint{2.533697in}{2.905132in}}%
\pgfpathlineto{\pgfqpoint{2.506977in}{2.773065in}}%
\pgfpathclose%
\pgfusepath{fill}%
\end{pgfscope}%
\begin{pgfscope}%
\pgfpathrectangle{\pgfqpoint{1.072000in}{0.528000in}}{\pgfqpoint{3.696000in}{3.696000in}}%
\pgfusepath{clip}%
\pgfsetbuttcap%
\pgfsetroundjoin%
\definecolor{currentfill}{rgb}{0.234377,0.305542,0.759680}%
\pgfsetfillcolor{currentfill}%
\pgfsetlinewidth{0.000000pt}%
\definecolor{currentstroke}{rgb}{0.000000,0.000000,0.000000}%
\pgfsetstrokecolor{currentstroke}%
\pgfsetdash{}{0pt}%
\pgfpathmoveto{\pgfqpoint{2.894212in}{1.161438in}}%
\pgfpathlineto{\pgfqpoint{2.941387in}{1.166598in}}%
\pgfpathlineto{\pgfqpoint{2.968706in}{1.159651in}}%
\pgfpathlineto{\pgfqpoint{2.921600in}{1.171468in}}%
\pgfpathlineto{\pgfqpoint{2.894212in}{1.161438in}}%
\pgfpathclose%
\pgfusepath{fill}%
\end{pgfscope}%
\begin{pgfscope}%
\pgfpathrectangle{\pgfqpoint{1.072000in}{0.528000in}}{\pgfqpoint{3.696000in}{3.696000in}}%
\pgfusepath{clip}%
\pgfsetbuttcap%
\pgfsetroundjoin%
\definecolor{currentfill}{rgb}{0.852378,0.346492,0.280346}%
\pgfsetfillcolor{currentfill}%
\pgfsetlinewidth{0.000000pt}%
\definecolor{currentstroke}{rgb}{0.000000,0.000000,0.000000}%
\pgfsetstrokecolor{currentstroke}%
\pgfsetdash{}{0pt}%
\pgfpathmoveto{\pgfqpoint{2.898024in}{2.970247in}}%
\pgfpathlineto{\pgfqpoint{2.945281in}{2.969654in}}%
\pgfpathlineto{\pgfqpoint{2.972614in}{2.880085in}}%
\pgfpathlineto{\pgfqpoint{2.925575in}{2.871029in}}%
\pgfpathlineto{\pgfqpoint{2.898024in}{2.970247in}}%
\pgfpathclose%
\pgfusepath{fill}%
\end{pgfscope}%
\begin{pgfscope}%
\pgfpathrectangle{\pgfqpoint{1.072000in}{0.528000in}}{\pgfqpoint{3.696000in}{3.696000in}}%
\pgfusepath{clip}%
\pgfsetbuttcap%
\pgfsetroundjoin%
\definecolor{currentfill}{rgb}{0.252663,0.332837,0.783665}%
\pgfsetfillcolor{currentfill}%
\pgfsetlinewidth{0.000000pt}%
\definecolor{currentstroke}{rgb}{0.000000,0.000000,0.000000}%
\pgfsetstrokecolor{currentstroke}%
\pgfsetdash{}{0pt}%
\pgfpathmoveto{\pgfqpoint{2.650617in}{1.206832in}}%
\pgfpathlineto{\pgfqpoint{2.697414in}{1.207777in}}%
\pgfpathlineto{\pgfqpoint{2.725645in}{1.178711in}}%
\pgfpathlineto{\pgfqpoint{2.678896in}{1.186708in}}%
\pgfpathlineto{\pgfqpoint{2.650617in}{1.206832in}}%
\pgfpathclose%
\pgfusepath{fill}%
\end{pgfscope}%
\begin{pgfscope}%
\pgfpathrectangle{\pgfqpoint{1.072000in}{0.528000in}}{\pgfqpoint{3.696000in}{3.696000in}}%
\pgfusepath{clip}%
\pgfsetbuttcap%
\pgfsetroundjoin%
\definecolor{currentfill}{rgb}{0.959385,0.610306,0.489382}%
\pgfsetfillcolor{currentfill}%
\pgfsetlinewidth{0.000000pt}%
\definecolor{currentstroke}{rgb}{0.000000,0.000000,0.000000}%
\pgfsetstrokecolor{currentstroke}%
\pgfsetdash{}{0pt}%
\pgfpathmoveto{\pgfqpoint{2.499856in}{2.518153in}}%
\pgfpathlineto{\pgfqpoint{2.545567in}{2.575881in}}%
\pgfpathlineto{\pgfqpoint{2.571971in}{2.746325in}}%
\pgfpathlineto{\pgfqpoint{2.526038in}{2.690821in}}%
\pgfpathlineto{\pgfqpoint{2.499856in}{2.518153in}}%
\pgfpathclose%
\pgfusepath{fill}%
\end{pgfscope}%
\begin{pgfscope}%
\pgfpathrectangle{\pgfqpoint{1.072000in}{0.528000in}}{\pgfqpoint{3.696000in}{3.696000in}}%
\pgfusepath{clip}%
\pgfsetbuttcap%
\pgfsetroundjoin%
\definecolor{currentfill}{rgb}{0.243520,0.319189,0.771672}%
\pgfsetfillcolor{currentfill}%
\pgfsetlinewidth{0.000000pt}%
\definecolor{currentstroke}{rgb}{0.000000,0.000000,0.000000}%
\pgfsetstrokecolor{currentstroke}%
\pgfsetdash{}{0pt}%
\pgfpathmoveto{\pgfqpoint{3.212134in}{1.170698in}}%
\pgfpathlineto{\pgfqpoint{3.260047in}{1.195283in}}%
\pgfpathlineto{\pgfqpoint{3.286330in}{1.179550in}}%
\pgfpathlineto{\pgfqpoint{3.238680in}{1.175813in}}%
\pgfpathlineto{\pgfqpoint{3.212134in}{1.170698in}}%
\pgfpathclose%
\pgfusepath{fill}%
\end{pgfscope}%
\begin{pgfscope}%
\pgfpathrectangle{\pgfqpoint{1.072000in}{0.528000in}}{\pgfqpoint{3.696000in}{3.696000in}}%
\pgfusepath{clip}%
\pgfsetbuttcap%
\pgfsetroundjoin%
\definecolor{currentfill}{rgb}{0.728970,0.817464,0.973188}%
\pgfsetfillcolor{currentfill}%
\pgfsetlinewidth{0.000000pt}%
\definecolor{currentstroke}{rgb}{0.000000,0.000000,0.000000}%
\pgfsetstrokecolor{currentstroke}%
\pgfsetdash{}{0pt}%
\pgfpathmoveto{\pgfqpoint{3.234171in}{1.916383in}}%
\pgfpathlineto{\pgfqpoint{3.280568in}{1.769041in}}%
\pgfpathlineto{\pgfqpoint{3.308138in}{1.900219in}}%
\pgfpathlineto{\pgfqpoint{3.261867in}{2.059939in}}%
\pgfpathlineto{\pgfqpoint{3.234171in}{1.916383in}}%
\pgfpathclose%
\pgfusepath{fill}%
\end{pgfscope}%
\begin{pgfscope}%
\pgfpathrectangle{\pgfqpoint{1.072000in}{0.528000in}}{\pgfqpoint{3.696000in}{3.696000in}}%
\pgfusepath{clip}%
\pgfsetbuttcap%
\pgfsetroundjoin%
\definecolor{currentfill}{rgb}{0.711554,0.033337,0.154485}%
\pgfsetfillcolor{currentfill}%
\pgfsetlinewidth{0.000000pt}%
\definecolor{currentstroke}{rgb}{0.000000,0.000000,0.000000}%
\pgfsetstrokecolor{currentstroke}%
\pgfsetdash{}{0pt}%
\pgfpathmoveto{\pgfqpoint{3.270474in}{3.105629in}}%
\pgfpathlineto{\pgfqpoint{3.318413in}{3.125236in}}%
\pgfpathlineto{\pgfqpoint{3.344967in}{3.148332in}}%
\pgfpathlineto{\pgfqpoint{3.297190in}{3.141894in}}%
\pgfpathlineto{\pgfqpoint{3.270474in}{3.105629in}}%
\pgfpathclose%
\pgfusepath{fill}%
\end{pgfscope}%
\begin{pgfscope}%
\pgfpathrectangle{\pgfqpoint{1.072000in}{0.528000in}}{\pgfqpoint{3.696000in}{3.696000in}}%
\pgfusepath{clip}%
\pgfsetbuttcap%
\pgfsetroundjoin%
\definecolor{currentfill}{rgb}{0.947345,0.794696,0.716991}%
\pgfsetfillcolor{currentfill}%
\pgfsetlinewidth{0.000000pt}%
\definecolor{currentstroke}{rgb}{0.000000,0.000000,0.000000}%
\pgfsetstrokecolor{currentstroke}%
\pgfsetdash{}{0pt}%
\pgfpathmoveto{\pgfqpoint{2.539590in}{2.226110in}}%
\pgfpathlineto{\pgfqpoint{2.585802in}{2.250217in}}%
\pgfpathlineto{\pgfqpoint{2.611919in}{2.453996in}}%
\pgfpathlineto{\pgfqpoint{2.565517in}{2.428025in}}%
\pgfpathlineto{\pgfqpoint{2.539590in}{2.226110in}}%
\pgfpathclose%
\pgfusepath{fill}%
\end{pgfscope}%
\begin{pgfscope}%
\pgfpathrectangle{\pgfqpoint{1.072000in}{0.528000in}}{\pgfqpoint{3.696000in}{3.696000in}}%
\pgfusepath{clip}%
\pgfsetbuttcap%
\pgfsetroundjoin%
\definecolor{currentfill}{rgb}{0.752704,0.157576,0.184258}%
\pgfsetfillcolor{currentfill}%
\pgfsetlinewidth{0.000000pt}%
\definecolor{currentstroke}{rgb}{0.000000,0.000000,0.000000}%
\pgfsetstrokecolor{currentstroke}%
\pgfsetdash{}{0pt}%
\pgfpathmoveto{\pgfqpoint{2.560779in}{3.011369in}}%
\pgfpathlineto{\pgfqpoint{2.606918in}{3.056447in}}%
\pgfpathlineto{\pgfqpoint{2.634528in}{3.119139in}}%
\pgfpathlineto{\pgfqpoint{2.588162in}{3.091597in}}%
\pgfpathlineto{\pgfqpoint{2.560779in}{3.011369in}}%
\pgfpathclose%
\pgfusepath{fill}%
\end{pgfscope}%
\begin{pgfscope}%
\pgfpathrectangle{\pgfqpoint{1.072000in}{0.528000in}}{\pgfqpoint{3.696000in}{3.696000in}}%
\pgfusepath{clip}%
\pgfsetbuttcap%
\pgfsetroundjoin%
\definecolor{currentfill}{rgb}{0.852378,0.346492,0.280346}%
\pgfsetfillcolor{currentfill}%
\pgfsetlinewidth{0.000000pt}%
\definecolor{currentstroke}{rgb}{0.000000,0.000000,0.000000}%
\pgfsetstrokecolor{currentstroke}%
\pgfsetdash{}{0pt}%
\pgfpathmoveto{\pgfqpoint{3.019872in}{2.922975in}}%
\pgfpathlineto{\pgfqpoint{3.067425in}{2.977648in}}%
\pgfpathlineto{\pgfqpoint{3.094295in}{2.932522in}}%
\pgfpathlineto{\pgfqpoint{3.046770in}{2.843259in}}%
\pgfpathlineto{\pgfqpoint{3.019872in}{2.922975in}}%
\pgfpathclose%
\pgfusepath{fill}%
\end{pgfscope}%
\begin{pgfscope}%
\pgfpathrectangle{\pgfqpoint{1.072000in}{0.528000in}}{\pgfqpoint{3.696000in}{3.696000in}}%
\pgfusepath{clip}%
\pgfsetbuttcap%
\pgfsetroundjoin%
\definecolor{currentfill}{rgb}{0.830187,0.304733,0.254891}%
\pgfsetfillcolor{currentfill}%
\pgfsetlinewidth{0.000000pt}%
\definecolor{currentstroke}{rgb}{0.000000,0.000000,0.000000}%
\pgfsetstrokecolor{currentstroke}%
\pgfsetdash{}{0pt}%
\pgfpathmoveto{\pgfqpoint{3.094295in}{2.932522in}}%
\pgfpathlineto{\pgfqpoint{3.142138in}{3.007551in}}%
\pgfpathlineto{\pgfqpoint{3.168906in}{2.998194in}}%
\pgfpathlineto{\pgfqpoint{3.121001in}{2.899182in}}%
\pgfpathlineto{\pgfqpoint{3.094295in}{2.932522in}}%
\pgfpathclose%
\pgfusepath{fill}%
\end{pgfscope}%
\begin{pgfscope}%
\pgfpathrectangle{\pgfqpoint{1.072000in}{0.528000in}}{\pgfqpoint{3.696000in}{3.696000in}}%
\pgfusepath{clip}%
\pgfsetbuttcap%
\pgfsetroundjoin%
\definecolor{currentfill}{rgb}{0.820401,0.286765,0.245160}%
\pgfsetfillcolor{currentfill}%
\pgfsetlinewidth{0.000000pt}%
\definecolor{currentstroke}{rgb}{0.000000,0.000000,0.000000}%
\pgfsetstrokecolor{currentstroke}%
\pgfsetdash{}{0pt}%
\pgfpathmoveto{\pgfqpoint{3.413151in}{3.025725in}}%
\pgfpathlineto{\pgfqpoint{3.459628in}{2.910675in}}%
\pgfpathlineto{\pgfqpoint{3.485545in}{2.912803in}}%
\pgfpathlineto{\pgfqpoint{3.439296in}{3.032240in}}%
\pgfpathlineto{\pgfqpoint{3.413151in}{3.025725in}}%
\pgfpathclose%
\pgfusepath{fill}%
\end{pgfscope}%
\begin{pgfscope}%
\pgfpathrectangle{\pgfqpoint{1.072000in}{0.528000in}}{\pgfqpoint{3.696000in}{3.696000in}}%
\pgfusepath{clip}%
\pgfsetbuttcap%
\pgfsetroundjoin%
\definecolor{currentfill}{rgb}{0.960581,0.762501,0.667964}%
\pgfsetfillcolor{currentfill}%
\pgfsetlinewidth{0.000000pt}%
\definecolor{currentstroke}{rgb}{0.000000,0.000000,0.000000}%
\pgfsetstrokecolor{currentstroke}%
\pgfsetdash{}{0pt}%
\pgfpathmoveto{\pgfqpoint{1.606401in}{2.527340in}}%
\pgfpathlineto{\pgfqpoint{1.656665in}{2.397557in}}%
\pgfpathlineto{\pgfqpoint{1.691750in}{2.282711in}}%
\pgfpathlineto{\pgfqpoint{1.641498in}{2.421031in}}%
\pgfpathlineto{\pgfqpoint{1.606401in}{2.527340in}}%
\pgfpathclose%
\pgfusepath{fill}%
\end{pgfscope}%
\begin{pgfscope}%
\pgfpathrectangle{\pgfqpoint{1.072000in}{0.528000in}}{\pgfqpoint{3.696000in}{3.696000in}}%
\pgfusepath{clip}%
\pgfsetbuttcap%
\pgfsetroundjoin%
\definecolor{currentfill}{rgb}{0.238948,0.312365,0.765676}%
\pgfsetfillcolor{currentfill}%
\pgfsetlinewidth{0.000000pt}%
\definecolor{currentstroke}{rgb}{0.000000,0.000000,0.000000}%
\pgfsetstrokecolor{currentstroke}%
\pgfsetdash{}{0pt}%
\pgfpathmoveto{\pgfqpoint{3.015948in}{1.158365in}}%
\pgfpathlineto{\pgfqpoint{3.063376in}{1.169916in}}%
\pgfpathlineto{\pgfqpoint{3.090326in}{1.159496in}}%
\pgfpathlineto{\pgfqpoint{3.043036in}{1.167349in}}%
\pgfpathlineto{\pgfqpoint{3.015948in}{1.158365in}}%
\pgfpathclose%
\pgfusepath{fill}%
\end{pgfscope}%
\begin{pgfscope}%
\pgfpathrectangle{\pgfqpoint{1.072000in}{0.528000in}}{\pgfqpoint{3.696000in}{3.696000in}}%
\pgfusepath{clip}%
\pgfsetbuttcap%
\pgfsetroundjoin%
\definecolor{currentfill}{rgb}{0.810616,0.268797,0.235428}%
\pgfsetfillcolor{currentfill}%
\pgfsetlinewidth{0.000000pt}%
\definecolor{currentstroke}{rgb}{0.000000,0.000000,0.000000}%
\pgfsetstrokecolor{currentstroke}%
\pgfsetdash{}{0pt}%
\pgfpathmoveto{\pgfqpoint{2.533697in}{2.905132in}}%
\pgfpathlineto{\pgfqpoint{2.579587in}{2.963952in}}%
\pgfpathlineto{\pgfqpoint{2.606918in}{3.056447in}}%
\pgfpathlineto{\pgfqpoint{2.560779in}{3.011369in}}%
\pgfpathlineto{\pgfqpoint{2.533697in}{2.905132in}}%
\pgfpathclose%
\pgfusepath{fill}%
\end{pgfscope}%
\begin{pgfscope}%
\pgfpathrectangle{\pgfqpoint{1.072000in}{0.528000in}}{\pgfqpoint{3.696000in}{3.696000in}}%
\pgfusepath{clip}%
\pgfsetbuttcap%
\pgfsetroundjoin%
\definecolor{currentfill}{rgb}{0.895885,0.433075,0.338681}%
\pgfsetfillcolor{currentfill}%
\pgfsetlinewidth{0.000000pt}%
\definecolor{currentstroke}{rgb}{0.000000,0.000000,0.000000}%
\pgfsetstrokecolor{currentstroke}%
\pgfsetdash{}{0pt}%
\pgfpathmoveto{\pgfqpoint{3.433392in}{2.894705in}}%
\pgfpathlineto{\pgfqpoint{3.479347in}{2.744727in}}%
\pgfpathlineto{\pgfqpoint{3.505394in}{2.760031in}}%
\pgfpathlineto{\pgfqpoint{3.459628in}{2.910675in}}%
\pgfpathlineto{\pgfqpoint{3.433392in}{2.894705in}}%
\pgfpathclose%
\pgfusepath{fill}%
\end{pgfscope}%
\begin{pgfscope}%
\pgfpathrectangle{\pgfqpoint{1.072000in}{0.528000in}}{\pgfqpoint{3.696000in}{3.696000in}}%
\pgfusepath{clip}%
\pgfsetbuttcap%
\pgfsetroundjoin%
\definecolor{currentfill}{rgb}{0.969522,0.700833,0.587508}%
\pgfsetfillcolor{currentfill}%
\pgfsetlinewidth{0.000000pt}%
\definecolor{currentstroke}{rgb}{0.000000,0.000000,0.000000}%
\pgfsetstrokecolor{currentstroke}%
\pgfsetdash{}{0pt}%
\pgfpathmoveto{\pgfqpoint{2.519533in}{2.386619in}}%
\pgfpathlineto{\pgfqpoint{2.565517in}{2.428025in}}%
\pgfpathlineto{\pgfqpoint{2.591755in}{2.617521in}}%
\pgfpathlineto{\pgfqpoint{2.545567in}{2.575881in}}%
\pgfpathlineto{\pgfqpoint{2.519533in}{2.386619in}}%
\pgfpathclose%
\pgfusepath{fill}%
\end{pgfscope}%
\begin{pgfscope}%
\pgfpathrectangle{\pgfqpoint{1.072000in}{0.528000in}}{\pgfqpoint{3.696000in}{3.696000in}}%
\pgfusepath{clip}%
\pgfsetbuttcap%
\pgfsetroundjoin%
\definecolor{currentfill}{rgb}{0.640828,0.760752,0.997846}%
\pgfsetfillcolor{currentfill}%
\pgfsetlinewidth{0.000000pt}%
\definecolor{currentstroke}{rgb}{0.000000,0.000000,0.000000}%
\pgfsetstrokecolor{currentstroke}%
\pgfsetdash{}{0pt}%
\pgfpathmoveto{\pgfqpoint{2.936109in}{1.723607in}}%
\pgfpathlineto{\pgfqpoint{2.983297in}{1.636792in}}%
\pgfpathlineto{\pgfqpoint{3.010612in}{1.806864in}}%
\pgfpathlineto{\pgfqpoint{2.963343in}{1.906069in}}%
\pgfpathlineto{\pgfqpoint{2.936109in}{1.723607in}}%
\pgfpathclose%
\pgfusepath{fill}%
\end{pgfscope}%
\begin{pgfscope}%
\pgfpathrectangle{\pgfqpoint{1.072000in}{0.528000in}}{\pgfqpoint{3.696000in}{3.696000in}}%
\pgfusepath{clip}%
\pgfsetbuttcap%
\pgfsetroundjoin%
\definecolor{currentfill}{rgb}{0.871493,0.862309,0.857016}%
\pgfsetfillcolor{currentfill}%
\pgfsetlinewidth{0.000000pt}%
\definecolor{currentstroke}{rgb}{0.000000,0.000000,0.000000}%
\pgfsetstrokecolor{currentstroke}%
\pgfsetdash{}{0pt}%
\pgfpathmoveto{\pgfqpoint{2.606296in}{2.049851in}}%
\pgfpathlineto{\pgfqpoint{2.652980in}{2.042876in}}%
\pgfpathlineto{\pgfqpoint{2.679245in}{2.253286in}}%
\pgfpathlineto{\pgfqpoint{2.632372in}{2.259419in}}%
\pgfpathlineto{\pgfqpoint{2.606296in}{2.049851in}}%
\pgfpathclose%
\pgfusepath{fill}%
\end{pgfscope}%
\begin{pgfscope}%
\pgfpathrectangle{\pgfqpoint{1.072000in}{0.528000in}}{\pgfqpoint{3.696000in}{3.696000in}}%
\pgfusepath{clip}%
\pgfsetbuttcap%
\pgfsetroundjoin%
\definecolor{currentfill}{rgb}{0.705673,0.015556,0.150233}%
\pgfsetfillcolor{currentfill}%
\pgfsetlinewidth{0.000000pt}%
\definecolor{currentstroke}{rgb}{0.000000,0.000000,0.000000}%
\pgfsetstrokecolor{currentstroke}%
\pgfsetdash{}{0pt}%
\pgfpathmoveto{\pgfqpoint{2.634528in}{3.119139in}}%
\pgfpathlineto{\pgfqpoint{2.681323in}{3.120820in}}%
\pgfpathlineto{\pgfqpoint{2.709251in}{3.133967in}}%
\pgfpathlineto{\pgfqpoint{2.662314in}{3.155786in}}%
\pgfpathlineto{\pgfqpoint{2.634528in}{3.119139in}}%
\pgfpathclose%
\pgfusepath{fill}%
\end{pgfscope}%
\begin{pgfscope}%
\pgfpathrectangle{\pgfqpoint{1.072000in}{0.528000in}}{\pgfqpoint{3.696000in}{3.696000in}}%
\pgfusepath{clip}%
\pgfsetbuttcap%
\pgfsetroundjoin%
\definecolor{currentfill}{rgb}{0.309060,0.413498,0.850128}%
\pgfsetfillcolor{currentfill}%
\pgfsetlinewidth{0.000000pt}%
\definecolor{currentstroke}{rgb}{0.000000,0.000000,0.000000}%
\pgfsetstrokecolor{currentstroke}%
\pgfsetdash{}{0pt}%
\pgfpathmoveto{\pgfqpoint{3.483068in}{1.251509in}}%
\pgfpathlineto{\pgfqpoint{3.532387in}{1.320331in}}%
\pgfpathlineto{\pgfqpoint{3.557543in}{1.288614in}}%
\pgfpathlineto{\pgfqpoint{3.508575in}{1.236128in}}%
\pgfpathlineto{\pgfqpoint{3.483068in}{1.251509in}}%
\pgfpathclose%
\pgfusepath{fill}%
\end{pgfscope}%
\begin{pgfscope}%
\pgfpathrectangle{\pgfqpoint{1.072000in}{0.528000in}}{\pgfqpoint{3.696000in}{3.696000in}}%
\pgfusepath{clip}%
\pgfsetbuttcap%
\pgfsetroundjoin%
\definecolor{currentfill}{rgb}{0.962701,0.628218,0.507636}%
\pgfsetfillcolor{currentfill}%
\pgfsetlinewidth{0.000000pt}%
\definecolor{currentstroke}{rgb}{0.000000,0.000000,0.000000}%
\pgfsetstrokecolor{currentstroke}%
\pgfsetdash{}{0pt}%
\pgfpathmoveto{\pgfqpoint{3.426294in}{2.667919in}}%
\pgfpathlineto{\pgfqpoint{3.471866in}{2.489179in}}%
\pgfpathlineto{\pgfqpoint{3.498398in}{2.536341in}}%
\pgfpathlineto{\pgfqpoint{3.452970in}{2.714098in}}%
\pgfpathlineto{\pgfqpoint{3.426294in}{2.667919in}}%
\pgfpathclose%
\pgfusepath{fill}%
\end{pgfscope}%
\begin{pgfscope}%
\pgfpathrectangle{\pgfqpoint{1.072000in}{0.528000in}}{\pgfqpoint{3.696000in}{3.696000in}}%
\pgfusepath{clip}%
\pgfsetbuttcap%
\pgfsetroundjoin%
\definecolor{currentfill}{rgb}{0.333490,0.446265,0.874452}%
\pgfsetfillcolor{currentfill}%
\pgfsetlinewidth{0.000000pt}%
\definecolor{currentstroke}{rgb}{0.000000,0.000000,0.000000}%
\pgfsetstrokecolor{currentstroke}%
\pgfsetdash{}{0pt}%
\pgfpathmoveto{\pgfqpoint{3.557543in}{1.288614in}}%
\pgfpathlineto{\pgfqpoint{3.607383in}{1.370739in}}%
\pgfpathlineto{\pgfqpoint{3.632262in}{1.338669in}}%
\pgfpathlineto{\pgfqpoint{3.582762in}{1.270841in}}%
\pgfpathlineto{\pgfqpoint{3.557543in}{1.288614in}}%
\pgfpathclose%
\pgfusepath{fill}%
\end{pgfscope}%
\begin{pgfscope}%
\pgfpathrectangle{\pgfqpoint{1.072000in}{0.528000in}}{\pgfqpoint{3.696000in}{3.696000in}}%
\pgfusepath{clip}%
\pgfsetbuttcap%
\pgfsetroundjoin%
\definecolor{currentfill}{rgb}{0.248091,0.326013,0.777669}%
\pgfsetfillcolor{currentfill}%
\pgfsetlinewidth{0.000000pt}%
\definecolor{currentstroke}{rgb}{0.000000,0.000000,0.000000}%
\pgfsetstrokecolor{currentstroke}%
\pgfsetdash{}{0pt}%
\pgfpathmoveto{\pgfqpoint{3.137826in}{1.166874in}}%
\pgfpathlineto{\pgfqpoint{3.185615in}{1.191567in}}%
\pgfpathlineto{\pgfqpoint{3.212134in}{1.170698in}}%
\pgfpathlineto{\pgfqpoint{3.164565in}{1.166223in}}%
\pgfpathlineto{\pgfqpoint{3.137826in}{1.166874in}}%
\pgfpathclose%
\pgfusepath{fill}%
\end{pgfscope}%
\begin{pgfscope}%
\pgfpathrectangle{\pgfqpoint{1.072000in}{0.528000in}}{\pgfqpoint{3.696000in}{3.696000in}}%
\pgfusepath{clip}%
\pgfsetbuttcap%
\pgfsetroundjoin%
\definecolor{currentfill}{rgb}{0.289996,0.386836,0.828926}%
\pgfsetfillcolor{currentfill}%
\pgfsetlinewidth{0.000000pt}%
\definecolor{currentstroke}{rgb}{0.000000,0.000000,0.000000}%
\pgfsetstrokecolor{currentstroke}%
\pgfsetdash{}{0pt}%
\pgfpathmoveto{\pgfqpoint{3.408717in}{1.224739in}}%
\pgfpathlineto{\pgfqpoint{3.457628in}{1.283075in}}%
\pgfpathlineto{\pgfqpoint{3.483068in}{1.251509in}}%
\pgfpathlineto{\pgfqpoint{3.434500in}{1.211046in}}%
\pgfpathlineto{\pgfqpoint{3.408717in}{1.224739in}}%
\pgfpathclose%
\pgfusepath{fill}%
\end{pgfscope}%
\begin{pgfscope}%
\pgfpathrectangle{\pgfqpoint{1.072000in}{0.528000in}}{\pgfqpoint{3.696000in}{3.696000in}}%
\pgfusepath{clip}%
\pgfsetbuttcap%
\pgfsetroundjoin%
\definecolor{currentfill}{rgb}{0.818056,0.855590,0.914638}%
\pgfsetfillcolor{currentfill}%
\pgfsetlinewidth{0.000000pt}%
\definecolor{currentstroke}{rgb}{0.000000,0.000000,0.000000}%
\pgfsetstrokecolor{currentstroke}%
\pgfsetdash{}{0pt}%
\pgfpathmoveto{\pgfqpoint{3.261867in}{2.059939in}}%
\pgfpathlineto{\pgfqpoint{3.308138in}{1.900219in}}%
\pgfpathlineto{\pgfqpoint{3.335720in}{2.027224in}}%
\pgfpathlineto{\pgfqpoint{3.289570in}{2.196185in}}%
\pgfpathlineto{\pgfqpoint{3.261867in}{2.059939in}}%
\pgfpathclose%
\pgfusepath{fill}%
\end{pgfscope}%
\begin{pgfscope}%
\pgfpathrectangle{\pgfqpoint{1.072000in}{0.528000in}}{\pgfqpoint{3.696000in}{3.696000in}}%
\pgfusepath{clip}%
\pgfsetbuttcap%
\pgfsetroundjoin%
\definecolor{currentfill}{rgb}{0.768929,0.189213,0.197965}%
\pgfsetfillcolor{currentfill}%
\pgfsetlinewidth{0.000000pt}%
\definecolor{currentstroke}{rgb}{0.000000,0.000000,0.000000}%
\pgfsetstrokecolor{currentstroke}%
\pgfsetdash{}{0pt}%
\pgfpathmoveto{\pgfqpoint{3.168906in}{2.998194in}}%
\pgfpathlineto{\pgfqpoint{3.216973in}{3.061020in}}%
\pgfpathlineto{\pgfqpoint{3.243734in}{3.077602in}}%
\pgfpathlineto{\pgfqpoint{3.195641in}{3.008207in}}%
\pgfpathlineto{\pgfqpoint{3.168906in}{2.998194in}}%
\pgfpathclose%
\pgfusepath{fill}%
\end{pgfscope}%
\begin{pgfscope}%
\pgfpathrectangle{\pgfqpoint{1.072000in}{0.528000in}}{\pgfqpoint{3.696000in}{3.696000in}}%
\pgfusepath{clip}%
\pgfsetbuttcap%
\pgfsetroundjoin%
\definecolor{currentfill}{rgb}{0.743754,0.825125,0.965798}%
\pgfsetfillcolor{currentfill}%
\pgfsetlinewidth{0.000000pt}%
\definecolor{currentstroke}{rgb}{0.000000,0.000000,0.000000}%
\pgfsetstrokecolor{currentstroke}%
\pgfsetdash{}{0pt}%
\pgfpathmoveto{\pgfqpoint{3.159812in}{1.905949in}}%
\pgfpathlineto{\pgfqpoint{3.206535in}{1.770462in}}%
\pgfpathlineto{\pgfqpoint{3.234171in}{1.916383in}}%
\pgfpathlineto{\pgfqpoint{3.187517in}{2.064211in}}%
\pgfpathlineto{\pgfqpoint{3.159812in}{1.905949in}}%
\pgfpathclose%
\pgfusepath{fill}%
\end{pgfscope}%
\begin{pgfscope}%
\pgfpathrectangle{\pgfqpoint{1.072000in}{0.528000in}}{\pgfqpoint{3.696000in}{3.696000in}}%
\pgfusepath{clip}%
\pgfsetbuttcap%
\pgfsetroundjoin%
\definecolor{currentfill}{rgb}{0.248091,0.326013,0.777669}%
\pgfsetfillcolor{currentfill}%
\pgfsetlinewidth{0.000000pt}%
\definecolor{currentstroke}{rgb}{0.000000,0.000000,0.000000}%
\pgfsetstrokecolor{currentstroke}%
\pgfsetdash{}{0pt}%
\pgfpathmoveto{\pgfqpoint{2.819463in}{1.178031in}}%
\pgfpathlineto{\pgfqpoint{2.866568in}{1.188596in}}%
\pgfpathlineto{\pgfqpoint{2.894212in}{1.161438in}}%
\pgfpathlineto{\pgfqpoint{2.847172in}{1.165191in}}%
\pgfpathlineto{\pgfqpoint{2.819463in}{1.178031in}}%
\pgfpathclose%
\pgfusepath{fill}%
\end{pgfscope}%
\begin{pgfscope}%
\pgfpathrectangle{\pgfqpoint{1.072000in}{0.528000in}}{\pgfqpoint{3.696000in}{3.696000in}}%
\pgfusepath{clip}%
\pgfsetbuttcap%
\pgfsetroundjoin%
\definecolor{currentfill}{rgb}{0.705673,0.015556,0.150233}%
\pgfsetfillcolor{currentfill}%
\pgfsetlinewidth{0.000000pt}%
\definecolor{currentstroke}{rgb}{0.000000,0.000000,0.000000}%
\pgfsetstrokecolor{currentstroke}%
\pgfsetdash{}{0pt}%
\pgfpathmoveto{\pgfqpoint{3.318413in}{3.125236in}}%
\pgfpathlineto{\pgfqpoint{3.366034in}{3.098486in}}%
\pgfpathlineto{\pgfqpoint{3.392395in}{3.111859in}}%
\pgfpathlineto{\pgfqpoint{3.344967in}{3.148332in}}%
\pgfpathlineto{\pgfqpoint{3.318413in}{3.125236in}}%
\pgfpathclose%
\pgfusepath{fill}%
\end{pgfscope}%
\begin{pgfscope}%
\pgfpathrectangle{\pgfqpoint{1.072000in}{0.528000in}}{\pgfqpoint{3.696000in}{3.696000in}}%
\pgfusepath{clip}%
\pgfsetbuttcap%
\pgfsetroundjoin%
\definecolor{currentfill}{rgb}{0.746838,0.140021,0.179996}%
\pgfsetfillcolor{currentfill}%
\pgfsetlinewidth{0.000000pt}%
\definecolor{currentstroke}{rgb}{0.000000,0.000000,0.000000}%
\pgfsetstrokecolor{currentstroke}%
\pgfsetdash{}{0pt}%
\pgfpathmoveto{\pgfqpoint{3.366034in}{3.098486in}}%
\pgfpathlineto{\pgfqpoint{3.413151in}{3.025725in}}%
\pgfpathlineto{\pgfqpoint{3.439296in}{3.032240in}}%
\pgfpathlineto{\pgfqpoint{3.392395in}{3.111859in}}%
\pgfpathlineto{\pgfqpoint{3.366034in}{3.098486in}}%
\pgfpathclose%
\pgfusepath{fill}%
\end{pgfscope}%
\begin{pgfscope}%
\pgfpathrectangle{\pgfqpoint{1.072000in}{0.528000in}}{\pgfqpoint{3.696000in}{3.696000in}}%
\pgfusepath{clip}%
\pgfsetbuttcap%
\pgfsetroundjoin%
\definecolor{currentfill}{rgb}{0.698454,0.799450,0.984577}%
\pgfsetfillcolor{currentfill}%
\pgfsetlinewidth{0.000000pt}%
\definecolor{currentstroke}{rgb}{0.000000,0.000000,0.000000}%
\pgfsetstrokecolor{currentstroke}%
\pgfsetdash{}{0pt}%
\pgfpathmoveto{\pgfqpoint{3.010612in}{1.806864in}}%
\pgfpathlineto{\pgfqpoint{3.057743in}{1.702438in}}%
\pgfpathlineto{\pgfqpoint{3.085233in}{1.868871in}}%
\pgfpathlineto{\pgfqpoint{3.038066in}{1.985792in}}%
\pgfpathlineto{\pgfqpoint{3.010612in}{1.806864in}}%
\pgfpathclose%
\pgfusepath{fill}%
\end{pgfscope}%
\begin{pgfscope}%
\pgfpathrectangle{\pgfqpoint{1.072000in}{0.528000in}}{\pgfqpoint{3.696000in}{3.696000in}}%
\pgfusepath{clip}%
\pgfsetbuttcap%
\pgfsetroundjoin%
\definecolor{currentfill}{rgb}{0.261805,0.346484,0.795658}%
\pgfsetfillcolor{currentfill}%
\pgfsetlinewidth{0.000000pt}%
\definecolor{currentstroke}{rgb}{0.000000,0.000000,0.000000}%
\pgfsetstrokecolor{currentstroke}%
\pgfsetdash{}{0pt}%
\pgfpathmoveto{\pgfqpoint{2.697414in}{1.207777in}}%
\pgfpathlineto{\pgfqpoint{2.744326in}{1.214709in}}%
\pgfpathlineto{\pgfqpoint{2.772496in}{1.175290in}}%
\pgfpathlineto{\pgfqpoint{2.725645in}{1.178711in}}%
\pgfpathlineto{\pgfqpoint{2.697414in}{1.207777in}}%
\pgfpathclose%
\pgfusepath{fill}%
\end{pgfscope}%
\begin{pgfscope}%
\pgfpathrectangle{\pgfqpoint{1.072000in}{0.528000in}}{\pgfqpoint{3.696000in}{3.696000in}}%
\pgfusepath{clip}%
\pgfsetbuttcap%
\pgfsetroundjoin%
\definecolor{currentfill}{rgb}{0.708720,0.805721,0.981117}%
\pgfsetfillcolor{currentfill}%
\pgfsetlinewidth{0.000000pt}%
\definecolor{currentstroke}{rgb}{0.000000,0.000000,0.000000}%
\pgfsetstrokecolor{currentstroke}%
\pgfsetdash{}{0pt}%
\pgfpathmoveto{\pgfqpoint{2.888846in}{1.804446in}}%
\pgfpathlineto{\pgfqpoint{2.936109in}{1.723607in}}%
\pgfpathlineto{\pgfqpoint{2.963343in}{1.906069in}}%
\pgfpathlineto{\pgfqpoint{2.915964in}{1.996702in}}%
\pgfpathlineto{\pgfqpoint{2.888846in}{1.804446in}}%
\pgfpathclose%
\pgfusepath{fill}%
\end{pgfscope}%
\begin{pgfscope}%
\pgfpathrectangle{\pgfqpoint{1.072000in}{0.528000in}}{\pgfqpoint{3.696000in}{3.696000in}}%
\pgfusepath{clip}%
\pgfsetbuttcap%
\pgfsetroundjoin%
\definecolor{currentfill}{rgb}{0.248091,0.326013,0.777669}%
\pgfsetfillcolor{currentfill}%
\pgfsetlinewidth{0.000000pt}%
\definecolor{currentstroke}{rgb}{0.000000,0.000000,0.000000}%
\pgfsetstrokecolor{currentstroke}%
\pgfsetdash{}{0pt}%
\pgfpathmoveto{\pgfqpoint{2.941387in}{1.166598in}}%
\pgfpathlineto{\pgfqpoint{2.988739in}{1.182659in}}%
\pgfpathlineto{\pgfqpoint{3.015948in}{1.158365in}}%
\pgfpathlineto{\pgfqpoint{2.968706in}{1.159651in}}%
\pgfpathlineto{\pgfqpoint{2.941387in}{1.166598in}}%
\pgfpathclose%
\pgfusepath{fill}%
\end{pgfscope}%
\begin{pgfscope}%
\pgfpathrectangle{\pgfqpoint{1.072000in}{0.528000in}}{\pgfqpoint{3.696000in}{3.696000in}}%
\pgfusepath{clip}%
\pgfsetbuttcap%
\pgfsetroundjoin%
\definecolor{currentfill}{rgb}{0.280550,0.373423,0.818011}%
\pgfsetfillcolor{currentfill}%
\pgfsetlinewidth{0.000000pt}%
\definecolor{currentstroke}{rgb}{0.000000,0.000000,0.000000}%
\pgfsetstrokecolor{currentstroke}%
\pgfsetdash{}{0pt}%
\pgfpathmoveto{\pgfqpoint{3.334398in}{1.206344in}}%
\pgfpathlineto{\pgfqpoint{3.382996in}{1.257122in}}%
\pgfpathlineto{\pgfqpoint{3.408717in}{1.224739in}}%
\pgfpathlineto{\pgfqpoint{3.360439in}{1.192792in}}%
\pgfpathlineto{\pgfqpoint{3.334398in}{1.206344in}}%
\pgfpathclose%
\pgfusepath{fill}%
\end{pgfscope}%
\begin{pgfscope}%
\pgfpathrectangle{\pgfqpoint{1.072000in}{0.528000in}}{\pgfqpoint{3.696000in}{3.696000in}}%
\pgfusepath{clip}%
\pgfsetbuttcap%
\pgfsetroundjoin%
\definecolor{currentfill}{rgb}{0.871493,0.862309,0.857016}%
\pgfsetfillcolor{currentfill}%
\pgfsetlinewidth{0.000000pt}%
\definecolor{currentstroke}{rgb}{0.000000,0.000000,0.000000}%
\pgfsetstrokecolor{currentstroke}%
\pgfsetdash{}{0pt}%
\pgfpathmoveto{\pgfqpoint{2.652980in}{2.042876in}}%
\pgfpathlineto{\pgfqpoint{2.699907in}{2.021690in}}%
\pgfpathlineto{\pgfqpoint{2.726360in}{2.231661in}}%
\pgfpathlineto{\pgfqpoint{2.679245in}{2.253286in}}%
\pgfpathlineto{\pgfqpoint{2.652980in}{2.042876in}}%
\pgfpathclose%
\pgfusepath{fill}%
\end{pgfscope}%
\begin{pgfscope}%
\pgfpathrectangle{\pgfqpoint{1.072000in}{0.528000in}}{\pgfqpoint{3.696000in}{3.696000in}}%
\pgfusepath{clip}%
\pgfsetbuttcap%
\pgfsetroundjoin%
\definecolor{currentfill}{rgb}{0.711554,0.033337,0.154485}%
\pgfsetfillcolor{currentfill}%
\pgfsetlinewidth{0.000000pt}%
\definecolor{currentstroke}{rgb}{0.000000,0.000000,0.000000}%
\pgfsetstrokecolor{currentstroke}%
\pgfsetdash{}{0pt}%
\pgfpathmoveto{\pgfqpoint{2.681323in}{3.120820in}}%
\pgfpathlineto{\pgfqpoint{2.728398in}{3.103398in}}%
\pgfpathlineto{\pgfqpoint{2.756400in}{3.091391in}}%
\pgfpathlineto{\pgfqpoint{2.709251in}{3.133967in}}%
\pgfpathlineto{\pgfqpoint{2.681323in}{3.120820in}}%
\pgfpathclose%
\pgfusepath{fill}%
\end{pgfscope}%
\begin{pgfscope}%
\pgfpathrectangle{\pgfqpoint{1.072000in}{0.528000in}}{\pgfqpoint{3.696000in}{3.696000in}}%
\pgfusepath{clip}%
\pgfsetbuttcap%
\pgfsetroundjoin%
\definecolor{currentfill}{rgb}{0.728970,0.817464,0.973188}%
\pgfsetfillcolor{currentfill}%
\pgfsetlinewidth{0.000000pt}%
\definecolor{currentstroke}{rgb}{0.000000,0.000000,0.000000}%
\pgfsetstrokecolor{currentstroke}%
\pgfsetdash{}{0pt}%
\pgfpathmoveto{\pgfqpoint{3.085233in}{1.868871in}}%
\pgfpathlineto{\pgfqpoint{3.132207in}{1.747956in}}%
\pgfpathlineto{\pgfqpoint{3.159812in}{1.905949in}}%
\pgfpathlineto{\pgfqpoint{3.112853in}{2.039271in}}%
\pgfpathlineto{\pgfqpoint{3.085233in}{1.868871in}}%
\pgfpathclose%
\pgfusepath{fill}%
\end{pgfscope}%
\begin{pgfscope}%
\pgfpathrectangle{\pgfqpoint{1.072000in}{0.528000in}}{\pgfqpoint{3.696000in}{3.696000in}}%
\pgfusepath{clip}%
\pgfsetbuttcap%
\pgfsetroundjoin%
\definecolor{currentfill}{rgb}{0.905783,0.455186,0.355336}%
\pgfsetfillcolor{currentfill}%
\pgfsetlinewidth{0.000000pt}%
\definecolor{currentstroke}{rgb}{0.000000,0.000000,0.000000}%
\pgfsetstrokecolor{currentstroke}%
\pgfsetdash{}{0pt}%
\pgfpathmoveto{\pgfqpoint{2.526038in}{2.690821in}}%
\pgfpathlineto{\pgfqpoint{2.571971in}{2.746325in}}%
\pgfpathlineto{\pgfqpoint{2.598781in}{2.890102in}}%
\pgfpathlineto{\pgfqpoint{2.552614in}{2.841195in}}%
\pgfpathlineto{\pgfqpoint{2.526038in}{2.690821in}}%
\pgfpathclose%
\pgfusepath{fill}%
\end{pgfscope}%
\begin{pgfscope}%
\pgfpathrectangle{\pgfqpoint{1.072000in}{0.528000in}}{\pgfqpoint{3.696000in}{3.696000in}}%
\pgfusepath{clip}%
\pgfsetbuttcap%
\pgfsetroundjoin%
\definecolor{currentfill}{rgb}{0.830187,0.304733,0.254891}%
\pgfsetfillcolor{currentfill}%
\pgfsetlinewidth{0.000000pt}%
\definecolor{currentstroke}{rgb}{0.000000,0.000000,0.000000}%
\pgfsetstrokecolor{currentstroke}%
\pgfsetdash{}{0pt}%
\pgfpathmoveto{\pgfqpoint{2.945281in}{2.969654in}}%
\pgfpathlineto{\pgfqpoint{2.992695in}{2.988501in}}%
\pgfpathlineto{\pgfqpoint{3.019872in}{2.922975in}}%
\pgfpathlineto{\pgfqpoint{2.972614in}{2.880085in}}%
\pgfpathlineto{\pgfqpoint{2.945281in}{2.969654in}}%
\pgfpathclose%
\pgfusepath{fill}%
\end{pgfscope}%
\begin{pgfscope}%
\pgfpathrectangle{\pgfqpoint{1.072000in}{0.528000in}}{\pgfqpoint{3.696000in}{3.696000in}}%
\pgfusepath{clip}%
\pgfsetbuttcap%
\pgfsetroundjoin%
\definecolor{currentfill}{rgb}{0.717435,0.051118,0.158737}%
\pgfsetfillcolor{currentfill}%
\pgfsetlinewidth{0.000000pt}%
\definecolor{currentstroke}{rgb}{0.000000,0.000000,0.000000}%
\pgfsetstrokecolor{currentstroke}%
\pgfsetdash{}{0pt}%
\pgfpathmoveto{\pgfqpoint{3.243734in}{3.077602in}}%
\pgfpathlineto{\pgfqpoint{3.291751in}{3.102008in}}%
\pgfpathlineto{\pgfqpoint{3.318413in}{3.125236in}}%
\pgfpathlineto{\pgfqpoint{3.270474in}{3.105629in}}%
\pgfpathlineto{\pgfqpoint{3.243734in}{3.077602in}}%
\pgfpathclose%
\pgfusepath{fill}%
\end{pgfscope}%
\begin{pgfscope}%
\pgfpathrectangle{\pgfqpoint{1.072000in}{0.528000in}}{\pgfqpoint{3.696000in}{3.696000in}}%
\pgfusepath{clip}%
\pgfsetbuttcap%
\pgfsetroundjoin%
\definecolor{currentfill}{rgb}{0.967317,0.657471,0.538160}%
\pgfsetfillcolor{currentfill}%
\pgfsetlinewidth{0.000000pt}%
\definecolor{currentstroke}{rgb}{0.000000,0.000000,0.000000}%
\pgfsetstrokecolor{currentstroke}%
\pgfsetdash{}{0pt}%
\pgfpathmoveto{\pgfqpoint{3.399347in}{2.605735in}}%
\pgfpathlineto{\pgfqpoint{3.445045in}{2.426067in}}%
\pgfpathlineto{\pgfqpoint{3.471866in}{2.489179in}}%
\pgfpathlineto{\pgfqpoint{3.426294in}{2.667919in}}%
\pgfpathlineto{\pgfqpoint{3.399347in}{2.605735in}}%
\pgfpathclose%
\pgfusepath{fill}%
\end{pgfscope}%
\begin{pgfscope}%
\pgfpathrectangle{\pgfqpoint{1.072000in}{0.528000in}}{\pgfqpoint{3.696000in}{3.696000in}}%
\pgfusepath{clip}%
\pgfsetbuttcap%
\pgfsetroundjoin%
\definecolor{currentfill}{rgb}{0.891817,0.851973,0.829085}%
\pgfsetfillcolor{currentfill}%
\pgfsetlinewidth{0.000000pt}%
\definecolor{currentstroke}{rgb}{0.000000,0.000000,0.000000}%
\pgfsetstrokecolor{currentstroke}%
\pgfsetdash{}{0pt}%
\pgfpathmoveto{\pgfqpoint{3.289570in}{2.196185in}}%
\pgfpathlineto{\pgfqpoint{3.335720in}{2.027224in}}%
\pgfpathlineto{\pgfqpoint{3.363258in}{2.146017in}}%
\pgfpathlineto{\pgfqpoint{3.317225in}{2.321161in}}%
\pgfpathlineto{\pgfqpoint{3.289570in}{2.196185in}}%
\pgfpathclose%
\pgfusepath{fill}%
\end{pgfscope}%
\begin{pgfscope}%
\pgfpathrectangle{\pgfqpoint{1.072000in}{0.528000in}}{\pgfqpoint{3.696000in}{3.696000in}}%
\pgfusepath{clip}%
\pgfsetbuttcap%
\pgfsetroundjoin%
\definecolor{currentfill}{rgb}{0.257234,0.339661,0.789661}%
\pgfsetfillcolor{currentfill}%
\pgfsetlinewidth{0.000000pt}%
\definecolor{currentstroke}{rgb}{0.000000,0.000000,0.000000}%
\pgfsetstrokecolor{currentstroke}%
\pgfsetdash{}{0pt}%
\pgfpathmoveto{\pgfqpoint{3.063376in}{1.169916in}}%
\pgfpathlineto{\pgfqpoint{3.111055in}{1.196264in}}%
\pgfpathlineto{\pgfqpoint{3.137826in}{1.166874in}}%
\pgfpathlineto{\pgfqpoint{3.090326in}{1.159496in}}%
\pgfpathlineto{\pgfqpoint{3.063376in}{1.169916in}}%
\pgfpathclose%
\pgfusepath{fill}%
\end{pgfscope}%
\begin{pgfscope}%
\pgfpathrectangle{\pgfqpoint{1.072000in}{0.528000in}}{\pgfqpoint{3.696000in}{3.696000in}}%
\pgfusepath{clip}%
\pgfsetbuttcap%
\pgfsetroundjoin%
\definecolor{currentfill}{rgb}{0.956371,0.775144,0.686416}%
\pgfsetfillcolor{currentfill}%
\pgfsetlinewidth{0.000000pt}%
\definecolor{currentstroke}{rgb}{0.000000,0.000000,0.000000}%
\pgfsetstrokecolor{currentstroke}%
\pgfsetdash{}{0pt}%
\pgfpathmoveto{\pgfqpoint{2.585802in}{2.250217in}}%
\pgfpathlineto{\pgfqpoint{2.632372in}{2.259419in}}%
\pgfpathlineto{\pgfqpoint{2.658680in}{2.464185in}}%
\pgfpathlineto{\pgfqpoint{2.611919in}{2.453996in}}%
\pgfpathlineto{\pgfqpoint{2.585802in}{2.250217in}}%
\pgfpathclose%
\pgfusepath{fill}%
\end{pgfscope}%
\begin{pgfscope}%
\pgfpathrectangle{\pgfqpoint{1.072000in}{0.528000in}}{\pgfqpoint{3.696000in}{3.696000in}}%
\pgfusepath{clip}%
\pgfsetbuttcap%
\pgfsetroundjoin%
\definecolor{currentfill}{rgb}{0.729196,0.086679,0.167240}%
\pgfsetfillcolor{currentfill}%
\pgfsetlinewidth{0.000000pt}%
\definecolor{currentstroke}{rgb}{0.000000,0.000000,0.000000}%
\pgfsetstrokecolor{currentstroke}%
\pgfsetdash{}{0pt}%
\pgfpathmoveto{\pgfqpoint{2.728398in}{3.103398in}}%
\pgfpathlineto{\pgfqpoint{2.775627in}{3.076105in}}%
\pgfpathlineto{\pgfqpoint{2.803620in}{3.040390in}}%
\pgfpathlineto{\pgfqpoint{2.756400in}{3.091391in}}%
\pgfpathlineto{\pgfqpoint{2.728398in}{3.103398in}}%
\pgfpathclose%
\pgfusepath{fill}%
\end{pgfscope}%
\begin{pgfscope}%
\pgfpathrectangle{\pgfqpoint{1.072000in}{0.528000in}}{\pgfqpoint{3.696000in}{3.696000in}}%
\pgfusepath{clip}%
\pgfsetbuttcap%
\pgfsetroundjoin%
\definecolor{currentfill}{rgb}{0.763363,0.835092,0.955658}%
\pgfsetfillcolor{currentfill}%
\pgfsetlinewidth{0.000000pt}%
\definecolor{currentstroke}{rgb}{0.000000,0.000000,0.000000}%
\pgfsetstrokecolor{currentstroke}%
\pgfsetdash{}{0pt}%
\pgfpathmoveto{\pgfqpoint{2.841543in}{1.876535in}}%
\pgfpathlineto{\pgfqpoint{2.888846in}{1.804446in}}%
\pgfpathlineto{\pgfqpoint{2.915964in}{1.996702in}}%
\pgfpathlineto{\pgfqpoint{2.868518in}{2.076157in}}%
\pgfpathlineto{\pgfqpoint{2.841543in}{1.876535in}}%
\pgfpathclose%
\pgfusepath{fill}%
\end{pgfscope}%
\begin{pgfscope}%
\pgfpathrectangle{\pgfqpoint{1.072000in}{0.528000in}}{\pgfqpoint{3.696000in}{3.696000in}}%
\pgfusepath{clip}%
\pgfsetbuttcap%
\pgfsetroundjoin%
\definecolor{currentfill}{rgb}{0.717435,0.051118,0.158737}%
\pgfsetfillcolor{currentfill}%
\pgfsetlinewidth{0.000000pt}%
\definecolor{currentstroke}{rgb}{0.000000,0.000000,0.000000}%
\pgfsetstrokecolor{currentstroke}%
\pgfsetdash{}{0pt}%
\pgfpathmoveto{\pgfqpoint{2.606918in}{3.056447in}}%
\pgfpathlineto{\pgfqpoint{2.653529in}{3.078264in}}%
\pgfpathlineto{\pgfqpoint{2.681323in}{3.120820in}}%
\pgfpathlineto{\pgfqpoint{2.634528in}{3.119139in}}%
\pgfpathlineto{\pgfqpoint{2.606918in}{3.056447in}}%
\pgfpathclose%
\pgfusepath{fill}%
\end{pgfscope}%
\begin{pgfscope}%
\pgfpathrectangle{\pgfqpoint{1.072000in}{0.528000in}}{\pgfqpoint{3.696000in}{3.696000in}}%
\pgfusepath{clip}%
\pgfsetbuttcap%
\pgfsetroundjoin%
\definecolor{currentfill}{rgb}{0.275827,0.366717,0.812553}%
\pgfsetfillcolor{currentfill}%
\pgfsetlinewidth{0.000000pt}%
\definecolor{currentstroke}{rgb}{0.000000,0.000000,0.000000}%
\pgfsetstrokecolor{currentstroke}%
\pgfsetdash{}{0pt}%
\pgfpathmoveto{\pgfqpoint{3.260047in}{1.195283in}}%
\pgfpathlineto{\pgfqpoint{3.308403in}{1.241237in}}%
\pgfpathlineto{\pgfqpoint{3.334398in}{1.206344in}}%
\pgfpathlineto{\pgfqpoint{3.286330in}{1.179550in}}%
\pgfpathlineto{\pgfqpoint{3.260047in}{1.195283in}}%
\pgfpathclose%
\pgfusepath{fill}%
\end{pgfscope}%
\begin{pgfscope}%
\pgfpathrectangle{\pgfqpoint{1.072000in}{0.528000in}}{\pgfqpoint{3.696000in}{3.696000in}}%
\pgfusepath{clip}%
\pgfsetbuttcap%
\pgfsetroundjoin%
\definecolor{currentfill}{rgb}{0.895885,0.433075,0.338681}%
\pgfsetfillcolor{currentfill}%
\pgfsetlinewidth{0.000000pt}%
\definecolor{currentstroke}{rgb}{0.000000,0.000000,0.000000}%
\pgfsetstrokecolor{currentstroke}%
\pgfsetdash{}{0pt}%
\pgfpathmoveto{\pgfqpoint{3.406870in}{2.865235in}}%
\pgfpathlineto{\pgfqpoint{3.452970in}{2.714098in}}%
\pgfpathlineto{\pgfqpoint{3.479347in}{2.744727in}}%
\pgfpathlineto{\pgfqpoint{3.433392in}{2.894705in}}%
\pgfpathlineto{\pgfqpoint{3.406870in}{2.865235in}}%
\pgfpathclose%
\pgfusepath{fill}%
\end{pgfscope}%
\begin{pgfscope}%
\pgfpathrectangle{\pgfqpoint{1.072000in}{0.528000in}}{\pgfqpoint{3.696000in}{3.696000in}}%
\pgfusepath{clip}%
\pgfsetbuttcap%
\pgfsetroundjoin%
\definecolor{currentfill}{rgb}{0.859385,0.864431,0.872111}%
\pgfsetfillcolor{currentfill}%
\pgfsetlinewidth{0.000000pt}%
\definecolor{currentstroke}{rgb}{0.000000,0.000000,0.000000}%
\pgfsetstrokecolor{currentstroke}%
\pgfsetdash{}{0pt}%
\pgfpathmoveto{\pgfqpoint{2.699907in}{2.021690in}}%
\pgfpathlineto{\pgfqpoint{2.747016in}{1.986450in}}%
\pgfpathlineto{\pgfqpoint{2.773654in}{2.194628in}}%
\pgfpathlineto{\pgfqpoint{2.726360in}{2.231661in}}%
\pgfpathlineto{\pgfqpoint{2.699907in}{2.021690in}}%
\pgfpathclose%
\pgfusepath{fill}%
\end{pgfscope}%
\begin{pgfscope}%
\pgfpathrectangle{\pgfqpoint{1.072000in}{0.528000in}}{\pgfqpoint{3.696000in}{3.696000in}}%
\pgfusepath{clip}%
\pgfsetbuttcap%
\pgfsetroundjoin%
\definecolor{currentfill}{rgb}{0.967317,0.657471,0.538160}%
\pgfsetfillcolor{currentfill}%
\pgfsetlinewidth{0.000000pt}%
\definecolor{currentstroke}{rgb}{0.000000,0.000000,0.000000}%
\pgfsetstrokecolor{currentstroke}%
\pgfsetdash{}{0pt}%
\pgfpathmoveto{\pgfqpoint{1.556147in}{2.652053in}}%
\pgfpathlineto{\pgfqpoint{1.606401in}{2.527340in}}%
\pgfpathlineto{\pgfqpoint{1.641498in}{2.421031in}}%
\pgfpathlineto{\pgfqpoint{1.591085in}{2.557926in}}%
\pgfpathlineto{\pgfqpoint{1.556147in}{2.652053in}}%
\pgfpathclose%
\pgfusepath{fill}%
\end{pgfscope}%
\begin{pgfscope}%
\pgfpathrectangle{\pgfqpoint{1.072000in}{0.528000in}}{\pgfqpoint{3.696000in}{3.696000in}}%
\pgfusepath{clip}%
\pgfsetbuttcap%
\pgfsetroundjoin%
\definecolor{currentfill}{rgb}{0.810616,0.268797,0.235428}%
\pgfsetfillcolor{currentfill}%
\pgfsetlinewidth{0.000000pt}%
\definecolor{currentstroke}{rgb}{0.000000,0.000000,0.000000}%
\pgfsetstrokecolor{currentstroke}%
\pgfsetdash{}{0pt}%
\pgfpathmoveto{\pgfqpoint{3.386745in}{3.008307in}}%
\pgfpathlineto{\pgfqpoint{3.433392in}{2.894705in}}%
\pgfpathlineto{\pgfqpoint{3.459628in}{2.910675in}}%
\pgfpathlineto{\pgfqpoint{3.413151in}{3.025725in}}%
\pgfpathlineto{\pgfqpoint{3.386745in}{3.008307in}}%
\pgfpathclose%
\pgfusepath{fill}%
\end{pgfscope}%
\begin{pgfscope}%
\pgfpathrectangle{\pgfqpoint{1.072000in}{0.528000in}}{\pgfqpoint{3.696000in}{3.696000in}}%
\pgfusepath{clip}%
\pgfsetbuttcap%
\pgfsetroundjoin%
\definecolor{currentfill}{rgb}{0.969522,0.700833,0.587508}%
\pgfsetfillcolor{currentfill}%
\pgfsetlinewidth{0.000000pt}%
\definecolor{currentstroke}{rgb}{0.000000,0.000000,0.000000}%
\pgfsetstrokecolor{currentstroke}%
\pgfsetdash{}{0pt}%
\pgfpathmoveto{\pgfqpoint{3.372161in}{2.527122in}}%
\pgfpathlineto{\pgfqpoint{3.417973in}{2.347251in}}%
\pgfpathlineto{\pgfqpoint{3.445045in}{2.426067in}}%
\pgfpathlineto{\pgfqpoint{3.399347in}{2.605735in}}%
\pgfpathlineto{\pgfqpoint{3.372161in}{2.527122in}}%
\pgfpathclose%
\pgfusepath{fill}%
\end{pgfscope}%
\begin{pgfscope}%
\pgfpathrectangle{\pgfqpoint{1.072000in}{0.528000in}}{\pgfqpoint{3.696000in}{3.696000in}}%
\pgfusepath{clip}%
\pgfsetbuttcap%
\pgfsetroundjoin%
\definecolor{currentfill}{rgb}{0.940879,0.805596,0.735167}%
\pgfsetfillcolor{currentfill}%
\pgfsetlinewidth{0.000000pt}%
\definecolor{currentstroke}{rgb}{0.000000,0.000000,0.000000}%
\pgfsetstrokecolor{currentstroke}%
\pgfsetdash{}{0pt}%
\pgfpathmoveto{\pgfqpoint{3.317225in}{2.321161in}}%
\pgfpathlineto{\pgfqpoint{3.363258in}{2.146017in}}%
\pgfpathlineto{\pgfqpoint{3.390694in}{2.253438in}}%
\pgfpathlineto{\pgfqpoint{3.344772in}{2.432035in}}%
\pgfpathlineto{\pgfqpoint{3.317225in}{2.321161in}}%
\pgfpathclose%
\pgfusepath{fill}%
\end{pgfscope}%
\begin{pgfscope}%
\pgfpathrectangle{\pgfqpoint{1.072000in}{0.528000in}}{\pgfqpoint{3.696000in}{3.696000in}}%
\pgfusepath{clip}%
\pgfsetbuttcap%
\pgfsetroundjoin%
\definecolor{currentfill}{rgb}{0.809329,0.852974,0.922323}%
\pgfsetfillcolor{currentfill}%
\pgfsetlinewidth{0.000000pt}%
\definecolor{currentstroke}{rgb}{0.000000,0.000000,0.000000}%
\pgfsetstrokecolor{currentstroke}%
\pgfsetdash{}{0pt}%
\pgfpathmoveto{\pgfqpoint{2.794248in}{1.937723in}}%
\pgfpathlineto{\pgfqpoint{2.841543in}{1.876535in}}%
\pgfpathlineto{\pgfqpoint{2.868518in}{2.076157in}}%
\pgfpathlineto{\pgfqpoint{2.821061in}{2.142548in}}%
\pgfpathlineto{\pgfqpoint{2.794248in}{1.937723in}}%
\pgfpathclose%
\pgfusepath{fill}%
\end{pgfscope}%
\begin{pgfscope}%
\pgfpathrectangle{\pgfqpoint{1.072000in}{0.528000in}}{\pgfqpoint{3.696000in}{3.696000in}}%
\pgfusepath{clip}%
\pgfsetbuttcap%
\pgfsetroundjoin%
\definecolor{currentfill}{rgb}{0.758112,0.168122,0.188827}%
\pgfsetfillcolor{currentfill}%
\pgfsetlinewidth{0.000000pt}%
\definecolor{currentstroke}{rgb}{0.000000,0.000000,0.000000}%
\pgfsetstrokecolor{currentstroke}%
\pgfsetdash{}{0pt}%
\pgfpathmoveto{\pgfqpoint{2.775627in}{3.076105in}}%
\pgfpathlineto{\pgfqpoint{2.822929in}{3.049005in}}%
\pgfpathlineto{\pgfqpoint{2.850829in}{2.995651in}}%
\pgfpathlineto{\pgfqpoint{2.803620in}{3.040390in}}%
\pgfpathlineto{\pgfqpoint{2.775627in}{3.076105in}}%
\pgfpathclose%
\pgfusepath{fill}%
\end{pgfscope}%
\begin{pgfscope}%
\pgfpathrectangle{\pgfqpoint{1.072000in}{0.528000in}}{\pgfqpoint{3.696000in}{3.696000in}}%
\pgfusepath{clip}%
\pgfsetbuttcap%
\pgfsetroundjoin%
\definecolor{currentfill}{rgb}{0.839351,0.861167,0.894494}%
\pgfsetfillcolor{currentfill}%
\pgfsetlinewidth{0.000000pt}%
\definecolor{currentstroke}{rgb}{0.000000,0.000000,0.000000}%
\pgfsetstrokecolor{currentstroke}%
\pgfsetdash{}{0pt}%
\pgfpathmoveto{\pgfqpoint{2.747016in}{1.986450in}}%
\pgfpathlineto{\pgfqpoint{2.794248in}{1.937723in}}%
\pgfpathlineto{\pgfqpoint{2.821061in}{2.142548in}}%
\pgfpathlineto{\pgfqpoint{2.773654in}{2.194628in}}%
\pgfpathlineto{\pgfqpoint{2.747016in}{1.986450in}}%
\pgfpathclose%
\pgfusepath{fill}%
\end{pgfscope}%
\begin{pgfscope}%
\pgfpathrectangle{\pgfqpoint{1.072000in}{0.528000in}}{\pgfqpoint{3.696000in}{3.696000in}}%
\pgfusepath{clip}%
\pgfsetbuttcap%
\pgfsetroundjoin%
\definecolor{currentfill}{rgb}{0.830187,0.304733,0.254891}%
\pgfsetfillcolor{currentfill}%
\pgfsetlinewidth{0.000000pt}%
\definecolor{currentstroke}{rgb}{0.000000,0.000000,0.000000}%
\pgfsetstrokecolor{currentstroke}%
\pgfsetdash{}{0pt}%
\pgfpathmoveto{\pgfqpoint{2.552614in}{2.841195in}}%
\pgfpathlineto{\pgfqpoint{2.598781in}{2.890102in}}%
\pgfpathlineto{\pgfqpoint{2.625985in}{3.001631in}}%
\pgfpathlineto{\pgfqpoint{2.579587in}{2.963952in}}%
\pgfpathlineto{\pgfqpoint{2.552614in}{2.841195in}}%
\pgfpathclose%
\pgfusepath{fill}%
\end{pgfscope}%
\begin{pgfscope}%
\pgfpathrectangle{\pgfqpoint{1.072000in}{0.528000in}}{\pgfqpoint{3.696000in}{3.696000in}}%
\pgfusepath{clip}%
\pgfsetbuttcap%
\pgfsetroundjoin%
\definecolor{currentfill}{rgb}{0.839351,0.861167,0.894494}%
\pgfsetfillcolor{currentfill}%
\pgfsetlinewidth{0.000000pt}%
\definecolor{currentstroke}{rgb}{0.000000,0.000000,0.000000}%
\pgfsetstrokecolor{currentstroke}%
\pgfsetdash{}{0pt}%
\pgfpathmoveto{\pgfqpoint{3.187517in}{2.064211in}}%
\pgfpathlineto{\pgfqpoint{3.234171in}{1.916383in}}%
\pgfpathlineto{\pgfqpoint{3.261867in}{2.059939in}}%
\pgfpathlineto{\pgfqpoint{3.215282in}{2.216731in}}%
\pgfpathlineto{\pgfqpoint{3.187517in}{2.064211in}}%
\pgfpathclose%
\pgfusepath{fill}%
\end{pgfscope}%
\begin{pgfscope}%
\pgfpathrectangle{\pgfqpoint{1.072000in}{0.528000in}}{\pgfqpoint{3.696000in}{3.696000in}}%
\pgfusepath{clip}%
\pgfsetbuttcap%
\pgfsetroundjoin%
\definecolor{currentfill}{rgb}{0.939254,0.539581,0.423900}%
\pgfsetfillcolor{currentfill}%
\pgfsetlinewidth{0.000000pt}%
\definecolor{currentstroke}{rgb}{0.000000,0.000000,0.000000}%
\pgfsetstrokecolor{currentstroke}%
\pgfsetdash{}{0pt}%
\pgfpathmoveto{\pgfqpoint{2.545567in}{2.575881in}}%
\pgfpathlineto{\pgfqpoint{2.591755in}{2.617521in}}%
\pgfpathlineto{\pgfqpoint{2.618372in}{2.784691in}}%
\pgfpathlineto{\pgfqpoint{2.571971in}{2.746325in}}%
\pgfpathlineto{\pgfqpoint{2.545567in}{2.575881in}}%
\pgfpathclose%
\pgfusepath{fill}%
\end{pgfscope}%
\begin{pgfscope}%
\pgfpathrectangle{\pgfqpoint{1.072000in}{0.528000in}}{\pgfqpoint{3.696000in}{3.696000in}}%
\pgfusepath{clip}%
\pgfsetbuttcap%
\pgfsetroundjoin%
\definecolor{currentfill}{rgb}{0.962708,0.753557,0.655601}%
\pgfsetfillcolor{currentfill}%
\pgfsetlinewidth{0.000000pt}%
\definecolor{currentstroke}{rgb}{0.000000,0.000000,0.000000}%
\pgfsetstrokecolor{currentstroke}%
\pgfsetdash{}{0pt}%
\pgfpathmoveto{\pgfqpoint{3.344772in}{2.432035in}}%
\pgfpathlineto{\pgfqpoint{3.390694in}{2.253438in}}%
\pgfpathlineto{\pgfqpoint{3.417973in}{2.347251in}}%
\pgfpathlineto{\pgfqpoint{3.372161in}{2.527122in}}%
\pgfpathlineto{\pgfqpoint{3.344772in}{2.432035in}}%
\pgfpathclose%
\pgfusepath{fill}%
\end{pgfscope}%
\begin{pgfscope}%
\pgfpathrectangle{\pgfqpoint{1.072000in}{0.528000in}}{\pgfqpoint{3.696000in}{3.696000in}}%
\pgfusepath{clip}%
\pgfsetbuttcap%
\pgfsetroundjoin%
\definecolor{currentfill}{rgb}{0.777378,0.840921,0.946149}%
\pgfsetfillcolor{currentfill}%
\pgfsetlinewidth{0.000000pt}%
\definecolor{currentstroke}{rgb}{0.000000,0.000000,0.000000}%
\pgfsetstrokecolor{currentstroke}%
\pgfsetdash{}{0pt}%
\pgfpathmoveto{\pgfqpoint{2.963343in}{1.906069in}}%
\pgfpathlineto{\pgfqpoint{3.010612in}{1.806864in}}%
\pgfpathlineto{\pgfqpoint{3.038066in}{1.985792in}}%
\pgfpathlineto{\pgfqpoint{2.990722in}{2.094660in}}%
\pgfpathlineto{\pgfqpoint{2.963343in}{1.906069in}}%
\pgfpathclose%
\pgfusepath{fill}%
\end{pgfscope}%
\begin{pgfscope}%
\pgfpathrectangle{\pgfqpoint{1.072000in}{0.528000in}}{\pgfqpoint{3.696000in}{3.696000in}}%
\pgfusepath{clip}%
\pgfsetbuttcap%
\pgfsetroundjoin%
\definecolor{currentfill}{rgb}{0.967317,0.657471,0.538160}%
\pgfsetfillcolor{currentfill}%
\pgfsetlinewidth{0.000000pt}%
\definecolor{currentstroke}{rgb}{0.000000,0.000000,0.000000}%
\pgfsetstrokecolor{currentstroke}%
\pgfsetdash{}{0pt}%
\pgfpathmoveto{\pgfqpoint{2.565517in}{2.428025in}}%
\pgfpathlineto{\pgfqpoint{2.611919in}{2.453996in}}%
\pgfpathlineto{\pgfqpoint{2.638355in}{2.643012in}}%
\pgfpathlineto{\pgfqpoint{2.591755in}{2.617521in}}%
\pgfpathlineto{\pgfqpoint{2.565517in}{2.428025in}}%
\pgfpathclose%
\pgfusepath{fill}%
\end{pgfscope}%
\begin{pgfscope}%
\pgfpathrectangle{\pgfqpoint{1.072000in}{0.528000in}}{\pgfqpoint{3.696000in}{3.696000in}}%
\pgfusepath{clip}%
\pgfsetbuttcap%
\pgfsetroundjoin%
\definecolor{currentfill}{rgb}{0.763520,0.178667,0.193396}%
\pgfsetfillcolor{currentfill}%
\pgfsetlinewidth{0.000000pt}%
\definecolor{currentstroke}{rgb}{0.000000,0.000000,0.000000}%
\pgfsetstrokecolor{currentstroke}%
\pgfsetdash{}{0pt}%
\pgfpathmoveto{\pgfqpoint{2.579587in}{2.963952in}}%
\pgfpathlineto{\pgfqpoint{2.625985in}{3.001631in}}%
\pgfpathlineto{\pgfqpoint{2.653529in}{3.078264in}}%
\pgfpathlineto{\pgfqpoint{2.606918in}{3.056447in}}%
\pgfpathlineto{\pgfqpoint{2.579587in}{2.963952in}}%
\pgfpathclose%
\pgfusepath{fill}%
\end{pgfscope}%
\begin{pgfscope}%
\pgfpathrectangle{\pgfqpoint{1.072000in}{0.528000in}}{\pgfqpoint{3.696000in}{3.696000in}}%
\pgfusepath{clip}%
\pgfsetbuttcap%
\pgfsetroundjoin%
\definecolor{currentfill}{rgb}{0.271104,0.360011,0.807095}%
\pgfsetfillcolor{currentfill}%
\pgfsetlinewidth{0.000000pt}%
\definecolor{currentstroke}{rgb}{0.000000,0.000000,0.000000}%
\pgfsetstrokecolor{currentstroke}%
\pgfsetdash{}{0pt}%
\pgfpathmoveto{\pgfqpoint{2.866568in}{1.188596in}}%
\pgfpathlineto{\pgfqpoint{2.913842in}{1.208609in}}%
\pgfpathlineto{\pgfqpoint{2.941387in}{1.166598in}}%
\pgfpathlineto{\pgfqpoint{2.894212in}{1.161438in}}%
\pgfpathlineto{\pgfqpoint{2.866568in}{1.188596in}}%
\pgfpathclose%
\pgfusepath{fill}%
\end{pgfscope}%
\begin{pgfscope}%
\pgfpathrectangle{\pgfqpoint{1.072000in}{0.528000in}}{\pgfqpoint{3.696000in}{3.696000in}}%
\pgfusepath{clip}%
\pgfsetbuttcap%
\pgfsetroundjoin%
\definecolor{currentfill}{rgb}{0.280550,0.373423,0.818011}%
\pgfsetfillcolor{currentfill}%
\pgfsetlinewidth{0.000000pt}%
\definecolor{currentstroke}{rgb}{0.000000,0.000000,0.000000}%
\pgfsetstrokecolor{currentstroke}%
\pgfsetdash{}{0pt}%
\pgfpathmoveto{\pgfqpoint{2.744326in}{1.214709in}}%
\pgfpathlineto{\pgfqpoint{2.791373in}{1.228860in}}%
\pgfpathlineto{\pgfqpoint{2.819463in}{1.178031in}}%
\pgfpathlineto{\pgfqpoint{2.772496in}{1.175290in}}%
\pgfpathlineto{\pgfqpoint{2.744326in}{1.214709in}}%
\pgfpathclose%
\pgfusepath{fill}%
\end{pgfscope}%
\begin{pgfscope}%
\pgfpathrectangle{\pgfqpoint{1.072000in}{0.528000in}}{\pgfqpoint{3.696000in}{3.696000in}}%
\pgfusepath{clip}%
\pgfsetbuttcap%
\pgfsetroundjoin%
\definecolor{currentfill}{rgb}{0.280550,0.373423,0.818011}%
\pgfsetfillcolor{currentfill}%
\pgfsetlinewidth{0.000000pt}%
\definecolor{currentstroke}{rgb}{0.000000,0.000000,0.000000}%
\pgfsetstrokecolor{currentstroke}%
\pgfsetdash{}{0pt}%
\pgfpathmoveto{\pgfqpoint{3.185615in}{1.191567in}}%
\pgfpathlineto{\pgfqpoint{3.233778in}{1.234998in}}%
\pgfpathlineto{\pgfqpoint{3.260047in}{1.195283in}}%
\pgfpathlineto{\pgfqpoint{3.212134in}{1.170698in}}%
\pgfpathlineto{\pgfqpoint{3.185615in}{1.191567in}}%
\pgfpathclose%
\pgfusepath{fill}%
\end{pgfscope}%
\begin{pgfscope}%
\pgfpathrectangle{\pgfqpoint{1.072000in}{0.528000in}}{\pgfqpoint{3.696000in}{3.696000in}}%
\pgfusepath{clip}%
\pgfsetbuttcap%
\pgfsetroundjoin%
\definecolor{currentfill}{rgb}{0.705673,0.015556,0.150233}%
\pgfsetfillcolor{currentfill}%
\pgfsetlinewidth{0.000000pt}%
\definecolor{currentstroke}{rgb}{0.000000,0.000000,0.000000}%
\pgfsetstrokecolor{currentstroke}%
\pgfsetdash{}{0pt}%
\pgfpathmoveto{\pgfqpoint{3.291751in}{3.102008in}}%
\pgfpathlineto{\pgfqpoint{3.339484in}{3.078706in}}%
\pgfpathlineto{\pgfqpoint{3.366034in}{3.098486in}}%
\pgfpathlineto{\pgfqpoint{3.318413in}{3.125236in}}%
\pgfpathlineto{\pgfqpoint{3.291751in}{3.102008in}}%
\pgfpathclose%
\pgfusepath{fill}%
\end{pgfscope}%
\begin{pgfscope}%
\pgfpathrectangle{\pgfqpoint{1.072000in}{0.528000in}}{\pgfqpoint{3.696000in}{3.696000in}}%
\pgfusepath{clip}%
\pgfsetbuttcap%
\pgfsetroundjoin%
\definecolor{currentfill}{rgb}{0.768929,0.189213,0.197965}%
\pgfsetfillcolor{currentfill}%
\pgfsetlinewidth{0.000000pt}%
\definecolor{currentstroke}{rgb}{0.000000,0.000000,0.000000}%
\pgfsetstrokecolor{currentstroke}%
\pgfsetdash{}{0pt}%
\pgfpathmoveto{\pgfqpoint{2.822929in}{3.049005in}}%
\pgfpathlineto{\pgfqpoint{2.870281in}{3.030245in}}%
\pgfpathlineto{\pgfqpoint{2.898024in}{2.970247in}}%
\pgfpathlineto{\pgfqpoint{2.850829in}{2.995651in}}%
\pgfpathlineto{\pgfqpoint{2.822929in}{3.049005in}}%
\pgfpathclose%
\pgfusepath{fill}%
\end{pgfscope}%
\begin{pgfscope}%
\pgfpathrectangle{\pgfqpoint{1.072000in}{0.528000in}}{\pgfqpoint{3.696000in}{3.696000in}}%
\pgfusepath{clip}%
\pgfsetbuttcap%
\pgfsetroundjoin%
\definecolor{currentfill}{rgb}{0.271104,0.360011,0.807095}%
\pgfsetfillcolor{currentfill}%
\pgfsetlinewidth{0.000000pt}%
\definecolor{currentstroke}{rgb}{0.000000,0.000000,0.000000}%
\pgfsetstrokecolor{currentstroke}%
\pgfsetdash{}{0pt}%
\pgfpathmoveto{\pgfqpoint{2.988739in}{1.182659in}}%
\pgfpathlineto{\pgfqpoint{3.036317in}{1.211371in}}%
\pgfpathlineto{\pgfqpoint{3.063376in}{1.169916in}}%
\pgfpathlineto{\pgfqpoint{3.015948in}{1.158365in}}%
\pgfpathlineto{\pgfqpoint{2.988739in}{1.182659in}}%
\pgfpathclose%
\pgfusepath{fill}%
\end{pgfscope}%
\begin{pgfscope}%
\pgfpathrectangle{\pgfqpoint{1.072000in}{0.528000in}}{\pgfqpoint{3.696000in}{3.696000in}}%
\pgfusepath{clip}%
\pgfsetbuttcap%
\pgfsetroundjoin%
\definecolor{currentfill}{rgb}{0.785153,0.220851,0.211673}%
\pgfsetfillcolor{currentfill}%
\pgfsetlinewidth{0.000000pt}%
\definecolor{currentstroke}{rgb}{0.000000,0.000000,0.000000}%
\pgfsetstrokecolor{currentstroke}%
\pgfsetdash{}{0pt}%
\pgfpathmoveto{\pgfqpoint{3.067425in}{2.977648in}}%
\pgfpathlineto{\pgfqpoint{3.115237in}{3.023276in}}%
\pgfpathlineto{\pgfqpoint{3.142138in}{3.007551in}}%
\pgfpathlineto{\pgfqpoint{3.094295in}{2.932522in}}%
\pgfpathlineto{\pgfqpoint{3.067425in}{2.977648in}}%
\pgfpathclose%
\pgfusepath{fill}%
\end{pgfscope}%
\begin{pgfscope}%
\pgfpathrectangle{\pgfqpoint{1.072000in}{0.528000in}}{\pgfqpoint{3.696000in}{3.696000in}}%
\pgfusepath{clip}%
\pgfsetbuttcap%
\pgfsetroundjoin%
\definecolor{currentfill}{rgb}{0.740957,0.122240,0.175744}%
\pgfsetfillcolor{currentfill}%
\pgfsetlinewidth{0.000000pt}%
\definecolor{currentstroke}{rgb}{0.000000,0.000000,0.000000}%
\pgfsetstrokecolor{currentstroke}%
\pgfsetdash{}{0pt}%
\pgfpathmoveto{\pgfqpoint{3.339484in}{3.078706in}}%
\pgfpathlineto{\pgfqpoint{3.386745in}{3.008307in}}%
\pgfpathlineto{\pgfqpoint{3.413151in}{3.025725in}}%
\pgfpathlineto{\pgfqpoint{3.366034in}{3.098486in}}%
\pgfpathlineto{\pgfqpoint{3.339484in}{3.078706in}}%
\pgfpathclose%
\pgfusepath{fill}%
\end{pgfscope}%
\begin{pgfscope}%
\pgfpathrectangle{\pgfqpoint{1.072000in}{0.528000in}}{\pgfqpoint{3.696000in}{3.696000in}}%
\pgfusepath{clip}%
\pgfsetbuttcap%
\pgfsetroundjoin%
\definecolor{currentfill}{rgb}{0.758112,0.168122,0.188827}%
\pgfsetfillcolor{currentfill}%
\pgfsetlinewidth{0.000000pt}%
\definecolor{currentstroke}{rgb}{0.000000,0.000000,0.000000}%
\pgfsetstrokecolor{currentstroke}%
\pgfsetdash{}{0pt}%
\pgfpathmoveto{\pgfqpoint{3.142138in}{3.007551in}}%
\pgfpathlineto{\pgfqpoint{3.190147in}{3.052755in}}%
\pgfpathlineto{\pgfqpoint{3.216973in}{3.061020in}}%
\pgfpathlineto{\pgfqpoint{3.168906in}{2.998194in}}%
\pgfpathlineto{\pgfqpoint{3.142138in}{3.007551in}}%
\pgfpathclose%
\pgfusepath{fill}%
\end{pgfscope}%
\begin{pgfscope}%
\pgfpathrectangle{\pgfqpoint{1.072000in}{0.528000in}}{\pgfqpoint{3.696000in}{3.696000in}}%
\pgfusepath{clip}%
\pgfsetbuttcap%
\pgfsetroundjoin%
\definecolor{currentfill}{rgb}{0.839351,0.861167,0.894494}%
\pgfsetfillcolor{currentfill}%
\pgfsetlinewidth{0.000000pt}%
\definecolor{currentstroke}{rgb}{0.000000,0.000000,0.000000}%
\pgfsetstrokecolor{currentstroke}%
\pgfsetdash{}{0pt}%
\pgfpathmoveto{\pgfqpoint{3.112853in}{2.039271in}}%
\pgfpathlineto{\pgfqpoint{3.159812in}{1.905949in}}%
\pgfpathlineto{\pgfqpoint{3.187517in}{2.064211in}}%
\pgfpathlineto{\pgfqpoint{3.140577in}{2.206512in}}%
\pgfpathlineto{\pgfqpoint{3.112853in}{2.039271in}}%
\pgfpathclose%
\pgfusepath{fill}%
\end{pgfscope}%
\begin{pgfscope}%
\pgfpathrectangle{\pgfqpoint{1.072000in}{0.528000in}}{\pgfqpoint{3.696000in}{3.696000in}}%
\pgfusepath{clip}%
\pgfsetbuttcap%
\pgfsetroundjoin%
\definecolor{currentfill}{rgb}{0.818056,0.855590,0.914638}%
\pgfsetfillcolor{currentfill}%
\pgfsetlinewidth{0.000000pt}%
\definecolor{currentstroke}{rgb}{0.000000,0.000000,0.000000}%
\pgfsetstrokecolor{currentstroke}%
\pgfsetdash{}{0pt}%
\pgfpathmoveto{\pgfqpoint{3.038066in}{1.985792in}}%
\pgfpathlineto{\pgfqpoint{3.085233in}{1.868871in}}%
\pgfpathlineto{\pgfqpoint{3.112853in}{2.039271in}}%
\pgfpathlineto{\pgfqpoint{3.065656in}{2.165475in}}%
\pgfpathlineto{\pgfqpoint{3.038066in}{1.985792in}}%
\pgfpathclose%
\pgfusepath{fill}%
\end{pgfscope}%
\begin{pgfscope}%
\pgfpathrectangle{\pgfqpoint{1.072000in}{0.528000in}}{\pgfqpoint{3.696000in}{3.696000in}}%
\pgfusepath{clip}%
\pgfsetbuttcap%
\pgfsetroundjoin%
\definecolor{currentfill}{rgb}{0.960581,0.762501,0.667964}%
\pgfsetfillcolor{currentfill}%
\pgfsetlinewidth{0.000000pt}%
\definecolor{currentstroke}{rgb}{0.000000,0.000000,0.000000}%
\pgfsetstrokecolor{currentstroke}%
\pgfsetdash{}{0pt}%
\pgfpathmoveto{\pgfqpoint{2.632372in}{2.259419in}}%
\pgfpathlineto{\pgfqpoint{2.679245in}{2.253286in}}%
\pgfpathlineto{\pgfqpoint{2.705739in}{2.458476in}}%
\pgfpathlineto{\pgfqpoint{2.658680in}{2.464185in}}%
\pgfpathlineto{\pgfqpoint{2.632372in}{2.259419in}}%
\pgfpathclose%
\pgfusepath{fill}%
\end{pgfscope}%
\begin{pgfscope}%
\pgfpathrectangle{\pgfqpoint{1.072000in}{0.528000in}}{\pgfqpoint{3.696000in}{3.696000in}}%
\pgfusepath{clip}%
\pgfsetbuttcap%
\pgfsetroundjoin%
\definecolor{currentfill}{rgb}{0.705673,0.015556,0.150233}%
\pgfsetfillcolor{currentfill}%
\pgfsetlinewidth{0.000000pt}%
\definecolor{currentstroke}{rgb}{0.000000,0.000000,0.000000}%
\pgfsetstrokecolor{currentstroke}%
\pgfsetdash{}{0pt}%
\pgfpathmoveto{\pgfqpoint{2.653529in}{3.078264in}}%
\pgfpathlineto{\pgfqpoint{2.700485in}{3.081627in}}%
\pgfpathlineto{\pgfqpoint{2.728398in}{3.103398in}}%
\pgfpathlineto{\pgfqpoint{2.681323in}{3.120820in}}%
\pgfpathlineto{\pgfqpoint{2.653529in}{3.078264in}}%
\pgfpathclose%
\pgfusepath{fill}%
\end{pgfscope}%
\begin{pgfscope}%
\pgfpathrectangle{\pgfqpoint{1.072000in}{0.528000in}}{\pgfqpoint{3.696000in}{3.696000in}}%
\pgfusepath{clip}%
\pgfsetbuttcap%
\pgfsetroundjoin%
\definecolor{currentfill}{rgb}{0.717435,0.051118,0.158737}%
\pgfsetfillcolor{currentfill}%
\pgfsetlinewidth{0.000000pt}%
\definecolor{currentstroke}{rgb}{0.000000,0.000000,0.000000}%
\pgfsetstrokecolor{currentstroke}%
\pgfsetdash{}{0pt}%
\pgfpathmoveto{\pgfqpoint{3.216973in}{3.061020in}}%
\pgfpathlineto{\pgfqpoint{3.264997in}{3.080613in}}%
\pgfpathlineto{\pgfqpoint{3.291751in}{3.102008in}}%
\pgfpathlineto{\pgfqpoint{3.243734in}{3.077602in}}%
\pgfpathlineto{\pgfqpoint{3.216973in}{3.061020in}}%
\pgfpathclose%
\pgfusepath{fill}%
\end{pgfscope}%
\begin{pgfscope}%
\pgfpathrectangle{\pgfqpoint{1.072000in}{0.528000in}}{\pgfqpoint{3.696000in}{3.696000in}}%
\pgfusepath{clip}%
\pgfsetbuttcap%
\pgfsetroundjoin%
\definecolor{currentfill}{rgb}{0.905783,0.455186,0.355336}%
\pgfsetfillcolor{currentfill}%
\pgfsetlinewidth{0.000000pt}%
\definecolor{currentstroke}{rgb}{0.000000,0.000000,0.000000}%
\pgfsetstrokecolor{currentstroke}%
\pgfsetdash{}{0pt}%
\pgfpathmoveto{\pgfqpoint{3.380087in}{2.821807in}}%
\pgfpathlineto{\pgfqpoint{3.426294in}{2.667919in}}%
\pgfpathlineto{\pgfqpoint{3.452970in}{2.714098in}}%
\pgfpathlineto{\pgfqpoint{3.406870in}{2.865235in}}%
\pgfpathlineto{\pgfqpoint{3.380087in}{2.821807in}}%
\pgfpathclose%
\pgfusepath{fill}%
\end{pgfscope}%
\begin{pgfscope}%
\pgfpathrectangle{\pgfqpoint{1.072000in}{0.528000in}}{\pgfqpoint{3.696000in}{3.696000in}}%
\pgfusepath{clip}%
\pgfsetbuttcap%
\pgfsetroundjoin%
\definecolor{currentfill}{rgb}{0.839351,0.861167,0.894494}%
\pgfsetfillcolor{currentfill}%
\pgfsetlinewidth{0.000000pt}%
\definecolor{currentstroke}{rgb}{0.000000,0.000000,0.000000}%
\pgfsetstrokecolor{currentstroke}%
\pgfsetdash{}{0pt}%
\pgfpathmoveto{\pgfqpoint{2.915964in}{1.996702in}}%
\pgfpathlineto{\pgfqpoint{2.963343in}{1.906069in}}%
\pgfpathlineto{\pgfqpoint{2.990722in}{2.094660in}}%
\pgfpathlineto{\pgfqpoint{2.943236in}{2.192294in}}%
\pgfpathlineto{\pgfqpoint{2.915964in}{1.996702in}}%
\pgfpathclose%
\pgfusepath{fill}%
\end{pgfscope}%
\begin{pgfscope}%
\pgfpathrectangle{\pgfqpoint{1.072000in}{0.528000in}}{\pgfqpoint{3.696000in}{3.696000in}}%
\pgfusepath{clip}%
\pgfsetbuttcap%
\pgfsetroundjoin%
\definecolor{currentfill}{rgb}{0.919376,0.831273,0.782874}%
\pgfsetfillcolor{currentfill}%
\pgfsetlinewidth{0.000000pt}%
\definecolor{currentstroke}{rgb}{0.000000,0.000000,0.000000}%
\pgfsetstrokecolor{currentstroke}%
\pgfsetdash{}{0pt}%
\pgfpathmoveto{\pgfqpoint{3.215282in}{2.216731in}}%
\pgfpathlineto{\pgfqpoint{3.261867in}{2.059939in}}%
\pgfpathlineto{\pgfqpoint{3.289570in}{2.196185in}}%
\pgfpathlineto{\pgfqpoint{3.243050in}{2.358464in}}%
\pgfpathlineto{\pgfqpoint{3.215282in}{2.216731in}}%
\pgfpathclose%
\pgfusepath{fill}%
\end{pgfscope}%
\begin{pgfscope}%
\pgfpathrectangle{\pgfqpoint{1.072000in}{0.528000in}}{\pgfqpoint{3.696000in}{3.696000in}}%
\pgfusepath{clip}%
\pgfsetbuttcap%
\pgfsetroundjoin%
\definecolor{currentfill}{rgb}{0.785153,0.220851,0.211673}%
\pgfsetfillcolor{currentfill}%
\pgfsetlinewidth{0.000000pt}%
\definecolor{currentstroke}{rgb}{0.000000,0.000000,0.000000}%
\pgfsetstrokecolor{currentstroke}%
\pgfsetdash{}{0pt}%
\pgfpathmoveto{\pgfqpoint{2.992695in}{2.988501in}}%
\pgfpathlineto{\pgfqpoint{3.040322in}{3.013743in}}%
\pgfpathlineto{\pgfqpoint{3.067425in}{2.977648in}}%
\pgfpathlineto{\pgfqpoint{3.019872in}{2.922975in}}%
\pgfpathlineto{\pgfqpoint{2.992695in}{2.988501in}}%
\pgfpathclose%
\pgfusepath{fill}%
\end{pgfscope}%
\begin{pgfscope}%
\pgfpathrectangle{\pgfqpoint{1.072000in}{0.528000in}}{\pgfqpoint{3.696000in}{3.696000in}}%
\pgfusepath{clip}%
\pgfsetbuttcap%
\pgfsetroundjoin%
\definecolor{currentfill}{rgb}{0.289996,0.386836,0.828926}%
\pgfsetfillcolor{currentfill}%
\pgfsetlinewidth{0.000000pt}%
\definecolor{currentstroke}{rgb}{0.000000,0.000000,0.000000}%
\pgfsetstrokecolor{currentstroke}%
\pgfsetdash{}{0pt}%
\pgfpathmoveto{\pgfqpoint{3.111055in}{1.196264in}}%
\pgfpathlineto{\pgfqpoint{3.159059in}{1.238861in}}%
\pgfpathlineto{\pgfqpoint{3.185615in}{1.191567in}}%
\pgfpathlineto{\pgfqpoint{3.137826in}{1.166874in}}%
\pgfpathlineto{\pgfqpoint{3.111055in}{1.196264in}}%
\pgfpathclose%
\pgfusepath{fill}%
\end{pgfscope}%
\begin{pgfscope}%
\pgfpathrectangle{\pgfqpoint{1.072000in}{0.528000in}}{\pgfqpoint{3.696000in}{3.696000in}}%
\pgfusepath{clip}%
\pgfsetbuttcap%
\pgfsetroundjoin%
\definecolor{currentfill}{rgb}{0.939254,0.539581,0.423900}%
\pgfsetfillcolor{currentfill}%
\pgfsetlinewidth{0.000000pt}%
\definecolor{currentstroke}{rgb}{0.000000,0.000000,0.000000}%
\pgfsetstrokecolor{currentstroke}%
\pgfsetdash{}{0pt}%
\pgfpathmoveto{\pgfqpoint{1.506113in}{2.767179in}}%
\pgfpathlineto{\pgfqpoint{1.556147in}{2.652053in}}%
\pgfpathlineto{\pgfqpoint{1.591085in}{2.557926in}}%
\pgfpathlineto{\pgfqpoint{1.540710in}{2.688669in}}%
\pgfpathlineto{\pgfqpoint{1.506113in}{2.767179in}}%
\pgfpathclose%
\pgfusepath{fill}%
\end{pgfscope}%
\begin{pgfscope}%
\pgfpathrectangle{\pgfqpoint{1.072000in}{0.528000in}}{\pgfqpoint{3.696000in}{3.696000in}}%
\pgfusepath{clip}%
\pgfsetbuttcap%
\pgfsetroundjoin%
\definecolor{currentfill}{rgb}{0.358415,0.478426,0.896795}%
\pgfsetfillcolor{currentfill}%
\pgfsetlinewidth{0.000000pt}%
\definecolor{currentstroke}{rgb}{0.000000,0.000000,0.000000}%
\pgfsetstrokecolor{currentstroke}%
\pgfsetdash{}{0pt}%
\pgfpathmoveto{\pgfqpoint{3.457628in}{1.283075in}}%
\pgfpathlineto{\pgfqpoint{3.507272in}{1.366847in}}%
\pgfpathlineto{\pgfqpoint{3.532387in}{1.320331in}}%
\pgfpathlineto{\pgfqpoint{3.483068in}{1.251509in}}%
\pgfpathlineto{\pgfqpoint{3.457628in}{1.283075in}}%
\pgfpathclose%
\pgfusepath{fill}%
\end{pgfscope}%
\begin{pgfscope}%
\pgfpathrectangle{\pgfqpoint{1.072000in}{0.528000in}}{\pgfqpoint{3.696000in}{3.696000in}}%
\pgfusepath{clip}%
\pgfsetbuttcap%
\pgfsetroundjoin%
\definecolor{currentfill}{rgb}{0.810616,0.268797,0.235428}%
\pgfsetfillcolor{currentfill}%
\pgfsetlinewidth{0.000000pt}%
\definecolor{currentstroke}{rgb}{0.000000,0.000000,0.000000}%
\pgfsetstrokecolor{currentstroke}%
\pgfsetdash{}{0pt}%
\pgfpathmoveto{\pgfqpoint{3.360110in}{2.980680in}}%
\pgfpathlineto{\pgfqpoint{3.406870in}{2.865235in}}%
\pgfpathlineto{\pgfqpoint{3.433392in}{2.894705in}}%
\pgfpathlineto{\pgfqpoint{3.386745in}{3.008307in}}%
\pgfpathlineto{\pgfqpoint{3.360110in}{2.980680in}}%
\pgfpathclose%
\pgfusepath{fill}%
\end{pgfscope}%
\begin{pgfscope}%
\pgfpathrectangle{\pgfqpoint{1.072000in}{0.528000in}}{\pgfqpoint{3.696000in}{3.696000in}}%
\pgfusepath{clip}%
\pgfsetbuttcap%
\pgfsetroundjoin%
\definecolor{currentfill}{rgb}{0.869655,0.379274,0.300941}%
\pgfsetfillcolor{currentfill}%
\pgfsetlinewidth{0.000000pt}%
\definecolor{currentstroke}{rgb}{0.000000,0.000000,0.000000}%
\pgfsetstrokecolor{currentstroke}%
\pgfsetdash{}{0pt}%
\pgfpathmoveto{\pgfqpoint{2.571971in}{2.746325in}}%
\pgfpathlineto{\pgfqpoint{2.618372in}{2.784691in}}%
\pgfpathlineto{\pgfqpoint{2.645389in}{2.921218in}}%
\pgfpathlineto{\pgfqpoint{2.598781in}{2.890102in}}%
\pgfpathlineto{\pgfqpoint{2.571971in}{2.746325in}}%
\pgfpathclose%
\pgfusepath{fill}%
\end{pgfscope}%
\begin{pgfscope}%
\pgfpathrectangle{\pgfqpoint{1.072000in}{0.528000in}}{\pgfqpoint{3.696000in}{3.696000in}}%
\pgfusepath{clip}%
\pgfsetbuttcap%
\pgfsetroundjoin%
\definecolor{currentfill}{rgb}{0.768929,0.189213,0.197965}%
\pgfsetfillcolor{currentfill}%
\pgfsetlinewidth{0.000000pt}%
\definecolor{currentstroke}{rgb}{0.000000,0.000000,0.000000}%
\pgfsetstrokecolor{currentstroke}%
\pgfsetdash{}{0pt}%
\pgfpathmoveto{\pgfqpoint{2.870281in}{3.030245in}}%
\pgfpathlineto{\pgfqpoint{2.917713in}{3.023025in}}%
\pgfpathlineto{\pgfqpoint{2.945281in}{2.969654in}}%
\pgfpathlineto{\pgfqpoint{2.898024in}{2.970247in}}%
\pgfpathlineto{\pgfqpoint{2.870281in}{3.030245in}}%
\pgfpathclose%
\pgfusepath{fill}%
\end{pgfscope}%
\begin{pgfscope}%
\pgfpathrectangle{\pgfqpoint{1.072000in}{0.528000in}}{\pgfqpoint{3.696000in}{3.696000in}}%
\pgfusepath{clip}%
\pgfsetbuttcap%
\pgfsetroundjoin%
\definecolor{currentfill}{rgb}{0.388852,0.516298,0.921373}%
\pgfsetfillcolor{currentfill}%
\pgfsetlinewidth{0.000000pt}%
\definecolor{currentstroke}{rgb}{0.000000,0.000000,0.000000}%
\pgfsetstrokecolor{currentstroke}%
\pgfsetdash{}{0pt}%
\pgfpathmoveto{\pgfqpoint{3.532387in}{1.320331in}}%
\pgfpathlineto{\pgfqpoint{3.582545in}{1.415647in}}%
\pgfpathlineto{\pgfqpoint{3.607383in}{1.370739in}}%
\pgfpathlineto{\pgfqpoint{3.557543in}{1.288614in}}%
\pgfpathlineto{\pgfqpoint{3.532387in}{1.320331in}}%
\pgfpathclose%
\pgfusepath{fill}%
\end{pgfscope}%
\begin{pgfscope}%
\pgfpathrectangle{\pgfqpoint{1.072000in}{0.528000in}}{\pgfqpoint{3.696000in}{3.696000in}}%
\pgfusepath{clip}%
\pgfsetbuttcap%
\pgfsetroundjoin%
\definecolor{currentfill}{rgb}{0.338377,0.452819,0.879317}%
\pgfsetfillcolor{currentfill}%
\pgfsetlinewidth{0.000000pt}%
\definecolor{currentstroke}{rgb}{0.000000,0.000000,0.000000}%
\pgfsetstrokecolor{currentstroke}%
\pgfsetdash{}{0pt}%
\pgfpathmoveto{\pgfqpoint{3.382996in}{1.257122in}}%
\pgfpathlineto{\pgfqpoint{3.432225in}{1.331637in}}%
\pgfpathlineto{\pgfqpoint{3.457628in}{1.283075in}}%
\pgfpathlineto{\pgfqpoint{3.408717in}{1.224739in}}%
\pgfpathlineto{\pgfqpoint{3.382996in}{1.257122in}}%
\pgfpathclose%
\pgfusepath{fill}%
\end{pgfscope}%
\begin{pgfscope}%
\pgfpathrectangle{\pgfqpoint{1.072000in}{0.528000in}}{\pgfqpoint{3.696000in}{3.696000in}}%
\pgfusepath{clip}%
\pgfsetbuttcap%
\pgfsetroundjoin%
\definecolor{currentfill}{rgb}{0.960581,0.762501,0.667964}%
\pgfsetfillcolor{currentfill}%
\pgfsetlinewidth{0.000000pt}%
\definecolor{currentstroke}{rgb}{0.000000,0.000000,0.000000}%
\pgfsetstrokecolor{currentstroke}%
\pgfsetdash{}{0pt}%
\pgfpathmoveto{\pgfqpoint{2.679245in}{2.253286in}}%
\pgfpathlineto{\pgfqpoint{2.726360in}{2.231661in}}%
\pgfpathlineto{\pgfqpoint{2.753032in}{2.436826in}}%
\pgfpathlineto{\pgfqpoint{2.705739in}{2.458476in}}%
\pgfpathlineto{\pgfqpoint{2.679245in}{2.253286in}}%
\pgfpathclose%
\pgfusepath{fill}%
\end{pgfscope}%
\begin{pgfscope}%
\pgfpathrectangle{\pgfqpoint{1.072000in}{0.528000in}}{\pgfqpoint{3.696000in}{3.696000in}}%
\pgfusepath{clip}%
\pgfsetbuttcap%
\pgfsetroundjoin%
\definecolor{currentfill}{rgb}{0.962701,0.628218,0.507636}%
\pgfsetfillcolor{currentfill}%
\pgfsetlinewidth{0.000000pt}%
\definecolor{currentstroke}{rgb}{0.000000,0.000000,0.000000}%
\pgfsetstrokecolor{currentstroke}%
\pgfsetdash{}{0pt}%
\pgfpathmoveto{\pgfqpoint{2.611919in}{2.453996in}}%
\pgfpathlineto{\pgfqpoint{2.658680in}{2.464185in}}%
\pgfpathlineto{\pgfqpoint{2.685301in}{2.652674in}}%
\pgfpathlineto{\pgfqpoint{2.638355in}{2.643012in}}%
\pgfpathlineto{\pgfqpoint{2.611919in}{2.453996in}}%
\pgfpathclose%
\pgfusepath{fill}%
\end{pgfscope}%
\begin{pgfscope}%
\pgfpathrectangle{\pgfqpoint{1.072000in}{0.528000in}}{\pgfqpoint{3.696000in}{3.696000in}}%
\pgfusepath{clip}%
\pgfsetbuttcap%
\pgfsetroundjoin%
\definecolor{currentfill}{rgb}{0.705673,0.015556,0.150233}%
\pgfsetfillcolor{currentfill}%
\pgfsetlinewidth{0.000000pt}%
\definecolor{currentstroke}{rgb}{0.000000,0.000000,0.000000}%
\pgfsetstrokecolor{currentstroke}%
\pgfsetdash{}{0pt}%
\pgfpathmoveto{\pgfqpoint{2.700485in}{3.081627in}}%
\pgfpathlineto{\pgfqpoint{2.747673in}{3.072994in}}%
\pgfpathlineto{\pgfqpoint{2.775627in}{3.076105in}}%
\pgfpathlineto{\pgfqpoint{2.728398in}{3.103398in}}%
\pgfpathlineto{\pgfqpoint{2.700485in}{3.081627in}}%
\pgfpathclose%
\pgfusepath{fill}%
\end{pgfscope}%
\begin{pgfscope}%
\pgfpathrectangle{\pgfqpoint{1.072000in}{0.528000in}}{\pgfqpoint{3.696000in}{3.696000in}}%
\pgfusepath{clip}%
\pgfsetbuttcap%
\pgfsetroundjoin%
\definecolor{currentfill}{rgb}{0.729196,0.086679,0.167240}%
\pgfsetfillcolor{currentfill}%
\pgfsetlinewidth{0.000000pt}%
\definecolor{currentstroke}{rgb}{0.000000,0.000000,0.000000}%
\pgfsetstrokecolor{currentstroke}%
\pgfsetdash{}{0pt}%
\pgfpathmoveto{\pgfqpoint{2.625985in}{3.001631in}}%
\pgfpathlineto{\pgfqpoint{2.672783in}{3.021156in}}%
\pgfpathlineto{\pgfqpoint{2.700485in}{3.081627in}}%
\pgfpathlineto{\pgfqpoint{2.653529in}{3.078264in}}%
\pgfpathlineto{\pgfqpoint{2.625985in}{3.001631in}}%
\pgfpathclose%
\pgfusepath{fill}%
\end{pgfscope}%
\begin{pgfscope}%
\pgfpathrectangle{\pgfqpoint{1.072000in}{0.528000in}}{\pgfqpoint{3.696000in}{3.696000in}}%
\pgfusepath{clip}%
\pgfsetbuttcap%
\pgfsetroundjoin%
\definecolor{currentfill}{rgb}{0.921406,0.491420,0.383408}%
\pgfsetfillcolor{currentfill}%
\pgfsetlinewidth{0.000000pt}%
\definecolor{currentstroke}{rgb}{0.000000,0.000000,0.000000}%
\pgfsetstrokecolor{currentstroke}%
\pgfsetdash{}{0pt}%
\pgfpathmoveto{\pgfqpoint{3.353062in}{2.763308in}}%
\pgfpathlineto{\pgfqpoint{3.399347in}{2.605735in}}%
\pgfpathlineto{\pgfqpoint{3.426294in}{2.667919in}}%
\pgfpathlineto{\pgfqpoint{3.380087in}{2.821807in}}%
\pgfpathlineto{\pgfqpoint{3.353062in}{2.763308in}}%
\pgfpathclose%
\pgfusepath{fill}%
\end{pgfscope}%
\begin{pgfscope}%
\pgfpathrectangle{\pgfqpoint{1.072000in}{0.528000in}}{\pgfqpoint{3.696000in}{3.696000in}}%
\pgfusepath{clip}%
\pgfsetbuttcap%
\pgfsetroundjoin%
\definecolor{currentfill}{rgb}{0.891817,0.851973,0.829085}%
\pgfsetfillcolor{currentfill}%
\pgfsetlinewidth{0.000000pt}%
\definecolor{currentstroke}{rgb}{0.000000,0.000000,0.000000}%
\pgfsetstrokecolor{currentstroke}%
\pgfsetdash{}{0pt}%
\pgfpathmoveto{\pgfqpoint{2.868518in}{2.076157in}}%
\pgfpathlineto{\pgfqpoint{2.915964in}{1.996702in}}%
\pgfpathlineto{\pgfqpoint{2.943236in}{2.192294in}}%
\pgfpathlineto{\pgfqpoint{2.895661in}{2.276428in}}%
\pgfpathlineto{\pgfqpoint{2.868518in}{2.076157in}}%
\pgfpathclose%
\pgfusepath{fill}%
\end{pgfscope}%
\begin{pgfscope}%
\pgfpathrectangle{\pgfqpoint{1.072000in}{0.528000in}}{\pgfqpoint{3.696000in}{3.696000in}}%
\pgfusepath{clip}%
\pgfsetbuttcap%
\pgfsetroundjoin%
\definecolor{currentfill}{rgb}{0.304174,0.406945,0.845263}%
\pgfsetfillcolor{currentfill}%
\pgfsetlinewidth{0.000000pt}%
\definecolor{currentstroke}{rgb}{0.000000,0.000000,0.000000}%
\pgfsetstrokecolor{currentstroke}%
\pgfsetdash{}{0pt}%
\pgfpathmoveto{\pgfqpoint{2.791373in}{1.228860in}}%
\pgfpathlineto{\pgfqpoint{2.838583in}{1.251480in}}%
\pgfpathlineto{\pgfqpoint{2.866568in}{1.188596in}}%
\pgfpathlineto{\pgfqpoint{2.819463in}{1.178031in}}%
\pgfpathlineto{\pgfqpoint{2.791373in}{1.228860in}}%
\pgfpathclose%
\pgfusepath{fill}%
\end{pgfscope}%
\begin{pgfscope}%
\pgfpathrectangle{\pgfqpoint{1.072000in}{0.528000in}}{\pgfqpoint{3.696000in}{3.696000in}}%
\pgfusepath{clip}%
\pgfsetbuttcap%
\pgfsetroundjoin%
\definecolor{currentfill}{rgb}{0.294718,0.393542,0.834384}%
\pgfsetfillcolor{currentfill}%
\pgfsetlinewidth{0.000000pt}%
\definecolor{currentstroke}{rgb}{0.000000,0.000000,0.000000}%
\pgfsetstrokecolor{currentstroke}%
\pgfsetdash{}{0pt}%
\pgfpathmoveto{\pgfqpoint{2.913842in}{1.208609in}}%
\pgfpathlineto{\pgfqpoint{2.961326in}{1.239545in}}%
\pgfpathlineto{\pgfqpoint{2.988739in}{1.182659in}}%
\pgfpathlineto{\pgfqpoint{2.941387in}{1.166598in}}%
\pgfpathlineto{\pgfqpoint{2.913842in}{1.208609in}}%
\pgfpathclose%
\pgfusepath{fill}%
\end{pgfscope}%
\begin{pgfscope}%
\pgfpathrectangle{\pgfqpoint{1.072000in}{0.528000in}}{\pgfqpoint{3.696000in}{3.696000in}}%
\pgfusepath{clip}%
\pgfsetbuttcap%
\pgfsetroundjoin%
\definecolor{currentfill}{rgb}{0.921406,0.491420,0.383408}%
\pgfsetfillcolor{currentfill}%
\pgfsetlinewidth{0.000000pt}%
\definecolor{currentstroke}{rgb}{0.000000,0.000000,0.000000}%
\pgfsetstrokecolor{currentstroke}%
\pgfsetdash{}{0pt}%
\pgfpathmoveto{\pgfqpoint{2.591755in}{2.617521in}}%
\pgfpathlineto{\pgfqpoint{2.638355in}{2.643012in}}%
\pgfpathlineto{\pgfqpoint{2.665163in}{2.806897in}}%
\pgfpathlineto{\pgfqpoint{2.618372in}{2.784691in}}%
\pgfpathlineto{\pgfqpoint{2.591755in}{2.617521in}}%
\pgfpathclose%
\pgfusepath{fill}%
\end{pgfscope}%
\begin{pgfscope}%
\pgfpathrectangle{\pgfqpoint{1.072000in}{0.528000in}}{\pgfqpoint{3.696000in}{3.696000in}}%
\pgfusepath{clip}%
\pgfsetbuttcap%
\pgfsetroundjoin%
\definecolor{currentfill}{rgb}{0.960581,0.762501,0.667964}%
\pgfsetfillcolor{currentfill}%
\pgfsetlinewidth{0.000000pt}%
\definecolor{currentstroke}{rgb}{0.000000,0.000000,0.000000}%
\pgfsetstrokecolor{currentstroke}%
\pgfsetdash{}{0pt}%
\pgfpathmoveto{\pgfqpoint{3.243050in}{2.358464in}}%
\pgfpathlineto{\pgfqpoint{3.289570in}{2.196185in}}%
\pgfpathlineto{\pgfqpoint{3.317225in}{2.321161in}}%
\pgfpathlineto{\pgfqpoint{3.270764in}{2.485637in}}%
\pgfpathlineto{\pgfqpoint{3.243050in}{2.358464in}}%
\pgfpathclose%
\pgfusepath{fill}%
\end{pgfscope}%
\begin{pgfscope}%
\pgfpathrectangle{\pgfqpoint{1.072000in}{0.528000in}}{\pgfqpoint{3.696000in}{3.696000in}}%
\pgfusepath{clip}%
\pgfsetbuttcap%
\pgfsetroundjoin%
\definecolor{currentfill}{rgb}{0.790562,0.231397,0.216242}%
\pgfsetfillcolor{currentfill}%
\pgfsetlinewidth{0.000000pt}%
\definecolor{currentstroke}{rgb}{0.000000,0.000000,0.000000}%
\pgfsetstrokecolor{currentstroke}%
\pgfsetdash{}{0pt}%
\pgfpathmoveto{\pgfqpoint{2.598781in}{2.890102in}}%
\pgfpathlineto{\pgfqpoint{2.645389in}{2.921218in}}%
\pgfpathlineto{\pgfqpoint{2.672783in}{3.021156in}}%
\pgfpathlineto{\pgfqpoint{2.625985in}{3.001631in}}%
\pgfpathlineto{\pgfqpoint{2.598781in}{2.890102in}}%
\pgfpathclose%
\pgfusepath{fill}%
\end{pgfscope}%
\begin{pgfscope}%
\pgfpathrectangle{\pgfqpoint{1.072000in}{0.528000in}}{\pgfqpoint{3.696000in}{3.696000in}}%
\pgfusepath{clip}%
\pgfsetbuttcap%
\pgfsetroundjoin%
\definecolor{currentfill}{rgb}{0.333490,0.446265,0.874452}%
\pgfsetfillcolor{currentfill}%
\pgfsetlinewidth{0.000000pt}%
\definecolor{currentstroke}{rgb}{0.000000,0.000000,0.000000}%
\pgfsetstrokecolor{currentstroke}%
\pgfsetdash{}{0pt}%
\pgfpathmoveto{\pgfqpoint{3.308403in}{1.241237in}}%
\pgfpathlineto{\pgfqpoint{3.357299in}{1.308828in}}%
\pgfpathlineto{\pgfqpoint{3.382996in}{1.257122in}}%
\pgfpathlineto{\pgfqpoint{3.334398in}{1.206344in}}%
\pgfpathlineto{\pgfqpoint{3.308403in}{1.241237in}}%
\pgfpathclose%
\pgfusepath{fill}%
\end{pgfscope}%
\begin{pgfscope}%
\pgfpathrectangle{\pgfqpoint{1.072000in}{0.528000in}}{\pgfqpoint{3.696000in}{3.696000in}}%
\pgfusepath{clip}%
\pgfsetbuttcap%
\pgfsetroundjoin%
\definecolor{currentfill}{rgb}{0.705673,0.015556,0.150233}%
\pgfsetfillcolor{currentfill}%
\pgfsetlinewidth{0.000000pt}%
\definecolor{currentstroke}{rgb}{0.000000,0.000000,0.000000}%
\pgfsetstrokecolor{currentstroke}%
\pgfsetdash{}{0pt}%
\pgfpathmoveto{\pgfqpoint{3.264997in}{3.080613in}}%
\pgfpathlineto{\pgfqpoint{3.312772in}{3.053718in}}%
\pgfpathlineto{\pgfqpoint{3.339484in}{3.078706in}}%
\pgfpathlineto{\pgfqpoint{3.291751in}{3.102008in}}%
\pgfpathlineto{\pgfqpoint{3.264997in}{3.080613in}}%
\pgfpathclose%
\pgfusepath{fill}%
\end{pgfscope}%
\begin{pgfscope}%
\pgfpathrectangle{\pgfqpoint{1.072000in}{0.528000in}}{\pgfqpoint{3.696000in}{3.696000in}}%
\pgfusepath{clip}%
\pgfsetbuttcap%
\pgfsetroundjoin%
\definecolor{currentfill}{rgb}{0.891817,0.851973,0.829085}%
\pgfsetfillcolor{currentfill}%
\pgfsetlinewidth{0.000000pt}%
\definecolor{currentstroke}{rgb}{0.000000,0.000000,0.000000}%
\pgfsetstrokecolor{currentstroke}%
\pgfsetdash{}{0pt}%
\pgfpathmoveto{\pgfqpoint{2.990722in}{2.094660in}}%
\pgfpathlineto{\pgfqpoint{3.038066in}{1.985792in}}%
\pgfpathlineto{\pgfqpoint{3.065656in}{2.165475in}}%
\pgfpathlineto{\pgfqpoint{3.018246in}{2.280577in}}%
\pgfpathlineto{\pgfqpoint{2.990722in}{2.094660in}}%
\pgfpathclose%
\pgfusepath{fill}%
\end{pgfscope}%
\begin{pgfscope}%
\pgfpathrectangle{\pgfqpoint{1.072000in}{0.528000in}}{\pgfqpoint{3.696000in}{3.696000in}}%
\pgfusepath{clip}%
\pgfsetbuttcap%
\pgfsetroundjoin%
\definecolor{currentfill}{rgb}{0.928116,0.822197,0.765141}%
\pgfsetfillcolor{currentfill}%
\pgfsetlinewidth{0.000000pt}%
\definecolor{currentstroke}{rgb}{0.000000,0.000000,0.000000}%
\pgfsetstrokecolor{currentstroke}%
\pgfsetdash{}{0pt}%
\pgfpathmoveto{\pgfqpoint{3.140577in}{2.206512in}}%
\pgfpathlineto{\pgfqpoint{3.187517in}{2.064211in}}%
\pgfpathlineto{\pgfqpoint{3.215282in}{2.216731in}}%
\pgfpathlineto{\pgfqpoint{3.168359in}{2.364245in}}%
\pgfpathlineto{\pgfqpoint{3.140577in}{2.206512in}}%
\pgfpathclose%
\pgfusepath{fill}%
\end{pgfscope}%
\begin{pgfscope}%
\pgfpathrectangle{\pgfqpoint{1.072000in}{0.528000in}}{\pgfqpoint{3.696000in}{3.696000in}}%
\pgfusepath{clip}%
\pgfsetbuttcap%
\pgfsetroundjoin%
\definecolor{currentfill}{rgb}{0.740957,0.122240,0.175744}%
\pgfsetfillcolor{currentfill}%
\pgfsetlinewidth{0.000000pt}%
\definecolor{currentstroke}{rgb}{0.000000,0.000000,0.000000}%
\pgfsetstrokecolor{currentstroke}%
\pgfsetdash{}{0pt}%
\pgfpathmoveto{\pgfqpoint{3.312772in}{3.053718in}}%
\pgfpathlineto{\pgfqpoint{3.360110in}{2.980680in}}%
\pgfpathlineto{\pgfqpoint{3.386745in}{3.008307in}}%
\pgfpathlineto{\pgfqpoint{3.339484in}{3.078706in}}%
\pgfpathlineto{\pgfqpoint{3.312772in}{3.053718in}}%
\pgfpathclose%
\pgfusepath{fill}%
\end{pgfscope}%
\begin{pgfscope}%
\pgfpathrectangle{\pgfqpoint{1.072000in}{0.528000in}}{\pgfqpoint{3.696000in}{3.696000in}}%
\pgfusepath{clip}%
\pgfsetbuttcap%
\pgfsetroundjoin%
\definecolor{currentfill}{rgb}{0.956371,0.775144,0.686416}%
\pgfsetfillcolor{currentfill}%
\pgfsetlinewidth{0.000000pt}%
\definecolor{currentstroke}{rgb}{0.000000,0.000000,0.000000}%
\pgfsetstrokecolor{currentstroke}%
\pgfsetdash{}{0pt}%
\pgfpathmoveto{\pgfqpoint{2.726360in}{2.231661in}}%
\pgfpathlineto{\pgfqpoint{2.773654in}{2.194628in}}%
\pgfpathlineto{\pgfqpoint{2.800493in}{2.399201in}}%
\pgfpathlineto{\pgfqpoint{2.753032in}{2.436826in}}%
\pgfpathlineto{\pgfqpoint{2.726360in}{2.231661in}}%
\pgfpathclose%
\pgfusepath{fill}%
\end{pgfscope}%
\begin{pgfscope}%
\pgfpathrectangle{\pgfqpoint{1.072000in}{0.528000in}}{\pgfqpoint{3.696000in}{3.696000in}}%
\pgfusepath{clip}%
\pgfsetbuttcap%
\pgfsetroundjoin%
\definecolor{currentfill}{rgb}{0.928116,0.822197,0.765141}%
\pgfsetfillcolor{currentfill}%
\pgfsetlinewidth{0.000000pt}%
\definecolor{currentstroke}{rgb}{0.000000,0.000000,0.000000}%
\pgfsetstrokecolor{currentstroke}%
\pgfsetdash{}{0pt}%
\pgfpathmoveto{\pgfqpoint{2.821061in}{2.142548in}}%
\pgfpathlineto{\pgfqpoint{2.868518in}{2.076157in}}%
\pgfpathlineto{\pgfqpoint{2.895661in}{2.276428in}}%
\pgfpathlineto{\pgfqpoint{2.848059in}{2.345643in}}%
\pgfpathlineto{\pgfqpoint{2.821061in}{2.142548in}}%
\pgfpathclose%
\pgfusepath{fill}%
\end{pgfscope}%
\begin{pgfscope}%
\pgfpathrectangle{\pgfqpoint{1.072000in}{0.528000in}}{\pgfqpoint{3.696000in}{3.696000in}}%
\pgfusepath{clip}%
\pgfsetbuttcap%
\pgfsetroundjoin%
\definecolor{currentfill}{rgb}{0.939254,0.539581,0.423900}%
\pgfsetfillcolor{currentfill}%
\pgfsetlinewidth{0.000000pt}%
\definecolor{currentstroke}{rgb}{0.000000,0.000000,0.000000}%
\pgfsetstrokecolor{currentstroke}%
\pgfsetdash{}{0pt}%
\pgfpathmoveto{\pgfqpoint{3.325813in}{2.688373in}}%
\pgfpathlineto{\pgfqpoint{3.372161in}{2.527122in}}%
\pgfpathlineto{\pgfqpoint{3.399347in}{2.605735in}}%
\pgfpathlineto{\pgfqpoint{3.353062in}{2.763308in}}%
\pgfpathlineto{\pgfqpoint{3.325813in}{2.688373in}}%
\pgfpathclose%
\pgfusepath{fill}%
\end{pgfscope}%
\begin{pgfscope}%
\pgfpathrectangle{\pgfqpoint{1.072000in}{0.528000in}}{\pgfqpoint{3.696000in}{3.696000in}}%
\pgfusepath{clip}%
\pgfsetbuttcap%
\pgfsetroundjoin%
\definecolor{currentfill}{rgb}{0.947345,0.794696,0.716991}%
\pgfsetfillcolor{currentfill}%
\pgfsetlinewidth{0.000000pt}%
\definecolor{currentstroke}{rgb}{0.000000,0.000000,0.000000}%
\pgfsetstrokecolor{currentstroke}%
\pgfsetdash{}{0pt}%
\pgfpathmoveto{\pgfqpoint{2.773654in}{2.194628in}}%
\pgfpathlineto{\pgfqpoint{2.821061in}{2.142548in}}%
\pgfpathlineto{\pgfqpoint{2.848059in}{2.345643in}}%
\pgfpathlineto{\pgfqpoint{2.800493in}{2.399201in}}%
\pgfpathlineto{\pgfqpoint{2.773654in}{2.194628in}}%
\pgfpathclose%
\pgfusepath{fill}%
\end{pgfscope}%
\begin{pgfscope}%
\pgfpathrectangle{\pgfqpoint{1.072000in}{0.528000in}}{\pgfqpoint{3.696000in}{3.696000in}}%
\pgfusepath{clip}%
\pgfsetbuttcap%
\pgfsetroundjoin%
\definecolor{currentfill}{rgb}{0.919376,0.831273,0.782874}%
\pgfsetfillcolor{currentfill}%
\pgfsetlinewidth{0.000000pt}%
\definecolor{currentstroke}{rgb}{0.000000,0.000000,0.000000}%
\pgfsetstrokecolor{currentstroke}%
\pgfsetdash{}{0pt}%
\pgfpathmoveto{\pgfqpoint{3.065656in}{2.165475in}}%
\pgfpathlineto{\pgfqpoint{3.112853in}{2.039271in}}%
\pgfpathlineto{\pgfqpoint{3.140577in}{2.206512in}}%
\pgfpathlineto{\pgfqpoint{3.093356in}{2.338227in}}%
\pgfpathlineto{\pgfqpoint{3.065656in}{2.165475in}}%
\pgfpathclose%
\pgfusepath{fill}%
\end{pgfscope}%
\begin{pgfscope}%
\pgfpathrectangle{\pgfqpoint{1.072000in}{0.528000in}}{\pgfqpoint{3.696000in}{3.696000in}}%
\pgfusepath{clip}%
\pgfsetbuttcap%
\pgfsetroundjoin%
\definecolor{currentfill}{rgb}{0.309060,0.413498,0.850128}%
\pgfsetfillcolor{currentfill}%
\pgfsetlinewidth{0.000000pt}%
\definecolor{currentstroke}{rgb}{0.000000,0.000000,0.000000}%
\pgfsetstrokecolor{currentstroke}%
\pgfsetdash{}{0pt}%
\pgfpathmoveto{\pgfqpoint{3.036317in}{1.211371in}}%
\pgfpathlineto{\pgfqpoint{3.084181in}{1.254095in}}%
\pgfpathlineto{\pgfqpoint{3.111055in}{1.196264in}}%
\pgfpathlineto{\pgfqpoint{3.063376in}{1.169916in}}%
\pgfpathlineto{\pgfqpoint{3.036317in}{1.211371in}}%
\pgfpathclose%
\pgfusepath{fill}%
\end{pgfscope}%
\begin{pgfscope}%
\pgfpathrectangle{\pgfqpoint{1.072000in}{0.528000in}}{\pgfqpoint{3.696000in}{3.696000in}}%
\pgfusepath{clip}%
\pgfsetbuttcap%
\pgfsetroundjoin%
\definecolor{currentfill}{rgb}{0.969289,0.684982,0.568975}%
\pgfsetfillcolor{currentfill}%
\pgfsetlinewidth{0.000000pt}%
\definecolor{currentstroke}{rgb}{0.000000,0.000000,0.000000}%
\pgfsetstrokecolor{currentstroke}%
\pgfsetdash{}{0pt}%
\pgfpathmoveto{\pgfqpoint{3.270764in}{2.485637in}}%
\pgfpathlineto{\pgfqpoint{3.317225in}{2.321161in}}%
\pgfpathlineto{\pgfqpoint{3.344772in}{2.432035in}}%
\pgfpathlineto{\pgfqpoint{3.298368in}{2.595912in}}%
\pgfpathlineto{\pgfqpoint{3.270764in}{2.485637in}}%
\pgfpathclose%
\pgfusepath{fill}%
\end{pgfscope}%
\begin{pgfscope}%
\pgfpathrectangle{\pgfqpoint{1.072000in}{0.528000in}}{\pgfqpoint{3.696000in}{3.696000in}}%
\pgfusepath{clip}%
\pgfsetbuttcap%
\pgfsetroundjoin%
\definecolor{currentfill}{rgb}{0.711554,0.033337,0.154485}%
\pgfsetfillcolor{currentfill}%
\pgfsetlinewidth{0.000000pt}%
\definecolor{currentstroke}{rgb}{0.000000,0.000000,0.000000}%
\pgfsetstrokecolor{currentstroke}%
\pgfsetdash{}{0pt}%
\pgfpathmoveto{\pgfqpoint{2.747673in}{3.072994in}}%
\pgfpathlineto{\pgfqpoint{2.795010in}{3.059097in}}%
\pgfpathlineto{\pgfqpoint{2.822929in}{3.049005in}}%
\pgfpathlineto{\pgfqpoint{2.775627in}{3.076105in}}%
\pgfpathlineto{\pgfqpoint{2.747673in}{3.072994in}}%
\pgfpathclose%
\pgfusepath{fill}%
\end{pgfscope}%
\begin{pgfscope}%
\pgfpathrectangle{\pgfqpoint{1.072000in}{0.528000in}}{\pgfqpoint{3.696000in}{3.696000in}}%
\pgfusepath{clip}%
\pgfsetbuttcap%
\pgfsetroundjoin%
\definecolor{currentfill}{rgb}{0.711554,0.033337,0.154485}%
\pgfsetfillcolor{currentfill}%
\pgfsetlinewidth{0.000000pt}%
\definecolor{currentstroke}{rgb}{0.000000,0.000000,0.000000}%
\pgfsetstrokecolor{currentstroke}%
\pgfsetdash{}{0pt}%
\pgfpathmoveto{\pgfqpoint{3.190147in}{3.052755in}}%
\pgfpathlineto{\pgfqpoint{3.238131in}{3.059092in}}%
\pgfpathlineto{\pgfqpoint{3.264997in}{3.080613in}}%
\pgfpathlineto{\pgfqpoint{3.216973in}{3.061020in}}%
\pgfpathlineto{\pgfqpoint{3.190147in}{3.052755in}}%
\pgfpathclose%
\pgfusepath{fill}%
\end{pgfscope}%
\begin{pgfscope}%
\pgfpathrectangle{\pgfqpoint{1.072000in}{0.528000in}}{\pgfqpoint{3.696000in}{3.696000in}}%
\pgfusepath{clip}%
\pgfsetbuttcap%
\pgfsetroundjoin%
\definecolor{currentfill}{rgb}{0.959385,0.610306,0.489382}%
\pgfsetfillcolor{currentfill}%
\pgfsetlinewidth{0.000000pt}%
\definecolor{currentstroke}{rgb}{0.000000,0.000000,0.000000}%
\pgfsetstrokecolor{currentstroke}%
\pgfsetdash{}{0pt}%
\pgfpathmoveto{\pgfqpoint{3.298368in}{2.595912in}}%
\pgfpathlineto{\pgfqpoint{3.344772in}{2.432035in}}%
\pgfpathlineto{\pgfqpoint{3.372161in}{2.527122in}}%
\pgfpathlineto{\pgfqpoint{3.325813in}{2.688373in}}%
\pgfpathlineto{\pgfqpoint{3.298368in}{2.595912in}}%
\pgfpathclose%
\pgfusepath{fill}%
\end{pgfscope}%
\begin{pgfscope}%
\pgfpathrectangle{\pgfqpoint{1.072000in}{0.528000in}}{\pgfqpoint{3.696000in}{3.696000in}}%
\pgfusepath{clip}%
\pgfsetbuttcap%
\pgfsetroundjoin%
\definecolor{currentfill}{rgb}{0.333490,0.446265,0.874452}%
\pgfsetfillcolor{currentfill}%
\pgfsetlinewidth{0.000000pt}%
\definecolor{currentstroke}{rgb}{0.000000,0.000000,0.000000}%
\pgfsetstrokecolor{currentstroke}%
\pgfsetdash{}{0pt}%
\pgfpathmoveto{\pgfqpoint{3.233778in}{1.234998in}}%
\pgfpathlineto{\pgfqpoint{3.282404in}{1.297772in}}%
\pgfpathlineto{\pgfqpoint{3.308403in}{1.241237in}}%
\pgfpathlineto{\pgfqpoint{3.260047in}{1.195283in}}%
\pgfpathlineto{\pgfqpoint{3.233778in}{1.234998in}}%
\pgfpathclose%
\pgfusepath{fill}%
\end{pgfscope}%
\begin{pgfscope}%
\pgfpathrectangle{\pgfqpoint{1.072000in}{0.528000in}}{\pgfqpoint{3.696000in}{3.696000in}}%
\pgfusepath{clip}%
\pgfsetbuttcap%
\pgfsetroundjoin%
\definecolor{currentfill}{rgb}{0.820401,0.286765,0.245160}%
\pgfsetfillcolor{currentfill}%
\pgfsetlinewidth{0.000000pt}%
\definecolor{currentstroke}{rgb}{0.000000,0.000000,0.000000}%
\pgfsetstrokecolor{currentstroke}%
\pgfsetdash{}{0pt}%
\pgfpathmoveto{\pgfqpoint{3.333261in}{2.942059in}}%
\pgfpathlineto{\pgfqpoint{3.380087in}{2.821807in}}%
\pgfpathlineto{\pgfqpoint{3.406870in}{2.865235in}}%
\pgfpathlineto{\pgfqpoint{3.360110in}{2.980680in}}%
\pgfpathlineto{\pgfqpoint{3.333261in}{2.942059in}}%
\pgfpathclose%
\pgfusepath{fill}%
\end{pgfscope}%
\begin{pgfscope}%
\pgfpathrectangle{\pgfqpoint{1.072000in}{0.528000in}}{\pgfqpoint{3.696000in}{3.696000in}}%
\pgfusepath{clip}%
\pgfsetbuttcap%
\pgfsetroundjoin%
\definecolor{currentfill}{rgb}{0.758112,0.168122,0.188827}%
\pgfsetfillcolor{currentfill}%
\pgfsetlinewidth{0.000000pt}%
\definecolor{currentstroke}{rgb}{0.000000,0.000000,0.000000}%
\pgfsetstrokecolor{currentstroke}%
\pgfsetdash{}{0pt}%
\pgfpathmoveto{\pgfqpoint{2.917713in}{3.023025in}}%
\pgfpathlineto{\pgfqpoint{2.965273in}{3.024192in}}%
\pgfpathlineto{\pgfqpoint{2.992695in}{2.988501in}}%
\pgfpathlineto{\pgfqpoint{2.945281in}{2.969654in}}%
\pgfpathlineto{\pgfqpoint{2.917713in}{3.023025in}}%
\pgfpathclose%
\pgfusepath{fill}%
\end{pgfscope}%
\begin{pgfscope}%
\pgfpathrectangle{\pgfqpoint{1.072000in}{0.528000in}}{\pgfqpoint{3.696000in}{3.696000in}}%
\pgfusepath{clip}%
\pgfsetbuttcap%
\pgfsetroundjoin%
\definecolor{currentfill}{rgb}{0.729196,0.086679,0.167240}%
\pgfsetfillcolor{currentfill}%
\pgfsetlinewidth{0.000000pt}%
\definecolor{currentstroke}{rgb}{0.000000,0.000000,0.000000}%
\pgfsetstrokecolor{currentstroke}%
\pgfsetdash{}{0pt}%
\pgfpathmoveto{\pgfqpoint{3.115237in}{3.023276in}}%
\pgfpathlineto{\pgfqpoint{3.163189in}{3.044686in}}%
\pgfpathlineto{\pgfqpoint{3.190147in}{3.052755in}}%
\pgfpathlineto{\pgfqpoint{3.142138in}{3.007551in}}%
\pgfpathlineto{\pgfqpoint{3.115237in}{3.023276in}}%
\pgfpathclose%
\pgfusepath{fill}%
\end{pgfscope}%
\begin{pgfscope}%
\pgfpathrectangle{\pgfqpoint{1.072000in}{0.528000in}}{\pgfqpoint{3.696000in}{3.696000in}}%
\pgfusepath{clip}%
\pgfsetbuttcap%
\pgfsetroundjoin%
\definecolor{currentfill}{rgb}{0.959385,0.610306,0.489382}%
\pgfsetfillcolor{currentfill}%
\pgfsetlinewidth{0.000000pt}%
\definecolor{currentstroke}{rgb}{0.000000,0.000000,0.000000}%
\pgfsetstrokecolor{currentstroke}%
\pgfsetdash{}{0pt}%
\pgfpathmoveto{\pgfqpoint{2.658680in}{2.464185in}}%
\pgfpathlineto{\pgfqpoint{2.705739in}{2.458476in}}%
\pgfpathlineto{\pgfqpoint{2.732527in}{2.646858in}}%
\pgfpathlineto{\pgfqpoint{2.685301in}{2.652674in}}%
\pgfpathlineto{\pgfqpoint{2.658680in}{2.464185in}}%
\pgfpathclose%
\pgfusepath{fill}%
\end{pgfscope}%
\begin{pgfscope}%
\pgfpathrectangle{\pgfqpoint{1.072000in}{0.528000in}}{\pgfqpoint{3.696000in}{3.696000in}}%
\pgfusepath{clip}%
\pgfsetbuttcap%
\pgfsetroundjoin%
\definecolor{currentfill}{rgb}{0.711554,0.033337,0.154485}%
\pgfsetfillcolor{currentfill}%
\pgfsetlinewidth{0.000000pt}%
\definecolor{currentstroke}{rgb}{0.000000,0.000000,0.000000}%
\pgfsetstrokecolor{currentstroke}%
\pgfsetdash{}{0pt}%
\pgfpathmoveto{\pgfqpoint{2.672783in}{3.021156in}}%
\pgfpathlineto{\pgfqpoint{2.719878in}{3.026699in}}%
\pgfpathlineto{\pgfqpoint{2.747673in}{3.072994in}}%
\pgfpathlineto{\pgfqpoint{2.700485in}{3.081627in}}%
\pgfpathlineto{\pgfqpoint{2.672783in}{3.021156in}}%
\pgfpathclose%
\pgfusepath{fill}%
\end{pgfscope}%
\begin{pgfscope}%
\pgfpathrectangle{\pgfqpoint{1.072000in}{0.528000in}}{\pgfqpoint{3.696000in}{3.696000in}}%
\pgfusepath{clip}%
\pgfsetbuttcap%
\pgfsetroundjoin%
\definecolor{currentfill}{rgb}{0.839365,0.321856,0.264924}%
\pgfsetfillcolor{currentfill}%
\pgfsetlinewidth{0.000000pt}%
\definecolor{currentstroke}{rgb}{0.000000,0.000000,0.000000}%
\pgfsetstrokecolor{currentstroke}%
\pgfsetdash{}{0pt}%
\pgfpathmoveto{\pgfqpoint{2.618372in}{2.784691in}}%
\pgfpathlineto{\pgfqpoint{2.665163in}{2.806897in}}%
\pgfpathlineto{\pgfqpoint{2.692350in}{2.936908in}}%
\pgfpathlineto{\pgfqpoint{2.645389in}{2.921218in}}%
\pgfpathlineto{\pgfqpoint{2.618372in}{2.784691in}}%
\pgfpathclose%
\pgfusepath{fill}%
\end{pgfscope}%
\begin{pgfscope}%
\pgfpathrectangle{\pgfqpoint{1.072000in}{0.528000in}}{\pgfqpoint{3.696000in}{3.696000in}}%
\pgfusepath{clip}%
\pgfsetbuttcap%
\pgfsetroundjoin%
\definecolor{currentfill}{rgb}{0.740957,0.122240,0.175744}%
\pgfsetfillcolor{currentfill}%
\pgfsetlinewidth{0.000000pt}%
\definecolor{currentstroke}{rgb}{0.000000,0.000000,0.000000}%
\pgfsetstrokecolor{currentstroke}%
\pgfsetdash{}{0pt}%
\pgfpathmoveto{\pgfqpoint{3.040322in}{3.013743in}}%
\pgfpathlineto{\pgfqpoint{3.088138in}{3.030631in}}%
\pgfpathlineto{\pgfqpoint{3.115237in}{3.023276in}}%
\pgfpathlineto{\pgfqpoint{3.067425in}{2.977648in}}%
\pgfpathlineto{\pgfqpoint{3.040322in}{3.013743in}}%
\pgfpathclose%
\pgfusepath{fill}%
\end{pgfscope}%
\begin{pgfscope}%
\pgfpathrectangle{\pgfqpoint{1.072000in}{0.528000in}}{\pgfqpoint{3.696000in}{3.696000in}}%
\pgfusepath{clip}%
\pgfsetbuttcap%
\pgfsetroundjoin%
\definecolor{currentfill}{rgb}{0.940879,0.805596,0.735167}%
\pgfsetfillcolor{currentfill}%
\pgfsetlinewidth{0.000000pt}%
\definecolor{currentstroke}{rgb}{0.000000,0.000000,0.000000}%
\pgfsetstrokecolor{currentstroke}%
\pgfsetdash{}{0pt}%
\pgfpathmoveto{\pgfqpoint{2.943236in}{2.192294in}}%
\pgfpathlineto{\pgfqpoint{2.990722in}{2.094660in}}%
\pgfpathlineto{\pgfqpoint{3.018246in}{2.280577in}}%
\pgfpathlineto{\pgfqpoint{2.970669in}{2.381722in}}%
\pgfpathlineto{\pgfqpoint{2.943236in}{2.192294in}}%
\pgfpathclose%
\pgfusepath{fill}%
\end{pgfscope}%
\begin{pgfscope}%
\pgfpathrectangle{\pgfqpoint{1.072000in}{0.528000in}}{\pgfqpoint{3.696000in}{3.696000in}}%
\pgfusepath{clip}%
\pgfsetbuttcap%
\pgfsetroundjoin%
\definecolor{currentfill}{rgb}{0.717435,0.051118,0.158737}%
\pgfsetfillcolor{currentfill}%
\pgfsetlinewidth{0.000000pt}%
\definecolor{currentstroke}{rgb}{0.000000,0.000000,0.000000}%
\pgfsetstrokecolor{currentstroke}%
\pgfsetdash{}{0pt}%
\pgfpathmoveto{\pgfqpoint{2.795010in}{3.059097in}}%
\pgfpathlineto{\pgfqpoint{2.842458in}{3.045130in}}%
\pgfpathlineto{\pgfqpoint{2.870281in}{3.030245in}}%
\pgfpathlineto{\pgfqpoint{2.822929in}{3.049005in}}%
\pgfpathlineto{\pgfqpoint{2.795010in}{3.059097in}}%
\pgfpathclose%
\pgfusepath{fill}%
\end{pgfscope}%
\begin{pgfscope}%
\pgfpathrectangle{\pgfqpoint{1.072000in}{0.528000in}}{\pgfqpoint{3.696000in}{3.696000in}}%
\pgfusepath{clip}%
\pgfsetbuttcap%
\pgfsetroundjoin%
\definecolor{currentfill}{rgb}{0.965899,0.740142,0.637058}%
\pgfsetfillcolor{currentfill}%
\pgfsetlinewidth{0.000000pt}%
\definecolor{currentstroke}{rgb}{0.000000,0.000000,0.000000}%
\pgfsetstrokecolor{currentstroke}%
\pgfsetdash{}{0pt}%
\pgfpathmoveto{\pgfqpoint{3.168359in}{2.364245in}}%
\pgfpathlineto{\pgfqpoint{3.215282in}{2.216731in}}%
\pgfpathlineto{\pgfqpoint{3.243050in}{2.358464in}}%
\pgfpathlineto{\pgfqpoint{3.196144in}{2.507369in}}%
\pgfpathlineto{\pgfqpoint{3.168359in}{2.364245in}}%
\pgfpathclose%
\pgfusepath{fill}%
\end{pgfscope}%
\begin{pgfscope}%
\pgfpathrectangle{\pgfqpoint{1.072000in}{0.528000in}}{\pgfqpoint{3.696000in}{3.696000in}}%
\pgfusepath{clip}%
\pgfsetbuttcap%
\pgfsetroundjoin%
\definecolor{currentfill}{rgb}{0.905783,0.455186,0.355336}%
\pgfsetfillcolor{currentfill}%
\pgfsetlinewidth{0.000000pt}%
\definecolor{currentstroke}{rgb}{0.000000,0.000000,0.000000}%
\pgfsetstrokecolor{currentstroke}%
\pgfsetdash{}{0pt}%
\pgfpathmoveto{\pgfqpoint{2.638355in}{2.643012in}}%
\pgfpathlineto{\pgfqpoint{2.685301in}{2.652674in}}%
\pgfpathlineto{\pgfqpoint{2.712271in}{2.814207in}}%
\pgfpathlineto{\pgfqpoint{2.665163in}{2.806897in}}%
\pgfpathlineto{\pgfqpoint{2.638355in}{2.643012in}}%
\pgfpathclose%
\pgfusepath{fill}%
\end{pgfscope}%
\begin{pgfscope}%
\pgfpathrectangle{\pgfqpoint{1.072000in}{0.528000in}}{\pgfqpoint{3.696000in}{3.696000in}}%
\pgfusepath{clip}%
\pgfsetbuttcap%
\pgfsetroundjoin%
\definecolor{currentfill}{rgb}{0.338377,0.452819,0.879317}%
\pgfsetfillcolor{currentfill}%
\pgfsetlinewidth{0.000000pt}%
\definecolor{currentstroke}{rgb}{0.000000,0.000000,0.000000}%
\pgfsetstrokecolor{currentstroke}%
\pgfsetdash{}{0pt}%
\pgfpathmoveto{\pgfqpoint{3.159059in}{1.238861in}}%
\pgfpathlineto{\pgfqpoint{3.207463in}{1.298490in}}%
\pgfpathlineto{\pgfqpoint{3.233778in}{1.234998in}}%
\pgfpathlineto{\pgfqpoint{3.185615in}{1.191567in}}%
\pgfpathlineto{\pgfqpoint{3.159059in}{1.238861in}}%
\pgfpathclose%
\pgfusepath{fill}%
\end{pgfscope}%
\begin{pgfscope}%
\pgfpathrectangle{\pgfqpoint{1.072000in}{0.528000in}}{\pgfqpoint{3.696000in}{3.696000in}}%
\pgfusepath{clip}%
\pgfsetbuttcap%
\pgfsetroundjoin%
\definecolor{currentfill}{rgb}{0.763520,0.178667,0.193396}%
\pgfsetfillcolor{currentfill}%
\pgfsetlinewidth{0.000000pt}%
\definecolor{currentstroke}{rgb}{0.000000,0.000000,0.000000}%
\pgfsetstrokecolor{currentstroke}%
\pgfsetdash{}{0pt}%
\pgfpathmoveto{\pgfqpoint{2.645389in}{2.921218in}}%
\pgfpathlineto{\pgfqpoint{2.692350in}{2.936908in}}%
\pgfpathlineto{\pgfqpoint{2.719878in}{3.026699in}}%
\pgfpathlineto{\pgfqpoint{2.672783in}{3.021156in}}%
\pgfpathlineto{\pgfqpoint{2.645389in}{2.921218in}}%
\pgfpathclose%
\pgfusepath{fill}%
\end{pgfscope}%
\begin{pgfscope}%
\pgfpathrectangle{\pgfqpoint{1.072000in}{0.528000in}}{\pgfqpoint{3.696000in}{3.696000in}}%
\pgfusepath{clip}%
\pgfsetbuttcap%
\pgfsetroundjoin%
\definecolor{currentfill}{rgb}{0.705673,0.015556,0.150233}%
\pgfsetfillcolor{currentfill}%
\pgfsetlinewidth{0.000000pt}%
\definecolor{currentstroke}{rgb}{0.000000,0.000000,0.000000}%
\pgfsetstrokecolor{currentstroke}%
\pgfsetdash{}{0pt}%
\pgfpathmoveto{\pgfqpoint{3.238131in}{3.059092in}}%
\pgfpathlineto{\pgfqpoint{3.285896in}{3.022272in}}%
\pgfpathlineto{\pgfqpoint{3.312772in}{3.053718in}}%
\pgfpathlineto{\pgfqpoint{3.264997in}{3.080613in}}%
\pgfpathlineto{\pgfqpoint{3.238131in}{3.059092in}}%
\pgfpathclose%
\pgfusepath{fill}%
\end{pgfscope}%
\begin{pgfscope}%
\pgfpathrectangle{\pgfqpoint{1.072000in}{0.528000in}}{\pgfqpoint{3.696000in}{3.696000in}}%
\pgfusepath{clip}%
\pgfsetbuttcap%
\pgfsetroundjoin%
\definecolor{currentfill}{rgb}{0.746838,0.140021,0.179996}%
\pgfsetfillcolor{currentfill}%
\pgfsetlinewidth{0.000000pt}%
\definecolor{currentstroke}{rgb}{0.000000,0.000000,0.000000}%
\pgfsetstrokecolor{currentstroke}%
\pgfsetdash{}{0pt}%
\pgfpathmoveto{\pgfqpoint{3.285896in}{3.022272in}}%
\pgfpathlineto{\pgfqpoint{3.333261in}{2.942059in}}%
\pgfpathlineto{\pgfqpoint{3.360110in}{2.980680in}}%
\pgfpathlineto{\pgfqpoint{3.312772in}{3.053718in}}%
\pgfpathlineto{\pgfqpoint{3.285896in}{3.022272in}}%
\pgfpathclose%
\pgfusepath{fill}%
\end{pgfscope}%
\begin{pgfscope}%
\pgfpathrectangle{\pgfqpoint{1.072000in}{0.528000in}}{\pgfqpoint{3.696000in}{3.696000in}}%
\pgfusepath{clip}%
\pgfsetbuttcap%
\pgfsetroundjoin%
\definecolor{currentfill}{rgb}{0.333490,0.446265,0.874452}%
\pgfsetfillcolor{currentfill}%
\pgfsetlinewidth{0.000000pt}%
\definecolor{currentstroke}{rgb}{0.000000,0.000000,0.000000}%
\pgfsetstrokecolor{currentstroke}%
\pgfsetdash{}{0pt}%
\pgfpathmoveto{\pgfqpoint{2.838583in}{1.251480in}}%
\pgfpathlineto{\pgfqpoint{2.885988in}{1.283727in}}%
\pgfpathlineto{\pgfqpoint{2.913842in}{1.208609in}}%
\pgfpathlineto{\pgfqpoint{2.866568in}{1.188596in}}%
\pgfpathlineto{\pgfqpoint{2.838583in}{1.251480in}}%
\pgfpathclose%
\pgfusepath{fill}%
\end{pgfscope}%
\begin{pgfscope}%
\pgfpathrectangle{\pgfqpoint{1.072000in}{0.528000in}}{\pgfqpoint{3.696000in}{3.696000in}}%
\pgfusepath{clip}%
\pgfsetbuttcap%
\pgfsetroundjoin%
\definecolor{currentfill}{rgb}{0.839365,0.321856,0.264924}%
\pgfsetfillcolor{currentfill}%
\pgfsetlinewidth{0.000000pt}%
\definecolor{currentstroke}{rgb}{0.000000,0.000000,0.000000}%
\pgfsetstrokecolor{currentstroke}%
\pgfsetdash{}{0pt}%
\pgfpathmoveto{\pgfqpoint{3.306201in}{2.890415in}}%
\pgfpathlineto{\pgfqpoint{3.353062in}{2.763308in}}%
\pgfpathlineto{\pgfqpoint{3.380087in}{2.821807in}}%
\pgfpathlineto{\pgfqpoint{3.333261in}{2.942059in}}%
\pgfpathlineto{\pgfqpoint{3.306201in}{2.890415in}}%
\pgfpathclose%
\pgfusepath{fill}%
\end{pgfscope}%
\begin{pgfscope}%
\pgfpathrectangle{\pgfqpoint{1.072000in}{0.528000in}}{\pgfqpoint{3.696000in}{3.696000in}}%
\pgfusepath{clip}%
\pgfsetbuttcap%
\pgfsetroundjoin%
\definecolor{currentfill}{rgb}{0.333490,0.446265,0.874452}%
\pgfsetfillcolor{currentfill}%
\pgfsetlinewidth{0.000000pt}%
\definecolor{currentstroke}{rgb}{0.000000,0.000000,0.000000}%
\pgfsetstrokecolor{currentstroke}%
\pgfsetdash{}{0pt}%
\pgfpathmoveto{\pgfqpoint{2.961326in}{1.239545in}}%
\pgfpathlineto{\pgfqpoint{3.009069in}{1.282594in}}%
\pgfpathlineto{\pgfqpoint{3.036317in}{1.211371in}}%
\pgfpathlineto{\pgfqpoint{2.988739in}{1.182659in}}%
\pgfpathlineto{\pgfqpoint{2.961326in}{1.239545in}}%
\pgfpathclose%
\pgfusepath{fill}%
\end{pgfscope}%
\begin{pgfscope}%
\pgfpathrectangle{\pgfqpoint{1.072000in}{0.528000in}}{\pgfqpoint{3.696000in}{3.696000in}}%
\pgfusepath{clip}%
\pgfsetbuttcap%
\pgfsetroundjoin%
\definecolor{currentfill}{rgb}{0.959385,0.610306,0.489382}%
\pgfsetfillcolor{currentfill}%
\pgfsetlinewidth{0.000000pt}%
\definecolor{currentstroke}{rgb}{0.000000,0.000000,0.000000}%
\pgfsetstrokecolor{currentstroke}%
\pgfsetdash{}{0pt}%
\pgfpathmoveto{\pgfqpoint{2.705739in}{2.458476in}}%
\pgfpathlineto{\pgfqpoint{2.753032in}{2.436826in}}%
\pgfpathlineto{\pgfqpoint{2.779971in}{2.625690in}}%
\pgfpathlineto{\pgfqpoint{2.732527in}{2.646858in}}%
\pgfpathlineto{\pgfqpoint{2.705739in}{2.458476in}}%
\pgfpathclose%
\pgfusepath{fill}%
\end{pgfscope}%
\begin{pgfscope}%
\pgfpathrectangle{\pgfqpoint{1.072000in}{0.528000in}}{\pgfqpoint{3.696000in}{3.696000in}}%
\pgfusepath{clip}%
\pgfsetbuttcap%
\pgfsetroundjoin%
\definecolor{currentfill}{rgb}{0.960581,0.762501,0.667964}%
\pgfsetfillcolor{currentfill}%
\pgfsetlinewidth{0.000000pt}%
\definecolor{currentstroke}{rgb}{0.000000,0.000000,0.000000}%
\pgfsetstrokecolor{currentstroke}%
\pgfsetdash{}{0pt}%
\pgfpathmoveto{\pgfqpoint{3.018246in}{2.280577in}}%
\pgfpathlineto{\pgfqpoint{3.065656in}{2.165475in}}%
\pgfpathlineto{\pgfqpoint{3.093356in}{2.338227in}}%
\pgfpathlineto{\pgfqpoint{3.045890in}{2.455560in}}%
\pgfpathlineto{\pgfqpoint{3.018246in}{2.280577in}}%
\pgfpathclose%
\pgfusepath{fill}%
\end{pgfscope}%
\begin{pgfscope}%
\pgfpathrectangle{\pgfqpoint{1.072000in}{0.528000in}}{\pgfqpoint{3.696000in}{3.696000in}}%
\pgfusepath{clip}%
\pgfsetbuttcap%
\pgfsetroundjoin%
\definecolor{currentfill}{rgb}{0.965899,0.740142,0.637058}%
\pgfsetfillcolor{currentfill}%
\pgfsetlinewidth{0.000000pt}%
\definecolor{currentstroke}{rgb}{0.000000,0.000000,0.000000}%
\pgfsetstrokecolor{currentstroke}%
\pgfsetdash{}{0pt}%
\pgfpathmoveto{\pgfqpoint{3.093356in}{2.338227in}}%
\pgfpathlineto{\pgfqpoint{3.140577in}{2.206512in}}%
\pgfpathlineto{\pgfqpoint{3.168359in}{2.364245in}}%
\pgfpathlineto{\pgfqpoint{3.121118in}{2.497362in}}%
\pgfpathlineto{\pgfqpoint{3.093356in}{2.338227in}}%
\pgfpathclose%
\pgfusepath{fill}%
\end{pgfscope}%
\begin{pgfscope}%
\pgfpathrectangle{\pgfqpoint{1.072000in}{0.528000in}}{\pgfqpoint{3.696000in}{3.696000in}}%
\pgfusepath{clip}%
\pgfsetbuttcap%
\pgfsetroundjoin%
\definecolor{currentfill}{rgb}{0.962708,0.753557,0.655601}%
\pgfsetfillcolor{currentfill}%
\pgfsetlinewidth{0.000000pt}%
\definecolor{currentstroke}{rgb}{0.000000,0.000000,0.000000}%
\pgfsetstrokecolor{currentstroke}%
\pgfsetdash{}{0pt}%
\pgfpathmoveto{\pgfqpoint{2.895661in}{2.276428in}}%
\pgfpathlineto{\pgfqpoint{2.943236in}{2.192294in}}%
\pgfpathlineto{\pgfqpoint{2.970669in}{2.381722in}}%
\pgfpathlineto{\pgfqpoint{2.922986in}{2.467192in}}%
\pgfpathlineto{\pgfqpoint{2.895661in}{2.276428in}}%
\pgfpathclose%
\pgfusepath{fill}%
\end{pgfscope}%
\begin{pgfscope}%
\pgfpathrectangle{\pgfqpoint{1.072000in}{0.528000in}}{\pgfqpoint{3.696000in}{3.696000in}}%
\pgfusepath{clip}%
\pgfsetbuttcap%
\pgfsetroundjoin%
\definecolor{currentfill}{rgb}{0.729196,0.086679,0.167240}%
\pgfsetfillcolor{currentfill}%
\pgfsetlinewidth{0.000000pt}%
\definecolor{currentstroke}{rgb}{0.000000,0.000000,0.000000}%
\pgfsetstrokecolor{currentstroke}%
\pgfsetdash{}{0pt}%
\pgfpathmoveto{\pgfqpoint{2.965273in}{3.024192in}}%
\pgfpathlineto{\pgfqpoint{3.012985in}{3.025599in}}%
\pgfpathlineto{\pgfqpoint{3.040322in}{3.013743in}}%
\pgfpathlineto{\pgfqpoint{2.992695in}{2.988501in}}%
\pgfpathlineto{\pgfqpoint{2.965273in}{3.024192in}}%
\pgfpathclose%
\pgfusepath{fill}%
\end{pgfscope}%
\begin{pgfscope}%
\pgfpathrectangle{\pgfqpoint{1.072000in}{0.528000in}}{\pgfqpoint{3.696000in}{3.696000in}}%
\pgfusepath{clip}%
\pgfsetbuttcap%
\pgfsetroundjoin%
\definecolor{currentfill}{rgb}{0.705673,0.015556,0.150233}%
\pgfsetfillcolor{currentfill}%
\pgfsetlinewidth{0.000000pt}%
\definecolor{currentstroke}{rgb}{0.000000,0.000000,0.000000}%
\pgfsetstrokecolor{currentstroke}%
\pgfsetdash{}{0pt}%
\pgfpathmoveto{\pgfqpoint{2.719878in}{3.026699in}}%
\pgfpathlineto{\pgfqpoint{2.767192in}{3.022586in}}%
\pgfpathlineto{\pgfqpoint{2.795010in}{3.059097in}}%
\pgfpathlineto{\pgfqpoint{2.747673in}{3.072994in}}%
\pgfpathlineto{\pgfqpoint{2.719878in}{3.026699in}}%
\pgfpathclose%
\pgfusepath{fill}%
\end{pgfscope}%
\begin{pgfscope}%
\pgfpathrectangle{\pgfqpoint{1.072000in}{0.528000in}}{\pgfqpoint{3.696000in}{3.696000in}}%
\pgfusepath{clip}%
\pgfsetbuttcap%
\pgfsetroundjoin%
\definecolor{currentfill}{rgb}{0.964911,0.640159,0.519806}%
\pgfsetfillcolor{currentfill}%
\pgfsetlinewidth{0.000000pt}%
\definecolor{currentstroke}{rgb}{0.000000,0.000000,0.000000}%
\pgfsetstrokecolor{currentstroke}%
\pgfsetdash{}{0pt}%
\pgfpathmoveto{\pgfqpoint{3.196144in}{2.507369in}}%
\pgfpathlineto{\pgfqpoint{3.243050in}{2.358464in}}%
\pgfpathlineto{\pgfqpoint{3.270764in}{2.485637in}}%
\pgfpathlineto{\pgfqpoint{3.223871in}{2.632386in}}%
\pgfpathlineto{\pgfqpoint{3.196144in}{2.507369in}}%
\pgfpathclose%
\pgfusepath{fill}%
\end{pgfscope}%
\begin{pgfscope}%
\pgfpathrectangle{\pgfqpoint{1.072000in}{0.528000in}}{\pgfqpoint{3.696000in}{3.696000in}}%
\pgfusepath{clip}%
\pgfsetbuttcap%
\pgfsetroundjoin%
\definecolor{currentfill}{rgb}{0.717435,0.051118,0.158737}%
\pgfsetfillcolor{currentfill}%
\pgfsetlinewidth{0.000000pt}%
\definecolor{currentstroke}{rgb}{0.000000,0.000000,0.000000}%
\pgfsetstrokecolor{currentstroke}%
\pgfsetdash{}{0pt}%
\pgfpathmoveto{\pgfqpoint{2.842458in}{3.045130in}}%
\pgfpathlineto{\pgfqpoint{2.890012in}{3.033063in}}%
\pgfpathlineto{\pgfqpoint{2.917713in}{3.023025in}}%
\pgfpathlineto{\pgfqpoint{2.870281in}{3.030245in}}%
\pgfpathlineto{\pgfqpoint{2.842458in}{3.045130in}}%
\pgfpathclose%
\pgfusepath{fill}%
\end{pgfscope}%
\begin{pgfscope}%
\pgfpathrectangle{\pgfqpoint{1.072000in}{0.528000in}}{\pgfqpoint{3.696000in}{3.696000in}}%
\pgfusepath{clip}%
\pgfsetbuttcap%
\pgfsetroundjoin%
\definecolor{currentfill}{rgb}{0.705673,0.015556,0.150233}%
\pgfsetfillcolor{currentfill}%
\pgfsetlinewidth{0.000000pt}%
\definecolor{currentstroke}{rgb}{0.000000,0.000000,0.000000}%
\pgfsetstrokecolor{currentstroke}%
\pgfsetdash{}{0pt}%
\pgfpathmoveto{\pgfqpoint{3.163189in}{3.044686in}}%
\pgfpathlineto{\pgfqpoint{3.211117in}{3.032232in}}%
\pgfpathlineto{\pgfqpoint{3.238131in}{3.059092in}}%
\pgfpathlineto{\pgfqpoint{3.190147in}{3.052755in}}%
\pgfpathlineto{\pgfqpoint{3.163189in}{3.044686in}}%
\pgfpathclose%
\pgfusepath{fill}%
\end{pgfscope}%
\begin{pgfscope}%
\pgfpathrectangle{\pgfqpoint{1.072000in}{0.528000in}}{\pgfqpoint{3.696000in}{3.696000in}}%
\pgfusepath{clip}%
\pgfsetbuttcap%
\pgfsetroundjoin%
\definecolor{currentfill}{rgb}{0.869655,0.379274,0.300941}%
\pgfsetfillcolor{currentfill}%
\pgfsetlinewidth{0.000000pt}%
\definecolor{currentstroke}{rgb}{0.000000,0.000000,0.000000}%
\pgfsetstrokecolor{currentstroke}%
\pgfsetdash{}{0pt}%
\pgfpathmoveto{\pgfqpoint{3.278937in}{2.823090in}}%
\pgfpathlineto{\pgfqpoint{3.325813in}{2.688373in}}%
\pgfpathlineto{\pgfqpoint{3.353062in}{2.763308in}}%
\pgfpathlineto{\pgfqpoint{3.306201in}{2.890415in}}%
\pgfpathlineto{\pgfqpoint{3.278937in}{2.823090in}}%
\pgfpathclose%
\pgfusepath{fill}%
\end{pgfscope}%
\begin{pgfscope}%
\pgfpathrectangle{\pgfqpoint{1.072000in}{0.528000in}}{\pgfqpoint{3.696000in}{3.696000in}}%
\pgfusepath{clip}%
\pgfsetbuttcap%
\pgfsetroundjoin%
\definecolor{currentfill}{rgb}{0.962701,0.628218,0.507636}%
\pgfsetfillcolor{currentfill}%
\pgfsetlinewidth{0.000000pt}%
\definecolor{currentstroke}{rgb}{0.000000,0.000000,0.000000}%
\pgfsetstrokecolor{currentstroke}%
\pgfsetdash{}{0pt}%
\pgfpathmoveto{\pgfqpoint{2.753032in}{2.436826in}}%
\pgfpathlineto{\pgfqpoint{2.800493in}{2.399201in}}%
\pgfpathlineto{\pgfqpoint{2.827570in}{2.588969in}}%
\pgfpathlineto{\pgfqpoint{2.779971in}{2.625690in}}%
\pgfpathlineto{\pgfqpoint{2.753032in}{2.436826in}}%
\pgfpathclose%
\pgfusepath{fill}%
\end{pgfscope}%
\begin{pgfscope}%
\pgfpathrectangle{\pgfqpoint{1.072000in}{0.528000in}}{\pgfqpoint{3.696000in}{3.696000in}}%
\pgfusepath{clip}%
\pgfsetbuttcap%
\pgfsetroundjoin%
\definecolor{currentfill}{rgb}{0.969522,0.700833,0.587508}%
\pgfsetfillcolor{currentfill}%
\pgfsetlinewidth{0.000000pt}%
\definecolor{currentstroke}{rgb}{0.000000,0.000000,0.000000}%
\pgfsetstrokecolor{currentstroke}%
\pgfsetdash{}{0pt}%
\pgfpathmoveto{\pgfqpoint{2.848059in}{2.345643in}}%
\pgfpathlineto{\pgfqpoint{2.895661in}{2.276428in}}%
\pgfpathlineto{\pgfqpoint{2.922986in}{2.467192in}}%
\pgfpathlineto{\pgfqpoint{2.875264in}{2.536269in}}%
\pgfpathlineto{\pgfqpoint{2.848059in}{2.345643in}}%
\pgfpathclose%
\pgfusepath{fill}%
\end{pgfscope}%
\begin{pgfscope}%
\pgfpathrectangle{\pgfqpoint{1.072000in}{0.528000in}}{\pgfqpoint{3.696000in}{3.696000in}}%
\pgfusepath{clip}%
\pgfsetbuttcap%
\pgfsetroundjoin%
\definecolor{currentfill}{rgb}{0.425199,0.559058,0.946061}%
\pgfsetfillcolor{currentfill}%
\pgfsetlinewidth{0.000000pt}%
\definecolor{currentstroke}{rgb}{0.000000,0.000000,0.000000}%
\pgfsetstrokecolor{currentstroke}%
\pgfsetdash{}{0pt}%
\pgfpathmoveto{\pgfqpoint{3.432225in}{1.331637in}}%
\pgfpathlineto{\pgfqpoint{3.482167in}{1.428358in}}%
\pgfpathlineto{\pgfqpoint{3.507272in}{1.366847in}}%
\pgfpathlineto{\pgfqpoint{3.457628in}{1.283075in}}%
\pgfpathlineto{\pgfqpoint{3.432225in}{1.331637in}}%
\pgfpathclose%
\pgfusepath{fill}%
\end{pgfscope}%
\begin{pgfscope}%
\pgfpathrectangle{\pgfqpoint{1.072000in}{0.528000in}}{\pgfqpoint{3.696000in}{3.696000in}}%
\pgfusepath{clip}%
\pgfsetbuttcap%
\pgfsetroundjoin%
\definecolor{currentfill}{rgb}{0.820401,0.286765,0.245160}%
\pgfsetfillcolor{currentfill}%
\pgfsetlinewidth{0.000000pt}%
\definecolor{currentstroke}{rgb}{0.000000,0.000000,0.000000}%
\pgfsetstrokecolor{currentstroke}%
\pgfsetdash{}{0pt}%
\pgfpathmoveto{\pgfqpoint{2.665163in}{2.806897in}}%
\pgfpathlineto{\pgfqpoint{2.712271in}{2.814207in}}%
\pgfpathlineto{\pgfqpoint{2.739585in}{2.939734in}}%
\pgfpathlineto{\pgfqpoint{2.692350in}{2.936908in}}%
\pgfpathlineto{\pgfqpoint{2.665163in}{2.806897in}}%
\pgfpathclose%
\pgfusepath{fill}%
\end{pgfscope}%
\begin{pgfscope}%
\pgfpathrectangle{\pgfqpoint{1.072000in}{0.528000in}}{\pgfqpoint{3.696000in}{3.696000in}}%
\pgfusepath{clip}%
\pgfsetbuttcap%
\pgfsetroundjoin%
\definecolor{currentfill}{rgb}{0.451739,0.588181,0.960201}%
\pgfsetfillcolor{currentfill}%
\pgfsetlinewidth{0.000000pt}%
\definecolor{currentstroke}{rgb}{0.000000,0.000000,0.000000}%
\pgfsetstrokecolor{currentstroke}%
\pgfsetdash{}{0pt}%
\pgfpathmoveto{\pgfqpoint{3.507272in}{1.366847in}}%
\pgfpathlineto{\pgfqpoint{3.557721in}{1.473662in}}%
\pgfpathlineto{\pgfqpoint{3.582545in}{1.415647in}}%
\pgfpathlineto{\pgfqpoint{3.532387in}{1.320331in}}%
\pgfpathlineto{\pgfqpoint{3.507272in}{1.366847in}}%
\pgfpathclose%
\pgfusepath{fill}%
\end{pgfscope}%
\begin{pgfscope}%
\pgfpathrectangle{\pgfqpoint{1.072000in}{0.528000in}}{\pgfqpoint{3.696000in}{3.696000in}}%
\pgfusepath{clip}%
\pgfsetbuttcap%
\pgfsetroundjoin%
\definecolor{currentfill}{rgb}{0.939254,0.539581,0.423900}%
\pgfsetfillcolor{currentfill}%
\pgfsetlinewidth{0.000000pt}%
\definecolor{currentstroke}{rgb}{0.000000,0.000000,0.000000}%
\pgfsetstrokecolor{currentstroke}%
\pgfsetdash{}{0pt}%
\pgfpathmoveto{\pgfqpoint{3.223871in}{2.632386in}}%
\pgfpathlineto{\pgfqpoint{3.270764in}{2.485637in}}%
\pgfpathlineto{\pgfqpoint{3.298368in}{2.595912in}}%
\pgfpathlineto{\pgfqpoint{3.251483in}{2.737600in}}%
\pgfpathlineto{\pgfqpoint{3.223871in}{2.632386in}}%
\pgfpathclose%
\pgfusepath{fill}%
\end{pgfscope}%
\begin{pgfscope}%
\pgfpathrectangle{\pgfqpoint{1.072000in}{0.528000in}}{\pgfqpoint{3.696000in}{3.696000in}}%
\pgfusepath{clip}%
\pgfsetbuttcap%
\pgfsetroundjoin%
\definecolor{currentfill}{rgb}{0.358415,0.478426,0.896795}%
\pgfsetfillcolor{currentfill}%
\pgfsetlinewidth{0.000000pt}%
\definecolor{currentstroke}{rgb}{0.000000,0.000000,0.000000}%
\pgfsetstrokecolor{currentstroke}%
\pgfsetdash{}{0pt}%
\pgfpathmoveto{\pgfqpoint{3.084181in}{1.254095in}}%
\pgfpathlineto{\pgfqpoint{3.132397in}{1.311668in}}%
\pgfpathlineto{\pgfqpoint{3.159059in}{1.238861in}}%
\pgfpathlineto{\pgfqpoint{3.111055in}{1.196264in}}%
\pgfpathlineto{\pgfqpoint{3.084181in}{1.254095in}}%
\pgfpathclose%
\pgfusepath{fill}%
\end{pgfscope}%
\begin{pgfscope}%
\pgfpathrectangle{\pgfqpoint{1.072000in}{0.528000in}}{\pgfqpoint{3.696000in}{3.696000in}}%
\pgfusepath{clip}%
\pgfsetbuttcap%
\pgfsetroundjoin%
\definecolor{currentfill}{rgb}{0.967317,0.657471,0.538160}%
\pgfsetfillcolor{currentfill}%
\pgfsetlinewidth{0.000000pt}%
\definecolor{currentstroke}{rgb}{0.000000,0.000000,0.000000}%
\pgfsetstrokecolor{currentstroke}%
\pgfsetdash{}{0pt}%
\pgfpathmoveto{\pgfqpoint{2.800493in}{2.399201in}}%
\pgfpathlineto{\pgfqpoint{2.848059in}{2.345643in}}%
\pgfpathlineto{\pgfqpoint{2.875264in}{2.536269in}}%
\pgfpathlineto{\pgfqpoint{2.827570in}{2.588969in}}%
\pgfpathlineto{\pgfqpoint{2.800493in}{2.399201in}}%
\pgfpathclose%
\pgfusepath{fill}%
\end{pgfscope}%
\begin{pgfscope}%
\pgfpathrectangle{\pgfqpoint{1.072000in}{0.528000in}}{\pgfqpoint{3.696000in}{3.696000in}}%
\pgfusepath{clip}%
\pgfsetbuttcap%
\pgfsetroundjoin%
\definecolor{currentfill}{rgb}{0.905783,0.455186,0.355336}%
\pgfsetfillcolor{currentfill}%
\pgfsetlinewidth{0.000000pt}%
\definecolor{currentstroke}{rgb}{0.000000,0.000000,0.000000}%
\pgfsetstrokecolor{currentstroke}%
\pgfsetdash{}{0pt}%
\pgfpathmoveto{\pgfqpoint{3.251483in}{2.737600in}}%
\pgfpathlineto{\pgfqpoint{3.298368in}{2.595912in}}%
\pgfpathlineto{\pgfqpoint{3.325813in}{2.688373in}}%
\pgfpathlineto{\pgfqpoint{3.278937in}{2.823090in}}%
\pgfpathlineto{\pgfqpoint{3.251483in}{2.737600in}}%
\pgfpathclose%
\pgfusepath{fill}%
\end{pgfscope}%
\begin{pgfscope}%
\pgfpathrectangle{\pgfqpoint{1.072000in}{0.528000in}}{\pgfqpoint{3.696000in}{3.696000in}}%
\pgfusepath{clip}%
\pgfsetbuttcap%
\pgfsetroundjoin%
\definecolor{currentfill}{rgb}{0.895885,0.433075,0.338681}%
\pgfsetfillcolor{currentfill}%
\pgfsetlinewidth{0.000000pt}%
\definecolor{currentstroke}{rgb}{0.000000,0.000000,0.000000}%
\pgfsetstrokecolor{currentstroke}%
\pgfsetdash{}{0pt}%
\pgfpathmoveto{\pgfqpoint{2.685301in}{2.652674in}}%
\pgfpathlineto{\pgfqpoint{2.732527in}{2.646858in}}%
\pgfpathlineto{\pgfqpoint{2.759632in}{2.807638in}}%
\pgfpathlineto{\pgfqpoint{2.712271in}{2.814207in}}%
\pgfpathlineto{\pgfqpoint{2.685301in}{2.652674in}}%
\pgfpathclose%
\pgfusepath{fill}%
\end{pgfscope}%
\begin{pgfscope}%
\pgfpathrectangle{\pgfqpoint{1.072000in}{0.528000in}}{\pgfqpoint{3.696000in}{3.696000in}}%
\pgfusepath{clip}%
\pgfsetbuttcap%
\pgfsetroundjoin%
\definecolor{currentfill}{rgb}{0.763520,0.178667,0.193396}%
\pgfsetfillcolor{currentfill}%
\pgfsetlinewidth{0.000000pt}%
\definecolor{currentstroke}{rgb}{0.000000,0.000000,0.000000}%
\pgfsetstrokecolor{currentstroke}%
\pgfsetdash{}{0pt}%
\pgfpathmoveto{\pgfqpoint{3.258843in}{2.981041in}}%
\pgfpathlineto{\pgfqpoint{3.306201in}{2.890415in}}%
\pgfpathlineto{\pgfqpoint{3.333261in}{2.942059in}}%
\pgfpathlineto{\pgfqpoint{3.285896in}{3.022272in}}%
\pgfpathlineto{\pgfqpoint{3.258843in}{2.981041in}}%
\pgfpathclose%
\pgfusepath{fill}%
\end{pgfscope}%
\begin{pgfscope}%
\pgfpathrectangle{\pgfqpoint{1.072000in}{0.528000in}}{\pgfqpoint{3.696000in}{3.696000in}}%
\pgfusepath{clip}%
\pgfsetbuttcap%
\pgfsetroundjoin%
\definecolor{currentfill}{rgb}{0.711554,0.033337,0.154485}%
\pgfsetfillcolor{currentfill}%
\pgfsetlinewidth{0.000000pt}%
\definecolor{currentstroke}{rgb}{0.000000,0.000000,0.000000}%
\pgfsetstrokecolor{currentstroke}%
\pgfsetdash{}{0pt}%
\pgfpathmoveto{\pgfqpoint{3.088138in}{3.030631in}}%
\pgfpathlineto{\pgfqpoint{3.136051in}{3.026690in}}%
\pgfpathlineto{\pgfqpoint{3.163189in}{3.044686in}}%
\pgfpathlineto{\pgfqpoint{3.115237in}{3.023276in}}%
\pgfpathlineto{\pgfqpoint{3.088138in}{3.030631in}}%
\pgfpathclose%
\pgfusepath{fill}%
\end{pgfscope}%
\begin{pgfscope}%
\pgfpathrectangle{\pgfqpoint{1.072000in}{0.528000in}}{\pgfqpoint{3.696000in}{3.696000in}}%
\pgfusepath{clip}%
\pgfsetbuttcap%
\pgfsetroundjoin%
\definecolor{currentfill}{rgb}{0.409611,0.540759,0.935545}%
\pgfsetfillcolor{currentfill}%
\pgfsetlinewidth{0.000000pt}%
\definecolor{currentstroke}{rgb}{0.000000,0.000000,0.000000}%
\pgfsetstrokecolor{currentstroke}%
\pgfsetdash{}{0pt}%
\pgfpathmoveto{\pgfqpoint{3.357299in}{1.308828in}}%
\pgfpathlineto{\pgfqpoint{3.406819in}{1.397200in}}%
\pgfpathlineto{\pgfqpoint{3.432225in}{1.331637in}}%
\pgfpathlineto{\pgfqpoint{3.382996in}{1.257122in}}%
\pgfpathlineto{\pgfqpoint{3.357299in}{1.308828in}}%
\pgfpathclose%
\pgfusepath{fill}%
\end{pgfscope}%
\begin{pgfscope}%
\pgfpathrectangle{\pgfqpoint{1.072000in}{0.528000in}}{\pgfqpoint{3.696000in}{3.696000in}}%
\pgfusepath{clip}%
\pgfsetbuttcap%
\pgfsetroundjoin%
\definecolor{currentfill}{rgb}{0.746838,0.140021,0.179996}%
\pgfsetfillcolor{currentfill}%
\pgfsetlinewidth{0.000000pt}%
\definecolor{currentstroke}{rgb}{0.000000,0.000000,0.000000}%
\pgfsetstrokecolor{currentstroke}%
\pgfsetdash{}{0pt}%
\pgfpathmoveto{\pgfqpoint{2.692350in}{2.936908in}}%
\pgfpathlineto{\pgfqpoint{2.739585in}{2.939734in}}%
\pgfpathlineto{\pgfqpoint{2.767192in}{3.022586in}}%
\pgfpathlineto{\pgfqpoint{2.719878in}{3.026699in}}%
\pgfpathlineto{\pgfqpoint{2.692350in}{2.936908in}}%
\pgfpathclose%
\pgfusepath{fill}%
\end{pgfscope}%
\begin{pgfscope}%
\pgfpathrectangle{\pgfqpoint{1.072000in}{0.528000in}}{\pgfqpoint{3.696000in}{3.696000in}}%
\pgfusepath{clip}%
\pgfsetbuttcap%
\pgfsetroundjoin%
\definecolor{currentfill}{rgb}{0.711554,0.033337,0.154485}%
\pgfsetfillcolor{currentfill}%
\pgfsetlinewidth{0.000000pt}%
\definecolor{currentstroke}{rgb}{0.000000,0.000000,0.000000}%
\pgfsetstrokecolor{currentstroke}%
\pgfsetdash{}{0pt}%
\pgfpathmoveto{\pgfqpoint{3.211117in}{3.032232in}}%
\pgfpathlineto{\pgfqpoint{3.258843in}{2.981041in}}%
\pgfpathlineto{\pgfqpoint{3.285896in}{3.022272in}}%
\pgfpathlineto{\pgfqpoint{3.238131in}{3.059092in}}%
\pgfpathlineto{\pgfqpoint{3.211117in}{3.032232in}}%
\pgfpathclose%
\pgfusepath{fill}%
\end{pgfscope}%
\begin{pgfscope}%
\pgfpathrectangle{\pgfqpoint{1.072000in}{0.528000in}}{\pgfqpoint{3.696000in}{3.696000in}}%
\pgfusepath{clip}%
\pgfsetbuttcap%
\pgfsetroundjoin%
\definecolor{currentfill}{rgb}{0.969289,0.684982,0.568975}%
\pgfsetfillcolor{currentfill}%
\pgfsetlinewidth{0.000000pt}%
\definecolor{currentstroke}{rgb}{0.000000,0.000000,0.000000}%
\pgfsetstrokecolor{currentstroke}%
\pgfsetdash{}{0pt}%
\pgfpathmoveto{\pgfqpoint{2.970669in}{2.381722in}}%
\pgfpathlineto{\pgfqpoint{3.018246in}{2.280577in}}%
\pgfpathlineto{\pgfqpoint{3.045890in}{2.455560in}}%
\pgfpathlineto{\pgfqpoint{2.998239in}{2.556143in}}%
\pgfpathlineto{\pgfqpoint{2.970669in}{2.381722in}}%
\pgfpathclose%
\pgfusepath{fill}%
\end{pgfscope}%
\begin{pgfscope}%
\pgfpathrectangle{\pgfqpoint{1.072000in}{0.528000in}}{\pgfqpoint{3.696000in}{3.696000in}}%
\pgfusepath{clip}%
\pgfsetbuttcap%
\pgfsetroundjoin%
\definecolor{currentfill}{rgb}{0.705673,0.015556,0.150233}%
\pgfsetfillcolor{currentfill}%
\pgfsetlinewidth{0.000000pt}%
\definecolor{currentstroke}{rgb}{0.000000,0.000000,0.000000}%
\pgfsetstrokecolor{currentstroke}%
\pgfsetdash{}{0pt}%
\pgfpathmoveto{\pgfqpoint{2.767192in}{3.022586in}}%
\pgfpathlineto{\pgfqpoint{2.814671in}{3.012088in}}%
\pgfpathlineto{\pgfqpoint{2.842458in}{3.045130in}}%
\pgfpathlineto{\pgfqpoint{2.795010in}{3.059097in}}%
\pgfpathlineto{\pgfqpoint{2.767192in}{3.022586in}}%
\pgfpathclose%
\pgfusepath{fill}%
\end{pgfscope}%
\begin{pgfscope}%
\pgfpathrectangle{\pgfqpoint{1.072000in}{0.528000in}}{\pgfqpoint{3.696000in}{3.696000in}}%
\pgfusepath{clip}%
\pgfsetbuttcap%
\pgfsetroundjoin%
\definecolor{currentfill}{rgb}{0.962701,0.628218,0.507636}%
\pgfsetfillcolor{currentfill}%
\pgfsetlinewidth{0.000000pt}%
\definecolor{currentstroke}{rgb}{0.000000,0.000000,0.000000}%
\pgfsetstrokecolor{currentstroke}%
\pgfsetdash{}{0pt}%
\pgfpathmoveto{\pgfqpoint{3.121118in}{2.497362in}}%
\pgfpathlineto{\pgfqpoint{3.168359in}{2.364245in}}%
\pgfpathlineto{\pgfqpoint{3.196144in}{2.507369in}}%
\pgfpathlineto{\pgfqpoint{3.148883in}{2.637731in}}%
\pgfpathlineto{\pgfqpoint{3.121118in}{2.497362in}}%
\pgfpathclose%
\pgfusepath{fill}%
\end{pgfscope}%
\begin{pgfscope}%
\pgfpathrectangle{\pgfqpoint{1.072000in}{0.528000in}}{\pgfqpoint{3.696000in}{3.696000in}}%
\pgfusepath{clip}%
\pgfsetbuttcap%
\pgfsetroundjoin%
\definecolor{currentfill}{rgb}{0.711554,0.033337,0.154485}%
\pgfsetfillcolor{currentfill}%
\pgfsetlinewidth{0.000000pt}%
\definecolor{currentstroke}{rgb}{0.000000,0.000000,0.000000}%
\pgfsetstrokecolor{currentstroke}%
\pgfsetdash{}{0pt}%
\pgfpathmoveto{\pgfqpoint{2.890012in}{3.033063in}}%
\pgfpathlineto{\pgfqpoint{2.937682in}{3.020930in}}%
\pgfpathlineto{\pgfqpoint{2.965273in}{3.024192in}}%
\pgfpathlineto{\pgfqpoint{2.917713in}{3.023025in}}%
\pgfpathlineto{\pgfqpoint{2.890012in}{3.033063in}}%
\pgfpathclose%
\pgfusepath{fill}%
\end{pgfscope}%
\begin{pgfscope}%
\pgfpathrectangle{\pgfqpoint{1.072000in}{0.528000in}}{\pgfqpoint{3.696000in}{3.696000in}}%
\pgfusepath{clip}%
\pgfsetbuttcap%
\pgfsetroundjoin%
\definecolor{currentfill}{rgb}{0.404421,0.534643,0.932002}%
\pgfsetfillcolor{currentfill}%
\pgfsetlinewidth{0.000000pt}%
\definecolor{currentstroke}{rgb}{0.000000,0.000000,0.000000}%
\pgfsetstrokecolor{currentstroke}%
\pgfsetdash{}{0pt}%
\pgfpathmoveto{\pgfqpoint{3.282404in}{1.297772in}}%
\pgfpathlineto{\pgfqpoint{3.331575in}{1.379532in}}%
\pgfpathlineto{\pgfqpoint{3.357299in}{1.308828in}}%
\pgfpathlineto{\pgfqpoint{3.308403in}{1.241237in}}%
\pgfpathlineto{\pgfqpoint{3.282404in}{1.297772in}}%
\pgfpathclose%
\pgfusepath{fill}%
\end{pgfscope}%
\begin{pgfscope}%
\pgfpathrectangle{\pgfqpoint{1.072000in}{0.528000in}}{\pgfqpoint{3.696000in}{3.696000in}}%
\pgfusepath{clip}%
\pgfsetbuttcap%
\pgfsetroundjoin%
\definecolor{currentfill}{rgb}{0.711554,0.033337,0.154485}%
\pgfsetfillcolor{currentfill}%
\pgfsetlinewidth{0.000000pt}%
\definecolor{currentstroke}{rgb}{0.000000,0.000000,0.000000}%
\pgfsetstrokecolor{currentstroke}%
\pgfsetdash{}{0pt}%
\pgfpathmoveto{\pgfqpoint{3.012985in}{3.025599in}}%
\pgfpathlineto{\pgfqpoint{3.060826in}{3.017118in}}%
\pgfpathlineto{\pgfqpoint{3.088138in}{3.030631in}}%
\pgfpathlineto{\pgfqpoint{3.040322in}{3.013743in}}%
\pgfpathlineto{\pgfqpoint{3.012985in}{3.025599in}}%
\pgfpathclose%
\pgfusepath{fill}%
\end{pgfscope}%
\begin{pgfscope}%
\pgfpathrectangle{\pgfqpoint{1.072000in}{0.528000in}}{\pgfqpoint{3.696000in}{3.696000in}}%
\pgfusepath{clip}%
\pgfsetbuttcap%
\pgfsetroundjoin%
\definecolor{currentfill}{rgb}{0.964911,0.640159,0.519806}%
\pgfsetfillcolor{currentfill}%
\pgfsetlinewidth{0.000000pt}%
\definecolor{currentstroke}{rgb}{0.000000,0.000000,0.000000}%
\pgfsetstrokecolor{currentstroke}%
\pgfsetdash{}{0pt}%
\pgfpathmoveto{\pgfqpoint{3.045890in}{2.455560in}}%
\pgfpathlineto{\pgfqpoint{3.093356in}{2.338227in}}%
\pgfpathlineto{\pgfqpoint{3.121118in}{2.497362in}}%
\pgfpathlineto{\pgfqpoint{3.073607in}{2.612556in}}%
\pgfpathlineto{\pgfqpoint{3.045890in}{2.455560in}}%
\pgfpathclose%
\pgfusepath{fill}%
\end{pgfscope}%
\begin{pgfscope}%
\pgfpathrectangle{\pgfqpoint{1.072000in}{0.528000in}}{\pgfqpoint{3.696000in}{3.696000in}}%
\pgfusepath{clip}%
\pgfsetbuttcap%
\pgfsetroundjoin%
\definecolor{currentfill}{rgb}{0.378598,0.503856,0.913692}%
\pgfsetfillcolor{currentfill}%
\pgfsetlinewidth{0.000000pt}%
\definecolor{currentstroke}{rgb}{0.000000,0.000000,0.000000}%
\pgfsetstrokecolor{currentstroke}%
\pgfsetdash{}{0pt}%
\pgfpathmoveto{\pgfqpoint{2.885988in}{1.283727in}}%
\pgfpathlineto{\pgfqpoint{2.933630in}{1.326565in}}%
\pgfpathlineto{\pgfqpoint{2.961326in}{1.239545in}}%
\pgfpathlineto{\pgfqpoint{2.913842in}{1.208609in}}%
\pgfpathlineto{\pgfqpoint{2.885988in}{1.283727in}}%
\pgfpathclose%
\pgfusepath{fill}%
\end{pgfscope}%
\begin{pgfscope}%
\pgfpathrectangle{\pgfqpoint{1.072000in}{0.528000in}}{\pgfqpoint{3.696000in}{3.696000in}}%
\pgfusepath{clip}%
\pgfsetbuttcap%
\pgfsetroundjoin%
\definecolor{currentfill}{rgb}{0.895885,0.433075,0.338681}%
\pgfsetfillcolor{currentfill}%
\pgfsetlinewidth{0.000000pt}%
\definecolor{currentstroke}{rgb}{0.000000,0.000000,0.000000}%
\pgfsetstrokecolor{currentstroke}%
\pgfsetdash{}{0pt}%
\pgfpathmoveto{\pgfqpoint{2.732527in}{2.646858in}}%
\pgfpathlineto{\pgfqpoint{2.779971in}{2.625690in}}%
\pgfpathlineto{\pgfqpoint{2.807186in}{2.787558in}}%
\pgfpathlineto{\pgfqpoint{2.759632in}{2.807638in}}%
\pgfpathlineto{\pgfqpoint{2.732527in}{2.646858in}}%
\pgfpathclose%
\pgfusepath{fill}%
\end{pgfscope}%
\begin{pgfscope}%
\pgfpathrectangle{\pgfqpoint{1.072000in}{0.528000in}}{\pgfqpoint{3.696000in}{3.696000in}}%
\pgfusepath{clip}%
\pgfsetbuttcap%
\pgfsetroundjoin%
\definecolor{currentfill}{rgb}{0.790562,0.231397,0.216242}%
\pgfsetfillcolor{currentfill}%
\pgfsetlinewidth{0.000000pt}%
\definecolor{currentstroke}{rgb}{0.000000,0.000000,0.000000}%
\pgfsetstrokecolor{currentstroke}%
\pgfsetdash{}{0pt}%
\pgfpathmoveto{\pgfqpoint{3.231601in}{2.925617in}}%
\pgfpathlineto{\pgfqpoint{3.278937in}{2.823090in}}%
\pgfpathlineto{\pgfqpoint{3.306201in}{2.890415in}}%
\pgfpathlineto{\pgfqpoint{3.258843in}{2.981041in}}%
\pgfpathlineto{\pgfqpoint{3.231601in}{2.925617in}}%
\pgfpathclose%
\pgfusepath{fill}%
\end{pgfscope}%
\begin{pgfscope}%
\pgfpathrectangle{\pgfqpoint{1.072000in}{0.528000in}}{\pgfqpoint{3.696000in}{3.696000in}}%
\pgfusepath{clip}%
\pgfsetbuttcap%
\pgfsetroundjoin%
\definecolor{currentfill}{rgb}{0.810616,0.268797,0.235428}%
\pgfsetfillcolor{currentfill}%
\pgfsetlinewidth{0.000000pt}%
\definecolor{currentstroke}{rgb}{0.000000,0.000000,0.000000}%
\pgfsetstrokecolor{currentstroke}%
\pgfsetdash{}{0pt}%
\pgfpathmoveto{\pgfqpoint{2.712271in}{2.814207in}}%
\pgfpathlineto{\pgfqpoint{2.759632in}{2.807638in}}%
\pgfpathlineto{\pgfqpoint{2.787033in}{2.931647in}}%
\pgfpathlineto{\pgfqpoint{2.739585in}{2.939734in}}%
\pgfpathlineto{\pgfqpoint{2.712271in}{2.814207in}}%
\pgfpathclose%
\pgfusepath{fill}%
\end{pgfscope}%
\begin{pgfscope}%
\pgfpathrectangle{\pgfqpoint{1.072000in}{0.528000in}}{\pgfqpoint{3.696000in}{3.696000in}}%
\pgfusepath{clip}%
\pgfsetbuttcap%
\pgfsetroundjoin%
\definecolor{currentfill}{rgb}{0.705673,0.015556,0.150233}%
\pgfsetfillcolor{currentfill}%
\pgfsetlinewidth{0.000000pt}%
\definecolor{currentstroke}{rgb}{0.000000,0.000000,0.000000}%
\pgfsetstrokecolor{currentstroke}%
\pgfsetdash{}{0pt}%
\pgfpathmoveto{\pgfqpoint{3.136051in}{3.026690in}}%
\pgfpathlineto{\pgfqpoint{3.183924in}{2.993230in}}%
\pgfpathlineto{\pgfqpoint{3.211117in}{3.032232in}}%
\pgfpathlineto{\pgfqpoint{3.163189in}{3.044686in}}%
\pgfpathlineto{\pgfqpoint{3.136051in}{3.026690in}}%
\pgfpathclose%
\pgfusepath{fill}%
\end{pgfscope}%
\begin{pgfscope}%
\pgfpathrectangle{\pgfqpoint{1.072000in}{0.528000in}}{\pgfqpoint{3.696000in}{3.696000in}}%
\pgfusepath{clip}%
\pgfsetbuttcap%
\pgfsetroundjoin%
\definecolor{currentfill}{rgb}{0.740957,0.122240,0.175744}%
\pgfsetfillcolor{currentfill}%
\pgfsetlinewidth{0.000000pt}%
\definecolor{currentstroke}{rgb}{0.000000,0.000000,0.000000}%
\pgfsetstrokecolor{currentstroke}%
\pgfsetdash{}{0pt}%
\pgfpathmoveto{\pgfqpoint{2.739585in}{2.939734in}}%
\pgfpathlineto{\pgfqpoint{2.787033in}{2.931647in}}%
\pgfpathlineto{\pgfqpoint{2.814671in}{3.012088in}}%
\pgfpathlineto{\pgfqpoint{2.767192in}{3.022586in}}%
\pgfpathlineto{\pgfqpoint{2.739585in}{2.939734in}}%
\pgfpathclose%
\pgfusepath{fill}%
\end{pgfscope}%
\begin{pgfscope}%
\pgfpathrectangle{\pgfqpoint{1.072000in}{0.528000in}}{\pgfqpoint{3.696000in}{3.696000in}}%
\pgfusepath{clip}%
\pgfsetbuttcap%
\pgfsetroundjoin%
\definecolor{currentfill}{rgb}{0.959385,0.610306,0.489382}%
\pgfsetfillcolor{currentfill}%
\pgfsetlinewidth{0.000000pt}%
\definecolor{currentstroke}{rgb}{0.000000,0.000000,0.000000}%
\pgfsetstrokecolor{currentstroke}%
\pgfsetdash{}{0pt}%
\pgfpathmoveto{\pgfqpoint{2.922986in}{2.467192in}}%
\pgfpathlineto{\pgfqpoint{2.970669in}{2.381722in}}%
\pgfpathlineto{\pgfqpoint{2.998239in}{2.556143in}}%
\pgfpathlineto{\pgfqpoint{2.950473in}{2.639027in}}%
\pgfpathlineto{\pgfqpoint{2.922986in}{2.467192in}}%
\pgfpathclose%
\pgfusepath{fill}%
\end{pgfscope}%
\begin{pgfscope}%
\pgfpathrectangle{\pgfqpoint{1.072000in}{0.528000in}}{\pgfqpoint{3.696000in}{3.696000in}}%
\pgfusepath{clip}%
\pgfsetbuttcap%
\pgfsetroundjoin%
\definecolor{currentfill}{rgb}{0.926883,0.505422,0.394866}%
\pgfsetfillcolor{currentfill}%
\pgfsetlinewidth{0.000000pt}%
\definecolor{currentstroke}{rgb}{0.000000,0.000000,0.000000}%
\pgfsetstrokecolor{currentstroke}%
\pgfsetdash{}{0pt}%
\pgfpathmoveto{\pgfqpoint{3.148883in}{2.637731in}}%
\pgfpathlineto{\pgfqpoint{3.196144in}{2.507369in}}%
\pgfpathlineto{\pgfqpoint{3.223871in}{2.632386in}}%
\pgfpathlineto{\pgfqpoint{3.176588in}{2.756161in}}%
\pgfpathlineto{\pgfqpoint{3.148883in}{2.637731in}}%
\pgfpathclose%
\pgfusepath{fill}%
\end{pgfscope}%
\begin{pgfscope}%
\pgfpathrectangle{\pgfqpoint{1.072000in}{0.528000in}}{\pgfqpoint{3.696000in}{3.696000in}}%
\pgfusepath{clip}%
\pgfsetbuttcap%
\pgfsetroundjoin%
\definecolor{currentfill}{rgb}{0.705673,0.015556,0.150233}%
\pgfsetfillcolor{currentfill}%
\pgfsetlinewidth{0.000000pt}%
\definecolor{currentstroke}{rgb}{0.000000,0.000000,0.000000}%
\pgfsetstrokecolor{currentstroke}%
\pgfsetdash{}{0pt}%
\pgfpathmoveto{\pgfqpoint{2.814671in}{3.012088in}}%
\pgfpathlineto{\pgfqpoint{2.862287in}{2.996421in}}%
\pgfpathlineto{\pgfqpoint{2.890012in}{3.033063in}}%
\pgfpathlineto{\pgfqpoint{2.842458in}{3.045130in}}%
\pgfpathlineto{\pgfqpoint{2.814671in}{3.012088in}}%
\pgfpathclose%
\pgfusepath{fill}%
\end{pgfscope}%
\begin{pgfscope}%
\pgfpathrectangle{\pgfqpoint{1.072000in}{0.528000in}}{\pgfqpoint{3.696000in}{3.696000in}}%
\pgfusepath{clip}%
\pgfsetbuttcap%
\pgfsetroundjoin%
\definecolor{currentfill}{rgb}{0.388852,0.516298,0.921373}%
\pgfsetfillcolor{currentfill}%
\pgfsetlinewidth{0.000000pt}%
\definecolor{currentstroke}{rgb}{0.000000,0.000000,0.000000}%
\pgfsetstrokecolor{currentstroke}%
\pgfsetdash{}{0pt}%
\pgfpathmoveto{\pgfqpoint{3.009069in}{1.282594in}}%
\pgfpathlineto{\pgfqpoint{3.057126in}{1.338540in}}%
\pgfpathlineto{\pgfqpoint{3.084181in}{1.254095in}}%
\pgfpathlineto{\pgfqpoint{3.036317in}{1.211371in}}%
\pgfpathlineto{\pgfqpoint{3.009069in}{1.282594in}}%
\pgfpathclose%
\pgfusepath{fill}%
\end{pgfscope}%
\begin{pgfscope}%
\pgfpathrectangle{\pgfqpoint{1.072000in}{0.528000in}}{\pgfqpoint{3.696000in}{3.696000in}}%
\pgfusepath{clip}%
\pgfsetbuttcap%
\pgfsetroundjoin%
\definecolor{currentfill}{rgb}{0.729196,0.086679,0.167240}%
\pgfsetfillcolor{currentfill}%
\pgfsetlinewidth{0.000000pt}%
\definecolor{currentstroke}{rgb}{0.000000,0.000000,0.000000}%
\pgfsetstrokecolor{currentstroke}%
\pgfsetdash{}{0pt}%
\pgfpathmoveto{\pgfqpoint{3.183924in}{2.993230in}}%
\pgfpathlineto{\pgfqpoint{3.231601in}{2.925617in}}%
\pgfpathlineto{\pgfqpoint{3.258843in}{2.981041in}}%
\pgfpathlineto{\pgfqpoint{3.211117in}{3.032232in}}%
\pgfpathlineto{\pgfqpoint{3.183924in}{2.993230in}}%
\pgfpathclose%
\pgfusepath{fill}%
\end{pgfscope}%
\begin{pgfscope}%
\pgfpathrectangle{\pgfqpoint{1.072000in}{0.528000in}}{\pgfqpoint{3.696000in}{3.696000in}}%
\pgfusepath{clip}%
\pgfsetbuttcap%
\pgfsetroundjoin%
\definecolor{currentfill}{rgb}{0.409611,0.540759,0.935545}%
\pgfsetfillcolor{currentfill}%
\pgfsetlinewidth{0.000000pt}%
\definecolor{currentstroke}{rgb}{0.000000,0.000000,0.000000}%
\pgfsetstrokecolor{currentstroke}%
\pgfsetdash{}{0pt}%
\pgfpathmoveto{\pgfqpoint{3.207463in}{1.298490in}}%
\pgfpathlineto{\pgfqpoint{3.256342in}{1.375130in}}%
\pgfpathlineto{\pgfqpoint{3.282404in}{1.297772in}}%
\pgfpathlineto{\pgfqpoint{3.233778in}{1.234998in}}%
\pgfpathlineto{\pgfqpoint{3.207463in}{1.298490in}}%
\pgfpathclose%
\pgfusepath{fill}%
\end{pgfscope}%
\begin{pgfscope}%
\pgfpathrectangle{\pgfqpoint{1.072000in}{0.528000in}}{\pgfqpoint{3.696000in}{3.696000in}}%
\pgfusepath{clip}%
\pgfsetbuttcap%
\pgfsetroundjoin%
\definecolor{currentfill}{rgb}{0.830187,0.304733,0.254891}%
\pgfsetfillcolor{currentfill}%
\pgfsetlinewidth{0.000000pt}%
\definecolor{currentstroke}{rgb}{0.000000,0.000000,0.000000}%
\pgfsetstrokecolor{currentstroke}%
\pgfsetdash{}{0pt}%
\pgfpathmoveto{\pgfqpoint{3.204175in}{2.851702in}}%
\pgfpathlineto{\pgfqpoint{3.251483in}{2.737600in}}%
\pgfpathlineto{\pgfqpoint{3.278937in}{2.823090in}}%
\pgfpathlineto{\pgfqpoint{3.231601in}{2.925617in}}%
\pgfpathlineto{\pgfqpoint{3.204175in}{2.851702in}}%
\pgfpathclose%
\pgfusepath{fill}%
\end{pgfscope}%
\begin{pgfscope}%
\pgfpathrectangle{\pgfqpoint{1.072000in}{0.528000in}}{\pgfqpoint{3.696000in}{3.696000in}}%
\pgfusepath{clip}%
\pgfsetbuttcap%
\pgfsetroundjoin%
\definecolor{currentfill}{rgb}{0.905783,0.455186,0.355336}%
\pgfsetfillcolor{currentfill}%
\pgfsetlinewidth{0.000000pt}%
\definecolor{currentstroke}{rgb}{0.000000,0.000000,0.000000}%
\pgfsetstrokecolor{currentstroke}%
\pgfsetdash{}{0pt}%
\pgfpathmoveto{\pgfqpoint{2.779971in}{2.625690in}}%
\pgfpathlineto{\pgfqpoint{2.827570in}{2.588969in}}%
\pgfpathlineto{\pgfqpoint{2.854882in}{2.753532in}}%
\pgfpathlineto{\pgfqpoint{2.807186in}{2.787558in}}%
\pgfpathlineto{\pgfqpoint{2.779971in}{2.625690in}}%
\pgfpathclose%
\pgfusepath{fill}%
\end{pgfscope}%
\begin{pgfscope}%
\pgfpathrectangle{\pgfqpoint{1.072000in}{0.528000in}}{\pgfqpoint{3.696000in}{3.696000in}}%
\pgfusepath{clip}%
\pgfsetbuttcap%
\pgfsetroundjoin%
\definecolor{currentfill}{rgb}{0.705673,0.015556,0.150233}%
\pgfsetfillcolor{currentfill}%
\pgfsetlinewidth{0.000000pt}%
\definecolor{currentstroke}{rgb}{0.000000,0.000000,0.000000}%
\pgfsetstrokecolor{currentstroke}%
\pgfsetdash{}{0pt}%
\pgfpathmoveto{\pgfqpoint{2.937682in}{3.020930in}}%
\pgfpathlineto{\pgfqpoint{2.985467in}{3.003510in}}%
\pgfpathlineto{\pgfqpoint{3.012985in}{3.025599in}}%
\pgfpathlineto{\pgfqpoint{2.965273in}{3.024192in}}%
\pgfpathlineto{\pgfqpoint{2.937682in}{3.020930in}}%
\pgfpathclose%
\pgfusepath{fill}%
\end{pgfscope}%
\begin{pgfscope}%
\pgfpathrectangle{\pgfqpoint{1.072000in}{0.528000in}}{\pgfqpoint{3.696000in}{3.696000in}}%
\pgfusepath{clip}%
\pgfsetbuttcap%
\pgfsetroundjoin%
\definecolor{currentfill}{rgb}{0.877149,0.394645,0.311724}%
\pgfsetfillcolor{currentfill}%
\pgfsetlinewidth{0.000000pt}%
\definecolor{currentstroke}{rgb}{0.000000,0.000000,0.000000}%
\pgfsetstrokecolor{currentstroke}%
\pgfsetdash{}{0pt}%
\pgfpathmoveto{\pgfqpoint{3.176588in}{2.756161in}}%
\pgfpathlineto{\pgfqpoint{3.223871in}{2.632386in}}%
\pgfpathlineto{\pgfqpoint{3.251483in}{2.737600in}}%
\pgfpathlineto{\pgfqpoint{3.204175in}{2.851702in}}%
\pgfpathlineto{\pgfqpoint{3.176588in}{2.756161in}}%
\pgfpathclose%
\pgfusepath{fill}%
\end{pgfscope}%
\begin{pgfscope}%
\pgfpathrectangle{\pgfqpoint{1.072000in}{0.528000in}}{\pgfqpoint{3.696000in}{3.696000in}}%
\pgfusepath{clip}%
\pgfsetbuttcap%
\pgfsetroundjoin%
\definecolor{currentfill}{rgb}{0.939254,0.539581,0.423900}%
\pgfsetfillcolor{currentfill}%
\pgfsetlinewidth{0.000000pt}%
\definecolor{currentstroke}{rgb}{0.000000,0.000000,0.000000}%
\pgfsetstrokecolor{currentstroke}%
\pgfsetdash{}{0pt}%
\pgfpathmoveto{\pgfqpoint{2.875264in}{2.536269in}}%
\pgfpathlineto{\pgfqpoint{2.922986in}{2.467192in}}%
\pgfpathlineto{\pgfqpoint{2.950473in}{2.639027in}}%
\pgfpathlineto{\pgfqpoint{2.902665in}{2.704469in}}%
\pgfpathlineto{\pgfqpoint{2.875264in}{2.536269in}}%
\pgfpathclose%
\pgfusepath{fill}%
\end{pgfscope}%
\begin{pgfscope}%
\pgfpathrectangle{\pgfqpoint{1.072000in}{0.528000in}}{\pgfqpoint{3.696000in}{3.696000in}}%
\pgfusepath{clip}%
\pgfsetbuttcap%
\pgfsetroundjoin%
\definecolor{currentfill}{rgb}{0.921406,0.491420,0.383408}%
\pgfsetfillcolor{currentfill}%
\pgfsetlinewidth{0.000000pt}%
\definecolor{currentstroke}{rgb}{0.000000,0.000000,0.000000}%
\pgfsetstrokecolor{currentstroke}%
\pgfsetdash{}{0pt}%
\pgfpathmoveto{\pgfqpoint{2.827570in}{2.588969in}}%
\pgfpathlineto{\pgfqpoint{2.875264in}{2.536269in}}%
\pgfpathlineto{\pgfqpoint{2.902665in}{2.704469in}}%
\pgfpathlineto{\pgfqpoint{2.854882in}{2.753532in}}%
\pgfpathlineto{\pgfqpoint{2.827570in}{2.588969in}}%
\pgfpathclose%
\pgfusepath{fill}%
\end{pgfscope}%
\begin{pgfscope}%
\pgfpathrectangle{\pgfqpoint{1.072000in}{0.528000in}}{\pgfqpoint{3.696000in}{3.696000in}}%
\pgfusepath{clip}%
\pgfsetbuttcap%
\pgfsetroundjoin%
\definecolor{currentfill}{rgb}{0.705673,0.015556,0.150233}%
\pgfsetfillcolor{currentfill}%
\pgfsetlinewidth{0.000000pt}%
\definecolor{currentstroke}{rgb}{0.000000,0.000000,0.000000}%
\pgfsetstrokecolor{currentstroke}%
\pgfsetdash{}{0pt}%
\pgfpathmoveto{\pgfqpoint{3.060826in}{3.017118in}}%
\pgfpathlineto{\pgfqpoint{3.108717in}{2.989389in}}%
\pgfpathlineto{\pgfqpoint{3.136051in}{3.026690in}}%
\pgfpathlineto{\pgfqpoint{3.088138in}{3.030631in}}%
\pgfpathlineto{\pgfqpoint{3.060826in}{3.017118in}}%
\pgfpathclose%
\pgfusepath{fill}%
\end{pgfscope}%
\begin{pgfscope}%
\pgfpathrectangle{\pgfqpoint{1.072000in}{0.528000in}}{\pgfqpoint{3.696000in}{3.696000in}}%
\pgfusepath{clip}%
\pgfsetbuttcap%
\pgfsetroundjoin%
\definecolor{currentfill}{rgb}{0.939254,0.539581,0.423900}%
\pgfsetfillcolor{currentfill}%
\pgfsetlinewidth{0.000000pt}%
\definecolor{currentstroke}{rgb}{0.000000,0.000000,0.000000}%
\pgfsetstrokecolor{currentstroke}%
\pgfsetdash{}{0pt}%
\pgfpathmoveto{\pgfqpoint{2.998239in}{2.556143in}}%
\pgfpathlineto{\pgfqpoint{3.045890in}{2.455560in}}%
\pgfpathlineto{\pgfqpoint{3.073607in}{2.612556in}}%
\pgfpathlineto{\pgfqpoint{3.025899in}{2.708098in}}%
\pgfpathlineto{\pgfqpoint{2.998239in}{2.556143in}}%
\pgfpathclose%
\pgfusepath{fill}%
\end{pgfscope}%
\begin{pgfscope}%
\pgfpathrectangle{\pgfqpoint{1.072000in}{0.528000in}}{\pgfqpoint{3.696000in}{3.696000in}}%
\pgfusepath{clip}%
\pgfsetbuttcap%
\pgfsetroundjoin%
\definecolor{currentfill}{rgb}{0.926883,0.505422,0.394866}%
\pgfsetfillcolor{currentfill}%
\pgfsetlinewidth{0.000000pt}%
\definecolor{currentstroke}{rgb}{0.000000,0.000000,0.000000}%
\pgfsetstrokecolor{currentstroke}%
\pgfsetdash{}{0pt}%
\pgfpathmoveto{\pgfqpoint{3.073607in}{2.612556in}}%
\pgfpathlineto{\pgfqpoint{3.121118in}{2.497362in}}%
\pgfpathlineto{\pgfqpoint{3.148883in}{2.637731in}}%
\pgfpathlineto{\pgfqpoint{3.101332in}{2.746344in}}%
\pgfpathlineto{\pgfqpoint{3.073607in}{2.612556in}}%
\pgfpathclose%
\pgfusepath{fill}%
\end{pgfscope}%
\begin{pgfscope}%
\pgfpathrectangle{\pgfqpoint{1.072000in}{0.528000in}}{\pgfqpoint{3.696000in}{3.696000in}}%
\pgfusepath{clip}%
\pgfsetbuttcap%
\pgfsetroundjoin%
\definecolor{currentfill}{rgb}{0.810616,0.268797,0.235428}%
\pgfsetfillcolor{currentfill}%
\pgfsetlinewidth{0.000000pt}%
\definecolor{currentstroke}{rgb}{0.000000,0.000000,0.000000}%
\pgfsetstrokecolor{currentstroke}%
\pgfsetdash{}{0pt}%
\pgfpathmoveto{\pgfqpoint{2.759632in}{2.807638in}}%
\pgfpathlineto{\pgfqpoint{2.807186in}{2.787558in}}%
\pgfpathlineto{\pgfqpoint{2.834647in}{2.913362in}}%
\pgfpathlineto{\pgfqpoint{2.787033in}{2.931647in}}%
\pgfpathlineto{\pgfqpoint{2.759632in}{2.807638in}}%
\pgfpathclose%
\pgfusepath{fill}%
\end{pgfscope}%
\begin{pgfscope}%
\pgfpathrectangle{\pgfqpoint{1.072000in}{0.528000in}}{\pgfqpoint{3.696000in}{3.696000in}}%
\pgfusepath{clip}%
\pgfsetbuttcap%
\pgfsetroundjoin%
\definecolor{currentfill}{rgb}{0.740957,0.122240,0.175744}%
\pgfsetfillcolor{currentfill}%
\pgfsetlinewidth{0.000000pt}%
\definecolor{currentstroke}{rgb}{0.000000,0.000000,0.000000}%
\pgfsetstrokecolor{currentstroke}%
\pgfsetdash{}{0pt}%
\pgfpathmoveto{\pgfqpoint{2.787033in}{2.931647in}}%
\pgfpathlineto{\pgfqpoint{2.834647in}{2.913362in}}%
\pgfpathlineto{\pgfqpoint{2.862287in}{2.996421in}}%
\pgfpathlineto{\pgfqpoint{2.814671in}{3.012088in}}%
\pgfpathlineto{\pgfqpoint{2.787033in}{2.931647in}}%
\pgfpathclose%
\pgfusepath{fill}%
\end{pgfscope}%
\begin{pgfscope}%
\pgfpathrectangle{\pgfqpoint{1.072000in}{0.528000in}}{\pgfqpoint{3.696000in}{3.696000in}}%
\pgfusepath{clip}%
\pgfsetbuttcap%
\pgfsetroundjoin%
\definecolor{currentfill}{rgb}{0.705673,0.015556,0.150233}%
\pgfsetfillcolor{currentfill}%
\pgfsetlinewidth{0.000000pt}%
\definecolor{currentstroke}{rgb}{0.000000,0.000000,0.000000}%
\pgfsetstrokecolor{currentstroke}%
\pgfsetdash{}{0pt}%
\pgfpathmoveto{\pgfqpoint{2.862287in}{2.996421in}}%
\pgfpathlineto{\pgfqpoint{2.910020in}{2.974341in}}%
\pgfpathlineto{\pgfqpoint{2.937682in}{3.020930in}}%
\pgfpathlineto{\pgfqpoint{2.890012in}{3.033063in}}%
\pgfpathlineto{\pgfqpoint{2.862287in}{2.996421in}}%
\pgfpathclose%
\pgfusepath{fill}%
\end{pgfscope}%
\begin{pgfscope}%
\pgfpathrectangle{\pgfqpoint{1.072000in}{0.528000in}}{\pgfqpoint{3.696000in}{3.696000in}}%
\pgfusepath{clip}%
\pgfsetbuttcap%
\pgfsetroundjoin%
\definecolor{currentfill}{rgb}{0.763520,0.178667,0.193396}%
\pgfsetfillcolor{currentfill}%
\pgfsetlinewidth{0.000000pt}%
\definecolor{currentstroke}{rgb}{0.000000,0.000000,0.000000}%
\pgfsetstrokecolor{currentstroke}%
\pgfsetdash{}{0pt}%
\pgfpathmoveto{\pgfqpoint{3.156544in}{2.935497in}}%
\pgfpathlineto{\pgfqpoint{3.204175in}{2.851702in}}%
\pgfpathlineto{\pgfqpoint{3.231601in}{2.925617in}}%
\pgfpathlineto{\pgfqpoint{3.183924in}{2.993230in}}%
\pgfpathlineto{\pgfqpoint{3.156544in}{2.935497in}}%
\pgfpathclose%
\pgfusepath{fill}%
\end{pgfscope}%
\begin{pgfscope}%
\pgfpathrectangle{\pgfqpoint{1.072000in}{0.528000in}}{\pgfqpoint{3.696000in}{3.696000in}}%
\pgfusepath{clip}%
\pgfsetbuttcap%
\pgfsetroundjoin%
\definecolor{currentfill}{rgb}{0.717435,0.051118,0.158737}%
\pgfsetfillcolor{currentfill}%
\pgfsetlinewidth{0.000000pt}%
\definecolor{currentstroke}{rgb}{0.000000,0.000000,0.000000}%
\pgfsetstrokecolor{currentstroke}%
\pgfsetdash{}{0pt}%
\pgfpathmoveto{\pgfqpoint{3.108717in}{2.989389in}}%
\pgfpathlineto{\pgfqpoint{3.156544in}{2.935497in}}%
\pgfpathlineto{\pgfqpoint{3.183924in}{2.993230in}}%
\pgfpathlineto{\pgfqpoint{3.136051in}{3.026690in}}%
\pgfpathlineto{\pgfqpoint{3.108717in}{2.989389in}}%
\pgfpathclose%
\pgfusepath{fill}%
\end{pgfscope}%
\begin{pgfscope}%
\pgfpathrectangle{\pgfqpoint{1.072000in}{0.528000in}}{\pgfqpoint{3.696000in}{3.696000in}}%
\pgfusepath{clip}%
\pgfsetbuttcap%
\pgfsetroundjoin%
\definecolor{currentfill}{rgb}{0.705673,0.015556,0.150233}%
\pgfsetfillcolor{currentfill}%
\pgfsetlinewidth{0.000000pt}%
\definecolor{currentstroke}{rgb}{0.000000,0.000000,0.000000}%
\pgfsetstrokecolor{currentstroke}%
\pgfsetdash{}{0pt}%
\pgfpathmoveto{\pgfqpoint{2.985467in}{3.003510in}}%
\pgfpathlineto{\pgfqpoint{3.033333in}{2.973999in}}%
\pgfpathlineto{\pgfqpoint{3.060826in}{3.017118in}}%
\pgfpathlineto{\pgfqpoint{3.012985in}{3.025599in}}%
\pgfpathlineto{\pgfqpoint{2.985467in}{3.003510in}}%
\pgfpathclose%
\pgfusepath{fill}%
\end{pgfscope}%
\begin{pgfscope}%
\pgfpathrectangle{\pgfqpoint{1.072000in}{0.528000in}}{\pgfqpoint{3.696000in}{3.696000in}}%
\pgfusepath{clip}%
\pgfsetbuttcap%
\pgfsetroundjoin%
\definecolor{currentfill}{rgb}{0.425199,0.559058,0.946061}%
\pgfsetfillcolor{currentfill}%
\pgfsetlinewidth{0.000000pt}%
\definecolor{currentstroke}{rgb}{0.000000,0.000000,0.000000}%
\pgfsetstrokecolor{currentstroke}%
\pgfsetdash{}{0pt}%
\pgfpathmoveto{\pgfqpoint{3.132397in}{1.311668in}}%
\pgfpathlineto{\pgfqpoint{3.181032in}{1.384262in}}%
\pgfpathlineto{\pgfqpoint{3.207463in}{1.298490in}}%
\pgfpathlineto{\pgfqpoint{3.159059in}{1.238861in}}%
\pgfpathlineto{\pgfqpoint{3.132397in}{1.311668in}}%
\pgfpathclose%
\pgfusepath{fill}%
\end{pgfscope}%
\begin{pgfscope}%
\pgfpathrectangle{\pgfqpoint{1.072000in}{0.528000in}}{\pgfqpoint{3.696000in}{3.696000in}}%
\pgfusepath{clip}%
\pgfsetbuttcap%
\pgfsetroundjoin%
\definecolor{currentfill}{rgb}{0.869655,0.379274,0.300941}%
\pgfsetfillcolor{currentfill}%
\pgfsetlinewidth{0.000000pt}%
\definecolor{currentstroke}{rgb}{0.000000,0.000000,0.000000}%
\pgfsetstrokecolor{currentstroke}%
\pgfsetdash{}{0pt}%
\pgfpathmoveto{\pgfqpoint{3.101332in}{2.746344in}}%
\pgfpathlineto{\pgfqpoint{3.148883in}{2.637731in}}%
\pgfpathlineto{\pgfqpoint{3.176588in}{2.756161in}}%
\pgfpathlineto{\pgfqpoint{3.128999in}{2.854053in}}%
\pgfpathlineto{\pgfqpoint{3.101332in}{2.746344in}}%
\pgfpathclose%
\pgfusepath{fill}%
\end{pgfscope}%
\begin{pgfscope}%
\pgfpathrectangle{\pgfqpoint{1.072000in}{0.528000in}}{\pgfqpoint{3.696000in}{3.696000in}}%
\pgfusepath{clip}%
\pgfsetbuttcap%
\pgfsetroundjoin%
\definecolor{currentfill}{rgb}{0.820401,0.286765,0.245160}%
\pgfsetfillcolor{currentfill}%
\pgfsetlinewidth{0.000000pt}%
\definecolor{currentstroke}{rgb}{0.000000,0.000000,0.000000}%
\pgfsetstrokecolor{currentstroke}%
\pgfsetdash{}{0pt}%
\pgfpathmoveto{\pgfqpoint{2.807186in}{2.787558in}}%
\pgfpathlineto{\pgfqpoint{2.854882in}{2.753532in}}%
\pgfpathlineto{\pgfqpoint{2.882389in}{2.884112in}}%
\pgfpathlineto{\pgfqpoint{2.834647in}{2.913362in}}%
\pgfpathlineto{\pgfqpoint{2.807186in}{2.787558in}}%
\pgfpathclose%
\pgfusepath{fill}%
\end{pgfscope}%
\begin{pgfscope}%
\pgfpathrectangle{\pgfqpoint{1.072000in}{0.528000in}}{\pgfqpoint{3.696000in}{3.696000in}}%
\pgfusepath{clip}%
\pgfsetbuttcap%
\pgfsetroundjoin%
\definecolor{currentfill}{rgb}{0.905783,0.455186,0.355336}%
\pgfsetfillcolor{currentfill}%
\pgfsetlinewidth{0.000000pt}%
\definecolor{currentstroke}{rgb}{0.000000,0.000000,0.000000}%
\pgfsetstrokecolor{currentstroke}%
\pgfsetdash{}{0pt}%
\pgfpathmoveto{\pgfqpoint{2.950473in}{2.639027in}}%
\pgfpathlineto{\pgfqpoint{2.998239in}{2.556143in}}%
\pgfpathlineto{\pgfqpoint{3.025899in}{2.708098in}}%
\pgfpathlineto{\pgfqpoint{2.978075in}{2.784030in}}%
\pgfpathlineto{\pgfqpoint{2.950473in}{2.639027in}}%
\pgfpathclose%
\pgfusepath{fill}%
\end{pgfscope}%
\begin{pgfscope}%
\pgfpathrectangle{\pgfqpoint{1.072000in}{0.528000in}}{\pgfqpoint{3.696000in}{3.696000in}}%
\pgfusepath{clip}%
\pgfsetbuttcap%
\pgfsetroundjoin%
\definecolor{currentfill}{rgb}{0.810616,0.268797,0.235428}%
\pgfsetfillcolor{currentfill}%
\pgfsetlinewidth{0.000000pt}%
\definecolor{currentstroke}{rgb}{0.000000,0.000000,0.000000}%
\pgfsetstrokecolor{currentstroke}%
\pgfsetdash{}{0pt}%
\pgfpathmoveto{\pgfqpoint{3.128999in}{2.854053in}}%
\pgfpathlineto{\pgfqpoint{3.176588in}{2.756161in}}%
\pgfpathlineto{\pgfqpoint{3.204175in}{2.851702in}}%
\pgfpathlineto{\pgfqpoint{3.156544in}{2.935497in}}%
\pgfpathlineto{\pgfqpoint{3.128999in}{2.854053in}}%
\pgfpathclose%
\pgfusepath{fill}%
\end{pgfscope}%
\begin{pgfscope}%
\pgfpathrectangle{\pgfqpoint{1.072000in}{0.528000in}}{\pgfqpoint{3.696000in}{3.696000in}}%
\pgfusepath{clip}%
\pgfsetbuttcap%
\pgfsetroundjoin%
\definecolor{currentfill}{rgb}{0.746838,0.140021,0.179996}%
\pgfsetfillcolor{currentfill}%
\pgfsetlinewidth{0.000000pt}%
\definecolor{currentstroke}{rgb}{0.000000,0.000000,0.000000}%
\pgfsetstrokecolor{currentstroke}%
\pgfsetdash{}{0pt}%
\pgfpathmoveto{\pgfqpoint{2.834647in}{2.913362in}}%
\pgfpathlineto{\pgfqpoint{2.882389in}{2.884112in}}%
\pgfpathlineto{\pgfqpoint{2.910020in}{2.974341in}}%
\pgfpathlineto{\pgfqpoint{2.862287in}{2.996421in}}%
\pgfpathlineto{\pgfqpoint{2.834647in}{2.913362in}}%
\pgfpathclose%
\pgfusepath{fill}%
\end{pgfscope}%
\begin{pgfscope}%
\pgfpathrectangle{\pgfqpoint{1.072000in}{0.528000in}}{\pgfqpoint{3.696000in}{3.696000in}}%
\pgfusepath{clip}%
\pgfsetbuttcap%
\pgfsetroundjoin%
\definecolor{currentfill}{rgb}{0.711554,0.033337,0.154485}%
\pgfsetfillcolor{currentfill}%
\pgfsetlinewidth{0.000000pt}%
\definecolor{currentstroke}{rgb}{0.000000,0.000000,0.000000}%
\pgfsetstrokecolor{currentstroke}%
\pgfsetdash{}{0pt}%
\pgfpathmoveto{\pgfqpoint{2.910020in}{2.974341in}}%
\pgfpathlineto{\pgfqpoint{2.957846in}{2.942527in}}%
\pgfpathlineto{\pgfqpoint{2.985467in}{3.003510in}}%
\pgfpathlineto{\pgfqpoint{2.937682in}{3.020930in}}%
\pgfpathlineto{\pgfqpoint{2.910020in}{2.974341in}}%
\pgfpathclose%
\pgfusepath{fill}%
\end{pgfscope}%
\begin{pgfscope}%
\pgfpathrectangle{\pgfqpoint{1.072000in}{0.528000in}}{\pgfqpoint{3.696000in}{3.696000in}}%
\pgfusepath{clip}%
\pgfsetbuttcap%
\pgfsetroundjoin%
\definecolor{currentfill}{rgb}{0.877149,0.394645,0.311724}%
\pgfsetfillcolor{currentfill}%
\pgfsetlinewidth{0.000000pt}%
\definecolor{currentstroke}{rgb}{0.000000,0.000000,0.000000}%
\pgfsetstrokecolor{currentstroke}%
\pgfsetdash{}{0pt}%
\pgfpathmoveto{\pgfqpoint{3.025899in}{2.708098in}}%
\pgfpathlineto{\pgfqpoint{3.073607in}{2.612556in}}%
\pgfpathlineto{\pgfqpoint{3.101332in}{2.746344in}}%
\pgfpathlineto{\pgfqpoint{3.053580in}{2.832218in}}%
\pgfpathlineto{\pgfqpoint{3.025899in}{2.708098in}}%
\pgfpathclose%
\pgfusepath{fill}%
\end{pgfscope}%
\begin{pgfscope}%
\pgfpathrectangle{\pgfqpoint{1.072000in}{0.528000in}}{\pgfqpoint{3.696000in}{3.696000in}}%
\pgfusepath{clip}%
\pgfsetbuttcap%
\pgfsetroundjoin%
\definecolor{currentfill}{rgb}{0.839365,0.321856,0.264924}%
\pgfsetfillcolor{currentfill}%
\pgfsetlinewidth{0.000000pt}%
\definecolor{currentstroke}{rgb}{0.000000,0.000000,0.000000}%
\pgfsetstrokecolor{currentstroke}%
\pgfsetdash{}{0pt}%
\pgfpathmoveto{\pgfqpoint{2.854882in}{2.753532in}}%
\pgfpathlineto{\pgfqpoint{2.902665in}{2.704469in}}%
\pgfpathlineto{\pgfqpoint{2.930216in}{2.841881in}}%
\pgfpathlineto{\pgfqpoint{2.882389in}{2.884112in}}%
\pgfpathlineto{\pgfqpoint{2.854882in}{2.753532in}}%
\pgfpathclose%
\pgfusepath{fill}%
\end{pgfscope}%
\begin{pgfscope}%
\pgfpathrectangle{\pgfqpoint{1.072000in}{0.528000in}}{\pgfqpoint{3.696000in}{3.696000in}}%
\pgfusepath{clip}%
\pgfsetbuttcap%
\pgfsetroundjoin%
\definecolor{currentfill}{rgb}{0.869655,0.379274,0.300941}%
\pgfsetfillcolor{currentfill}%
\pgfsetlinewidth{0.000000pt}%
\definecolor{currentstroke}{rgb}{0.000000,0.000000,0.000000}%
\pgfsetstrokecolor{currentstroke}%
\pgfsetdash{}{0pt}%
\pgfpathmoveto{\pgfqpoint{2.902665in}{2.704469in}}%
\pgfpathlineto{\pgfqpoint{2.950473in}{2.639027in}}%
\pgfpathlineto{\pgfqpoint{2.978075in}{2.784030in}}%
\pgfpathlineto{\pgfqpoint{2.930216in}{2.841881in}}%
\pgfpathlineto{\pgfqpoint{2.902665in}{2.704469in}}%
\pgfpathclose%
\pgfusepath{fill}%
\end{pgfscope}%
\begin{pgfscope}%
\pgfpathrectangle{\pgfqpoint{1.072000in}{0.528000in}}{\pgfqpoint{3.696000in}{3.696000in}}%
\pgfusepath{clip}%
\pgfsetbuttcap%
\pgfsetroundjoin%
\definecolor{currentfill}{rgb}{0.717435,0.051118,0.158737}%
\pgfsetfillcolor{currentfill}%
\pgfsetlinewidth{0.000000pt}%
\definecolor{currentstroke}{rgb}{0.000000,0.000000,0.000000}%
\pgfsetstrokecolor{currentstroke}%
\pgfsetdash{}{0pt}%
\pgfpathmoveto{\pgfqpoint{3.033333in}{2.973999in}}%
\pgfpathlineto{\pgfqpoint{3.081209in}{2.925829in}}%
\pgfpathlineto{\pgfqpoint{3.108717in}{2.989389in}}%
\pgfpathlineto{\pgfqpoint{3.060826in}{3.017118in}}%
\pgfpathlineto{\pgfqpoint{3.033333in}{2.973999in}}%
\pgfpathclose%
\pgfusepath{fill}%
\end{pgfscope}%
\begin{pgfscope}%
\pgfpathrectangle{\pgfqpoint{1.072000in}{0.528000in}}{\pgfqpoint{3.696000in}{3.696000in}}%
\pgfusepath{clip}%
\pgfsetbuttcap%
\pgfsetroundjoin%
\definecolor{currentfill}{rgb}{0.430507,0.564883,0.948889}%
\pgfsetfillcolor{currentfill}%
\pgfsetlinewidth{0.000000pt}%
\definecolor{currentstroke}{rgb}{0.000000,0.000000,0.000000}%
\pgfsetstrokecolor{currentstroke}%
\pgfsetdash{}{0pt}%
\pgfpathmoveto{\pgfqpoint{2.933630in}{1.326565in}}%
\pgfpathlineto{\pgfqpoint{2.981557in}{1.380647in}}%
\pgfpathlineto{\pgfqpoint{3.009069in}{1.282594in}}%
\pgfpathlineto{\pgfqpoint{2.961326in}{1.239545in}}%
\pgfpathlineto{\pgfqpoint{2.933630in}{1.326565in}}%
\pgfpathclose%
\pgfusepath{fill}%
\end{pgfscope}%
\begin{pgfscope}%
\pgfpathrectangle{\pgfqpoint{1.072000in}{0.528000in}}{\pgfqpoint{3.696000in}{3.696000in}}%
\pgfusepath{clip}%
\pgfsetbuttcap%
\pgfsetroundjoin%
\definecolor{currentfill}{rgb}{0.752704,0.157576,0.184258}%
\pgfsetfillcolor{currentfill}%
\pgfsetlinewidth{0.000000pt}%
\definecolor{currentstroke}{rgb}{0.000000,0.000000,0.000000}%
\pgfsetstrokecolor{currentstroke}%
\pgfsetdash{}{0pt}%
\pgfpathmoveto{\pgfqpoint{3.081209in}{2.925829in}}%
\pgfpathlineto{\pgfqpoint{3.128999in}{2.854053in}}%
\pgfpathlineto{\pgfqpoint{3.156544in}{2.935497in}}%
\pgfpathlineto{\pgfqpoint{3.108717in}{2.989389in}}%
\pgfpathlineto{\pgfqpoint{3.081209in}{2.925829in}}%
\pgfpathclose%
\pgfusepath{fill}%
\end{pgfscope}%
\begin{pgfscope}%
\pgfpathrectangle{\pgfqpoint{1.072000in}{0.528000in}}{\pgfqpoint{3.696000in}{3.696000in}}%
\pgfusepath{clip}%
\pgfsetbuttcap%
\pgfsetroundjoin%
\definecolor{currentfill}{rgb}{0.810616,0.268797,0.235428}%
\pgfsetfillcolor{currentfill}%
\pgfsetlinewidth{0.000000pt}%
\definecolor{currentstroke}{rgb}{0.000000,0.000000,0.000000}%
\pgfsetstrokecolor{currentstroke}%
\pgfsetdash{}{0pt}%
\pgfpathmoveto{\pgfqpoint{3.053580in}{2.832218in}}%
\pgfpathlineto{\pgfqpoint{3.101332in}{2.746344in}}%
\pgfpathlineto{\pgfqpoint{3.128999in}{2.854053in}}%
\pgfpathlineto{\pgfqpoint{3.081209in}{2.925829in}}%
\pgfpathlineto{\pgfqpoint{3.053580in}{2.832218in}}%
\pgfpathclose%
\pgfusepath{fill}%
\end{pgfscope}%
\begin{pgfscope}%
\pgfpathrectangle{\pgfqpoint{1.072000in}{0.528000in}}{\pgfqpoint{3.696000in}{3.696000in}}%
\pgfusepath{clip}%
\pgfsetbuttcap%
\pgfsetroundjoin%
\definecolor{currentfill}{rgb}{0.763520,0.178667,0.193396}%
\pgfsetfillcolor{currentfill}%
\pgfsetlinewidth{0.000000pt}%
\definecolor{currentstroke}{rgb}{0.000000,0.000000,0.000000}%
\pgfsetstrokecolor{currentstroke}%
\pgfsetdash{}{0pt}%
\pgfpathmoveto{\pgfqpoint{2.882389in}{2.884112in}}%
\pgfpathlineto{\pgfqpoint{2.930216in}{2.841881in}}%
\pgfpathlineto{\pgfqpoint{2.957846in}{2.942527in}}%
\pgfpathlineto{\pgfqpoint{2.910020in}{2.974341in}}%
\pgfpathlineto{\pgfqpoint{2.882389in}{2.884112in}}%
\pgfpathclose%
\pgfusepath{fill}%
\end{pgfscope}%
\begin{pgfscope}%
\pgfpathrectangle{\pgfqpoint{1.072000in}{0.528000in}}{\pgfqpoint{3.696000in}{3.696000in}}%
\pgfusepath{clip}%
\pgfsetbuttcap%
\pgfsetroundjoin%
\definecolor{currentfill}{rgb}{0.532568,0.669801,0.990393}%
\pgfsetfillcolor{currentfill}%
\pgfsetlinewidth{0.000000pt}%
\definecolor{currentstroke}{rgb}{0.000000,0.000000,0.000000}%
\pgfsetstrokecolor{currentstroke}%
\pgfsetdash{}{0pt}%
\pgfpathmoveto{\pgfqpoint{3.482167in}{1.428358in}}%
\pgfpathlineto{\pgfqpoint{3.532872in}{1.544405in}}%
\pgfpathlineto{\pgfqpoint{3.557721in}{1.473662in}}%
\pgfpathlineto{\pgfqpoint{3.507272in}{1.366847in}}%
\pgfpathlineto{\pgfqpoint{3.482167in}{1.428358in}}%
\pgfpathclose%
\pgfusepath{fill}%
\end{pgfscope}%
\begin{pgfscope}%
\pgfpathrectangle{\pgfqpoint{1.072000in}{0.528000in}}{\pgfqpoint{3.696000in}{3.696000in}}%
\pgfusepath{clip}%
\pgfsetbuttcap%
\pgfsetroundjoin%
\definecolor{currentfill}{rgb}{0.729196,0.086679,0.167240}%
\pgfsetfillcolor{currentfill}%
\pgfsetlinewidth{0.000000pt}%
\definecolor{currentstroke}{rgb}{0.000000,0.000000,0.000000}%
\pgfsetstrokecolor{currentstroke}%
\pgfsetdash{}{0pt}%
\pgfpathmoveto{\pgfqpoint{2.957846in}{2.942527in}}%
\pgfpathlineto{\pgfqpoint{3.005723in}{2.896577in}}%
\pgfpathlineto{\pgfqpoint{3.033333in}{2.973999in}}%
\pgfpathlineto{\pgfqpoint{2.985467in}{3.003510in}}%
\pgfpathlineto{\pgfqpoint{2.957846in}{2.942527in}}%
\pgfpathclose%
\pgfusepath{fill}%
\end{pgfscope}%
\begin{pgfscope}%
\pgfpathrectangle{\pgfqpoint{1.072000in}{0.528000in}}{\pgfqpoint{3.696000in}{3.696000in}}%
\pgfusepath{clip}%
\pgfsetbuttcap%
\pgfsetroundjoin%
\definecolor{currentfill}{rgb}{0.505423,0.643995,0.983157}%
\pgfsetfillcolor{currentfill}%
\pgfsetlinewidth{0.000000pt}%
\definecolor{currentstroke}{rgb}{0.000000,0.000000,0.000000}%
\pgfsetstrokecolor{currentstroke}%
\pgfsetdash{}{0pt}%
\pgfpathmoveto{\pgfqpoint{3.406819in}{1.397200in}}%
\pgfpathlineto{\pgfqpoint{3.457025in}{1.504282in}}%
\pgfpathlineto{\pgfqpoint{3.482167in}{1.428358in}}%
\pgfpathlineto{\pgfqpoint{3.432225in}{1.331637in}}%
\pgfpathlineto{\pgfqpoint{3.406819in}{1.397200in}}%
\pgfpathclose%
\pgfusepath{fill}%
\end{pgfscope}%
\begin{pgfscope}%
\pgfpathrectangle{\pgfqpoint{1.072000in}{0.528000in}}{\pgfqpoint{3.696000in}{3.696000in}}%
\pgfusepath{clip}%
\pgfsetbuttcap%
\pgfsetroundjoin%
\definecolor{currentfill}{rgb}{0.830187,0.304733,0.254891}%
\pgfsetfillcolor{currentfill}%
\pgfsetlinewidth{0.000000pt}%
\definecolor{currentstroke}{rgb}{0.000000,0.000000,0.000000}%
\pgfsetstrokecolor{currentstroke}%
\pgfsetdash{}{0pt}%
\pgfpathmoveto{\pgfqpoint{2.978075in}{2.784030in}}%
\pgfpathlineto{\pgfqpoint{3.025899in}{2.708098in}}%
\pgfpathlineto{\pgfqpoint{3.053580in}{2.832218in}}%
\pgfpathlineto{\pgfqpoint{3.005723in}{2.896577in}}%
\pgfpathlineto{\pgfqpoint{2.978075in}{2.784030in}}%
\pgfpathclose%
\pgfusepath{fill}%
\end{pgfscope}%
\begin{pgfscope}%
\pgfpathrectangle{\pgfqpoint{1.072000in}{0.528000in}}{\pgfqpoint{3.696000in}{3.696000in}}%
\pgfusepath{clip}%
\pgfsetbuttcap%
\pgfsetroundjoin%
\definecolor{currentfill}{rgb}{0.790562,0.231397,0.216242}%
\pgfsetfillcolor{currentfill}%
\pgfsetlinewidth{0.000000pt}%
\definecolor{currentstroke}{rgb}{0.000000,0.000000,0.000000}%
\pgfsetstrokecolor{currentstroke}%
\pgfsetdash{}{0pt}%
\pgfpathmoveto{\pgfqpoint{2.930216in}{2.841881in}}%
\pgfpathlineto{\pgfqpoint{2.978075in}{2.784030in}}%
\pgfpathlineto{\pgfqpoint{3.005723in}{2.896577in}}%
\pgfpathlineto{\pgfqpoint{2.957846in}{2.942527in}}%
\pgfpathlineto{\pgfqpoint{2.930216in}{2.841881in}}%
\pgfpathclose%
\pgfusepath{fill}%
\end{pgfscope}%
\begin{pgfscope}%
\pgfpathrectangle{\pgfqpoint{1.072000in}{0.528000in}}{\pgfqpoint{3.696000in}{3.696000in}}%
\pgfusepath{clip}%
\pgfsetbuttcap%
\pgfsetroundjoin%
\definecolor{currentfill}{rgb}{0.763520,0.178667,0.193396}%
\pgfsetfillcolor{currentfill}%
\pgfsetlinewidth{0.000000pt}%
\definecolor{currentstroke}{rgb}{0.000000,0.000000,0.000000}%
\pgfsetstrokecolor{currentstroke}%
\pgfsetdash{}{0pt}%
\pgfpathmoveto{\pgfqpoint{3.005723in}{2.896577in}}%
\pgfpathlineto{\pgfqpoint{3.053580in}{2.832218in}}%
\pgfpathlineto{\pgfqpoint{3.081209in}{2.925829in}}%
\pgfpathlineto{\pgfqpoint{3.033333in}{2.973999in}}%
\pgfpathlineto{\pgfqpoint{3.005723in}{2.896577in}}%
\pgfpathclose%
\pgfusepath{fill}%
\end{pgfscope}%
\begin{pgfscope}%
\pgfpathrectangle{\pgfqpoint{1.072000in}{0.528000in}}{\pgfqpoint{3.696000in}{3.696000in}}%
\pgfusepath{clip}%
\pgfsetbuttcap%
\pgfsetroundjoin%
\definecolor{currentfill}{rgb}{0.494638,0.633022,0.978983}%
\pgfsetfillcolor{currentfill}%
\pgfsetlinewidth{0.000000pt}%
\definecolor{currentstroke}{rgb}{0.000000,0.000000,0.000000}%
\pgfsetstrokecolor{currentstroke}%
\pgfsetdash{}{0pt}%
\pgfpathmoveto{\pgfqpoint{3.331575in}{1.379532in}}%
\pgfpathlineto{\pgfqpoint{3.381356in}{1.478849in}}%
\pgfpathlineto{\pgfqpoint{3.406819in}{1.397200in}}%
\pgfpathlineto{\pgfqpoint{3.357299in}{1.308828in}}%
\pgfpathlineto{\pgfqpoint{3.331575in}{1.379532in}}%
\pgfpathclose%
\pgfusepath{fill}%
\end{pgfscope}%
\begin{pgfscope}%
\pgfpathrectangle{\pgfqpoint{1.072000in}{0.528000in}}{\pgfqpoint{3.696000in}{3.696000in}}%
\pgfusepath{clip}%
\pgfsetbuttcap%
\pgfsetroundjoin%
\definecolor{currentfill}{rgb}{0.451739,0.588181,0.960201}%
\pgfsetfillcolor{currentfill}%
\pgfsetlinewidth{0.000000pt}%
\definecolor{currentstroke}{rgb}{0.000000,0.000000,0.000000}%
\pgfsetstrokecolor{currentstroke}%
\pgfsetdash{}{0pt}%
\pgfpathmoveto{\pgfqpoint{3.057126in}{1.338540in}}%
\pgfpathlineto{\pgfqpoint{3.105556in}{1.407631in}}%
\pgfpathlineto{\pgfqpoint{3.132397in}{1.311668in}}%
\pgfpathlineto{\pgfqpoint{3.084181in}{1.254095in}}%
\pgfpathlineto{\pgfqpoint{3.057126in}{1.338540in}}%
\pgfpathclose%
\pgfusepath{fill}%
\end{pgfscope}%
\begin{pgfscope}%
\pgfpathrectangle{\pgfqpoint{1.072000in}{0.528000in}}{\pgfqpoint{3.696000in}{3.696000in}}%
\pgfusepath{clip}%
\pgfsetbuttcap%
\pgfsetroundjoin%
\definecolor{currentfill}{rgb}{0.494638,0.633022,0.978983}%
\pgfsetfillcolor{currentfill}%
\pgfsetlinewidth{0.000000pt}%
\definecolor{currentstroke}{rgb}{0.000000,0.000000,0.000000}%
\pgfsetstrokecolor{currentstroke}%
\pgfsetdash{}{0pt}%
\pgfpathmoveto{\pgfqpoint{3.256342in}{1.375130in}}%
\pgfpathlineto{\pgfqpoint{3.305762in}{1.467837in}}%
\pgfpathlineto{\pgfqpoint{3.331575in}{1.379532in}}%
\pgfpathlineto{\pgfqpoint{3.282404in}{1.297772in}}%
\pgfpathlineto{\pgfqpoint{3.256342in}{1.375130in}}%
\pgfpathclose%
\pgfusepath{fill}%
\end{pgfscope}%
\begin{pgfscope}%
\pgfpathrectangle{\pgfqpoint{1.072000in}{0.528000in}}{\pgfqpoint{3.696000in}{3.696000in}}%
\pgfusepath{clip}%
\pgfsetbuttcap%
\pgfsetroundjoin%
\definecolor{currentfill}{rgb}{0.505423,0.643995,0.983157}%
\pgfsetfillcolor{currentfill}%
\pgfsetlinewidth{0.000000pt}%
\definecolor{currentstroke}{rgb}{0.000000,0.000000,0.000000}%
\pgfsetstrokecolor{currentstroke}%
\pgfsetdash{}{0pt}%
\pgfpathmoveto{\pgfqpoint{3.181032in}{1.384262in}}%
\pgfpathlineto{\pgfqpoint{3.230147in}{1.471271in}}%
\pgfpathlineto{\pgfqpoint{3.256342in}{1.375130in}}%
\pgfpathlineto{\pgfqpoint{3.207463in}{1.298490in}}%
\pgfpathlineto{\pgfqpoint{3.181032in}{1.384262in}}%
\pgfpathclose%
\pgfusepath{fill}%
\end{pgfscope}%
\begin{pgfscope}%
\pgfpathrectangle{\pgfqpoint{1.072000in}{0.528000in}}{\pgfqpoint{3.696000in}{3.696000in}}%
\pgfusepath{clip}%
\pgfsetbuttcap%
\pgfsetroundjoin%
\definecolor{currentfill}{rgb}{0.494638,0.633022,0.978983}%
\pgfsetfillcolor{currentfill}%
\pgfsetlinewidth{0.000000pt}%
\definecolor{currentstroke}{rgb}{0.000000,0.000000,0.000000}%
\pgfsetstrokecolor{currentstroke}%
\pgfsetdash{}{0pt}%
\pgfpathmoveto{\pgfqpoint{2.981557in}{1.380647in}}%
\pgfpathlineto{\pgfqpoint{3.029818in}{1.446202in}}%
\pgfpathlineto{\pgfqpoint{3.057126in}{1.338540in}}%
\pgfpathlineto{\pgfqpoint{3.009069in}{1.282594in}}%
\pgfpathlineto{\pgfqpoint{2.981557in}{1.380647in}}%
\pgfpathclose%
\pgfusepath{fill}%
\end{pgfscope}%
\begin{pgfscope}%
\pgfpathrectangle{\pgfqpoint{1.072000in}{0.528000in}}{\pgfqpoint{3.696000in}{3.696000in}}%
\pgfusepath{clip}%
\pgfsetbuttcap%
\pgfsetroundjoin%
\definecolor{currentfill}{rgb}{0.532568,0.669801,0.990393}%
\pgfsetfillcolor{currentfill}%
\pgfsetlinewidth{0.000000pt}%
\definecolor{currentstroke}{rgb}{0.000000,0.000000,0.000000}%
\pgfsetstrokecolor{currentstroke}%
\pgfsetdash{}{0pt}%
\pgfpathmoveto{\pgfqpoint{3.105556in}{1.407631in}}%
\pgfpathlineto{\pgfqpoint{3.154414in}{1.489467in}}%
\pgfpathlineto{\pgfqpoint{3.181032in}{1.384262in}}%
\pgfpathlineto{\pgfqpoint{3.132397in}{1.311668in}}%
\pgfpathlineto{\pgfqpoint{3.105556in}{1.407631in}}%
\pgfpathclose%
\pgfusepath{fill}%
\end{pgfscope}%
\begin{pgfscope}%
\pgfpathrectangle{\pgfqpoint{1.072000in}{0.528000in}}{\pgfqpoint{3.696000in}{3.696000in}}%
\pgfusepath{clip}%
\pgfsetbuttcap%
\pgfsetroundjoin%
\definecolor{currentfill}{rgb}{0.624703,0.748318,0.998719}%
\pgfsetfillcolor{currentfill}%
\pgfsetlinewidth{0.000000pt}%
\definecolor{currentstroke}{rgb}{0.000000,0.000000,0.000000}%
\pgfsetstrokecolor{currentstroke}%
\pgfsetdash{}{0pt}%
\pgfpathmoveto{\pgfqpoint{3.457025in}{1.504282in}}%
\pgfpathlineto{\pgfqpoint{3.507949in}{1.626756in}}%
\pgfpathlineto{\pgfqpoint{3.532872in}{1.544405in}}%
\pgfpathlineto{\pgfqpoint{3.482167in}{1.428358in}}%
\pgfpathlineto{\pgfqpoint{3.457025in}{1.504282in}}%
\pgfpathclose%
\pgfusepath{fill}%
\end{pgfscope}%
\begin{pgfscope}%
\pgfpathrectangle{\pgfqpoint{1.072000in}{0.528000in}}{\pgfqpoint{3.696000in}{3.696000in}}%
\pgfusepath{clip}%
\pgfsetbuttcap%
\pgfsetroundjoin%
\definecolor{currentfill}{rgb}{0.603162,0.731527,0.999565}%
\pgfsetfillcolor{currentfill}%
\pgfsetlinewidth{0.000000pt}%
\definecolor{currentstroke}{rgb}{0.000000,0.000000,0.000000}%
\pgfsetstrokecolor{currentstroke}%
\pgfsetdash{}{0pt}%
\pgfpathmoveto{\pgfqpoint{3.381356in}{1.478849in}}%
\pgfpathlineto{\pgfqpoint{3.431791in}{1.593167in}}%
\pgfpathlineto{\pgfqpoint{3.457025in}{1.504282in}}%
\pgfpathlineto{\pgfqpoint{3.406819in}{1.397200in}}%
\pgfpathlineto{\pgfqpoint{3.381356in}{1.478849in}}%
\pgfpathclose%
\pgfusepath{fill}%
\end{pgfscope}%
\begin{pgfscope}%
\pgfpathrectangle{\pgfqpoint{1.072000in}{0.528000in}}{\pgfqpoint{3.696000in}{3.696000in}}%
\pgfusepath{clip}%
\pgfsetbuttcap%
\pgfsetroundjoin%
\definecolor{currentfill}{rgb}{0.597777,0.727330,0.999777}%
\pgfsetfillcolor{currentfill}%
\pgfsetlinewidth{0.000000pt}%
\definecolor{currentstroke}{rgb}{0.000000,0.000000,0.000000}%
\pgfsetstrokecolor{currentstroke}%
\pgfsetdash{}{0pt}%
\pgfpathmoveto{\pgfqpoint{3.305762in}{1.467837in}}%
\pgfpathlineto{\pgfqpoint{3.355774in}{1.574666in}}%
\pgfpathlineto{\pgfqpoint{3.381356in}{1.478849in}}%
\pgfpathlineto{\pgfqpoint{3.331575in}{1.379532in}}%
\pgfpathlineto{\pgfqpoint{3.305762in}{1.467837in}}%
\pgfpathclose%
\pgfusepath{fill}%
\end{pgfscope}%
\begin{pgfscope}%
\pgfpathrectangle{\pgfqpoint{1.072000in}{0.528000in}}{\pgfqpoint{3.696000in}{3.696000in}}%
\pgfusepath{clip}%
\pgfsetbuttcap%
\pgfsetroundjoin%
\definecolor{currentfill}{rgb}{0.570616,0.704109,0.997195}%
\pgfsetfillcolor{currentfill}%
\pgfsetlinewidth{0.000000pt}%
\definecolor{currentstroke}{rgb}{0.000000,0.000000,0.000000}%
\pgfsetstrokecolor{currentstroke}%
\pgfsetdash{}{0pt}%
\pgfpathmoveto{\pgfqpoint{3.029818in}{1.446202in}}%
\pgfpathlineto{\pgfqpoint{3.078463in}{1.522928in}}%
\pgfpathlineto{\pgfqpoint{3.105556in}{1.407631in}}%
\pgfpathlineto{\pgfqpoint{3.057126in}{1.338540in}}%
\pgfpathlineto{\pgfqpoint{3.029818in}{1.446202in}}%
\pgfpathclose%
\pgfusepath{fill}%
\end{pgfscope}%
\begin{pgfscope}%
\pgfpathrectangle{\pgfqpoint{1.072000in}{0.528000in}}{\pgfqpoint{3.696000in}{3.696000in}}%
\pgfusepath{clip}%
\pgfsetbuttcap%
\pgfsetroundjoin%
\definecolor{currentfill}{rgb}{0.603162,0.731527,0.999565}%
\pgfsetfillcolor{currentfill}%
\pgfsetlinewidth{0.000000pt}%
\definecolor{currentstroke}{rgb}{0.000000,0.000000,0.000000}%
\pgfsetstrokecolor{currentstroke}%
\pgfsetdash{}{0pt}%
\pgfpathmoveto{\pgfqpoint{3.230147in}{1.471271in}}%
\pgfpathlineto{\pgfqpoint{3.279794in}{1.571215in}}%
\pgfpathlineto{\pgfqpoint{3.305762in}{1.467837in}}%
\pgfpathlineto{\pgfqpoint{3.256342in}{1.375130in}}%
\pgfpathlineto{\pgfqpoint{3.230147in}{1.471271in}}%
\pgfpathclose%
\pgfusepath{fill}%
\end{pgfscope}%
\begin{pgfscope}%
\pgfpathrectangle{\pgfqpoint{1.072000in}{0.528000in}}{\pgfqpoint{3.696000in}{3.696000in}}%
\pgfusepath{clip}%
\pgfsetbuttcap%
\pgfsetroundjoin%
\definecolor{currentfill}{rgb}{0.624703,0.748318,0.998719}%
\pgfsetfillcolor{currentfill}%
\pgfsetlinewidth{0.000000pt}%
\definecolor{currentstroke}{rgb}{0.000000,0.000000,0.000000}%
\pgfsetstrokecolor{currentstroke}%
\pgfsetdash{}{0pt}%
\pgfpathmoveto{\pgfqpoint{3.154414in}{1.489467in}}%
\pgfpathlineto{\pgfqpoint{3.203749in}{1.582903in}}%
\pgfpathlineto{\pgfqpoint{3.230147in}{1.471271in}}%
\pgfpathlineto{\pgfqpoint{3.181032in}{1.384262in}}%
\pgfpathlineto{\pgfqpoint{3.154414in}{1.489467in}}%
\pgfpathclose%
\pgfusepath{fill}%
\end{pgfscope}%
\begin{pgfscope}%
\pgfpathrectangle{\pgfqpoint{1.072000in}{0.528000in}}{\pgfqpoint{3.696000in}{3.696000in}}%
\pgfusepath{clip}%
\pgfsetbuttcap%
\pgfsetroundjoin%
\definecolor{currentfill}{rgb}{0.724041,0.814910,0.975651}%
\pgfsetfillcolor{currentfill}%
\pgfsetlinewidth{0.000000pt}%
\definecolor{currentstroke}{rgb}{0.000000,0.000000,0.000000}%
\pgfsetstrokecolor{currentstroke}%
\pgfsetdash{}{0pt}%
\pgfpathmoveto{\pgfqpoint{3.431791in}{1.593167in}}%
\pgfpathlineto{\pgfqpoint{3.482892in}{1.718789in}}%
\pgfpathlineto{\pgfqpoint{3.507949in}{1.626756in}}%
\pgfpathlineto{\pgfqpoint{3.457025in}{1.504282in}}%
\pgfpathlineto{\pgfqpoint{3.431791in}{1.593167in}}%
\pgfpathclose%
\pgfusepath{fill}%
\end{pgfscope}%
\begin{pgfscope}%
\pgfpathrectangle{\pgfqpoint{1.072000in}{0.528000in}}{\pgfqpoint{3.696000in}{3.696000in}}%
\pgfusepath{clip}%
\pgfsetbuttcap%
\pgfsetroundjoin%
\definecolor{currentfill}{rgb}{0.656683,0.771806,0.994914}%
\pgfsetfillcolor{currentfill}%
\pgfsetlinewidth{0.000000pt}%
\definecolor{currentstroke}{rgb}{0.000000,0.000000,0.000000}%
\pgfsetstrokecolor{currentstroke}%
\pgfsetdash{}{0pt}%
\pgfpathmoveto{\pgfqpoint{3.078463in}{1.522928in}}%
\pgfpathlineto{\pgfqpoint{3.127539in}{1.609902in}}%
\pgfpathlineto{\pgfqpoint{3.154414in}{1.489467in}}%
\pgfpathlineto{\pgfqpoint{3.105556in}{1.407631in}}%
\pgfpathlineto{\pgfqpoint{3.078463in}{1.522928in}}%
\pgfpathclose%
\pgfusepath{fill}%
\end{pgfscope}%
\begin{pgfscope}%
\pgfpathrectangle{\pgfqpoint{1.072000in}{0.528000in}}{\pgfqpoint{3.696000in}{3.696000in}}%
\pgfusepath{clip}%
\pgfsetbuttcap%
\pgfsetroundjoin%
\definecolor{currentfill}{rgb}{0.713852,0.808857,0.979386}%
\pgfsetfillcolor{currentfill}%
\pgfsetlinewidth{0.000000pt}%
\definecolor{currentstroke}{rgb}{0.000000,0.000000,0.000000}%
\pgfsetstrokecolor{currentstroke}%
\pgfsetdash{}{0pt}%
\pgfpathmoveto{\pgfqpoint{3.355774in}{1.574666in}}%
\pgfpathlineto{\pgfqpoint{3.406401in}{1.692634in}}%
\pgfpathlineto{\pgfqpoint{3.431791in}{1.593167in}}%
\pgfpathlineto{\pgfqpoint{3.381356in}{1.478849in}}%
\pgfpathlineto{\pgfqpoint{3.355774in}{1.574666in}}%
\pgfpathclose%
\pgfusepath{fill}%
\end{pgfscope}%
\begin{pgfscope}%
\pgfpathrectangle{\pgfqpoint{1.072000in}{0.528000in}}{\pgfqpoint{3.696000in}{3.696000in}}%
\pgfusepath{clip}%
\pgfsetbuttcap%
\pgfsetroundjoin%
\definecolor{currentfill}{rgb}{0.713852,0.808857,0.979386}%
\pgfsetfillcolor{currentfill}%
\pgfsetlinewidth{0.000000pt}%
\definecolor{currentstroke}{rgb}{0.000000,0.000000,0.000000}%
\pgfsetstrokecolor{currentstroke}%
\pgfsetdash{}{0pt}%
\pgfpathmoveto{\pgfqpoint{3.279794in}{1.571215in}}%
\pgfpathlineto{\pgfqpoint{3.330004in}{1.681693in}}%
\pgfpathlineto{\pgfqpoint{3.355774in}{1.574666in}}%
\pgfpathlineto{\pgfqpoint{3.305762in}{1.467837in}}%
\pgfpathlineto{\pgfqpoint{3.279794in}{1.571215in}}%
\pgfpathclose%
\pgfusepath{fill}%
\end{pgfscope}%
\begin{pgfscope}%
\pgfpathrectangle{\pgfqpoint{1.072000in}{0.528000in}}{\pgfqpoint{3.696000in}{3.696000in}}%
\pgfusepath{clip}%
\pgfsetbuttcap%
\pgfsetroundjoin%
\definecolor{currentfill}{rgb}{0.724041,0.814910,0.975651}%
\pgfsetfillcolor{currentfill}%
\pgfsetlinewidth{0.000000pt}%
\definecolor{currentstroke}{rgb}{0.000000,0.000000,0.000000}%
\pgfsetstrokecolor{currentstroke}%
\pgfsetdash{}{0pt}%
\pgfpathmoveto{\pgfqpoint{3.203749in}{1.582903in}}%
\pgfpathlineto{\pgfqpoint{3.253599in}{1.685991in}}%
\pgfpathlineto{\pgfqpoint{3.279794in}{1.571215in}}%
\pgfpathlineto{\pgfqpoint{3.230147in}{1.471271in}}%
\pgfpathlineto{\pgfqpoint{3.203749in}{1.582903in}}%
\pgfpathclose%
\pgfusepath{fill}%
\end{pgfscope}%
\begin{pgfscope}%
\pgfpathrectangle{\pgfqpoint{1.072000in}{0.528000in}}{\pgfqpoint{3.696000in}{3.696000in}}%
\pgfusepath{clip}%
\pgfsetbuttcap%
\pgfsetroundjoin%
\definecolor{currentfill}{rgb}{0.748682,0.827679,0.963334}%
\pgfsetfillcolor{currentfill}%
\pgfsetlinewidth{0.000000pt}%
\definecolor{currentstroke}{rgb}{0.000000,0.000000,0.000000}%
\pgfsetstrokecolor{currentstroke}%
\pgfsetdash{}{0pt}%
\pgfpathmoveto{\pgfqpoint{3.127539in}{1.609902in}}%
\pgfpathlineto{\pgfqpoint{3.177081in}{1.705510in}}%
\pgfpathlineto{\pgfqpoint{3.203749in}{1.582903in}}%
\pgfpathlineto{\pgfqpoint{3.154414in}{1.489467in}}%
\pgfpathlineto{\pgfqpoint{3.127539in}{1.609902in}}%
\pgfpathclose%
\pgfusepath{fill}%
\end{pgfscope}%
\begin{pgfscope}%
\pgfpathrectangle{\pgfqpoint{1.072000in}{0.528000in}}{\pgfqpoint{3.696000in}{3.696000in}}%
\pgfusepath{clip}%
\pgfsetbuttcap%
\pgfsetroundjoin%
\definecolor{currentfill}{rgb}{0.822420,0.856898,0.910795}%
\pgfsetfillcolor{currentfill}%
\pgfsetlinewidth{0.000000pt}%
\definecolor{currentstroke}{rgb}{0.000000,0.000000,0.000000}%
\pgfsetstrokecolor{currentstroke}%
\pgfsetdash{}{0pt}%
\pgfpathmoveto{\pgfqpoint{3.406401in}{1.692634in}}%
\pgfpathlineto{\pgfqpoint{3.457639in}{1.817745in}}%
\pgfpathlineto{\pgfqpoint{3.482892in}{1.718789in}}%
\pgfpathlineto{\pgfqpoint{3.431791in}{1.593167in}}%
\pgfpathlineto{\pgfqpoint{3.406401in}{1.692634in}}%
\pgfpathclose%
\pgfusepath{fill}%
\end{pgfscope}%
\begin{pgfscope}%
\pgfpathrectangle{\pgfqpoint{1.072000in}{0.528000in}}{\pgfqpoint{3.696000in}{3.696000in}}%
\pgfusepath{clip}%
\pgfsetbuttcap%
\pgfsetroundjoin%
\definecolor{currentfill}{rgb}{0.813693,0.854282,0.918480}%
\pgfsetfillcolor{currentfill}%
\pgfsetlinewidth{0.000000pt}%
\definecolor{currentstroke}{rgb}{0.000000,0.000000,0.000000}%
\pgfsetstrokecolor{currentstroke}%
\pgfsetdash{}{0pt}%
\pgfpathmoveto{\pgfqpoint{3.330004in}{1.681693in}}%
\pgfpathlineto{\pgfqpoint{3.380787in}{1.799373in}}%
\pgfpathlineto{\pgfqpoint{3.406401in}{1.692634in}}%
\pgfpathlineto{\pgfqpoint{3.355774in}{1.574666in}}%
\pgfpathlineto{\pgfqpoint{3.330004in}{1.681693in}}%
\pgfpathclose%
\pgfusepath{fill}%
\end{pgfscope}%
\begin{pgfscope}%
\pgfpathrectangle{\pgfqpoint{1.072000in}{0.528000in}}{\pgfqpoint{3.696000in}{3.696000in}}%
\pgfusepath{clip}%
\pgfsetbuttcap%
\pgfsetroundjoin%
\definecolor{currentfill}{rgb}{0.822420,0.856898,0.910795}%
\pgfsetfillcolor{currentfill}%
\pgfsetlinewidth{0.000000pt}%
\definecolor{currentstroke}{rgb}{0.000000,0.000000,0.000000}%
\pgfsetstrokecolor{currentstroke}%
\pgfsetdash{}{0pt}%
\pgfpathmoveto{\pgfqpoint{3.253599in}{1.685991in}}%
\pgfpathlineto{\pgfqpoint{3.303979in}{1.795949in}}%
\pgfpathlineto{\pgfqpoint{3.330004in}{1.681693in}}%
\pgfpathlineto{\pgfqpoint{3.279794in}{1.571215in}}%
\pgfpathlineto{\pgfqpoint{3.253599in}{1.685991in}}%
\pgfpathclose%
\pgfusepath{fill}%
\end{pgfscope}%
\begin{pgfscope}%
\pgfpathrectangle{\pgfqpoint{1.072000in}{0.528000in}}{\pgfqpoint{3.696000in}{3.696000in}}%
\pgfusepath{clip}%
\pgfsetbuttcap%
\pgfsetroundjoin%
\definecolor{currentfill}{rgb}{0.839351,0.861167,0.894494}%
\pgfsetfillcolor{currentfill}%
\pgfsetlinewidth{0.000000pt}%
\definecolor{currentstroke}{rgb}{0.000000,0.000000,0.000000}%
\pgfsetstrokecolor{currentstroke}%
\pgfsetdash{}{0pt}%
\pgfpathmoveto{\pgfqpoint{3.177081in}{1.705510in}}%
\pgfpathlineto{\pgfqpoint{3.227112in}{1.807408in}}%
\pgfpathlineto{\pgfqpoint{3.253599in}{1.685991in}}%
\pgfpathlineto{\pgfqpoint{3.203749in}{1.582903in}}%
\pgfpathlineto{\pgfqpoint{3.177081in}{1.705510in}}%
\pgfpathclose%
\pgfusepath{fill}%
\end{pgfscope}%
\begin{pgfscope}%
\pgfpathrectangle{\pgfqpoint{1.072000in}{0.528000in}}{\pgfqpoint{3.696000in}{3.696000in}}%
\pgfusepath{clip}%
\pgfsetbuttcap%
\pgfsetroundjoin%
\definecolor{currentfill}{rgb}{0.906154,0.842091,0.806151}%
\pgfsetfillcolor{currentfill}%
\pgfsetlinewidth{0.000000pt}%
\definecolor{currentstroke}{rgb}{0.000000,0.000000,0.000000}%
\pgfsetstrokecolor{currentstroke}%
\pgfsetdash{}{0pt}%
\pgfpathmoveto{\pgfqpoint{3.380787in}{1.799373in}}%
\pgfpathlineto{\pgfqpoint{3.432120in}{1.920052in}}%
\pgfpathlineto{\pgfqpoint{3.457639in}{1.817745in}}%
\pgfpathlineto{\pgfqpoint{3.406401in}{1.692634in}}%
\pgfpathlineto{\pgfqpoint{3.380787in}{1.799373in}}%
\pgfpathclose%
\pgfusepath{fill}%
\end{pgfscope}%
\begin{pgfscope}%
\pgfpathrectangle{\pgfqpoint{1.072000in}{0.528000in}}{\pgfqpoint{3.696000in}{3.696000in}}%
\pgfusepath{clip}%
\pgfsetbuttcap%
\pgfsetroundjoin%
\definecolor{currentfill}{rgb}{0.906154,0.842091,0.806151}%
\pgfsetfillcolor{currentfill}%
\pgfsetlinewidth{0.000000pt}%
\definecolor{currentstroke}{rgb}{0.000000,0.000000,0.000000}%
\pgfsetstrokecolor{currentstroke}%
\pgfsetdash{}{0pt}%
\pgfpathmoveto{\pgfqpoint{3.303979in}{1.795949in}}%
\pgfpathlineto{\pgfqpoint{3.354883in}{1.909192in}}%
\pgfpathlineto{\pgfqpoint{3.380787in}{1.799373in}}%
\pgfpathlineto{\pgfqpoint{3.330004in}{1.681693in}}%
\pgfpathlineto{\pgfqpoint{3.303979in}{1.795949in}}%
\pgfpathclose%
\pgfusepath{fill}%
\end{pgfscope}%
\begin{pgfscope}%
\pgfpathrectangle{\pgfqpoint{1.072000in}{0.528000in}}{\pgfqpoint{3.696000in}{3.696000in}}%
\pgfusepath{clip}%
\pgfsetbuttcap%
\pgfsetroundjoin%
\definecolor{currentfill}{rgb}{0.916071,0.833977,0.788693}%
\pgfsetfillcolor{currentfill}%
\pgfsetlinewidth{0.000000pt}%
\definecolor{currentstroke}{rgb}{0.000000,0.000000,0.000000}%
\pgfsetstrokecolor{currentstroke}%
\pgfsetdash{}{0pt}%
\pgfpathmoveto{\pgfqpoint{3.227112in}{1.807408in}}%
\pgfpathlineto{\pgfqpoint{3.277634in}{1.912530in}}%
\pgfpathlineto{\pgfqpoint{3.303979in}{1.795949in}}%
\pgfpathlineto{\pgfqpoint{3.253599in}{1.685991in}}%
\pgfpathlineto{\pgfqpoint{3.227112in}{1.807408in}}%
\pgfpathclose%
\pgfusepath{fill}%
\end{pgfscope}%
\begin{pgfscope}%
\pgfpathrectangle{\pgfqpoint{1.072000in}{0.528000in}}{\pgfqpoint{3.696000in}{3.696000in}}%
\pgfusepath{clip}%
\pgfsetbuttcap%
\pgfsetroundjoin%
\definecolor{currentfill}{rgb}{0.959518,0.766973,0.674145}%
\pgfsetfillcolor{currentfill}%
\pgfsetlinewidth{0.000000pt}%
\definecolor{currentstroke}{rgb}{0.000000,0.000000,0.000000}%
\pgfsetstrokecolor{currentstroke}%
\pgfsetdash{}{0pt}%
\pgfpathmoveto{\pgfqpoint{3.354883in}{1.909192in}}%
\pgfpathlineto{\pgfqpoint{3.406270in}{2.021409in}}%
\pgfpathlineto{\pgfqpoint{3.432120in}{1.920052in}}%
\pgfpathlineto{\pgfqpoint{3.380787in}{1.799373in}}%
\pgfpathlineto{\pgfqpoint{3.354883in}{1.909192in}}%
\pgfpathclose%
\pgfusepath{fill}%
\end{pgfscope}%
\begin{pgfscope}%
\pgfpathrectangle{\pgfqpoint{1.072000in}{0.528000in}}{\pgfqpoint{3.696000in}{3.696000in}}%
\pgfusepath{clip}%
\pgfsetbuttcap%
\pgfsetroundjoin%
\definecolor{currentfill}{rgb}{0.960581,0.762501,0.667964}%
\pgfsetfillcolor{currentfill}%
\pgfsetlinewidth{0.000000pt}%
\definecolor{currentstroke}{rgb}{0.000000,0.000000,0.000000}%
\pgfsetstrokecolor{currentstroke}%
\pgfsetdash{}{0pt}%
\pgfpathmoveto{\pgfqpoint{3.277634in}{1.912530in}}%
\pgfpathlineto{\pgfqpoint{3.328624in}{2.017137in}}%
\pgfpathlineto{\pgfqpoint{3.354883in}{1.909192in}}%
\pgfpathlineto{\pgfqpoint{3.303979in}{1.795949in}}%
\pgfpathlineto{\pgfqpoint{3.277634in}{1.912530in}}%
\pgfpathclose%
\pgfusepath{fill}%
\end{pgfscope}%
\begin{pgfscope}%
\pgfpathrectangle{\pgfqpoint{1.072000in}{0.528000in}}{\pgfqpoint{3.696000in}{3.696000in}}%
\pgfusepath{clip}%
\pgfsetbuttcap%
\pgfsetroundjoin%
\definecolor{currentfill}{rgb}{0.967317,0.657471,0.538160}%
\pgfsetfillcolor{currentfill}%
\pgfsetlinewidth{0.000000pt}%
\definecolor{currentstroke}{rgb}{0.000000,0.000000,0.000000}%
\pgfsetstrokecolor{currentstroke}%
\pgfsetdash{}{0pt}%
\pgfpathmoveto{\pgfqpoint{3.328624in}{2.017137in}}%
\pgfpathlineto{\pgfqpoint{3.380031in}{2.116930in}}%
\pgfpathlineto{\pgfqpoint{3.406270in}{2.021409in}}%
\pgfpathlineto{\pgfqpoint{3.354883in}{1.909192in}}%
\pgfpathlineto{\pgfqpoint{3.328624in}{2.017137in}}%
\pgfpathclose%
\pgfusepath{fill}%
\end{pgfscope}%
\begin{pgfscope}%
\pgfsetbuttcap%
\pgfsetmiterjoin%
\definecolor{currentfill}{rgb}{1.000000,1.000000,1.000000}%
\pgfsetfillcolor{currentfill}%
\pgfsetlinewidth{0.000000pt}%
\definecolor{currentstroke}{rgb}{0.000000,0.000000,0.000000}%
\pgfsetstrokecolor{currentstroke}%
\pgfsetstrokeopacity{0.000000}%
\pgfsetdash{}{0pt}%
\pgfpathmoveto{\pgfqpoint{5.016000in}{1.452000in}}%
\pgfpathlineto{\pgfqpoint{5.385600in}{1.452000in}}%
\pgfpathlineto{\pgfqpoint{5.385600in}{3.300000in}}%
\pgfpathlineto{\pgfqpoint{5.016000in}{3.300000in}}%
\pgfpathlineto{\pgfqpoint{5.016000in}{1.452000in}}%
\pgfpathclose%
\pgfusepath{fill}%
\end{pgfscope}%
\begin{pgfscope}%
\pgfsys@transformshift{5.020000in}{1.450000in}%
\pgftext[left,bottom]{\includegraphics[interpolate=true,width=0.370000in,height=1.850000in]{image_01-img0.png}}%
\end{pgfscope}%
\begin{pgfscope}%
\pgfsetbuttcap%
\pgfsetroundjoin%
\definecolor{currentfill}{rgb}{0.000000,0.000000,0.000000}%
\pgfsetfillcolor{currentfill}%
\pgfsetlinewidth{0.803000pt}%
\definecolor{currentstroke}{rgb}{0.000000,0.000000,0.000000}%
\pgfsetstrokecolor{currentstroke}%
\pgfsetdash{}{0pt}%
\pgfsys@defobject{currentmarker}{\pgfqpoint{0.000000in}{0.000000in}}{\pgfqpoint{0.048611in}{0.000000in}}{%
\pgfpathmoveto{\pgfqpoint{0.000000in}{0.000000in}}%
\pgfpathlineto{\pgfqpoint{0.048611in}{0.000000in}}%
\pgfusepath{stroke,fill}%
}%
\begin{pgfscope}%
\pgfsys@transformshift{5.385600in}{1.910439in}%
\pgfsys@useobject{currentmarker}{}%
\end{pgfscope}%
\end{pgfscope}%
\begin{pgfscope}%
\definecolor{textcolor}{rgb}{0.000000,0.000000,0.000000}%
\pgfsetstrokecolor{textcolor}%
\pgfsetfillcolor{textcolor}%
\pgftext[x=5.482822in, y=1.862214in, left, base]{\color{textcolor}{\rmfamily\fontsize{10.000000}{12.000000}\selectfont\catcode`\^=\active\def^{\ifmmode\sp\else\^{}\fi}\catcode`\%=\active\def%{\%}$\mathdefault{\ensuremath{-}0.5}$}}%
\end{pgfscope}%
\begin{pgfscope}%
\pgfsetbuttcap%
\pgfsetroundjoin%
\definecolor{currentfill}{rgb}{0.000000,0.000000,0.000000}%
\pgfsetfillcolor{currentfill}%
\pgfsetlinewidth{0.803000pt}%
\definecolor{currentstroke}{rgb}{0.000000,0.000000,0.000000}%
\pgfsetstrokecolor{currentstroke}%
\pgfsetdash{}{0pt}%
\pgfsys@defobject{currentmarker}{\pgfqpoint{0.000000in}{0.000000in}}{\pgfqpoint{0.048611in}{0.000000in}}{%
\pgfpathmoveto{\pgfqpoint{0.000000in}{0.000000in}}%
\pgfpathlineto{\pgfqpoint{0.048611in}{0.000000in}}%
\pgfusepath{stroke,fill}%
}%
\begin{pgfscope}%
\pgfsys@transformshift{5.385600in}{2.376146in}%
\pgfsys@useobject{currentmarker}{}%
\end{pgfscope}%
\end{pgfscope}%
\begin{pgfscope}%
\definecolor{textcolor}{rgb}{0.000000,0.000000,0.000000}%
\pgfsetstrokecolor{textcolor}%
\pgfsetfillcolor{textcolor}%
\pgftext[x=5.482822in, y=2.327921in, left, base]{\color{textcolor}{\rmfamily\fontsize{10.000000}{12.000000}\selectfont\catcode`\^=\active\def^{\ifmmode\sp\else\^{}\fi}\catcode`\%=\active\def%{\%}$\mathdefault{0.0}$}}%
\end{pgfscope}%
\begin{pgfscope}%
\pgfsetbuttcap%
\pgfsetroundjoin%
\definecolor{currentfill}{rgb}{0.000000,0.000000,0.000000}%
\pgfsetfillcolor{currentfill}%
\pgfsetlinewidth{0.803000pt}%
\definecolor{currentstroke}{rgb}{0.000000,0.000000,0.000000}%
\pgfsetstrokecolor{currentstroke}%
\pgfsetdash{}{0pt}%
\pgfsys@defobject{currentmarker}{\pgfqpoint{0.000000in}{0.000000in}}{\pgfqpoint{0.048611in}{0.000000in}}{%
\pgfpathmoveto{\pgfqpoint{0.000000in}{0.000000in}}%
\pgfpathlineto{\pgfqpoint{0.048611in}{0.000000in}}%
\pgfusepath{stroke,fill}%
}%
\begin{pgfscope}%
\pgfsys@transformshift{5.385600in}{2.841853in}%
\pgfsys@useobject{currentmarker}{}%
\end{pgfscope}%
\end{pgfscope}%
\begin{pgfscope}%
\definecolor{textcolor}{rgb}{0.000000,0.000000,0.000000}%
\pgfsetstrokecolor{textcolor}%
\pgfsetfillcolor{textcolor}%
\pgftext[x=5.482822in, y=2.793628in, left, base]{\color{textcolor}{\rmfamily\fontsize{10.000000}{12.000000}\selectfont\catcode`\^=\active\def^{\ifmmode\sp\else\^{}\fi}\catcode`\%=\active\def%{\%}$\mathdefault{0.5}$}}%
\end{pgfscope}%
\begin{pgfscope}%
\pgfsetrectcap%
\pgfsetmiterjoin%
\pgfsetlinewidth{0.803000pt}%
\definecolor{currentstroke}{rgb}{0.000000,0.000000,0.000000}%
\pgfsetstrokecolor{currentstroke}%
\pgfsetdash{}{0pt}%
\pgfpathmoveto{\pgfqpoint{5.016000in}{1.452000in}}%
\pgfpathlineto{\pgfqpoint{5.200800in}{1.452000in}}%
\pgfpathlineto{\pgfqpoint{5.385600in}{1.452000in}}%
\pgfpathlineto{\pgfqpoint{5.385600in}{3.300000in}}%
\pgfpathlineto{\pgfqpoint{5.200800in}{3.300000in}}%
\pgfpathlineto{\pgfqpoint{5.016000in}{3.300000in}}%
\pgfpathlineto{\pgfqpoint{5.016000in}{1.452000in}}%
\pgfpathclose%
\pgfusepath{stroke}%
\end{pgfscope}%
\end{pgfpicture}%
\makeatother%
\endgroup%

		\end{center}
		\caption{A surface given by the equation $Z=$}
	\end{figure}
	
	\begin{example}
		
	\end{example}
	
	\begin{example}
		
	\end{example}
	
	\begin{definition}
		
	\end{definition}
	
	\begin{example}
		Let's check that the definition of a vector field on a manifold indeed generalizes the definition on $\mathbb{R}^n$.
	\end{example}
	
	\begin{definition}
		A function $f$ is said to be \textbf{morse} if all its critical points are non-degenerate.
	\end{definition}

	
	\ifSubfilesClassLoaded{
			\bibliographystyle{apalike}
			\bibliography{../refs}
	}{}



	
\end{document}
