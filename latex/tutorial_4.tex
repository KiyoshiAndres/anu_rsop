\documentclass[12pt]{article}

\usepackage{geometry}
\usepackage{xcolor}
\usepackage{color,soul}
\usepackage{mathrsfs}
\usepackage{tikz-cd}
\usepackage{tcolorbox}
\usepackage{amsthm}
\usepackage{amsmath}
\usepackage{amssymb}
\usepackage{fancyhdr}
\pagestyle{fancy}
\usepackage{graphicx}


\usepackage{titlesec}
\titleformat{\section}[block]{\color{black}\Large\bfseries\filcenter}{}{1em}{}



\setlength{\headheight}{40pt}
\usepackage{graphicx}

\usepackage{imakeidx}
\makeindex[columns=2, title=Index, intoc]


\usepackage{hyperref}
\usepackage{cleveref}

\hypersetup{
	colorlinks=true,
	linkcolor=blue,
	filecolor=magenta,      
	urlcolor=cyan,
	pdftitle={Sharelatex Example},
	pdfpagemode=FullScreen,
}

\urlstyle{same}


\newcommand{\FHom}{\mathcal{H} o m}
\newcommand{\sheafification}{\mathcal{F}^+}
\newcommand{\Id}{\textrm{Id}}
\newcommand{\Hom}{\textrm{Hom}}
\newcommand{\Spec}{\textrm{Spec }}
\newcommand{\Sch}{\textrm{Sch}}
\newcommand{\Set}{\textrm{Set}}
\newcommand{\Grass}{\textrm{Grass}}
\newcommand{\Mod}{\textrm{Mod}}
\newcommand{\id}{\textrm{id}}
\newcommand{\Ob}{\textrm{Ob}}
\newcommand{\Ext}{\textrm{Ext}}
\newcommand{\Tor}{\textrm{Tor}}
\newcommand{\Invlim}{\lim\limits_{\longleftarrow}}
\newcommand{\Dirlim}{\lim\limits_{\longrightarrow}}
\newcommand{\Gal}{\textrm{Gal}}
\newcommand{\F}{\mathscr{F}}
\newcommand{\res}{\text{res}}
\newcommand{\primep}{\mathfrak{p}}
\newcommand{\bigo}{\mathcal{O}}
\newcommand\smallo{
	\mathchoice
	{{\scriptstyle\mathcal{O}}}% \displaystyle
	{{\scriptstyle\mathcal{O}}}% \textstyle
	{{\scriptscriptstyle\mathcal{O}}}% \scriptstyle
	{\scalebox{.7}{$\scriptscriptstyle\mathcal{O}$}}%\scriptscriptstyle
}
\newcommand{\Aut}{\textrm{Aut}}
\newcommand{\projs}{\mathbf{Proj}(S)}
\newcommand{\Proj}{\mathbf{Proj}}
\newcommand{\Ass}{\textrm{Ass}}
\newcommand{\Div}{\textrm{Div}}
\newcommand{\pdiv}{\textrm{div}}
\newcommand{\Cov}{\textrm{Cov}}
\newcommand{\Op}{\textrm{Op}}


\newtheoremstyle{theorem}{}{}{}{}{\color{blue}\bfseries}{.}{ }{}
\theoremstyle{theorem}
\newtheorem{theorem}{Theorem}[section]
\newtheorem{lemma}[theorem]{Lemma}
\newtheorem{corollary}[theorem]{Corollary}
\newtheorem{proposition}[theorem]{Proposition}

\theoremstyle{definition}
\newtheorem{definition}{Definition} [section]
\newtheorem{example}{Example}
\newtheorem{xca}[theorem]{Exercise}



\theoremstyle{remark}
\newtheorem{remark}{Remark}

\newtheoremstyle{gremark}{}{}{}{}{\color{red}\bfseries}{.}{ }{}
\theoremstyle{gremark}
\newtheorem{gremark}{Future Expansion Necessary}

\newtheoremstyle{discussion}{}{}{}{}{\color{orange}\bfseries}{.}{ }{}
\theoremstyle{discussion}
\newtheorem{discussion}{Discussion}

\newtheoremstyle{notation}{}{}{}{}{\color{orange}\bfseries}{.}{ }{}
\theoremstyle{notation}
\newtheorem{notation}{Notation}

\bibliographystyle{apalike}

\newtcolorbox{mybox}[3][]
{
	colframe = #2!25,
	colback  = #2!10,
	coltitle = #2!20!black,  
	#1,
}




\title{Workshop 4}
\author{Kiyoshi Andres Takeuchi Romo}

\begin{document}
	
	\maketitle
	
	
	\begin{proposition}
		Let $v_1,\ldots, v_n$ be pairwise orthogonal non-zero vectors. Then they are linearly independent.
	\end{proposition}

	\begin{proof}
		Suppose $c_1v_1+\ldots+c_nv_n=0$. Then, for taking inner product with $v_i$, we get $c_iv_i\cdot v_i=0$ since the other components vanish by orthogonality. And since $v_i$ is non-zero, $c_i=0$. 
	\end{proof}

	
	
	\pagebreak
	
	

	\begin{definition}
		A set of matrices forms a \textbf{reducible set} if there exists a nonzero proper subspace of the underlying vector space that is invariant under all of those matrices. 
	\end{definition}
		
	\begin{proposition}
		Do the following set of matrices form a reducible or irreducible set? Explain your answer.
	\end{proposition}

	\begin{proof}
		Yes, $\alpha(1,1,1,1)$ is an invariant subspace.
	\end{proof}
		
		
		
		\pagebreak
		
		
	\begin{proposition}
		A matrix that commutes with every matrix of an irreducible representation must be a multiple of the identity matrix. That is, if $D$ is irreducible and if $AD(g) = D(g)A$ for all $g\in G$, then $A = \alpha I$.
	\end{proposition}

	\begin{lemma}
		Let $\rho:G\to GL(n,\mathbb{C})$ be a scalar representation. That is, for any $g$, there exists some $\alpha_g\in \mathbb{C}$ such that $\rho(g)=\alpha_g I$. Then if $\rho$ is irreducible, $n=1$.
	\end{lemma}

	\begin{proof}
		Suppose $n\geq 1$. Then let $W$ be any $1$-dimensional subspace of $\mathbb{C}^n$. Then by definition, for $w\in W$, $\rho(g)w=\alpha_g Iw=\alpha w\in W$ since $W$ is a subspace. So we have a contradiction. $n=1$.
	\end{proof}

	
	\begin{proposition}
		Use Schur’s lemma to show that an irreducible matrix representation, in $GL(n, \mathbb{C})$, of an abelian	group must be one-dimensional.
	\end{proposition}

	\begin{proof}
		Let $\rho:G\to GL(n, \mathbb{C})$ be a an irreducible representation of $G$. Fix $h\in G$ and let $g\in G$ be arbitrary. Then, since $\rho$ is a group homomorphism, and $G$ is abelian, $\rho(g)\rho(h)=\rho(h)\rho(g)$. Therefore $\rho(h)=\alpha I$ is a scalar multiple of the identity by Schur's Lemma. Since $h$ was arbitrary (albeit fixed), $\rho(h)$ is a scalar multiple of the identity matrix for all $h\in G$.
		By the Lemma $n=1$.
	\end{proof}

		\begin{proposition}[Schur's Lemma]
			Let $V$ and $W$ be vector spaces. Let $\rho_V$ and $\rho_W$ be irreducible representations of $G$ on $V$ and $W$ respectively.
			
			\begin{enumerate}
				\item If $V$ and $W$ are not isomorphic, then there are no non-trivial $G$-linear maps between them.
				\item If $V=W$ finite-dimensional over an algebraically closed field, and if $\rho_V=\rho_W$, then the only nontrivial G-linear maps are the identity, and scalar multiples of the identity.
			\end{enumerate}
		\end{proposition}

	\begin{proof}
		
	\end{proof}


	\pagebreak
	
	
	\begin{definition}
		\textbf{infinitessimal operator}
	\end{definition}
	
	\begin{proposition}
		Suppose $T(a)$ is the translation operator that acts on a function $f(x)$ to make it $f(x + a)$. That is:
		 $T (a)f (x) = f (x + a).$
		Show that the infinitessimal operator corresponding to the Lie group of translation operators is given by
		
		$$\partial $$
	\end{proposition}

	\begin{proof}
		\[
		\left[e^{ia p_x} f\right](x)
		\;=\; \sum_{n=0}^\infty \frac{(ia\,p_x)^n}{n!}\,f(x)
		\;=\; \sum_{n=0}^\infty \frac{(ia)^n}{n!}\,\left(-i\frac{d}{dx}\right)^n f(x)
		\;=\; f(x+a).
		\]
	\end{proof}
	
	
	\begin{lemma}
		
	\end{lemma}

	\begin{proof}
		\noindent
		\textbf{Step 1. Define \(\,g(t)\,\) by \(g(t) = f(x + t)\).}
		
		\[
		g(0) \;=\; f(x), 
		\quad\text{and for the \(n\)-th derivative we get}
		\]
		\[
		g^{(n)}(0) 
		\;=\; \left.\frac{d^n}{dt^n} \bigl[f(x + t)\bigr]\right|_{t=0}
		\;=\; f^{(n)}(x).
		\]
		
		\noindent
		\textbf{Step 2. Maclaurin series for \(\,g(t)\).}
		
		\[
		\text{Because \(g\) is sufficiently smooth, its Taylor expansion about 0 is}
		\]
		\[
		g(t)
		\;=\; \sum_{n=0}^{\infty} \frac{g^{(n)}(0)}{n!} \, t^n
		\;=\; \sum_{n=0}^{\infty} \frac{f^{(n)}(x)}{n!} \, t^n.
		\]
		
		\noindent
		\textbf{Step 3. Substitute \(\,t = a\).}
		
		\[
		\text{Since } g(t) = f(x + t), \text{ setting } t = a \text{ yields}
		\]
		\[
		f(x + a) 
		\;=\; g(a) 
		\;=\; \sum_{n=0}^{\infty} \frac{f^{(n)}(x)}{n!} \, a^n,
		\]
		\[
		\text{which is the usual Taylor expansion of } f(x+a) \text{ about } x.
		\]
	\end{proof}
	
	
	
	
	
\end{document}